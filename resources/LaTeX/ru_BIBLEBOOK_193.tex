\section{BOOK 192}
\subsection{CHAPTER 1}
\begin{tcolorbox}
\textsubscript{1} Иуда, раб Иисуса Христа, брат Иакова, призванным, которые освящены Богом Отцем и сохранены Иисусом Христом:
\end{tcolorbox}
\begin{tcolorbox}
\textsubscript{2} милость вам и мир и любовь да умножатся.
\end{tcolorbox}
\begin{tcolorbox}
\textsubscript{3} Возлюбленные! имея все усердие писать вам об общем спасении, я почел за нужное написать вам увещание--подвизаться за веру, однажды преданную святым.
\end{tcolorbox}
\begin{tcolorbox}
\textsubscript{4} Ибо вкрались некоторые люди, издревле предназначенные к сему осуждению, нечестивые, обращающие благодать Бога нашего в [повод к] распутству и отвергающиеся единого Владыки Бога и Господа нашего Иисуса Христа.
\end{tcolorbox}
\begin{tcolorbox}
\textsubscript{5} Я хочу напомнить вам, уже знающим это, что Господь, избавив народ из земли Египетской, потом неверовавших погубил,
\end{tcolorbox}
\begin{tcolorbox}
\textsubscript{6} и ангелов, не сохранивших своего достоинства, но оставивших свое жилище, соблюдает в вечных узах, под мраком, на суд великого дня.
\end{tcolorbox}
\begin{tcolorbox}
\textsubscript{7} Как Содом и Гоморра и окрестные города, подобно им блудодействовавшие и ходившие за иною плотию, подвергшись казни огня вечного, поставлены в пример, --
\end{tcolorbox}
\begin{tcolorbox}
\textsubscript{8} так точно будет и с сими мечтателями, которые оскверняют плоть, отвергают начальства и злословят высокие власти.
\end{tcolorbox}
\begin{tcolorbox}
\textsubscript{9} Михаил Архангел, когда говорил с диаволом, споря о Моисеевом теле, не смел произнести укоризненного суда, но сказал: 'да запретит тебе Господь'.
\end{tcolorbox}
\begin{tcolorbox}
\textsubscript{10} А сии злословят то, чего не знают; что же по природе, как бессловесные животные, знают, тем растлевают себя.
\end{tcolorbox}
\begin{tcolorbox}
\textsubscript{11} Горе им, потому что идут путем Каиновым, предаются обольщению мзды, как Валаам, и в упорстве погибают, как Корей.
\end{tcolorbox}
\begin{tcolorbox}
\textsubscript{12} Таковые бывают соблазном на ваших вечерях любви; пиршествуя с вами, без страха утучняют себя. Это безводные облака, носимые ветром; осенние деревья, бесплодные, дважды умершие, исторгнутые;
\end{tcolorbox}
\begin{tcolorbox}
\textsubscript{13} свирепые морские волны, пенящиеся срамотами своими; звезды блуждающие, которым блюдется мрак тьмы на веки.
\end{tcolorbox}
\begin{tcolorbox}
\textsubscript{14} О них пророчествовал и Енох, седьмый от Адама, говоря: 'се, идет Господь со тьмами святых Ангелов Своих--
\end{tcolorbox}
\begin{tcolorbox}
\textsubscript{15} сотворить суд над всеми и обличить всех между ними нечестивых во всех делах, которые произвело их нечестие, и во всех жестоких словах, которые произносили на Него нечестивые грешники'.
\end{tcolorbox}
\begin{tcolorbox}
\textsubscript{16} Это ропотники, ничем не довольные, поступающие по своим похотям (нечестиво и беззаконно); уста их произносят надутые слова; они оказывают лицеприятие для корысти.
\end{tcolorbox}
\begin{tcolorbox}
\textsubscript{17} Но вы, возлюбленные, помните предсказанное Апостолами Господа нашего Иисуса Христа.
\end{tcolorbox}
\begin{tcolorbox}
\textsubscript{18} Они говорили вам, что в последнее время появятся ругатели, поступающие по своим нечестивым похотям.
\end{tcolorbox}
\begin{tcolorbox}
\textsubscript{19} Это люди, отделяющие себя (от единства веры), душевные, не имеющие духа.
\end{tcolorbox}
\begin{tcolorbox}
\textsubscript{20} А вы, возлюбленные, назидая себя на святейшей вере вашей, молясь Духом Святым,
\end{tcolorbox}
\begin{tcolorbox}
\textsubscript{21} сохраняйте себя в любви Божией, ожидая милости от Господа нашего Иисуса Христа, для вечной жизни.
\end{tcolorbox}
\begin{tcolorbox}
\textsubscript{22} И к одним будьте милостивы, с рассмотрением,
\end{tcolorbox}
\begin{tcolorbox}
\textsubscript{23} а других страхом спасайте, исторгая из огня, обличайте же со страхом, гнушаясь даже одеждою, которая осквернена плотью.
\end{tcolorbox}
\begin{tcolorbox}
\textsubscript{24} Могущему же соблюсти вас от падения и поставить пред славою Своею непорочными в радости,
\end{tcolorbox}
\begin{tcolorbox}
\textsubscript{25} Единому Премудрому Богу, Спасителю нашему чрез Иисуса Христа Господа нашего, слава и величие, сила и власть прежде всех веков, ныне и во все веки. Аминь.
\end{tcolorbox}
