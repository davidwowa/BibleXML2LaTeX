\section{BOOK 153}
\subsection{CHAPTER 1}
\begin{tcolorbox}
\textsubscript{1} Павел и Силуан и Тимофей--церкви Фессалоникской в Боге Отце и Господе Иисусе Христе: благодать вам и мир от Бога Отца нашего и Господа Иисуса Христа.
\end{tcolorbox}
\begin{tcolorbox}
\textsubscript{2} Всегда благодарим Бога за всех вас, вспоминая о вас в молитвах наших,
\end{tcolorbox}
\begin{tcolorbox}
\textsubscript{3} непрестанно памятуя ваше дело веры и труд любви и терпение упования на Господа нашего Иисуса Христа пред Богом и Отцем нашим,
\end{tcolorbox}
\begin{tcolorbox}
\textsubscript{4} зная избрание ваше, возлюбленные Богом братия;
\end{tcolorbox}
\begin{tcolorbox}
\textsubscript{5} потому что наше благовествование у вас было не в слове только, но и в силе и во Святом Духе, и со многим удостоверением, как вы [сами] знаете, каковы были мы для вас между вами.
\end{tcolorbox}
\begin{tcolorbox}
\textsubscript{6} И вы сделались подражателями нам и Господу, приняв слово при многих скорбях с радостью Духа Святаго,
\end{tcolorbox}
\begin{tcolorbox}
\textsubscript{7} так что вы стали образцом для всех верующих в Македонии и Ахаии.
\end{tcolorbox}
\begin{tcolorbox}
\textsubscript{8} Ибо от вас пронеслось слово Господне не только в Македонии и Ахаии, но и во всяком месте прошла [слава] о вере вашей в Бога, так что нам ни о чем не нужно рассказывать.
\end{tcolorbox}
\begin{tcolorbox}
\textsubscript{9} Ибо сами они сказывают о нас, какой вход имели мы к вам, и как вы обратились к Богу от идолов, [чтобы] служить Богу живому и истинному
\end{tcolorbox}
\begin{tcolorbox}
\textsubscript{10} и ожидать с небес Сына Его, Которого Он воскресил из мертвых, Иисуса, избавляющего нас от грядущего гнева.
\end{tcolorbox}
\subsection{CHAPTER 2}
\begin{tcolorbox}
\textsubscript{1} Вы сами знаете, братия, о нашем входе к вам, что он был не бездейственный;
\end{tcolorbox}
\begin{tcolorbox}
\textsubscript{2} но, прежде пострадав и быв поруганы в Филиппах, как вы знаете, мы дерзнули в Боге нашем проповедать вам благовестие Божие с великим подвигом.
\end{tcolorbox}
\begin{tcolorbox}
\textsubscript{3} Ибо в учении нашем нет ни заблуждения, ни нечистых [побуждений], ни лукавства;
\end{tcolorbox}
\begin{tcolorbox}
\textsubscript{4} но, как Бог удостоил нас того, чтобы вверить [нам] благовестие, так мы и говорим, угождая не человекам, но Богу, испытующему сердца наши.
\end{tcolorbox}
\begin{tcolorbox}
\textsubscript{5} Ибо никогда не было у нас перед вами ни слов ласкательства, как вы знаете, ни видов корысти: Бог свидетель!
\end{tcolorbox}
\begin{tcolorbox}
\textsubscript{6} Не ищем славы человеческой ни от вас, ни от других:
\end{tcolorbox}
\begin{tcolorbox}
\textsubscript{7} мы могли явиться с важностью, как Апостолы Христовы, но были тихи среди вас, подобно как кормилица нежно обходится с детьми своими.
\end{tcolorbox}
\begin{tcolorbox}
\textsubscript{8} Так мы, из усердия к вам, восхотели передать вам не только благовестие Божие, но и души наши, потому что вы стали нам любезны.
\end{tcolorbox}
\begin{tcolorbox}
\textsubscript{9} Ибо вы помните, братия, труд наш и изнурение: ночью и днем работая, чтобы не отяготить кого из вас, мы проповедывали у вас благовестие Божие.
\end{tcolorbox}
\begin{tcolorbox}
\textsubscript{10} Свидетели вы и Бог, как свято и праведно и безукоризненно поступали мы перед вами, верующими,
\end{tcolorbox}
\begin{tcolorbox}
\textsubscript{11} потому что вы знаете, как каждого из вас, как отец детей своих,
\end{tcolorbox}
\begin{tcolorbox}
\textsubscript{12} мы просили и убеждали и умоляли поступать достойно Бога, призвавшего вас в Свое Царство и славу.
\end{tcolorbox}
\begin{tcolorbox}
\textsubscript{13} Посему и мы непрестанно благодарим Бога, что, приняв от нас слышанное слово Божие, вы приняли не [как] слово человеческое, но [как] слово Божие, --каково оно есть по истине, --которое и действует в вас, верующих.
\end{tcolorbox}
\begin{tcolorbox}
\textsubscript{14} Ибо вы, братия, сделались подражателями церквам Божиим во Христе Иисусе, находящимся в Иудее, потому что и вы то же претерпели от своих единоплеменников, что и те от Иудеев,
\end{tcolorbox}
\begin{tcolorbox}
\textsubscript{15} которые убили и Господа Иисуса и Его пророков, и нас изгнали, и Богу не угождают, и всем человекам противятся,
\end{tcolorbox}
\begin{tcolorbox}
\textsubscript{16} которые препятствуют нам говорить язычникам, чтобы спаслись, и через это всегда наполняют меру грехов своих; но приближается на них гнев до конца.
\end{tcolorbox}
\begin{tcolorbox}
\textsubscript{17} Мы же, братия, быв разлучены с вами на короткое время лицем, а не сердцем, тем с большим желанием старались увидеть лице ваше.
\end{tcolorbox}
\begin{tcolorbox}
\textsubscript{18} И потому мы, я Павел, и раз и два хотели прийти к вам, но воспрепятствовал нам сатана.
\end{tcolorbox}
\begin{tcolorbox}
\textsubscript{19} Ибо кто наша надежда, или радость, или венец похвалы? Не и вы ли пред Господом нашим Иисусом Христом в пришествие Его?
\end{tcolorbox}
\begin{tcolorbox}
\textsubscript{20} Ибо вы--слава наша и радость.
\end{tcolorbox}
\subsection{CHAPTER 3}
\begin{tcolorbox}
\textsubscript{1} И потому, не терпя более, мы восхотели остаться в Афинах одни,
\end{tcolorbox}
\begin{tcolorbox}
\textsubscript{2} и послали Тимофея, брата нашего и служителя Божия и сотрудника нашего в благовествовании Христовом, чтобы утвердить вас и утешить в вере вашей,
\end{tcolorbox}
\begin{tcolorbox}
\textsubscript{3} чтобы никто не поколебался в скорбях сих: ибо вы сами знаете, что так нам суждено.
\end{tcolorbox}
\begin{tcolorbox}
\textsubscript{4} Ибо мы и тогда, как были у вас, предсказывали вам, что будем страдать, как и случилось, и вы знаете.
\end{tcolorbox}
\begin{tcolorbox}
\textsubscript{5} Посему и я, не терпя более, послал узнать о вере вашей, чтобы как не искусил вас искуситель и не сделался тщетным труд наш.
\end{tcolorbox}
\begin{tcolorbox}
\textsubscript{6} Теперь же, когда пришел к нам от вас Тимофей и принес нам добрую весть о вере и любви вашей, и что вы всегда имеете добрую память о нас, желая нас видеть, как и мы вас,
\end{tcolorbox}
\begin{tcolorbox}
\textsubscript{7} то мы, при всей скорби и нужде нашей, утешились вами, братия, ради вашей веры;
\end{tcolorbox}
\begin{tcolorbox}
\textsubscript{8} ибо теперь мы живы, когда вы стоите в Господе.
\end{tcolorbox}
\begin{tcolorbox}
\textsubscript{9} Какую благодарность можем мы воздать Богу за вас, за всю радость, которою радуемся о вас пред Богом нашим,
\end{tcolorbox}
\begin{tcolorbox}
\textsubscript{10} ночь и день всеусердно молясь о том, чтобы видеть лице ваше и дополнить, чего недоставало вере вашей?
\end{tcolorbox}
\begin{tcolorbox}
\textsubscript{11} Сам же Бог и Отец наш и Господь наш Иисус Христос да управит путь наш к вам.
\end{tcolorbox}
\begin{tcolorbox}
\textsubscript{12} А вас Господь да исполнит и преисполнит любовью друг к другу и ко всем, какою мы исполнены к вам,
\end{tcolorbox}
\begin{tcolorbox}
\textsubscript{13} чтобы утвердить сердца ваши непорочными во святыне пред Богом и Отцем нашим в пришествие Господа нашего Иисуса Христа со всеми святыми Его. Аминь.
\end{tcolorbox}
\subsection{CHAPTER 4}
\begin{tcolorbox}
\textsubscript{1} За сим, братия, просим и умоляем вас Христом Иисусом, чтобы вы, приняв от нас, как должно вам поступать и угождать Богу, более в том преуспевали,
\end{tcolorbox}
\begin{tcolorbox}
\textsubscript{2} ибо вы знаете, какие мы дали вам заповеди от Господа Иисуса.
\end{tcolorbox}
\begin{tcolorbox}
\textsubscript{3} Ибо воля Божия есть освящение ваше, чтобы вы воздерживались от блуда;
\end{tcolorbox}
\begin{tcolorbox}
\textsubscript{4} чтобы каждый из вас умел соблюдать свой сосуд в святости и чести,
\end{tcolorbox}
\begin{tcolorbox}
\textsubscript{5} а не в страсти похотения, как и язычники, не знающие Бога;
\end{tcolorbox}
\begin{tcolorbox}
\textsubscript{6} чтобы вы ни в чем не поступали с братом своим противозаконно и корыстолюбиво: потому что Господь--мститель за все это, как и прежде мы говорили вам и свидетельствовали.
\end{tcolorbox}
\begin{tcolorbox}
\textsubscript{7} Ибо призвал нас Бог не к нечистоте, но к святости.
\end{tcolorbox}
\begin{tcolorbox}
\textsubscript{8} Итак непокорный непокорен не человеку, но Богу, Который и дал нам Духа Своего Святаго.
\end{tcolorbox}
\begin{tcolorbox}
\textsubscript{9} О братолюбии же нет нужды писать к вам; ибо вы сами научены Богом любить друг друга,
\end{tcolorbox}
\begin{tcolorbox}
\textsubscript{10} ибо вы так и поступаете со всеми братиями по всей Македонии. Умоляем же вас, братия, более преуспевать
\end{tcolorbox}
\begin{tcolorbox}
\textsubscript{11} и усердно стараться о том, чтобы жить тихо, делать свое [дело] и работать своими собственными руками, как мы заповедывали вам;
\end{tcolorbox}
\begin{tcolorbox}
\textsubscript{12} чтобы вы поступали благоприлично перед внешними и ни в чем не нуждались.
\end{tcolorbox}
\begin{tcolorbox}
\textsubscript{13} Не хочу же оставить вас, братия, в неведении об умерших, дабы вы не скорбели, как прочие, не имеющие надежды.
\end{tcolorbox}
\begin{tcolorbox}
\textsubscript{14} Ибо, если мы веруем, что Иисус умер и воскрес, то и умерших в Иисусе Бог приведет с Ним.
\end{tcolorbox}
\begin{tcolorbox}
\textsubscript{15} Ибо сие говорим вам словом Господним, что мы живущие, оставшиеся до пришествия Господня, не предупредим умерших,
\end{tcolorbox}
\begin{tcolorbox}
\textsubscript{16} потому что Сам Господь при возвещении, при гласе Архангела и трубе Божией, сойдет с неба, и мертвые во Христе воскреснут прежде;
\end{tcolorbox}
\begin{tcolorbox}
\textsubscript{17} потом мы, оставшиеся в живых, вместе с ними восхищены будем на облаках в сретение Господу на воздухе, и так всегда с Господом будем.
\end{tcolorbox}
\begin{tcolorbox}
\textsubscript{18} Итак утешайте друг друга сими словами.
\end{tcolorbox}
\subsection{CHAPTER 5}
\begin{tcolorbox}
\textsubscript{1} О временах же и сроках нет нужды писать к вам, братия,
\end{tcolorbox}
\begin{tcolorbox}
\textsubscript{2} ибо сами вы достоверно знаете, что день Господень так придет, как тать ночью.
\end{tcolorbox}
\begin{tcolorbox}
\textsubscript{3} Ибо, когда будут говорить: 'мир и безопасность', тогда внезапно постигнет их пагуба, подобно как мука родами [постигает] имеющую во чреве, и не избегнут.
\end{tcolorbox}
\begin{tcolorbox}
\textsubscript{4} Но вы, братия, не во тьме, чтобы день застал вас, как тать.
\end{tcolorbox}
\begin{tcolorbox}
\textsubscript{5} Ибо все вы--сыны света и сыны дня: мы--не [сыны] ночи, ни тьмы.
\end{tcolorbox}
\begin{tcolorbox}
\textsubscript{6} Итак, не будем спать, как и прочие, но будем бодрствовать и трезвиться.
\end{tcolorbox}
\begin{tcolorbox}
\textsubscript{7} Ибо спящие спят ночью, и упивающиеся упиваются ночью.
\end{tcolorbox}
\begin{tcolorbox}
\textsubscript{8} Мы же, будучи [сынами] дня, да трезвимся, облекшись в броню веры и любви и в шлем надежды спасения,
\end{tcolorbox}
\begin{tcolorbox}
\textsubscript{9} потому что Бог определил нас не на гнев, но к получению спасения через Господа нашего Иисуса Христа,
\end{tcolorbox}
\begin{tcolorbox}
\textsubscript{10} умершего за нас, чтобы мы, бодрствуем ли, или спим, жили вместе с Ним.
\end{tcolorbox}
\begin{tcolorbox}
\textsubscript{11} Посему увещавайте друг друга и назидайте один другого, как вы и делаете.
\end{tcolorbox}
\begin{tcolorbox}
\textsubscript{12} Просим же вас, братия, уважать трудящихся у вас, и предстоятелей ваших в Господе, и вразумляющих вас,
\end{tcolorbox}
\begin{tcolorbox}
\textsubscript{13} и почитать их преимущественно с любовью за дело их; будьте в мире между собою.
\end{tcolorbox}
\begin{tcolorbox}
\textsubscript{14} Умоляем также вас, братия, вразумляйте бесчинных, утешайте малодушных, поддерживайте слабых, будьте долготерпеливы ко всем.
\end{tcolorbox}
\begin{tcolorbox}
\textsubscript{15} Смотрите, чтобы кто кому не воздавал злом за зло; но всегда ищите добра и друг другу и всем.
\end{tcolorbox}
\begin{tcolorbox}
\textsubscript{16} Всегда радуйтесь.
\end{tcolorbox}
\begin{tcolorbox}
\textsubscript{17} Непрестанно молитесь.
\end{tcolorbox}
\begin{tcolorbox}
\textsubscript{18} За все благодарите: ибо такова о вас воля Божия во Христе Иисусе.
\end{tcolorbox}
\begin{tcolorbox}
\textsubscript{19} Духа не угашайте.
\end{tcolorbox}
\begin{tcolorbox}
\textsubscript{20} Пророчества не уничижайте.
\end{tcolorbox}
\begin{tcolorbox}
\textsubscript{21} Все испытывайте, хорошего держитесь.
\end{tcolorbox}
\begin{tcolorbox}
\textsubscript{22} Удерживайтесь от всякого рода зла.
\end{tcolorbox}
\begin{tcolorbox}
\textsubscript{23} Сам же Бог мира да освятит вас во всей полноте, и ваш дух и душа и тело во всей целости да сохранится без порока в пришествие Господа нашего Иисуса Христа.
\end{tcolorbox}
\begin{tcolorbox}
\textsubscript{24} Верен Призывающий вас, Который и сотворит [сие].
\end{tcolorbox}
\begin{tcolorbox}
\textsubscript{25} Братия! молитесь о нас.
\end{tcolorbox}
\begin{tcolorbox}
\textsubscript{26} Приветствуйте всех братьев лобзанием святым.
\end{tcolorbox}
\begin{tcolorbox}
\textsubscript{27} Заклинаю вас Господом прочитать сие послание всем святым братиям.
\end{tcolorbox}
\begin{tcolorbox}
\textsubscript{28} Благодать Господа нашего Иисуса Христа с вами. Аминь.
\end{tcolorbox}
