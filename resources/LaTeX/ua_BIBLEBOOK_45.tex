\section{BOOK 44}
\subsection{CHAPTER 1}
\begin{tcolorbox}
\textsubscript{1} Павло, раб Ісуса Христа, покликаний апостол, вибраний для звіщання Євангелії Божої,
\end{tcolorbox}
\begin{tcolorbox}
\textsubscript{2} яке Він перед тим приобіцяв через Своїх пророків у святих Писаннях,
\end{tcolorbox}
\begin{tcolorbox}
\textsubscript{3} про Сина Свого, що тілом був із насіння Давидового,
\end{tcolorbox}
\begin{tcolorbox}
\textsubscript{4} і об'явився Сином Божим у силі, за духом святости, через воскресення з мертвих, про Ісуса Христа, Господа нашого,
\end{tcolorbox}
\begin{tcolorbox}
\textsubscript{5} що через Нього прийняли ми благодать і апостольство на послух віри через Ім'я Його між усіма народами,
\end{tcolorbox}
\begin{tcolorbox}
\textsubscript{6} між якими й ви, покликані Ісуса Христа,
\end{tcolorbox}
\begin{tcolorbox}
\textsubscript{7} усім, хто знаходиться в Римі, улюбленим Божим, вибраним святим, благодать вам та мир від Бога, Отця нашого, і Господа Ісуса Христа!
\end{tcolorbox}
\begin{tcolorbox}
\textsubscript{8} Отже, насамперед дякую Богові моєму через Ісуса Христа за всіх вас, що віра ваша звіщається по всьому світові.
\end{tcolorbox}
\begin{tcolorbox}
\textsubscript{9} Бо свідок мені Бог, Якому служу духом своїм у звіщанні Євангелії Його Сина, що я безперестанно згадую про вас,
\end{tcolorbox}
\begin{tcolorbox}
\textsubscript{10} і в молитвах своїх завжди молюся, щоб воля Божа щасливо попровадила мене коли прийти до вас.
\end{tcolorbox}
\begin{tcolorbox}
\textsubscript{11} Бо прагну вас бачити, щоб подати вам якого дара духовного для зміцнення вас,
\end{tcolorbox}
\begin{tcolorbox}
\textsubscript{12} цебто потішитись разом між вами спільною вірою і вашою, і моєю.
\end{tcolorbox}
\begin{tcolorbox}
\textsubscript{13} Не хочу ж, щоб ви не знали, браття, що багато разів мав я замір прийти до вас, але мені перешкоджувано аж досі, щоб мати який плід і в вас, як і в інших народів.
\end{tcolorbox}
\begin{tcolorbox}
\textsubscript{14} А гелленам і чужоземцям, розумним і немудрим я боржник.
\end{tcolorbox}
\begin{tcolorbox}
\textsubscript{15} Отже, щодо мене, я готовий і вам, хто знаходиться в Римі, звіщати Євангелію.
\end{tcolorbox}
\begin{tcolorbox}
\textsubscript{16} Бо я не соромлюсь Євангелії, бож вона сила Божа на спасіння кожному, хто вірує, перше ж юдеєві, а потім гелленові.
\end{tcolorbox}
\begin{tcolorbox}
\textsubscript{17} Правда бо Божа з'являється в ній з віри в віру, як написано: А праведний житиме вірою.
\end{tcolorbox}
\begin{tcolorbox}
\textsubscript{18} Бо гнів Божий з'являється з неба на всяку безбожність і неправду людей, що правду гамують неправдою,
\end{tcolorbox}
\begin{tcolorbox}
\textsubscript{19} тому, що те, що можна знати про Бога, явне для них, бо їм Бог об'явив.
\end{tcolorbox}
\begin{tcolorbox}
\textsubscript{20} Бо Його невидиме від створення світу, власне Його вічна сила й Божество, думанням про твори стає видиме. Так що нема їм виправдання,
\end{tcolorbox}
\begin{tcolorbox}
\textsubscript{21} бо, пізнавши Бога, не прославляли Його, як Бога, і не дякували, але знікчемніли своїми думками, і запаморочилось нерозумне їхнє серце.
\end{tcolorbox}
\begin{tcolorbox}
\textsubscript{22} Називаючи себе мудрими, вони потуманіли,
\end{tcolorbox}
\begin{tcolorbox}
\textsubscript{23} і славу нетлінного Бога змінили на подобу образа тлінної людини, і птахів, і чотириногих, і гадів.
\end{tcolorbox}
\begin{tcolorbox}
\textsubscript{24} Тому то й видав їх Бог у пожадливостях їхніх сердець на нечистість, щоб вони самі знеславляли тіла свої.
\end{tcolorbox}
\begin{tcolorbox}
\textsubscript{25} Вони Божу правду замінили на неправду, і честь віддавали, і служили створінню більш, як Творцеві, що благословенний навіки, амінь.
\end{tcolorbox}
\begin{tcolorbox}
\textsubscript{26} Через це Бог їх видав на пожадливість ганебну, бо їхні жінки замінили природне єднання на протиприродне.
\end{tcolorbox}
\begin{tcolorbox}
\textsubscript{27} Так само й чоловіки, позоставивши природне єднання з жіночою статтю, розпалилися своєю пожадливістю один до одного, і чоловіки з чоловіками сором чинили. І вони прийняли в собі відплату, відповідну їхньому блудові.
\end{tcolorbox}
\begin{tcolorbox}
\textsubscript{28} А що вони не вважали за потрібне мати Бога в пізнанні, видав їх Бог на розум перевернений, щоб чинили непристойне.
\end{tcolorbox}
\begin{tcolorbox}
\textsubscript{29} Вони повні всякої неправди, лукавства, зажерливости, злоби, повні заздрости, убивства, суперечки, омани, лихих звичаїв,
\end{tcolorbox}
\begin{tcolorbox}
\textsubscript{30} обмовники, наклепники, богоненавидники, напасники, чваньки, пишні, винахідники зла, неслухняні батькам,
\end{tcolorbox}
\begin{tcolorbox}
\textsubscript{31} нерозумні, зрадники, нелюбовні, немилостиві.
\end{tcolorbox}
\begin{tcolorbox}
\textsubscript{32} Вони знають присуд Божий, що ті, хто чинить таке, варті смерти, а проте не тільки самі чинять, але й хвалять тих, хто робить таке.
\end{tcolorbox}
\subsection{CHAPTER 2}
\begin{tcolorbox}
\textsubscript{1} Ось тому без виправдання ти, кожний чоловіче, що судиш, бо в чому осуджуєш іншого, сам себе осуджуєш, бо чиниш те саме й ти, що судиш.
\end{tcolorbox}
\begin{tcolorbox}
\textsubscript{2} А ми знаємо, що суд Божий поправді на тих, хто чинить таке.
\end{tcolorbox}
\begin{tcolorbox}
\textsubscript{3} Чи ти думаєш, чоловіче, судячи тих, хто чинить таке, а сам робиш таке саме, що ти втечеш від суду Божого?
\end{tcolorbox}
\begin{tcolorbox}
\textsubscript{4} Або погорджуєш багатством Його добрости, лагідности та довготерпіння, не знаючи, що Божа добрість провадить тебе до покаяння?
\end{tcolorbox}
\begin{tcolorbox}
\textsubscript{5} Та через жорстокість свою й нерозкаяність серця збираєш собі гнів на день гніву та об'явлення справедливого суду Бога,
\end{tcolorbox}
\begin{tcolorbox}
\textsubscript{6} що кожному віддасть за його вчинками:
\end{tcolorbox}
\begin{tcolorbox}
\textsubscript{7} тим, хто витривалістю в добрім ділі шукає слави, і чести, і нетління, життя вічне,
\end{tcolorbox}
\begin{tcolorbox}
\textsubscript{8} а сварливим та тим, хто противиться правді, але кориться неправді, лютість та гнів.
\end{tcolorbox}
\begin{tcolorbox}
\textsubscript{9} Недоля та утиск на всяку душу людини, хто чинить зле, юдея ж перше та геллена,
\end{tcolorbox}
\begin{tcolorbox}
\textsubscript{10} а слава, і честь, і мир усякому, хто чинить добре, юдеєві ж перше та гелленові.
\end{tcolorbox}
\begin{tcolorbox}
\textsubscript{11} Бо не дивиться Бог на обличчя!
\end{tcolorbox}
\begin{tcolorbox}
\textsubscript{12} Котрі бо згрішили без Закону, без Закону й загинуть, а котрі згрішили в Законі, приймуть суд за Законом.
\end{tcolorbox}
\begin{tcolorbox}
\textsubscript{13} Бо не слухачі Закону справедливі перед Богом, але виконавці Закону виправдані будуть.
\end{tcolorbox}
\begin{tcolorbox}
\textsubscript{14} Бо коли погани, що не мають Закону, з природи чинять законне, вони, не мавши Закону, самі собі Закон,
\end{tcolorbox}
\begin{tcolorbox}
\textsubscript{15} що виявляють діло Закону, написане в серцях своїх, як свідчить їм сумління та їхні думки, що то осуджують, то виправдують одна одну,
\end{tcolorbox}
\begin{tcolorbox}
\textsubscript{16} дня, коли Бог, згідно з моїм благовістям, буде судити таємні речі людей через Ісуса Христа.
\end{tcolorbox}
\begin{tcolorbox}
\textsubscript{17} Ось ти звешся юдеєм, і спираєшся на Закона, і хвалишся Богом,
\end{tcolorbox}
\begin{tcolorbox}
\textsubscript{18} і знаєш волю Його, і розумієш, що краще, навчившись із Закону,
\end{tcolorbox}
\begin{tcolorbox}
\textsubscript{19} і маєш певність, що ти провідник для сліпих, світло для тих, хто знаходиться в темряві,
\end{tcolorbox}
\begin{tcolorbox}
\textsubscript{20} виховник нерозумним, учитель дітям, що ти маєш зразок знання й правди в Законі.
\end{tcolorbox}
\begin{tcolorbox}
\textsubscript{21} Отож, ти, що іншого навчаєш, себе самого не вчиш! Проповідуєш не красти, а сам крадеш!
\end{tcolorbox}
\begin{tcolorbox}
\textsubscript{22} Наказуючи не чинити перелюбу, чиниш перелюб! Гидуючи ідолами, чиниш святокрадство!
\end{tcolorbox}
\begin{tcolorbox}
\textsubscript{23} Ти, що хвалишся Законом, зневажаєш Бога переступом Закону!
\end{tcolorbox}
\begin{tcolorbox}
\textsubscript{24} Бо через вас зневажається Боже Ймення в поган, як написано.
\end{tcolorbox}
\begin{tcolorbox}
\textsubscript{25} Обрізання корисне, коли виконуєш Закона; а коли ти переступник Закону, то обрізання твоє стало необрізанням.
\end{tcolorbox}
\begin{tcolorbox}
\textsubscript{26} Отож, коли необрізаний зберігає постанови Закону, то чи не порахується його необрізання за обрізання?
\end{tcolorbox}
\begin{tcolorbox}
\textsubscript{27} І необрізаний з природи, виконуючи Закона, чи не осудить тебе, переступника Закону з Писанням і обрізанням?
\end{tcolorbox}
\begin{tcolorbox}
\textsubscript{28} Бо не той юдей, що є ним назовні, і не то обрізання, що назовні на тілі,
\end{tcolorbox}
\begin{tcolorbox}
\textsubscript{29} але той, що є юдей потаємно, духово, і обрізання серця духом, а не за буквою; і йому похвала не від людей, а від Бога.
\end{tcolorbox}
\subsection{CHAPTER 3}
\begin{tcolorbox}
\textsubscript{1} Отож, що має більшого юдей, або яка користь від обрізання?
\end{tcolorbox}
\begin{tcolorbox}
\textsubscript{2} Багато, на всякий спосіб, а насамперед, що їм довірені були Слова Божі.
\end{tcolorbox}
\begin{tcolorbox}
\textsubscript{3} Бо що ж, що не вірували деякі? Чи ж їхнє недовірство знищить вірність Божу?
\end{tcolorbox}
\begin{tcolorbox}
\textsubscript{4} Зовсім ні! Бож Бог правдивий, а кожна людина неправдива, як написано: Щоб був Ти виправданий у словах Своїх, і переміг, коли будеш судитися.
\end{tcolorbox}
\begin{tcolorbox}
\textsubscript{5} А коли наша неправда виставляє правду Божу, то що скажемо? Чи ж Бог несправедливий, коли гнів виявляє? Говорю по-людському.
\end{tcolorbox}
\begin{tcolorbox}
\textsubscript{6} Зовсім ні! Бож як Бог судитиме світ?
\end{tcolorbox}
\begin{tcolorbox}
\textsubscript{7} Бо коли Божа правда через мою неправду збільшилась на славу Йому, пощо судити ще й мене, як грішника?
\end{tcolorbox}
\begin{tcolorbox}
\textsubscript{8} І чи не так, як нас лають, і як деякі говорять, ніби ми кажемо: Робімо зле, щоб вийшло добре? Справедливий осуд на таких!
\end{tcolorbox}
\begin{tcolorbox}
\textsubscript{9} То що ж? Маємо перевагу? Анітрохи! Бож ми перед тим довели, що юдеї й геллени усі під гріхом,
\end{tcolorbox}
\begin{tcolorbox}
\textsubscript{10} як написано: Нема праведного ані одного;
\end{tcolorbox}
\begin{tcolorbox}
\textsubscript{11} нема, хто розумів би; немає, хто Бога шукав би,
\end{tcolorbox}
\begin{tcolorbox}
\textsubscript{12} усі повідступали, разом стали непотрібні, нема доброчинця, нема ні одного!
\end{tcolorbox}
\begin{tcolorbox}
\textsubscript{13} Гріб відкритий їхнє горло, язиком своїм кажуть неправду, отрута зміїна на їхніх губах,
\end{tcolorbox}
\begin{tcolorbox}
\textsubscript{14} уста їхні повні прокляття й гіркоти!
\end{tcolorbox}
\begin{tcolorbox}
\textsubscript{15} Швидкі їхні ноги, щоб кров проливати,
\end{tcolorbox}
\begin{tcolorbox}
\textsubscript{16} руїна та злидні на їхніх дорогах,
\end{tcolorbox}
\begin{tcolorbox}
\textsubscript{17} а дороги миру вони не пізнали!
\end{tcolorbox}
\begin{tcolorbox}
\textsubscript{18} Нема страху Божого перед очима їхніми...
\end{tcolorbox}
\begin{tcolorbox}
\textsubscript{19} А ми знаємо, що скільки говорить Закон, він говорить до тих, хто під Законом, щоб замкнути всякі уста, і щоб став увесь світ винний Богові.
\end{tcolorbox}
\begin{tcolorbox}
\textsubscript{20} Бо жадне тіло ділами Закону не виправдається перед Ним, Законом бо гріх пізнається.
\end{tcolorbox}
\begin{tcolorbox}
\textsubscript{21} А тепер, без Закону, правда Божа з'явилась, про яку свідчать Закон і Пророки.
\end{tcolorbox}
\begin{tcolorbox}
\textsubscript{22} А Божа правда через віру в Ісуса Христа в усіх і на всіх, хто вірує, бо різниці немає,
\end{tcolorbox}
\begin{tcolorbox}
\textsubscript{23} бо всі згрішили, і позбавлені Божої слави,
\end{tcolorbox}
\begin{tcolorbox}
\textsubscript{24} але дарма виправдуються Його благодаттю, через відкуплення, що в Ісусі Христі,
\end{tcolorbox}
\begin{tcolorbox}
\textsubscript{25} що Його Бог дав у жертву примирення в крові Його через віру, щоб виявити Свою правду через відпущення давніше вчинених гріхів,
\end{tcolorbox}
\begin{tcolorbox}
\textsubscript{26} за довготерпіння Божого, щоб виявити Свою правду за теперішнього часу, щоб бути Йому праведним, і виправдувати того, хто вірує в Ісуса.
\end{tcolorbox}
\begin{tcolorbox}
\textsubscript{27} Тож де похвальба? Виключена. Яким законом? Законом діл? Ні, але законом віри.
\end{tcolorbox}
\begin{tcolorbox}
\textsubscript{28} Отож, ми визнаємо, що людина виправдується вірою, без діл Закону.
\end{tcolorbox}
\begin{tcolorbox}
\textsubscript{29} Хіба ж Бог тільки для юдеїв, а не й для поган? Так, і для поган,
\end{tcolorbox}
\begin{tcolorbox}
\textsubscript{30} бо є один тільки Бог, що виправдає обрізання з віри й необрізання через віру.
\end{tcolorbox}
\begin{tcolorbox}
\textsubscript{31} Тож чи не нищимо ми Закона вірою? Зовсім ні, але зміцнюємо Закона.
\end{tcolorbox}
\subsection{CHAPTER 4}
\begin{tcolorbox}
\textsubscript{1} Що ж, скажемо, знайшов Авраам, наш отець за тілом?
\end{tcolorbox}
\begin{tcolorbox}
\textsubscript{2} Бо коли Авраам виправдався ділами, то він має похвалу, та не в Бога.
\end{tcolorbox}
\begin{tcolorbox}
\textsubscript{3} Що бо Писання говорить? Увірував Авраам Богові, і це йому залічено в праведність.
\end{tcolorbox}
\begin{tcolorbox}
\textsubscript{4} А заплата виконавцеві не рахується з милости, але з обов'язку.
\end{tcolorbox}
\begin{tcolorbox}
\textsubscript{5} А тому, хто не виконує, але вірує в Того, Хто виправдує нечестивого, віра його порахується в праведність.
\end{tcolorbox}
\begin{tcolorbox}
\textsubscript{6} Як і Давид називає блаженною людину, якій рахує Бог праведність без діл:
\end{tcolorbox}
\begin{tcolorbox}
\textsubscript{7} Блаженні, кому прощені беззаконня, і кому прикриті гріхи.
\end{tcolorbox}
\begin{tcolorbox}
\textsubscript{8} Блаженна людина, якій Господь не порахує гріха!
\end{tcolorbox}
\begin{tcolorbox}
\textsubscript{9} Чи ж це блаженство з обрізання чи з необрізання? Бо говоримо, що віра залічена Авраамові в праведність.
\end{tcolorbox}
\begin{tcolorbox}
\textsubscript{10} Як же залічена? Як був в обрізанні, чи в необрізанні? Не в обрізанні, але в необрізанні!
\end{tcolorbox}
\begin{tcolorbox}
\textsubscript{11} І прийняв він ознаку обрізання, печать праведности через віру, що її в необрізанні мав, щоб йому бути отцем усіх віруючих, хоч були необрізані, щоб і їм залічено праведність,
\end{tcolorbox}
\begin{tcolorbox}
\textsubscript{12} і отцем обрізаних, не тільки тих, хто з обрізання, але й тих, хто ходить по слідах віри, що її в необрізанні мав наш отець Авраам.
\end{tcolorbox}
\begin{tcolorbox}
\textsubscript{13} Бо обітницю Авраамові чи його насінню, що бути йому спадкоємцем світу, дано не Законом, але праведністю віри.
\end{tcolorbox}
\begin{tcolorbox}
\textsubscript{14} Бо коли спадкоємці ті, хто з Закону, то спорожніла віра й знівечилась обітниця.
\end{tcolorbox}
\begin{tcolorbox}
\textsubscript{15} Бо Закон чинить гнів; де ж немає Закону, немає й переступу.
\end{tcolorbox}
\begin{tcolorbox}
\textsubscript{16} Через це з віри, щоб було з милости, щоб обітниця певна була всім нащадкам, не тільки тому, хто з Закону, але й тому, хто з віри Авраама, що отець усім нам,
\end{tcolorbox}
\begin{tcolorbox}
\textsubscript{17} як написано: Отцем багатьох народів Я поставив тебе, перед Богом, Якому він вірив, Який оживляє мертвих і кличе неіснуюче, як існуюче.
\end{tcolorbox}
\begin{tcolorbox}
\textsubscript{18} Він проти надії увірував у надії, що стане батьком багатьох народів, за сказаним: Таке численне буде насіння твоє!
\end{tcolorbox}
\begin{tcolorbox}
\textsubscript{19} І не знеміг він у вірі, і не вважав свого тіла за вже омертвіле, бувши майже сторічним, ні утроби Сариної за змертвілу,
\end{tcolorbox}
\begin{tcolorbox}
\textsubscript{20} і не мав сумніву в обітницю Божу через недовірство, але зміцнився в вірі, і віддав славу Богові,
\end{tcolorbox}
\begin{tcolorbox}
\textsubscript{21} і був зовсім певний, що Він має силу й виконати те, що обіцяв.
\end{tcolorbox}
\begin{tcolorbox}
\textsubscript{22} Тому й залічено це йому в праведність.
\end{tcolorbox}
\begin{tcolorbox}
\textsubscript{23} Та не написано за нього одного, що залічено йому,
\end{tcolorbox}
\begin{tcolorbox}
\textsubscript{24} а за нас, залічиться й нам, що віруємо в Того, Хто воскресив із мертвих Ісуса, Господа нашого,
\end{tcolorbox}
\begin{tcolorbox}
\textsubscript{25} що був виданий за наші гріхи, і воскрес для виправдання нашого.
\end{tcolorbox}
\subsection{CHAPTER 5}
\begin{tcolorbox}
\textsubscript{1} Отож, виправдавшись вірою, майте мир із Богом через Господа нашого Ісуса Христа,
\end{tcolorbox}
\begin{tcolorbox}
\textsubscript{2} через Якого ми вірою одержали доступ до тієї благодаті, що в ній стоїмо, і хвалимось надією слави Божої.
\end{tcolorbox}
\begin{tcolorbox}
\textsubscript{3} І не тільки нею, але й хвалимося в утисках, знаючи, що утиски приносять терпеливість,
\end{tcolorbox}
\begin{tcolorbox}
\textsubscript{4} а терпеливість досвід, а досвід надію,
\end{tcolorbox}
\begin{tcolorbox}
\textsubscript{5} а надія не засоромить, бо любов Божа вилилася в наші серця Святим Духом, даним нам.
\end{tcolorbox}
\begin{tcolorbox}
\textsubscript{6} Бо Христос, коли ми були ще недужі, своєї пори помер за нечестивих.
\end{tcolorbox}
\begin{tcolorbox}
\textsubscript{7} Бо навряд чи помре хто за праведника, ще бо за доброго може хто й відважиться вмерти.
\end{tcolorbox}
\begin{tcolorbox}
\textsubscript{8} А Бог доводить Свою любов до нас тим, що Христос умер за нас, коли ми були ще грішниками.
\end{tcolorbox}
\begin{tcolorbox}
\textsubscript{9} Тож тим більше спасемося Ним від гніву тепер, коли кров'ю Його ми виправдані.
\end{tcolorbox}
\begin{tcolorbox}
\textsubscript{10} Бо коли ми, бувши ворогами, примирилися з Богом через смерть Сина Його, то тим більше, примирившися, спасемося життям Його.
\end{tcolorbox}
\begin{tcolorbox}
\textsubscript{11} І не тільки це, але й хвалимося в Бозі через Господа нашого Ісуса Христа, що через Нього одержали ми тепер примирення.
\end{tcolorbox}
\begin{tcolorbox}
\textsubscript{12} Тому то, як через одного чоловіка ввійшов до світу гріх, а гріхом смерть, так прийшла й смерть у всіх людей через те, що всі згрішили.
\end{tcolorbox}
\begin{tcolorbox}
\textsubscript{13} Гріх бо був у світі й до Закону, але гріх не ставиться в провину, коли немає Закону.
\end{tcolorbox}
\begin{tcolorbox}
\textsubscript{14} Та смерть панувала від Адама аж до Мойсея і над тими, хто не згрішив, подібно переступу Адама, який є образ майбутнього.
\end{tcolorbox}
\begin{tcolorbox}
\textsubscript{15} Але не такий дар благодаті, як переступ. Бо коли за переступ одного померло багато, то тим більш благодать Божа й дар через благодать однієї Людини, Ісуса Христа, щедро спливли на багатьох.
\end{tcolorbox}
\begin{tcolorbox}
\textsubscript{16} І дар не такий, як те, що сталось від одного, що згрішив; бо суд за один прогріх на осуд, а дар благодаті на виправдання від багатьох прогріхів.
\end{tcolorbox}
\begin{tcolorbox}
\textsubscript{17} Бо коли за переступ одного смерть панувала через одного, то тим більше ті, хто приймає рясноту благодаті й дар праведности, запанують у житті через одного Ісуса Христа.
\end{tcolorbox}
\begin{tcolorbox}
\textsubscript{18} Ось тому, як через переступ одного на всіх людей прийшов осуд, так і через праведність Одного прийшло виправдання для життя на всіх людей.
\end{tcolorbox}
\begin{tcolorbox}
\textsubscript{19} Бо як через непослух одного чоловіка багато-хто стали грішними, так і через послух Одного багато-хто стануть праведними.
\end{tcolorbox}
\begin{tcolorbox}
\textsubscript{20} Закон же прийшов, щоб збільшився переступ. А де збільшився гріх, там зарясніла благодать,
\end{tcolorbox}
\begin{tcolorbox}
\textsubscript{21} щоб, як гріх панував через смерть, так само й благодать запанувала через праведність для життя вічного Ісусом Христом, Господом нашим.
\end{tcolorbox}
\subsection{CHAPTER 6}
\begin{tcolorbox}
\textsubscript{1} Що ж скажемо? Позостанемся в гріху, щоб благодать примножилась? Зовсім ні!
\end{tcolorbox}
\begin{tcolorbox}
\textsubscript{2} Ми, що вмерли для гріха, як ще будемо жити в нім?
\end{tcolorbox}
\begin{tcolorbox}
\textsubscript{3} Чи ви не знаєте, що ми всі, хто христився у Христа Ісуса, у смерть Його христилися?
\end{tcolorbox}
\begin{tcolorbox}
\textsubscript{4} Отож, ми поховані з Ним хрищенням у смерть, щоб, як воскрес Христос із мертвих славою Отця, так щоб і ми стали ходити в обновленні життя.
\end{tcolorbox}
\begin{tcolorbox}
\textsubscript{5} Бо коли ми з'єдналися подобою смерти Його, то з'єднаємось і подобою воскресення,
\end{tcolorbox}
\begin{tcolorbox}
\textsubscript{6} знаючи те, що наш давній чоловік розп'ятий із Ним, щоб знищилось тіло гріховне, щоб не бути нам більше рабами гріха,
\end{tcolorbox}
\begin{tcolorbox}
\textsubscript{7} бо хто вмер, той звільнивсь від гріха!
\end{tcolorbox}
\begin{tcolorbox}
\textsubscript{8} А коли ми померли з Христом, то віруємо, що й жити з Ним будемо,
\end{tcolorbox}
\begin{tcolorbox}
\textsubscript{9} знаючи, що Христос, воскреснувши з мертвих, уже більш не вмирає, смерть над Ним не панує вже більше!
\end{tcolorbox}
\begin{tcolorbox}
\textsubscript{10} Бо що вмер Він, то один раз умер для гріха, а що живе, то для Бога живе.
\end{tcolorbox}
\begin{tcolorbox}
\textsubscript{11} Так само ж і ви вважайте себе за мертвих для гріха й за живих для Бога в Христі Ісусі, Господі нашім.
\end{tcolorbox}
\begin{tcolorbox}
\textsubscript{12} Тож нехай не панує гріх у смертельному вашому тілі, щоб вам слухатись його пожадливостей,
\end{tcolorbox}
\begin{tcolorbox}
\textsubscript{13} і не віддавайте членів своїх гріхові за знаряддя неправедности, але віддавайте себе Богові, як ожилих із мертвих, а члени ваші Богові за знаряддя праведности.
\end{tcolorbox}
\begin{tcolorbox}
\textsubscript{14} Бо хай гріх не панує над вами, ви бо не під Законом, а під благодаттю.
\end{tcolorbox}
\begin{tcolorbox}
\textsubscript{15} Що ж? Чи будемо грішити, бо ми не під Законом, а під благодаттю? Зовсім ні!
\end{tcolorbox}
\begin{tcolorbox}
\textsubscript{16} Хіба ви не знаєте, що кому віддаєте себе за рабів на послух, то ви й раби того, кого слухаєтесь, або гріха на смерть, або послуху на праведність?
\end{tcolorbox}
\begin{tcolorbox}
\textsubscript{17} Тож дяка Богові, що ви, бувши рабами гріха, від серця послухались того роду науки, якому ви себе віддали.
\end{tcolorbox}
\begin{tcolorbox}
\textsubscript{18} А звільнившися від гріха, стали рабами праведности.
\end{tcolorbox}
\begin{tcolorbox}
\textsubscript{19} Говорю я по-людському, через неміч вашого тіла. Бо як ви віддавали були члени ваші за рабів нечистості й беззаконню на беззаконня, так тепер віддайте члени ваші за рабів праведности на освячення.
\end{tcolorbox}
\begin{tcolorbox}
\textsubscript{20} Бо коли були ви рабами гріха, то були вільні від праведности.
\end{tcolorbox}
\begin{tcolorbox}
\textsubscript{21} Який же плід ви мали тоді? Такі речі, що ними соромитесь тепер, бо кінець їх то смерть.
\end{tcolorbox}
\begin{tcolorbox}
\textsubscript{22} А тепер, звільнившися від гріха й ставши рабами Богові, маєте плід ваш на освячення, а кінець життя вічне.
\end{tcolorbox}
\begin{tcolorbox}
\textsubscript{23} Бо заплата за гріх смерть, а дар Божий вічне життя в Христі Ісусі, Господі нашім!
\end{tcolorbox}
\subsection{CHAPTER 7}
\begin{tcolorbox}
\textsubscript{1} Чи ви не знаєте, браття, бо говорю тим, хто знає Закона, що Закон панує над людиною, поки вона живе?
\end{tcolorbox}
\begin{tcolorbox}
\textsubscript{2} Бо заміжня жінка, поки живе чоловік, прив'язана до нього Законом; а коли помре чоловік, вона звільняється від закону чоловіка.
\end{tcolorbox}
\begin{tcolorbox}
\textsubscript{3} Тому то, поки живе чоловік, вона буде вважатися перелюбницею, якщо стане дружиною іншому чоловікові; коли ж чоловік помре, вона вільна від Закону, і не буде перелюбницею, якщо стане за дружину іншому чоловікові.
\end{tcolorbox}
\begin{tcolorbox}
\textsubscript{4} Так, мої браття, і ви вмерли для Закону через тіло Христове, щоб належати вам іншому, Воскреслому з мертвих, щоб приносити плід Богові.
\end{tcolorbox}
\begin{tcolorbox}
\textsubscript{5} Бо коли ми жили за тілом, то пристрасті гріховні, що походять від Закону, діяли в наших членах, щоб приносити плід смерти.
\end{tcolorbox}
\begin{tcolorbox}
\textsubscript{6} А тепер ми звільнились від Закону, умерши для того, чим були зв'язані, щоб служити нам обновленням духа, а не старістю букви.
\end{tcolorbox}
\begin{tcolorbox}
\textsubscript{7} Що ж скажемо? Чи Закон то гріх? Зовсім ні! Але я не пізнав гріха, як тільки через Закон, бо я не знав би пожадливости, коли б Закон не наказував: Не пожадай.
\end{tcolorbox}
\begin{tcolorbox}
\textsubscript{8} Але гріх, узявши привід від заповіді, зробив у мені всяку пожадливість, бо без Закону гріх мертвий.
\end{tcolorbox}
\begin{tcolorbox}
\textsubscript{9} А я колись жив без Закону, але, коли прийшла заповідь, то гріх ожив,
\end{tcolorbox}
\begin{tcolorbox}
\textsubscript{10} а я вмер; і сталася мені та заповідь, що для життя, на смерть,
\end{tcolorbox}
\begin{tcolorbox}
\textsubscript{11} бо гріх, узявши причину від заповіді, звів мене, і нею вмертвив.
\end{tcolorbox}
\begin{tcolorbox}
\textsubscript{12} Тому то Закон святий, і заповідь свята, і праведна, і добра.
\end{tcolorbox}
\begin{tcolorbox}
\textsubscript{13} Тож чи добре стало мені смертю? Зовсім ні! Але гріх, щоб стати гріхом, приніс мені смерть добром, щоб гріх став міцно грішний через заповідь.
\end{tcolorbox}
\begin{tcolorbox}
\textsubscript{14} Бо ми знаємо, що Закон духовний, а я тілесний, проданий під гріх.
\end{tcolorbox}
\begin{tcolorbox}
\textsubscript{15} Бо що я виконую, не розумію; я бо чиню не те, що хочу, але що ненавиджу, те я роблю.
\end{tcolorbox}
\begin{tcolorbox}
\textsubscript{16} А коли роблю те, чого я не хочу, то згоджуюсь із Законом, що він добрий,
\end{tcolorbox}
\begin{tcolorbox}
\textsubscript{17} а тому вже не я це виконую, але гріх, що живе в мені.
\end{tcolorbox}
\begin{tcolorbox}
\textsubscript{18} Знаю бо, що не живе в мені, цебто в тілі моїм, добре; бо бажання лежить у мені, але щоб виконати добре, того не знаходжу.
\end{tcolorbox}
\begin{tcolorbox}
\textsubscript{19} Бо не роблю я доброго, що хочу, але зле, чого не хочу, це чиню.
\end{tcolorbox}
\begin{tcolorbox}
\textsubscript{20} Коли ж я роблю те, чого не хочу, то вже не я це виконую, але гріх, що живе в мені.
\end{tcolorbox}
\begin{tcolorbox}
\textsubscript{21} Тож знаходжу закона, коли хочу робити добро, що зло лежить у мені.
\end{tcolorbox}
\begin{tcolorbox}
\textsubscript{22} Бо маю задоволення в Законі Божому за внутрішнім чоловіком,
\end{tcolorbox}
\begin{tcolorbox}
\textsubscript{23} та бачу інший закон у членах своїх, що воює проти закону мого розуму, і полонить мене законом гріховним, що знаходиться в членах моїх.
\end{tcolorbox}
\begin{tcolorbox}
\textsubscript{24} Нещасна я людина! Хто мене визволить від тіла цієї смерти?
\end{tcolorbox}
\begin{tcolorbox}
\textsubscript{25} Дякую Богові через Ісуса Христа, Господа нашого. Тому то я сам служу розумом Законові Божому, але тілом закону гріховному...
\end{tcolorbox}
\subsection{CHAPTER 8}
\begin{tcolorbox}
\textsubscript{1} Тож немає тепер жадного осуду тим, хто ходить у Христі Ісусі не за тілом, а за духом,
\end{tcolorbox}
\begin{tcolorbox}
\textsubscript{2} бо закон духа життя в Христі Ісусі визволив мене від закону гріха й смерти.
\end{tcolorbox}
\begin{tcolorbox}
\textsubscript{3} Бо що було неможливе для Закону, у чому був він безсилий тілом, Бог послав Сина Свого в подобі гріховного тіла, і за гріх осудив гріх у тілі,
\end{tcolorbox}
\begin{tcolorbox}
\textsubscript{4} щоб виконалось виправдання Закону на нас, що ходимо не за тілом, а за духом.
\end{tcolorbox}
\begin{tcolorbox}
\textsubscript{5} Бо ті, хто ходить за тілом, думають про тілесне, а хто за духом про духовне.
\end{tcolorbox}
\begin{tcolorbox}
\textsubscript{6} Бо думка тілесна то смерть, а думка духовна життя та мир,
\end{tcolorbox}
\begin{tcolorbox}
\textsubscript{7} думка бо тілесна ворожнеча на Бога, бо не кориться Законові Божому, та й не може.
\end{tcolorbox}
\begin{tcolorbox}
\textsubscript{8} І ті, хто ходить за тілом, не можуть догодити Богові.
\end{tcolorbox}
\begin{tcolorbox}
\textsubscript{9} А ви не в тілі, але в дусі, бо Дух Божий живе в вас. А коли хто не має Христового Духа, той не Його.
\end{tcolorbox}
\begin{tcolorbox}
\textsubscript{10} А коли Христос у вас, то хоч тіло мертве через гріх, але дух живий через праведність.
\end{tcolorbox}
\begin{tcolorbox}
\textsubscript{11} А коли живе в вас Дух Того, Хто воскресив Ісуса з мертвих, то Той, хто підняв Христа з мертвих, оживить і смертельні тіла ваші через Свого Духа, що живе в вас.
\end{tcolorbox}
\begin{tcolorbox}
\textsubscript{12} Тому то, браття, ми не боржники тіла, щоб жити за тілом;
\end{tcolorbox}
\begin{tcolorbox}
\textsubscript{13} бо коли живете за тілом, то маєте вмерти, а коли духом умертвляєте тілесні вчинки, то будете жити.
\end{tcolorbox}
\begin{tcolorbox}
\textsubscript{14} Бо всі, хто водиться Духом Божим, вони сини Божі;
\end{tcolorbox}
\begin{tcolorbox}
\textsubscript{15} бо не взяли ви духа неволі знов на страх, але взяли ви Духа синівства, що через Нього кличемо: Авва, Отче!
\end{tcolorbox}
\begin{tcolorbox}
\textsubscript{16} Сам Цей Дух свідчить разом із духом нашим, що ми діти Божі.
\end{tcolorbox}
\begin{tcolorbox}
\textsubscript{17} А коли діти, то й спадкоємці, спадкоємці ж Божі, а співспадкоємці Христові, коли тільки разом із Ним ми терпимо, щоб разом із Ним і прославитись.
\end{tcolorbox}
\begin{tcolorbox}
\textsubscript{18} Бо я думаю, що страждання теперішнього часу нічого не варті супроти тієї слави, що має з'явитися в нас.
\end{tcolorbox}
\begin{tcolorbox}
\textsubscript{19} Бо чекання створіння очікує з'явлення синів Божих,
\end{tcolorbox}
\begin{tcolorbox}
\textsubscript{20} бо створіння покорилось марноті не добровільно, але через того, хто скорив його, в надії,
\end{tcolorbox}
\begin{tcolorbox}
\textsubscript{21} що й саме створіння визволиться від неволі тління на волю слави синів Божих.
\end{tcolorbox}
\begin{tcolorbox}
\textsubscript{22} Бо знаємо, що все створіння разом зідхає й разом мучиться аж досі.
\end{tcolorbox}
\begin{tcolorbox}
\textsubscript{23} Але не тільки воно, але й ми самі, маючи зачаток Духа, і ми самі в собі зідхаємо, очікуючи синівства, відкуплення нашого тіла.
\end{tcolorbox}
\begin{tcolorbox}
\textsubscript{24} Надією бо ми спаслися. Надія ж, коли бачить, не є надія, бо хто що бачить, чому б того й надіявся?
\end{tcolorbox}
\begin{tcolorbox}
\textsubscript{25} А коли сподіваємось, чого не бачимо, то очікуємо того з терпеливістю.
\end{tcolorbox}
\begin{tcolorbox}
\textsubscript{26} Так само ж і Дух допомагає нам у наших немочах; бо ми не знаємо, про що маємо молитись, як належить, але Сам Дух заступається за нас невимовними зідханнями.
\end{tcolorbox}
\begin{tcolorbox}
\textsubscript{27} А Той, Хто досліджує серця, знає, яка думка Духа, бо з волі Божої заступається за святих.
\end{tcolorbox}
\begin{tcolorbox}
\textsubscript{28} І знаємо, що тим, хто любить Бога, хто покликаний Його постановою, усе допомагає на добре.
\end{tcolorbox}
\begin{tcolorbox}
\textsubscript{29} Бо кого Він передбачив, тих і призначив, щоб були подібні до образу Сина Його, щоб Він був перворідним поміж багатьма братами.
\end{tcolorbox}
\begin{tcolorbox}
\textsubscript{30} А кого Він призначив, тих і покликав, а кого покликав, тих і виправдав, а кого виправдав, тих і прославив.
\end{tcolorbox}
\begin{tcolorbox}
\textsubscript{31} Що ж скажем на це? Коли за нас Бог, то хто проти нас?
\end{tcolorbox}
\begin{tcolorbox}
\textsubscript{32} Той же, Хто Сина Свого не пожалів, але видав Його за всіх нас, як же не дав би Він нам із Ним і всього?
\end{tcolorbox}
\begin{tcolorbox}
\textsubscript{33} Хто оскаржувати буде Божих вибранців? Бог Той, що виправдує.
\end{tcolorbox}
\begin{tcolorbox}
\textsubscript{34} Хто ж той, що засуджує? Христос Ісус є Той, що вмер, надто й воскрес, Він праворуч Бога, і Він і заступається за нас.
\end{tcolorbox}
\begin{tcolorbox}
\textsubscript{35} Хто нас розлучить від любови Христової? Чи недоля, чи утиск, чи переслідування, чи голод, чи нагота, чи небезпека, чи меч?
\end{tcolorbox}
\begin{tcolorbox}
\textsubscript{36} Як написано: За Тебе нас цілий день умертвляють, нас уважають за овець, приречених на заколення.
\end{tcolorbox}
\begin{tcolorbox}
\textsubscript{37} Але в цьому всьому ми перемагаємо Тим, Хто нас полюбив.
\end{tcolorbox}
\begin{tcolorbox}
\textsubscript{38} Бо я пересвідчився, що ні смерть, ні життя, ні Анголи, ні влади, ні теперішнє, ні майбутнє, ні сили,
\end{tcolorbox}
\begin{tcolorbox}
\textsubscript{39} ні вишина, ні глибина, ані інше яке створіння не зможе відлучити нас від любови Божої, яка в Христі Ісусі, Господі нашім!
\end{tcolorbox}
\subsection{CHAPTER 9}
\begin{tcolorbox}
\textsubscript{1} Кажу правду в Христі, не обманюю, як свідчить мені моє сумління через Духа Святого,
\end{tcolorbox}
\begin{tcolorbox}
\textsubscript{2} що маю велику скорботу й невпинну муку для серця свого!
\end{tcolorbox}
\begin{tcolorbox}
\textsubscript{3} Бо я бажав би сам бути відлучений від Христа замість братів моїх, рідних мені тілом;
\end{tcolorbox}
\begin{tcolorbox}
\textsubscript{4} вони ізраїльтяни, що їм належить синівство, і слава, і заповіти, і законодавство, і Богослужба, і обітниці,
\end{tcolorbox}
\begin{tcolorbox}
\textsubscript{5} що їхні й отці, і від них же тілом Христос, що Він над усіма Бог, благословенний, навіки, амінь.
\end{tcolorbox}
\begin{tcolorbox}
\textsubscript{6} Не так, щоб Слово Боже не збулося. Бо не всі ті ізраїльтяни, хто від Ізраїля,
\end{tcolorbox}
\begin{tcolorbox}
\textsubscript{7} і не всі діти Авраамові, хто від насіння його, але: в Ісаку буде насіння тобі.
\end{tcolorbox}
\begin{tcolorbox}
\textsubscript{8} Цебто, не тілесні діти то діти Божі, але діти обітниці признаються за насіння.
\end{tcolorbox}
\begin{tcolorbox}
\textsubscript{9} А слово обітниці таке: На той час прийду, і буде син у Сари.
\end{tcolorbox}
\begin{tcolorbox}
\textsubscript{10} І не тільки це, але й Ревекка зачала дітей від одного ложа отця нашого Ісака,
\end{tcolorbox}
\begin{tcolorbox}
\textsubscript{11} бо коли вони ще не народились, і нічого доброго чи злого не вчинили, щоб позосталась постанова Божа у вибранні
\end{tcolorbox}
\begin{tcolorbox}
\textsubscript{12} не від учинків, але від Того, Хто кличе, сказано їй: Більший служитиме меншому,
\end{tcolorbox}
\begin{tcolorbox}
\textsubscript{13} як і написано: Полюбив Я Якова, а Ісава зненавидів.
\end{tcolorbox}
\begin{tcolorbox}
\textsubscript{14} Що ж скажемо? Може в Бога неправда? Зовсім ні!
\end{tcolorbox}
\begin{tcolorbox}
\textsubscript{15} Бо Він каже Мойсеєві: Помилую, кого хочу помилувати, і змилосерджуся, над ким хочу змилосердитись.
\end{tcolorbox}
\begin{tcolorbox}
\textsubscript{16} Отож, не залежить це ні від того, хто хоче, ні від того, хто біжить, але від Бога, що милує.
\end{tcolorbox}
\begin{tcolorbox}
\textsubscript{17} Бо Писання говорить фараонові: Власне на те Я поставив тебе, щоб на тобі показати Свою силу, і щоб звістилось по цілій землі Моє Ймення.
\end{tcolorbox}
\begin{tcolorbox}
\textsubscript{18} Отож, кого хоче Він милує, і кого хоче ожорсточує.
\end{tcolorbox}
\begin{tcolorbox}
\textsubscript{19} А ти скажеш мені: Чого ж іще Він докоряє, бо хто може противитись волі Його?
\end{tcolorbox}
\begin{tcolorbox}
\textsubscript{20} Отже, хто ти, чоловіче, що ти сперечаєшся з Богом? Чи скаже твориво творцеві: Пощо ти зробив мене так?
\end{tcolorbox}
\begin{tcolorbox}
\textsubscript{21} Чи ганчар не має влади над глиною, щоб із того самого місива зробити одну посудину на честь, а одну на нечесть?
\end{tcolorbox}
\begin{tcolorbox}
\textsubscript{22} Тож Бог, бажаючи показати гнів і виявити могутність Свою, щадив із великим терпінням посудини гніву, що готові були на погибіль,
\end{tcolorbox}
\begin{tcolorbox}
\textsubscript{23} і щоб виявити багатство слави Своєї на посудинах милосердя, що їх приготував на славу,
\end{tcolorbox}
\begin{tcolorbox}
\textsubscript{24} на нас, що їх і покликав не тільки від юдеїв, але й від поган.
\end{tcolorbox}
\begin{tcolorbox}
\textsubscript{25} Як і в Осії Він говорить: Назву Своїм народом не людей Моїх, і не улюблену улюбленою,
\end{tcolorbox}
\begin{tcolorbox}
\textsubscript{26} і на місці, де сказано їм: Ви не Мій народ, там названі будуть синами Бога Живого!
\end{tcolorbox}
\begin{tcolorbox}
\textsubscript{27} А Ісая взиває про Ізраїля: Коли б число синів Ізраїлевих було, як морський пісок, то тільки останок спасеться,
\end{tcolorbox}
\begin{tcolorbox}
\textsubscript{28} бо вирок закінчений та скорочений учинить Господь на землі!
\end{tcolorbox}
\begin{tcolorbox}
\textsubscript{29} І як Ісая віщував: Коли б Господь Саваот не лишив нам насіння, то ми стали б, як Содом, і подібні були б до Гоморри!
\end{tcolorbox}
\begin{tcolorbox}
\textsubscript{30} Що ж скажемо? Що погани, які не шукали праведности, досягли праведности, тієї праведности, що від віри,
\end{tcolorbox}
\begin{tcolorbox}
\textsubscript{31} а Ізраїль, що шукав Закона праведности, не досяг Закону праведности.
\end{tcolorbox}
\begin{tcolorbox}
\textsubscript{32} Чому? Бо шукали не з віри, але якби з учинків Закону; вони бо спіткнулись об камінь спотикання,
\end{tcolorbox}
\begin{tcolorbox}
\textsubscript{33} як написано: Ось Я кладу на Сіоні камінь спотикання та скелю спокуси, і кожен, хто вірує в Нього, не посоромиться!
\end{tcolorbox}
\subsection{CHAPTER 10}
\begin{tcolorbox}
\textsubscript{1} Браття, бажання мого серця й молитва до Бога за Ізраїля на спасіння.
\end{tcolorbox}
\begin{tcolorbox}
\textsubscript{2} Бо я свідчу їм, що вони мають ревність про Бога, але не за розумом.
\end{tcolorbox}
\begin{tcolorbox}
\textsubscript{3} Вони бо, не розуміючи праведности Божої, і силкуючись поставити власну праведність, не покорились праведності Божій.
\end{tcolorbox}
\begin{tcolorbox}
\textsubscript{4} Бо кінець Закону Христос на праведність кожному, хто вірує.
\end{tcolorbox}
\begin{tcolorbox}
\textsubscript{5} Мойсей бо пише про праведність, що від Закону, що людина, яка його виконує, буде ним жити.
\end{tcolorbox}
\begin{tcolorbox}
\textsubscript{6} А про праведність, що від віри, говорить так: Не кажи в своїм серці: Хто вийде на небо? цебто звести додолу Христа,
\end{tcolorbox}
\begin{tcolorbox}
\textsubscript{7} або: Хто зійде в безодню? цебто вивести з мертвих Христа.
\end{tcolorbox}
\begin{tcolorbox}
\textsubscript{8} Але що каже ще? Близько тебе слово, в устах твоїх і в серці твоїм, цебто слово віри, що його проповідуємо.
\end{tcolorbox}
\begin{tcolorbox}
\textsubscript{9} Бо коли ти устами своїми визнаватимеш Ісуса за Господа, і будеш вірувати в своїм серці, що Бог воскресив Його з мертвих, то спасешся,
\end{tcolorbox}
\begin{tcolorbox}
\textsubscript{10} бо серцем віруємо для праведности, а устами ісповідуємо для спасіння.
\end{tcolorbox}
\begin{tcolorbox}
\textsubscript{11} Каже бо Писання: Кожен, хто вірує в Нього, не буде засоромлений.
\end{tcolorbox}
\begin{tcolorbox}
\textsubscript{12} Бо нема різниці поміж юдеєм та гелленом, бо той же Господь є Господом усіх, багатий для всіх, хто кличе Його.
\end{tcolorbox}
\begin{tcolorbox}
\textsubscript{13} Бо кожен, хто покличе Господнє Ім'я, буде спасений.
\end{tcolorbox}
\begin{tcolorbox}
\textsubscript{14} Але як покличуть Того, в Кого не ввірували? А як увірують у Того, що про Нього не чули? А як почують без проповідника?
\end{tcolorbox}
\begin{tcolorbox}
\textsubscript{15} І як будуть проповідувати, коли не будуть послані? Як написано: Які гарні ноги благовісників миру, благовісників добра.
\end{tcolorbox}
\begin{tcolorbox}
\textsubscript{16} Але не всі послухались Євангелії. Бо Ісая каже: Господи, хто повірив тому, що почув був від нас?
\end{tcolorbox}
\begin{tcolorbox}
\textsubscript{17} Тож віра від слухання, а слухання через Слово Христове.
\end{tcolorbox}
\begin{tcolorbox}
\textsubscript{18} Та кажу: Чи не чули вони? Отож: По всій землі їхній голос пішов, і їхні слова в кінці світу!
\end{tcolorbox}
\begin{tcolorbox}
\textsubscript{19} Але кажу: Чи Ізраїль не знав? Перший Мойсей говорить: Я викличу заздрість у вас ненародом, роздражню вас нерозумним народом.
\end{tcolorbox}
\begin{tcolorbox}
\textsubscript{20} А Ісая сміливо говорить: Знайшли Мене ті, хто Мене не шукав, відкрився Я тим, хто не питався про Мене!
\end{tcolorbox}
\begin{tcolorbox}
\textsubscript{21} А про Ізраїля каже: Я руки Свої цілий день простягав до людей неслухняних і суперечних!
\end{tcolorbox}
\subsection{CHAPTER 11}
\begin{tcolorbox}
\textsubscript{1} Отож я питаю: Чи ж Бог відкинув народа Свого? Зовсім ні! Бо й я ізраїльтянин, із насіння Авраамового, Веніяминового племени.
\end{tcolorbox}
\begin{tcolorbox}
\textsubscript{2} Не відкинув Бог народа Свого, що його перше знав. Чи ви не знаєте, що говорить Писання, де про Іллю, як він скаржиться Богові на Ізраїля, кажучи:
\end{tcolorbox}
\begin{tcolorbox}
\textsubscript{3} Господи, вони повбивали пророків Твоїх, і Твої жертівники поруйнували, і лишився я сам, і шукають моєї душі.
\end{tcolorbox}
\begin{tcolorbox}
\textsubscript{4} Та що каже йому Божа відповідь: Я для Себе зоставив сім тисяч мужа, що перед Ваалом колін не схилили.
\end{tcolorbox}
\begin{tcolorbox}
\textsubscript{5} Також і теперішнього часу залишився останок за вибором благодаті.
\end{tcolorbox}
\begin{tcolorbox}
\textsubscript{6} А коли за благодаттю, то не з учинків, інакше благодать не була б благодаттю. А коли з учинків, то це більше не благодать, інакше вчинок не є вже вчинок.
\end{tcolorbox}
\begin{tcolorbox}
\textsubscript{7} Що ж? Чого Ізраїль шукає, того не одержав, та одержали вибрані, а останні затверділи,
\end{tcolorbox}
\begin{tcolorbox}
\textsubscript{8} як написано: Бог дав їм духа засипання, очі, щоб не бачили, і вуха, щоб не чули, аж до сьогоднішнього дня.
\end{tcolorbox}
\begin{tcolorbox}
\textsubscript{9} А Давид каже: Нехай станеться стіл їхній за сітку й за пастку, і на спокусу, та їм на заплату;
\end{tcolorbox}
\begin{tcolorbox}
\textsubscript{10} нехай потемніють їхні очі, щоб не бачили, хай назавжди зігнеться хребет їхній!
\end{tcolorbox}
\begin{tcolorbox}
\textsubscript{11} Тож питаю: Чи ж спіткнулись вони, щоб упасти? Зовсім ні! Але з їхнього занепаду спасіння поганам, щоб викликати заздрість у них.
\end{tcolorbox}
\begin{tcolorbox}
\textsubscript{12} А коли їхній занепад багатство для світу, а їхнє упокорення багатство поганам, скільки ж більш повнота їхня?
\end{tcolorbox}
\begin{tcolorbox}
\textsubscript{13} Кажу бо я вам, поганам: через те, що я апостол поганів, я хвалю свою службу,
\end{tcolorbox}
\begin{tcolorbox}
\textsubscript{14} може як викличу заздрість у своїх за тілом, і спасу декого з них.
\end{tcolorbox}
\begin{tcolorbox}
\textsubscript{15} Коли ж відкинення їх то примирення світу, то що їхнє прийняття, як не життя з мертвих?
\end{tcolorbox}
\begin{tcolorbox}
\textsubscript{16} А коли святий первісток, то й тісто святе; а коли святий корінь, то й віття святе.
\end{tcolorbox}
\begin{tcolorbox}
\textsubscript{17} Коли ж деякі з галузок відломилися, а ти, бувши дике оливне дерево, прищепився між них і став спільником товщу оливного кореня,
\end{tcolorbox}
\begin{tcolorbox}
\textsubscript{18} то не вихваляйся перед галузками; а коли вихваляєшся, то знай, що не ти носиш кореня, але корінь тебе.
\end{tcolorbox}
\begin{tcolorbox}
\textsubscript{19} Отже скажеш: Галузки відломилися, щоб я прищепився.
\end{tcolorbox}
\begin{tcolorbox}
\textsubscript{20} Добре. Вони відломились невірством, а ти тримаєшся вірою; не величайся, але бійся.
\end{tcolorbox}
\begin{tcolorbox}
\textsubscript{21} Бо коли Бог природних галузок не пожалував, то Він і тебе не пожалує!
\end{tcolorbox}
\begin{tcolorbox}
\textsubscript{22} Отже, бач добрість і суворість Божу, на відпалих суворість, а на тебе добрість Божа, коли перебудеш у добрості, коли ж ні, то й ти будеш відтятий.
\end{tcolorbox}
\begin{tcolorbox}
\textsubscript{23} Та й вони, коли не зостануться в невірстві, прищепляться, бо має Бог силу їх знов прищепити.
\end{tcolorbox}
\begin{tcolorbox}
\textsubscript{24} Бо коли ти відтятий з оливки, дикої з природи, і проти природи защеплений до доброї оливки, то скільки ж більше ті, що природні, прищепляться до своєї власної оливки?
\end{tcolorbox}
\begin{tcolorbox}
\textsubscript{25} Бо не хочу я, браття, щоб ви не знали цієї таємниці, щоб не були ви високої думки про себе, що жорстокість сталась Ізраїлеві почасти, аж поки не ввійде повне число поган,
\end{tcolorbox}
\begin{tcolorbox}
\textsubscript{26} і так увесь Ізраїль спасеться, як написано: Прийде з Сіону Спаситель, і відверне безбожність від Якова,
\end{tcolorbox}
\begin{tcolorbox}
\textsubscript{27} і це заповіт їм від Мене, коли відійму гріхи їхні!
\end{tcolorbox}
\begin{tcolorbox}
\textsubscript{28} Тож вони за Євангелією вороги ради вас, а за вибором улюблені ради отців.
\end{tcolorbox}
\begin{tcolorbox}
\textsubscript{29} Бо дари й покликання Божі невідмінні.
\end{tcolorbox}
\begin{tcolorbox}
\textsubscript{30} Бо як і ви були колись неслухняні Богові, а тепер помилувані через їхній непослух,
\end{tcolorbox}
\begin{tcolorbox}
\textsubscript{31} так і вони тепер спротивились для помилування вас, щоб і самі були помилувані.
\end{tcolorbox}
\begin{tcolorbox}
\textsubscript{32} Бо замкнув Бог усіх у непослух, щоб помилувати всіх.
\end{tcolorbox}
\begin{tcolorbox}
\textsubscript{33} О глибино багатства, і премудрости, і знання Божого! Які недовідомі присуди Його, і недосліджені дороги Його!
\end{tcolorbox}
\begin{tcolorbox}
\textsubscript{34} Бо хто розум Господній пізнав? Або хто був дорадник Йому?
\end{tcolorbox}
\begin{tcolorbox}
\textsubscript{35} Або хто давніш Йому дав, і йому буде віддано?
\end{tcolorbox}
\begin{tcolorbox}
\textsubscript{36} Бо все з Нього, через Нього і для Нього! Йому слава навіки. Амінь.
\end{tcolorbox}
\subsection{CHAPTER 12}
\begin{tcolorbox}
\textsubscript{1} Тож благаю вас, браття, через Боже милосердя, повіддавайте ваші тіла на жертву живу, святу, приємну Богові, як розумну службу вашу,
\end{tcolorbox}
\begin{tcolorbox}
\textsubscript{2} і не стосуйтесь до віку цього, але перемініться відновою вашого розуму, щоб пізнати вам, що то є воля Божа, добро, приємність та досконалість.
\end{tcolorbox}
\begin{tcolorbox}
\textsubscript{3} Через дану мені благодать кажу кожному з вас не думати про себе більш, ніж належить думати, але думати скромно, у міру віри, як кожному Бог наділив.
\end{tcolorbox}
\begin{tcolorbox}
\textsubscript{4} Бо як в однім тілі маємо багато членів, а всі члени мають не однакове діяння,
\end{tcolorbox}
\begin{tcolorbox}
\textsubscript{5} так багато нас є одне тіло в Христі, а зосібна ми один одному члени.
\end{tcolorbox}
\begin{tcolorbox}
\textsubscript{6} І ми маємо різні дари, згідно з благодаттю, даною нам: коли пророцтво то виконуй його в міру віри,
\end{tcolorbox}
\begin{tcolorbox}
\textsubscript{7} а коли служіння будь на служіння, коли вчитель на навчання,
\end{tcolorbox}
\begin{tcolorbox}
\textsubscript{8} коли втішитель на потішання, хто подає у простоті, хто головує то з пильністю, хто милосердствує то з привітністю!
\end{tcolorbox}
\begin{tcolorbox}
\textsubscript{9} Любов нехай буде нелицемірна; ненавидьте зло та туліться до доброго!
\end{tcolorbox}
\begin{tcolorbox}
\textsubscript{10} Любіть один одного братньою любов'ю; випереджайте один одного пошаною!
\end{tcolorbox}
\begin{tcolorbox}
\textsubscript{11} У ревності не лінуйтеся, духом палайте, служіть Господеві,
\end{tcolorbox}
\begin{tcolorbox}
\textsubscript{12} тіштесь надією, утиски терпіть, перебувайте в молитві,
\end{tcolorbox}
\begin{tcolorbox}
\textsubscript{13} беріть уділ у потребах святих, будьте гостинні до чужинців!
\end{tcolorbox}
\begin{tcolorbox}
\textsubscript{14} Благословляйте тих, хто вас переслідує; благословляйте, а не проклинайте!
\end{tcolorbox}
\begin{tcolorbox}
\textsubscript{15} Тіштеся з тими, хто тішиться, і плачте з отими, хто плаче!
\end{tcolorbox}
\begin{tcolorbox}
\textsubscript{16} Думайте між собою однаково; не величайтеся, але наслідуйте слухняних; не вважайте за мудрих себе!
\end{tcolorbox}
\begin{tcolorbox}
\textsubscript{17} Не платіть нікому злом за зло, дбайте про добре перед усіма людьми!
\end{tcolorbox}
\begin{tcolorbox}
\textsubscript{18} Коли можливо, якщо це залежить від вас, живіть у мирі зо всіма людьми!
\end{tcolorbox}
\begin{tcolorbox}
\textsubscript{19} Не мстіться самі, улюблені, але дайте місце гніву Божому, бо написано: Мені помста належить, Я відплачу, говорить Господь.
\end{tcolorbox}
\begin{tcolorbox}
\textsubscript{20} Отож, як твій ворог голодний, нагодуй його; як він прагне, напій його, бо, роблячи це, ти згортаєш розпалене вугілля йому на голову.
\end{tcolorbox}
\begin{tcolorbox}
\textsubscript{21} Не будь переможений злом, але перемагай зло добром!
\end{tcolorbox}
\subsection{CHAPTER 13}
\begin{tcolorbox}
\textsubscript{1} Нехай кожна людина кориться вищій владі, бо немає влади, як не від Бога, і влади існуючі встановлені від Бога.
\end{tcolorbox}
\begin{tcolorbox}
\textsubscript{2} Тому той, хто противиться владі, противиться Божій постанові; а ті, хто противиться, самі візьмуть осуд на себе.
\end{tcolorbox}
\begin{tcolorbox}
\textsubscript{3} Бо володарі пострах не на добрі діла, а на злі. Хочеш не боятися влади? Роби добро, і матимеш похвалу від неї,
\end{tcolorbox}
\begin{tcolorbox}
\textsubscript{4} бо володар Божий слуга, тобі на добро. А як чиниш ти зле, то бійся, бо недармо він носить меча, він бо Божий слуга, месник у гніві злочинцеві!
\end{tcolorbox}
\begin{tcolorbox}
\textsubscript{5} Тому треба коритися не тільки ради страху кари, але й ради сумління.
\end{tcolorbox}
\begin{tcolorbox}
\textsubscript{6} Через це ви й податки даєте, бо вони служителі Божі, саме тим завжди зайняті.
\end{tcolorbox}
\begin{tcolorbox}
\textsubscript{7} Тож віддайте належне усім: кому податок податок, кому мито мито, кому страх страх, кому честь честь.
\end{tcolorbox}
\begin{tcolorbox}
\textsubscript{8} Не будьте винні нікому нічого, крім того, щоб любити один одного. Бо хто іншого любить, той виконав Закона.
\end{tcolorbox}
\begin{tcolorbox}
\textsubscript{9} Бо заповіді: Не чини перелюбу, Не вбивай, Не кради, Не свідкуй неправдиво, Не пожадай й які інші, вони містяться всі в цьому слові: Люби свого ближнього, як самого себе!
\end{tcolorbox}
\begin{tcolorbox}
\textsubscript{10} Любов не чинить зла ближньому, тож любов виконання Закону.
\end{tcolorbox}
\begin{tcolorbox}
\textsubscript{11} І це тому, що знаєте час, що пора нам уже пробудитись від сну. Бо тепер спасіння ближче до нас, аніж тоді, коли ми ввірували.
\end{tcolorbox}
\begin{tcolorbox}
\textsubscript{12} Ніч минула, а день наблизився, тож відкиньмо вчинки темряви й зодягнімось у зброю світла.
\end{tcolorbox}
\begin{tcolorbox}
\textsubscript{13} Як удень, поступаймо доброчесно, не в гульні та п'янстві, не в перелюбі та розпусті, не в сварні та заздрощах,
\end{tcolorbox}
\begin{tcolorbox}
\textsubscript{14} але зодягніться Господом Ісусом Христом, а догодження тілу не обертайте на пожадливість!
\end{tcolorbox}
\subsection{CHAPTER 14}
\begin{tcolorbox}
\textsubscript{1} Слабого в вірі приймайте, але не для суперечок про погляди.
\end{tcolorbox}
\begin{tcolorbox}
\textsubscript{2} Один бо вірує, що можна їсти все, а немічний споживає ярину.
\end{tcolorbox}
\begin{tcolorbox}
\textsubscript{3} Хто їсть, нехай не погорджує тим, хто не їсть. А хто не їсть, нехай не осуджує того, хто їсть, Бог бо прийняв його.
\end{tcolorbox}
\begin{tcolorbox}
\textsubscript{4} Ти хто такий, що судиш чужого раба? Він для пана свого стоїть або падає; але він устоїть, бо має Бог силу поставити його.
\end{tcolorbox}
\begin{tcolorbox}
\textsubscript{5} Один вирізнює день від дня, інший же про кожен день судить однаково. Нехай кожен за власною думкою тримається свого переконання.
\end{tcolorbox}
\begin{tcolorbox}
\textsubscript{6} Хто вважає на день, для Господа вважає, а хто не вважає на день, для Господа не вважає. Хто їсть, для Господа їсть, бо дякує Богові. А хто не їсть, для Господа не їсть, і дякує Богові.
\end{tcolorbox}
\begin{tcolorbox}
\textsubscript{7} Бо ніхто з нас не живе сам для себе, і не вмирає ніхто сам для себе.
\end{tcolorbox}
\begin{tcolorbox}
\textsubscript{8} Бо коли живемо для Господа живемо, і коли вмираємо для Господа вмираємо. І чи живемо, чи вмираємо ми Господні!
\end{tcolorbox}
\begin{tcolorbox}
\textsubscript{9} Бо Христос на те й умер, і ожив, щоб панувати і над мертвими, і над живими.
\end{tcolorbox}
\begin{tcolorbox}
\textsubscript{10} А ти нащо осуджуєш брата свого? Чи чого ти погорджуєш братом своїм? Бо всі станемо перед судним престолом Божим.
\end{tcolorbox}
\begin{tcolorbox}
\textsubscript{11} Бо написано: Я живу, каже Господь, і схилиться кожне коліно передо Мною, і визнає Бога кожен язик!
\end{tcolorbox}
\begin{tcolorbox}
\textsubscript{12} Тому кожен із нас сам за себе дасть відповідь Богові.
\end{tcolorbox}
\begin{tcolorbox}
\textsubscript{13} Отож, не будемо більше осуджувати один одного, але краще судіть про те, щоб не давати братові спотикання та спокуси.
\end{tcolorbox}
\begin{tcolorbox}
\textsubscript{14} Я знаю, і пересвідчений у Господі Ісусі, що нема нічого нечистого в самому собі; тільки коли хто вважає що за нечисте, тому воно нечисте.
\end{tcolorbox}
\begin{tcolorbox}
\textsubscript{15} Коли ж через поживу сумує твій брат, то вже не за любов'ю поводишся ти, не губи своєю поживою того, за кого Христос був умер.
\end{tcolorbox}
\begin{tcolorbox}
\textsubscript{16} Нехай ваше добре не зневажається.
\end{tcolorbox}
\begin{tcolorbox}
\textsubscript{17} Бо Царство Боже не пожива й питво, але праведність, і мир, і радість у Дусі Святім.
\end{tcolorbox}
\begin{tcolorbox}
\textsubscript{18} Хто цим служить Христові, той Богові милий і шанований поміж людьми.
\end{tcolorbox}
\begin{tcolorbox}
\textsubscript{19} Отож, пильнуймо про мир, та про те, що на збудування один одного!
\end{tcolorbox}
\begin{tcolorbox}
\textsubscript{20} Не руйнуй діла Божого ради поживи, усе бо чисте, але зле людині, що їсть на спотикання.
\end{tcolorbox}
\begin{tcolorbox}
\textsubscript{21} Добре не їсти м'яса, ані пити вина, ані робити такого, від чого брат твій гіршиться, або спокушується, або слабне.
\end{tcolorbox}
\begin{tcolorbox}
\textsubscript{22} Ти маєш віру? Май її сам про себе перед Богом. Блаженний той, хто не осуджує самого себе за те, про що випробовується!
\end{tcolorbox}
\begin{tcolorbox}
\textsubscript{23} А хто має сумнів, коли їсть, буде осуджений, бо не робить із віри, а що не від віри, те гріх.
\end{tcolorbox}
\subsection{CHAPTER 15}
\begin{tcolorbox}
\textsubscript{1} Ми, сильні, повинні нести слабості безсилих, а не собі догоджати.
\end{tcolorbox}
\begin{tcolorbox}
\textsubscript{2} Кожен із нас нехай догоджає ближньому на добро для збудування.
\end{tcolorbox}
\begin{tcolorbox}
\textsubscript{3} Бо й Христос не Собі догоджав, але як написано: Зневаги тих, хто Тебе зневажає, упали на Мене.
\end{tcolorbox}
\begin{tcolorbox}
\textsubscript{4} А все, що давніше написане, написане нам на науку, щоб терпінням і потіхою з Писання ми мали надію.
\end{tcolorbox}
\begin{tcolorbox}
\textsubscript{5} А Бог терпеливости й потіхи нехай дасть вам бути однодумними між собою за Христом Ісусом,
\end{tcolorbox}
\begin{tcolorbox}
\textsubscript{6} щоб ви однодушно, одними устами славили Бога й Отця Господа нашого Ісуса Христа.
\end{tcolorbox}
\begin{tcolorbox}
\textsubscript{7} Приймайте тому один одного, як і Христос прийняв нас до Божої слави.
\end{tcolorbox}
\begin{tcolorbox}
\textsubscript{8} Кажу ж, що Христос для обрізаних став за служку ради Божої правди, щоб отцям потвердити обітниці,
\end{tcolorbox}
\begin{tcolorbox}
\textsubscript{9} а для поган щоб славили Бога за милосердя, як написано: Тому я хвалитиму Тебе, Господи, серед поган, і Ім'я Твоє буду виспівувати!
\end{tcolorbox}
\begin{tcolorbox}
\textsubscript{10} І ще каже: Тіштесь, погани, з народом Його!
\end{tcolorbox}
\begin{tcolorbox}
\textsubscript{11} І ще: Хваліть, усі погани, Господа, виславляйте Його, усі люди!
\end{tcolorbox}
\begin{tcolorbox}
\textsubscript{12} І ще каже Ісая: Буде корінь Єссеїв, що постане, щоб панувати над поганами, погани на Нього надіятись будуть!
\end{tcolorbox}
\begin{tcolorbox}
\textsubscript{13} Бог же надії нехай вас наповнить усякою радістю й миром у вірі, щоб ви збагатились надією, силою Духа Святого!
\end{tcolorbox}
\begin{tcolorbox}
\textsubscript{14} І я про вас сам пересвідчений, браття мої, що й самі ви повні добрости, наповнені всяким знанням, і можете й один одного навчати.
\end{tcolorbox}
\begin{tcolorbox}
\textsubscript{15} А писав я вам почасти трохи сміліше, якби вам нагадуючи благодаттю, що дана мені від Бога,
\end{tcolorbox}
\begin{tcolorbox}
\textsubscript{16} щоб був я слугою Христа Ісуса між поганами, і виконував святу службу Євангелії Божої, щоб приношення поган стало приємне й освячене Духом Святим.
\end{tcolorbox}
\begin{tcolorbox}
\textsubscript{17} Тож маю я чим похвалитись у Христі Ісусі, щодо Божих речей,
\end{tcolorbox}
\begin{tcolorbox}
\textsubscript{18} бо не смію казати того, чого не зробив через мене Христос на послух поган, словом і чином,
\end{tcolorbox}
\begin{tcolorbox}
\textsubscript{19} силою ознак і чудес, силою Духа Божого, так що я поширив Євангелію Христову від Єрусалиму й околиць аж до Ілліріка.
\end{tcolorbox}
\begin{tcolorbox}
\textsubscript{20} При тому пильнував я звіщати Євангелію не там, де Христове Ім'я було знане, щоб не будувати на основі чужій,
\end{tcolorbox}
\begin{tcolorbox}
\textsubscript{21} але як написано: Кому не звіщалось про Нього, побачать, і ті, хто не чув, зрозуміють!
\end{tcolorbox}
\begin{tcolorbox}
\textsubscript{22} Тому часто я мав перешкоди, щоб прибути до вас.
\end{tcolorbox}
\begin{tcolorbox}
\textsubscript{23} А тепер, не маючи більше місця в країнах оцих, але з давніх літ мавши бажання прибути до вас,
\end{tcolorbox}
\begin{tcolorbox}
\textsubscript{24} коли тільки піду до Еспанії, прибуду до вас. Бо маю надію, як буду проходити, побачити вас, і що ви проведете мене туди, коли перше почасти матиму я задоволення з вами побути.
\end{tcolorbox}
\begin{tcolorbox}
\textsubscript{25} А тепер я йду до Єрусалиму послужити святим,
\end{tcolorbox}
\begin{tcolorbox}
\textsubscript{26} бо Македонія й Ахая визнали за добре подати деяку поміч незаможним святим, що в Єрусалимі живуть.
\end{tcolorbox}
\begin{tcolorbox}
\textsubscript{27} Бо визнали за добре, та й боржники вони їхні. Бо коли погани стали спільниками в їх духовнім, то повинні й у тілеснім послужити їм.
\end{tcolorbox}
\begin{tcolorbox}
\textsubscript{28} Як це докінчу та достачу їм плід цей, тоді через ваше місто я піду до Еспанії.
\end{tcolorbox}
\begin{tcolorbox}
\textsubscript{29} І знаю, що коли прийду до вас, то прийду в повноті Христового благословення.
\end{tcolorbox}
\begin{tcolorbox}
\textsubscript{30} Благаю ж вас, браття, Господом нашим Ісусом Христом і любов'ю Духа, помагайте мені в молитвах за мене до Бога,
\end{tcolorbox}
\begin{tcolorbox}
\textsubscript{31} щоб мені визволитися від неслухняних в Юдеї, і щоб служба моя в Єрусалимі була приємна святим,
\end{tcolorbox}
\begin{tcolorbox}
\textsubscript{32} щоб із волі Божої з радістю прийти до вас і відпочити з вами!
\end{tcolorbox}
\begin{tcolorbox}
\textsubscript{33} А Бог миру нехай буде зо всіма вами. Амінь.
\end{tcolorbox}
\subsection{CHAPTER 16}
\begin{tcolorbox}
\textsubscript{1} Поручаю ж вам сестру нашу Фіву, служебницю Церкви в Кенхреях,
\end{tcolorbox}
\begin{tcolorbox}
\textsubscript{2} щоб ви прийняли її в Господі, як личить святим, і допомагайте їй, у якій речі буде вона чого потребувати від вас, бо й вона опікунка була багатьом і самому мені.
\end{tcolorbox}
\begin{tcolorbox}
\textsubscript{3} Вітайте Прискиллу й Акилу, співробітників моїх у Христі Ісусі,
\end{tcolorbox}
\begin{tcolorbox}
\textsubscript{4} що голови свої за душу мою клали, яким не я сам дякую, але й усі Церкви з поган, і їхню домашню Церкву.
\end{tcolorbox}
\begin{tcolorbox}
\textsubscript{5} Вітайте улюбленого мого Епенета, він первісток Ахаї для Христа.
\end{tcolorbox}
\begin{tcolorbox}
\textsubscript{6} Вітайте Марію, що напрацювалася багато для вас.
\end{tcolorbox}
\begin{tcolorbox}
\textsubscript{7} Вітайте Андроніка й Юнія, родичів моїх і співв'язнів моїх, що славні вони між апостолами, що й у Христі були перше мене.
\end{tcolorbox}
\begin{tcolorbox}
\textsubscript{8} Вітайте Амплія, мого улюбленого в Господі.
\end{tcolorbox}
\begin{tcolorbox}
\textsubscript{9} Вітайте Урбана, співробітника нашого в Христі, і улюбленого мого Стахія.
\end{tcolorbox}
\begin{tcolorbox}
\textsubscript{10} Вітайте Апеллеса, випробуваного в Христі. Вітайте Аристовулових.
\end{tcolorbox}
\begin{tcolorbox}
\textsubscript{11} Вітайте мого родича Іродіона. Вітайте Наркисових, що в Господі.
\end{tcolorbox}
\begin{tcolorbox}
\textsubscript{12} Вітайте Трифену й Трифосу, що працюють у Господі. Вітайте улюблену Персиду, що багато попрацювала в Господі.
\end{tcolorbox}
\begin{tcolorbox}
\textsubscript{13} Вітайте вибраного в Господі Руфа, і матір його та мою.
\end{tcolorbox}
\begin{tcolorbox}
\textsubscript{14} Вітайте Асинкрита, Флегонта, Єрма, Патрова, Єрмія і братів, що з ними.
\end{tcolorbox}
\begin{tcolorbox}
\textsubscript{15} Вітайте Філолога та Юлію, Нірея й сестру його, і Олімпіяна, і всіх святих, що з ними.
\end{tcolorbox}
\begin{tcolorbox}
\textsubscript{16} Вітайте один одного святим поцілунком. Вітають вас усі Церкви Христові!
\end{tcolorbox}
\begin{tcolorbox}
\textsubscript{17} Благаю ж вас, браття, щоб ви остерігалися тих, хто чинить розділення й згіршення проти науки, якої ви навчилися, і уникайте їх,
\end{tcolorbox}
\begin{tcolorbox}
\textsubscript{18} бо такі не служать Господеві нашому Ісусу Христу, але власному череву; вони добрими та гарними словами зводять серця простодушних.
\end{tcolorbox}
\begin{tcolorbox}
\textsubscript{19} Ваша ж слухняність дійшла до всіх. І я тішусь за вас, але хочу, щоб були ви мудрі в доброму, а прості в злому.
\end{tcolorbox}
\begin{tcolorbox}
\textsubscript{20} А Бог миру потопче незабаром сатану під ваші ноги. Благодать Господа нашого Ісуса Христа нехай буде з вами! Амінь.
\end{tcolorbox}
\begin{tcolorbox}
\textsubscript{21} Вітає вас мій співробітник Тимофій, і Лукій, і Ясон, і Сосипатр, мої родичі.
\end{tcolorbox}
\begin{tcolorbox}
\textsubscript{22} Вітаю вас у Господі й я, Тертій, що цього листа написав.
\end{tcolorbox}
\begin{tcolorbox}
\textsubscript{23} Вітає вас Гай, гостинний для мене й цілої Церкви. Вітає вас міський доморядник Ераст і брат Кварт.
\end{tcolorbox}
\begin{tcolorbox}
\textsubscript{24} Благодать Господа нашого Ісуса Христа нехай буде зо всіма вами! Амінь.
\end{tcolorbox}
\begin{tcolorbox}
\textsubscript{25} А Тому, хто може поставити вас міцно згідно з моєю Євангелією й проповіддю Ісуса Христа, за об'явленням таємниці, що від вічних часів була замовчана,
\end{tcolorbox}
\begin{tcolorbox}
\textsubscript{26} а тепер виявлена, і через пророцькі писання, з наказу вічного Бога, на послух вірі по всіх народах провіщена,
\end{tcolorbox}
\begin{tcolorbox}
\textsubscript{27} єдиному мудрому Богові, через Ісуса Христа, слава навіки! Амінь.
\end{tcolorbox}
