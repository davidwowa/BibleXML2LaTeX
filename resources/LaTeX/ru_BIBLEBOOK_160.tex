\section{BOOK 159}
\subsection{CHAPTER 1}
\begin{tcolorbox}
\textsubscript{1} Павел, Апостол Иисуса Христа по повелению Бога, Спасителя нашего, и Господа Иисуса Христа, надежды нашей,
\end{tcolorbox}
\begin{tcolorbox}
\textsubscript{2} Тимофею, истинному сыну в вере: благодать, милость, мир от Бога, Отца нашего, и Христа Иисуса, Господа нашего.
\end{tcolorbox}
\begin{tcolorbox}
\textsubscript{3} Отходя в Македонию, я просил тебя пребыть в Ефесе и увещевать некоторых, чтобы они не учили иному
\end{tcolorbox}
\begin{tcolorbox}
\textsubscript{4} и не занимались баснями и родословиями бесконечными, которые производят больше споры, нежели Божие назидание в вере.
\end{tcolorbox}
\begin{tcolorbox}
\textsubscript{5} Цель же увещания есть любовь от чистого сердца и доброй совести и нелицемерной веры,
\end{tcolorbox}
\begin{tcolorbox}
\textsubscript{6} от чего отступив, некоторые уклонились в пустословие,
\end{tcolorbox}
\begin{tcolorbox}
\textsubscript{7} желая быть законоучителями, но не разумея ни того, о чем говорят, ни того, что утверждают.
\end{tcolorbox}
\begin{tcolorbox}
\textsubscript{8} А мы знаем, что закон добр, если кто законно употребляет его,
\end{tcolorbox}
\begin{tcolorbox}
\textsubscript{9} зная, что закон положен не для праведника, но для беззаконных и непокоривых, нечестивых и грешников, развратных и оскверненных, для оскорбителей отца и матери, для человекоубийц,
\end{tcolorbox}
\begin{tcolorbox}
\textsubscript{10} для блудников, мужеложников, человекохищников, (клеветников, скотоложников,) лжецов, клятвопреступников, и для всего, что противно здравому учению,
\end{tcolorbox}
\begin{tcolorbox}
\textsubscript{11} по славному благовестию блаженного Бога, которое мне вверено.
\end{tcolorbox}
\begin{tcolorbox}
\textsubscript{12} Благодарю давшего мне силу, Христа Иисуса, Господа нашего, что Он признал меня верным, определив на служение,
\end{tcolorbox}
\begin{tcolorbox}
\textsubscript{13} меня, который прежде был хулитель и гонитель и обидчик, но помилован потому, что [так] поступал по неведению, в неверии;
\end{tcolorbox}
\begin{tcolorbox}
\textsubscript{14} благодать же Господа нашего (Иисуса Христа) открылась [во мне] обильно с верою и любовью во Христе Иисусе.
\end{tcolorbox}
\begin{tcolorbox}
\textsubscript{15} Верно и всякого принятия достойно слово, что Христос Иисус пришел в мир спасти грешников, из которых я первый.
\end{tcolorbox}
\begin{tcolorbox}
\textsubscript{16} Но для того я и помилован, чтобы Иисус Христос во мне первом показал все долготерпение, в пример тем, которые будут веровать в Него к жизни вечной.
\end{tcolorbox}
\begin{tcolorbox}
\textsubscript{17} Царю же веков нетленному, невидимому, единому премудрому Богу честь и слава во веки веков. Аминь.
\end{tcolorbox}
\begin{tcolorbox}
\textsubscript{18} Преподаю тебе, сын [мой] Тимофей, сообразно с бывшими о тебе пророчествами, такое завещание, чтобы ты воинствовал согласно с ними, как добрый воин,
\end{tcolorbox}
\begin{tcolorbox}
\textsubscript{19} имея веру и добрую совесть, которую некоторые отвергнув, потерпели кораблекрушение в вере;
\end{tcolorbox}
\begin{tcolorbox}
\textsubscript{20} таковы Именей и Александр, которых я предал сатане, чтобы они научились не богохульствовать.
\end{tcolorbox}
\subsection{CHAPTER 2}
\begin{tcolorbox}
\textsubscript{1} Итак прежде всего прошу совершать молитвы, прошения, моления, благодарения за всех человеков,
\end{tcolorbox}
\begin{tcolorbox}
\textsubscript{2} за царей и за всех начальствующих, дабы проводить нам жизнь тихую и безмятежную во всяком благочестии и чистоте,
\end{tcolorbox}
\begin{tcolorbox}
\textsubscript{3} ибо это хорошо и угодно Спасителю нашему Богу,
\end{tcolorbox}
\begin{tcolorbox}
\textsubscript{4} Который хочет, чтобы все люди спаслись и достигли познания истины.
\end{tcolorbox}
\begin{tcolorbox}
\textsubscript{5} Ибо един Бог, един и посредник между Богом и человеками, человек Христос Иисус,
\end{tcolorbox}
\begin{tcolorbox}
\textsubscript{6} предавший Себя для искупления всех. [Таково было] в свое время свидетельство,
\end{tcolorbox}
\begin{tcolorbox}
\textsubscript{7} для которого я поставлен проповедником и Апостолом, --истину говорю во Христе, не лгу, --учителем язычников в вере и истине.
\end{tcolorbox}
\begin{tcolorbox}
\textsubscript{8} Итак желаю, чтобы на всяком месте произносили молитвы мужи, воздевая чистые руки без гнева и сомнения;
\end{tcolorbox}
\begin{tcolorbox}
\textsubscript{9} чтобы также и жены, в приличном одеянии, со стыдливостью и целомудрием, украшали себя не плетением [волос], не золотом, не жемчугом, не многоценною одеждою,
\end{tcolorbox}
\begin{tcolorbox}
\textsubscript{10} но добрыми делами, как прилично женам, посвящающим себя благочестию.
\end{tcolorbox}
\begin{tcolorbox}
\textsubscript{11} Жена да учится в безмолвии, со всякою покорностью;
\end{tcolorbox}
\begin{tcolorbox}
\textsubscript{12} а учить жене не позволяю, ни властвовать над мужем, но быть в безмолвии.
\end{tcolorbox}
\begin{tcolorbox}
\textsubscript{13} Ибо прежде создан Адам, а потом Ева;
\end{tcolorbox}
\begin{tcolorbox}
\textsubscript{14} и не Адам прельщен; но жена, прельстившись, впала в преступление;
\end{tcolorbox}
\begin{tcolorbox}
\textsubscript{15} впрочем спасется через чадородие, если пребудет в вере и любви и в святости с целомудрием.
\end{tcolorbox}
\subsection{CHAPTER 3}
\begin{tcolorbox}
\textsubscript{1} Верно слово: если кто епископства желает, доброго дела желает.
\end{tcolorbox}
\begin{tcolorbox}
\textsubscript{2} Но епископ должен быть непорочен, одной жены муж, трезв, целомудрен, благочинен, честен, страннолюбив, учителен,
\end{tcolorbox}
\begin{tcolorbox}
\textsubscript{3} не пьяница, не бийца, не сварлив, не корыстолюбив, но тих, миролюбив, не сребролюбив,
\end{tcolorbox}
\begin{tcolorbox}
\textsubscript{4} хорошо управляющий домом своим, детей содержащий в послушании со всякою честностью;
\end{tcolorbox}
\begin{tcolorbox}
\textsubscript{5} ибо, кто не умеет управлять собственным домом, тот будет ли пещись о Церкви Божией?
\end{tcolorbox}
\begin{tcolorbox}
\textsubscript{6} Не [должен быть] из новообращенных, чтобы не возгордился и не подпал осуждению с диаволом.
\end{tcolorbox}
\begin{tcolorbox}
\textsubscript{7} Надлежит ему также иметь доброе свидетельство от внешних, чтобы не впасть в нарекание и сеть диавольскую.
\end{tcolorbox}
\begin{tcolorbox}
\textsubscript{8} Диаконы также [должны быть] честны, не двоязычны, не пристрастны к вину, не корыстолюбивы,
\end{tcolorbox}
\begin{tcolorbox}
\textsubscript{9} хранящие таинство веры в чистой совести.
\end{tcolorbox}
\begin{tcolorbox}
\textsubscript{10} И таких надобно прежде испытывать, потом, если беспорочны, [допускать] до служения.
\end{tcolorbox}
\begin{tcolorbox}
\textsubscript{11} Равно и жены [их должны быть] честны, не клеветницы, трезвы, верны во всем.
\end{tcolorbox}
\begin{tcolorbox}
\textsubscript{12} Диакон должен быть муж одной жены, хорошо управляющий детьми и домом своим.
\end{tcolorbox}
\begin{tcolorbox}
\textsubscript{13} Ибо хорошо служившие приготовляют себе высшую степень и великое дерзновение в вере во Христа Иисуса.
\end{tcolorbox}
\begin{tcolorbox}
\textsubscript{14} Сие пишу тебе, надеясь вскоре придти к тебе,
\end{tcolorbox}
\begin{tcolorbox}
\textsubscript{15} чтобы, если замедлю, ты знал, как должно поступать в доме Божием, который есть Церковь Бога живаго, столп и утверждение истины.
\end{tcolorbox}
\begin{tcolorbox}
\textsubscript{16} И беспрекословно--великая благочестия тайна: Бог явился во плоти, оправдал Себя в Духе, показал Себя Ангелам, проповедан в народах, принят верою в мире, вознесся во славе.
\end{tcolorbox}
\subsection{CHAPTER 4}
\begin{tcolorbox}
\textsubscript{1} Дух же ясно говорит, что в последние времена отступят некоторые от веры, внимая духам обольстителям и учениям бесовским,
\end{tcolorbox}
\begin{tcolorbox}
\textsubscript{2} через лицемерие лжесловесников, сожженных в совести своей,
\end{tcolorbox}
\begin{tcolorbox}
\textsubscript{3} запрещающих вступать в брак [и] употреблять в пищу то, что Бог сотворил, дабы верные и познавшие истину вкушали с благодарением.
\end{tcolorbox}
\begin{tcolorbox}
\textsubscript{4} Ибо всякое творение Божие хорошо, и ничто не предосудительно, если принимается с благодарением,
\end{tcolorbox}
\begin{tcolorbox}
\textsubscript{5} потому что освящается словом Божиим и молитвою.
\end{tcolorbox}
\begin{tcolorbox}
\textsubscript{6} Внушая сие братиям, будешь добрый служитель Иисуса Христа, питаемый словами веры и добрым учением, которому ты последовал.
\end{tcolorbox}
\begin{tcolorbox}
\textsubscript{7} Негодных же и бабьих басен отвращайся, а упражняй себя в благочестии,
\end{tcolorbox}
\begin{tcolorbox}
\textsubscript{8} ибо телесное упражнение мало полезно, а благочестие на все полезно, имея обетование жизни настоящей и будущей.
\end{tcolorbox}
\begin{tcolorbox}
\textsubscript{9} Слово сие верно и всякого принятия достойно.
\end{tcolorbox}
\begin{tcolorbox}
\textsubscript{10} Ибо мы для того и трудимся и поношения терпим, что уповаем на Бога живаго, Который есть Спаситель всех человеков, а наипаче верных.
\end{tcolorbox}
\begin{tcolorbox}
\textsubscript{11} Проповедуй сие и учи.
\end{tcolorbox}
\begin{tcolorbox}
\textsubscript{12} Никто да не пренебрегает юностью твоею; но будь образцом для верных в слове, в житии, в любви, в духе, в вере, в чистоте.
\end{tcolorbox}
\begin{tcolorbox}
\textsubscript{13} Доколе не приду, занимайся чтением, наставлением, учением.
\end{tcolorbox}
\begin{tcolorbox}
\textsubscript{14} Не неради о пребывающем в тебе даровании, которое дано тебе по пророчеству с возложением рук священства.
\end{tcolorbox}
\begin{tcolorbox}
\textsubscript{15} О сем заботься, в сем пребывай, дабы успех твой для всех был очевиден.
\end{tcolorbox}
\begin{tcolorbox}
\textsubscript{16} Вникай в себя и в учение; занимайся сим постоянно: ибо, так поступая, и себя спасешь и слушающих тебя.
\end{tcolorbox}
\subsection{CHAPTER 5}
\begin{tcolorbox}
\textsubscript{1} Старца не укоряй, но увещевай, как отца; младших, как братьев;
\end{tcolorbox}
\begin{tcolorbox}
\textsubscript{2} стариц, как матерей; молодых, как сестер, со всякою чистотою.
\end{tcolorbox}
\begin{tcolorbox}
\textsubscript{3} Вдовиц почитай, истинных вдовиц.
\end{tcolorbox}
\begin{tcolorbox}
\textsubscript{4} Если же какая вдовица имеет детей или внучат, то они прежде пусть учатся почитать свою семью и воздавать должное родителям, ибо сие угодно Богу.
\end{tcolorbox}
\begin{tcolorbox}
\textsubscript{5} Истинная вдовица и одинокая надеется на Бога и пребывает в молениях и молитвах день и ночь;
\end{tcolorbox}
\begin{tcolorbox}
\textsubscript{6} а сластолюбивая заживо умерла.
\end{tcolorbox}
\begin{tcolorbox}
\textsubscript{7} И сие внушай им, чтобы были беспорочны.
\end{tcolorbox}
\begin{tcolorbox}
\textsubscript{8} Если же кто о своих и особенно о домашних не печется, тот отрекся от веры и хуже неверного.
\end{tcolorbox}
\begin{tcolorbox}
\textsubscript{9} Вдовица должна быть избираема не менее, как шестидесятилетняя, бывшая женою одного мужа,
\end{tcolorbox}
\begin{tcolorbox}
\textsubscript{10} известная по добрым делам, если она воспитала детей, принимала странников, умывала ноги святым, помогала бедствующим и была усердна ко всякому доброму делу.
\end{tcolorbox}
\begin{tcolorbox}
\textsubscript{11} Молодых же вдовиц не принимай, ибо они, впадая в роскошь в противность Христу, желают вступать в брак.
\end{tcolorbox}
\begin{tcolorbox}
\textsubscript{12} Они подлежат осуждению, потому что отвергли прежнюю веру;
\end{tcolorbox}
\begin{tcolorbox}
\textsubscript{13} притом же они, будучи праздны, приучаются ходить по домам и [бывают] не только праздны, но и болтливы, любопытны, и говорят, чего не должно.
\end{tcolorbox}
\begin{tcolorbox}
\textsubscript{14} Итак я желаю, чтобы молодые вдовы вступали в брак, рождали детей, управляли домом и не подавали противнику никакого повода к злоречию;
\end{tcolorbox}
\begin{tcolorbox}
\textsubscript{15} ибо некоторые уже совратились вслед сатаны.
\end{tcolorbox}
\begin{tcolorbox}
\textsubscript{16} Если какой верный или верная имеет вдов, то должны их довольствовать и не обременять Церкви, чтобы она могла довольствовать истинных вдовиц.
\end{tcolorbox}
\begin{tcolorbox}
\textsubscript{17} Достойно начальствующим пресвитерам должно оказывать сугубую честь, особенно тем, которые трудятся в слове и учении.
\end{tcolorbox}
\begin{tcolorbox}
\textsubscript{18} Ибо Писание говорит: не заграждай рта у вола молотящего; и: трудящийся достоин награды своей.
\end{tcolorbox}
\begin{tcolorbox}
\textsubscript{19} Обвинение на пресвитера не иначе принимай, как при двух или трех свидетелях.
\end{tcolorbox}
\begin{tcolorbox}
\textsubscript{20} Согрешающих обличай перед всеми, чтобы и прочие страх имели.
\end{tcolorbox}
\begin{tcolorbox}
\textsubscript{21} Пред Богом и Господом Иисусом Христом и избранными Ангелами заклинаю тебя сохранить сие без предубеждения, ничего не делая по пристрастию.
\end{tcolorbox}
\begin{tcolorbox}
\textsubscript{22} Рук ни на кого не возлагай поспешно, и не делайся участником в чужих грехах. Храни себя чистым.
\end{tcolorbox}
\begin{tcolorbox}
\textsubscript{23} Впредь пей не [одну] воду, но употребляй немного вина, ради желудка твоего и частых твоих недугов.
\end{tcolorbox}
\begin{tcolorbox}
\textsubscript{24} Грехи некоторых людей явны и прямо ведут к осуждению, а некоторых [открываются] впоследствии.
\end{tcolorbox}
\begin{tcolorbox}
\textsubscript{25} Равным образом и добрые дела явны; а если и не таковы, скрыться не могут.
\end{tcolorbox}
\subsection{CHAPTER 6}
\begin{tcolorbox}
\textsubscript{1} Рабы, под игом находящиеся, должны почитать господ своих достойными всякой чести, дабы не было хулы на имя Божие и учение.
\end{tcolorbox}
\begin{tcolorbox}
\textsubscript{2} Те, которые имеют господами верных, не должны обращаться с ними небрежно, потому что они братья; но тем более должны служить им, что они верные и возлюбленные и благодетельствуют [им]. Учи сему и увещевай.
\end{tcolorbox}
\begin{tcolorbox}
\textsubscript{3} Кто учит иному и не следует здравым словам Господа нашего Иисуса Христа и учению о благочестии,
\end{tcolorbox}
\begin{tcolorbox}
\textsubscript{4} тот горд, ничего не знает, но заражен [страстью] к состязаниям и словопрениям, от которых происходят зависть, распри, злоречия, лукавые подозрения.
\end{tcolorbox}
\begin{tcolorbox}
\textsubscript{5} Пустые споры между людьми поврежденного ума, чуждыми истины, которые думают, будто благочестие служит для прибытка. Удаляйся от таких.
\end{tcolorbox}
\begin{tcolorbox}
\textsubscript{6} Великое приобретение--быть благочестивым и довольным.
\end{tcolorbox}
\begin{tcolorbox}
\textsubscript{7} Ибо мы ничего не принесли в мир; явно, что ничего не можем и вынести [из него].
\end{tcolorbox}
\begin{tcolorbox}
\textsubscript{8} Имея пропитание и одежду, будем довольны тем.
\end{tcolorbox}
\begin{tcolorbox}
\textsubscript{9} А желающие обогащаться впадают в искушение и в сеть и во многие безрассудные и вредные похоти, которые погружают людей в бедствие и пагубу;
\end{tcolorbox}
\begin{tcolorbox}
\textsubscript{10} ибо корень всех зол есть сребролюбие, которому предавшись, некоторые уклонились от веры и сами себя подвергли многим скорбям.
\end{tcolorbox}
\begin{tcolorbox}
\textsubscript{11} Ты же, человек Божий, убегай сего, а преуспевай в правде, благочестии, вере, любви, терпении, кротости.
\end{tcolorbox}
\begin{tcolorbox}
\textsubscript{12} Подвизайся добрым подвигом веры, держись вечной жизни, к которой ты и призван, и исповедал доброе исповедание перед многими свидетелями.
\end{tcolorbox}
\begin{tcolorbox}
\textsubscript{13} Пред Богом, все животворящим, и пред Христом Иисусом, Который засвидетельствовал пред Понтием Пилатом доброе исповедание, завещеваю тебе
\end{tcolorbox}
\begin{tcolorbox}
\textsubscript{14} соблюсти заповедь чисто и неукоризненно, даже до явления Господа нашего Иисуса Христа,
\end{tcolorbox}
\begin{tcolorbox}
\textsubscript{15} которое в свое время откроет блаженный и единый сильный Царь царствующих и Господь господствующих,
\end{tcolorbox}
\begin{tcolorbox}
\textsubscript{16} единый имеющий бессмертие, Который обитает в неприступном свете, Которого никто из человеков не видел и видеть не может. Ему честь и держава вечная! Аминь.
\end{tcolorbox}
\begin{tcolorbox}
\textsubscript{17} Богатых в настоящем веке увещевай, чтобы они не высоко думали [о] [себе] и уповали не на богатство неверное, но на Бога живаго, дающего нам всё обильно для наслаждения;
\end{tcolorbox}
\begin{tcolorbox}
\textsubscript{18} чтобы они благодетельствовали, богатели добрыми делами, были щедры и общительны,
\end{tcolorbox}
\begin{tcolorbox}
\textsubscript{19} собирая себе сокровище, доброе основание для будущего, чтобы достигнуть вечной жизни.
\end{tcolorbox}
\begin{tcolorbox}
\textsubscript{20} О, Тимофей! храни преданное тебе, отвращаясь негодного пустословия и прекословий лжеименного знания,
\end{tcolorbox}
\begin{tcolorbox}
\textsubscript{21} которому предавшись, некоторые уклонились от веры. Благодать с тобою. Аминь.
\end{tcolorbox}
