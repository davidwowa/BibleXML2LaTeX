Book 47
\subsection{CHAPTER 1}
\begin{tcolorbox}
\textsubscript{1} Апостол Павло, поставлений ні від людей, ані від чоловіка, але від Ісуса Христа й Бога Отця, що з мертвих Його воскресив,
\end{tcolorbox}
\begin{tcolorbox}
\textsubscript{2} і присутня зо мною вся браття, до Церков галатійських:
\end{tcolorbox}
\begin{tcolorbox}
\textsubscript{3} благодать вам і мир від Бога, Отця нашого, і Господа Ісуса Христа,
\end{tcolorbox}
\begin{tcolorbox}
\textsubscript{4} що за наші гріхи дав Самого Себе, щоб від злого сучасного віку нас визволити, за волею Бога й Отця нашого,
\end{tcolorbox}
\begin{tcolorbox}
\textsubscript{5} Йому слава на віки вічні, амінь!
\end{tcolorbox}
\begin{tcolorbox}
\textsubscript{6} Дивуюся я, що ви так скоро відхилюєтесь від того, хто покликав Христовою благодаттю вас, на іншу Євангелію,
\end{tcolorbox}
\begin{tcolorbox}
\textsubscript{7} що не інша вона, але деякі є, що вас непокоять, і хочуть перевернути Христову Євангелію.
\end{tcolorbox}
\begin{tcolorbox}
\textsubscript{8} Але якби й ми або Ангол із неба зачав благовістити вам не те, що ми вам благовістили, нехай буде проклятий!
\end{tcolorbox}
\begin{tcolorbox}
\textsubscript{9} Як ми перше казали, і тепер знов кажу: коли хто вам не те благовістить, що ви прийняли, нехай буде проклятий!
\end{tcolorbox}
\begin{tcolorbox}
\textsubscript{10} Бо тепер чи я в людей шукаю признання чи в Бога? Чи людям дбаю я догоджати? Бо коли б догоджав я ще людям, я не був би рабом Христовим.
\end{tcolorbox}
\begin{tcolorbox}
\textsubscript{11} Звіщаю ж вам, браття, що Євангелія, яку я благовістив, вона не від людей.
\end{tcolorbox}
\begin{tcolorbox}
\textsubscript{12} Бо я не прийняв, ні навчився її від людини, але відкриттям Ісуса Христа.
\end{tcolorbox}
\begin{tcolorbox}
\textsubscript{13} Чули бо ви про моє поступовання перше в юдействі, що Божу Церкву жорстоко я переслідував та руйнував її.
\end{tcolorbox}
\begin{tcolorbox}
\textsubscript{14} І я перевищував в юдействі багатьох своїх ровесників роду мого, бувши запеклим прихильником моїх отцівських передань.
\end{tcolorbox}
\begin{tcolorbox}
\textsubscript{15} Коли ж Бог, що вибрав мене від утроби матері моєї і покликав благодаттю Своєю, уподобав
\end{tcolorbox}
\begin{tcolorbox}
\textsubscript{16} виявити мною Сина Свого, щоб благовістив я Його між поганами, я не радився зараз із тілом та кров'ю,
\end{tcolorbox}
\begin{tcolorbox}
\textsubscript{17} і не відправився в Єрусалим до апостолів, що передо мною були, а пішов я в Арабію, і знову вернувся в Дамаск.
\end{tcolorbox}
\begin{tcolorbox}
\textsubscript{18} По трьох роках потому пішов я в Єрусалим побачити Кифу, і в нього пробув днів із п'ятнадцять.
\end{tcolorbox}
\begin{tcolorbox}
\textsubscript{19} А іншого з апостолів я не бачив, крім Якова, брата Господнього.
\end{tcolorbox}
\begin{tcolorbox}
\textsubscript{20} А що вам пишу, ось кажу перед Богом, що я не обманюю!
\end{tcolorbox}
\begin{tcolorbox}
\textsubscript{21} Потому пішов я до сирських та кілікійських країн.
\end{tcolorbox}
\begin{tcolorbox}
\textsubscript{22} Церквам же Христовим в Юдеї я знаний не був особисто,
\end{tcolorbox}
\begin{tcolorbox}
\textsubscript{23} тільки чули вони, що той, що колись переслідував їх, благовістить тепер віру, що колись руйнував був її.
\end{tcolorbox}
\begin{tcolorbox}
\textsubscript{24} І славили Бога вони через мене!
\end{tcolorbox}
\subsection{CHAPTER 2}
\begin{tcolorbox}
\textsubscript{1} Потому, по чотирнадцяти роках, я знову ходив в Єрусалим із Варнавою, взявши й Тита з собою.
\end{tcolorbox}
\begin{tcolorbox}
\textsubscript{2} А пішов я за відкриттям. І подав їм Євангелію, що її проповідую між поганами, особливо знатнішим, чи не дарма змагаюся я чи змагався.
\end{tcolorbox}
\begin{tcolorbox}
\textsubscript{3} Але й Тит, що зо мною, бувши греком, не був до обрізання змушений.
\end{tcolorbox}
\begin{tcolorbox}
\textsubscript{4} А щодо прибулих фальшивих братів, що прийшли підглядати нашу вільність, яку маємо в Христі Ісусі, щоб нас поневолити,
\end{tcolorbox}
\begin{tcolorbox}
\textsubscript{5} то ми їх не послухали ані на хвилю, і не піддалися були, щоб тривала в вас правда Євангелії.
\end{tcolorbox}
\begin{tcolorbox}
\textsubscript{6} Щождо тих, що за щось уважають себе, та якими колись вони були, то ні в чому різниці для мене нема, не дивиться Бог на особу людини! Бо ті, що за щось уважають себе, нічого мені не додали,
\end{tcolorbox}
\begin{tcolorbox}
\textsubscript{7} але навпаки, побачивши, що мені припоручена Євангелія для необрізаних, як Петрові для обрізаних,
\end{tcolorbox}
\begin{tcolorbox}
\textsubscript{8} бо Той, хто помагав Петрові в апостольстві між обрізаними, помагав і мені між поганами,
\end{tcolorbox}
\begin{tcolorbox}
\textsubscript{9} і, пізнавши ту благодать, що дана мені, Яків, і Кифа, і Іван, що стовпами вважаються, подали мені та Варнаві правиці спільноти, щоб ми для поган працювали, вони ж для обрізаних,
\end{tcolorbox}
\begin{tcolorbox}
\textsubscript{10} тільки щоб ми пам'ятали про вбогих, що я й пильнував був чинити таке.
\end{tcolorbox}
\begin{tcolorbox}
\textsubscript{11} Коли ж Кифа прийшов був до Антіохії, то відкрито я виступив супроти нього, заслуговував бо він на осуд.
\end{tcolorbox}
\begin{tcolorbox}
\textsubscript{12} Бо він перед тим, як прийшли були дехто від Якова, споживав із поганами. А коли прибули, став ховатися та відлучатися, боячися обрізаних.
\end{tcolorbox}
\begin{tcolorbox}
\textsubscript{13} А з ним лицемірили й інші юдеї, так що навіть Варнава пристав був до їхнього лицемірства.
\end{tcolorbox}
\begin{tcolorbox}
\textsubscript{14} А коли я побачив, що не йдуть вони рівно за євангельською правдою, то перед усіма сказав Кифі: Коли ти, бувши юдеєм, живеш по-поганському, а не по-юдейському, то нащо поган ти примушуєш жити по-юдейському?
\end{tcolorbox}
\begin{tcolorbox}
\textsubscript{15} Ми юдеї природою, а не грішники з поган...
\end{tcolorbox}
\begin{tcolorbox}
\textsubscript{16} А коли ми дізнались, що людина не може бути виправдана ділами Закону, але тільки вірою в Христа Ісуса, то ми ввірували в Христа Ісуса, щоб нам виправдатися вірою в Христа, а не ділами Закону. Бо жадна людина ділами Закону не буде виправдана!
\end{tcolorbox}
\begin{tcolorbox}
\textsubscript{17} Коли ж, шукаючи виправдання в Христі, ми й самі показалися грішниками, то хіба Христос слуга гріху? Зовсім ні!
\end{tcolorbox}
\begin{tcolorbox}
\textsubscript{18} Бо коли я будую знов те, що був зруйнував, то самого себе роблю злочинцем.
\end{tcolorbox}
\begin{tcolorbox}
\textsubscript{19} Бо Законом я вмер для Закону, щоб жити для Бога. Я розп'ятий з Христом.
\end{tcolorbox}
\begin{tcolorbox}
\textsubscript{20} І живу вже не я, а Христос проживає в мені. А що я живу в тілі тепер, живу вірою в Божого Сина, що мене полюбив, і видав за мене Самого Себе.
\end{tcolorbox}
\begin{tcolorbox}
\textsubscript{21} Божої благодаті я не відкидаю. Бо коли набувається правда Законом, то надармо Христос був умер!
\end{tcolorbox}
\subsection{CHAPTER 3}
\begin{tcolorbox}
\textsubscript{1} О, ви нерозумні галати! Хто вас звів не коритися правді, вас, яким перед очима Ісус Христос переднакреслений був, як ніби між вами розп'ятий?
\end{tcolorbox}
\begin{tcolorbox}
\textsubscript{2} Це одне хочу знати від вас: чи ви прийняли Духа ділами Закону, чи із проповіді про віру?
\end{tcolorbox}
\begin{tcolorbox}
\textsubscript{3} Чи ж ви аж такі нерозумні? Духом почавши, кінчите тепер тілом?
\end{tcolorbox}
\begin{tcolorbox}
\textsubscript{4} Чи ви так багато терпіли надармо? Коли б тільки надармо!
\end{tcolorbox}
\begin{tcolorbox}
\textsubscript{5} Отже, Той, Хто вам Духа дає й чуда чинить між вами, чи чинить ділами Закону, чи із проповіді про віру?
\end{tcolorbox}
\begin{tcolorbox}
\textsubscript{6} Так як Авраам був увірував в Бога, і це залічено за праведність йому.
\end{tcolorbox}
\begin{tcolorbox}
\textsubscript{7} Тож знайте, що ті, хто від віри, то сини Авраамові.
\end{tcolorbox}
\begin{tcolorbox}
\textsubscript{8} І Писання, передбачивши, що вірою Бог виправдає поган, благовістило Авраамові: Благословляться в тобі всі народи!
\end{tcolorbox}
\begin{tcolorbox}
\textsubscript{9} Тому ті, хто від віри, будуть поблагословлені з вірним Авраамом.
\end{tcolorbox}
\begin{tcolorbox}
\textsubscript{10} А всі ті, хто на діла Закону покладається, вони під прокляттям. Бо написано: Проклятий усякий, хто не триває в усьому, що написано в книзі Закону, щоб чинити оте!
\end{tcolorbox}
\begin{tcolorbox}
\textsubscript{11} А що перед Богом Законом ніхто не виправдується, то це ясно, бо праведний житиме вірою.
\end{tcolorbox}
\begin{tcolorbox}
\textsubscript{12} А Закон не від віри, але хто чинитиме те, той житиме ним.
\end{tcolorbox}
\begin{tcolorbox}
\textsubscript{13} Христос відкупив нас від прокляття Закону, ставши прокляттям за нас, бо написано: Проклятий усякий, хто висить на дереві,
\end{tcolorbox}
\begin{tcolorbox}
\textsubscript{14} щоб Авраамове благословення в Ісусі Христі поширилося на поган, щоб обітницю Духа прийняти нам вірою.
\end{tcolorbox}
\begin{tcolorbox}
\textsubscript{15} Браття, кажу я по-людському: навіть людського затвердженого заповіту ніхто не відкидає та до нього не додає.
\end{tcolorbox}
\begin{tcolorbox}
\textsubscript{16} А обітниці дані були Авраамові й насінню його. Не говориться: і насінням, як про багатьох, але як про одного: і Насінню твоєму, яке є Христос.
\end{tcolorbox}
\begin{tcolorbox}
\textsubscript{17} А я кажу це, що заповіту, від Бога затвердженого, Закон, що прийшов по чотириста тридцяти роках, не відкидає, щоб обітницю він зруйнував.
\end{tcolorbox}
\begin{tcolorbox}
\textsubscript{18} Бо коли від Закону спадщина, то вже не з обітниці; Авраамові ж Бог дарував із обітниці.
\end{tcolorbox}
\begin{tcolorbox}
\textsubscript{19} Що ж Закон? Він був даний з причини переступів, аж поки прийде Насіння, якому обітниця дана була; він учинений був Анголами рукою посередника.
\end{tcolorbox}
\begin{tcolorbox}
\textsubscript{20} Але посередник не є для одного, Бог же один.
\end{tcolorbox}
\begin{tcolorbox}
\textsubscript{21} Отож, чи ж Закон проти Божих обітниць? Зовсім ні! Якби бо був даний Закон, щоб він міг оживляти, то праведність справді була б від Закону!
\end{tcolorbox}
\begin{tcolorbox}
\textsubscript{22} Та все зачинило Писання під гріх, щоб віруючим була дана обітниця з віри в Ісуса Христа.
\end{tcolorbox}
\begin{tcolorbox}
\textsubscript{23} Але поки прийшла віра, під Законом стережено нас, замкнених до приходу віри, що мала об'явитись.
\end{tcolorbox}
\begin{tcolorbox}
\textsubscript{24} Тому то Закон виховником був до Христа, щоб нам виправдатися вірою.
\end{tcolorbox}
\begin{tcolorbox}
\textsubscript{25} А як віра прийшла, то вже ми не під виховником.
\end{tcolorbox}
\begin{tcolorbox}
\textsubscript{26} Бо ви всі сини Божі через віру в Христа Ісуса!
\end{tcolorbox}
\begin{tcolorbox}
\textsubscript{27} Бо ви всі, що в Христа охристилися, у Христа зодягнулися!
\end{tcolorbox}
\begin{tcolorbox}
\textsubscript{28} Нема юдея, ні грека, нема раба, ані вільного, нема чоловічої статі, ані жіночої, бо всі ви один у Христі Ісусі!
\end{tcolorbox}
\begin{tcolorbox}
\textsubscript{29} А коли ви Христові, то ви Авраамове насіння й за обітницею спадкоємці.
\end{tcolorbox}
\subsection{CHAPTER 4}
\begin{tcolorbox}
\textsubscript{1} Тож кажу я: поки спадкоємець дитина, він нічим від раба не різниться, хоч він пан над усім,
\end{tcolorbox}
\begin{tcolorbox}
\textsubscript{2} але під опікунами та керівниками знаходиться він аж до часу, що визначив батько.
\end{tcolorbox}
\begin{tcolorbox}
\textsubscript{3} Так і ми, поки дітьми були, то були поневолені стихіями світу.
\end{tcolorbox}
\begin{tcolorbox}
\textsubscript{4} Як настало ж виповнення часу, Бог послав Свого Сина, що родився від жони, та став під Законом,
\end{tcolorbox}
\begin{tcolorbox}
\textsubscript{5} щоб викупити підзаконних, щоб усиновлення ми прийняли.
\end{tcolorbox}
\begin{tcolorbox}
\textsubscript{6} А що ви сини, Бог послав у ваші серця Духа Сина Свого, що викликує: Авва, Отче!
\end{tcolorbox}
\begin{tcolorbox}
\textsubscript{7} Тому ти вже не раб, але син. А як син, то й спадкоємець Божий через Христа.
\end{tcolorbox}
\begin{tcolorbox}
\textsubscript{8} Та тоді, не знаючи Бога, служили ви тим, що з істоти богами вони не були.
\end{tcolorbox}
\begin{tcolorbox}
\textsubscript{9} А тепер, як пізнали ви Бога, чи краще як Бог вас пізнав, як вертаєтесь знов до слабих та вбогих стихій, яким хочете знов, як давніше, служити?
\end{tcolorbox}
\begin{tcolorbox}
\textsubscript{10} Ви вважаєте пильно на дні та на місяці, і на пори та роки.
\end{tcolorbox}
\begin{tcolorbox}
\textsubscript{11} Я боюся за вас, чи не дармо я працював коло вас?...
\end{tcolorbox}
\begin{tcolorbox}
\textsubscript{12} Прошу я вас, браття, будьте, як я, бо й я такий самий, як ви. Нічим ви мене не покривдили!
\end{tcolorbox}
\begin{tcolorbox}
\textsubscript{13} І знаєте ви, що в немочі тіла я перше звіщав вам Євангелію,
\end{tcolorbox}
\begin{tcolorbox}
\textsubscript{14} ви ж моєю спокусою в тілі моїм не погордували, і мене не відкинули, але, немов Ангола Божого, ви прийняли мене, як Христа Ісуса!
\end{tcolorbox}
\begin{tcolorbox}
\textsubscript{15} Тож де ваше тодішнє блаженство? Свідкую бо вам, що якби було можна, то ви вибрали б очі свої та мені віддали б!
\end{tcolorbox}
\begin{tcolorbox}
\textsubscript{16} Чи ж я став для вас ворогом, правду говорячи вам?
\end{tcolorbox}
\begin{tcolorbox}
\textsubscript{17} Недобре пильнують про вас, але вас відлучити хочуть, щоб ви пильнували про них.
\end{tcolorbox}
\begin{tcolorbox}
\textsubscript{18} То добре, пильнувати про добре постійно, а не тільки тоді, як приходжу до вас.
\end{tcolorbox}
\begin{tcolorbox}
\textsubscript{19} Дітки мої, я знову для вас терплю муки породу, поки образ Христа не відіб'ється в вас!
\end{tcolorbox}
\begin{tcolorbox}
\textsubscript{20} Я хотів би тепер бути в вас та змінити свій голос, бо маю я сумнів за вас.
\end{tcolorbox}
\begin{tcolorbox}
\textsubscript{21} Скажіть мені ви, що хочете бути під Законом: чи не слухаєтесь ви Закону?
\end{tcolorbox}
\begin{tcolorbox}
\textsubscript{22} Бо написано: Мав Авраам двох синів, одного від рабині, а другого від вільної.
\end{tcolorbox}
\begin{tcolorbox}
\textsubscript{23} Але той, хто був від рабині, народився за тілом, а хто був від вільної, за обітницею.
\end{tcolorbox}
\begin{tcolorbox}
\textsubscript{24} Розуміти це треба інакше, бо це два заповіти: один від гори Сінай, що в рабство народжує, а він то Аґар.
\end{tcolorbox}
\begin{tcolorbox}
\textsubscript{25} Бо Аґар то гора Сінай в Арабії, а відповідає сучасному Єрусалимові, який у рабстві з своїми дітьми.
\end{tcolorbox}
\begin{tcolorbox}
\textsubscript{26} А вишній Єрусалим вільний, він мати всім нам!
\end{tcolorbox}
\begin{tcolorbox}
\textsubscript{27} Бо написано: Звеселися, неплідна, ти, що не родиш! Гукай та викликуй ти, що в породі не мучилась, бо в полишеної значно більше дітей, ніж у тієї, що має вона чоловіка!
\end{tcolorbox}
\begin{tcolorbox}
\textsubscript{28} А ви, браття, діти обітниці за Ісаком!
\end{tcolorbox}
\begin{tcolorbox}
\textsubscript{29} Але як і тоді, хто родився за тілом, переслідував тих, хто родився за духом, так само й тепер.
\end{tcolorbox}
\begin{tcolorbox}
\textsubscript{30} Та що каже Писання? Прожени рабиню й сина її, бо не буде спадкувати син рабині разом із сином вільної.
\end{tcolorbox}
\begin{tcolorbox}
\textsubscript{31} Тому, браття, не сини ми рабині, але вільної!
\end{tcolorbox}
\subsection{CHAPTER 5}
\begin{tcolorbox}
\textsubscript{1} Христос для волі нас визволив. Тож стійте в ній та не піддавайтеся знову в ярмо рабства!
\end{tcolorbox}
\begin{tcolorbox}
\textsubscript{2} Ось я, Павло, кажу вам, що коли ви обрізуєтесь, то нема вам тоді жадної користи від Христа.
\end{tcolorbox}
\begin{tcolorbox}
\textsubscript{3} І свідкую я знову всякому чоловікові, який обрізується, що повинен він виконати ввесь Закон.
\end{tcolorbox}
\begin{tcolorbox}
\textsubscript{4} Ви, що Законом виправдуєтесь, полишилися без Христа, відпали від благодаті!
\end{tcolorbox}
\begin{tcolorbox}
\textsubscript{5} Бо ми в дусі з віри чекаємо надії праведности.
\end{tcolorbox}
\begin{tcolorbox}
\textsubscript{6} Бо сили не має в Христі Ісусі ані обрізання, ані необрізання, але віра, що чинна любов'ю.
\end{tcolorbox}
\begin{tcolorbox}
\textsubscript{7} Бігли ви добре. Хто заборонив вам коритися правді?
\end{tcolorbox}
\begin{tcolorbox}
\textsubscript{8} Таке переконання не від Того, Хто вас покликав.
\end{tcolorbox}
\begin{tcolorbox}
\textsubscript{9} Трохи розчини квасить усе тісто!
\end{tcolorbox}
\begin{tcolorbox}
\textsubscript{10} Я в Господі маю надію на вас, що нічого іншого думати не будете ви. А хто вас непокоїть, осуджений буде, хоч би він хто був!
\end{tcolorbox}
\begin{tcolorbox}
\textsubscript{11} Чого ж, браття, мене ще переслідують, коли я обрізання ще проповідую? Тоді спокуса хреста в ніщо обертається!
\end{tcolorbox}
\begin{tcolorbox}
\textsubscript{12} О, коли б були навіть відсічені ті, хто підбурює вас!
\end{tcolorbox}
\begin{tcolorbox}
\textsubscript{13} Бо ви, браття, на волю покликані, але щоб ваша воля не стала приводом догоджати тілу, а любов'ю служити один одному!
\end{tcolorbox}
\begin{tcolorbox}
\textsubscript{14} Бо ввесь Закон в однім слові міститься: Люби свого ближнього, як самого себе!
\end{tcolorbox}
\begin{tcolorbox}
\textsubscript{15} Коли ж ви гризете та їсте один одного, то глядіть, щоб не знищили ви один одного!
\end{tcolorbox}
\begin{tcolorbox}
\textsubscript{16} І кажу: ходіть за духом, і не вчините пожадливости тіла,
\end{tcolorbox}
\begin{tcolorbox}
\textsubscript{17} бо тіло бажає противного духові, а дух противного тілу, і супротивні вони один одному, щоб ви чинили не те, чого хочете.
\end{tcolorbox}
\begin{tcolorbox}
\textsubscript{18} Коли ж дух вас провадить, то ви не під Законом.
\end{tcolorbox}
\begin{tcolorbox}
\textsubscript{19} Учинки тіла явні, то є: перелюб, нечистість, розпуста,
\end{tcolorbox}
\begin{tcolorbox}
\textsubscript{20} ідолослуження, чари, ворожнечі, сварка, заздрість, гнів, суперечки, незгоди, єресі,
\end{tcolorbox}
\begin{tcolorbox}
\textsubscript{21} завидки, п'янство, гулянки й подібне до цього. Я про це попереджую вас, як і попереджав був, що хто чинить таке, не вспадкують вони Царства Божого!
\end{tcolorbox}
\begin{tcolorbox}
\textsubscript{22} А плід духа: любов, радість, мир, довготерпіння, добрість, милосердя, віра,
\end{tcolorbox}
\begin{tcolorbox}
\textsubscript{23} лагідність, здержливість: Закону нема на таких!
\end{tcolorbox}
\begin{tcolorbox}
\textsubscript{24} А ті, що Христові Ісусові, розп'яли вони тіло з пожадливостями та з похотями.
\end{tcolorbox}
\begin{tcolorbox}
\textsubscript{25} Коли духом живемо, то й духом ходімо!
\end{tcolorbox}
\begin{tcolorbox}
\textsubscript{26} Не будьмо чванливі, не дражнімо один одного, не завидуймо один одному!
\end{tcolorbox}
\subsection{CHAPTER 6}
\begin{tcolorbox}
\textsubscript{1} Браття, як людина й упаде в який прогріх, то ви, духовні, виправляйте такого духом лагідности, сам себе доглядаючи, щоб не спокусився й ти!
\end{tcolorbox}
\begin{tcolorbox}
\textsubscript{2} Носіть тягарі один одного, і так виконаєте закона Христового.
\end{tcolorbox}
\begin{tcolorbox}
\textsubscript{3} Коли бо хто думає, що він щось, бувши ніщо, сам себе той обманює.
\end{tcolorbox}
\begin{tcolorbox}
\textsubscript{4} Нехай кожен досліджує діло своє, і тоді матиме тільки в собі похвалу, а не в іншому!
\end{tcolorbox}
\begin{tcolorbox}
\textsubscript{5} Бо кожен нестиме свій власний тягар!
\end{tcolorbox}
\begin{tcolorbox}
\textsubscript{6} А хто слова навчається, нехай ділиться всяким добром із навчаючим.
\end{tcolorbox}
\begin{tcolorbox}
\textsubscript{7} Не обманюйтеся, Бог осміяний бути не може. Бо що тільки людина посіє, те саме й пожне!
\end{tcolorbox}
\begin{tcolorbox}
\textsubscript{8} Бо хто сіє для власного тіла свого, той від тіла тління пожне. А хто сіє для духа, той від духа пожне життя вічне.
\end{tcolorbox}
\begin{tcolorbox}
\textsubscript{9} А роблячи добре, не знуджуймося, бо часу свого пожнемо, коли не ослабнемо.
\end{tcolorbox}
\begin{tcolorbox}
\textsubscript{10} Тож тому, поки маємо час, усім робімо добро, а найбільш одновірним!
\end{tcolorbox}
\begin{tcolorbox}
\textsubscript{11} Погляньте, якими великими буквами я написав вам своєю рукою!
\end{tcolorbox}
\begin{tcolorbox}
\textsubscript{12} Усі ті, хто бажає хвалитися тілом, змушують вас обрізуватись, щоб тільки вони не були переслідувані за хреста Христового.
\end{tcolorbox}
\begin{tcolorbox}
\textsubscript{13} Бо навіть і ті, хто обрізується, самі не зберігають Закона, а хочуть, щоб ви обрізувались, щоб хвалитися їм вашим тілом.
\end{tcolorbox}
\begin{tcolorbox}
\textsubscript{14} А щодо мене, то нехай нічим не хвалюся, хіба тільки хрестом Господа нашого Ісуса Христа, що ним розп'ятий світ для мене, а я для світу.
\end{tcolorbox}
\begin{tcolorbox}
\textsubscript{15} Бо сили немає ані обрізання, ані необрізання, а створіння нове.
\end{tcolorbox}
\begin{tcolorbox}
\textsubscript{16} А всі ті, хто піде за цим правилом, мир та милість на них, і на Ізраїля Божого!
\end{tcolorbox}
\begin{tcolorbox}
\textsubscript{17} Зрештою, хай ніхто не турбує мене, бо ношу я Ісусові рани на тілі своїм!...
\end{tcolorbox}
\begin{tcolorbox}
\textsubscript{18} Благодать Господа нашого Ісуса Христа нехай буде з духом вашим, браття! Амінь.
\end{tcolorbox}
