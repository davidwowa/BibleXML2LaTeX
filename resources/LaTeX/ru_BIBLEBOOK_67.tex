\section{BOOK 66}
\subsection{CHAPTER 1}
\begin{tcolorbox}
\textsubscript{1} Видение Исаии, сына Амосова, которое он видел о Иудее и Иерусалиме, во дни Озии, Иоафама, Ахаза, Езекии--царей Иудейских.
\end{tcolorbox}
\begin{tcolorbox}
\textsubscript{2} Слушайте, небеса, и внимай, земля, потому что Господь говорит: Я воспитал и возвысил сыновей, а они возмутились против Меня.
\end{tcolorbox}
\begin{tcolorbox}
\textsubscript{3} Вол знает владетеля своего, и осел--ясли господина своего; а Израиль не знает [Меня], народ Мой не разумеет.
\end{tcolorbox}
\begin{tcolorbox}
\textsubscript{4} Увы, народ грешный, народ обремененный беззакониями, племя злодеев, сыны погибельные! Оставили Господа, презрели Святаго Израилева, --повернулись назад.
\end{tcolorbox}
\begin{tcolorbox}
\textsubscript{5} Во что вас бить еще, продолжающие свое упорство? Вся голова в язвах, и все сердце исчахло.
\end{tcolorbox}
\begin{tcolorbox}
\textsubscript{6} От подошвы ноги до темени головы нет у него здорового места: язвы, пятна, гноящиеся раны, неочищенные и необвязанные и не смягченные елеем.
\end{tcolorbox}
\begin{tcolorbox}
\textsubscript{7} Земля ваша опустошена; города ваши сожжены огнем; поля ваши в ваших глазах съедают чужие; все опустело, как после разорения чужими.
\end{tcolorbox}
\begin{tcolorbox}
\textsubscript{8} И осталась дщерь Сиона, как шатер в винограднике, как шалаш в огороде, как осажденный город.
\end{tcolorbox}
\begin{tcolorbox}
\textsubscript{9} Если бы Господь Саваоф не оставил нам небольшого остатка, то мы были бы то же, что Содом, уподобились бы Гоморре.
\end{tcolorbox}
\begin{tcolorbox}
\textsubscript{10} Слушайте слово Господне, князья Содомские; внимай закону Бога нашего, народ Гоморрский!
\end{tcolorbox}
\begin{tcolorbox}
\textsubscript{11} К чему Мне множество жертв ваших? говорит Господь. Я пресыщен всесожжениями овнов и туком откормленного скота, и крови тельцов и агнцев и козлов не хочу.
\end{tcolorbox}
\begin{tcolorbox}
\textsubscript{12} Когда вы приходите являться пред лице Мое, кто требует от вас, чтобы вы топтали дворы Мои?
\end{tcolorbox}
\begin{tcolorbox}
\textsubscript{13} Не носите больше даров тщетных: курение отвратительно для Меня; новомесячий и суббот, праздничных собраний не могу терпеть: беззаконие--и празднование!
\end{tcolorbox}
\begin{tcolorbox}
\textsubscript{14} Новомесячия ваши и праздники ваши ненавидит душа Моя: они бремя для Меня; Мне тяжело нести их.
\end{tcolorbox}
\begin{tcolorbox}
\textsubscript{15} И когда вы простираете руки ваши, Я закрываю от вас очи Мои; и когда вы умножаете моления ваши, Я не слышу: ваши руки полны крови.
\end{tcolorbox}
\begin{tcolorbox}
\textsubscript{16} Омойтесь, очиститесь; удалите злые деяния ваши от очей Моих; перестаньте делать зло;
\end{tcolorbox}
\begin{tcolorbox}
\textsubscript{17} научитесь делать добро, ищите правды, спасайте угнетенного, защищайте сироту, вступайтесь за вдову.
\end{tcolorbox}
\begin{tcolorbox}
\textsubscript{18} Тогда придите--и рассудим, говорит Господь. Если будут грехи ваши, как багряное, --как снег убелю; если будут красны, как пурпур, --как волну убелю.
\end{tcolorbox}
\begin{tcolorbox}
\textsubscript{19} Если захотите и послушаетесь, то будете вкушать блага земли;
\end{tcolorbox}
\begin{tcolorbox}
\textsubscript{20} если же отречетесь и будете упорствовать, то меч пожрет вас: ибо уста Господни говорят.
\end{tcolorbox}
\begin{tcolorbox}
\textsubscript{21} Как сделалась блудницею верная столица, исполненная правосудия! Правда обитала в ней, а теперь--убийцы.
\end{tcolorbox}
\begin{tcolorbox}
\textsubscript{22} Серебро твое стало изгарью, вино твое испорчено водою;
\end{tcolorbox}
\begin{tcolorbox}
\textsubscript{23} князья твои--законопреступники и сообщники воров; все они любят подарки и гоняются за мздою; не защищают сироты, и дело вдовы не доходит до них.
\end{tcolorbox}
\begin{tcolorbox}
\textsubscript{24} Посему говорит Господь, Господь Саваоф, Сильный Израилев: о, удовлетворю Я Себя над противниками Моими и отмщу врагам Моим!
\end{tcolorbox}
\begin{tcolorbox}
\textsubscript{25} И обращу на тебя руку Мою и, как в щелочи, очищу с тебя примесь, и отделю от тебя все свинцовое;
\end{tcolorbox}
\begin{tcolorbox}
\textsubscript{26} и опять буду поставлять тебе судей, как прежде, и советников, как вначале; тогда будут говорить о тебе: 'город правды, столица верная'.
\end{tcolorbox}
\begin{tcolorbox}
\textsubscript{27} Сион спасется правосудием, и обратившиеся [сыны] его--правдою;
\end{tcolorbox}
\begin{tcolorbox}
\textsubscript{28} всем же отступникам и грешникам--погибель, и оставившие Господа истребятся.
\end{tcolorbox}
\begin{tcolorbox}
\textsubscript{29} Они будут постыжены за дубравы, которые столь вожделенны для вас, и посрамлены за сады, которые вы избрали себе;
\end{tcolorbox}
\begin{tcolorbox}
\textsubscript{30} ибо вы будете, как дуб, [которого] лист опал, и как сад, в котором нет воды.
\end{tcolorbox}
\begin{tcolorbox}
\textsubscript{31} И сильный будет отрепьем, и дело его--искрою; и будут гореть вместе, --и никто не потушит.
\end{tcolorbox}
\subsection{CHAPTER 2}
\begin{tcolorbox}
\textsubscript{1} Слово, которое было в видении к Исаии, сыну Амосову, о Иудее и Иерусалиме.
\end{tcolorbox}
\begin{tcolorbox}
\textsubscript{2} И будет в последние дни, гора дома Господня будет поставлена во главу гор и возвысится над холмами, и потекут к ней все народы.
\end{tcolorbox}
\begin{tcolorbox}
\textsubscript{3} И пойдут многие народы и скажут: придите, и взойдем на гору Господню, в дом Бога Иаковлева, и научит Он нас Своим путям и будем ходить по стезям Его; ибо от Сиона выйдет закон, и слово Господне--из Иерусалима.
\end{tcolorbox}
\begin{tcolorbox}
\textsubscript{4} И будет Он судить народы, и обличит многие племена; и перекуют мечи свои на орала, и копья свои--на серпы: не поднимет народ на народ меча, и не будут более учиться воевать.
\end{tcolorbox}
\begin{tcolorbox}
\textsubscript{5} О, дом Иакова! Придите, и будем ходить во свете Господнем.
\end{tcolorbox}
\begin{tcolorbox}
\textsubscript{6} Но Ты отринул народ Твой, дом Иакова, потому что они многое переняли от востока: и чародеи [у них], как у Филистимлян, и с сынами чужих они в общении.
\end{tcolorbox}
\begin{tcolorbox}
\textsubscript{7} И наполнилась земля его серебром и золотом, и нет числа сокровищам его; и наполнилась земля его конями, и нет числа колесницам его;
\end{tcolorbox}
\begin{tcolorbox}
\textsubscript{8} и наполнилась земля его идолами: они поклоняются делу рук своих, тому, что сделали персты их.
\end{tcolorbox}
\begin{tcolorbox}
\textsubscript{9} И преклонился человек, и унизился муж, --и Ты не простишь их.
\end{tcolorbox}
\begin{tcolorbox}
\textsubscript{10} Иди в скалу и сокройся в землю от страха Господа и от славы величия Его.
\end{tcolorbox}
\begin{tcolorbox}
\textsubscript{11} Поникнут гордые взгляды человека, и высокое людское унизится; и один Господь будет высок в тот день.
\end{tcolorbox}
\begin{tcolorbox}
\textsubscript{12} Ибо [грядет] день Господа Саваофа на все гордое и высокомерное и на все превознесенное, --и оно будет унижено, --
\end{tcolorbox}
\begin{tcolorbox}
\textsubscript{13} и на все кедры Ливанские, высокие и превозносящиеся, и на все дубы Васанские,
\end{tcolorbox}
\begin{tcolorbox}
\textsubscript{14} и на все высокие горы, и на все возвышающиеся холмы,
\end{tcolorbox}
\begin{tcolorbox}
\textsubscript{15} и на всякую высокую башню, и на всякую крепкую стену,
\end{tcolorbox}
\begin{tcolorbox}
\textsubscript{16} и на все корабли Фарсисские, и на все вожделенные украшения их.
\end{tcolorbox}
\begin{tcolorbox}
\textsubscript{17} И падет величие человеческое, и высокое людское унизится; и один Господь будет высок в тот день,
\end{tcolorbox}
\begin{tcolorbox}
\textsubscript{18} и идолы совсем исчезнут.
\end{tcolorbox}
\begin{tcolorbox}
\textsubscript{19} И войдут [люди] в расселины скал и в пропасти земли от страха Господа и от славы величия Его, когда Он восстанет сокрушить землю.
\end{tcolorbox}
\begin{tcolorbox}
\textsubscript{20} В тот день человек бросит кротам и летучим мышам серебряных своих идолов и золотых своих идолов, которых сделал себе для поклонения им,
\end{tcolorbox}
\begin{tcolorbox}
\textsubscript{21} чтобы войти в ущелья скал и в расселины гор от страха Господа и от славы величия Его, когда Он восстанет сокрушить землю.
\end{tcolorbox}
\begin{tcolorbox}
\textsubscript{22} Перестаньте вы надеяться на человека, которого дыхание в ноздрях его, ибо что он значит?
\end{tcolorbox}
\subsection{CHAPTER 3}
\begin{tcolorbox}
\textsubscript{1} Вот, Господь, Господь Саваоф, отнимет у Иерусалима и у Иуды посох и трость, всякое подкрепление хлебом и всякое подкрепление водою,
\end{tcolorbox}
\begin{tcolorbox}
\textsubscript{2} храброго вождя и воина, судью и пророка, и прозорливца и старца,
\end{tcolorbox}
\begin{tcolorbox}
\textsubscript{3} пятидесятника и вельможу и советника, и мудрого художника и искусного в слове.
\end{tcolorbox}
\begin{tcolorbox}
\textsubscript{4} И дам им отроков в начальники, и дети будут господствовать над ними.
\end{tcolorbox}
\begin{tcolorbox}
\textsubscript{5} И в народе один будет угнетаем другим, и каждый--ближним своим; юноша будет нагло превозноситься над старцем, и простолюдин над вельможею.
\end{tcolorbox}
\begin{tcolorbox}
\textsubscript{6} Тогда ухватится человек за брата своего, в семействе отца своего, [и скажет]: у тебя [есть] одежда, будь нашим вождем, и да будут эти развалины под рукою твоею.
\end{tcolorbox}
\begin{tcolorbox}
\textsubscript{7} А [он] с клятвою скажет: не могу исцелить [ран общества]; и в моем доме нет ни хлеба, ни одежды; не делайте меня вождем народа.
\end{tcolorbox}
\begin{tcolorbox}
\textsubscript{8} Так рушился Иерусалим, и пал Иуда, потому что язык их и дела их--против Господа, оскорбительны для очей славы Его.
\end{tcolorbox}
\begin{tcolorbox}
\textsubscript{9} Выражение лиц их свидетельствует против них, и о грехе своем они рассказывают открыто, как Содомляне, не скрывают: горе душе их! ибо сами на себя навлекают зло.
\end{tcolorbox}
\begin{tcolorbox}
\textsubscript{10} Скажите праведнику, что благо [ему], ибо он будет вкушать плоды дел своих;
\end{tcolorbox}
\begin{tcolorbox}
\textsubscript{11} а беззаконнику--горе, ибо будет ему возмездие за [дела] рук его.
\end{tcolorbox}
\begin{tcolorbox}
\textsubscript{12} Притеснители народа Моего--дети, и женщины господствуют над ним. Народ Мой! вожди твои вводят тебя в заблуждение и путь стезей твоих испортили.
\end{tcolorbox}
\begin{tcolorbox}
\textsubscript{13} Восстал Господь на суд--и стоит, чтобы судить народы.
\end{tcolorbox}
\begin{tcolorbox}
\textsubscript{14} Господь вступает в суд со старейшинами народа Своего и с князьями его: вы опустошили виноградник; награбленное у бедного--в ваших домах;
\end{tcolorbox}
\begin{tcolorbox}
\textsubscript{15} что вы тесните народ Мой и угнетаете бедных? говорит Господь, Господь Саваоф.
\end{tcolorbox}
\begin{tcolorbox}
\textsubscript{16} И сказал Господь: за то, что дочери Сиона надменны и ходят, подняв шею и обольщая взорами, и выступают величавою поступью и гремят цепочками на ногах, --
\end{tcolorbox}
\begin{tcolorbox}
\textsubscript{17} оголит Господь темя дочерей Сиона и обнажит Господь срамоту их;
\end{tcolorbox}
\begin{tcolorbox}
\textsubscript{18} в тот день отнимет Господь красивые цепочки на ногах и звездочки, и луночки,
\end{tcolorbox}
\begin{tcolorbox}
\textsubscript{19} серьги, и ожерелья, и опахала, увясла и запястья, и пояса, и сосудцы с духами, и привески волшебные,
\end{tcolorbox}
\begin{tcolorbox}
\textsubscript{20} перстни и кольца в носу,
\end{tcolorbox}
\begin{tcolorbox}
\textsubscript{21} верхнюю одежду и нижнюю, и платки, и кошельки,
\end{tcolorbox}
\begin{tcolorbox}
\textsubscript{22} светлые тонкие епанчи и повязки, и покрывала.
\end{tcolorbox}
\begin{tcolorbox}
\textsubscript{23} И будет вместо благовония зловоние, и вместо пояса будет веревка, и вместо завитых волос--плешь, и вместо широкой епанчи--узкое вретище, вместо красоты--клеймо.
\end{tcolorbox}
\begin{tcolorbox}
\textsubscript{24} Мужи твои падут от меча, и храбрые твои--на войне.
\end{tcolorbox}
\begin{tcolorbox}
\textsubscript{25} И будут воздыхать и плакать ворота [столицы],
\end{tcolorbox}
\begin{tcolorbox}
\textsubscript{26} и будет она сидеть на земле опустошенная.
\end{tcolorbox}
\subsection{CHAPTER 4}
\begin{tcolorbox}
\textsubscript{1} И ухватятся семь женщин за одного мужчину в тот день, и скажут: 'свой хлеб будем есть и свою одежду будем носить, только пусть будем называться твоим именем, --сними с нас позор'.
\end{tcolorbox}
\begin{tcolorbox}
\textsubscript{2} В тот день отрасль Господа явится в красоте и чести, и плод земли--в величии и славе, для уцелевших [сынов] Израиля.
\end{tcolorbox}
\begin{tcolorbox}
\textsubscript{3} Тогда оставшиеся на Сионе и уцелевшие в Иерусалиме будут именоваться святыми, все вписанные в книгу для житья в Иерусалиме,
\end{tcolorbox}
\begin{tcolorbox}
\textsubscript{4} когда Господь омоет скверну дочерей Сиона и очистит кровь Иерусалима из среды его духом суда и духом огня.
\end{tcolorbox}
\begin{tcolorbox}
\textsubscript{5} И сотворит Господь над всяким местом горы Сиона и над собраниями ее облако и дым во время дня и блистание пылающего огня во время ночи; ибо над всем чтимым будет покров.
\end{tcolorbox}
\begin{tcolorbox}
\textsubscript{6} И будет шатер для осенения днем от зноя и для убежища и защиты от непогод и дождя.
\end{tcolorbox}
\subsection{CHAPTER 5}
\begin{tcolorbox}
\textsubscript{1} Воспою Возлюбленному моему песнь Возлюбленного моего о винограднике Его. У Возлюбленного моего был виноградник на вершине утучненной горы,
\end{tcolorbox}
\begin{tcolorbox}
\textsubscript{2} и Он обнес его оградою, и очистил его от камней, и насадил в нем отборные виноградные лозы, и построил башню посреди его, и выкопал в нем точило, и ожидал, что он принесет добрые грозды, а он принес дикие ягоды.
\end{tcolorbox}
\begin{tcolorbox}
\textsubscript{3} И ныне, жители Иерусалима и мужи Иуды, рассудите Меня с виноградником Моим.
\end{tcolorbox}
\begin{tcolorbox}
\textsubscript{4} Что еще надлежало бы сделать для виноградника Моего, чего Я не сделал ему? Почему, когда Я ожидал, что он принесет добрые грозды, он принес дикие ягоды?
\end{tcolorbox}
\begin{tcolorbox}
\textsubscript{5} Итак Я скажу вам, что сделаю с виноградником Моим: отниму у него ограду, и будет он опустошаем; разрушу стены его, и будет попираем,
\end{tcolorbox}
\begin{tcolorbox}
\textsubscript{6} и оставлю его в запустении: не будут ни обрезывать, ни вскапывать его, --и зарастет он тернами и волчцами, и повелю облакам не проливать на него дождя.
\end{tcolorbox}
\begin{tcolorbox}
\textsubscript{7} Виноградник Господа Саваофа есть дом Израилев, и мужи Иуды--любимое насаждение Его. И ждал Он правосудия, но вот--кровопролитие; [ждал] правды, и вот--вопль.
\end{tcolorbox}
\begin{tcolorbox}
\textsubscript{8} Горе вам, прибавляющие дом к дому, присоединяющие поле к полю, так что [другим] не остается места, как будто вы одни поселены на земле.
\end{tcolorbox}
\begin{tcolorbox}
\textsubscript{9} В уши мои [сказал] Господь Саваоф: многочисленные домы эти будут пусты, большие и красивые--без жителей;
\end{tcolorbox}
\begin{tcolorbox}
\textsubscript{10} десять участков в винограднике дадут один бат, и хомер посеянного зерна едва принесет ефу.
\end{tcolorbox}
\begin{tcolorbox}
\textsubscript{11} Горе тем, которые с раннего утра ищут сикеры и до позднего вечера разгорячают себя вином;
\end{tcolorbox}
\begin{tcolorbox}
\textsubscript{12} и цитра и гусли, тимпан и свирель и вино на пиршествах их; а на дела Господа они не взирают и о деяниях рук Его не помышляют.
\end{tcolorbox}
\begin{tcolorbox}
\textsubscript{13} За то народ мой пойдет в плен непредвиденно, и вельможи его будут голодать, и богачи его будут томиться жаждою.
\end{tcolorbox}
\begin{tcolorbox}
\textsubscript{14} За то преисподняя расширилась и без меры раскрыла пасть свою: и сойдет [туда] слава их и богатство их, и шум их и [всё], что веселит их.
\end{tcolorbox}
\begin{tcolorbox}
\textsubscript{15} И преклонится человек, и смирится муж, и глаза гордых поникнут;
\end{tcolorbox}
\begin{tcolorbox}
\textsubscript{16} а Господь Саваоф превознесется в суде, и Бог Святый явит святость Свою в правде.
\end{tcolorbox}
\begin{tcolorbox}
\textsubscript{17} И будут пастись овцы по своей воле, и чужие будут питаться оставленными жирными пажитями богатых.
\end{tcolorbox}
\begin{tcolorbox}
\textsubscript{18} Горе тем, которые влекут на себя беззаконие вервями суетности, и грех--как бы ремнями колесничными;
\end{tcolorbox}
\begin{tcolorbox}
\textsubscript{19} которые говорят: 'пусть Он поспешит и ускорит дело Свое, чтобы мы видели, и пусть приблизится и придет в исполнение совет Святаго Израилева, чтобы мы узнали!'
\end{tcolorbox}
\begin{tcolorbox}
\textsubscript{20} Горе тем, которые зло называют добром, и добро--злом, тьму почитают светом, и свет--тьмою, горькое почитают сладким, и сладкое--горьким!
\end{tcolorbox}
\begin{tcolorbox}
\textsubscript{21} Горе тем, которые мудры в своих глазах и разумны пред самими собою!
\end{tcolorbox}
\begin{tcolorbox}
\textsubscript{22} Горе тем, которые храбры пить вино и сильны приготовлять крепкий напиток,
\end{tcolorbox}
\begin{tcolorbox}
\textsubscript{23} которые за подарки оправдывают виновного и правых лишают законного!
\end{tcolorbox}
\begin{tcolorbox}
\textsubscript{24} За то, как огонь съедает солому, и пламя истребляет сено, так истлеет корень их, и цвет их разнесется, как прах; потому что они отвергли закон Господа Саваофа и презрели слово Святаго Израилева.
\end{tcolorbox}
\begin{tcolorbox}
\textsubscript{25} За то возгорится гнев Господа на народ Его, и прострет Он руку Свою на него и поразит его, так что содрогнутся горы, и трупы их будут как помет на улицах. И при всем этом гнев Его не отвратится, и рука Его еще будет простерта.
\end{tcolorbox}
\begin{tcolorbox}
\textsubscript{26} И поднимет знамя народам дальним, и даст знак живущему на краю земли, --и вот, он легко и скоро придет;
\end{tcolorbox}
\begin{tcolorbox}
\textsubscript{27} не будет у него ни усталого, ни изнемогающего; ни один не задремлет и не заснет, и не снимется пояс с чресл его, и не разорвется ремень у обуви его;
\end{tcolorbox}
\begin{tcolorbox}
\textsubscript{28} стрелы его заострены, и все луки его натянуты; копыта коней его подобны кремню, и колеса его--как вихрь;
\end{tcolorbox}
\begin{tcolorbox}
\textsubscript{29} рев его--как рев львицы; он рыкает подобно скимнам, и заревет, и схватит добычу и унесет, и никто не отнимет.
\end{tcolorbox}
\begin{tcolorbox}
\textsubscript{30} И заревет на него в тот день как бы рев [разъяренного] моря; и взглянет он на землю, и вот--тьма, горе, и свет померк в облаках.
\end{tcolorbox}
\subsection{CHAPTER 6}
\begin{tcolorbox}
\textsubscript{1} В год смерти царя Озии видел я Господа, сидящего на престоле высоком и превознесенном, и края риз Его наполняли весь храм.
\end{tcolorbox}
\begin{tcolorbox}
\textsubscript{2} Вокруг Него стояли Серафимы; у каждого из них по шести крыл: двумя закрывал каждый лице свое, и двумя закрывал ноги свои, и двумя летал.
\end{tcolorbox}
\begin{tcolorbox}
\textsubscript{3} И взывали они друг ко другу и говорили: Свят, Свят, Свят Господь Саваоф! вся земля полна славы Его!
\end{tcolorbox}
\begin{tcolorbox}
\textsubscript{4} И поколебались верхи врат от гласа восклицающих, и дом наполнился курениями.
\end{tcolorbox}
\begin{tcolorbox}
\textsubscript{5} И сказал я: горе мне! погиб я! ибо я человек с нечистыми устами, и живу среди народа также с нечистыми устами, --и глаза мои видели Царя, Господа Саваофа.
\end{tcolorbox}
\begin{tcolorbox}
\textsubscript{6} Тогда прилетел ко мне один из Серафимов, и в руке у него горящий уголь, который он взял клещами с жертвенника,
\end{tcolorbox}
\begin{tcolorbox}
\textsubscript{7} и коснулся уст моих и сказал: вот, это коснулось уст твоих, и беззаконие твое удалено от тебя, и грех твой очищен.
\end{tcolorbox}
\begin{tcolorbox}
\textsubscript{8} И услышал я голос Господа, говорящего: кого Мне послать? и кто пойдет для Нас? И я сказал: вот я, пошли меня.
\end{tcolorbox}
\begin{tcolorbox}
\textsubscript{9} И сказал Он: пойди и скажи этому народу: слухом услышите--и не уразумеете, и очами смотреть будете--и не увидите.
\end{tcolorbox}
\begin{tcolorbox}
\textsubscript{10} Ибо огрубело сердце народа сего, и ушами с трудом слышат, и очи свои сомкнули, да не узрят очами, и не услышат ушами, и не уразумеют сердцем, и не обратятся, чтобы Я исцелил их.
\end{tcolorbox}
\begin{tcolorbox}
\textsubscript{11} И сказал я: надолго ли, Господи? Он сказал: доколе не опустеют города, и останутся без жителей, и домы без людей, и доколе земля эта совсем не опустеет.
\end{tcolorbox}
\begin{tcolorbox}
\textsubscript{12} И удалит Господь людей, и великое запустение будет на этой земле.
\end{tcolorbox}
\begin{tcolorbox}
\textsubscript{13} И если еще останется десятая часть на ней и возвратится, и она опять будет разорена; [но] как от теревинфа и как от дуба, когда они и срублены, [остается] корень их, так святое семя [будет] корнем ее.
\end{tcolorbox}
\subsection{CHAPTER 7}
\begin{tcolorbox}
\textsubscript{1} И было во дни Ахаза, сына Иоафамова, сына Озии, царя Иудейского, Рецин, царь Сирийский, и факей, сын Ремалиин, царь Израильский, пошли против Иерусалима, чтобы завоевать его, но не могли завоевать.
\end{tcolorbox}
\begin{tcolorbox}
\textsubscript{2} И было возвещено дому Давидову и сказано: Сирияне расположились в земле Ефремовой; и всколебалось сердце его и сердце народа его, как колеблются от ветра дерева в лесу.
\end{tcolorbox}
\begin{tcolorbox}
\textsubscript{3} И сказал Господь Исаии: выйди ты и сын твой Шеар-ясув навстречу Ахазу, к концу водопровода верхнего пруда, на дорогу к полю белильничьему,
\end{tcolorbox}
\begin{tcolorbox}
\textsubscript{4} и скажи ему: наблюдай и будь спокоен; не страшись и да не унывает сердце твое от двух концов этих дымящихся головней, от разгоревшегося гнева Рецина и Сириян и сына Ремалиина.
\end{tcolorbox}
\begin{tcolorbox}
\textsubscript{5} Сирия, Ефрем и сын Ремалиин умышляют против тебя зло, говоря:
\end{tcolorbox}
\begin{tcolorbox}
\textsubscript{6} пойдем на Иудею и возмутим ее, и овладеем ею и поставим в ней царем сына Тавеилова.
\end{tcolorbox}
\begin{tcolorbox}
\textsubscript{7} Но Господь Бог так говорит: это не состоится и не сбудется;
\end{tcolorbox}
\begin{tcolorbox}
\textsubscript{8} ибо глава Сирии--Дамаск, и глава Дамаска--Рецин; а чрез шестьдесят пять лет Ефрем перестанет быть народом;
\end{tcolorbox}
\begin{tcolorbox}
\textsubscript{9} и глава Ефрема--Самария, и глава Самарии--сын Ремалиин. Если вы не верите, то потому, что вы не удостоверены.
\end{tcolorbox}
\begin{tcolorbox}
\textsubscript{10} И продолжал Господь говорить к Ахазу, и сказал:
\end{tcolorbox}
\begin{tcolorbox}
\textsubscript{11} проси себе знамения у Господа Бога твоего: проси или в глубине, или на высоте.
\end{tcolorbox}
\begin{tcolorbox}
\textsubscript{12} И сказал Ахаз: не буду просить и не буду искушать Господа.
\end{tcolorbox}
\begin{tcolorbox}
\textsubscript{13} Тогда сказал [Исаия]: слушайте же, дом Давидов! разве мало для вас затруднять людей, что вы хотите затруднять и Бога моего?
\end{tcolorbox}
\begin{tcolorbox}
\textsubscript{14} Итак Сам Господь даст вам знамение: се, Дева во чреве приимет и родит Сына, и нарекут имя Ему: Еммануил.
\end{tcolorbox}
\begin{tcolorbox}
\textsubscript{15} Он будет питаться молоком и медом, доколе не будет разуметь отвергать худое и избирать доброе;
\end{tcolorbox}
\begin{tcolorbox}
\textsubscript{16} ибо прежде нежели этот младенец будет разуметь отвергать худое и избирать доброе, земля та, которой ты страшишься, будет оставлена обоими царями ее.
\end{tcolorbox}
\begin{tcolorbox}
\textsubscript{17} Но наведет Господь на тебя и на народ твой и на дом отца твоего дни, какие не приходили со времени отпадения Ефрема от Иуды, наведет царя Ассирийского.
\end{tcolorbox}
\begin{tcolorbox}
\textsubscript{18} И будет в тот день: даст знак Господь мухе, которая при устье реки Египетской, и пчеле, которая в земле Ассирийской, --
\end{tcolorbox}
\begin{tcolorbox}
\textsubscript{19} и прилетят и усядутся все они по долинам опустелым и по расселинам скал, и по всем колючим кустарникам, и по всем деревам.
\end{tcolorbox}
\begin{tcolorbox}
\textsubscript{20} В тот день обреет Господь бритвою, нанятою по ту сторону реки, царем Ассирийским, голову и волоса на ногах, и даже отнимет бороду.
\end{tcolorbox}
\begin{tcolorbox}
\textsubscript{21} И будет в тот день: кто будет содержать корову и двух овец,
\end{tcolorbox}
\begin{tcolorbox}
\textsubscript{22} по изобилию молока, которое они дадут, будет есть масло; маслом и медом будут питаться все, оставшиеся в этой земле.
\end{tcolorbox}
\begin{tcolorbox}
\textsubscript{23} И будет в тот день: на всяком месте, где росла тысяча виноградных лоз на тысячу сребренников, будет терновник и колючий кустарник.
\end{tcolorbox}
\begin{tcolorbox}
\textsubscript{24} Со стрелами и луками будут ходить туда, ибо вся земля будет терновником и колючим кустарником.
\end{tcolorbox}
\begin{tcolorbox}
\textsubscript{25} И ни на одну из гор, которые расчищались бороздниками, не пойдешь, боясь терновника и колючего кустарника: туда будут выгонять волов, и мелкий скот будет топтать их.
\end{tcolorbox}
\subsection{CHAPTER 8}
\begin{tcolorbox}
\textsubscript{1} И сказал мне Господь: возьми себе большой свиток и начертай на нем человеческим письмом: Магер-шелал-хаш-баз.
\end{tcolorbox}
\begin{tcolorbox}
\textsubscript{2} И я взял себе верных свидетелей: Урию священника и Захарию, сына Варахиина, --
\end{tcolorbox}
\begin{tcolorbox}
\textsubscript{3} и приступил я к пророчице, и она зачала и родила сына. И сказал мне Господь: нареки ему имя: Магер-шелал-хаш-баз,
\end{tcolorbox}
\begin{tcolorbox}
\textsubscript{4} ибо прежде нежели дитя будет уметь выговорить: отец мой, мать моя, --богатства Дамаска и добычи Самарийские понесут перед царем Ассирийским.
\end{tcolorbox}
\begin{tcolorbox}
\textsubscript{5} И продолжал Господь говорить ко мне и сказал еще:
\end{tcolorbox}
\begin{tcolorbox}
\textsubscript{6} за то, что этот народ пренебрегает водами Силоама, текущими тихо, и восхищается Рецином и сыном Ремалииным,
\end{tcolorbox}
\begin{tcolorbox}
\textsubscript{7} наведет на него Господь воды реки бурные и большие--царя Ассирийского со всею славою его; и поднимется она во всех протоках своих и выступит из всех берегов своих;
\end{tcolorbox}
\begin{tcolorbox}
\textsubscript{8} и пойдет по Иудее, наводнит ее и высоко поднимется--дойдет до шеи; и распростертие крыльев ее будет во всю широту земли Твоей, Еммануил!
\end{tcolorbox}
\begin{tcolorbox}
\textsubscript{9} Враждуйте, народы, но трепещите, и внимайте, все отдаленные земли! Вооружайтесь, но трепещите; вооружайтесь, но трепещите!
\end{tcolorbox}
\begin{tcolorbox}
\textsubscript{10} Замышляйте замыслы, но они рушатся; говорите слово, но оно не состоится: ибо с нами Бог!
\end{tcolorbox}
\begin{tcolorbox}
\textsubscript{11} Ибо так говорил мне Господь, [держа на мне] крепкую руку и внушая мне не ходить путем сего народа, и сказал:
\end{tcolorbox}
\begin{tcolorbox}
\textsubscript{12} 'Не называйте заговором всего того, что народ сей называет заговором; и не бойтесь того, чего он боится, и не страшитесь.
\end{tcolorbox}
\begin{tcolorbox}
\textsubscript{13} Господа Саваофа--Его чтите свято, и Он--страх ваш, и Он--трепет ваш!
\end{tcolorbox}
\begin{tcolorbox}
\textsubscript{14} И будет Он освящением и камнем преткновения, и скалою соблазна для обоих домов Израиля, петлею и сетью для жителей Иерусалима.
\end{tcolorbox}
\begin{tcolorbox}
\textsubscript{15} И многие из них преткнутся и упадут, и разобьются, и запутаются в сети, и будут уловлены.
\end{tcolorbox}
\begin{tcolorbox}
\textsubscript{16} Завяжи свидетельство, и запечатай откровение при учениках Моих'.
\end{tcolorbox}
\begin{tcolorbox}
\textsubscript{17} Итак я надеюсь на Господа, сокрывшего лице Свое от дома Иаковлева, и уповаю на Него.
\end{tcolorbox}
\begin{tcolorbox}
\textsubscript{18} Вот я и дети, которых дал мне Господь, как указания и предзнаменования в Израиле от Господа Саваофа, живущего на горе Сионе.
\end{tcolorbox}
\begin{tcolorbox}
\textsubscript{19} И когда скажут вам: обратитесь к вызывателям умерших и к чародеям, к шептунам и чревовещателям, --тогда отвечайте: не должен ли народ обращаться к своему Богу? спрашивают ли мертвых о живых?
\end{tcolorbox}
\begin{tcolorbox}
\textsubscript{20} [Обращайтесь] к закону и откровению. Если они не говорят, как это слово, то нет в них света.
\end{tcolorbox}
\begin{tcolorbox}
\textsubscript{21} И будут они бродить по земле, жестоко угнетенные и голодные; и во время голода будут злиться, хулить царя своего и Бога своего.
\end{tcolorbox}
\begin{tcolorbox}
\textsubscript{22} И взглянут вверх, и посмотрят на землю; и вот--горе и мрак, густая тьма, и будут повержены во тьму. Но не всегда будет мрак там, где теперь он сгустел.
\end{tcolorbox}
\subsection{CHAPTER 9}
\begin{tcolorbox}
\textsubscript{1} Прежнее время умалило землю Завулонову и землю Неффалимову; но последующее возвеличит приморский путь, Заиорданскую страну, Галилею языческую.
\end{tcolorbox}
\begin{tcolorbox}
\textsubscript{2} Народ, ходящий во тьме, увидит свет великий; на живущих в стране тени смертной свет воссияет.
\end{tcolorbox}
\begin{tcolorbox}
\textsubscript{3} Ты умножишь народ, увеличишь радость его. Он будет веселиться пред Тобою, как веселятся во время жатвы, как радуются при разделе добычи.
\end{tcolorbox}
\begin{tcolorbox}
\textsubscript{4} Ибо ярмо, тяготившее его, и жезл, поражавший его, и трость притеснителя его Ты сокрушишь, как в день Мадиама.
\end{tcolorbox}
\begin{tcolorbox}
\textsubscript{5} Ибо всякая обувь воина во время брани и одежда, обагренная кровью, будут отданы на сожжение, в пищу огню.
\end{tcolorbox}
\begin{tcolorbox}
\textsubscript{6} Ибо младенец родился нам--Сын дан нам; владычество на раменах Его, и нарекут имя Ему: Чудный, Советник, Бог крепкий, Отец вечности, Князь мира.
\end{tcolorbox}
\begin{tcolorbox}
\textsubscript{7} Умножению владычества Его и мира нет предела на престоле Давида и в царстве его, чтобы Ему утвердить его и укрепить его судом и правдою отныне и до века. Ревность Господа Саваофа соделает это.
\end{tcolorbox}
\begin{tcolorbox}
\textsubscript{8} Слово посылает Господь на Иакова, и оно нисходит на Израиля,
\end{tcolorbox}
\begin{tcolorbox}
\textsubscript{9} чтобы знал весь народ, Ефрем и жители Самарии, которые с гордостью и надменным сердцем говорят:
\end{tcolorbox}
\begin{tcolorbox}
\textsubscript{10} кирпичи пали--построим из тесаного камня; сикоморы вырублены--заменим их кедрами.
\end{tcolorbox}
\begin{tcolorbox}
\textsubscript{11} И воздвигнет Господь против него врагов Рецина, и неприятелей его вооружит:
\end{tcolorbox}
\begin{tcolorbox}
\textsubscript{12} Сириян с востока, а Филистимлян с запада; и будут они пожирать Израиля полным ртом. При всем этом не отвратится гнев Его, и рука Его еще простерта.
\end{tcolorbox}
\begin{tcolorbox}
\textsubscript{13} Но народ не обращается к Биющему его, и к Господу Саваофу не прибегает.
\end{tcolorbox}
\begin{tcolorbox}
\textsubscript{14} И отсечет Господь у Израиля голову и хвост, пальму и трость, в один день:
\end{tcolorbox}
\begin{tcolorbox}
\textsubscript{15} старец и знатный, --это голова; а пророк-лжеучитель есть хвост.
\end{tcolorbox}
\begin{tcolorbox}
\textsubscript{16} И вожди сего народа введут его в заблуждение, и водимые ими погибнут.
\end{tcolorbox}
\begin{tcolorbox}
\textsubscript{17} Поэтому о юношах его не порадуется Господь, и сирот его и вдов его не помилует: ибо все они--лицемеры и злодеи, и уста всех говорят нечестиво. При всем этом не отвратится гнев Его, и рука Его еще простерта.
\end{tcolorbox}
\begin{tcolorbox}
\textsubscript{18} Ибо беззаконие, как огонь, разгорелось, пожирает терновник и колючий кустарник и пылает в чащах леса, и поднимаются столбы дыма.
\end{tcolorbox}
\begin{tcolorbox}
\textsubscript{19} Ярость Господа Саваофа опалит землю, и народ сделается как бы пищею огня; не пощадит человек брата своего.
\end{tcolorbox}
\begin{tcolorbox}
\textsubscript{20} И будут резать по правую сторону, и останутся голодны; и будут есть по левую, и не будут сыты; каждый будет пожирать плоть мышцы своей:
\end{tcolorbox}
\begin{tcolorbox}
\textsubscript{21} Манассия--Ефрема, и Ефрем--Манассию, оба вместе--Иуду. При всем этом не отвратится гнев Его, и рука Его еще простерта.
\end{tcolorbox}
\subsection{CHAPTER 10}
\begin{tcolorbox}
\textsubscript{1} Горе тем, которые постановляют несправедливые законы и пишут жестокие решения,
\end{tcolorbox}
\begin{tcolorbox}
\textsubscript{2} чтобы устранить бедных от правосудия и похитить права у малосильных из народа Моего, чтобы вдов сделать добычею своею и ограбить сирот.
\end{tcolorbox}
\begin{tcolorbox}
\textsubscript{3} И что вы будете делать в день посещения, когда придет гибель издалека? К кому прибегнете за помощью? И где оставите богатство ваше?
\end{tcolorbox}
\begin{tcolorbox}
\textsubscript{4} Без Меня согнутся между узниками и падут между убитыми. При всем этом не отвратится гнев Его, и рука Его еще простерта.
\end{tcolorbox}
\begin{tcolorbox}
\textsubscript{5} О, Ассур, жезл гнева Моего! и бич в руке его--Мое негодование!
\end{tcolorbox}
\begin{tcolorbox}
\textsubscript{6} Я пошлю его против народа нечестивого и против народа гнева Моего, дам ему повеление ограбить грабежом и добыть добычу и попирать его, как грязь на улицах.
\end{tcolorbox}
\begin{tcolorbox}
\textsubscript{7} Но он не так подумает и не так помыслит сердце его; у него будет на сердце--разорить и истребить немало народов.
\end{tcolorbox}
\begin{tcolorbox}
\textsubscript{8} Ибо он скажет: 'не все ли цари князья мои?
\end{tcolorbox}
\begin{tcolorbox}
\textsubscript{9} Халне не то же ли, что Кархемис? Емаф не то же ли, что Арпад? Самария не то же ли, что Дамаск?
\end{tcolorbox}
\begin{tcolorbox}
\textsubscript{10} Так как рука моя овладела царствами идольскими, в которых кумиров более, нежели в Иерусалиме и Самарии, --
\end{tcolorbox}
\begin{tcolorbox}
\textsubscript{11} то не сделаю ли того же с Иерусалимом и изваяниями его, что сделал с Самариею и идолами ее?'
\end{tcolorbox}
\begin{tcolorbox}
\textsubscript{12} И будет, когда Господь совершит все Свое дело на горе Сионе и в Иерусалиме, скажет: посмотрю на успех надменного сердца царя Ассирийского и на тщеславие высоко поднятых глаз его.
\end{tcolorbox}
\begin{tcolorbox}
\textsubscript{13} Он говорит: 'силою руки моей и моею мудростью я сделал это, потому что я умен: и переставляю пределы народов, и расхищаю сокровища их, и низвергаю с престолов, как исполин;
\end{tcolorbox}
\begin{tcolorbox}
\textsubscript{14} и рука моя захватила богатство народов, как гнезда; и как забирают оставленные в них яйца, так забрал я всю землю, и никто не пошевелил крылом, и не открыл рта, и не пискнул'.
\end{tcolorbox}
\begin{tcolorbox}
\textsubscript{15} Величается ли секира пред тем, кто рубит ею? Пила гордится ли пред тем, кто двигает ее? Как будто жезл восстает против того, кто поднимает его; как будто палка поднимается на того, кто не дерево!
\end{tcolorbox}
\begin{tcolorbox}
\textsubscript{16} За то Господь, Господь Саваоф, пошлет чахлость на тучных его, и между знаменитыми его возжет пламя, как пламя огня.
\end{tcolorbox}
\begin{tcolorbox}
\textsubscript{17} Свет Израиля будет огнем, и Святый его--пламенем, которое сожжет и пожрет терны его и волчцы его в один день;
\end{tcolorbox}
\begin{tcolorbox}
\textsubscript{18} и славный лес его и сад его, от души до тела, истребит; и он будет, как чахлый умирающий.
\end{tcolorbox}
\begin{tcolorbox}
\textsubscript{19} И остаток дерев леса его так будет малочислен, что дитя в состоянии будет сделать опись.
\end{tcolorbox}
\begin{tcolorbox}
\textsubscript{20} И будет в тот день: остаток Израиля и спасшиеся из дома Иакова не будут более полагаться на того, кто поразил их, но возложат упование на Господа, Святаго Израилева, чистосердечно.
\end{tcolorbox}
\begin{tcolorbox}
\textsubscript{21} Остаток обратится, остаток Иакова--к Богу сильному.
\end{tcolorbox}
\begin{tcolorbox}
\textsubscript{22} Ибо, хотя бы народа у тебя, Израиль, [было] столько, сколько песку морского, только остаток его обратится; истребление определено изобилующею правдою;
\end{tcolorbox}
\begin{tcolorbox}
\textsubscript{23} ибо определенное истребление совершит Господь, Господь Саваоф, во всей земле.
\end{tcolorbox}
\begin{tcolorbox}
\textsubscript{24} Посему так говорит Господь, Господь Саваоф: народ Мой, живущий на Сионе! не бойся Ассура. Он поразит тебя жезлом и трость свою поднимет на тебя, как Египет.
\end{tcolorbox}
\begin{tcolorbox}
\textsubscript{25} Еще немного, очень немного, и пройдет Мое негодование, и ярость Моя [обратится] на истребление их.
\end{tcolorbox}
\begin{tcolorbox}
\textsubscript{26} И поднимет Господь Саваоф бич на него, как во время поражения Мадиама у скалы Орива, или как [простер] на море жезл, и поднимет его, как на Египет.
\end{tcolorbox}
\begin{tcolorbox}
\textsubscript{27} И будет в тот день: снимется с рамен твоих бремя его, и ярмо его--с шеи твоей; и распадется ярмо от тука.
\end{tcolorbox}
\begin{tcolorbox}
\textsubscript{28} Он идет на Аиаф, проходит Мигрон, в Михмасе складывает свои запасы.
\end{tcolorbox}
\begin{tcolorbox}
\textsubscript{29} Проходят теснины; в Геве ночлег их; Рама трясется; Гива Саулова разбежалась.
\end{tcolorbox}
\begin{tcolorbox}
\textsubscript{30} Вой голосом твоим, дочь Галима; пусть услышит тебя Лаис, бедный Анафоф!
\end{tcolorbox}
\begin{tcolorbox}
\textsubscript{31} Мадмена разбежалась, жители Гевима спешат уходить.
\end{tcolorbox}
\begin{tcolorbox}
\textsubscript{32} Еще день простоит он в Нове; грозит рукою своею горе Сиону, холму Иерусалимскому.
\end{tcolorbox}
\begin{tcolorbox}
\textsubscript{33} Вот, Господь, Господь Саваоф, страшною силою сорвет ветви дерев, и величающиеся ростом будут срублены, высокие--повержены на землю.
\end{tcolorbox}
\begin{tcolorbox}
\textsubscript{34} И посечет чащу леса железом, и Ливан падет от Всемогущего.
\end{tcolorbox}
\subsection{CHAPTER 11}
\begin{tcolorbox}
\textsubscript{1} И произойдет отрасль от корня Иессеева, и ветвь произрастет от корня его;
\end{tcolorbox}
\begin{tcolorbox}
\textsubscript{2} и почиет на нем Дух Господень, дух премудрости и разума, дух совета и крепости, дух ведения и благочестия;
\end{tcolorbox}
\begin{tcolorbox}
\textsubscript{3} и страхом Господним исполнится, и будет судить не по взгляду очей Своих и не по слуху ушей Своих решать дела.
\end{tcolorbox}
\begin{tcolorbox}
\textsubscript{4} Он будет судить бедных по правде, и дела страдальцев земли решать по истине; и жезлом уст Своих поразит землю, и духом уст Своих убьет нечестивого.
\end{tcolorbox}
\begin{tcolorbox}
\textsubscript{5} И будет препоясанием чресл Его правда, и препоясанием бедр Его--истина.
\end{tcolorbox}
\begin{tcolorbox}
\textsubscript{6} Тогда волк будет жить вместе с ягненком, и барс будет лежать вместе с козленком; и теленок, и молодой лев, и вол будут вместе, и малое дитя будет водить их.
\end{tcolorbox}
\begin{tcolorbox}
\textsubscript{7} И корова будет пастись с медведицею, и детеныши их будут лежать вместе, и лев, как вол, будет есть солому.
\end{tcolorbox}
\begin{tcolorbox}
\textsubscript{8} И младенец будет играть над норою аспида, и дитя протянет руку свою на гнездо змеи.
\end{tcolorbox}
\begin{tcolorbox}
\textsubscript{9} Не будут делать зла и вреда на всей святой горе Моей, ибо земля будет наполнена ведением Господа, как воды наполняют море.
\end{tcolorbox}
\begin{tcolorbox}
\textsubscript{10} И будет в тот день: к корню Иессееву, который станет, как знамя для народов, обратятся язычники, --и покой его будет слава.
\end{tcolorbox}
\begin{tcolorbox}
\textsubscript{11} И будет в тот день: Господь снова прострет руку Свою, чтобы возвратить Себе остаток народа Своего, какой останется у Ассура, и в Египте, и в Патросе, и у Хуса, и у Елама, и в Сеннааре, и в Емафе, и на островах моря.
\end{tcolorbox}
\begin{tcolorbox}
\textsubscript{12} И поднимет знамя язычникам, и соберет изгнанников Израиля, и рассеянных Иудеев созовет от четырех концов земли.
\end{tcolorbox}
\begin{tcolorbox}
\textsubscript{13} И прекратится зависть Ефрема, и враждующие против Иуды будут истреблены. Ефрем не будет завидовать Иуде, и Иуда не будет притеснять Ефрема.
\end{tcolorbox}
\begin{tcolorbox}
\textsubscript{14} И полетят на плеча Филистимлян к западу, ограбят всех детей Востока; на Едома и Моава наложат руку свою, и дети Аммона будут подданными им.
\end{tcolorbox}
\begin{tcolorbox}
\textsubscript{15} И иссушит Господь залив моря Египетского, и прострет руку Свою на реку в сильном ветре Своем, и разобьет ее на семь ручьев, так что в сандалиях могут переходить ее.
\end{tcolorbox}
\begin{tcolorbox}
\textsubscript{16} Тогда для остатка народа Его, который останется у Ассура, будет большая дорога, как это было для Израиля, когда он выходил из земли Египетской.
\end{tcolorbox}
\subsection{CHAPTER 12}
\begin{tcolorbox}
\textsubscript{1} И скажешь в тот день: славлю Тебя, Господи; Ты гневался на меня, но отвратил гнев Твой и утешил меня.
\end{tcolorbox}
\begin{tcolorbox}
\textsubscript{2} Вот, Бог--спасение мое: уповаю на Него и не боюсь; ибо Господь--сила моя, и пение мое--Господь; и Он был мне во спасение.
\end{tcolorbox}
\begin{tcolorbox}
\textsubscript{3} И в радости будете почерпать воду из источников спасения,
\end{tcolorbox}
\begin{tcolorbox}
\textsubscript{4} и скажете в тот день: славьте Господа, призывайте имя Его; возвещайте в народах дела Его; напоминайте, что велико имя Его;
\end{tcolorbox}
\begin{tcolorbox}
\textsubscript{5} пойте Господу, ибо Он соделал великое, --да знают это по всей земле.
\end{tcolorbox}
\begin{tcolorbox}
\textsubscript{6} Веселись и радуйся, жительница Сиона, ибо велик посреди тебя Святый Израилев.
\end{tcolorbox}
\subsection{CHAPTER 13}
\begin{tcolorbox}
\textsubscript{1} Пророчество о Вавилоне, которое изрек Исаия, сын Амосов.
\end{tcolorbox}
\begin{tcolorbox}
\textsubscript{2} Поднимите знамя на открытой горе, возвысьте голос; махните им рукою, чтобы шли в ворота властелинов.
\end{tcolorbox}
\begin{tcolorbox}
\textsubscript{3} Я дал повеление избранным Моим и призвал для [совершения] гнева Моего сильных Моих, торжествующих в величии Моем.
\end{tcolorbox}
\begin{tcolorbox}
\textsubscript{4} Большой шум на горах, как бы от многолюдного народа, мятежный шум царств и народов, собравшихся вместе: Господь Саваоф обозревает боевое войско.
\end{tcolorbox}
\begin{tcolorbox}
\textsubscript{5} Идут из отдаленной страны, от края неба, Господь и орудия гнева Его, чтобы сокрушить всю землю.
\end{tcolorbox}
\begin{tcolorbox}
\textsubscript{6} Рыдайте, ибо день Господа близок, идет как разрушительная сила от Всемогущего.
\end{tcolorbox}
\begin{tcolorbox}
\textsubscript{7} Оттого руки у всех опустились, и сердце у каждого человека растаяло.
\end{tcolorbox}
\begin{tcolorbox}
\textsubscript{8} Ужаснулись, судороги и боли схватили их; мучатся, как рождающая, с изумлением смотрят друг на друга, лица у них разгорелись.
\end{tcolorbox}
\begin{tcolorbox}
\textsubscript{9} Вот, приходит день Господа лютый, с гневом и пылающею яростью, чтобы сделать землю пустынею и истребить с нее грешников ее.
\end{tcolorbox}
\begin{tcolorbox}
\textsubscript{10} Звезды небесные и светила не дают от себя света; солнце меркнет при восходе своем, и луна не сияет светом своим.
\end{tcolorbox}
\begin{tcolorbox}
\textsubscript{11} Я накажу мир за зло, и нечестивых--за беззакония их, и положу конец высокоумию гордых, и уничижу надменность притеснителей;
\end{tcolorbox}
\begin{tcolorbox}
\textsubscript{12} сделаю то, что люди будут дороже чистого золота, и мужи--дороже золота Офирского.
\end{tcolorbox}
\begin{tcolorbox}
\textsubscript{13} Для сего потрясу небо, и земля сдвинется с места своего от ярости Господа Саваофа, в день пылающего гнева Его.
\end{tcolorbox}
\begin{tcolorbox}
\textsubscript{14} Тогда каждый, как преследуемая серна и как покинутые овцы, обратится к народу своему, и каждый побежит в свою землю.
\end{tcolorbox}
\begin{tcolorbox}
\textsubscript{15} Но кто попадется, будет пронзен, и кого схватят, тот падет от меча.
\end{tcolorbox}
\begin{tcolorbox}
\textsubscript{16} И младенцы их будут разбиты пред глазами их; домы их будут разграблены и жены их обесчещены.
\end{tcolorbox}
\begin{tcolorbox}
\textsubscript{17} Вот, Я подниму против них Мидян, которые не ценят серебра и не пристрастны к золоту.
\end{tcolorbox}
\begin{tcolorbox}
\textsubscript{18} Луки их сразят юношей и не пощадят плода чрева: глаз их не сжалится над детьми.
\end{tcolorbox}
\begin{tcolorbox}
\textsubscript{19} И Вавилон, краса царств, гордость Халдеев, будет ниспровержен Богом, как Содом и Гоморра,
\end{tcolorbox}
\begin{tcolorbox}
\textsubscript{20} не заселится никогда, и в роды родов не будет жителей в нем; не раскинет Аравитянин шатра своего, и пастухи со стадами не будут отдыхать там.
\end{tcolorbox}
\begin{tcolorbox}
\textsubscript{21} Но будут обитать в нем звери пустыни, и домы наполнятся филинами; и страусы поселятся, и косматые будут скакать там.
\end{tcolorbox}
\begin{tcolorbox}
\textsubscript{22} Шакалы будут выть в чертогах их, и гиены--в увеселительных домах.
\end{tcolorbox}
\subsection{CHAPTER 14}
\begin{tcolorbox}
\textsubscript{1} Близко время его, и не замедлят дни его, ибо помилует Господь Иакова и снова возлюбит Израиля; и поселит их на земле их, и присоединятся к ним иноземцы и прилепятся к дому Иакова.
\end{tcolorbox}
\begin{tcolorbox}
\textsubscript{2} И возьмут их народы, и приведут на место их, и дом Израиля усвоит их себе на земле Господней рабами и рабынями, и возьмет в плен пленивших его, и будет господствовать над угнетателями своими.
\end{tcolorbox}
\begin{tcolorbox}
\textsubscript{3} И будет в тот день: когда Господь устроит тебя от скорби твоей и от страха и от тяжкого рабства, которому ты порабощен был,
\end{tcolorbox}
\begin{tcolorbox}
\textsubscript{4} ты произнесешь победную песнь на царя Вавилонского и скажешь: как не стало мучителя, пресеклось грабительство!
\end{tcolorbox}
\begin{tcolorbox}
\textsubscript{5} Сокрушил Господь жезл нечестивых, скипетр владык,
\end{tcolorbox}
\begin{tcolorbox}
\textsubscript{6} поражавший народы в ярости ударами неотвратимыми, во гневе господствовавший над племенами с неудержимым преследованием.
\end{tcolorbox}
\begin{tcolorbox}
\textsubscript{7} Вся земля отдыхает, покоится, восклицает от радости;
\end{tcolorbox}
\begin{tcolorbox}
\textsubscript{8} и кипарисы радуются о тебе, и кедры ливанские, [говоря]: 'с тех пор, как ты заснул, никто не приходит рубить нас'.
\end{tcolorbox}
\begin{tcolorbox}
\textsubscript{9} Ад преисподний пришел в движение ради тебя, чтобы встретить тебя при входе твоем; пробудил для тебя Рефаимов, всех вождей земли; поднял всех царей языческих с престолов их.
\end{tcolorbox}
\begin{tcolorbox}
\textsubscript{10} Все они будут говорить тебе: и ты сделался бессильным, как мы! и ты стал подобен нам!
\end{tcolorbox}
\begin{tcolorbox}
\textsubscript{11} В преисподнюю низвержена гордыня твоя со всем шумом твоим; под тобою подстилается червь, и черви--покров твой.
\end{tcolorbox}
\begin{tcolorbox}
\textsubscript{12} Как упал ты с неба, денница, сын зари! разбился о землю, попиравший народы.
\end{tcolorbox}
\begin{tcolorbox}
\textsubscript{13} А говорил в сердце своем: 'взойду на небо, выше звезд Божиих вознесу престол мой и сяду на горе в сонме богов, на краю севера;
\end{tcolorbox}
\begin{tcolorbox}
\textsubscript{14} взойду на высоты облачные, буду подобен Всевышнему'.
\end{tcolorbox}
\begin{tcolorbox}
\textsubscript{15} Но ты низвержен в ад, в глубины преисподней.
\end{tcolorbox}
\begin{tcolorbox}
\textsubscript{16} Видящие тебя всматриваются в тебя, размышляют о тебе: 'тот ли это человек, который колебал землю, потрясал царства,
\end{tcolorbox}
\begin{tcolorbox}
\textsubscript{17} вселенную сделал пустынею и разрушал города ее, пленников своих не отпускал домой?'
\end{tcolorbox}
\begin{tcolorbox}
\textsubscript{18} Все цари народов, все лежат с честью, каждый в своей усыпальнице;
\end{tcolorbox}
\begin{tcolorbox}
\textsubscript{19} а ты повержен вне гробницы своей, как презренная ветвь, как одежда убитых, сраженных мечом, которых опускают в каменные рвы, --ты, как попираемый труп,
\end{tcolorbox}
\begin{tcolorbox}
\textsubscript{20} не соединишься с ними в могиле; ибо ты разорил землю твою, убил народ твой: во веки не помянется племя злодеев.
\end{tcolorbox}
\begin{tcolorbox}
\textsubscript{21} Готовьте заклание сыновьям его за беззаконие отца их, чтобы не восстали и не завладели землею и не наполнили вселенной неприятелями.
\end{tcolorbox}
\begin{tcolorbox}
\textsubscript{22} И восстану на них, говорит Господь Саваоф, и истреблю имя Вавилона и весь остаток, и сына и внука, говорит Господь.
\end{tcolorbox}
\begin{tcolorbox}
\textsubscript{23} И сделаю его владением ежей и болотом, и вымету его метлою истребительною, говорит Господь Саваоф.
\end{tcolorbox}
\begin{tcolorbox}
\textsubscript{24} С клятвою говорит Господь Саваоф: как Я помыслил, так и будет; как Я определил, так и состоится,
\end{tcolorbox}
\begin{tcolorbox}
\textsubscript{25} чтобы сокрушить Ассура в земле Моей и растоптать его на горах Моих; и спадет с них ярмо его, и снимется бремя его с рамен их.
\end{tcolorbox}
\begin{tcolorbox}
\textsubscript{26} Таково определение, постановленное о всей земле, и вот рука, простертая на все народы,
\end{tcolorbox}
\begin{tcolorbox}
\textsubscript{27} ибо Господь Саваоф определил, и кто может отменить это? рука Его простерта, --и кто отвратит ее?
\end{tcolorbox}
\begin{tcolorbox}
\textsubscript{28} В год смерти царя Ахаза было такое пророческое слово:
\end{tcolorbox}
\begin{tcolorbox}
\textsubscript{29} не радуйся, земля Филистимская, что сокрушен жезл, который поражал тебя, ибо из корня змеиного выйдет аспид, и плодом его будет летучий дракон.
\end{tcolorbox}
\begin{tcolorbox}
\textsubscript{30} Тогда беднейшие будут накормлены, и нищие будут покоиться в безопасности; а твой корень уморю голодом, и он убьет остаток твой.
\end{tcolorbox}
\begin{tcolorbox}
\textsubscript{31} Рыдайте, ворота! вой голосом, город! Распадешься ты, вся земля Филистимская, ибо от севера дым идет, и нет отсталого в полчищах их.
\end{tcolorbox}
\begin{tcolorbox}
\textsubscript{32} Что же скажут вестники народа? --То, что Господь утвердил Сион, и в нем найдут убежище бедные из народа Его.
\end{tcolorbox}
\subsection{CHAPTER 15}
\begin{tcolorbox}
\textsubscript{1} Пророчество о Моаве. --Так! ночью будет разорен Ар-Моав и уничтожен; так! ночью будет разорен Кир-Моав и уничтожен!
\end{tcolorbox}
\begin{tcolorbox}
\textsubscript{2} Он восходит к Баиту и Дивону, восходит на высоты, чтобы плакать; Моав рыдает над Нево и Медевою; у всех их острижены головы, у всех обриты бороды.
\end{tcolorbox}
\begin{tcolorbox}
\textsubscript{3} На улицах его препоясываются вретищем; на кровлях его и площадях его всё рыдает, утопает в слезах.
\end{tcolorbox}
\begin{tcolorbox}
\textsubscript{4} И вопит Есевон и Елеала; голос их слышится до самой Иаацы; за ними и воины Моава рыдают; душа его возмущена в нем.
\end{tcolorbox}
\begin{tcolorbox}
\textsubscript{5} Рыдает сердце мое о Моаве; бегут из него к Сигору, до третьей Эглы; восходят на Лухит с плачем; по дороге Хоронаимской поднимают страшный крик;
\end{tcolorbox}
\begin{tcolorbox}
\textsubscript{6} потому что воды Нимрима иссякли, луга засохли, трава выгорела, не стало зелени.
\end{tcolorbox}
\begin{tcolorbox}
\textsubscript{7} Поэтому они остатки стяжания и, что сбережено ими, переносят за реку Аравийскую.
\end{tcolorbox}
\begin{tcolorbox}
\textsubscript{8} Ибо вопль по всем пределам Моава, до Эглаима плач его и до Беэр-Елима плач его;
\end{tcolorbox}
\begin{tcolorbox}
\textsubscript{9} потому что воды Димона наполнились кровью, и Я наведу на Димон еще новое--львов на убежавших из Моава и на оставшихся в стране.
\end{tcolorbox}
\subsection{CHAPTER 16}
\begin{tcolorbox}
\textsubscript{1} Посылайте агнцев владетелю земли из Селы в пустыне на гору дочери Сиона;
\end{tcolorbox}
\begin{tcolorbox}
\textsubscript{2} ибо блуждающей птице, выброшенной из гнезда, будут подобны дочери Моава у бродов Арнонских.
\end{tcolorbox}
\begin{tcolorbox}
\textsubscript{3} 'Составь совет, постанови решение; осени нас среди полудня, как ночью, тенью твоею, укрой изгнанных, не выдай скитающихся.
\end{tcolorbox}
\begin{tcolorbox}
\textsubscript{4} Пусть поживут у тебя мои изгнанные Моавитяне; будь им покровом от грабителя: ибо притеснителя не станет, грабеж прекратится, попирающие исчезнут с земли.
\end{tcolorbox}
\begin{tcolorbox}
\textsubscript{5} И утвердится престол милостью, и воссядет на нем в истине, в шатре Давидовом, судия, ищущий правды и стремящийся к правосудию'.
\end{tcolorbox}
\begin{tcolorbox}
\textsubscript{6} 'Слыхали мы о гордости Моава, гордости чрезмерной, о надменности его и высокомерии и неистовстве его: неискренна речь его'.
\end{tcolorbox}
\begin{tcolorbox}
\textsubscript{7} Поэтому возрыдает Моав о Моаве, --все будут рыдать; стенайте о твердынях Кирхарешета: они совершенно разрушены.
\end{tcolorbox}
\begin{tcolorbox}
\textsubscript{8} Поля Есевонские оскудели, также и виноградник Севамский; властители народов истребили лучшие лозы его, которые достигали до Иазера, расстилались по пустыне; побеги их расширялись, переходили за море.
\end{tcolorbox}
\begin{tcolorbox}
\textsubscript{9} Посему [и] я буду плакать о лозе Севамской плачем Иазера, буду обливать тебя слезами моими, Есевон и Елеала; ибо во время собирания винограда твоего и во время жатвы твоей нет более шумной радости.
\end{tcolorbox}
\begin{tcolorbox}
\textsubscript{10} Исчезло с плодоносной земли веселье и ликование, и в виноградниках не поют, не ликуют; виноградарь не топчет винограда в точилах: Я прекратил ликование.
\end{tcolorbox}
\begin{tcolorbox}
\textsubscript{11} Оттого внутренность моя стонет о Моаве, как гусли, и сердце мое--о Кирхарешете.
\end{tcolorbox}
\begin{tcolorbox}
\textsubscript{12} Хотя и явится Моав, и будет до утомления [подвизаться] на высотах, и придет к святилищу своему помолиться, но ничто не поможет.
\end{tcolorbox}
\begin{tcolorbox}
\textsubscript{13} Вот слово, которое изрек Господь о Моаве издавна.
\end{tcolorbox}
\begin{tcolorbox}
\textsubscript{14} Ныне же так говорит Господь: чрез три года, считая годами наемничьими, величие Моава будет унижено со всем великим многолюдством, и остаток [будет] очень малый и незначительный.
\end{tcolorbox}
\subsection{CHAPTER 17}
\begin{tcolorbox}
\textsubscript{1} Пророчество о Дамаске. --Вот, Дамаск исключается из [числа] городов и будет грудою развалин.
\end{tcolorbox}
\begin{tcolorbox}
\textsubscript{2} Города Ароерские будут покинуты, --останутся для стад, которые будут отдыхать там, и некому будет пугать их.
\end{tcolorbox}
\begin{tcolorbox}
\textsubscript{3} Не станет твердыни Ефремовой и царства Дамасского с остальною Сириею; с ними будет то же, что со славою сынов Израиля, говорит Господь Саваоф.
\end{tcolorbox}
\begin{tcolorbox}
\textsubscript{4} И будет в тот день: умалится слава Иакова, и тучное тело его сделается тощим.
\end{tcolorbox}
\begin{tcolorbox}
\textsubscript{5} То же будет, что по собрании хлеба жнецом, когда рука его пожнет колосья, и когда соберут колосья в долине Рефаимской.
\end{tcolorbox}
\begin{tcolorbox}
\textsubscript{6} И останутся у него, как бывает при обивании маслин, две-три ягоды на самой вершине, или четыре-пять на плодоносных ветвях, говорит Господь, Бог Израилев.
\end{tcolorbox}
\begin{tcolorbox}
\textsubscript{7} В тот день обратит человек взор свой к Творцу своему, и глаза его будут устремлены к Святому Израилеву;
\end{tcolorbox}
\begin{tcolorbox}
\textsubscript{8} и не взглянет на жертвенники, на дело рук своих, и не посмотрит на то, что сделали персты его, на кумиры Астарты и Ваала.
\end{tcolorbox}
\begin{tcolorbox}
\textsubscript{9} В тот день укрепленные города его будут, как развалины в лесах и на вершинах гор, оставленные пред сынами Израиля, --и будет пусто.
\end{tcolorbox}
\begin{tcolorbox}
\textsubscript{10} Ибо ты забыл Бога спасения твоего, и не воспоминал о скале прибежища твоего; оттого развел увеселительные сады и насадил черенки от чужой лозы.
\end{tcolorbox}
\begin{tcolorbox}
\textsubscript{11} В день насаждения твоего ты заботился, чтобы оно росло и чтобы посеянное тобою рано расцвело; но в день собирания не куча жатвы будет, но скорбь жестокая.
\end{tcolorbox}
\begin{tcolorbox}
\textsubscript{12} Увы! шум народов многих! шумят они, как шумит море. Рев племен! они ревут, как ревут сильные воды.
\end{tcolorbox}
\begin{tcolorbox}
\textsubscript{13} Ревут народы, как ревут сильные воды; но Он погрозил им и они далеко побежали, и были гонимы, как прах по горам от ветра и как пыль от вихря.
\end{tcolorbox}
\begin{tcolorbox}
\textsubscript{14} Вечер--и вот ужас! и прежде утра уже нет его. Такова участь грабителей наших, жребий разорителей наших.
\end{tcolorbox}
\subsection{CHAPTER 18}
\begin{tcolorbox}
\textsubscript{1} Горе земле, осеняющей крыльями по ту сторону рек Ефиопских,
\end{tcolorbox}
\begin{tcolorbox}
\textsubscript{2} посылающей послов по морю, и в папировых суднах по водам! Идите, быстрые послы, к народу крепкому и бодрому, к народу страшному от начала и доныне, к народу рослому и [всё] попирающему, которого землю разрезывают реки.
\end{tcolorbox}
\begin{tcolorbox}
\textsubscript{3} Все вы, населяющие вселенную и живущие на земле! смотрите, когда знамя поднимется на горах, и, когда загремит труба, слушайте!
\end{tcolorbox}
\begin{tcolorbox}
\textsubscript{4} Ибо так Господь сказал мне: Я спокойно смотрю из жилища Моего, как светлая теплота после дождя, как облако росы во время жатвенного зноя.
\end{tcolorbox}
\begin{tcolorbox}
\textsubscript{5} Ибо прежде собирания винограда, когда он отцветет, и грозд начнет созревать, Он отрежет ножом ветви и отнимет, и отрубит отрасли.
\end{tcolorbox}
\begin{tcolorbox}
\textsubscript{6} И оставят всё хищным птицам на горах и зверям полевым; и птицы будут проводить там лето, а все звери полевые будут зимовать там.
\end{tcolorbox}
\begin{tcolorbox}
\textsubscript{7} В то время будет принесен дар Господу Саваофу от народа крепкого и бодрого, от народа страшного от начала и доныне, от народа рослого и [всё] попирающего, которого землю разрезывают реки, --к месту имени Господа Саваофа, на гору Сион.
\end{tcolorbox}
\subsection{CHAPTER 19}
\begin{tcolorbox}
\textsubscript{1} Пророчество о Египте. --Вот, Господь восседит на облаке легком и грядет в Египет. И потрясутся от лица Его идолы Египетские, и сердце Египта растает в нем.
\end{tcolorbox}
\begin{tcolorbox}
\textsubscript{2} Я вооружу Египтян против Египтян; и будут сражаться брат против брата и друг против друга, город с городом, царство с царством.
\end{tcolorbox}
\begin{tcolorbox}
\textsubscript{3} И дух Египта изнеможет в нем, и разрушу совет его, и прибегнут они к идолам и к чародеям, и к вызывающим мертвых и к гадателям.
\end{tcolorbox}
\begin{tcolorbox}
\textsubscript{4} И предам Египтян в руки властителя жестокого, и свирепый царь будет господствовать над ними, говорит Господь, Господь Саваоф.
\end{tcolorbox}
\begin{tcolorbox}
\textsubscript{5} И истощатся воды в море и река иссякнет и высохнет;
\end{tcolorbox}
\begin{tcolorbox}
\textsubscript{6} и оскудеют реки, и каналы Египетские обмелеют и высохнут; камыш и тростник завянут.
\end{tcolorbox}
\begin{tcolorbox}
\textsubscript{7} Поля при реке, по берегам реки, и все, посеянное при реке, засохнет, развеется и исчезнет.
\end{tcolorbox}
\begin{tcolorbox}
\textsubscript{8} И восплачут рыбаки, и возрыдают все, бросающие уду в реку, и ставящие сети в воде впадут в уныние;
\end{tcolorbox}
\begin{tcolorbox}
\textsubscript{9} и будут в смущении обрабатывающие лен и ткачи белых полотен;
\end{tcolorbox}
\begin{tcolorbox}
\textsubscript{10} и будут сокрушены сети, и все, которые содержат садки для живой рыбы, упадут в духе.
\end{tcolorbox}
\begin{tcolorbox}
\textsubscript{11} Так! обезумели князья Цоанские; совет мудрых советников фараоновых стал бессмысленным. Как скажете вы фараону: 'я сын мудрецов, сын царей древних?'
\end{tcolorbox}
\begin{tcolorbox}
\textsubscript{12} Где они? где твои мудрецы? пусть они теперь скажут тебе; пусть узнают, что Господь Саваоф определил о Египте.
\end{tcolorbox}
\begin{tcolorbox}
\textsubscript{13} Обезумели князья Цоанские; обманулись князья Мемфисские, и совратил Египет с пути главы племен его.
\end{tcolorbox}
\begin{tcolorbox}
\textsubscript{14} Господь послал в него дух опьянения; и они ввели Египет в заблуждение во всех делах его, подобно тому, как пьяный бродит по блевотине своей.
\end{tcolorbox}
\begin{tcolorbox}
\textsubscript{15} И не будет в Египте такого дела, которое совершить умели бы голова и хвост, пальма и трость.
\end{tcolorbox}
\begin{tcolorbox}
\textsubscript{16} В тот день Египтяне будут подобны женщинам, и вострепещут и убоятся движения руки Господа Саваофа, которую Он поднимет на них.
\end{tcolorbox}
\begin{tcolorbox}
\textsubscript{17} Земля Иудина сделается ужасом для Египта; кто вспомнит о ней, тот затрепещет от определения Господа Саваофа, которое Он постановил о нем.
\end{tcolorbox}
\begin{tcolorbox}
\textsubscript{18} В тот день пять городов в земле Египетской будут говорить языком Ханаанским и клясться Господом Саваофом; один назовется городом солнца.
\end{tcolorbox}
\begin{tcolorbox}
\textsubscript{19} В тот день жертвенник Господу будет посреди земли Египетской, и памятник Господу--у пределов ее.
\end{tcolorbox}
\begin{tcolorbox}
\textsubscript{20} И будет он знамением и свидетельством о Господе Саваофе в земле Египетской, потому что они воззовут к Господу по причине притеснителей, и Он пошлет им спасителя и заступника, и избавит их.
\end{tcolorbox}
\begin{tcolorbox}
\textsubscript{21} И Господь явит Себя в Египте; и Египтяне в тот день познают Господа и принесут жертвы и дары, и дадут обеты Господу, и исполнят.
\end{tcolorbox}
\begin{tcolorbox}
\textsubscript{22} И поразит Господь Египет; поразит и исцелит; они обратятся к Господу, и Он услышит их, и исцелит их.
\end{tcolorbox}
\begin{tcolorbox}
\textsubscript{23} В тот день из Египта в Ассирию будет большая дорога, и будет приходить Ассур в Египет, и Египтяне--в Ассирию; и Египтяне вместе с Ассириянами будут служить Господу.
\end{tcolorbox}
\begin{tcolorbox}
\textsubscript{24} В тот день Израиль будет третьим с Египтом и Ассириею; благословение будет посреди земли,
\end{tcolorbox}
\begin{tcolorbox}
\textsubscript{25} которую благословит Господь Саваоф, говоря: благословен народ Мой--Египтяне, и дело рук Моих--Ассирияне, и наследие Мое--Израиль.
\end{tcolorbox}
\subsection{CHAPTER 20}
\begin{tcolorbox}
\textsubscript{1} В год, когда Тартан пришел к Азоту, быв послан от Саргона, царя Ассирийского, и воевал против Азота, и взял его,
\end{tcolorbox}
\begin{tcolorbox}
\textsubscript{2} в то самое время Господь сказал Исаии, сыну Амосову, так: пойди и сними вретище с чресл твоих и сбрось сандалии твои с ног твоих. Он так и сделал: ходил нагой и босой.
\end{tcolorbox}
\begin{tcolorbox}
\textsubscript{3} И сказал Господь: как раб Мой Исаия ходил нагой и босой три года, в указание и предзнаменование о Египте и Ефиопии,
\end{tcolorbox}
\begin{tcolorbox}
\textsubscript{4} так поведет царь Ассирийский пленников из Египта и переселенцев из Ефиопии, молодых и старых, нагими и босыми и с обнаженными чреслами, в посрамление Египту.
\end{tcolorbox}
\begin{tcolorbox}
\textsubscript{5} Тогда ужаснутся и устыдятся из-за Ефиопии, надежды своей, и из-за Египта, которым хвалились.
\end{tcolorbox}
\begin{tcolorbox}
\textsubscript{6} И скажут в тот день жители этой страны: вот каковы те, на которых мы надеялись и к которым прибегали за помощью, чтобы спастись от царя Ассирийского! и как спаслись бы мы?
\end{tcolorbox}
\subsection{CHAPTER 21}
\begin{tcolorbox}
\textsubscript{1} Пророчество о пустыне приморской. --Как бури на юге носятся, идет он от пустыни, из земли страшной.
\end{tcolorbox}
\begin{tcolorbox}
\textsubscript{2} Грозное видение показано мне: грабитель грабит, опустошитель опустошает; восходи, Елам, осаждай, Мид! всем стенаниям я положу конец.
\end{tcolorbox}
\begin{tcolorbox}
\textsubscript{3} От этого чресла мои трясутся; муки схватили меня, как муки рождающей. Я взволнован от того, что слышу; я смущен от того, что вижу.
\end{tcolorbox}
\begin{tcolorbox}
\textsubscript{4} Сердце мое трепещет; дрожь бьет меня; отрадная ночь моя превратилась в ужас для меня.
\end{tcolorbox}
\begin{tcolorbox}
\textsubscript{5} Приготовляют стол, расстилают покрывала, --едят, пьют. 'Вставайте, князья, мажьте щиты!'
\end{tcolorbox}
\begin{tcolorbox}
\textsubscript{6} Ибо так сказал мне Господь: пойди, поставь сторожа; пусть он сказывает, что увидит.
\end{tcolorbox}
\begin{tcolorbox}
\textsubscript{7} И увидел он едущих попарно всадников на конях, всадников на ослах, всадников на верблюдах; и вслушивался он прилежно, с большим вниманием, --
\end{tcolorbox}
\begin{tcolorbox}
\textsubscript{8} и закричал, [как] лев: господин мой! на страже стоял я весь день, и на месте моем оставался целые ночи:
\end{tcolorbox}
\begin{tcolorbox}
\textsubscript{9} и вот, едут люди, всадники на конях попарно. Потом он возгласил и сказал: пал, пал Вавилон, и все идолы богов его лежат на земле разбитые.
\end{tcolorbox}
\begin{tcolorbox}
\textsubscript{10} О, измолоченный мой и сын гумна моего! Что слышал я от Господа Саваофа, Бога Израилева, то и возвестил вам.
\end{tcolorbox}
\begin{tcolorbox}
\textsubscript{11} Пророчество о Думе. --Кричат мне с Сеира: сторож! сколько ночи? сторож! сколько ночи?
\end{tcolorbox}
\begin{tcolorbox}
\textsubscript{12} Сторож отвечает: приближается утро, но еще ночь. Если вы настоятельно спрашиваете, то обратитесь и приходите.
\end{tcolorbox}
\begin{tcolorbox}
\textsubscript{13} Пророчество об Аравии. --В лесу Аравийском ночуйте, караваны Деданские!
\end{tcolorbox}
\begin{tcolorbox}
\textsubscript{14} Живущие в земле Фемайской! несите воды навстречу жаждущим; с хлебом встречайте бегущих,
\end{tcolorbox}
\begin{tcolorbox}
\textsubscript{15} ибо они от мечей бегут, от меча обнаженного и от лука натянутого, и от лютости войны.
\end{tcolorbox}
\begin{tcolorbox}
\textsubscript{16} Ибо так сказал мне Господь: еще год, равный году наемничьему, и вся слава Кидарова исчезнет,
\end{tcolorbox}
\begin{tcolorbox}
\textsubscript{17} и луков у храбрых сынов Кидара останется немного: так сказал Господь, Бог Израилев.
\end{tcolorbox}
\subsection{CHAPTER 22}
\begin{tcolorbox}
\textsubscript{1} Пророчество о долине видения. --Что с тобою, что ты весь взошел на кровли?
\end{tcolorbox}
\begin{tcolorbox}
\textsubscript{2} Город шумный, волнующийся, город ликующий! Пораженные твои не мечом убиты и не в битве умерли;
\end{tcolorbox}
\begin{tcolorbox}
\textsubscript{3} все вожди твои бежали вместе, но были связаны стрелками; все найденные у тебя связаны вместе, как ни далеко бежали.
\end{tcolorbox}
\begin{tcolorbox}
\textsubscript{4} Потому говорю: оставьте меня, я буду плакать горько; не усиливайтесь утешать меня в разорении дочери народа моего.
\end{tcolorbox}
\begin{tcolorbox}
\textsubscript{5} Ибо день смятения и попрания и замешательства в долине видения от Господа, Бога Саваофа. Ломают стену, и крик восходит на горы.
\end{tcolorbox}
\begin{tcolorbox}
\textsubscript{6} И Елам несет колчан; люди на колесницах [и] всадники, и Кир обнажает щит.
\end{tcolorbox}
\begin{tcolorbox}
\textsubscript{7} И вот, лучшие долины твои полны колесницами, и всадники выстроились против ворот,
\end{tcolorbox}
\begin{tcolorbox}
\textsubscript{8} и снимают покров с Иудеи; и ты в тот день обращаешь взор на запас оружия в доме кедровом.
\end{tcolorbox}
\begin{tcolorbox}
\textsubscript{9} Но вы видите, что много проломов в стене города Давидова, и собираете воды в нижнем пруде;
\end{tcolorbox}
\begin{tcolorbox}
\textsubscript{10} и отмечаете домы в Иерусалиме, и разрушаете домы, чтобы укрепить стену;
\end{tcolorbox}
\begin{tcolorbox}
\textsubscript{11} и устрояете между двумя стенами хранилище для вод старого пруда. А на Того, Кто это делает, не взираете, и не смотрите на Того, Кто издавна определил это.
\end{tcolorbox}
\begin{tcolorbox}
\textsubscript{12} И Господь, Господь Саваоф, призывает вас в этот день плакать и сетовать, и остричь волоса и препоясаться вретищем.
\end{tcolorbox}
\begin{tcolorbox}
\textsubscript{13} Но вот, веселье и радость! Убивают волов, и режут овец; едят мясо, и пьют вино: 'будем есть и пить, ибо завтра умрем!'
\end{tcolorbox}
\begin{tcolorbox}
\textsubscript{14} И открыл мне в уши Господь Саваоф: не будет прощено вам это нечестие, доколе не умрете, сказал Господь, Господь Саваоф.
\end{tcolorbox}
\begin{tcolorbox}
\textsubscript{15} Так сказал Господь, Господь Саваоф: ступай, пойди к этому царедворцу, к Севне, начальнику дворца [и скажи ему]:
\end{tcolorbox}
\begin{tcolorbox}
\textsubscript{16} что у тебя здесь, и кто здесь у тебя, что ты здесь высекаешь себе гробницу? --Он высекает себе гробницу на возвышенности, вырубает в скале жилище себе.
\end{tcolorbox}
\begin{tcolorbox}
\textsubscript{17} Вот, Господь перебросит тебя, как бросает сильный человек, и сожмет тебя в ком;
\end{tcolorbox}
\begin{tcolorbox}
\textsubscript{18} свернув тебя в сверток, бросит тебя, как меч, в землю обширную; там ты умрешь, и там великолепные колесницы твои будут поношением для дома господина твоего.
\end{tcolorbox}
\begin{tcolorbox}
\textsubscript{19} И столкну тебя с места твоего, и свергну тебя со степени твоей.
\end{tcolorbox}
\begin{tcolorbox}
\textsubscript{20} И будет в тот день, призову раба Моего Елиакима, сына Хелкиина,
\end{tcolorbox}
\begin{tcolorbox}
\textsubscript{21} и одену его в одежду твою, и поясом твоим опояшу его, и власть твою передам в руки его; и будет он отцом для жителей Иерусалима и для дома Иудина.
\end{tcolorbox}
\begin{tcolorbox}
\textsubscript{22} И ключ дома Давидова возложу на рамена его; отворит он, и никто не запрет; запрет он, и никто не отворит.
\end{tcolorbox}
\begin{tcolorbox}
\textsubscript{23} И укреплю его как гвоздь в твердом месте; и будет он как седалище славы для дома отца своего.
\end{tcolorbox}
\begin{tcolorbox}
\textsubscript{24} И будет висеть на нем вся слава дома отца его, детей и внуков, всей домашней утвари до последних музыкальных орудий.
\end{tcolorbox}
\begin{tcolorbox}
\textsubscript{25} В тот день, говорит Господь Саваоф, пошатнется гвоздь, укрепленный в твердом месте, и будет выбит, и упадет, и распадется вся тяжесть, которая на нем: ибо Господь говорит.
\end{tcolorbox}
\subsection{CHAPTER 23}
\begin{tcolorbox}
\textsubscript{1} Пророчество о Тире. --Рыдайте, корабли Фарсиса, ибо он разрушен; нет домов, и некому входить в домы. Так им возвещено из земли Киттийской.
\end{tcolorbox}
\begin{tcolorbox}
\textsubscript{2} Умолкните, обитатели острова, который наполняли купцы Сидонские, плавающие по морю.
\end{tcolorbox}
\begin{tcolorbox}
\textsubscript{3} По великим водам привозились в него семена Сихора, жатва [большой] реки, и был он торжищем народов.
\end{tcolorbox}
\begin{tcolorbox}
\textsubscript{4} Устыдись, Сидон; ибо [вот что] говорит море, крепость морская: 'как бы ни мучилась я родами и ни рождала, и ни воспитывала юношей, ни возращала девиц'.
\end{tcolorbox}
\begin{tcolorbox}
\textsubscript{5} Когда весть дойдет до Египтян, содрогнутся они, услышав о Тире.
\end{tcolorbox}
\begin{tcolorbox}
\textsubscript{6} Переселяйтесь в Фарсис, рыдайте, обитатели острова!
\end{tcolorbox}
\begin{tcolorbox}
\textsubscript{7} Это ли ваш ликующий город, которого начало от дней древних? Ноги его несут его скитаться в стране далекой.
\end{tcolorbox}
\begin{tcolorbox}
\textsubscript{8} Кто определил это Тиру, который раздавал венцы, которого купцы [были] князья, торговцы--знаменитости земли?
\end{tcolorbox}
\begin{tcolorbox}
\textsubscript{9} Господь Саваоф определил это, чтобы посрамить надменность всякой славы, чтобы унизить все знаменитости земли.
\end{tcolorbox}
\begin{tcolorbox}
\textsubscript{10} Ходи по земле твоей, дочь Фарсиса, как река: нет более препоны.
\end{tcolorbox}
\begin{tcolorbox}
\textsubscript{11} Он простер руку Свою на море, потряс царства; Господь дал повеление о Ханаане разрушить крепости его
\end{tcolorbox}
\begin{tcolorbox}
\textsubscript{12} и сказал: ты не будешь более ликовать, посрамленная девица, дочь Сидона! Вставай, иди в Киттим, [но] и там не будет тебе покоя.
\end{tcolorbox}
\begin{tcolorbox}
\textsubscript{13} Вот земля Халдеев. Этого народа прежде не было; Ассур положил ему начало из обитателей пустынь. Они ставят башни свои, разрушают чертоги его, превращают его в развалины.
\end{tcolorbox}
\begin{tcolorbox}
\textsubscript{14} Рыдайте, корабли Фарсисские! Ибо твердыня ваша разорена.
\end{tcolorbox}
\begin{tcolorbox}
\textsubscript{15} И будет в тот день, забудут Тир на семьдесят лет, в мере дней одного царя. По окончании же семидесяти лет с Тиром будет то же, что поют о блуднице:
\end{tcolorbox}
\begin{tcolorbox}
\textsubscript{16} 'возьми цитру, ходи по городу, забытая блудница! Играй складно, пой много песен, чтобы вспомнили о тебе'.
\end{tcolorbox}
\begin{tcolorbox}
\textsubscript{17} И будет, по истечении семидесяти лет, Господь посетит Тир; и он снова начнет получать прибыль свою и будет блудодействовать со всеми царствами земными по всей вселенной.
\end{tcolorbox}
\begin{tcolorbox}
\textsubscript{18} Но торговля его и прибыль его будут посвящаемы Господу; не будут заперты и уложены в кладовые, ибо к живущим пред лицем Господа будет переходить прибыль от торговли его, чтобы они ели до сытости и имели одежду прочную.
\end{tcolorbox}
\subsection{CHAPTER 24}
\begin{tcolorbox}
\textsubscript{1} Вот, Господь опустошает землю и делает ее бесплодною; изменяет вид ее и рассевает живущих на ней.
\end{tcolorbox}
\begin{tcolorbox}
\textsubscript{2} И что будет с народом, то и со священником; что со слугою, то и с господином его; что со служанкою, то и с госпожею ее; что с покупающим, то и с продающим; что с заемщиком, то и с заимодавцем; что с ростовщиком, то и с дающим в рост.
\end{tcolorbox}
\begin{tcolorbox}
\textsubscript{3} Земля опустошена вконец и совершенно разграблена, ибо Господь изрек слово сие.
\end{tcolorbox}
\begin{tcolorbox}
\textsubscript{4} Сетует, уныла земля; поникла, уныла вселенная; поникли возвышавшиеся над народом земли.
\end{tcolorbox}
\begin{tcolorbox}
\textsubscript{5} И земля осквернена под живущими на ней, ибо они преступили законы, изменили устав, нарушили вечный завет.
\end{tcolorbox}
\begin{tcolorbox}
\textsubscript{6} За то проклятие поедает землю, и несут наказание живущие на ней; за то сожжены обитатели земли, и немного осталось людей.
\end{tcolorbox}
\begin{tcolorbox}
\textsubscript{7} Плачет сок грозда; болит виноградная лоза; воздыхают все веселившиеся сердцем.
\end{tcolorbox}
\begin{tcolorbox}
\textsubscript{8} Прекратилось веселье с тимпанами; умолк шум веселящихся; затихли звуки гуслей;
\end{tcolorbox}
\begin{tcolorbox}
\textsubscript{9} уже не пьют вина с песнями; горька сикера для пьющих ее.
\end{tcolorbox}
\begin{tcolorbox}
\textsubscript{10} Разрушен опустевший город, все домы заперты, нельзя войти.
\end{tcolorbox}
\begin{tcolorbox}
\textsubscript{11} Плачут о вине на улицах; помрачилась всякая радость; изгнано всякое веселие земли.
\end{tcolorbox}
\begin{tcolorbox}
\textsubscript{12} В городе осталось запустение, и ворота развалились.
\end{tcolorbox}
\begin{tcolorbox}
\textsubscript{13} А посреди земли, между народами, будет то же, что бывает при обивании маслин, при обирании [винограда], когда кончена уборка.
\end{tcolorbox}
\begin{tcolorbox}
\textsubscript{14} Они возвысят голос свой, восторжествуют в величии Господа, громко будут восклицать с моря.
\end{tcolorbox}
\begin{tcolorbox}
\textsubscript{15} Итак славьте Господа на востоке, на островах морских--имя Господа, Бога Израилева.
\end{tcolorbox}
\begin{tcolorbox}
\textsubscript{16} От края земли мы слышим песнь: 'Слава Праведному!' И сказал я: беда мне, беда мне! увы мне! злодеи злодействуют, и злодействуют злодеи злодейски.
\end{tcolorbox}
\begin{tcolorbox}
\textsubscript{17} Ужас и яма и петля для тебя, житель земли!
\end{tcolorbox}
\begin{tcolorbox}
\textsubscript{18} Тогда побежавший от крика ужаса упадет в яму; и кто выйдет из ямы, попадет в петлю; ибо окна с [небесной] высоты растворятся, и основания земли потрясутся.
\end{tcolorbox}
\begin{tcolorbox}
\textsubscript{19} Земля сокрушается, земля распадается, земля сильно потрясена;
\end{tcolorbox}
\begin{tcolorbox}
\textsubscript{20} шатается земля, как пьяный, и качается, как колыбель, и беззаконие ее тяготеет на ней; она упадет, и уже не встанет.
\end{tcolorbox}
\begin{tcolorbox}
\textsubscript{21} И будет в тот день: посетит Господь воинство выспреннее на высоте и царей земных на земле.
\end{tcolorbox}
\begin{tcolorbox}
\textsubscript{22} И будут собраны вместе, как узники, в ров, и будут заключены в темницу, и после многих дней будут наказаны.
\end{tcolorbox}
\begin{tcolorbox}
\textsubscript{23} И покраснеет луна, и устыдится солнце, когда Господь Саваоф воцарится на горе Сионе и в Иерусалиме, и пред старейшинами его [будет] слава.
\end{tcolorbox}
\subsection{CHAPTER 25}
\begin{tcolorbox}
\textsubscript{1} Господи! Ты Бог мой; превознесу Тебя, восхвалю имя Твое, ибо Ты совершил дивное; предопределения древние истинны, аминь.
\end{tcolorbox}
\begin{tcolorbox}
\textsubscript{2} Ты превратил город в груду камней, твердую крепость в развалины; чертогов иноплеменников уже не стало в городе; вовек не будет он восстановлен.
\end{tcolorbox}
\begin{tcolorbox}
\textsubscript{3} Посему будут прославлять Тебя народы сильные; города страшных племен будут бояться Тебя,
\end{tcolorbox}
\begin{tcolorbox}
\textsubscript{4} ибо Ты был убежищем бедного, убежищем нищего в тесное для него время, защитою от бури, тенью от зноя; ибо гневное дыхание тиранов было подобно буре против стены.
\end{tcolorbox}
\begin{tcolorbox}
\textsubscript{5} Как зной в месте безводном, Ты укротил буйство врагов; [как] зной тенью облака, подавлено ликование притеснителей.
\end{tcolorbox}
\begin{tcolorbox}
\textsubscript{6} И сделает Господь Саваоф на горе сей для всех народов трапезу из тучных яств, трапезу из чистых вин, из тука костей и самых чистых вин;
\end{tcolorbox}
\begin{tcolorbox}
\textsubscript{7} и уничтожит на горе сей покрывало, покрывающее все народы, покрывало, лежащее на всех племенах.
\end{tcolorbox}
\begin{tcolorbox}
\textsubscript{8} Поглощена будет смерть навеки, и отрет Господь Бог слезы со всех лиц, и снимет поношение с народа Своего по всей земле; ибо так говорит Господь.
\end{tcolorbox}
\begin{tcolorbox}
\textsubscript{9} И скажут в тот день: вот Он, Бог наш! на Него мы уповали, и Он спас нас! Сей есть Господь; на Него уповали мы; возрадуемся и возвеселимся во спасении Его!
\end{tcolorbox}
\begin{tcolorbox}
\textsubscript{10} Ибо рука Господа почиет на горе сей, и Моав будет попран на месте своем, как попирается солома в навозе.
\end{tcolorbox}
\begin{tcolorbox}
\textsubscript{11} И хотя он распрострет посреди его руки свои, как плавающий распростирает их для плавания; [но Бог] унизит гордость его вместе с лукавством рук его.
\end{tcolorbox}
\begin{tcolorbox}
\textsubscript{12} И твердыню высоких стен твоих обрушит, низвергнет, повергнет на землю, в прах.
\end{tcolorbox}
\subsection{CHAPTER 26}
\begin{tcolorbox}
\textsubscript{1} В тот день будет воспета песнь сия в земле Иудиной: город крепкий у нас; спасение дал Он вместо стены и вала.
\end{tcolorbox}
\begin{tcolorbox}
\textsubscript{2} Отворите ворота; да войдет народ праведный, хранящий истину.
\end{tcolorbox}
\begin{tcolorbox}
\textsubscript{3} Твердого духом Ты хранишь в совершенном мире, ибо на Тебя уповает он.
\end{tcolorbox}
\begin{tcolorbox}
\textsubscript{4} Уповайте на Господа вовеки, ибо Господь Бог есть твердыня вечная:
\end{tcolorbox}
\begin{tcolorbox}
\textsubscript{5} Он ниспроверг живших на высоте, высоко стоявший город; поверг его, поверг на землю, бросил его в прах.
\end{tcolorbox}
\begin{tcolorbox}
\textsubscript{6} Нога попирает его, ноги бедного, стопы нищих.
\end{tcolorbox}
\begin{tcolorbox}
\textsubscript{7} Путь праведника прям; Ты уравниваешь стезю праведника.
\end{tcolorbox}
\begin{tcolorbox}
\textsubscript{8} И на пути судов Твоих, Господи, мы уповали на Тебя; к имени Твоему и к воспоминанию о Тебе стремилась душа наша.
\end{tcolorbox}
\begin{tcolorbox}
\textsubscript{9} Душею моею я стремился к Тебе ночью, и духом моим я буду искать Тебя во внутренности моей с раннего утра: ибо когда суды Твои [совершаются] на земле, тогда живущие в мире научаются правде.
\end{tcolorbox}
\begin{tcolorbox}
\textsubscript{10} Если нечестивый будет помилован, то не научится он правде, --будет злодействовать в земле правых и не будет взирать на величие Господа.
\end{tcolorbox}
\begin{tcolorbox}
\textsubscript{11} Господи! рука Твоя была высоко поднята, но они не видали ее; увидят и устыдятся ненавидящие народ Твой; огонь пожрет врагов Твоих.
\end{tcolorbox}
\begin{tcolorbox}
\textsubscript{12} Господи! Ты даруешь нам мир; ибо и все дела наши Ты устрояешь для нас.
\end{tcolorbox}
\begin{tcolorbox}
\textsubscript{13} Господи Боже наш! другие владыки кроме Тебя господствовали над нами; но чрез Тебя только мы славим имя Твое.
\end{tcolorbox}
\begin{tcolorbox}
\textsubscript{14} Мертвые не оживут; рефаимы не встанут, потому что Ты посетил и истребил их, и уничтожил всякую память о них.
\end{tcolorbox}
\begin{tcolorbox}
\textsubscript{15} Ты умножил народ, Господи, умножил народ, --прославил Себя, распространил все пределы земли.
\end{tcolorbox}
\begin{tcolorbox}
\textsubscript{16} Господи! в бедствии он искал Тебя; изливал тихие моления, когда наказание Твое постигало его.
\end{tcolorbox}
\begin{tcolorbox}
\textsubscript{17} Как беременная женщина, при наступлении родов, мучится, вопит от болей своих, так были мы пред Тобою, Господи.
\end{tcolorbox}
\begin{tcolorbox}
\textsubscript{18} Были беременны, мучились, --и рождали как бы ветер; спасения не доставили земле, и прочие жители вселенной не пали.
\end{tcolorbox}
\begin{tcolorbox}
\textsubscript{19} Оживут мертвецы Твои, восстанут мертвые тела! Воспряните и торжествуйте, поверженные в прахе: ибо роса Твоя--роса растений, и земля извергнет мертвецов.
\end{tcolorbox}
\begin{tcolorbox}
\textsubscript{20} Пойди, народ мой, войди в покои твои и запри за собой двери твои, укройся на мгновение, доколе не пройдет гнев;
\end{tcolorbox}
\begin{tcolorbox}
\textsubscript{21} ибо вот, Господь выходит из жилища Своего наказать обитателей земли за их беззаконие, и земля откроет поглощенную ею кровь и уже не скроет убитых своих.
\end{tcolorbox}
\subsection{CHAPTER 27}
\begin{tcolorbox}
\textsubscript{1} В тот день поразит Господь мечом Своим тяжелым, и большим и крепким, левиафана, змея прямо бегущего, и левиафана, змея изгибающегося, и убьет чудовище морское.
\end{tcolorbox}
\begin{tcolorbox}
\textsubscript{2} В тот день воспойте о нем--о возлюбленном винограднике:
\end{tcolorbox}
\begin{tcolorbox}
\textsubscript{3} Я, Господь, хранитель его, в каждое мгновение напояю его; ночью и днем стерегу его, чтобы кто не ворвался в него.
\end{tcolorbox}
\begin{tcolorbox}
\textsubscript{4} Гнева нет во Мне. Но если бы кто противопоставил Мне [в нем] волчцы и терны, Я войною пойду против него, выжгу его совсем.
\end{tcolorbox}
\begin{tcolorbox}
\textsubscript{5} Разве прибегнет к защите Моей и заключит мир со Мною? тогда пусть заключит мир со Мною.
\end{tcolorbox}
\begin{tcolorbox}
\textsubscript{6} В грядущие [дни] укоренится Иаков, даст отпрыск и расцветет Израиль; и наполнится плодами вселенная.
\end{tcolorbox}
\begin{tcolorbox}
\textsubscript{7} Так ли Он поражал его, как поражал поражавших его? Так ли убивал его, как убиты убивавшие его?
\end{tcolorbox}
\begin{tcolorbox}
\textsubscript{8} Мерою Ты наказывал его, когда отвергал его; выбросил его сильным дуновением Своим как бы в день восточного ветра.
\end{tcolorbox}
\begin{tcolorbox}
\textsubscript{9} И чрез это загладится беззаконие Иакова; и плодом сего будет снятие греха с него, когда все камни жертвенников он обратит в куски извести, и не будут уже стоять дубравы и истуканы солнца.
\end{tcolorbox}
\begin{tcolorbox}
\textsubscript{10} Ибо укрепленный город опустеет, жилища [будут] покинуты и заброшены, как пустыня. Там будет пастись теленок, и там он будет покоиться и объедать ветви его.
\end{tcolorbox}
\begin{tcolorbox}
\textsubscript{11} Когда ветви его засохнут, их обломают; женщины придут и сожгут их. Так как это народ безрассудный, то не сжалится над ним Творец его, и не помилует его Создатель его.
\end{tcolorbox}
\begin{tcolorbox}
\textsubscript{12} Но будет в тот день: Господь потрясет всё от великой реки до потока Египетского, и вы, сыны Израиля, будете собраны один к другому;
\end{tcolorbox}
\begin{tcolorbox}
\textsubscript{13} и будет в тот день: вострубит великая труба, и придут затерявшиеся в Ассирийской земле и изгнанные в землю Египетскую и поклонятся Господу на горе святой в Иерусалиме.
\end{tcolorbox}
\subsection{CHAPTER 28}
\begin{tcolorbox}
\textsubscript{1} Горе венку гордости пьяных Ефремлян, увядшему цветкукрасивого убранства его, который на вершине тучной долинысраженных вином!
\end{tcolorbox}
\begin{tcolorbox}
\textsubscript{2} Вот, крепкий и сильный у Господа, как ливень с градом игубительный вихрь, как разлившееся наводнение бурных вод, с силоюповергает его на землю.
\end{tcolorbox}
\begin{tcolorbox}
\textsubscript{3} Ногами попирается венок гордости пьяных Ефремлян.
\end{tcolorbox}
\begin{tcolorbox}
\textsubscript{4} И с увядшим цветком красивого убранства его, который навершине тучной долины, делается то же, что бывает с созревшеюпрежде времени смоквою, которую, как скоро кто увидит, тотчасберет в руку и проглатывает ее.
\end{tcolorbox}
\begin{tcolorbox}
\textsubscript{5} В тот день Господь Саваоф будет великолепным венцом иславною диадемою для остатка народа Своего,
\end{tcolorbox}
\begin{tcolorbox}
\textsubscript{6} и духом правосудия для сидящего в судилище и мужеством дляотражающих неприятеля до ворот.
\end{tcolorbox}
\begin{tcolorbox}
\textsubscript{7} Но и эти шатаются от вина и сбиваются с пути от сикеры; священник и пророк спотыкаются от крепких напитков; побежденывином, обезумели от сикеры, в видении ошибаются, в сужденииспотыкаются.
\end{tcolorbox}
\begin{tcolorbox}
\textsubscript{8} Ибо все столы наполнены отвратительною блевотиною, нет [чистого] места.
\end{tcolorbox}
\begin{tcolorbox}
\textsubscript{9} А [говорят]: 'кого хочет он учить ведению? и кого вразумлятьпроповедью? отнятых от грудного молока, отлученных от сосцов [матери]?
\end{tcolorbox}
\begin{tcolorbox}
\textsubscript{10} Ибо всё заповедь на заповедь, заповедь на заповедь, правилона правило, правило на правило, тут немного и там немного'.
\end{tcolorbox}
\begin{tcolorbox}
\textsubscript{11} За то лепечущими устами и на чужом языке будут говорить кэтому народу.
\end{tcolorbox}
\begin{tcolorbox}
\textsubscript{12} Им говорили: 'вот г покой, дайте покой утружденному, и вот д успокоение'. Но они не хотели слушать.
\end{tcolorbox}
\begin{tcolorbox}
\textsubscript{13} И стало у них словом Господа: заповедь на заповедь, заповедьна заповедь, правило на правило, правило на правило, тутнемного, там немного, г так что они пойдут, и упадут навзничь, иразобьются, и попадут в сеть и будут уловлены.
\end{tcolorbox}
\begin{tcolorbox}
\textsubscript{14} Итак слушайте слово Господне, хульники, правители народасего, который в Иерусалиме.
\end{tcolorbox}
\begin{tcolorbox}
\textsubscript{15} Так как вы говорите: 'мы заключили союз со смертью и спреисподнею сделали договор: когда всепоражающий бич будетпроходить, он не дойдет до нас, г потому что ложь сделали мыубежищем для себя, и обманом прикроем себя'.
\end{tcolorbox}
\begin{tcolorbox}
\textsubscript{16} Посему так говорит Господь Бог: вот, Я полагаю в основаниена Сионе камень, г камень испытанный, краеугольный, драгоценный, крепко утвержденный: верующий в него не постыдится.
\end{tcolorbox}
\begin{tcolorbox}
\textsubscript{17} И поставлю суд мерилом и правду весами; и градом истребитсяубежище лжи, и воды потопят место укрывательства.
\end{tcolorbox}
\begin{tcolorbox}
\textsubscript{18} И союз ваш со смертью рушится, и договор ваш с преисподнеюне устоит. Когда пойдет всепоражающий бич, вы будете попраны.
\end{tcolorbox}
\begin{tcolorbox}
\textsubscript{19} Как скоро он пойдет, схватит вас; ходить же будет каждоеутро, день и ночь, и один слух о нем будет внушать ужас.
\end{tcolorbox}
\begin{tcolorbox}
\textsubscript{20} Слишком коротка будет постель, чтобы протянуться; слишкомузко и одеяло, чтобы завернуться в него.
\end{tcolorbox}
\begin{tcolorbox}
\textsubscript{21} Ибо восстанет Господь, как на горе Перациме; разгневается, как на долине Гаваонской, чтобы сделать дело Свое, необычайноедело, и совершить действие Свое, чудное Свое действие.
\end{tcolorbox}
\begin{tcolorbox}
\textsubscript{22} Итак не кощунствуйте, чтобы узы ваши не стали крепче; ибо яслышал от Господа, Бога Саваофа, что истребление определено длявсей земли.
\end{tcolorbox}
\begin{tcolorbox}
\textsubscript{23} Приклоните ухо, и послушайте моего голоса; будьте внимательны, и выслушайте речь мою.
\end{tcolorbox}
\begin{tcolorbox}
\textsubscript{24} Всегда ли земледелец пашет для посева, бороздит и боронитземлю свою?
\end{tcolorbox}
\begin{tcolorbox}
\textsubscript{25} Нет; когда уровняет поверхность ее, он сеет чернуху, илирассыпает тмин, или разбрасывает пшеницу рядами, и ячмень вопределенном месте, и полбу рядом с ним.
\end{tcolorbox}
\begin{tcolorbox}
\textsubscript{26} И такому порядку учит его Бог его; Он наставляет его.
\end{tcolorbox}
\begin{tcolorbox}
\textsubscript{27} Ибо не молотят чернухи катком зубчатым, и колес молотильныхне катают по тмину; но палкою выколачивают чернуху, и тмин гпалкою.
\end{tcolorbox}
\begin{tcolorbox}
\textsubscript{28} Зерновой хлеб вымолачивают, но не разбивают его; и водят понему молотильные колеса с конями их, но не растирают его.
\end{tcolorbox}
\begin{tcolorbox}
\textsubscript{29} И это происходит от Господа Саваофа: дивны судьбы Его, велика премудрость Его!
\end{tcolorbox}
\subsection{CHAPTER 29}
\begin{tcolorbox}
\textsubscript{1} Горе Ариилу, Ариилу, городу, в котором жил Давид! приложите год к году; пусть заколают жертвы.
\end{tcolorbox}
\begin{tcolorbox}
\textsubscript{2} Но Я стесню Ариил, и будет плач и сетование; и он останется у Меня, как Ариил.
\end{tcolorbox}
\begin{tcolorbox}
\textsubscript{3} Я расположусь станом вокруг тебя и стесню тебя стражею наблюдательною, и воздвигну против тебя укрепления.
\end{tcolorbox}
\begin{tcolorbox}
\textsubscript{4} И будешь унижен, с земли будешь говорить, и глуха будет речь твоя из-под праха, и голос твой будет, как голос чревовещателя, и из-под праха шептать будет речь твоя.
\end{tcolorbox}
\begin{tcolorbox}
\textsubscript{5} Множество врагов твоих будет, как мелкая пыль, и полчище лютых, как разлетающаяся плева; и это совершится внезапно, в одно мгновение.
\end{tcolorbox}
\begin{tcolorbox}
\textsubscript{6} Господь Саваоф посетит тебя громом и землетрясением, и сильным гласом, бурею и вихрем, и пламенем всепожирающего огня.
\end{tcolorbox}
\begin{tcolorbox}
\textsubscript{7} И как сон, как ночное сновидение, будет множество всех народов, воюющих против Ариила, и всех выступивших против него и укреплений его и стеснивших его.
\end{tcolorbox}
\begin{tcolorbox}
\textsubscript{8} И как голодному снится, будто он ест, но пробуждается, и душа его тоща; и как жаждущему снится, будто он пьет, но пробуждается, и вот он томится, и душа его жаждет: то же будет и множеству всех народов, воюющих против горы Сиона.
\end{tcolorbox}
\begin{tcolorbox}
\textsubscript{9} Изумляйтесь и дивитесь: они ослепили других, и сами ослепли; они пьяны, но не от вина, --шатаются, но не от сикеры;
\end{tcolorbox}
\begin{tcolorbox}
\textsubscript{10} ибо навел на вас Господь дух усыпления и сомкнул глаза ваши, пророки, и закрыл ваши головы, прозорливцы.
\end{tcolorbox}
\begin{tcolorbox}
\textsubscript{11} И всякое пророчество для вас то же, что слова в запечатанной книге, которую подают умеющему читать книгу и говорят: 'прочитай ее'; и тот отвечает: 'не могу, потому что она запечатана'.
\end{tcolorbox}
\begin{tcolorbox}
\textsubscript{12} И передают книгу тому, кто читать не умеет, и говорят: 'прочитай ее'; и тот отвечает: 'я не умею читать'.
\end{tcolorbox}
\begin{tcolorbox}
\textsubscript{13} И сказал Господь: так как этот народ приближается ко Мне устами своими, и языком своим чтит Меня, сердце же его далеко отстоит от Меня, и благоговение их предо Мною есть изучение заповедей человеческих;
\end{tcolorbox}
\begin{tcolorbox}
\textsubscript{14} то вот, Я еще необычайно поступлю с этим народом, чудно и дивно, так что мудрость мудрецов его погибнет, и разума у разумных его не станет.
\end{tcolorbox}
\begin{tcolorbox}
\textsubscript{15} Горе тем, которые думают скрыться в глубину, чтобы замысл свой утаить от Господа, которые делают дела свои во мраке и говорят: 'кто увидит нас? и кто узнает нас?'
\end{tcolorbox}
\begin{tcolorbox}
\textsubscript{16} Какое безрассудство! Разве можно считать горшечника, как глину? Скажет ли изделие о сделавшем его: 'не он сделал меня'? и скажет ли произведение о художнике своем: 'он не разумеет'?
\end{tcolorbox}
\begin{tcolorbox}
\textsubscript{17} Еще немного, очень немного, и Ливан не превратится ли в сад, а сад не будут ли почитать, как лес?
\end{tcolorbox}
\begin{tcolorbox}
\textsubscript{18} И в тот день глухие услышат слова книги, и прозрят из тьмы и мрака глаза слепых.
\end{tcolorbox}
\begin{tcolorbox}
\textsubscript{19} И страждущие более и более будут радоваться о Господе, и бедные люди будут торжествовать о Святом Израиля,
\end{tcolorbox}
\begin{tcolorbox}
\textsubscript{20} потому что не будет более обидчика, и хульник исчезнет, и будут истреблены все поборники неправды,
\end{tcolorbox}
\begin{tcolorbox}
\textsubscript{21} которые запутывают человека в словах, и требующему суда у ворот расставляют сети, и отталкивают правого.
\end{tcolorbox}
\begin{tcolorbox}
\textsubscript{22} Посему так говорит о доме Иакова Господь, Который искупил Авраама: тогда Иаков не будет в стыде, и лице его более не побледнеет.
\end{tcolorbox}
\begin{tcolorbox}
\textsubscript{23} Ибо когда увидит у себя детей своих, дело рук Моих, то они свято будут чтить имя Мое и свято чтить Святаго Иаковлева, и благоговеть пред Богом Израилевым.
\end{tcolorbox}
\begin{tcolorbox}
\textsubscript{24} Тогда блуждающие духом познают мудрость, и непокорные научатся послушанию.
\end{tcolorbox}
\subsection{CHAPTER 30}
\begin{tcolorbox}
\textsubscript{1} Горе непокорным сынам, говорит Господь, которые делают совещания, но без Меня, и заключают союзы, но не по духу Моему, чтобы прилагать грех ко греху:
\end{tcolorbox}
\begin{tcolorbox}
\textsubscript{2} не вопросив уст Моих, идут в Египет, чтобы подкрепить себя силою фараона и укрыться под тенью Египта.
\end{tcolorbox}
\begin{tcolorbox}
\textsubscript{3} Но сила фараона будет для вас стыдом, и убежище под тенью Египта--бесчестием;
\end{tcolorbox}
\begin{tcolorbox}
\textsubscript{4} потому что князья его уже в Цоане, и послы его дошли до Ханеса.
\end{tcolorbox}
\begin{tcolorbox}
\textsubscript{5} Все они будут постыжены из-за народа, [который] бесполезен для них; не будет от него ни помощи, ни пользы, но--стыд и срам.
\end{tcolorbox}
\begin{tcolorbox}
\textsubscript{6} Тяжести на животных, [идущих] на юг, по земле угнетения и тесноты, откуда [выходят] львицы и львы, аспиды и летучие змеи; они несут на хребтах ослов богатства свои и на горбах верблюдов сокровища свои к народу, который не принесет им пользы.
\end{tcolorbox}
\begin{tcolorbox}
\textsubscript{7} Ибо помощь Египта будет тщетна и напрасна; потому Я сказал им: сила их--сидеть спокойно.
\end{tcolorbox}
\begin{tcolorbox}
\textsubscript{8} Теперь пойди, начертай это на доске у них, и впиши это в книгу, чтобы осталось на будущее время, навсегда, навеки.
\end{tcolorbox}
\begin{tcolorbox}
\textsubscript{9} Ибо это народ мятежный, дети лживые, дети, которые не хотят слушать закона Господня,
\end{tcolorbox}
\begin{tcolorbox}
\textsubscript{10} которые провидящим говорят: 'перестаньте провидеть', и пророкам: 'не пророчествуйте нам правды, говорите нам лестное, предсказывайте приятное;
\end{tcolorbox}
\begin{tcolorbox}
\textsubscript{11} сойдите с дороги, уклонитесь от пути; устраните от глаз наших Святаго Израилева.'
\end{tcolorbox}
\begin{tcolorbox}
\textsubscript{12} Посему так говорит Святый Израилев: так как вы отвергаете слово сие, а надеетесь на обман и неправду, и опираетесь на то:
\end{tcolorbox}
\begin{tcolorbox}
\textsubscript{13} то беззаконие это будет для вас, как угрожающая падением трещина, обнаружившаяся в высокой стене, которой разрушение настанет внезапно, в одно мгновение.
\end{tcolorbox}
\begin{tcolorbox}
\textsubscript{14} И Он разрушит ее, как сокрушают глиняный сосуд, разбивая его без пощады, так что в обломках его не найдется и черепка, чтобы взять огня с очага или зачерпнуть воды из водоема;
\end{tcolorbox}
\begin{tcolorbox}
\textsubscript{15} ибо так говорит Господь Бог, Святый Израилев: оставаясь на месте и в покое, вы спаслись бы; в тишине и уповании крепость ваша; но вы не хотели
\end{tcolorbox}
\begin{tcolorbox}
\textsubscript{16} и говорили: 'нет, мы на конях убежим', --за то и побежите; 'мы на быстрых ускачем', --за то и преследующие вас будут быстры.
\end{tcolorbox}
\begin{tcolorbox}
\textsubscript{17} От угрозы одного [побежит] тысяча, от угрозы пятерых побежите так, что остаток ваш будет как веха на вершине горы и как знамя на холме.
\end{tcolorbox}
\begin{tcolorbox}
\textsubscript{18} И потому Господь медлит, чтобы помиловать вас, и потому еще удерживается, чтобы сжалиться над вами; ибо Господь есть Бог правды: блаженны все уповающие на Него!
\end{tcolorbox}
\begin{tcolorbox}
\textsubscript{19} Народ будет жить на Сионе в Иерусалиме; ты не будешь много плакать, --Он помилует тебя, по голосу вопля твоего, и как только услышит его, ответит тебе.
\end{tcolorbox}
\begin{tcolorbox}
\textsubscript{20} И даст вам Господь хлеб в горести и воду в нужде; и учители твои уже не будут скрываться, и глаза твои будут видеть учителей твоих;
\end{tcolorbox}
\begin{tcolorbox}
\textsubscript{21} и уши твои будут слышать слово, говорящее позади тебя: 'вот путь, идите по нему', если бы вы уклонились направо и если бы вы уклонились налево.
\end{tcolorbox}
\begin{tcolorbox}
\textsubscript{22} Тогда вы будете считать скверною оклад идолов из серебра твоего и оклад истуканов из золота твоего; ты бросишь их, как нечистоту; ты скажешь им: прочь отсюда.
\end{tcolorbox}
\begin{tcolorbox}
\textsubscript{23} И Он даст дождь на семя твое, которым засеешь поле, и хлеб, плод земли, и он будет обилен и сочен; стада твои в тот день будут пастись на обширных пастбищах.
\end{tcolorbox}
\begin{tcolorbox}
\textsubscript{24} И волы и ослы, возделывающие поле, будут есть корм соленый, очищенный лопатою и веялом.
\end{tcolorbox}
\begin{tcolorbox}
\textsubscript{25} И на всякой горе высокой и на всяком холме возвышенном потекут ручьи, потоки вод, в день великого поражения, когда упадут башни.
\end{tcolorbox}
\begin{tcolorbox}
\textsubscript{26} И свет луны будет, как свет солнца, а свет солнца будет светлее всемеро, как свет семи дней, в тот день, когда Господь обвяжет рану народа Своего и исцелит нанесенные ему язвы.
\end{tcolorbox}
\begin{tcolorbox}
\textsubscript{27} Вот, имя Господа идет издали, горит гнев Его, и пламя его сильно, уста Его исполнены негодования, и язык Его, как огонь поедающий,
\end{tcolorbox}
\begin{tcolorbox}
\textsubscript{28} и дыхание Его, как разлившийся поток, который поднимается даже до шеи, чтобы развеять народы до истощания; и будет в челюстях народов узда, направляющая к заблуждению.
\end{tcolorbox}
\begin{tcolorbox}
\textsubscript{29} А у вас будут песни, как в ночь священного праздника, и веселие сердца, как у идущего со свирелью на гору Господню, к твердыне Израилевой.
\end{tcolorbox}
\begin{tcolorbox}
\textsubscript{30} И возгремит Господь величественным гласом Своим и явит тяготеющую мышцу Свою в сильном гневе и в пламени поедающего огня, в буре и в наводнении и в каменном граде.
\end{tcolorbox}
\begin{tcolorbox}
\textsubscript{31} Ибо от гласа Господа содрогнется Ассур, жезлом поражаемый.
\end{tcolorbox}
\begin{tcolorbox}
\textsubscript{32} И всякое движение определенного ему жезла, который Господь направит на него, будет с тимпанами и цитрами, и Он пойдет против него войною опустошительною.
\end{tcolorbox}
\begin{tcolorbox}
\textsubscript{33} Ибо Тофет давно уже устроен; он приготовлен и для царя, глубок и широк; в костре его много огня и дров; дуновение Господа, как поток серы, зажжет его.
\end{tcolorbox}
\subsection{CHAPTER 31}
\begin{tcolorbox}
\textsubscript{1} Горе тем, которые идут в Египет за помощью, надеются на коней и полагаются на колесницы, потому что их много, и на всадников, потому что они весьма сильны, а на Святаго Израилева не взирают и к Господу не прибегают!
\end{tcolorbox}
\begin{tcolorbox}
\textsubscript{2} Но премудр Он; и наведет бедствие, и не отменит слов Своих; восстанет против дома нечестивых и против помощи делающих беззаконие.
\end{tcolorbox}
\begin{tcolorbox}
\textsubscript{3} И Египтяне--люди, а не Бог; и кони их--плоть, а не дух. И прострет руку Свою Господь, и споткнется защитник, и упадет защищаемый, и все вместе погибнут.
\end{tcolorbox}
\begin{tcolorbox}
\textsubscript{4} Ибо так сказал мне Господь: как лев, как скимен, ревущий над своею добычею, хотя бы множество пастухов кричало на него, от крика их не содрогнется и множеству их не уступит, --так Господь Саваоф сойдет сразиться за гору Сион и за холм его.
\end{tcolorbox}
\begin{tcolorbox}
\textsubscript{5} Как птицы--птенцов, так Господь Саваоф покроет Иерусалим, защитит и избавит, пощадит и спасет.
\end{tcolorbox}
\begin{tcolorbox}
\textsubscript{6} Обратитесь к Тому, от Которого вы столько отпали, сыны Израиля!
\end{tcolorbox}
\begin{tcolorbox}
\textsubscript{7} В тот день отбросит каждый человек своих серебряных идолов и золотых своих идолов, которых руки ваши сделали вам на грех.
\end{tcolorbox}
\begin{tcolorbox}
\textsubscript{8} И Ассур падет не от человеческого меча, и не человеческий меч потребит его, --он избежит от меча, и юноши его будут податью.
\end{tcolorbox}
\begin{tcolorbox}
\textsubscript{9} И от страха пробежит мимо крепости своей; и князья его будут пугаться знамени, говорит Господь, Которого огонь на Сионе и горнило в Иерусалиме.
\end{tcolorbox}
\subsection{CHAPTER 32}
\begin{tcolorbox}
\textsubscript{1} Вот, Царь будет царствовать по правде, и князья будут править по закону;
\end{tcolorbox}
\begin{tcolorbox}
\textsubscript{2} и каждый из них будет как защита от ветра и покров от непогоды, как источники вод в степи, как тень от высокой скалы в земле жаждущей.
\end{tcolorbox}
\begin{tcolorbox}
\textsubscript{3} И очи видящих не будут закрываемы, и уши слышащих будут внимать.
\end{tcolorbox}
\begin{tcolorbox}
\textsubscript{4} И сердце легкомысленных будет уметь рассуждать; и косноязычные будут говорить ясно.
\end{tcolorbox}
\begin{tcolorbox}
\textsubscript{5} Невежду уже не будут называть почтенным, и о коварном не скажут, что он честный.
\end{tcolorbox}
\begin{tcolorbox}
\textsubscript{6} Ибо невежда говорит глупое, и сердце его помышляет о беззаконном, чтобы действовать лицемерно и произносить хулу на Господа, душу голодного лишать хлеба и отнимать питье у жаждущего.
\end{tcolorbox}
\begin{tcolorbox}
\textsubscript{7} У коварного и действования гибельные: он замышляет ковы, чтобы погубить бедного словами лжи, хотя бы бедный был и прав.
\end{tcolorbox}
\begin{tcolorbox}
\textsubscript{8} А честный и мыслит о честном и твердо стоит во всем, что честно.
\end{tcolorbox}
\begin{tcolorbox}
\textsubscript{9} Женщины беспечные! встаньте, послушайте голоса моего; дочери беззаботные! приклоните слух к моим словам.
\end{tcolorbox}
\begin{tcolorbox}
\textsubscript{10} Еще несколько дней сверх года, и ужаснетесь, беспечные! ибо не будет обирания винограда, и время жатвы не настанет.
\end{tcolorbox}
\begin{tcolorbox}
\textsubscript{11} Содрогнитесь, беззаботные! ужаснитесь, беспечные! сбросьте одежды, обнажитесь и препояшьте чресла.
\end{tcolorbox}
\begin{tcolorbox}
\textsubscript{12} Будут бить себя в грудь о прекрасных полях, о виноградной лозе плодовитой.
\end{tcolorbox}
\begin{tcolorbox}
\textsubscript{13} На земле народа моего будут расти терны и волчцы, равно и на всех домах веселья в ликующем городе;
\end{tcolorbox}
\begin{tcolorbox}
\textsubscript{14} ибо чертоги будут оставлены; шумный город будет покинут; Офел и башня навсегда будут служить, вместо пещер, убежищем диких ослов и пасущихся стад,
\end{tcolorbox}
\begin{tcolorbox}
\textsubscript{15} доколе не излиется на нас Дух свыше, и пустыня не сделается садом, а сад не будут считать лесом.
\end{tcolorbox}
\begin{tcolorbox}
\textsubscript{16} Тогда суд водворится в этой пустыне, и правосудие будет пребывать на плодоносном поле.
\end{tcolorbox}
\begin{tcolorbox}
\textsubscript{17} И делом правды будет мир, и плодом правосудия--спокойствие и безопасность вовеки.
\end{tcolorbox}
\begin{tcolorbox}
\textsubscript{18} Тогда народ мой будет жить в обители мира и в селениях безопасных, и в покоищах блаженных.
\end{tcolorbox}
\begin{tcolorbox}
\textsubscript{19} И град будет падать на лес, и город спустится в долину.
\end{tcolorbox}
\begin{tcolorbox}
\textsubscript{20} Блаженны вы, сеющие при всех водах и посылающие туда вола и осла.
\end{tcolorbox}
\subsection{CHAPTER 33}
\begin{tcolorbox}
\textsubscript{1} Горе тебе, опустошитель, который не был опустошаем, и грабитель, которого не грабили! Когда кончишь опустошение, будешь опустошен и ты; когда прекратишь грабительства, разграбят и тебя.
\end{tcolorbox}
\begin{tcolorbox}
\textsubscript{2} Господи! помилуй нас; на Тебя уповаем мы; будь нашею мышцею с раннего утра и спасением нашим во время тесное.
\end{tcolorbox}
\begin{tcolorbox}
\textsubscript{3} От грозного гласа [Твоего] побегут народы; когда восстанешь, рассеются племена,
\end{tcolorbox}
\begin{tcolorbox}
\textsubscript{4} и будут собирать добычу вашу, как собирает гусеница; бросятся на нее, как бросается саранча.
\end{tcolorbox}
\begin{tcolorbox}
\textsubscript{5} Высок Господь, живущий в вышних; Он наполнит Сион судом и правдою.
\end{tcolorbox}
\begin{tcolorbox}
\textsubscript{6} И настанут безопасные времена твои, изобилие спасения, мудрости и ведения; страх Господень будет сокровищем твоим.
\end{tcolorbox}
\begin{tcolorbox}
\textsubscript{7} Вот, сильные их кричат на улицах; послы для мира горько плачут.
\end{tcolorbox}
\begin{tcolorbox}
\textsubscript{8} Опустели дороги; не стало путешествующих; он нарушил договор, разрушил города, --ни во что ставит людей.
\end{tcolorbox}
\begin{tcolorbox}
\textsubscript{9} Земля сетует, сохнет; Ливан постыжен, увял; Сарон похож стал на пустыню, и обнажены от листьев своих Васан и Кармил.
\end{tcolorbox}
\begin{tcolorbox}
\textsubscript{10} Ныне Я восстану, говорит Господь, ныне поднимусь, ныне вознесусь.
\end{tcolorbox}
\begin{tcolorbox}
\textsubscript{11} Вы беременны сеном, разродитесь соломою; дыхание ваше--огонь, который пожрет вас.
\end{tcolorbox}
\begin{tcolorbox}
\textsubscript{12} И будут народы, [как] горящая известь, [как] срубленный терновник, будут сожжены в огне.
\end{tcolorbox}
\begin{tcolorbox}
\textsubscript{13} Слушайте, дальние, что сделаю Я; и вы, ближние, познайте могущество Мое.
\end{tcolorbox}
\begin{tcolorbox}
\textsubscript{14} Устрашились грешники на Сионе; трепет овладел нечестивыми: 'кто из нас может жить при огне пожирающем? кто из нас может жить при вечном пламени?' --
\end{tcolorbox}
\begin{tcolorbox}
\textsubscript{15} Тот, кто ходит в правде и говорит истину; кто презирает корысть от притеснения, удерживает руки свои от взяток, затыкает уши свои, чтобы не слышать о кровопролитии, и закрывает глаза свои, чтобы не видеть зла;
\end{tcolorbox}
\begin{tcolorbox}
\textsubscript{16} тот будет обитать на высотах; убежище его--неприступные скалы; хлеб будет дан ему; вода у него не иссякнет.
\end{tcolorbox}
\begin{tcolorbox}
\textsubscript{17} Глаза твои увидят Царя в красоте Его, узрят землю отдаленную;
\end{tcolorbox}
\begin{tcolorbox}
\textsubscript{18} сердце твое будет [только] вспоминать об ужасах: 'где делавший перепись? где весивший [дань]? где осматривающий башни?'
\end{tcolorbox}
\begin{tcolorbox}
\textsubscript{19} Не увидишь более народа свирепого, народа с глухою, невнятною речью, с языком странным, непонятным.
\end{tcolorbox}
\begin{tcolorbox}
\textsubscript{20} Взгляни на Сион, город праздничных собраний наших; глаза твои увидят Иерусалим, жилище мирное, непоколебимую скинию; столпы ее никогда не исторгнутся, и ни одна вервь ее не порвется.
\end{tcolorbox}
\begin{tcolorbox}
\textsubscript{21} Там у нас великий Господь будет вместо рек, вместо широких каналов; туда не войдет ни одно весельное судно, и не пройдет большой корабль.
\end{tcolorbox}
\begin{tcolorbox}
\textsubscript{22} Ибо Господь--судия наш, Господь--законодатель наш, Господь--царь наш; Он спасет нас.
\end{tcolorbox}
\begin{tcolorbox}
\textsubscript{23} Ослабли веревки твои, не могут удержать мачты и натянуть паруса. Тогда будет большой раздел добычи, так что и хромые пойдут на грабеж.
\end{tcolorbox}
\begin{tcolorbox}
\textsubscript{24} И ни один из жителей не скажет: 'я болен'; народу, живущему там, будут отпущены согрешения.
\end{tcolorbox}
\subsection{CHAPTER 34}
\begin{tcolorbox}
\textsubscript{1} Приступите, народы, слушайте и внимайте, племена! да слышит земля и всё, что наполняет ее, вселенная и всё рождающееся в ней!
\end{tcolorbox}
\begin{tcolorbox}
\textsubscript{2} Ибо гнев Господа на все народы, и ярость Его на все воинство их. Он предал их заклятию, отдал их на заклание.
\end{tcolorbox}
\begin{tcolorbox}
\textsubscript{3} И убитые их будут разбросаны, и от трупов их поднимется смрад, и горы размокнут от крови их.
\end{tcolorbox}
\begin{tcolorbox}
\textsubscript{4} И истлеет все небесное воинство; и небеса свернутся, как свиток книжный; и все воинство их падет, как спадает лист с виноградной лозы, и как увядший лист--со смоковницы.
\end{tcolorbox}
\begin{tcolorbox}
\textsubscript{5} Ибо упился меч Мой на небесах: вот, для суда нисходит он на Едом и на народ, преданный Мною заклятию.
\end{tcolorbox}
\begin{tcolorbox}
\textsubscript{6} Меч Господа наполнится кровью, утучнеет от тука, от крови агнцев и козлов, от тука с почек овнов: ибо жертва у Господа в Восоре и большое заклание в земле Едома.
\end{tcolorbox}
\begin{tcolorbox}
\textsubscript{7} И буйволы падут с ними и тельцы вместе с волами, и упьется земля их кровью, и прах их утучнеет от тука.
\end{tcolorbox}
\begin{tcolorbox}
\textsubscript{8} Ибо день мщения у Господа, год возмездия за Сион.
\end{tcolorbox}
\begin{tcolorbox}
\textsubscript{9} И превратятся реки его в смолу, и прах его--в серу, и будет земля его горящею смолою:
\end{tcolorbox}
\begin{tcolorbox}
\textsubscript{10} не будет гаснуть ни днем, ни ночью; вечно будет восходить дым ее; будет от рода в род оставаться опустелою; во веки веков никто не пройдет по ней;
\end{tcolorbox}
\begin{tcolorbox}
\textsubscript{11} и завладеют ею пеликан и еж; и филин и ворон поселятся в ней; и протянут по ней вервь разорения и отвес уничтожения.
\end{tcolorbox}
\begin{tcolorbox}
\textsubscript{12} Никого не останется там из знатных ее, кого можно было бы призвать на царство, и все князья ее будут ничто.
\end{tcolorbox}
\begin{tcolorbox}
\textsubscript{13} И зарастут дворцы ее колючими растениями, крапивою и репейником--твердыни ее; и будет она жилищем шакалов, пристанищем страусов.
\end{tcolorbox}
\begin{tcolorbox}
\textsubscript{14} И звери пустыни будут встречаться с дикими кошками, и лешие будут перекликаться один с другим; там будет отдыхать ночное привидение и находить себе покой.
\end{tcolorbox}
\begin{tcolorbox}
\textsubscript{15} Там угнездится летучий змей, будет класть яйца и выводить детей и собирать их под тень свою; там и коршуны будут собираться один к другому.
\end{tcolorbox}
\begin{tcolorbox}
\textsubscript{16} Отыщите в книге Господней и прочитайте; ни одно из сих не преминет придти, и одно другим не заменится. Ибо сами уста Его повелели, и сам дух Его соберет их.
\end{tcolorbox}
\begin{tcolorbox}
\textsubscript{17} И Сам Он бросил им жребий, и Его рука разделила им ее мерою; во веки будут они владеть ею, из рода в род будут жить на ней.
\end{tcolorbox}
\subsection{CHAPTER 35}
\begin{tcolorbox}
\textsubscript{1} Возвеселится пустыня и сухая земля, и возрадуется страна необитаемая и расцветет как нарцисс;
\end{tcolorbox}
\begin{tcolorbox}
\textsubscript{2} великолепно будет цвести и радоваться, будет торжествовать и ликовать; слава Ливана дастся ей, великолепие Кармила и Сарона; они увидят славу Господа, величие Бога нашего.
\end{tcolorbox}
\begin{tcolorbox}
\textsubscript{3} Укрепите ослабевшие руки и утвердите колени дрожащие;
\end{tcolorbox}
\begin{tcolorbox}
\textsubscript{4} скажите робким душею: будьте тверды, не бойтесь; вот Бог ваш, придет отмщение, воздаяние Божие; Он придет и спасет вас.
\end{tcolorbox}
\begin{tcolorbox}
\textsubscript{5} Тогда откроются глаза слепых, и уши глухих отверзутся.
\end{tcolorbox}
\begin{tcolorbox}
\textsubscript{6} Тогда хромой вскочит, как олень, и язык немого будет петь; ибо пробьются воды в пустыне, и в степи--потоки.
\end{tcolorbox}
\begin{tcolorbox}
\textsubscript{7} И превратится призрак вод в озеро, и жаждущая земля--в источники вод; в жилище шакалов, где они покоятся, будет место для тростника и камыша.
\end{tcolorbox}
\begin{tcolorbox}
\textsubscript{8} И будет там большая дорога, и путь по ней назовется путем святым: нечистый не будет ходить по нему; но он будет для них [одних]; идущие этим путем, даже и неопытные, не заблудятся.
\end{tcolorbox}
\begin{tcolorbox}
\textsubscript{9} Льва не будет там, и хищный зверь не взойдет на него; его не найдется там, а будут ходить искупленные.
\end{tcolorbox}
\begin{tcolorbox}
\textsubscript{10} И возвратятся избавленные Господом, придут на Сион с радостным восклицанием; и радость вечная будет над головою их; они найдут радость и веселье, а печаль и воздыхание удалятся.
\end{tcolorbox}
\subsection{CHAPTER 36}
\begin{tcolorbox}
\textsubscript{1} И было в четырнадцатый год царя Езекии, пошел Сеннахирим, царь Ассирийский, против всех укрепленных городов Иудеи и взял их.
\end{tcolorbox}
\begin{tcolorbox}
\textsubscript{2} И послал царь Ассирийский из Лахиса в Иерусалим к царю Езекии Рабсака с большим войском; и он остановился у водопровода верхнего пруда на дороге поля белильничьего.
\end{tcolorbox}
\begin{tcolorbox}
\textsubscript{3} И вышел к нему Елиаким, сын Хелкиин, начальник дворца, и Севна писец, и Иоах, сын Асафов, дееписатель.
\end{tcolorbox}
\begin{tcolorbox}
\textsubscript{4} И сказал им Рабсак: скажите Езекии: так говорит царь великий, царь Ассирийский: что это за упование, на которое ты уповаешь?
\end{tcolorbox}
\begin{tcolorbox}
\textsubscript{5} Я думаю, [что] это одни пустые слова, [а] для войны нужны совет и сила: итак на кого ты уповаешь, что отложился от меня?
\end{tcolorbox}
\begin{tcolorbox}
\textsubscript{6} Вот, ты думаешь опереться на Египет, на эту трость надломленную, которая, если кто опрется на нее, войдет тому в руку и проколет ее! Таков фараон, царь Египетский, для всех уповающих на него.
\end{tcolorbox}
\begin{tcolorbox}
\textsubscript{7} А если скажешь мне: 'на Господа, Бога нашего мы уповаем', то на того ли, которого высоты и жертвенники отменил Езекия и сказал Иуде и Иерусалиму: 'пред сим только жертвенником поклоняйтесь'?
\end{tcolorbox}
\begin{tcolorbox}
\textsubscript{8} Итак вступи в союз с господином моим, царем Ассирийским; я дам тебе две тысячи коней; можешь ли достать себе всадников на них?
\end{tcolorbox}
\begin{tcolorbox}
\textsubscript{9} И как ты хочешь заставить отступить вождя, одного из малейших рабов господина моего, надеясь на Египет, ради колесниц и коней?
\end{tcolorbox}
\begin{tcolorbox}
\textsubscript{10} Да разве я без воли Господней пошел на землю сию, чтобы разорить ее? Господь сказал мне: пойди на землю сию и разори ее.
\end{tcolorbox}
\begin{tcolorbox}
\textsubscript{11} И сказал Елиаким и Севна и Иоах Рабсаку: говори рабам твоим по-арамейски, потому что мы понимаем, а не говори с нами по-- иудейски, вслух народа, который на стене.
\end{tcolorbox}
\begin{tcolorbox}
\textsubscript{12} И сказал Рабсак: разве [только] к господину твоему и к тебе послал меня господин мой сказать слова сии? Нет, [также] и к людям, которые сидят на стене, чтобы есть помет свой и пить мочу свою с вами.
\end{tcolorbox}
\begin{tcolorbox}
\textsubscript{13} И встал Рабсак, и возгласил громким голосом по-иудейски, и сказал: слушайте слово царя великого, царя Ассирийского!
\end{tcolorbox}
\begin{tcolorbox}
\textsubscript{14} Так говорит царь: пусть не обольщает вас Езекия, ибо он не может спасти вас;
\end{tcolorbox}
\begin{tcolorbox}
\textsubscript{15} и пусть не обнадеживает вас Езекия Господом, говоря: 'спасет нас Господь; не будет город сей отдан в руки царя Ассирийского'.
\end{tcolorbox}
\begin{tcolorbox}
\textsubscript{16} Не слушайте Езекии, ибо так говорит царь Ассирийский: примиритесь со мною и выйдите ко мне, и пусть каждый ест плоды виноградной лозы своей и смоковницы своей, и пусть каждый пьет воду из своего колодезя,
\end{tcolorbox}
\begin{tcolorbox}
\textsubscript{17} доколе я не приду и не возьму вас в землю такую же, как и ваша земля, в землю хлеба и вина, в землю плодов и виноградников.
\end{tcolorbox}
\begin{tcolorbox}
\textsubscript{18} [Итак] да не обольщает вас Езекия, говоря: 'Господь спасет нас'. Спасли ли боги народов, каждый свою землю, от руки царя Ассирийского?
\end{tcolorbox}
\begin{tcolorbox}
\textsubscript{19} Где боги Емафа и Арпада? Где боги Сепарваима? Спасли ли они Самарию от руки моей?
\end{tcolorbox}
\begin{tcolorbox}
\textsubscript{20} Который из всех богов земель сих спас землю свою от руки моей? Так неужели спасет Господь Иерусалим от руки моей?
\end{tcolorbox}
\begin{tcolorbox}
\textsubscript{21} Но они молчали и не отвечали ему ни слова, потому что от царя дано было приказание: не отвечайте ему.
\end{tcolorbox}
\begin{tcolorbox}
\textsubscript{22} И пришел Елиаким, сын Хелкиин, начальник дворца, и Севна писец, и Иоах, сын Асафов, дееписатель, к Езекии в разодранных одеждах и пересказали ему слова Рабсака.
\end{tcolorbox}
\subsection{CHAPTER 37}
\begin{tcolorbox}
\textsubscript{1} Когда услышал это царь Езекия, то разодрал одежды свои и покрылся вретищем, и пошел в дом Господень;
\end{tcolorbox}
\begin{tcolorbox}
\textsubscript{2} и послал Елиакима, начальника дворца, и Севну писца, и старших священников, покрытых вретищами, к пророку Исаии, сыну Амосову.
\end{tcolorbox}
\begin{tcolorbox}
\textsubscript{3} И они сказали ему: так говорит Езекия: день скорби и наказания и посрамления день сей, ибо младенцы дошли до отверстия утробы матерней, а силы нет родить.
\end{tcolorbox}
\begin{tcolorbox}
\textsubscript{4} Может быть, услышит Господь Бог твой слова Рабсака, которого послал царь Ассирийский, господин его, хулить Бога живаго и поносить словами, какие слышал Господь, Бог твой; вознеси же молитву об оставшихся, которые находятся еще в живых.
\end{tcolorbox}
\begin{tcolorbox}
\textsubscript{5} И пришли слуги царя Езекии к Исаии.
\end{tcolorbox}
\begin{tcolorbox}
\textsubscript{6} И сказал им Исаия: так скажите господину вашему: так говорит Господь: не бойся слов, которые слышал ты, которыми поносили Меня слуги царя Ассирийского.
\end{tcolorbox}
\begin{tcolorbox}
\textsubscript{7} Вот, Я пошлю в него дух, и он услышит весть, и возвратится в землю свою, и Я поражу его мечом в земле его.
\end{tcolorbox}
\begin{tcolorbox}
\textsubscript{8} И возвратился Рабсак и нашел царя Ассирийского воюющим против Ливны; ибо он слышал, что тот отошел от Лахиса.
\end{tcolorbox}
\begin{tcolorbox}
\textsubscript{9} И услышал он о Тиргаке, царе Ефиопском; [ему] сказали: вот, он вышел сразиться с тобою. Услышав это, он послал послов к Езекии, сказав:
\end{tcolorbox}
\begin{tcolorbox}
\textsubscript{10} так скажите Езекии, царю Иудейскому: пусть не обманывает тебя Бог твой, на Которого ты уповаешь, думая: 'не будет отдан Иерусалим в руки царя Ассирийского'.
\end{tcolorbox}
\begin{tcolorbox}
\textsubscript{11} Вот, ты слышал, что сделали цари Ассирийские со всеми землями, положив на них заклятие; ты ли уцелеешь?
\end{tcolorbox}
\begin{tcolorbox}
\textsubscript{12} Боги народов, которых разорили отцы мои, спасли ли их, [спасли] [ли] Гозан и Харан, и Рецеф, и сынов Едена, что в Фалассаре?
\end{tcolorbox}
\begin{tcolorbox}
\textsubscript{13} Где царь Емафа и царь Арпада, и царь города Сепарваима, Ены и Иввы?
\end{tcolorbox}
\begin{tcolorbox}
\textsubscript{14} И взял Езекия письмо из руки послов и прочитал его, и пошел в дом Господень, и развернул его Езекия пред лицем Господним;
\end{tcolorbox}
\begin{tcolorbox}
\textsubscript{15} и молился Езекия пред лицем Господним и говорил:
\end{tcolorbox}
\begin{tcolorbox}
\textsubscript{16} Господи Саваоф, Боже Израилев, седящий на Херувимах! Ты один Бог всех царств земли; Ты сотворил небо и землю.
\end{tcolorbox}
\begin{tcolorbox}
\textsubscript{17} Приклони, Господи, ухо Твое и услышь; открой, Господи, очи Твои и воззри, и услышь слова Сеннахирима, который послал поносить Тебя, Бога живаго.
\end{tcolorbox}
\begin{tcolorbox}
\textsubscript{18} Правда, о, Господи! цари Ассирийские опустошили все страны и земли их
\end{tcolorbox}
\begin{tcolorbox}
\textsubscript{19} и побросали богов их в огонь; но это были не боги, а изделие рук человеческих, дерево и камень, потому и истребили их.
\end{tcolorbox}
\begin{tcolorbox}
\textsubscript{20} И ныне, Господи Боже наш, спаси нас от руки его; и узнают все царства земли, что Ты, Господи, Бог один.
\end{tcolorbox}
\begin{tcolorbox}
\textsubscript{21} И послал Исаия, сын Амосов, к Езекии сказать: так говорит Господь, Бог Израилев: о чем ты молился Мне против Сеннахирима, царя Ассирийского, --
\end{tcolorbox}
\begin{tcolorbox}
\textsubscript{22} вот слово, которое Господь изрек о нем: презрит тебя, посмеется над тобою девствующая дочь Сиона, покачает вслед тебя головою дочь Иерусалима.
\end{tcolorbox}
\begin{tcolorbox}
\textsubscript{23} Кого ты порицал и поносил? и на кого возвысил голос и поднял так высоко глаза твои? на Святаго Израилева.
\end{tcolorbox}
\begin{tcolorbox}
\textsubscript{24} Чрез рабов твоих ты порицал Господа и сказал: 'со множеством колесниц моих я взошел на высоту гор, на ребра Ливана, и срубил рослые кедры его, отличные кипарисы его, и пришел на самую вершину его, в рощу сада его;
\end{tcolorbox}
\begin{tcolorbox}
\textsubscript{25} и откапывал я, и пил воду; и осушу ступнями ног моих все реки Египетские'.
\end{tcolorbox}
\begin{tcolorbox}
\textsubscript{26} Разве не слышал ты, что Я издавна сделал это, в древние дни предначертал это, а ныне выполнил тем, что ты опустошаешь крепкие города, [превращая] их в груды развалин?
\end{tcolorbox}
\begin{tcolorbox}
\textsubscript{27} И жители их сделались маломощны, трепещут и остаются в стыде; они стали как трава на поле и нежная зелень, как порост на кровлях и опаленный хлеб, прежде нежели выколосился.
\end{tcolorbox}
\begin{tcolorbox}
\textsubscript{28} Сядешь ли ты, выйдешь ли, войдешь ли, Я знаю [всё, знаю] и дерзость твою против Меня.
\end{tcolorbox}
\begin{tcolorbox}
\textsubscript{29} За твою дерзость против Меня и за то, что надмение твое дошло до ушей Моих, Я вложу кольцо Мое в ноздри твои и удила Мои в рот твой, и возвращу тебя назад тою же дорогою, которою ты пришел.
\end{tcolorbox}
\begin{tcolorbox}
\textsubscript{30} И вот, тебе, Езекия, знамение: ешьте в этот год выросшее от упавшего зерна, и на другой год--самородное; а на третий год сейте и жните, и садите виноградные сады, и ешьте плоды их.
\end{tcolorbox}
\begin{tcolorbox}
\textsubscript{31} И уцелевший в доме Иудином остаток пустит опять корень внизу и принесет плод вверху,
\end{tcolorbox}
\begin{tcolorbox}
\textsubscript{32} ибо из Иерусалима произойдет остаток, и спасенное--от горы Сиона. Ревность Господа Саваофа соделает это.
\end{tcolorbox}
\begin{tcolorbox}
\textsubscript{33} Посему так говорит Господь о царе Ассирийском: 'не войдет он в этот город и не бросит туда стрелы, и не приступит к нему со щитом, и не насыплет против него вала.
\end{tcolorbox}
\begin{tcolorbox}
\textsubscript{34} По той же дороге, по которой пришел, возвратится, а в город сей не войдет, говорит Господь.
\end{tcolorbox}
\begin{tcolorbox}
\textsubscript{35} Я буду охранять город сей, чтобы спасти его ради Себя и ради Давида, раба Моего'.
\end{tcolorbox}
\begin{tcolorbox}
\textsubscript{36} И вышел Ангел Господень и поразил в стане Ассирийском сто восемьдесят пять тысяч [человек]. И встали поутру, и вот, всё тела мертвые.
\end{tcolorbox}
\begin{tcolorbox}
\textsubscript{37} И отступил, и пошел, и возвратился Сеннахирим, царь Ассирийский, и жил в Ниневии.
\end{tcolorbox}
\begin{tcolorbox}
\textsubscript{38} И когда он поклонялся в доме Нисроха, бога своего, Адрамелех и Шарецер, сыновья его, убили его мечом, а сами убежали в землю Араратскую. И воцарился Асардан, сын его, вместо него.
\end{tcolorbox}
\subsection{CHAPTER 38}
\begin{tcolorbox}
\textsubscript{1} В те дни Езекия заболел смертельно. И пришел к нему пророк Исаия, сын Амосов, и сказал ему: так говорит Господь: сделай завещание для дома твоего, ибо ты умрешь, не выздоровеешь.
\end{tcolorbox}
\begin{tcolorbox}
\textsubscript{2} Тогда Езекия отворотился лицем к стене и молился Господу, говоря:
\end{tcolorbox}
\begin{tcolorbox}
\textsubscript{3} 'о, Господи! вспомни, что я ходил пред лицем Твоим верно и с преданным [Тебе] сердцем и делал угодное в очах Твоих'. И заплакал Езекия сильно.
\end{tcolorbox}
\begin{tcolorbox}
\textsubscript{4} И было слово Господне к Исаии, и сказано:
\end{tcolorbox}
\begin{tcolorbox}
\textsubscript{5} пойди и скажи Езекии: так говорит Господь, Бог Давида, отца твоего: Я услышал молитву твою, увидел слезы твои, и вот, Я прибавлю к дням твоим пятнадцать лет,
\end{tcolorbox}
\begin{tcolorbox}
\textsubscript{6} и от руки царя Ассирийского спасу тебя и город сей и защищу город сей.
\end{tcolorbox}
\begin{tcolorbox}
\textsubscript{7} И вот тебе знамение от Господа, что Господь исполнит слово, которое Он изрек.
\end{tcolorbox}
\begin{tcolorbox}
\textsubscript{8} Вот, я возвращу назад на десять ступеней солнечную тень, которая прошла по ступеням Ахазовым. И возвратилось солнце на десять ступеней по ступеням, по которым оно сходило.
\end{tcolorbox}
\begin{tcolorbox}
\textsubscript{9} Молитва Езекии, царя Иудейского, когда он болен был и выздоровел от болезни:
\end{tcolorbox}
\begin{tcolorbox}
\textsubscript{10} 'Я сказал в себе: в преполовение дней моих должен я идти во врата преисподней; я лишен остатка лет моих.
\end{tcolorbox}
\begin{tcolorbox}
\textsubscript{11} Я говорил: не увижу я Господа, Господа на земле живых; не увижу больше человека между живущими в мире;
\end{tcolorbox}
\begin{tcolorbox}
\textsubscript{12} жилище мое снимается с места и уносится от меня, как шалаш пастушеский; я должен отрезать подобно ткачу жизнь мою; Он отрежет меня от основы; день и ночь я ждал, что Ты пошлешь мне кончину.
\end{tcolorbox}
\begin{tcolorbox}
\textsubscript{13} Я ждал до утра; подобно льву, Он сокрушал все кости мои; день и ночь я ждал, что Ты пошлешь мне кончину.
\end{tcolorbox}
\begin{tcolorbox}
\textsubscript{14} Как журавль, как ласточка издавал я звуки, тосковал как голубь; уныло смотрели глаза мои к небу: Господи! тесно мне; спаси меня.
\end{tcolorbox}
\begin{tcolorbox}
\textsubscript{15} Что скажу я? Он сказал мне, Он и сделал. Тихо буду проводить все годы жизни моей, помня горесть души моей.
\end{tcolorbox}
\begin{tcolorbox}
\textsubscript{16} Господи! так живут, и во всем этом жизнь моего духа; Ты исцелишь меня, даруешь мне жизнь.
\end{tcolorbox}
\begin{tcolorbox}
\textsubscript{17} Вот, во благо мне была сильная горесть, и Ты избавил душу мою от рва погибели, бросил все грехи мои за хребет Свой.
\end{tcolorbox}
\begin{tcolorbox}
\textsubscript{18} Ибо не преисподняя славит Тебя, не смерть восхваляет Тебя, не нисшедшие в могилу уповают на истину Твою.
\end{tcolorbox}
\begin{tcolorbox}
\textsubscript{19} Живой, только живой прославит Тебя, как я ныне: отец возвестит детям истину Твою.
\end{tcolorbox}
\begin{tcolorbox}
\textsubscript{20} Господь спасет меня; и мы во все дни жизни нашей [со звуками] струн моих будем воспевать песни в доме Господнем'.
\end{tcolorbox}
\begin{tcolorbox}
\textsubscript{21} И сказал Исаия: пусть принесут пласт смокв и обложат им нарыв; и он выздоровеет.
\end{tcolorbox}
\begin{tcolorbox}
\textsubscript{22} А Езекия сказал: какое знамение, что я буду ходить в дом Господень?
\end{tcolorbox}
\subsection{CHAPTER 39}
\begin{tcolorbox}
\textsubscript{1} В то время Меродах Валадан, сын Валадана, царь Вавилонский, прислал к Езекии письмо и дары, ибо слышал, что он был болен и выздоровел.
\end{tcolorbox}
\begin{tcolorbox}
\textsubscript{2} И обрадовался посланным Езекия, и показал им дом сокровищ своих, серебро и золото, и ароматы, и драгоценные масти, весь оружейный свой дом и все, что находилось в сокровищницах его; ничего не осталось, чего не показал бы им Езекия в доме своем и во всем владении своем.
\end{tcolorbox}
\begin{tcolorbox}
\textsubscript{3} И пришел пророк Исаия к царю Езекии и сказал ему: что говорили эти люди? и откуда они приходили к тебе? Езекия сказал: из далекой земли приходили они ко мне, из Вавилона.
\end{tcolorbox}
\begin{tcolorbox}
\textsubscript{4} И сказал [Исаия]: что видели они в доме твоем? Езекия сказал: видели всё, что есть в доме моем; ничего не осталось в сокровищницах моих, чего я не показал бы им.
\end{tcolorbox}
\begin{tcolorbox}
\textsubscript{5} И сказал Исаия Езекии: выслушай слово Господа Саваофа:
\end{tcolorbox}
\begin{tcolorbox}
\textsubscript{6} вот, придут дни, и всё, что есть в доме твоем и что собрали отцы твои до сего дня, будет унесено в Вавилон; ничего не останется, говорит Господь.
\end{tcolorbox}
\begin{tcolorbox}
\textsubscript{7} И возьмут из сыновей твоих, которые произойдут от тебя, которых ты родишь, --и они будут евнухами во дворце царя Вавилонского.
\end{tcolorbox}
\begin{tcolorbox}
\textsubscript{8} И сказал Езекия Исаии: благо слово Господне, которое ты изрек; потому что, присовокупил он, мир и благосостояние пребудут во дни мои.
\end{tcolorbox}
\subsection{CHAPTER 40}
\begin{tcolorbox}
\textsubscript{1} Утешайте, утешайте народ Мой, говорит Бог ваш;
\end{tcolorbox}
\begin{tcolorbox}
\textsubscript{2} говорите к сердцу Иерусалима и возвещайте ему, что исполнилось время борьбы его, что за неправды его сделано удовлетворение, ибо он от руки Господней принял вдвое за все грехи свои.
\end{tcolorbox}
\begin{tcolorbox}
\textsubscript{3} Глас вопиющего в пустыне: приготовьте путь Господу, прямыми сделайте в степи стези Богу нашему;
\end{tcolorbox}
\begin{tcolorbox}
\textsubscript{4} всякий дол да наполнится, и всякая гора и холм да понизятся, кривизны выпрямятся и неровные пути сделаются гладкими;
\end{tcolorbox}
\begin{tcolorbox}
\textsubscript{5} и явится слава Господня, и узрит всякая плоть [спасение Божие]; ибо уста Господни изрекли это.
\end{tcolorbox}
\begin{tcolorbox}
\textsubscript{6} Голос говорит: возвещай! И сказал: что мне возвещать? Всякая плоть--трава, и вся красота ее--как цвет полевой.
\end{tcolorbox}
\begin{tcolorbox}
\textsubscript{7} Засыхает трава, увядает цвет, когда дунет на него дуновение Господа: так и народ--трава.
\end{tcolorbox}
\begin{tcolorbox}
\textsubscript{8} Трава засыхает, цвет увядает, а слово Бога нашего пребудет вечно.
\end{tcolorbox}
\begin{tcolorbox}
\textsubscript{9} Взойди на высокую гору, благовествующий Сион! возвысь с силою голос твой, благовествующий Иерусалим! возвысь, не бойся; скажи городам Иудиным: вот Бог ваш!
\end{tcolorbox}
\begin{tcolorbox}
\textsubscript{10} Вот, Господь Бог грядет с силою, и мышца Его со властью. Вот, награда Его с Ним и воздаяние Его пред лицем Его.
\end{tcolorbox}
\begin{tcolorbox}
\textsubscript{11} Как пастырь Он будет пасти стадо Свое; агнцев будет брать на руки и носить на груди Своей, и водить дойных.
\end{tcolorbox}
\begin{tcolorbox}
\textsubscript{12} Кто исчерпал воды горстью своею и пядью измерил небеса, и вместил в меру прах земли, и взвесил на весах горы и на чашах весовых холмы?
\end{tcolorbox}
\begin{tcolorbox}
\textsubscript{13} Кто уразумел дух Господа, и был советником у Него и учил Его?
\end{tcolorbox}
\begin{tcolorbox}
\textsubscript{14} С кем советуется Он, и кто вразумляет Его и наставляет Его на путь правды, и учит Его знанию, и указывает Ему путь мудрости?
\end{tcolorbox}
\begin{tcolorbox}
\textsubscript{15} Вот народы--как капля из ведра, и считаются как пылинка на весах. Вот, острова как порошинку поднимает Он.
\end{tcolorbox}
\begin{tcolorbox}
\textsubscript{16} И Ливана недостаточно для жертвенного огня, и животных на нем--для всесожжения.
\end{tcolorbox}
\begin{tcolorbox}
\textsubscript{17} Все народы пред Ним как ничто, --менее ничтожества и пустоты считаются у Него.
\end{tcolorbox}
\begin{tcolorbox}
\textsubscript{18} Итак кому уподобите вы Бога? И какое подобие найдете Ему?
\end{tcolorbox}
\begin{tcolorbox}
\textsubscript{19} Идола выливает художник, и золотильщик покрывает его золотом и приделывает серебряные цепочки.
\end{tcolorbox}
\begin{tcolorbox}
\textsubscript{20} А кто беден для такого приношения, выбирает негниющее дерево, приискивает себе искусного художника, чтобы сделать идола, который стоял бы твердо.
\end{tcolorbox}
\begin{tcolorbox}
\textsubscript{21} Разве не знаете? разве вы не слышали? разве вам не говорено было от начала? разве вы не уразумели из оснований земли?
\end{tcolorbox}
\begin{tcolorbox}
\textsubscript{22} Он есть Тот, Который восседает над кругом земли, и живущие на ней--как саранча [пред Ним]; Он распростер небеса, как тонкую ткань, и раскинул их, как шатер для жилья.
\end{tcolorbox}
\begin{tcolorbox}
\textsubscript{23} Он обращает князей в ничто, делает чем-то пустым судей земли.
\end{tcolorbox}
\begin{tcolorbox}
\textsubscript{24} Едва они посажены, едва посеяны, едва укоренился в земле ствол их, и как только Он дохнул на них, они высохли, и вихрь унес их, как солому.
\end{tcolorbox}
\begin{tcolorbox}
\textsubscript{25} Кому же вы уподобите Меня и с кем сравните? говорит Святый.
\end{tcolorbox}
\begin{tcolorbox}
\textsubscript{26} Поднимите глаза ваши на высоту [небес] и посмотрите, кто сотворил их? Кто выводит воинство их счетом? Он всех их называет по имени: по множеству могущества и великой силе у Него ничто не выбывает.
\end{tcolorbox}
\begin{tcolorbox}
\textsubscript{27} Как же говоришь ты, Иаков, и высказываешь, Израиль: 'путь мой сокрыт от Господа, и дело мое забыто у Бога моего'?
\end{tcolorbox}
\begin{tcolorbox}
\textsubscript{28} Разве ты не знаешь? разве ты не слышал, что вечный Господь Бог, сотворивший концы земли, не утомляется и не изнемогает? разум Его неисследим.
\end{tcolorbox}
\begin{tcolorbox}
\textsubscript{29} Он дает утомленному силу, и изнемогшему дарует крепость.
\end{tcolorbox}
\begin{tcolorbox}
\textsubscript{30} Утомляются и юноши и ослабевают, и молодые люди падают,
\end{tcolorbox}
\begin{tcolorbox}
\textsubscript{31} а надеющиеся на Господа обновятся в силе: поднимут крылья, как орлы, потекут--и не устанут, пойдут--и не утомятся.
\end{tcolorbox}
\subsection{CHAPTER 41}
\begin{tcolorbox}
\textsubscript{1} Умолкните предо Мною, острова, и народы да обновят свои силы; пусть они приблизятся и скажут: 'станем вместе на суд'.
\end{tcolorbox}
\begin{tcolorbox}
\textsubscript{2} Кто воздвиг от востока мужа правды, призвал его следовать за собою, предал ему народы и покорил царей? Он обратил их мечом его в прах, луком его в солому, разносимую ветром.
\end{tcolorbox}
\begin{tcolorbox}
\textsubscript{3} Он гонит их, идет спокойно дорогою, по которой никогда не ходил ногами своими.
\end{tcolorbox}
\begin{tcolorbox}
\textsubscript{4} Кто сделал и совершил это? Тот, Кто от начала вызывает роды; Я--Господь первый, и в последних--Я тот же.
\end{tcolorbox}
\begin{tcolorbox}
\textsubscript{5} Увидели острова и ужаснулись, концы земли затрепетали. Они сблизились и сошлись;
\end{tcolorbox}
\begin{tcolorbox}
\textsubscript{6} каждый помогает своему товарищу и говорит своему брату: 'крепись!'
\end{tcolorbox}
\begin{tcolorbox}
\textsubscript{7} Кузнец ободряет плавильщика, разглаживающий листы молотом--кующего на наковальне, говоря о спайке: 'хороша'; и укрепляет гвоздями, чтобы было твердо.
\end{tcolorbox}
\begin{tcolorbox}
\textsubscript{8} А ты, Израиль, раб Мой, Иаков, которого Я избрал, семя Авраама, друга Моего, --
\end{tcolorbox}
\begin{tcolorbox}
\textsubscript{9} ты, которого Я взял от концов земли и призвал от краев ее, и сказал тебе: 'ты Мой раб, Я избрал тебя и не отвергну тебя':
\end{tcolorbox}
\begin{tcolorbox}
\textsubscript{10} не бойся, ибо Я с тобою; не смущайся, ибо Я Бог твой; Я укреплю тебя, и помогу тебе, и поддержу тебя десницею правды Моей.
\end{tcolorbox}
\begin{tcolorbox}
\textsubscript{11} Вот, в стыде и посрамлении останутся все, раздраженные против тебя; будут как ничто и погибнут препирающиеся с тобою.
\end{tcolorbox}
\begin{tcolorbox}
\textsubscript{12} Будешь искать их, и не найдешь их, враждующих против тебя; борющиеся с тобою будут как ничто, совершенно ничто;
\end{tcolorbox}
\begin{tcolorbox}
\textsubscript{13} ибо Я Господь, Бог твой; держу тебя за правую руку твою, говорю тебе: 'не бойся, Я помогаю тебе'.
\end{tcolorbox}
\begin{tcolorbox}
\textsubscript{14} Не бойся, червь Иаков, малолюдный Израиль, --Я помогаю тебе, говорит Господь и Искупитель твой, Святый Израилев.
\end{tcolorbox}
\begin{tcolorbox}
\textsubscript{15} Вот, Я сделал тебя острым молотилом, новым, зубчатым; ты будешь молотить и растирать горы, и холмы сделаешь, как мякину.
\end{tcolorbox}
\begin{tcolorbox}
\textsubscript{16} Ты будешь веять их, и ветер разнесет их, и вихрь развеет их; а ты возрадуешься о Господе, будешь хвалиться Святым Израилевым.
\end{tcolorbox}
\begin{tcolorbox}
\textsubscript{17} Бедные и нищие ищут воды, и нет [ее]; язык их сохнет от жажды: Я, Господь, услышу их, Я, Бог Израилев, не оставлю их.
\end{tcolorbox}
\begin{tcolorbox}
\textsubscript{18} Открою на горах реки и среди долин источники; пустыню сделаю озером и сухую землю--источниками воды;
\end{tcolorbox}
\begin{tcolorbox}
\textsubscript{19} посажу в пустыне кедр, ситтим и мирту и маслину; насажу в степи кипарис, явор и бук вместе,
\end{tcolorbox}
\begin{tcolorbox}
\textsubscript{20} чтобы увидели и познали, и рассмотрели и уразумели, что рука Господня соделала это, и Святый Израилев сотворил сие.
\end{tcolorbox}
\begin{tcolorbox}
\textsubscript{21} Представьте дело ваше, говорит Господь; приведите ваши доказательства, говорит Царь Иакова.
\end{tcolorbox}
\begin{tcolorbox}
\textsubscript{22} Пусть они представят и скажут нам, что произойдет; пусть возвестят что-либо прежде, нежели оно произошло, и мы вникнем умом своим и узнаем, как оно кончилось, или пусть предвозвестят нам о будущем.
\end{tcolorbox}
\begin{tcolorbox}
\textsubscript{23} Скажите, что произойдет в будущем, и мы будем знать, что вы боги, или сделайте что-нибудь, доброе ли, худое ли, чтобы мы изумились и вместе с вами увидели.
\end{tcolorbox}
\begin{tcolorbox}
\textsubscript{24} Но вы ничто, и дело ваше ничтожно; мерзость тот, кто избирает вас.
\end{tcolorbox}
\begin{tcolorbox}
\textsubscript{25} Я воздвиг его от севера, и он придет; от восхода солнца будет призывать имя Мое и попирать владык, как грязь, и топтать, как горшечник глину.
\end{tcolorbox}
\begin{tcolorbox}
\textsubscript{26} Кто возвестил об этом изначала, чтобы нам знать, и задолго пред тем, чтобы нам можно было сказать: 'правда'? Но никто не сказал, никто не возвестил, никто не слыхал слов ваших.
\end{tcolorbox}
\begin{tcolorbox}
\textsubscript{27} Я первый [сказал] Сиону: 'вот оно!' и дал Иерусалиму благовестника.
\end{tcolorbox}
\begin{tcolorbox}
\textsubscript{28} Итак Я смотрел, и не было никого, и между ними не нашлось советника, чтоб Я мог спросить их, и они дали ответ.
\end{tcolorbox}
\begin{tcolorbox}
\textsubscript{29} Вот, все они ничто, ничтожны и дела их; ветер и пустота истуканы их.
\end{tcolorbox}
\subsection{CHAPTER 42}
\begin{tcolorbox}
\textsubscript{1} Вот, Отрок Мой, Которого Я держу за руку, избранный Мой, к которому благоволит душа Моя. Положу дух Мой на Него, и возвестит народам суд;
\end{tcolorbox}
\begin{tcolorbox}
\textsubscript{2} не возопиет и не возвысит голоса Своего, и не даст услышать его на улицах;
\end{tcolorbox}
\begin{tcolorbox}
\textsubscript{3} трости надломленной не переломит, и льна курящегося не угасит; будет производить суд по истине;
\end{tcolorbox}
\begin{tcolorbox}
\textsubscript{4} не ослабеет и не изнеможет, доколе на земле не утвердит суда, и на закон Его будут уповать острова.
\end{tcolorbox}
\begin{tcolorbox}
\textsubscript{5} Так говорит Господь Бог, сотворивший небеса и пространство их, распростерший землю с произведениями ее, дающий дыхание народу на ней и дух ходящим по ней.
\end{tcolorbox}
\begin{tcolorbox}
\textsubscript{6} Я, Господь, призвал Тебя в правду, и буду держать Тебя за руку и хранить Тебя, и поставлю Тебя в завет для народа, во свет для язычников,
\end{tcolorbox}
\begin{tcolorbox}
\textsubscript{7} чтобы открыть глаза слепых, чтобы узников вывести из заключения и сидящих во тьме--из темницы.
\end{tcolorbox}
\begin{tcolorbox}
\textsubscript{8} Я Господь, это--Мое имя, и не дам славы Моей иному и хвалы Моей истуканам.
\end{tcolorbox}
\begin{tcolorbox}
\textsubscript{9} Вот, [предсказанное] прежде сбылось, и новое Я возвещу; прежде нежели оно произойдет, Я возвещу вам.
\end{tcolorbox}
\begin{tcolorbox}
\textsubscript{10} Пойте Господу новую песнь, хвалу Ему от концов земли, вы, плавающие по морю, и всё, наполняющее его, острова и живущие на них.
\end{tcolorbox}
\begin{tcolorbox}
\textsubscript{11} Да возвысит голос пустыня и города ее, селения, где обитает Кидар; да торжествуют живущие на скалах, да возглашают с вершин гор.
\end{tcolorbox}
\begin{tcolorbox}
\textsubscript{12} Да воздадут Господу славу, и хвалу Его да возвестят на островах.
\end{tcolorbox}
\begin{tcolorbox}
\textsubscript{13} Господь выйдет, как исполин, как муж браней возбудит ревность; воззовет и поднимет воинский крик, и покажет Себя сильным против врагов Своих.
\end{tcolorbox}
\begin{tcolorbox}
\textsubscript{14} Долго молчал Я, терпел, удерживался; теперь буду кричать, как рождающая, буду разрушать и поглощать всё;
\end{tcolorbox}
\begin{tcolorbox}
\textsubscript{15} опустошу горы и холмы, и всю траву их иссушу; и реки сделаю островами, и осушу озера;
\end{tcolorbox}
\begin{tcolorbox}
\textsubscript{16} и поведу слепых дорогою, которой они не знают, неизвестными путями буду вести их; мрак сделаю светом пред ними, и кривые пути--прямыми: вот что Я сделаю для них и не оставлю их.
\end{tcolorbox}
\begin{tcolorbox}
\textsubscript{17} Тогда обратятся вспять и великим стыдом покроются надеющиеся на идолов, говорящие истуканам: 'вы наши боги'.
\end{tcolorbox}
\begin{tcolorbox}
\textsubscript{18} Слушайте, глухие, и смотрите, слепые, чтобы видеть.
\end{tcolorbox}
\begin{tcolorbox}
\textsubscript{19} Кто так слеп, как раб Мой, и глух, как вестник Мой, Мною посланный? Кто так слеп, как возлюбленный, так слеп, как раб Господа?
\end{tcolorbox}
\begin{tcolorbox}
\textsubscript{20} Ты видел многое, но не замечал; уши были открыты, но не слышал.
\end{tcolorbox}
\begin{tcolorbox}
\textsubscript{21} Господу угодно было, ради правды Своей, возвеличить и прославить закон.
\end{tcolorbox}
\begin{tcolorbox}
\textsubscript{22} Но это народ разоренный и разграбленный; все они связаны в подземельях и сокрыты в темницах; сделались добычею, и нет избавителя; ограблены, и никто не говорит: 'отдай назад!'
\end{tcolorbox}
\begin{tcolorbox}
\textsubscript{23} Кто из вас приклонил к этому ухо, вникнул и выслушал это для будущего?
\end{tcolorbox}
\begin{tcolorbox}
\textsubscript{24} Кто предал Иакова на разорение и Израиля грабителям? не Господь ли, против Которого мы грешили? Не хотели они ходить путями Его и не слушали закона Его.
\end{tcolorbox}
\begin{tcolorbox}
\textsubscript{25} И Он излил на них ярость гнева Своего и лютость войны: она окружила их пламенем со всех сторон, но они не примечали; и горела у них, но они не уразумели этого сердцем.
\end{tcolorbox}
\subsection{CHAPTER 43}
\begin{tcolorbox}
\textsubscript{1} Ныне же так говорит Господь, сотворивший тебя, Иаков, и устроивший тебя, Израиль: не бойся, ибо Я искупил тебя, назвал тебя по имени твоему; ты Мой.
\end{tcolorbox}
\begin{tcolorbox}
\textsubscript{2} Будешь ли переходить через воды, Я с тобою, --через реки ли, они не потопят тебя; пойдешь ли через огонь, не обожжешься, и пламя не опалит тебя.
\end{tcolorbox}
\begin{tcolorbox}
\textsubscript{3} Ибо Я Господь, Бог твой, Святый Израилев, Спаситель твой; в выкуп за тебя отдал Египет, Ефиопию и Савею за тебя.
\end{tcolorbox}
\begin{tcolorbox}
\textsubscript{4} Так как ты дорог в очах Моих, многоценен, и Я возлюбил тебя, то отдам [других] людей за тебя, и народы за душу твою.
\end{tcolorbox}
\begin{tcolorbox}
\textsubscript{5} Не бойся, ибо Я с тобою; от востока приведу племя твое и от запада соберу тебя.
\end{tcolorbox}
\begin{tcolorbox}
\textsubscript{6} Северу скажу: 'отдай'; и югу: 'не удерживай; веди сыновей Моих издалека и дочерей Моих от концов земли,
\end{tcolorbox}
\begin{tcolorbox}
\textsubscript{7} каждого кто называется Моим именем, кого Я сотворил для славы Моей, образовал и устроил.
\end{tcolorbox}
\begin{tcolorbox}
\textsubscript{8} Выведи народ слепой, хотя у него есть глаза, и глухой, хотя у него есть уши'.
\end{tcolorbox}
\begin{tcolorbox}
\textsubscript{9} Пусть все народы соберутся вместе, и совокупятся племена. Кто между ними предсказал это? пусть возвестят, что было от начала; пусть представят свидетелей от себя и оправдаются, чтобы можно было услышать и сказать: 'правда!'
\end{tcolorbox}
\begin{tcolorbox}
\textsubscript{10} А Мои свидетели, говорит Господь, вы и раб Мой, которого Я избрал, чтобы вы знали и верили Мне, и разумели, что это Я: прежде Меня не было Бога и после Меня не будет.
\end{tcolorbox}
\begin{tcolorbox}
\textsubscript{11} Я, Я Господь, и нет Спасителя кроме Меня.
\end{tcolorbox}
\begin{tcolorbox}
\textsubscript{12} Я предрек и спас, и возвестил; а иного нет у вас, и вы--свидетели Мои, говорит Господь, что Я Бог;
\end{tcolorbox}
\begin{tcolorbox}
\textsubscript{13} от [начала] дней Я Тот же, и никто не спасет от руки Моей; Я сделаю, и кто отменит это?
\end{tcolorbox}
\begin{tcolorbox}
\textsubscript{14} Так говорит Господь, Искупитель ваш, Святый Израилев: ради вас Я послал в Вавилон и сокрушил все запоры и Халдеев, величавшихся кораблями.
\end{tcolorbox}
\begin{tcolorbox}
\textsubscript{15} Я Господь, Святый ваш, Творец Израиля, Царь ваш.
\end{tcolorbox}
\begin{tcolorbox}
\textsubscript{16} Так говорит Господь, открывший в море дорогу, в сильных водах стезю,
\end{tcolorbox}
\begin{tcolorbox}
\textsubscript{17} выведший колесницы и коней, войско и силу; все легли вместе, не встали; потухли как светильня, погасли.
\end{tcolorbox}
\begin{tcolorbox}
\textsubscript{18} Но вы не вспоминаете прежнего и о древнем не помышляете.
\end{tcolorbox}
\begin{tcolorbox}
\textsubscript{19} Вот, Я делаю новое; ныне же оно явится; неужели вы и этого не хотите знать? Я проложу дорогу в степи, реки в пустыне.
\end{tcolorbox}
\begin{tcolorbox}
\textsubscript{20} Полевые звери прославят Меня, шакалы и страусы, потому что Я в пустынях дам воду, реки в сухой степи, чтобы поить избранный народ Мой.
\end{tcolorbox}
\begin{tcolorbox}
\textsubscript{21} Этот народ Я образовал для Себя; он будет возвещать славу Мою.
\end{tcolorbox}
\begin{tcolorbox}
\textsubscript{22} А ты, Иаков, не взывал ко Мне; ты, Израиль, не трудился для Меня.
\end{tcolorbox}
\begin{tcolorbox}
\textsubscript{23} Ты не приносил Мне агнцев твоих во всесожжение и жертвами твоими не чтил Меня. Я не заставлял тебя служить Мне хлебным приношением и не отягощал тебя фимиамом.
\end{tcolorbox}
\begin{tcolorbox}
\textsubscript{24} Ты не покупал Мне благовонной трости за серебро и туком жертв твоих не насыщал Меня; но ты грехами твоими затруднял Меня, беззакониями твоими отягощал Меня.
\end{tcolorbox}
\begin{tcolorbox}
\textsubscript{25} Я, Я Сам изглаживаю преступления твои ради Себя Самого и грехов твоих не помяну:
\end{tcolorbox}
\begin{tcolorbox}
\textsubscript{26} припомни Мне; станем судиться; говори ты, чтоб оправдаться.
\end{tcolorbox}
\begin{tcolorbox}
\textsubscript{27} Праотец твой согрешил, и ходатаи твои отступили от Меня.
\end{tcolorbox}
\begin{tcolorbox}
\textsubscript{28} За то Я предстоятелей святилища лишил священства и Иакова предал на заклятие и Израиля на поругание.
\end{tcolorbox}
\subsection{CHAPTER 44}
\begin{tcolorbox}
\textsubscript{1} А ныне слушай, Иаков, раб Мой, и Израиль, которого Я избрал.
\end{tcolorbox}
\begin{tcolorbox}
\textsubscript{2} Так говорит Господь, создавший тебя и образовавший тебя, помогающий тебе от утробы матерней: не бойся, раб Мой, Иаков, и возлюбленный [Израиль], которого Я избрал;
\end{tcolorbox}
\begin{tcolorbox}
\textsubscript{3} ибо Я изолью воды на жаждущее и потоки на иссохшее; излию дух Мой на племя твое и благословение Мое на потомков твоих.
\end{tcolorbox}
\begin{tcolorbox}
\textsubscript{4} И будут расти между травою, как ивы при потоках вод.
\end{tcolorbox}
\begin{tcolorbox}
\textsubscript{5} Один скажет: 'я Господень', другой назовется именем Иакова; а иной напишет рукою своею: 'я Господень', и прозовется именем Израиля.
\end{tcolorbox}
\begin{tcolorbox}
\textsubscript{6} Так говорит Господь, Царь Израиля, и Искупитель его, Господь Саваоф: Я первый и Я последний, и кроме Меня нет Бога,
\end{tcolorbox}
\begin{tcolorbox}
\textsubscript{7} ибо кто как Я? Пусть он расскажет, возвестит и в порядке представит Мне [всё] с того времени, как Я устроил народ древний, или пусть возвестят наступающее и будущее.
\end{tcolorbox}
\begin{tcolorbox}
\textsubscript{8} Не бойтесь и не страшитесь: не издавна ли Я возвестил тебе и предсказал? И вы Мои свидетели. Есть ли Бог кроме Меня? нет другой твердыни, никакой не знаю.
\end{tcolorbox}
\begin{tcolorbox}
\textsubscript{9} Делающие идолов все ничтожны, и вожделеннейшие их не приносят никакой пользы, и они сами себе свидетели в том. Они не видят и не разумеют, и потому будут посрамлены.
\end{tcolorbox}
\begin{tcolorbox}
\textsubscript{10} Кто сделал бога и вылил идола, не приносящего никакой пользы?
\end{tcolorbox}
\begin{tcolorbox}
\textsubscript{11} Все участвующие в этом будут постыжены, ибо и художники сами из людей же; пусть все они соберутся и станут; они устрашатся, и все будут постыжены.
\end{tcolorbox}
\begin{tcolorbox}
\textsubscript{12} Кузнец делает из железа топор и работает на угольях, молотами обделывает его и трудится над ним сильною рукою своею до того, что становится голоден и бессилен, не пьет воды и изнемогает.
\end{tcolorbox}
\begin{tcolorbox}
\textsubscript{13} Плотник [выбрав дерево], протягивает по нему линию, остроконечным орудием делает на нем очертание, потом обделывает его резцом и округляет его, и выделывает из него образ человека красивого вида, чтобы поставить его в доме.
\end{tcolorbox}
\begin{tcolorbox}
\textsubscript{14} Он рубит себе кедры, берет сосну и дуб, которые выберет между деревьями в лесу, садит ясень, а дождь возращает его.
\end{tcolorbox}
\begin{tcolorbox}
\textsubscript{15} И это служит человеку топливом, и [часть] из этого употребляет он на то, чтобы ему было тепло, и разводит огонь, и печет хлеб. И из того же делает бога, и поклоняется ему, делает идола, и повергается перед ним.
\end{tcolorbox}
\begin{tcolorbox}
\textsubscript{16} Часть дерева сожигает в огне, другою частью варит мясо в пищу, жарит жаркое и ест досыта, а также греется и говорит: 'хорошо, я согрелся; почувствовал огонь'.
\end{tcolorbox}
\begin{tcolorbox}
\textsubscript{17} А из остатков от того делает бога, идола своего, поклоняется ему, повергается перед ним и молится ему, и говорит: 'спаси меня, ибо ты бог мой'.
\end{tcolorbox}
\begin{tcolorbox}
\textsubscript{18} Не знают и не разумеют они: Он закрыл глаза их, чтобы не видели, [и] сердца их, чтобы не разумели.
\end{tcolorbox}
\begin{tcolorbox}
\textsubscript{19} И не возьмет он этого к своему сердцу, и нет у него столько знания и смысла, чтобы сказать: 'половину его я сжег в огне и на угольях его испек хлеб, изжарил мясо и съел; а из остатка его сделаю ли я мерзость? буду ли поклоняться куску дерева?'
\end{tcolorbox}
\begin{tcolorbox}
\textsubscript{20} Он гоняется за пылью; обманутое сердце ввело его в заблуждение, и он не может освободить души своей и сказать: 'не обман ли в правой руке моей?'
\end{tcolorbox}
\begin{tcolorbox}
\textsubscript{21} Помни это, Иаков и Израиль, ибо ты раб Мой; Я образовал тебя: раб Мой ты, Израиль, не забывай Меня.
\end{tcolorbox}
\begin{tcolorbox}
\textsubscript{22} Изглажу беззакония твои, как туман, и грехи твои, как облако; обратись ко Мне, ибо Я искупил тебя.
\end{tcolorbox}
\begin{tcolorbox}
\textsubscript{23} Торжествуйте, небеса, ибо Господь соделал это. Восклицайте, глубины земли; шумите от радости, горы, лес и все деревья в нем; ибо искупил Господь Иакова и прославится в Израиле.
\end{tcolorbox}
\begin{tcolorbox}
\textsubscript{24} Так говорит Господь, искупивший тебя и образовавший тебя от утробы матерней: Я Господь, Который сотворил все, один распростер небеса и Своею силою разостлал землю,
\end{tcolorbox}
\begin{tcolorbox}
\textsubscript{25} Который делает ничтожными знамения лжепророков и обнаруживает безумие волшебников, мудрецов прогоняет назад и знание их делает глупостью,
\end{tcolorbox}
\begin{tcolorbox}
\textsubscript{26} Который утверждает слово раба Своего и приводит в исполнение изречение Своих посланников, Который говорит Иерусалиму: 'ты будешь населен', и городам Иудиным: 'вы будете построены, и развалины его Я восстановлю',
\end{tcolorbox}
\begin{tcolorbox}
\textsubscript{27} Который бездне говорит: 'иссохни!' и реки твои Я иссушу,
\end{tcolorbox}
\begin{tcolorbox}
\textsubscript{28} Который говорит о Кире: пастырь Мой, и он исполнит всю волю Мою и скажет Иерусалиму: 'ты будешь построен!' и храму: 'ты будешь основан!'
\end{tcolorbox}
\subsection{CHAPTER 45}
\begin{tcolorbox}
\textsubscript{1} Так говорит Господь помазаннику Своему Киру: Я держу тебя за правую руку, чтобы покорить тебе народы, и сниму поясы с чресл царей, чтоб отворялись для тебя двери, и ворота не затворялись;
\end{tcolorbox}
\begin{tcolorbox}
\textsubscript{2} Я пойду пред тобою и горы уровняю, медные двери сокрушу и запоры железные сломаю;
\end{tcolorbox}
\begin{tcolorbox}
\textsubscript{3} и отдам тебе хранимые во тьме сокровища и сокрытые богатства, дабы ты познал, что Я Господь, называющий тебя по имени, Бог Израилев.
\end{tcolorbox}
\begin{tcolorbox}
\textsubscript{4} Ради Иакова, раба Моего, и Израиля, избранного Моего, Я назвал тебя по имени, почтил тебя, хотя ты не знал Меня.
\end{tcolorbox}
\begin{tcolorbox}
\textsubscript{5} Я Господь, и нет иного; нет Бога кроме Меня; Я препоясал тебя, хотя ты не знал Меня,
\end{tcolorbox}
\begin{tcolorbox}
\textsubscript{6} дабы узнали от восхода солнца и от запада, что нет кроме Меня; Я Господь, и нет иного.
\end{tcolorbox}
\begin{tcolorbox}
\textsubscript{7} Я образую свет и творю тьму, делаю мир и произвожу бедствия; Я, Господь, делаю все это.
\end{tcolorbox}
\begin{tcolorbox}
\textsubscript{8} Кропите, небеса, свыше, и облака да проливают правду; да раскроется земля и приносит спасение, и да произрастает вместе правда. Я, Господь, творю это.
\end{tcolorbox}
\begin{tcolorbox}
\textsubscript{9} Горе тому, кто препирается с Создателем своим, черепок из черепков земных! Скажет ли глина горшечнику: 'что ты делаешь?' и твое дело [скажет ли о тебе]: 'у него нет рук?'
\end{tcolorbox}
\begin{tcolorbox}
\textsubscript{10} Горе тому, кто говорит отцу: 'зачем ты произвел [меня] на свет?', а матери: 'зачем ты родила [меня]?'
\end{tcolorbox}
\begin{tcolorbox}
\textsubscript{11} Так говорит Господь, Святый Израиля и Создатель его: вы спрашиваете Меня о будущем сыновей Моих и хотите Мне указывать в деле рук Моих?
\end{tcolorbox}
\begin{tcolorbox}
\textsubscript{12} Я создал землю и сотворил на ней человека; Я--Мои руки распростерли небеса, и всему воинству их дал закон Я.
\end{tcolorbox}
\begin{tcolorbox}
\textsubscript{13} Я воздвиг его в правде и уровняю все пути его. Он построит город Мой и отпустит пленных Моих, не за выкуп и не за дары, говорит Господь Саваоф.
\end{tcolorbox}
\begin{tcolorbox}
\textsubscript{14} Так говорит Господь: труды Египтян и торговля Ефиоплян, и Савейцы, люди рослые, к тебе перейдут и будут твоими; они последуют за тобою, в цепях придут и повергнутся пред тобою, и будут умолять тебя, [говоря]: у тебя только Бог, и нет иного Бога.
\end{tcolorbox}
\begin{tcolorbox}
\textsubscript{15} Истинно Ты Бог сокровенный, Бог Израилев, Спаситель.
\end{tcolorbox}
\begin{tcolorbox}
\textsubscript{16} Все они будут постыжены и посрамлены; вместе с ними со стыдом пойдут и все, делающие идолов.
\end{tcolorbox}
\begin{tcolorbox}
\textsubscript{17} Израиль же будет спасен спасением вечным в Господе; вы не будете постыжены и посрамлены во веки веков.
\end{tcolorbox}
\begin{tcolorbox}
\textsubscript{18} Ибо так говорит Господь, сотворивший небеса, Он, Бог, образовавший землю и создавший ее; Он утвердил ее, не напрасно сотворил ее; Он образовал ее для жительства: Я Господь, и нет иного.
\end{tcolorbox}
\begin{tcolorbox}
\textsubscript{19} Не тайно Я говорил, не в темном месте земли; не говорил Я племени Иакова: 'напрасно ищете Меня'. Я Господь, изрекающий правду, открывающий истину.
\end{tcolorbox}
\begin{tcolorbox}
\textsubscript{20} Соберитесь и придите, приблизьтесь все, уцелевшие из народов. Невежды те, которые носят деревянного своего идола и молятся богу, который не спасает.
\end{tcolorbox}
\begin{tcolorbox}
\textsubscript{21} Объявите и скажите, посоветовавшись между собою: кто возвестил это из древних времен, наперед сказал это? Не Я ли, Господь? и нет иного Бога кроме Меня, Бога праведного и спасающего нет кроме Меня.
\end{tcolorbox}
\begin{tcolorbox}
\textsubscript{22} Ко Мне обратитесь, и будете спасены, все концы земли, ибо я Бог, и нет иного.
\end{tcolorbox}
\begin{tcolorbox}
\textsubscript{23} Мною клянусь: из уст Моих исходит правда, слово неизменное, что предо Мною преклонится всякое колено, Мною будет клясться всякий язык.
\end{tcolorbox}
\begin{tcolorbox}
\textsubscript{24} Только у Господа, будут говорить о Мне, правда и сила; к Нему придут и устыдятся все, враждовавшие против Него.
\end{tcolorbox}
\begin{tcolorbox}
\textsubscript{25} Господом будет оправдано и прославлено все племя Израилево.
\end{tcolorbox}
\subsection{CHAPTER 46}
\begin{tcolorbox}
\textsubscript{1} Пал Вил, низвергся Нево; истуканы их--на скоте и вьючных животных; ваша ноша сделалась бременем для усталых животных.
\end{tcolorbox}
\begin{tcolorbox}
\textsubscript{2} Низверглись, пали вместе; не могли защитить носивших, и сами пошли в плен.
\end{tcolorbox}
\begin{tcolorbox}
\textsubscript{3} Послушайте меня, дом Иаковлев и весь остаток дома Израилева, принятые [Мною] от чрева, носимые Мною от утробы [матерней]:
\end{tcolorbox}
\begin{tcolorbox}
\textsubscript{4} и до старости вашей Я тот же буду, и до седины вашей Я же буду носить [вас]; Я создал и буду носить, поддерживать и охранять вас.
\end{tcolorbox}
\begin{tcolorbox}
\textsubscript{5} Кому уподобите Меня, и [с кем] сравните, и с кем сличите, чтобы мы были сходны?
\end{tcolorbox}
\begin{tcolorbox}
\textsubscript{6} Высыпают золото из кошелька и весят серебро на весах, и нанимают серебряника, чтобы он сделал из него бога; кланяются ему и повергаются перед ним;
\end{tcolorbox}
\begin{tcolorbox}
\textsubscript{7} поднимают его на плечи, несут его и ставят его на свое место; он стоит, с места своего не двигается; кричат к нему, --он не отвечает, не спасает от беды.
\end{tcolorbox}
\begin{tcolorbox}
\textsubscript{8} Вспомните это и покажите себя мужами; примите это, отступники, к сердцу;
\end{tcolorbox}
\begin{tcolorbox}
\textsubscript{9} вспомните прежде бывшее, от [начала] века, ибо Я Бог, и нет иного Бога, и нет подобного Мне.
\end{tcolorbox}
\begin{tcolorbox}
\textsubscript{10} Я возвещаю от начала, что будет в конце, и от древних времен то, что еще не сделалось, говорю: Мой совет состоится, и все, что Мне угодно, Я сделаю.
\end{tcolorbox}
\begin{tcolorbox}
\textsubscript{11} Я воззвал орла от востока, из дальней страны, исполнителя определения Моего. Я сказал, и приведу это в исполнение; предначертал, и сделаю.
\end{tcolorbox}
\begin{tcolorbox}
\textsubscript{12} Послушайте Меня, жестокие сердцем, далекие от правды:
\end{tcolorbox}
\begin{tcolorbox}
\textsubscript{13} Я приблизил правду Мою, она не далеко, и спасение Мое не замедлит; и дам Сиону спасение, Израилю славу Мою.
\end{tcolorbox}
\subsection{CHAPTER 47}
\begin{tcolorbox}
\textsubscript{1} Сойди и сядь на прах, девица, дочь Вавилона; сиди на земле: престола нет, дочь Халдеев, и вперед не будут называть тебя нежною и роскошною.
\end{tcolorbox}
\begin{tcolorbox}
\textsubscript{2} Возьми жернова и мели муку; сними покрывало твое, подбери подол, открой голени, переходи через реки:
\end{tcolorbox}
\begin{tcolorbox}
\textsubscript{3} откроется нагота твоя, и даже виден будет стыд твой. Совершу мщение и не пощажу никого.
\end{tcolorbox}
\begin{tcolorbox}
\textsubscript{4} Искупитель наш--Господь Саваоф имя Ему, Святый Израилев.
\end{tcolorbox}
\begin{tcolorbox}
\textsubscript{5} Сиди молча и уйди в темноту, дочь Халдеев: ибо вперед не будут называть тебя госпожею царств.
\end{tcolorbox}
\begin{tcolorbox}
\textsubscript{6} Я прогневался на народ Мой, уничижил наследие Мое и предал их в руки твои; [а] ты не оказала им милосердия, на старца налагала крайне тяжкое иго твое.
\end{tcolorbox}
\begin{tcolorbox}
\textsubscript{7} И ты говорила: 'вечно буду госпожею', а не представляла того в уме твоем, не помышляла, что будет после.
\end{tcolorbox}
\begin{tcolorbox}
\textsubscript{8} Но ныне выслушай это, изнеженная, живущая беспечно, говорящая в сердце своем: 'я, --и другой подобной мне нет; не буду сидеть вдовою и не буду знать потери детей'.
\end{tcolorbox}
\begin{tcolorbox}
\textsubscript{9} Но внезапно, в один день, придет к тебе то и другое, потеря детей и вдовство; в полной мере придут они на тебя, несмотря на множество чародейств твоих и на великую силу волшебств твоих.
\end{tcolorbox}
\begin{tcolorbox}
\textsubscript{10} Ибо ты надеялась на злодейство твое, говорила: 'никто не видит меня'. Мудрость твоя и знание твое--они сбили тебя с пути; и ты говорила в сердце твоем: 'я, и никто кроме меня'.
\end{tcolorbox}
\begin{tcolorbox}
\textsubscript{11} И придет на тебя бедствие: ты не узнаешь, откуда оно поднимется; и нападет на тебя беда, которой ты не в силах будешь отвратить, и внезапно придет на тебя пагуба, о которой ты и не думаешь.
\end{tcolorbox}
\begin{tcolorbox}
\textsubscript{12} Оставайся же с твоими волшебствами и со множеством чародейств твоих, которыми ты занималась от юности твоей: может быть, пособишь себе, может быть, устоишь.
\end{tcolorbox}
\begin{tcolorbox}
\textsubscript{13} Ты утомлена множеством советов твоих; пусть же выступят наблюдатели небес и звездочеты и предвещатели по новолуниям, и спасут тебя от того, что должно приключиться тебе.
\end{tcolorbox}
\begin{tcolorbox}
\textsubscript{14} Вот они, как солома: огонь сожег их, --не избавили души своей от пламени; не осталось угля, чтобы погреться, ни огня, чтобы посидеть перед ним.
\end{tcolorbox}
\begin{tcolorbox}
\textsubscript{15} Такими стали для тебя те, с которыми ты трудилась, с которыми вела торговлю от юности твоей. Каждый побрел в свою сторону; никто не спасает тебя.
\end{tcolorbox}
\subsection{CHAPTER 48}
\begin{tcolorbox}
\textsubscript{1} Слушайте это, дом Иакова, называющиеся именем Израиля и происшедшие от источника Иудина, клянущиеся именем Господа и исповедающие Бога Израилева, хотя не по истине и не по правде.
\end{tcolorbox}
\begin{tcolorbox}
\textsubscript{2} Ибо они называют себя [происходящими] от святого города и опираются на Бога Израилева; Господь Саваоф--имя Ему.
\end{tcolorbox}
\begin{tcolorbox}
\textsubscript{3} Прежнее Я задолго объявлял; из Моих уст выходило оно, и Я возвещал это и внезапно делал, и все сбывалось.
\end{tcolorbox}
\begin{tcolorbox}
\textsubscript{4} Я знал, что ты упорен, и что в шее твоей жилы железные, и лоб твой--медный;
\end{tcolorbox}
\begin{tcolorbox}
\textsubscript{5} поэтому и объявлял тебе задолго, прежде нежели это приходило, и предъявлял тебе, чтобы ты не сказал: 'идол мой сделал это, и истукан мой и изваянный мой повелел этому быть'.
\end{tcolorbox}
\begin{tcolorbox}
\textsubscript{6} Ты слышал, --посмотри на все это! и неужели вы не признаёте этого? А ныне Я возвещаю тебе новое и сокровенное, и ты не знал этого.
\end{tcolorbox}
\begin{tcolorbox}
\textsubscript{7} Оно произошло ныне, а не задолго и не за день, и ты не слыхал о том, чтобы ты не сказал: 'вот! я знал это'.
\end{tcolorbox}
\begin{tcolorbox}
\textsubscript{8} Ты и не слыхал и не знал об этом, и ухо твое не было прежде открыто; ибо Я знал, что ты поступишь вероломно, и от самого чрева [матернего] ты прозван отступником.
\end{tcolorbox}
\begin{tcolorbox}
\textsubscript{9} Ради имени Моего отлагал гнев Мой, и ради славы Моей удерживал Себя от истребления тебя.
\end{tcolorbox}
\begin{tcolorbox}
\textsubscript{10} Вот, Я расплавил тебя, но не как серебро; испытал тебя в горниле страдания.
\end{tcolorbox}
\begin{tcolorbox}
\textsubscript{11} Ради Себя, ради Себя Самого делаю это, --ибо какое было бы нарекание [на имя Мое]! славы Моей не дам иному.
\end{tcolorbox}
\begin{tcolorbox}
\textsubscript{12} Послушай Меня, Иаков и Израиль, призванный Мой: Я тот же, Я первый и Я последний.
\end{tcolorbox}
\begin{tcolorbox}
\textsubscript{13} Моя рука основала землю, и Моя десница распростерла небеса; призову их, и они предстанут вместе.
\end{tcolorbox}
\begin{tcolorbox}
\textsubscript{14} Соберитесь все и слушайте: кто между ними предсказал это? Господь возлюбил его, и он исполнит волю Его над Вавилоном и явит мышцу Его над Халдеями.
\end{tcolorbox}
\begin{tcolorbox}
\textsubscript{15} Я, Я сказал, и призвал его; Я привел его, и путь его будет благоуспешен.
\end{tcolorbox}
\begin{tcolorbox}
\textsubscript{16} Приступите ко Мне, слушайте это: Я и сначала говорил не тайно; с того времени, как это происходит, Я был там; и ныне послал Меня Господь Бог и Дух Его.
\end{tcolorbox}
\begin{tcolorbox}
\textsubscript{17} Так говорит Господь, Искупитель твой, Святый Израилев: Я Господь, Бог твой, научающий тебя полезному, ведущий тебя по тому пути, по которому должно тебе идти.
\end{tcolorbox}
\begin{tcolorbox}
\textsubscript{18} О, если бы ты внимал заповедям Моим! тогда мир твой был бы как река, и правда твоя--как волны морские.
\end{tcolorbox}
\begin{tcolorbox}
\textsubscript{19} И семя твое было бы как песок, и происходящие из чресл твоих--как песчинки: не изгладилось бы, не истребилось бы имя его предо Мною.
\end{tcolorbox}
\begin{tcolorbox}
\textsubscript{20} Выходите из Вавилона, бегите от Халдеев, со гласом радости возвещайте и проповедуйте это, распространяйте эту весть до пределов земли; говорите: 'Господь искупил раба Своего Иакова'.
\end{tcolorbox}
\begin{tcolorbox}
\textsubscript{21} И не жаждут они в пустынях, чрез которые Он ведет их: Он источает им воду из камня; рассекает скалу, и льются воды.
\end{tcolorbox}
\begin{tcolorbox}
\textsubscript{22} Нечестивым же нет мира, говорит Господь.
\end{tcolorbox}
\subsection{CHAPTER 49}
\begin{tcolorbox}
\textsubscript{1} Слушайте Меня, острова, и внимайте, народы дальние: Господь призвал Меня от чрева, от утробы матери Моей называл имя Мое;
\end{tcolorbox}
\begin{tcolorbox}
\textsubscript{2} и соделал уста Мои как острый меч; тенью руки Своей покрывал Меня, и соделал Меня стрелою изостренною; в колчане Своем хранил Меня;
\end{tcolorbox}
\begin{tcolorbox}
\textsubscript{3} и сказал Мне: Ты раб Мой, Израиль, в Тебе Я прославлюсь.
\end{tcolorbox}
\begin{tcolorbox}
\textsubscript{4} А Я сказал: напрасно Я трудился, ни на что и вотще истощал силу Свою. Но Мое право у Господа, и награда Моя у Бога Моего.
\end{tcolorbox}
\begin{tcolorbox}
\textsubscript{5} И ныне говорит Господь, образовавший Меня от чрева в раба Себе, чтобы обратить к Нему Иакова и чтобы Израиль собрался к Нему; Я почтен в очах Господа, и Бог Мой--сила Моя.
\end{tcolorbox}
\begin{tcolorbox}
\textsubscript{6} И Он сказал: мало того, что Ты будешь рабом Моим для восстановления колен Иаковлевых и для возвращения остатков Израиля, но Я сделаю Тебя светом народов, чтобы спасение Мое простерлось до концов земли.
\end{tcolorbox}
\begin{tcolorbox}
\textsubscript{7} Так говорит Господь, Искупитель Израиля, Святый Его, презираемому всеми, поносимому народом, рабу властелинов: цари увидят, и встанут; князья поклонятся ради Господа, Который верен, ради Святаго Израилева, Который избрал Тебя.
\end{tcolorbox}
\begin{tcolorbox}
\textsubscript{8} Так говорит Господь: во время благоприятное Я услышал Тебя, и в день спасения помог Тебе; и Я буду охранять Тебя, и сделаю Тебя заветом народа, чтобы восстановить землю, чтобы возвратить наследникам наследия опустошенные,
\end{tcolorbox}
\begin{tcolorbox}
\textsubscript{9} сказать узникам: 'выходите', и тем, которые во тьме: 'покажитесь'. Они при дорогах будут пасти, и по всем холмам будут пажити их;
\end{tcolorbox}
\begin{tcolorbox}
\textsubscript{10} не будут терпеть голода и жажды, и не поразит их зной и солнце; ибо Милующий их будет вести их и приведет их к источникам вод.
\end{tcolorbox}
\begin{tcolorbox}
\textsubscript{11} И все горы Мои сделаю путем, и дороги Мои будут подняты.
\end{tcolorbox}
\begin{tcolorbox}
\textsubscript{12} Вот, одни придут издалека; и вот, одни от севера и моря, а другие из земли Синим.
\end{tcolorbox}
\begin{tcolorbox}
\textsubscript{13} Радуйтесь, небеса, и веселись, земля, и восклицайте, горы, от радости; ибо утешил Господь народ Свой и помиловал страдальцев Своих.
\end{tcolorbox}
\begin{tcolorbox}
\textsubscript{14} А Сион говорил: 'оставил меня Господь, и Бог мой забыл меня!'
\end{tcolorbox}
\begin{tcolorbox}
\textsubscript{15} Забудет ли женщина грудное дитя свое, чтобы не пожалеть сына чрева своего? но если бы и она забыла, то Я не забуду тебя.
\end{tcolorbox}
\begin{tcolorbox}
\textsubscript{16} Вот, Я начертал тебя на дланях [Моих]; стены твои всегда предо Мною.
\end{tcolorbox}
\begin{tcolorbox}
\textsubscript{17} Сыновья твои поспешат [к тебе], а разорители и опустошители твои уйдут от тебя.
\end{tcolorbox}
\begin{tcolorbox}
\textsubscript{18} Возведи очи твои и посмотри вокруг, --все они собираются, идут к тебе. Живу Я! говорит Господь, --всеми ими ты облечешься, как убранством, и нарядишься ими, как невеста.
\end{tcolorbox}
\begin{tcolorbox}
\textsubscript{19} Ибо развалины твои и пустыни твои, и разоренная земля твоя будут теперь слишком тесны для жителей, и поглощавшие тебя удалятся от тебя.
\end{tcolorbox}
\begin{tcolorbox}
\textsubscript{20} Дети, которые будут у тебя после потери прежних, будут говорить вслух тебе: 'тесно для меня место; уступи мне, чтобы я мог жить'.
\end{tcolorbox}
\begin{tcolorbox}
\textsubscript{21} И ты скажешь в сердце твоем: кто мне родил их? я была бездетна и бесплодна, отведена в плен и удалена; кто же возрастил их? вот, я оставалась одинокою; где же они были?
\end{tcolorbox}
\begin{tcolorbox}
\textsubscript{22} Так говорит Господь Бог: вот, Я подниму руку Мою к народам, и выставлю знамя Мое племенам, и принесут сыновей твоих на руках и дочерей твоих на плечах.
\end{tcolorbox}
\begin{tcolorbox}
\textsubscript{23} И будут цари питателями твоими, и царицы их кормилицами твоими; лицом до земли будут кланяться тебе и лизать прах ног твоих, и узнаешь, что Я Господь, что надеющиеся на Меня не постыдятся.
\end{tcolorbox}
\begin{tcolorbox}
\textsubscript{24} Может ли быть отнята у сильного добыча, и могут ли быть отняты у победителя взятые в плен?
\end{tcolorbox}
\begin{tcolorbox}
\textsubscript{25} Да! так говорит Господь: и плененные сильным будут отняты, и добыча тирана будет избавлена; потому что Я буду состязаться с противниками твоими и сыновей твоих Я спасу;
\end{tcolorbox}
\begin{tcolorbox}
\textsubscript{26} и притеснителей твоих накормлю собственною их плотью, и они будут упоены кровью своею, как молодым вином; и всякая плоть узнает, что Я Господь, Спаситель твой и Искупитель твой, Сильный Иаковлев.
\end{tcolorbox}
\subsection{CHAPTER 50}
\begin{tcolorbox}
\textsubscript{1} Так говорит Господь: где разводное письмо вашей матери, с которым Я отпустил ее? или которому из Моих заимодавцев Я продал вас? Вот, вы проданы за грехи ваши, и за преступления ваши отпущена мать ваша.
\end{tcolorbox}
\begin{tcolorbox}
\textsubscript{2} Почему, когда Я приходил, никого не было, и когда Я звал, никто не отвечал? Разве рука Моя коротка стала для того, чтобы избавлять, или нет силы во Мне, чтобы спасать? Вот, прещением Моим Я иссушаю море, превращаю реки в пустыню; рыбы в них гниют от недостатка воды и умирают от жажды.
\end{tcolorbox}
\begin{tcolorbox}
\textsubscript{3} Я облекаю небеса мраком, и вретище делаю покровом их.
\end{tcolorbox}
\begin{tcolorbox}
\textsubscript{4} Господь Бог дал Мне язык мудрых, чтобы Я мог словом подкреплять изнемогающего; каждое утро Он пробуждает, пробуждает ухо Мое, чтобы Я слушал, подобно учащимся.
\end{tcolorbox}
\begin{tcolorbox}
\textsubscript{5} Господь Бог открыл Мне ухо, и Я не воспротивился, не отступил назад.
\end{tcolorbox}
\begin{tcolorbox}
\textsubscript{6} Я предал хребет Мой биющим и ланиты Мои поражающим; лица Моего не закрывал от поруганий и оплевания.
\end{tcolorbox}
\begin{tcolorbox}
\textsubscript{7} И Господь Бог помогает Мне: поэтому Я не стыжусь, поэтому Я держу лице Мое, как кремень, и знаю, что не останусь в стыде.
\end{tcolorbox}
\begin{tcolorbox}
\textsubscript{8} Близок оправдывающий Меня: кто хочет состязаться со Мною? станем вместе. Кто хочет судиться со Мною? пусть подойдет ко Мне.
\end{tcolorbox}
\begin{tcolorbox}
\textsubscript{9} Вот, Господь Бог помогает Мне: кто осудит Меня? Вот, все они, как одежда, обветшают; моль съест их.
\end{tcolorbox}
\begin{tcolorbox}
\textsubscript{10} Кто из вас боится Господа, слушается гласа Раба Его? Кто ходит во мраке, без света, да уповает на имя Господа и да утверждается в Боге своем.
\end{tcolorbox}
\begin{tcolorbox}
\textsubscript{11} Вот, все вы, которые возжигаете огонь, вооруженные зажигательными стрелами, --идите в пламень огня вашего и стрел, раскаленных вами! Это будет вам от руки Моей; в мучении умрете.
\end{tcolorbox}
\subsection{CHAPTER 51}
\begin{tcolorbox}
\textsubscript{1} Послушайте Меня, стремящиеся к правде, ищущие Господа! Взгляните на скалу, из которой вы иссечены, в глубину рва, изкоторого вы извлечены.
\end{tcolorbox}
\begin{tcolorbox}
\textsubscript{2} Посмотрите на Авраама, отца вашего, и на Сарру, родившуювас: ибо Я призвал его одного и благословил его, и размножил его.
\end{tcolorbox}
\begin{tcolorbox}
\textsubscript{3} Так, Господь утешит Сион, утешит все развалины его и сделаетпустыни его, как рай, и степь его, как сад Господа; радость ивеселие будет в нем, славословие и песнопение.
\end{tcolorbox}
\begin{tcolorbox}
\textsubscript{4} Послушайте Меня, народ Мой, и племя Мое, приклоните ухо Комне! ибо от Меня произойдет закон, и суд Мой поставлю во свет длянародов.
\end{tcolorbox}
\begin{tcolorbox}
\textsubscript{5} Правда Моя близка; спасение Мое восходит, и мышца Моя будетсудить народы; острова будут уповать на Меня и надеяться на Мышцумою.
\end{tcolorbox}
\begin{tcolorbox}
\textsubscript{6} Поднимите глаза ваши к небесам, и посмотрите на землю вниз: ибо небеса исчезнут, как дым, и земля обветшает, как одежда, ижители ее также вымрут; а Мое спасение пребудет вечным, и Правдамоя не престанет.
\end{tcolorbox}
\begin{tcolorbox}
\textsubscript{7} Послушайте Меня, знающие правду, народ, у которого в сердцезакон Мой! Не бойтесь поношения от людей, и злословия их нестрашитесь.
\end{tcolorbox}
\begin{tcolorbox}
\textsubscript{8} Ибо, как одежду, съест их моль и, как волну, съест их червь; а правда Моя пребудет вовек, и спасение Мое г в роды родов.
\end{tcolorbox}
\begin{tcolorbox}
\textsubscript{9} Восстань, восстань, облекись крепостью, мышца Господня! Восстань, как в дни древние, в роды давние! Не ты ли Сразилараава, поразила крокодила?
\end{tcolorbox}
\begin{tcolorbox}
\textsubscript{10} Не ты ли иссушила море, воды великой бездны, превратилаглубины моря в дорогу, чтобы прошли искупленные?
\end{tcolorbox}
\begin{tcolorbox}
\textsubscript{11} И возвратятся избавленные Господом и придут на Сион спением, и радость вечная над головою их; они найдут радость ивеселье: печаль и вздохи удалятся.
\end{tcolorbox}
\begin{tcolorbox}
\textsubscript{12} Я, Я Сам г Утешитель ваш. Кто ты, что боишься человека, который умирает, и сына человеческого, который то же, что трава,
\end{tcolorbox}
\begin{tcolorbox}
\textsubscript{13} и забываешь Господа, Творца своего, распростершего небеса иосновавшего землю; и непрестанно, всякий день страшишься яростипритеснителя, как бы он готов был истребить? Но где яростьпритеснителя?
\end{tcolorbox}
\begin{tcolorbox}
\textsubscript{14} Скоро освобожден будет пленный, и не умрет в яме и не будетнуждаться в хлебе.
\end{tcolorbox}
\begin{tcolorbox}
\textsubscript{15} Я Господь, Бог твой, возмущающий море, так что волны егоревут: Господь Саваоф г имя Его.
\end{tcolorbox}
\begin{tcolorbox}
\textsubscript{16} И Я вложу слова Мои в уста твои, и тенью руки Моей покроютебя, чтобы устроить небеса и утвердить землю и сказать Сиону: 'ты Мой народ'.
\end{tcolorbox}
\begin{tcolorbox}
\textsubscript{17} Воспряни, воспряни, восстань, Иерусалим, ты, который изруки Господа выпил чашу ярости Его, выпил до дна чашу опьянения, осушил.
\end{tcolorbox}
\begin{tcolorbox}
\textsubscript{18} Некому было вести его из всех сыновей, рожденных им, инекому было поддержать его за руку из всех сыновей, [которых] онвозрастил.
\end{tcolorbox}
\begin{tcolorbox}
\textsubscript{19} Тебя постигли два [бедствия], кто пожалеет о тебе? гопустошение и истребление, голод и меч: кем я утешу тебя?
\end{tcolorbox}
\begin{tcolorbox}
\textsubscript{20} Сыновья твои изнемогли, лежат по углам всех улиц, как сернав тенетах, исполненные гнева Господа, прещения Бога твоего.
\end{tcolorbox}
\begin{tcolorbox}
\textsubscript{21} Итак выслушай это, страдалец и опьяневший, но не от вина.
\end{tcolorbox}
\begin{tcolorbox}
\textsubscript{22} Так говорит Господь твой, Господь и Бог твой, отмщающий за Свой народ: вот, Я беру из руки твоей чашу опьянения, дрожжи изчаши ярости Моей: ты не будешь уже пить их,
\end{tcolorbox}
\begin{tcolorbox}
\textsubscript{23} и подам ее в руки мучителям твоим, которые говорили тебе: 'пади ниц, чтобы нам пройти по тебе'; и ты хребет твой делал какбы землею и улицею для проходящих.
\end{tcolorbox}
\subsection{CHAPTER 52}
\begin{tcolorbox}
\textsubscript{1} Восстань, восстань, облекись в силу твою, Сион! Облекись в одежды величия твоего, Иерусалим, город святый! ибо уже не будет более входить в тебя необрезанный и нечистый.
\end{tcolorbox}
\begin{tcolorbox}
\textsubscript{2} Отряси с себя прах; встань, пленный Иерусалим! сними цепи с шеи твоей, пленная дочь Сиона!
\end{tcolorbox}
\begin{tcolorbox}
\textsubscript{3} ибо так говорит Господь: за ничто были вы проданы, и без серебра будете выкуплены;
\end{tcolorbox}
\begin{tcolorbox}
\textsubscript{4} ибо так говорит Господь Бог: народ Мой ходил прежде в Египет, чтобы там пожить, и Ассур теснил его ни за что.
\end{tcolorbox}
\begin{tcolorbox}
\textsubscript{5} И теперь что у Меня здесь? говорит Господь; народ Мой взят даром, властители их неистовствуют, говорит Господь, и постоянно, всякий день имя Мое бесславится.
\end{tcolorbox}
\begin{tcolorbox}
\textsubscript{6} Поэтому народ Мой узнает имя Мое; поэтому [узнает] в тот день, что Я тот же, Который сказал: 'вот Я!'
\end{tcolorbox}
\begin{tcolorbox}
\textsubscript{7} Как прекрасны на горах ноги благовестника, возвещающего мир, благовествующего радость, проповедующего спасение, говорящего Сиону: 'воцарился Бог твой!'
\end{tcolorbox}
\begin{tcolorbox}
\textsubscript{8} Голос сторожей твоих--они возвысили голос, и все вместе ликуют, ибо своими глазами видят, что Господь возвращается в Сион.
\end{tcolorbox}
\begin{tcolorbox}
\textsubscript{9} Торжествуйте, пойте вместе, развалины Иерусалима, ибо утешил Господь народ Свой, искупил Иерусалим.
\end{tcolorbox}
\begin{tcolorbox}
\textsubscript{10} Обнажил Господь святую мышцу Свою пред глазами всех народов; и все концы земли увидят спасение Бога нашего.
\end{tcolorbox}
\begin{tcolorbox}
\textsubscript{11} Идите, идите, выходите оттуда; не касайтесь нечистого; выходите из среды его, очистите себя, носящие сосуды Господни!
\end{tcolorbox}
\begin{tcolorbox}
\textsubscript{12} ибо вы выйдете неторопливо, и не побежите; потому что впереди вас пойдет Господь, и Бог Израилев будет стражем позади вас.
\end{tcolorbox}
\begin{tcolorbox}
\textsubscript{13} Вот, раб Мой будет благоуспешен, возвысится и вознесется, и возвеличится.
\end{tcolorbox}
\begin{tcolorbox}
\textsubscript{14} Как многие изумлялись, [смотря] на Тебя, --столько был обезображен паче всякого человека лик Его, и вид Его--паче сынов человеческих!
\end{tcolorbox}
\begin{tcolorbox}
\textsubscript{15} Так многие народы приведет Он в изумление; цари закроют пред Ним уста свои, ибо они увидят то, о чем не было говорено им, и узнают то, чего не слыхали.
\end{tcolorbox}
\subsection{CHAPTER 53}
\begin{tcolorbox}
\textsubscript{1} Кто поверил слышанному от нас, и кому открылась мышца Господня?
\end{tcolorbox}
\begin{tcolorbox}
\textsubscript{2} Ибо Он взошел пред Ним, как отпрыск и как росток из сухой земли; нет в Нем ни вида, ни величия; и мы видели Его, и не было в Нем вида, который привлекал бы нас к Нему.
\end{tcolorbox}
\begin{tcolorbox}
\textsubscript{3} Он был презрен и умален пред людьми, муж скорбей и изведавший болезни, и мы отвращали от Него лице свое; Он был презираем, и мы ни во что ставили Его.
\end{tcolorbox}
\begin{tcolorbox}
\textsubscript{4} Но Он взял на Себя наши немощи и понес наши болезни; а мы думали, [что] Он был поражаем, наказуем и уничижен Богом.
\end{tcolorbox}
\begin{tcolorbox}
\textsubscript{5} Но Он изъязвлен был за грехи наши и мучим за беззакония наши; наказание мира нашего [было] на Нем, и ранами Его мы исцелились.
\end{tcolorbox}
\begin{tcolorbox}
\textsubscript{6} Все мы блуждали, как овцы, совратились каждый на свою дорогу: и Господь возложил на Него грехи всех нас.
\end{tcolorbox}
\begin{tcolorbox}
\textsubscript{7} Он истязуем был, но страдал добровольно и не открывал уст Своих; как овца, веден был Он на заклание, и как агнец пред стригущим его безгласен, так Он не отверзал уст Своих.
\end{tcolorbox}
\begin{tcolorbox}
\textsubscript{8} От уз и суда Он был взят; но род Его кто изъяснит? ибо Он отторгнут от земли живых; за преступления народа Моего претерпел казнь.
\end{tcolorbox}
\begin{tcolorbox}
\textsubscript{9} Ему назначали гроб со злодеями, но Он погребен у богатого, потому что не сделал греха, и не было лжи в устах Его.
\end{tcolorbox}
\begin{tcolorbox}
\textsubscript{10} Но Господу угодно было поразить Его, и Он предал Его мучению; когда же душа Его принесет жертву умилостивления, Он узрит потомство долговечное, и воля Господня благоуспешно будет исполняться рукою Его.
\end{tcolorbox}
\begin{tcolorbox}
\textsubscript{11} На подвиг души Своей Он будет смотреть с довольством; чрез познание Его Он, Праведник, Раб Мой, оправдает многих и грехи их на Себе понесет.
\end{tcolorbox}
\begin{tcolorbox}
\textsubscript{12} Посему Я дам Ему часть между великими, и с сильными будет делить добычу, за то, что предал душу Свою на смерть, и к злодеям причтен был, тогда как Он понес на Себе грех многих и за преступников сделался ходатаем.
\end{tcolorbox}
\subsection{CHAPTER 54}
\begin{tcolorbox}
\textsubscript{1} Возвеселись, неплодная, нерождающая; воскликни и возгласи, немучившаяся родами; потому что у оставленной гораздо более детей, нежели у имеющей мужа, говорит Господь.
\end{tcolorbox}
\begin{tcolorbox}
\textsubscript{2} Распространи место шатра твоего, расширь покровы жилищ твоих; не стесняйся, пусти длиннее верви твои и утверди колья твои;
\end{tcolorbox}
\begin{tcolorbox}
\textsubscript{3} ибо ты распространишься направо и налево, и потомство твое завладеет народами и населит опустошенные города.
\end{tcolorbox}
\begin{tcolorbox}
\textsubscript{4} Не бойся, ибо не будешь постыжена; не смущайся, ибо не будешь в поругании: ты забудешь посрамление юности твоей и не будешь более вспоминать о бесславии вдовства твоего.
\end{tcolorbox}
\begin{tcolorbox}
\textsubscript{5} Ибо твой Творец есть супруг твой; Господь Саваоф--имя Его; и Искупитель твой--Святый Израилев: Богом всей земли назовется Он.
\end{tcolorbox}
\begin{tcolorbox}
\textsubscript{6} Ибо как жену, оставленную и скорбящую духом, призывает тебя Господь, и [как] жену юности, которая была отвержена, говорит Бог твой.
\end{tcolorbox}
\begin{tcolorbox}
\textsubscript{7} На малое время Я оставил тебя, но с великою милостью восприму тебя.
\end{tcolorbox}
\begin{tcolorbox}
\textsubscript{8} В жару гнева Я сокрыл от тебя лице Мое на время, но вечною милостью помилую тебя, говорит Искупитель твой, Господь.
\end{tcolorbox}
\begin{tcolorbox}
\textsubscript{9} Ибо это для Меня, как воды Ноя: как Я поклялся, что воды Ноя не придут более на землю, так поклялся не гневаться на тебя и не укорять тебя.
\end{tcolorbox}
\begin{tcolorbox}
\textsubscript{10} Горы сдвинутся и холмы поколеблются, --а милость Моя не отступит от тебя, и завет мира Моего не поколеблется, говорит милующий тебя Господь.
\end{tcolorbox}
\begin{tcolorbox}
\textsubscript{11} Бедная, бросаемая бурею, безутешная! Вот, Я положу камни твои на рубине и сделаю основание твое из сапфиров;
\end{tcolorbox}
\begin{tcolorbox}
\textsubscript{12} и сделаю окна твои из рубинов и ворота твои--из жемчужин, и всю ограду твою--из драгоценных камней.
\end{tcolorbox}
\begin{tcolorbox}
\textsubscript{13} И все сыновья твои будут научены Господом, и великий мир будет у сыновей твоих.
\end{tcolorbox}
\begin{tcolorbox}
\textsubscript{14} Ты утвердишься правдою, будешь далека от угнетения, ибо тебе бояться нечего, и от ужаса, ибо он не приблизится к тебе.
\end{tcolorbox}
\begin{tcolorbox}
\textsubscript{15} Вот, будут вооружаться [против тебя], но не от Меня; кто бы ни вооружился против тебя, падет.
\end{tcolorbox}
\begin{tcolorbox}
\textsubscript{16} Вот, Я сотворил кузнеца, который раздувает угли в огне и производит орудие для своего дела, --и Я творю губителя для истребления.
\end{tcolorbox}
\begin{tcolorbox}
\textsubscript{17} Ни одно орудие, сделанное против тебя, не будет успешно; и всякий язык, который будет состязаться с тобою на суде, --ты обвинишь. Это есть наследие рабов Господа, оправдание их от Меня, говорит Господь.
\end{tcolorbox}
\subsection{CHAPTER 55}
\begin{tcolorbox}
\textsubscript{1} Жаждущие! идите все к водам; даже и вы, у которых нет серебра, идите, покупайте и ешьте; идите, покупайте без серебра и без платы вино и молоко.
\end{tcolorbox}
\begin{tcolorbox}
\textsubscript{2} Для чего вам отвешивать серебро за то, что не хлеб, и трудовое свое за то, что не насыщает? Послушайте Меня внимательно и вкушайте благо, и душа ваша да насладится туком.
\end{tcolorbox}
\begin{tcolorbox}
\textsubscript{3} Приклоните ухо ваше и придите ко Мне: послушайте, и жива будет душа ваша, --и дам вам завет вечный, неизменные милости, [обещанные] Давиду.
\end{tcolorbox}
\begin{tcolorbox}
\textsubscript{4} Вот, Я дал Его свидетелем для народов, вождем и наставником народам.
\end{tcolorbox}
\begin{tcolorbox}
\textsubscript{5} Вот, ты призовешь народ, которого ты не знал, и народы, которые тебя не знали, поспешат к тебе ради Господа Бога твоего и ради Святаго Израилева, ибо Он прославил тебя.
\end{tcolorbox}
\begin{tcolorbox}
\textsubscript{6} Ищите Господа, когда можно найти Его; призывайте Его, когда Он близко.
\end{tcolorbox}
\begin{tcolorbox}
\textsubscript{7} Да оставит нечестивый путь свой и беззаконник--помыслы свои, и да обратится к Господу, и Он помилует его, и к Богу нашему, ибо Он многомилостив.
\end{tcolorbox}
\begin{tcolorbox}
\textsubscript{8} Мои мысли--не ваши мысли, ни ваши пути--пути Мои, говорит Господь.
\end{tcolorbox}
\begin{tcolorbox}
\textsubscript{9} Но как небо выше земли, так пути Мои выше путей ваших, и мысли Мои выше мыслей ваших.
\end{tcolorbox}
\begin{tcolorbox}
\textsubscript{10} Как дождь и снег нисходит с неба и туда не возвращается, но напояет землю и делает ее способною рождать и произращать, чтобы она давала семя тому, кто сеет, и хлеб тому, кто ест, --
\end{tcolorbox}
\begin{tcolorbox}
\textsubscript{11} так и слово Мое, которое исходит из уст Моих, --оно не возвращается ко Мне тщетным, но исполняет то, что Мне угодно, и совершает то, для чего Я послал его.
\end{tcolorbox}
\begin{tcolorbox}
\textsubscript{12} Итак вы выйдете с веселием и будете провожаемы с миром; горы и холмы будут петь пред вами песнь, и все дерева в поле рукоплескать вам.
\end{tcolorbox}
\begin{tcolorbox}
\textsubscript{13} Вместо терновника вырастет кипарис; вместо крапивы возрастет мирт; и это будет во славу Господа, в знамение вечное, несокрушимое.
\end{tcolorbox}
\subsection{CHAPTER 56}
\begin{tcolorbox}
\textsubscript{1} Так говорит Господь: сохраняйте суд и делайте правду; ибо близко спасение Мое и откровение правды Моей.
\end{tcolorbox}
\begin{tcolorbox}
\textsubscript{2} Блажен муж, который делает это, и сын человеческий, который крепко держится этого, который хранит субботу от осквернения и оберегает руку свою, чтобы не сделать никакого зла.
\end{tcolorbox}
\begin{tcolorbox}
\textsubscript{3} Да не говорит сын иноплеменника, присоединившийся к Господу: 'Господь совсем отделил меня от Своего народа', и да не говорит евнух: 'вот я сухое дерево'.
\end{tcolorbox}
\begin{tcolorbox}
\textsubscript{4} Ибо Господь так говорит об евнухах: которые хранят Мои субботы и избирают угодное Мне, и крепко держатся завета Моего, --
\end{tcolorbox}
\begin{tcolorbox}
\textsubscript{5} тем дам Я в доме Моем и в стенах Моих место и имя лучшее, нежели сыновьям и дочерям; дам им вечное имя, которое не истребится.
\end{tcolorbox}
\begin{tcolorbox}
\textsubscript{6} И сыновей иноплеменников, присоединившихся к Господу, чтобы служить Ему и любить имя Господа, быть рабами Его, всех, хранящих субботу от осквернения ее и твердо держащихся завета Моего,
\end{tcolorbox}
\begin{tcolorbox}
\textsubscript{7} Я приведу на святую гору Мою и обрадую их в Моем доме молитвы; всесожжения их и жертвы их [будут] благоприятны на жертвеннике Моем, ибо дом Мой назовется домом молитвы для всех народов.
\end{tcolorbox}
\begin{tcolorbox}
\textsubscript{8} Господь Бог, собирающий рассеянных Израильтян, говорит: к собранным у него Я буду еще собирать других.
\end{tcolorbox}
\begin{tcolorbox}
\textsubscript{9} Все звери полевые, все звери лесные! идите есть.
\end{tcolorbox}
\begin{tcolorbox}
\textsubscript{10} Стражи их слепы все и невежды: все они немые псы, не могущие лаять, бредящие лежа, любящие спать.
\end{tcolorbox}
\begin{tcolorbox}
\textsubscript{11} И это псы, жадные душею, не знающие сытости; и это пастыри бессмысленные: все смотрят на свою дорогу, каждый до последнего, на свою корысть;
\end{tcolorbox}
\begin{tcolorbox}
\textsubscript{12} приходите, [говорят], я достану вина, и мы напьемся сикеры; и завтра то же будет, что сегодня, да еще и больше.
\end{tcolorbox}
\subsection{CHAPTER 57}
\begin{tcolorbox}
\textsubscript{1} Праведник умирает, и никто не принимает этого к сердцу; и мужи благочестивые восхищаются [от земли], и никто не помыслит, что праведник восхищается от зла.
\end{tcolorbox}
\begin{tcolorbox}
\textsubscript{2} Он отходит к миру; ходящие прямым путем будут покоиться на ложах своих.
\end{tcolorbox}
\begin{tcolorbox}
\textsubscript{3} Но приблизьтесь сюда вы, сыновья чародейки, семя прелюбодея и блудницы!
\end{tcolorbox}
\begin{tcolorbox}
\textsubscript{4} Над кем вы глумитесь? против кого расширяете рот, высовываете язык? не дети ли вы преступления, семя лжи,
\end{tcolorbox}
\begin{tcolorbox}
\textsubscript{5} разжигаемые похотью к идолам под каждым ветвистым деревом, заколающие детей при ручьях, между расселинами скал?
\end{tcolorbox}
\begin{tcolorbox}
\textsubscript{6} В гладких камнях ручьев доля твоя; они, они жребий твой; им ты делаешь возлияние и приносишь жертвы: могу ли Я быть доволен этим?
\end{tcolorbox}
\begin{tcolorbox}
\textsubscript{7} На высокой и выдающейся горе ты ставишь ложе твое и туда восходишь приносить жертву.
\end{tcolorbox}
\begin{tcolorbox}
\textsubscript{8} За дверью также и за косяками ставишь памяти твои; ибо, отвратившись от Меня, ты обнажаешься и восходишь; распространяешь ложе твое и договариваешься с теми из них, с которыми любишь лежать, высматриваешь место.
\end{tcolorbox}
\begin{tcolorbox}
\textsubscript{9} Ты ходила также к царю с благовонною мастью и умножила масти твои, и далеко посылала послов твоих, и унижалась до преисподней.
\end{tcolorbox}
\begin{tcolorbox}
\textsubscript{10} От долгого пути твоего утомлялась, но не говорила: 'надежда потеряна!'; все еще находила живость в руке твоей, и потому не чувствовала ослабления.
\end{tcolorbox}
\begin{tcolorbox}
\textsubscript{11} Кого же ты испугалась и устрашилась, что сделалась неверною и Меня перестала помнить и хранить в твоем сердце? не оттого ли, что Я молчал, и притом долго, ты перестала бояться Меня?
\end{tcolorbox}
\begin{tcolorbox}
\textsubscript{12} Я покажу правду твою и дела твои, --и они будут не в пользу тебе.
\end{tcolorbox}
\begin{tcolorbox}
\textsubscript{13} Когда ты будешь вопить, спасет ли тебя сборище твое? --всех их унесет ветер, развеет дуновение; а надеющийся на Меня наследует землю и будет владеть святою горою Моею.
\end{tcolorbox}
\begin{tcolorbox}
\textsubscript{14} И сказал: поднимайте, поднимайте, ровняйте путь, убирайте преграду с пути народа Моего.
\end{tcolorbox}
\begin{tcolorbox}
\textsubscript{15} Ибо так говорит Высокий и Превознесенный, вечно Живущий, --Святый имя Его: Я живу на высоте [небес] и во святилище, и также с сокрушенными и смиренными духом, чтобы оживлять дух смиренных и оживлять сердца сокрушенных.
\end{tcolorbox}
\begin{tcolorbox}
\textsubscript{16} Ибо не вечно буду Я вести тяжбу и не до конца гневаться; иначе изнеможет предо Мною дух и всякое дыхание, Мною сотворенное.
\end{tcolorbox}
\begin{tcolorbox}
\textsubscript{17} За грех корыстолюбия его Я гневался и поражал его, скрывал лице и негодовал; но он, отвратившись, пошел по пути своего сердца.
\end{tcolorbox}
\begin{tcolorbox}
\textsubscript{18} Я видел пути его, и исцелю его, и буду водить его и утешать его и сетующих его.
\end{tcolorbox}
\begin{tcolorbox}
\textsubscript{19} Я исполню слово: мир, мир дальнему и ближнему, говорит Господь, и исцелю его.
\end{tcolorbox}
\begin{tcolorbox}
\textsubscript{20} А нечестивые--как море взволнованное, которое не может успокоиться и которого воды выбрасывают ил и грязь.
\end{tcolorbox}
\begin{tcolorbox}
\textsubscript{21} Нет мира нечестивым, говорит Бог мой.
\end{tcolorbox}
\subsection{CHAPTER 58}
\begin{tcolorbox}
\textsubscript{1} Взывай громко, не удерживайся; возвысь голос твой, подобно трубе, и укажи народу Моему на беззакония его, и дому Иаковлеву--на грехи его.
\end{tcolorbox}
\begin{tcolorbox}
\textsubscript{2} Они каждый день ищут Меня и хотят знать пути Мои, как бы народ, поступающий праведно и не оставляющий законов Бога своего; они вопрошают Меня о судах правды, желают приближения к Богу:
\end{tcolorbox}
\begin{tcolorbox}
\textsubscript{3} 'Почему мы постимся, а Ты не видишь? смиряем души свои, а Ты не знаешь?' --Вот, в день поста вашего вы исполняете волю вашу и требуете тяжких трудов от других.
\end{tcolorbox}
\begin{tcolorbox}
\textsubscript{4} Вот, вы поститесь для ссор и распрей и для того, чтобы дерзкою рукою бить других; вы не поститесь в это время так, чтобы голос ваш был услышан на высоте.
\end{tcolorbox}
\begin{tcolorbox}
\textsubscript{5} Таков ли тот пост, который Я избрал, день, в который томит человек душу свою, когда гнет голову свою, как тростник, и подстилает под себя рубище и пепел? Это ли назовешь постом и днем, угодным Господу?
\end{tcolorbox}
\begin{tcolorbox}
\textsubscript{6} Вот пост, который Я избрал: разреши оковы неправды, развяжи узы ярма, и угнетенных отпусти на свободу, и расторгни всякое ярмо;
\end{tcolorbox}
\begin{tcolorbox}
\textsubscript{7} раздели с голодным хлеб твой, и скитающихся бедных введи в дом; когда увидишь нагого, одень его, и от единокровного твоего не укрывайся.
\end{tcolorbox}
\begin{tcolorbox}
\textsubscript{8} Тогда откроется, как заря, свет твой, и исцеление твое скоро возрастет, и правда твоя пойдет пред тобою, и слава Господня будет сопровождать тебя.
\end{tcolorbox}
\begin{tcolorbox}
\textsubscript{9} Тогда ты воззовешь, и Господь услышит; возопиешь, и Он скажет: 'вот Я!' Когда ты удалишь из среды твоей ярмо, перестанешь поднимать перст и говорить оскорбительное,
\end{tcolorbox}
\begin{tcolorbox}
\textsubscript{10} и отдашь голодному душу твою и напитаешь душу страдальца: тогда свет твой взойдет во тьме, и мрак твой [будет] как полдень;
\end{tcolorbox}
\begin{tcolorbox}
\textsubscript{11} и будет Господь вождем твоим всегда, и во время засухи будет насыщать душу твою и утучнять кости твои, и ты будешь, как напоенный водою сад и как источник, которого воды никогда не иссякают.
\end{tcolorbox}
\begin{tcolorbox}
\textsubscript{12} И застроятся [потомками] твоими пустыни вековые: ты восстановишь основания многих поколений, и будут называть тебя восстановителем развалин, возобновителем путей для населения.
\end{tcolorbox}
\begin{tcolorbox}
\textsubscript{13} Если ты удержишь ногу твою ради субботы от исполнения прихотей твоих во святый день Мой, и будешь называть субботу отрадою, святым днем Господним, чествуемым, и почтишь ее тем, что не будешь заниматься обычными твоими делами, угождать твоей прихоти и пустословить, --
\end{tcolorbox}
\begin{tcolorbox}
\textsubscript{14} то будешь иметь радость в Господе, и Я возведу тебя на высоты земли и дам вкусить тебе наследие Иакова, отца твоего: уста Господни изрекли это.
\end{tcolorbox}
\subsection{CHAPTER 59}
\begin{tcolorbox}
\textsubscript{1} Вот, рука Господа не сократилась на то, чтобы спасать, и ухо Его не отяжелело для того, чтобы слышать.
\end{tcolorbox}
\begin{tcolorbox}
\textsubscript{2} Но беззакония ваши произвели разделение между вами и Богом вашим, и грехи ваши отвращают лице [Его] от вас, чтобы не слышать.
\end{tcolorbox}
\begin{tcolorbox}
\textsubscript{3} Ибо руки ваши осквернены кровью и персты ваши--беззаконием; уста ваши говорят ложь, язык ваш произносит неправду.
\end{tcolorbox}
\begin{tcolorbox}
\textsubscript{4} Никто не возвышает голоса за правду, и никто не вступается за истину; надеются на пустое и говорят ложь, зачинают зло и рождают злодейство;
\end{tcolorbox}
\begin{tcolorbox}
\textsubscript{5} высиживают змеиные яйца и ткут паутину; кто поест яиц их, --умрет, а если раздавит, --выползет ехидна.
\end{tcolorbox}
\begin{tcolorbox}
\textsubscript{6} Паутины их для одежды негодны, и они не покроются своим произведением; дела их--дела неправедные, и насилие в руках их.
\end{tcolorbox}
\begin{tcolorbox}
\textsubscript{7} Ноги их бегут ко злу, и они спешат на пролитие невинной крови; мысли их--мысли нечестивые; опустошение и гибель на стезях их.
\end{tcolorbox}
\begin{tcolorbox}
\textsubscript{8} Пути мира они не знают, и нет суда на стезях их; пути их искривлены, и никто, идущий по ним, не знает мира.
\end{tcolorbox}
\begin{tcolorbox}
\textsubscript{9} Потому-то и далек от нас суд, и правосудие не достигает до нас; ждем света, и вот тьма, --озарения, и ходим во мраке.
\end{tcolorbox}
\begin{tcolorbox}
\textsubscript{10} Осязаем, как слепые стену, и, как без глаз, ходим ощупью; спотыкаемся в полдень, как в сумерки, между живыми--как мертвые.
\end{tcolorbox}
\begin{tcolorbox}
\textsubscript{11} Все мы ревем, как медведи, и стонем, как голуби; ожидаем суда, и нет [его], --спасения, но оно далеко от нас.
\end{tcolorbox}
\begin{tcolorbox}
\textsubscript{12} Ибо преступления наши многочисленны пред Тобою, и грехи наши свидетельствуют против нас; ибо преступления наши с нами, и беззакония наши мы знаем.
\end{tcolorbox}
\begin{tcolorbox}
\textsubscript{13} Мы изменили и солгали пред Господом, и отступили от Бога нашего; говорили клевету и измену, зачинали и рождали из сердца лживые слова.
\end{tcolorbox}
\begin{tcolorbox}
\textsubscript{14} И суд отступил назад, и правда стала вдали, ибо истина преткнулась на площади, и честность не может войти.
\end{tcolorbox}
\begin{tcolorbox}
\textsubscript{15} И не стало истины, и удаляющийся от зла подвергается оскорблению. И Господь увидел это, и противно было очам Его, что нет суда.
\end{tcolorbox}
\begin{tcolorbox}
\textsubscript{16} И видел, что нет человека, и дивился, что нет заступника; и помогла Ему мышца Его, и правда Его поддержала Его.
\end{tcolorbox}
\begin{tcolorbox}
\textsubscript{17} И Он возложил на Себя правду, как броню, и шлем спасения на главу Свою; и облекся в ризу мщения, как в одежду, и покрыл Себя ревностью, как плащом.
\end{tcolorbox}
\begin{tcolorbox}
\textsubscript{18} По мере возмездия, по этой мере Он воздаст противникам Своим--яростью, врагам Своим--местью, островам воздаст должное.
\end{tcolorbox}
\begin{tcolorbox}
\textsubscript{19} И убоятся имени Господа на западе и славы Его--на восходе солнца. Если враг придет как река, дуновение Господа прогонит его.
\end{tcolorbox}
\begin{tcolorbox}
\textsubscript{20} И придет Искупитель Сиона и [сынов] Иакова, обратившихся от нечестия, говорит Господь.
\end{tcolorbox}
\begin{tcolorbox}
\textsubscript{21} И вот завет Мой с ними, говорит Господь: Дух Мой, Который на тебе, и слова Мои, которые вложил Я в уста твои, не отступят от уст твоих и от уст потомства твоего, и от уст потомков потомства твоего, говорит Господь, отныне и до века.
\end{tcolorbox}
\subsection{CHAPTER 60}
\begin{tcolorbox}
\textsubscript{1} Восстань, светись, [Иерусалим], ибо пришел свет твой, и слава Господня взошла над тобою.
\end{tcolorbox}
\begin{tcolorbox}
\textsubscript{2} Ибо вот, тьма покроет землю, и мрак--народы; а над тобою воссияет Господь, и слава Его явится над тобою.
\end{tcolorbox}
\begin{tcolorbox}
\textsubscript{3} И придут народы к свету твоему, и цари--к восходящему над тобою сиянию.
\end{tcolorbox}
\begin{tcolorbox}
\textsubscript{4} Возведи очи твои и посмотри вокруг: все они собираются, идут к тебе; сыновья твои издалека идут и дочерей твоих на руках несут.
\end{tcolorbox}
\begin{tcolorbox}
\textsubscript{5} Тогда увидишь, и возрадуешься, и затрепещет и расширится сердце твое, потому что богатство моря обратится к тебе, достояние народов придет к тебе.
\end{tcolorbox}
\begin{tcolorbox}
\textsubscript{6} Множество верблюдов покроет тебя--дромадеры из Мадиама и Ефы; все они из Савы придут, принесут золото и ладан и возвестят славу Господа.
\end{tcolorbox}
\begin{tcolorbox}
\textsubscript{7} Все овцы Кидарские будут собраны к тебе; овны Неваиофские послужат тебе: взойдут на алтарь Мой жертвою благоугодною, и Я прославлю дом славы Моей.
\end{tcolorbox}
\begin{tcolorbox}
\textsubscript{8} Кто это летят, как облака, и как голуби--к голубятням своим?
\end{tcolorbox}
\begin{tcolorbox}
\textsubscript{9} Так, Меня ждут острова и впереди их--корабли Фарсисские, чтобы перевезти сынов твоих издалека и с ними серебро их и золото их, во имя Господа Бога твоего и Святаго Израилева, потому что Он прославил тебя.
\end{tcolorbox}
\begin{tcolorbox}
\textsubscript{10} Тогда сыновья иноземцев будут строить стены твои, и цари их--служить тебе; ибо во гневе Моем Я поражал тебя, но в благоволении Моем буду милостив к тебе.
\end{tcolorbox}
\begin{tcolorbox}
\textsubscript{11} И будут всегда отверсты врата твои, не будут затворяться ни днем ни ночью, чтобы приносимо было к тебе достояние народов и приводимы были цари их.
\end{tcolorbox}
\begin{tcolorbox}
\textsubscript{12} Ибо народ и царства, которые не захотят служить тебе, --погибнут, и такие народы совершенно истребятся.
\end{tcolorbox}
\begin{tcolorbox}
\textsubscript{13} Слава Ливана придет к тебе, кипарис и певг и вместе кедр, чтобы украсить место святилища Моего, и Я прославлю подножие ног Моих.
\end{tcolorbox}
\begin{tcolorbox}
\textsubscript{14} И придут к тебе с покорностью сыновья угнетавших тебя, и падут к стопам ног твоих все, презиравшие тебя, и назовут тебя городом Господа, Сионом Святаго Израилева.
\end{tcolorbox}
\begin{tcolorbox}
\textsubscript{15} Вместо того, что ты был оставлен и ненавидим, так что никто не проходил чрез [тебя], Я соделаю тебя величием навеки, радостью в роды родов.
\end{tcolorbox}
\begin{tcolorbox}
\textsubscript{16} Ты будешь насыщаться молоком народов, и груди царские сосать будешь, и узнаешь, что Я Господь--Спаситель твой и Искупитель твой, Сильный Иаковлев.
\end{tcolorbox}
\begin{tcolorbox}
\textsubscript{17} Вместо меди буду доставлять тебе золото, и вместо железа серебро, и вместо дерева медь, и вместо камней железо; и поставлю правителем твоим мир и надзирателями твоими--правду.
\end{tcolorbox}
\begin{tcolorbox}
\textsubscript{18} Не слышно будет более насилия в земле твоей, опустошения и разорения--в пределах твоих; и будешь называть стены твои спасением и ворота твои--славою.
\end{tcolorbox}
\begin{tcolorbox}
\textsubscript{19} Не будет уже солнце служить тебе светом дневным, и сияние луны--светить тебе; но Господь будет тебе вечным светом, и Бог твой--славою твоею.
\end{tcolorbox}
\begin{tcolorbox}
\textsubscript{20} Не зайдет уже солнце твое, и луна твоя не сокроется, ибо Господь будет для тебя вечным светом, и окончатся дни сетования твоего.
\end{tcolorbox}
\begin{tcolorbox}
\textsubscript{21} И народ твой весь будет праведный, на веки наследует землю, --отрасль насаждения Моего, дело рук Моих, к прославлению Моему.
\end{tcolorbox}
\begin{tcolorbox}
\textsubscript{22} От малого произойдет тысяча, и от самого слабого--сильный народ. Я, Господь, ускорю совершить это в свое время.
\end{tcolorbox}
\subsection{CHAPTER 61}
\begin{tcolorbox}
\textsubscript{1} Дух Господа Бога на Мне, ибо Господь помазал Меня благовествовать нищим, послал Меня исцелять сокрушенных сердцем, проповедывать пленным освобождение и узникам открытие темницы,
\end{tcolorbox}
\begin{tcolorbox}
\textsubscript{2} проповедывать лето Господне благоприятное и день мщения Бога нашего, утешить всех сетующих,
\end{tcolorbox}
\begin{tcolorbox}
\textsubscript{3} возвестить сетующим на Сионе, что им вместо пепла дастся украшение, вместо плача--елей радости, вместо унылого духа--славная одежда, и назовут их сильными правдою, насаждением Господа во славу Его.
\end{tcolorbox}
\begin{tcolorbox}
\textsubscript{4} И застроят пустыни вековые, восстановят древние развалины и возобновят города разоренные, остававшиеся в запустении с давних родов.
\end{tcolorbox}
\begin{tcolorbox}
\textsubscript{5} И придут иноземцы и будут пасти стада ваши; и сыновья чужестранцев [будут] вашими земледельцами и вашими виноградарями.
\end{tcolorbox}
\begin{tcolorbox}
\textsubscript{6} А вы будете называться священниками Господа, служителями Бога нашего будут именовать вас; будете пользоваться достоянием народов и славиться славою их.
\end{tcolorbox}
\begin{tcolorbox}
\textsubscript{7} За посрамление вам будет вдвое; за поношение они будут радоваться своей доле, потому что в земле своей вдвое получат; веселие вечное будет у них.
\end{tcolorbox}
\begin{tcolorbox}
\textsubscript{8} Ибо Я, Господь, люблю правосудие, ненавижу грабительство с насилием, и воздам награду им по истине, и завет вечный поставлю с ними;
\end{tcolorbox}
\begin{tcolorbox}
\textsubscript{9} и будет известно между народами семя их, и потомство их--среди племен; все видящие их познают, что они семя, благословенное Господом.
\end{tcolorbox}
\begin{tcolorbox}
\textsubscript{10} Радостью буду радоваться о Господе, возвеселится душа моя о Боге моем; ибо Он облек меня в ризы спасения, одеждою правды одел меня, как на жениха возложил венец и, как невесту, украсил убранством.
\end{tcolorbox}
\begin{tcolorbox}
\textsubscript{11} Ибо, как земля производит растения свои, и как сад произращает посеянное в нем, так Господь Бог проявит правду и славу пред всеми народами.
\end{tcolorbox}
\subsection{CHAPTER 62}
\begin{tcolorbox}
\textsubscript{1} Не умолкну ради Сиона, и ради Иерусалима не успокоюсь, доколе не взойдет, как свет, правда его и спасение его--как горящий светильник.
\end{tcolorbox}
\begin{tcolorbox}
\textsubscript{2} И увидят народы правду твою и все цари--славу твою, и назовут тебя новым именем, которое нарекут уста Господа.
\end{tcolorbox}
\begin{tcolorbox}
\textsubscript{3} И будешь венцом славы в руке Господа и царскою диадемою на длани Бога твоего.
\end{tcolorbox}
\begin{tcolorbox}
\textsubscript{4} Не будут уже называть тебя 'оставленным', и землю твою не будут более называть 'пустынею', но будут называть тебя: 'Мое благоволение к нему', а землю твою--'замужнею', ибо Господь благоволит к тебе, и земля твоя сочетается.
\end{tcolorbox}
\begin{tcolorbox}
\textsubscript{5} Как юноша сочетается с девою, так сочетаются с тобою сыновья твои; и [как] жених радуется о невесте, так будет радоваться о тебе Бог твой.
\end{tcolorbox}
\begin{tcolorbox}
\textsubscript{6} На стенах твоих, Иерусалим, Я поставил сторожей, [которые] не будут умолкать ни днем, ни ночью. О, вы, напоминающие о Господе! не умолкайте, --
\end{tcolorbox}
\begin{tcolorbox}
\textsubscript{7} не умолкайте пред Ним, доколе Он не восстановит и доколе не сделает Иерусалима славою на земле.
\end{tcolorbox}
\begin{tcolorbox}
\textsubscript{8} Господь поклялся десницею Своею и крепкою мышцею Своею: не дам зерна твоего более в пищу врагам твоим, и сыновья чужих не будут пить вина твоего, над которым ты трудился;
\end{tcolorbox}
\begin{tcolorbox}
\textsubscript{9} но собирающие его будут есть его и славить Господа, и обирающие виноград будут пить [вино] его во дворах святилища Моего.
\end{tcolorbox}
\begin{tcolorbox}
\textsubscript{10} Проходите, проходите в ворота, приготовляйте путь народу! Ровняйте, ровняйте дорогу, убирайте камни, поднимите знамя для народов!
\end{tcolorbox}
\begin{tcolorbox}
\textsubscript{11} Вот, Господь объявляет до конца земли: скажите дщери Сиона: грядет Спаситель твой; награда Его с Ним и воздаяние Его пред Ним.
\end{tcolorbox}
\begin{tcolorbox}
\textsubscript{12} И назовут их народом святым, искупленным от Господа, а тебя назовут взысканным городом, неоставленным.
\end{tcolorbox}
\subsection{CHAPTER 63}
\begin{tcolorbox}
\textsubscript{1} Кто это идет от Едома, в червленых ризах от Восора, столь величественный в Своей одежде, выступающий в полноте силы Своей? 'Я--изрекающий правду, сильный, чтобы спасать'.
\end{tcolorbox}
\begin{tcolorbox}
\textsubscript{2} Отчего же одеяние Твое красно, и ризы у Тебя, как у топтавшего в точиле?
\end{tcolorbox}
\begin{tcolorbox}
\textsubscript{3} 'Я топтал точило один, и из народов никого не было со Мною; и Я топтал их во гневе Моем и попирал их в ярости Моей; кровь их брызгала на ризы Мои, и Я запятнал все одеяние Свое;
\end{tcolorbox}
\begin{tcolorbox}
\textsubscript{4} ибо день мщения--в сердце Моем, и год Моих искупленных настал.
\end{tcolorbox}
\begin{tcolorbox}
\textsubscript{5} Я смотрел, и не было помощника; дивился, что не было поддерживающего; но помогла Мне мышца Моя, и ярость Моя--она поддержала Меня:
\end{tcolorbox}
\begin{tcolorbox}
\textsubscript{6} и попрал Я народы во гневе Моем, и сокрушил их в ярости Моей, и вылил на землю кровь их'.
\end{tcolorbox}
\begin{tcolorbox}
\textsubscript{7} Воспомяну милости Господни и славу Господню за все, что Господь даровал нам, и великую благость [Его] к дому Израилеву, какую оказал Он ему по милосердию Своему и по множеству щедрот Своих.
\end{tcolorbox}
\begin{tcolorbox}
\textsubscript{8} Он сказал: 'подлинно они народ Мой, дети, которые не солгут', и Он был для них Спасителем.
\end{tcolorbox}
\begin{tcolorbox}
\textsubscript{9} Во всякой скорби их Он не оставлял их, и Ангел лица Его спасал их; по любви Своей и благосердию Своему Он искупил их, взял и носил их во все дни древние.
\end{tcolorbox}
\begin{tcolorbox}
\textsubscript{10} Но они возмутились и огорчили Святаго Духа Его; поэтому Он обратился в неприятеля их: Сам воевал против них.
\end{tcolorbox}
\begin{tcolorbox}
\textsubscript{11} Тогда народ Его вспомнил древние дни, Моисеевы: где Тот, Который вывел их из моря с пастырем овец Своих? где Тот, Который вложил в сердце его Святаго Духа Своего,
\end{tcolorbox}
\begin{tcolorbox}
\textsubscript{12} Который вел Моисея за правую руку величественною мышцею Своею, разделил пред ними воды, чтобы сделать Себе вечное имя,
\end{tcolorbox}
\begin{tcolorbox}
\textsubscript{13} Который вел их чрез бездны, как коня по степи, [и] они не спотыкались?
\end{tcolorbox}
\begin{tcolorbox}
\textsubscript{14} Как стадо сходит в долину, Дух Господень вел их к покою. Так вел Ты народ Твой, чтобы сделать Себе славное имя.
\end{tcolorbox}
\begin{tcolorbox}
\textsubscript{15} Призри с небес и посмотри из жилища святыни Твоей и славы Твоей: где ревность Твоя и могущество Твое? --благоутробие Твое и милости Твои ко мне удержаны.
\end{tcolorbox}
\begin{tcolorbox}
\textsubscript{16} Только Ты--Отец наш; ибо Авраам не узнаёт нас, и Израиль не признаёт нас своими; Ты, Господи, Отец наш, от века имя Твое: 'Искупитель наш'.
\end{tcolorbox}
\begin{tcolorbox}
\textsubscript{17} Для чего, Господи, Ты попустил нам совратиться с путей Твоих, ожесточиться сердцу нашему, чтобы не бояться Тебя? обратись ради рабов Твоих, ради колен наследия Твоего.
\end{tcolorbox}
\begin{tcolorbox}
\textsubscript{18} Короткое время владел им народ святыни Твоей: враги наши попрали святилище Твое.
\end{tcolorbox}
\begin{tcolorbox}
\textsubscript{19} Мы сделались такими, над которыми Ты как бы никогда не владычествовал и над которыми не именовалось имя Твое.
\end{tcolorbox}
\subsection{CHAPTER 64}
\begin{tcolorbox}
\textsubscript{1} О, если бы Ты расторг небеса [и] сошел! горы растаяли бы от лица Твоего,
\end{tcolorbox}
\begin{tcolorbox}
\textsubscript{2} как от плавящего огня, как от кипятящего воду, чтобы имя Твое сделать известным врагам Твоим; от лица Твоего содрогнулись бы народы.
\end{tcolorbox}
\begin{tcolorbox}
\textsubscript{3} Когда Ты совершал страшные дела, нами неожиданные, и нисходил, --горы таяли от лица Твоего.
\end{tcolorbox}
\begin{tcolorbox}
\textsubscript{4} Ибо от века не слыхали, не внимали ухом, и никакой глаз не видал другого бога, кроме Тебя, который столько сделал бы для надеющихся на него.
\end{tcolorbox}
\begin{tcolorbox}
\textsubscript{5} Ты милостиво встречал радующегося и делающего правду, поминающего Тебя на путях Твоих. Но вот, Ты прогневался, потому что мы издавна грешили; и как же мы будем спасены?
\end{tcolorbox}
\begin{tcolorbox}
\textsubscript{6} Все мы сделались--как нечистый, и вся праведность наша--как запачканная одежда; и все мы поблекли, как лист, и беззакония наши, как ветер, уносят нас.
\end{tcolorbox}
\begin{tcolorbox}
\textsubscript{7} И нет призывающего имя Твое, который положил бы крепко держаться за Тебя; поэтому Ты сокрыл от нас лице Твое и оставил нас погибать от беззаконий наших.
\end{tcolorbox}
\begin{tcolorbox}
\textsubscript{8} Но ныне, Господи, Ты--Отец наш; мы--глина, а Ты--образователь наш, и все мы--дело руки Твоей.
\end{tcolorbox}
\begin{tcolorbox}
\textsubscript{9} Не гневайся, Господи, без меры, и не вечно помни беззаконие. Воззри же: мы все народ Твой.
\end{tcolorbox}
\begin{tcolorbox}
\textsubscript{10} Города святыни Твоей сделались пустынею; пустынею стал Сион; Иерусалим опустошен.
\end{tcolorbox}
\begin{tcolorbox}
\textsubscript{11} Дом освящения нашего и славы нашей, где отцы наши прославляли Тебя, сожжен огнем, и все драгоценности наши разграблены.
\end{tcolorbox}
\begin{tcolorbox}
\textsubscript{12} После этого будешь ли еще удерживаться, Господи, будешь ли молчать и карать нас без меры?
\end{tcolorbox}
\subsection{CHAPTER 65}
\begin{tcolorbox}
\textsubscript{1} Я открылся не вопрошавшим обо Мне; Меня нашли не искавшие Меня. 'Вот Я! вот Я!' говорил Я народу, не именовавшемуся именем Моим.
\end{tcolorbox}
\begin{tcolorbox}
\textsubscript{2} Всякий день простирал Я руки Мои к народу непокорному, ходившему путем недобрым, по своим помышлениям, --
\end{tcolorbox}
\begin{tcolorbox}
\textsubscript{3} к народу, который постоянно оскорбляет Меня в лице, приносит жертвы в рощах и сожигает фимиам на черепках,
\end{tcolorbox}
\begin{tcolorbox}
\textsubscript{4} сидит в гробах и ночует в пещерах; ест свиное мясо, и мерзкое варево в сосудах у него;
\end{tcolorbox}
\begin{tcolorbox}
\textsubscript{5} который говорит: 'остановись, не подходи ко мне, потому что я свят для тебя'. Они--дым для обоняния Моего, огонь, горящий всякий день.
\end{tcolorbox}
\begin{tcolorbox}
\textsubscript{6} Вот что написано пред лицем Моим: не умолчу, но воздам, воздам в недро их
\end{tcolorbox}
\begin{tcolorbox}
\textsubscript{7} беззакония ваши, говорит Господь, и вместе беззакония отцов ваших, которые воскуряли фимиам на горах, и на холмах поносили Меня; и отмерю в недра их прежние деяния их.
\end{tcolorbox}
\begin{tcolorbox}
\textsubscript{8} Так говорит Господь: когда в виноградной кисти находится сок, тогда говорят: 'не повреди ее, ибо в ней благословение'; то же сделаю Я и ради рабов Моих, чтобы не всех погубить.
\end{tcolorbox}
\begin{tcolorbox}
\textsubscript{9} И произведу от Иакова семя, и от Иуды наследника гор Моих, и наследуют это избранные Мои, и рабы Мои будут жить там.
\end{tcolorbox}
\begin{tcolorbox}
\textsubscript{10} И будет Сарон пастбищем для овец и долина Ахор--местом отдыха для волов народа Моего, который взыскал Меня.
\end{tcolorbox}
\begin{tcolorbox}
\textsubscript{11} А вас, которые оставили Господа, забыли святую гору Мою, приготовляете трапезу для Гада и растворяете полную чашу для Мени, --
\end{tcolorbox}
\begin{tcolorbox}
\textsubscript{12} вас обрекаю Я мечу, и все вы преклонитесь на заклание: потому что Я звал, и вы не отвечали; говорил, и вы не слушали, но делали злое в очах Моих и избирали то, что было неугодно Мне.
\end{tcolorbox}
\begin{tcolorbox}
\textsubscript{13} Посему так говорит Господь Бог: вот, рабы Мои будут есть, а вы будете голодать; рабы Мои будут пить, а вы будете томиться жаждою;
\end{tcolorbox}
\begin{tcolorbox}
\textsubscript{14} рабы Мои будут веселиться, а вы будете в стыде; рабы Мои будут петь от сердечной радости, а вы будете кричать от сердечной скорби и рыдать от сокрушения духа.
\end{tcolorbox}
\begin{tcolorbox}
\textsubscript{15} И оставите имя ваше избранным Моим для проклятия; и убьет тебя Господь Бог, а рабов Своих назовет иным именем,
\end{tcolorbox}
\begin{tcolorbox}
\textsubscript{16} которым кто будет благословлять себя на земле, будет благословляться Богом истины; и кто будет клясться на земле, будет клясться Богом истины, --потому что прежние скорби будут забыты и сокрыты от очей Моих.
\end{tcolorbox}
\begin{tcolorbox}
\textsubscript{17} Ибо вот, Я творю новое небо и новую землю, и прежние уже не будут воспоминаемы и не придут на сердце.
\end{tcolorbox}
\begin{tcolorbox}
\textsubscript{18} А вы будете веселиться и радоваться вовеки о том, что Я творю: ибо вот, Я творю Иерусалим веселием и народ его радостью.
\end{tcolorbox}
\begin{tcolorbox}
\textsubscript{19} И буду радоваться о Иерусалиме и веселиться о народе Моем; и не услышится в нем более голос плача и голос вопля.
\end{tcolorbox}
\begin{tcolorbox}
\textsubscript{20} Там не будет более малолетнего и старца, который не достигал бы полноты дней своих; ибо столетний будет умирать юношею, но столетний грешник будет проклинаем.
\end{tcolorbox}
\begin{tcolorbox}
\textsubscript{21} И буду строить домы и жить в них, и насаждать виноградники и есть плоды их.
\end{tcolorbox}
\begin{tcolorbox}
\textsubscript{22} Не будут строить, чтобы другой жил, не будут насаждать, чтобы другой ел; ибо дни народа Моего будут, как дни дерева, и избранные Мои долго будут пользоваться изделием рук своих.
\end{tcolorbox}
\begin{tcolorbox}
\textsubscript{23} Не будут трудиться напрасно и рождать детей на горе; ибо будут семенем, благословенным от Господа, и потомки их с ними.
\end{tcolorbox}
\begin{tcolorbox}
\textsubscript{24} И будет, прежде нежели они воззовут, Я отвечу; они еще будут говорить, и Я уже услышу.
\end{tcolorbox}
\begin{tcolorbox}
\textsubscript{25} Волк и ягненок будут пастись вместе, и лев, как вол, будет есть солому, а для змея прах будет пищею: они не будут причинять зла и вреда на всей святой горе Моей, говорит Господь.
\end{tcolorbox}
\subsection{CHAPTER 66}
\begin{tcolorbox}
\textsubscript{1} Так говорит Господь: небо--престол Мой, а земля--подножие ног Моих; где же построите вы дом для Меня, и где место покоя Моего?
\end{tcolorbox}
\begin{tcolorbox}
\textsubscript{2} Ибо все это соделала рука Моя, и все сие было, говорит Господь. А вот на кого Я призрю: на смиренного и сокрушенного духом и на трепещущего пред словом Моим.
\end{tcolorbox}
\begin{tcolorbox}
\textsubscript{3} Заколающий вола--то же, что убивающий человека; приносящий агнца в жертву--то же, что задушающий пса; приносящий семидал--то же, что приносящий свиную кровь; воскуряющий фимиам--то же, что молящийся идолу; и как они избрали собственные свои пути, и душа их находит удовольствие в мерзостях их, --
\end{tcolorbox}
\begin{tcolorbox}
\textsubscript{4} так и Я употреблю их обольщение и наведу на них ужасное для них: потому что Я звал, и не было отвечающего, говорил, и они не слушали, а делали злое в очах Моих и избирали то, что неугодно Мне.
\end{tcolorbox}
\begin{tcolorbox}
\textsubscript{5} Выслушайте слово Господа, трепещущие пред словом Его: ваши братья, ненавидящие вас и изгоняющие вас за имя Мое, говорят: 'пусть явит Себя в славе Господь, и мы посмотрим на веселие ваше'. Но они будут постыжены.
\end{tcolorbox}
\begin{tcolorbox}
\textsubscript{6} Вот, шум из города, голос из храма, голос Господа, воздающего возмездие врагам Своим.
\end{tcolorbox}
\begin{tcolorbox}
\textsubscript{7} Еще не мучилась родами, а родила; прежде нежели наступили боли ее, разрешилась сыном.
\end{tcolorbox}
\begin{tcolorbox}
\textsubscript{8} Кто слыхал таковое? кто видал подобное этому? возникала ли страна в один день? рождался ли народ в один раз, как Сион, едва начал родами мучиться, родил сынов своих?
\end{tcolorbox}
\begin{tcolorbox}
\textsubscript{9} Доведу ли Я до родов, и не дам родить? говорит Господь. Или, давая силу родить, заключу ли [утробу]? говорит Бог твой.
\end{tcolorbox}
\begin{tcolorbox}
\textsubscript{10} Возвеселитесь с Иерусалимом и радуйтесь о нем, все любящие его! возрадуйтесь с ним радостью, все сетовавшие о нем,
\end{tcolorbox}
\begin{tcolorbox}
\textsubscript{11} чтобы вам питаться и насыщаться от сосцов утешений его, упиваться и наслаждаться преизбытком славы его.
\end{tcolorbox}
\begin{tcolorbox}
\textsubscript{12} Ибо так говорит Господь: вот, Я направляю к нему мир как реку, и богатство народов--как разливающийся поток для наслаждения вашего; на руках будут носить вас и на коленях ласкать.
\end{tcolorbox}
\begin{tcolorbox}
\textsubscript{13} Как утешает кого-либо мать его, так утешу Я вас, и вы будете утешены в Иерусалиме.
\end{tcolorbox}
\begin{tcolorbox}
\textsubscript{14} И увидите это, и возрадуется сердце ваше, и кости ваши расцветут, как молодая зелень, и откроется рука Господа рабам Его, а на врагов Своих Он разгневается.
\end{tcolorbox}
\begin{tcolorbox}
\textsubscript{15} Ибо вот, придет Господь в огне, и колесницы Его--как вихрь, чтобы излить гнев Свой с яростью и прещение Свое с пылающим огнем.
\end{tcolorbox}
\begin{tcolorbox}
\textsubscript{16} Ибо Господь с огнем и мечом Своим произведет суд над всякою плотью, и много будет пораженных Господом.
\end{tcolorbox}
\begin{tcolorbox}
\textsubscript{17} Те, которые освящают и очищают себя в рощах, один за другим, едят свиное мясо и мерзость и мышей, --все погибнут, говорит Господь.
\end{tcolorbox}
\begin{tcolorbox}
\textsubscript{18} Ибо Я [знаю] деяния их и мысли их; и вот, приду собрать все народы и языки, и они придут и увидят славу Мою.
\end{tcolorbox}
\begin{tcolorbox}
\textsubscript{19} И положу на них знамение, и пошлю из спасенных от них к народам: в Фарсис, к Пулу и Луду, к натягивающим лук, к Тубалу и Явану, на дальние острова, которые не слышали обо Мне и не видели славы Моей: и они возвестят народам славу Мою
\end{tcolorbox}
\begin{tcolorbox}
\textsubscript{20} и представят всех братьев ваших от всех народов в дар Господу на конях и колесницах, и на носилках, и на мулах, и на быстрых верблюдах, на святую гору Мою, в Иерусалим, говорит Господь, --подобно тому, как сыны Израилевы приносят дар в дом Господа в чистом сосуде.
\end{tcolorbox}
\begin{tcolorbox}
\textsubscript{21} Из них буду брать также в священники и левиты, говорит Господь.
\end{tcolorbox}
\begin{tcolorbox}
\textsubscript{22} Ибо, как новое небо и новая земля, которые Я сотворю, всегда будут пред лицем Моим, говорит Господь, так будет и семя ваше и имя ваше.
\end{tcolorbox}
\begin{tcolorbox}
\textsubscript{23} Тогда из месяца в месяц и из субботы в субботу будет приходить всякая плоть пред лице Мое на поклонение, говорит Господь.
\end{tcolorbox}
\begin{tcolorbox}
\textsubscript{24} И будут выходить и увидят трупы людей, отступивших от Меня: ибо червь их не умрет, и огонь их не угаснет; и будут они мерзостью для всякой плоти.
\end{tcolorbox}
