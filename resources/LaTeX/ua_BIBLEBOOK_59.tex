\section{BOOK 58}
\subsection{CHAPTER 1}
\begin{tcolorbox}
\textsubscript{1} Яків, раб Бога й Господа Ісуса Христа, дванадцятьом племенам, які в Розпорошенні, вітаю я вас!
\end{tcolorbox}
\begin{tcolorbox}
\textsubscript{2} Майте, брати мої, повну радість, коли впадаєте в усілякі випробовування,
\end{tcolorbox}
\begin{tcolorbox}
\textsubscript{3} знаючи, що досвідчення вашої віри дає терпеливість.
\end{tcolorbox}
\begin{tcolorbox}
\textsubscript{4} А терпеливість нехай має чин досконалий, щоб ви досконалі та бездоганні були, і недостачі ні в чому не мали.
\end{tcolorbox}
\begin{tcolorbox}
\textsubscript{5} А якщо кому з вас не стачає мудрости, нехай просить від Бога, що всім дає просто, та не докоряє, і буде вона йому дана.
\end{tcolorbox}
\begin{tcolorbox}
\textsubscript{6} Але нехай просить із вірою, без жадного сумніву. Бо хто має сумнів, той подібний до морської хвилі, яку жене й кидає вітер.
\end{tcolorbox}
\begin{tcolorbox}
\textsubscript{7} Нехай бо така людина не гадає, що дістане що від Господа.
\end{tcolorbox}
\begin{tcolorbox}
\textsubscript{8} Двоєдушна людина непостійна на всіх дорогах своїх.
\end{tcolorbox}
\begin{tcolorbox}
\textsubscript{9} А понижений брат нехай хвалиться високістю своєю,
\end{tcolorbox}
\begin{tcolorbox}
\textsubscript{10} а багатий пониженням своїм, бо він промине, як той цвіт трав'яний,
\end{tcolorbox}
\begin{tcolorbox}
\textsubscript{11} бо сонце зійшло зо спекотою, і траву посушило, і відпав цвіт її, і зникла краса її виду... Так само зів'яне й багатий у дорогах своїх!
\end{tcolorbox}
\begin{tcolorbox}
\textsubscript{12} Блаженна людина, що витерпить пробу, бо, бувши випробувана, дістане вінця життя, якого Господь обіцяв тим, хто любить Його.
\end{tcolorbox}
\begin{tcolorbox}
\textsubscript{13} Випробовуваний, хай не каже ніхто: Я від Бога спокушуваний. Бо Бог злом не спокушується, і нікого Він Сам не спокушує.
\end{tcolorbox}
\begin{tcolorbox}
\textsubscript{14} Але кожен спокушується, як надиться й зводиться пожадливістю власною.
\end{tcolorbox}
\begin{tcolorbox}
\textsubscript{15} Пожадливість потому, зачавши, народжує гріх, а зроблений гріх народжує смерть.
\end{tcolorbox}
\begin{tcolorbox}
\textsubscript{16} Не обманюйтесь, брати мої любі!
\end{tcolorbox}
\begin{tcolorbox}
\textsubscript{17} Усяке добре давання та дар досконалий походить згори від Отця світил, що в Нього нема переміни чи тіні відміни.
\end{tcolorbox}
\begin{tcolorbox}
\textsubscript{18} Захотівши, Він нас породив словом правди, щоб ми стали якимсь первопочином творів Його.
\end{tcolorbox}
\begin{tcolorbox}
\textsubscript{19} Отож, мої брати любі, нехай буде кожна людина швидка послухати, забарна говорити, повільна на гнів.
\end{tcolorbox}
\begin{tcolorbox}
\textsubscript{20} Бо гнів людський не чинить правди Божої.
\end{tcolorbox}
\begin{tcolorbox}
\textsubscript{21} Тому то відкиньте всіляку нечисть та залишок злоби, і прийміть із лагідністю всіяне слово, що може спасти ваші душі.
\end{tcolorbox}
\begin{tcolorbox}
\textsubscript{22} Будьте ж виконавцями слова, а не слухачами самими, що себе самих обманюють.
\end{tcolorbox}
\begin{tcolorbox}
\textsubscript{23} Бо хто слухач слова, а не виконавець, той подібний людині, що риси обличчя свого розглядає у дзеркалі,
\end{tcolorbox}
\begin{tcolorbox}
\textsubscript{24} бо розгляне себе та й відійде, і зараз забуде, яка вона є.
\end{tcolorbox}
\begin{tcolorbox}
\textsubscript{25} А хто заглядає в закон досконалий, закон волі, і в нім пробуває, той не буде забудько слухач, але виконавець діла, і він буде блаженний у діянні своїм!
\end{tcolorbox}
\begin{tcolorbox}
\textsubscript{26} Коли ж хто гадає, що він побожний, і свого язика не вгамовує, та своє серце обманює, марна побожність того!
\end{tcolorbox}
\begin{tcolorbox}
\textsubscript{27} Чиста й непорочна побожність перед Богом і Отцем оця: зглянутися над сиротами та вдовицями в утисках їхніх, себе берегти чистим від світу.
\end{tcolorbox}
\subsection{CHAPTER 2}
\begin{tcolorbox}
\textsubscript{1} Брати мої, не зважаючи на обличчя, майте віру в нашого Господа слави, Ісуса Христа.
\end{tcolorbox}
\begin{tcolorbox}
\textsubscript{2} Бо коли до вашого зібрання ввійде чоловік із золотим перснем, у шаті блискучій, увійде й бідар у вбогім вбранні,
\end{tcolorbox}
\begin{tcolorbox}
\textsubscript{3} і ви поглянете на того, хто в шаті блискучій, і скажете йому: Ти сідай вигідно отут, а бідареві прокажете: Ти стань там, чи сідай собі тут на підніжку моїм,
\end{tcolorbox}
\begin{tcolorbox}
\textsubscript{4} то чи не стало між вами поділення, і не стали ви злодумними суддями?
\end{tcolorbox}
\begin{tcolorbox}
\textsubscript{5} Послухайте, мої брати любі, чи ж не вибрав Бог бідарів цього світу за багатих вірою й за спадкоємців Царства, яке обіцяв Він тим, хто любить Його?
\end{tcolorbox}
\begin{tcolorbox}
\textsubscript{6} А ви бідаря зневажили! Хіба не багачі переслідують вас, хіба не вони тягнуть вас на суди?
\end{tcolorbox}
\begin{tcolorbox}
\textsubscript{7} Хіба не вони зневажають те добре ім'я, що ви ним називаєтесь?
\end{tcolorbox}
\begin{tcolorbox}
\textsubscript{8} Коли ви Закона Царського виконуєте, за Писанням: Люби свого ближнього, як самого себе, то ви робите добре.
\end{tcolorbox}
\begin{tcolorbox}
\textsubscript{9} Коли ж дивитеся на обличчя, то чините гріх, бо Закон удоводнює, що ви винуватці.
\end{tcolorbox}
\begin{tcolorbox}
\textsubscript{10} Бо хто всього Закона виконує, а згрішить в одному, той винним у всьому стає.
\end{tcolorbox}
\begin{tcolorbox}
\textsubscript{11} Бо Той, Хто сказав: Не чини перелюбства, також наказав: Не вбивай. А хоч ти перелюбства не чиниш, а вб'єш, то ти переступник Закону.
\end{tcolorbox}
\begin{tcolorbox}
\textsubscript{12} Отак говоріть і отак чиніть, як такі, що будете суджені законом волі.
\end{tcolorbox}
\begin{tcolorbox}
\textsubscript{13} Бо суд немилосердний на того, хто не вчинив милосердя. Милосердя бо ставиться вище за суд.
\end{tcolorbox}
\begin{tcolorbox}
\textsubscript{14} Яка користь, брати мої, коли хто говорить, що має віру, але діл не має? Чи може спасти його віра?
\end{tcolorbox}
\begin{tcolorbox}
\textsubscript{15} Коли ж брат чи сестра будуть нагі, і позбавлені денного покорму,
\end{tcolorbox}
\begin{tcolorbox}
\textsubscript{16} а хтонебудь із вас до них скаже: Ідіть з миром, грійтесь та їжте, та не дасть їм потрібного тілу, що ж то поможе?
\end{tcolorbox}
\begin{tcolorbox}
\textsubscript{17} Так само й віра, коли діл не має, мертва в собі!
\end{tcolorbox}
\begin{tcolorbox}
\textsubscript{18} Але скаже хтонебудь: Маєш ти віру, а я маю діла; покажи мені віру свою без діл твоїх, а я покажу тобі віру свою від діл моїх.
\end{tcolorbox}
\begin{tcolorbox}
\textsubscript{19} Чи віруєш ти, що Бог один? Добре робиш! Та й демони вірують, і тремтять.
\end{tcolorbox}
\begin{tcolorbox}
\textsubscript{20} Чи хочеш ти знати, о марна людино, що віра без діл мертва?
\end{tcolorbox}
\begin{tcolorbox}
\textsubscript{21} Авраам, отець наш, чи він не з діл виправданий був, як поклав був на жертівника свого сина Ісака?
\end{tcolorbox}
\begin{tcolorbox}
\textsubscript{22} Чи ти бачиш, що віра помогла його ділам, і вдосконалилась віра із діл?
\end{tcolorbox}
\begin{tcolorbox}
\textsubscript{23} І здійснилося Писання, що каже: Авраам же ввірував Богові, і це йому зараховане в праведність, і був названий він другом Божим.
\end{tcolorbox}
\begin{tcolorbox}
\textsubscript{24} Отож, чи ви бачите, що людина виправдується від діл, а не тільки від віри?
\end{tcolorbox}
\begin{tcolorbox}
\textsubscript{25} Чи так само і блудниця Рахав не з діл виправдалась, коли прийняла посланців, і дорогою іншою випустила?
\end{tcolorbox}
\begin{tcolorbox}
\textsubscript{26} Бо як тіло без духа мертве, так і віра без діл мертва!
\end{tcolorbox}
\subsection{CHAPTER 3}
\begin{tcolorbox}
\textsubscript{1} Не багато-хто ставайте, брати мої, учителями, знавши, що більший осуд приймемо.
\end{tcolorbox}
\begin{tcolorbox}
\textsubscript{2} Бо багато ми всі помиляємось. Коли хто не помиляється в слові, то це муж досконалий, спроможний приборкувати й усе тіло.
\end{tcolorbox}
\begin{tcolorbox}
\textsubscript{3} От і коням вкладаєм уздечки до рота, щоб корилися нам, і ми всім їхнім тілом керуємо.
\end{tcolorbox}
\begin{tcolorbox}
\textsubscript{4} От і кораблі, хоч які величезні та гнані вітрами жорстокими, проте найменшим стерном скеровуються, куди хоче стерничий.
\end{tcolorbox}
\begin{tcolorbox}
\textsubscript{5} Так само й язик, малий член, але хвалиться вельми! Ось маленький огонь, а запалює величезного ліса!
\end{tcolorbox}
\begin{tcolorbox}
\textsubscript{6} І язик то огонь. Як світ неправости, поставлений так поміж нашими членами, язик сквернить усе тіло, запалює круг життя, і сам запалюється від геєнни.
\end{tcolorbox}
\begin{tcolorbox}
\textsubscript{7} Бо всяка природа звірів і пташок, гадів і морських потвор приборкується, і приборкана буде природою людською,
\end{tcolorbox}
\begin{tcolorbox}
\textsubscript{8} та не може ніхто із людей язика вгамувати, він зло безупинне, він повний отрути смертельної!
\end{tcolorbox}
\begin{tcolorbox}
\textsubscript{9} Ним ми благословляємо Бога й Отця, і ним проклинаєм людей, що створені на Божу подобу.
\end{tcolorbox}
\begin{tcolorbox}
\textsubscript{10} Із тих самих уст виходить благословення й прокляття. Не повинно, брати мої, щоб так це було!
\end{tcolorbox}
\begin{tcolorbox}
\textsubscript{11} Хіба з одного отвору виходить вода солодка й гірка?
\end{tcolorbox}
\begin{tcolorbox}
\textsubscript{12} Хіба може, брати мої, фіґове дерево родити оливки, або виноград фіґи? Солодка вода не тече з солонця.
\end{tcolorbox}
\begin{tcolorbox}
\textsubscript{13} Хто мудрий і розумний між вами? Нехай він покаже діла свої в лагідній мудрості добрим поводженням!
\end{tcolorbox}
\begin{tcolorbox}
\textsubscript{14} Коли ж гірку заздрість та сварку ви маєте в серці своєму, то не величайтесь та не говоріть неправди на правду,
\end{tcolorbox}
\begin{tcolorbox}
\textsubscript{15} це не мудрість, що ніби зверху походить вона, але земна, тілесна та демонська.
\end{tcolorbox}
\begin{tcolorbox}
\textsubscript{16} Бо де заздрість та сварка, там безлад та всяка зла річ!
\end{tcolorbox}
\begin{tcolorbox}
\textsubscript{17} А мудрість, що зверху вона, насамперед чиста, а потім спокійна, лагідна, покірлива, повна милосердя та добрих плодів, безстороння та нелукава.
\end{tcolorbox}
\begin{tcolorbox}
\textsubscript{18} А плід правди сіється творцями миру.
\end{tcolorbox}
\subsection{CHAPTER 4}
\begin{tcolorbox}
\textsubscript{1} Звідки війни та свари між вами? Чи не звідси, від ваших пожадливостей, які в ваших членах воюють?
\end{tcolorbox}
\begin{tcolorbox}
\textsubscript{2} Бажаєте ви та й не маєте, убиваєте й заздрите та досягнути не можете, сваритеся та воюєте та не маєте, бо не прохаєте,
\end{tcolorbox}
\begin{tcolorbox}
\textsubscript{3} прохаєте та не одержуєте, бо прохаєте на зле, щоб ужити на розкоші свої.
\end{tcolorbox}
\begin{tcolorbox}
\textsubscript{4} Перелюбники та перелюбниці, чи ж ви не знаєте, що дружба зо світом то ворожнеча супроти Бога? Бо хто хоче бути світові приятелем, той ворогом Божим стається.
\end{tcolorbox}
\begin{tcolorbox}
\textsubscript{5} Чи ви думаєте, що даремно Писання говорить: Жадає аж до заздрости дух, що в нас пробуває?
\end{tcolorbox}
\begin{tcolorbox}
\textsubscript{6} Та ще більшу благодать дає, через що й промовляє: Бог противиться гордим, а смиренним дає благодать.
\end{tcolorbox}
\begin{tcolorbox}
\textsubscript{7} Тож підкоріться Богові та спротивляйтесь дияволові, то й утече він від вас.
\end{tcolorbox}
\begin{tcolorbox}
\textsubscript{8} Наблизьтесь до Бога, то й Бог наблизиться до вас. Очистьте руки, грішні, та серця освятіть, двоєдушні!
\end{tcolorbox}
\begin{tcolorbox}
\textsubscript{9} Журіться, сумуйте та плачте! Хай обернеться сміх ваш у плач, а радість у сум!
\end{tcolorbox}
\begin{tcolorbox}
\textsubscript{10} Упокоріться перед Господнім лицем, і Він вас підійме!
\end{tcolorbox}
\begin{tcolorbox}
\textsubscript{11} Не обмовляйте, брати, один одного! Бо хто брата свого обмовляє або судить брата, той Закона обмовляє та судить Закона. А коли ти Закона осуджуєш, то ти не виконавець Закона, але суддя.
\end{tcolorbox}
\begin{tcolorbox}
\textsubscript{12} Один Законодавець і Суддя, що може спасти й погубити. А ти хто такий, що осуджуєш ближнього?
\end{tcolorbox}
\begin{tcolorbox}
\textsubscript{13} А ну тепер ви, що говорите: Сьогодні чи взавтра ми підем у те чи те місто, і там рік проживемо, та будемо торгувати й заробляти,
\end{tcolorbox}
\begin{tcolorbox}
\textsubscript{14} ви, що не відаєте, що трапиться взавтра, яке ваше життя? Бо це пара, що на хвильку з'являється, а потім зникає!...
\end{tcolorbox}
\begin{tcolorbox}
\textsubscript{15} Замість того, щоб вам говорити: Як схоче Господь та будемо живі, то зробимо це або те.
\end{tcolorbox}
\begin{tcolorbox}
\textsubscript{16} А тепер ви хвалитеся в своїх гордощах, лиха всяка подібна хвальба!
\end{tcolorbox}
\begin{tcolorbox}
\textsubscript{17} Отож, хто знає, як чинити добро, та не чинить, той має гріх!
\end{tcolorbox}
\subsection{CHAPTER 5}
\begin{tcolorbox}
\textsubscript{1} А ну ж тепер ви, багачі, плачте й ридайте над лихом своїм, що вас має спіткати:
\end{tcolorbox}
\begin{tcolorbox}
\textsubscript{2} ваше багатство згнило, а ваші вбрання міль поїла!
\end{tcolorbox}
\begin{tcolorbox}
\textsubscript{3} Золото ваше та срібло поіржавіло, а їхня іржа буде свідчити проти вас, і поїсть ваше тіло, немов той огонь! Ви скарби зібрали собі на останні дні!
\end{tcolorbox}
\begin{tcolorbox}
\textsubscript{4} Ось голосить заплата, що ви затримали в робітників, які жали на ваших полях, і голосіння женців досягли вух Господа Саваота!
\end{tcolorbox}
\begin{tcolorbox}
\textsubscript{5} Ви розкошували на землі й насолоджувались, серця свої вигодували, немов би на день заколення.
\end{tcolorbox}
\begin{tcolorbox}
\textsubscript{6} Ви Праведного засудили й убили, Він вам не противився!
\end{tcolorbox}
\begin{tcolorbox}
\textsubscript{7} Отож, браття, довготерпіть аж до приходу Господа! Ось чекає рільник дорогоцінного плоду землі, довготерпить за нього, аж поки одержить дощ ранній та пізній.
\end{tcolorbox}
\begin{tcolorbox}
\textsubscript{8} Довготерпіть же й ви, зміцніть серця ваші, бо наблизився прихід Господній!
\end{tcolorbox}
\begin{tcolorbox}
\textsubscript{9} Не нарікайте один на одного, браття, щоб вас не засуджено, он Суддя стоїть перед дверима!
\end{tcolorbox}
\begin{tcolorbox}
\textsubscript{10} Візьміть, браття, пророків за приклад страждання та довготерпіння, вони промовляли Господнім Ім'ям!
\end{tcolorbox}
\begin{tcolorbox}
\textsubscript{11} Отож, за блаженних ми маємо тих, хто витерпів. Ви чули про Йовове терпіння та бачили Господній кінець його, що вельми Господь милостивий та щедрий.
\end{tcolorbox}
\begin{tcolorbox}
\textsubscript{12} А найперше, браття мої, не кляніться ні небом, ані землею, і ніякою іншою клятвою! Слово ж ваше хай буде: Так, так та Ні, ні, щоб не впасти вам в осуд.
\end{tcolorbox}
\begin{tcolorbox}
\textsubscript{13} Чи страждає хто з вас? Нехай молиться! Чи тішиться хтось? Хай співає псалми!
\end{tcolorbox}
\begin{tcolorbox}
\textsubscript{14} Чи хворіє хто з вас? Хай покличе пресвітерів Церкви, і над ним хай помоляться, намастивши його оливою в Господнє Ім'я,
\end{tcolorbox}
\begin{tcolorbox}
\textsubscript{15} і молитва віри вздоровить недужого, і Господь його підійме, а коли він гріхи був учинив, то вони йому простяться.
\end{tcolorbox}
\begin{tcolorbox}
\textsubscript{16} Отже, признавайтесь один перед одним у своїх прогріхах, і моліться один за одного, щоб вам уздоровитись. Бо дуже могутня ревна молитва праведного!
\end{tcolorbox}
\begin{tcolorbox}
\textsubscript{17} Ілля був людина, подібна до нас пристрастями, і він помолився молитвою, щоб дощу не було, і дощу не було на землі аж три роки й шість місяців...
\end{tcolorbox}
\begin{tcolorbox}
\textsubscript{18} І він знов помолився, і дощу дало небо, а земля вродила свій плід!
\end{tcolorbox}
\begin{tcolorbox}
\textsubscript{19} Браття мої, коли хто з-поміж вас заблудить від правди, і його хто наверне,
\end{tcolorbox}
\begin{tcolorbox}
\textsubscript{20} хай знає, що той, хто грішника навернув від його блудної дороги, той душу його спасає від смерти та безліч гріхів покриває!
\end{tcolorbox}
