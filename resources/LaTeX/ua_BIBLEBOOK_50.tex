\section{BOOK 49}
\subsection{CHAPTER 1}
\begin{tcolorbox}
\textsubscript{1} Павло й Тимофій, раби Христа Ісуса, до всіх святих у Христі Ісусі, що знаходяться в Филипах, з єпископами та дияконами:
\end{tcolorbox}
\begin{tcolorbox}
\textsubscript{2} благодать вам і мир від Бога, Отця нашого, і Господа Ісуса Христа!
\end{tcolorbox}
\begin{tcolorbox}
\textsubscript{3} Дякую Богові своєму при кожній згадці про вас,
\end{tcolorbox}
\begin{tcolorbox}
\textsubscript{4} і завжди в усякій молитві своїй за всіх вас чиню я молитву з радощами,
\end{tcolorbox}
\begin{tcolorbox}
\textsubscript{5} за участь вашу в Євангелії від першого дня аж дотепер.
\end{tcolorbox}
\begin{tcolorbox}
\textsubscript{6} Я певний того, що той, хто в вас розпочав добре діло, виконає його аж до дня Христа Ісуса.
\end{tcolorbox}
\begin{tcolorbox}
\textsubscript{7} Бо то справедливо мені думати це про всіх вас, бо я маю вас у серці, а ви всі в кайданах моїх, і в обороні, і в утвердженні Євангелії спільники мої в благодаті.
\end{tcolorbox}
\begin{tcolorbox}
\textsubscript{8} Бо Бог мені свідок, що тужу я за вами всіма в сердечній любові Христа Ісуса.
\end{tcolorbox}
\begin{tcolorbox}
\textsubscript{9} І молюсь я про те, щоб ваша любов примножалась ще більше та більше в пізнанні й усякім дослідженні,
\end{tcolorbox}
\begin{tcolorbox}
\textsubscript{10} щоб ви досліджували те, що краще, щоб чисті та цілі були Христового дня,
\end{tcolorbox}
\begin{tcolorbox}
\textsubscript{11} наповнені плодів праведности через Ісуса Христа, на славу та на хвалу Божу.
\end{tcolorbox}
\begin{tcolorbox}
\textsubscript{12} Бажаю ж я, браття, щоб відали ви, що те, що сталось мені, вийшло більше на успіх Євангелії,
\end{tcolorbox}
\begin{tcolorbox}
\textsubscript{13} бо в усій преторії та всім іншим стали відомі кайдани мої за Христа.
\end{tcolorbox}
\begin{tcolorbox}
\textsubscript{14} А багато братів у Господі через кайдани мої посміліли та ще більше відважилися Слово Боже звіщати безстрашно.
\end{tcolorbox}
\begin{tcolorbox}
\textsubscript{15} Одні, правда, і через заздрощі та колотнечу, другі ж із доброї волі Христа проповідують;
\end{tcolorbox}
\begin{tcolorbox}
\textsubscript{16} а інші з любови, знаючи, що я поставлений на оборону Євангелії;
\end{tcolorbox}
\begin{tcolorbox}
\textsubscript{17} а інші через підступ звіщають Христа нещиро, думаючи, що додадуть тягару до кайданів моїх.
\end{tcolorbox}
\begin{tcolorbox}
\textsubscript{18} Але що ж? У всякому разі, чи облудно, чи щиро, Христос проповідується, а тим я радію та й буду радіти.
\end{tcolorbox}
\begin{tcolorbox}
\textsubscript{19} Бо знаю, що це буде мені на спасіння через вашу молитву й допомогу Духа Ісуса Христа,
\end{tcolorbox}
\begin{tcolorbox}
\textsubscript{20} через чекання й надію мою, що я ні в чому не буду посоромлений, але цілою сміливістю, як завжди, так і тепер Христос буде звеличений у тілі моїм, чи то життям, чи то смертю.
\end{tcolorbox}
\begin{tcolorbox}
\textsubscript{21} Бо для мене життя то Христос, а смерть то надбання.
\end{tcolorbox}
\begin{tcolorbox}
\textsubscript{22} А коли життя в тілі то для мене плід діла, то не знаю, що вибрати.
\end{tcolorbox}
\begin{tcolorbox}
\textsubscript{23} Тягнуть мене одне й друге, хоч я маю бажання померти та бути з Христом, бо це значно ліпше.
\end{tcolorbox}
\begin{tcolorbox}
\textsubscript{24} А щоб полишатися в тілі, то це потрібніш ради вас.
\end{tcolorbox}
\begin{tcolorbox}
\textsubscript{25} І оце знаю певно, що залишусь я, і пробуватиму з вами всіма вам на користь та на радощі в вірі,
\end{tcolorbox}
\begin{tcolorbox}
\textsubscript{26} щоб ваша хвала через мене примножилася в Христі Ісусі, коли знову прийду я до вас.
\end{tcolorbox}
\begin{tcolorbox}
\textsubscript{27} Тільки живіть згідно з Христовою Євангелією, щоб, чи прийду я й побачу вас, чи й не бувши почув я про вас, що ви стоїте в однім дусі, борючись однодушно за віру євангельську,
\end{tcolorbox}
\begin{tcolorbox}
\textsubscript{28} і ні в чому не боячися противників; це їм доказ загибелі, вам же спасіння. А це від Бога!
\end{tcolorbox}
\begin{tcolorbox}
\textsubscript{29} Бо вчинено вам за Христа добродійство, не тільки вірувати в Нього, але і страждати за Нього,
\end{tcolorbox}
\begin{tcolorbox}
\textsubscript{30} маючи таку саму боротьбу, яку ви бачили в мені, а тепер чуєте про мене.
\end{tcolorbox}
\subsection{CHAPTER 2}
\begin{tcolorbox}
\textsubscript{1} Отож, коли є в Христі яка заохота, коли є яка потіха любови, коли є яка спільнота духа, коли є яке серце та милосердя,
\end{tcolorbox}
\begin{tcolorbox}
\textsubscript{2} то доповніть радість мою: щоб думали ви одне й те, щоб мали ту саму любов, одну згоду й один розум!
\end{tcolorbox}
\begin{tcolorbox}
\textsubscript{3} Не робіть нічого підступом або з чванливости, але в покорі майте один одного за більшого від себе.
\end{tcolorbox}
\begin{tcolorbox}
\textsubscript{4} Нехай кожен дбає не про своє, але кожен і про інших.
\end{tcolorbox}
\begin{tcolorbox}
\textsubscript{5} Нехай у вас будуть ті самі думки, що й у Христі Ісусі!
\end{tcolorbox}
\begin{tcolorbox}
\textsubscript{6} Він, бувши в Божій подобі, не вважав за захват бути Богові рівним,
\end{tcolorbox}
\begin{tcolorbox}
\textsubscript{7} але Він умалив Самого Себе, прийнявши вигляд раба, ставши подібним до людини; і подобою ставши, як людина,
\end{tcolorbox}
\begin{tcolorbox}
\textsubscript{8} Він упокорив Себе, бувши слухняний аж до смерти, і то смерти хресної...
\end{tcolorbox}
\begin{tcolorbox}
\textsubscript{9} Тому й Бог повищив Його, та дав Йому Ім'я, що вище над кожне ім'я,
\end{tcolorbox}
\begin{tcolorbox}
\textsubscript{10} щоб перед Ісусовим Ім'ям вклонялося кожне коліно небесних, і земних, і підземних,
\end{tcolorbox}
\begin{tcolorbox}
\textsubscript{11} і щоб кожен язик визнавав: Ісус Христос то Господь, на славу Бога Отця!
\end{tcolorbox}
\begin{tcolorbox}
\textsubscript{12} Отож, мої любі, як ви завжди слухняні були не тільки в моїй присутності, але значно більше тепер, у моїй відсутності, зо страхом і тремтінням виконуйте своє спасіння.
\end{tcolorbox}
\begin{tcolorbox}
\textsubscript{13} Бо то Бог викликає в вас і хотіння, і чин за доброю волею Своєю.
\end{tcolorbox}
\begin{tcolorbox}
\textsubscript{14} Робіть усе без нарікання та сумніву,
\end{tcolorbox}
\begin{tcolorbox}
\textsubscript{15} щоб були ви бездоганні та щирі, невинні діти Божі серед лукавого та розпусного роду, що в ньому ви сяєте, як світла в світі,
\end{tcolorbox}
\begin{tcolorbox}
\textsubscript{16} додержуючи слово життя на похвалу мені в день Христа, що я біг не надармо, що я працював не надармо.
\end{tcolorbox}
\begin{tcolorbox}
\textsubscript{17} Та хоч і стаю я жертвою при жертві і при службі вашої віри, я радію та тішуся разом із вами всіма.
\end{tcolorbox}
\begin{tcolorbox}
\textsubscript{18} Тіштесь тим самим і ви, і тіштеся разом зо мною!
\end{tcolorbox}
\begin{tcolorbox}
\textsubscript{19} Надіюся в Господі Ісусі незабаром послати до вас Тимофія, щоб і я зміцнів духом, розізнавши про вас.
\end{tcolorbox}
\begin{tcolorbox}
\textsubscript{20} Бо я однодумця не маю ні одного, щоб щиріше подбав він про вас.
\end{tcolorbox}
\begin{tcolorbox}
\textsubscript{21} Усі бо шукають свого, а не Христового Ісусового.
\end{tcolorbox}
\begin{tcolorbox}
\textsubscript{22} Та ви знаєте досвід його, бо він, немов батькові син, зо мною служив для Євангелії.
\end{tcolorbox}
\begin{tcolorbox}
\textsubscript{23} Отже, маю надію негайно послати цього, як тільки довідаюся, що буде зо мною.
\end{tcolorbox}
\begin{tcolorbox}
\textsubscript{24} Але в Господі маю надію, що й сам незабаром прибуду до вас.
\end{tcolorbox}
\begin{tcolorbox}
\textsubscript{25} Але я вважав за потрібне послати до вас брата Епафродита, свого співробітника та співбойовника, вашого апостола й служителя в потребі моїй,
\end{tcolorbox}
\begin{tcolorbox}
\textsubscript{26} бо він побивався за вами всіма, і сумував через те, що ви чули, що він хворував.
\end{tcolorbox}
\begin{tcolorbox}
\textsubscript{27} Бо смертельно він був хворував. Але змилувався Бог над ним, і не тільки над ним, але й надо мною, щоб я смутку на смуток не мав.
\end{tcolorbox}
\begin{tcolorbox}
\textsubscript{28} Отож, тим швидше послав я його, щоб тішились ви, його знову побачивши, і щоб без смутку я був.
\end{tcolorbox}
\begin{tcolorbox}
\textsubscript{29} Тож прийміть його в Господі з повною радістю, і майте в пошані таких,
\end{tcolorbox}
\begin{tcolorbox}
\textsubscript{30} бо за діло Христове наблизився був аж до смерти, наражаючи на небезпеку життя, щоб доповнити ваш нестаток служіння для мене.
\end{tcolorbox}
\subsection{CHAPTER 3}
\begin{tcolorbox}
\textsubscript{1} Зрештою, браття мої, радійте у Господі! Писати вам те саме не прикро мені, а для вас це навчальне.
\end{tcolorbox}
\begin{tcolorbox}
\textsubscript{2} Стережіться собак, стережіться працівників лихих, стережіться обрізання!
\end{tcolorbox}
\begin{tcolorbox}
\textsubscript{3} Бо обрізання то ми, що служимо Богові духом, а хвалимося Христом Ісусом, і не кладемо надії на тіло,
\end{tcolorbox}
\begin{tcolorbox}
\textsubscript{4} хоч і я міг би мати надію на тіло. Як хто інший на тіло надіятись думає, то тим більше я,
\end{tcolorbox}
\begin{tcolorbox}
\textsubscript{5} обрізаний восьмого дня, з роду Ізраїля, з племени Веніяминового, єврей із євреїв, фарисей за Законом.
\end{tcolorbox}
\begin{tcolorbox}
\textsubscript{6} Через горливість я був переслідував Церкву, бувши невинний, щодо правди в Законі.
\end{tcolorbox}
\begin{tcolorbox}
\textsubscript{7} Але те, що для мене було за надбання, те ради Христа я за втрату вважав.
\end{tcolorbox}
\begin{tcolorbox}
\textsubscript{8} Тож усе я вважаю за втрату ради переважного познання Христа Ісуса, мого Господа, що я ради Нього відмовився всього, і вважаю все за сміття, щоб придбати Христа,
\end{tcolorbox}
\begin{tcolorbox}
\textsubscript{9} щоб знайтися в Нім не з власною праведністю, яка від Закону, але з тією, що з віри в Христа, праведністю від Бога за вірою,
\end{tcolorbox}
\begin{tcolorbox}
\textsubscript{10} щоб пізнати Його й силу Його воскресення, та участь у муках Його, уподоблюючись Його смерті,
\end{tcolorbox}
\begin{tcolorbox}
\textsubscript{11} аби досягнути якось воскресення з мертвих.
\end{tcolorbox}
\begin{tcolorbox}
\textsubscript{12} Не тому, що я вже досягнув, або вже вдосконалився, але прагну, чи не досягну я того, чим і Христос Ісус досягнув був мене.
\end{tcolorbox}
\begin{tcolorbox}
\textsubscript{13} Браття, я себе не вважаю, що я досягнув. Та тільки, забуваючи те, що позаду, і спішачи до того, що попереду,
\end{tcolorbox}
\begin{tcolorbox}
\textsubscript{14} я женусь до мети за нагородою високого поклику Божого в Христі Ісусі.
\end{tcolorbox}
\begin{tcolorbox}
\textsubscript{15} Тож усі, хто досконалий, думаймо це; коли ж думаєте ви щось інше, то Бог вам відкриє й це.
\end{tcolorbox}
\begin{tcolorbox}
\textsubscript{16} Та до чого дійшли ми, поступаймо в тім самім далі.
\end{tcolorbox}
\begin{tcolorbox}
\textsubscript{17} Будьте до мене подібні, браття, і дивіться на тих, хто поводиться так, як маєте ви за взір нас.
\end{tcolorbox}
\begin{tcolorbox}
\textsubscript{18} Багато бо хто, що про них я вам часто казав, а тепер говорю навіть плачучи, поводяться, як вороги хреста Христового.
\end{tcolorbox}
\begin{tcolorbox}
\textsubscript{19} Їхній кінець то загибіль, шлунок їхній бог, а слава в їхньому соромі... Вони думають тільки про земне!
\end{tcolorbox}
\begin{tcolorbox}
\textsubscript{20} Життя ж наше на небесах, звідки ждемо й Спасителя, Господа Ісуса Христа,
\end{tcolorbox}
\begin{tcolorbox}
\textsubscript{21} Який перемінить тіло нашого пониження, щоб стало подібне до славного тіла Його, силою, якою Він може і все підкорити Собі.
\end{tcolorbox}
\subsection{CHAPTER 4}
\begin{tcolorbox}
\textsubscript{1} Отож, мої браття улюблені, за якими так сильно тужу, моя радосте й вінче, так у Господі стійте, улюблені!
\end{tcolorbox}
\begin{tcolorbox}
\textsubscript{2} Благаю Еводію, благаю й Синтихію думати однаково в Господі.
\end{tcolorbox}
\begin{tcolorbox}
\textsubscript{3} Так, благаю й тебе, товаришу вірний, допомагай тим, хто в боротьбі за Євангелію помагали мені та Климентові й іншим моїм співробітникам, яких імення записані в Книзі Життя.
\end{tcolorbox}
\begin{tcolorbox}
\textsubscript{4} Радійте в Господі завсіди, і знову кажу: радійте!
\end{tcolorbox}
\begin{tcolorbox}
\textsubscript{5} Ваша лагідність хай буде відома всім людям. Господь близько!
\end{tcolorbox}
\begin{tcolorbox}
\textsubscript{6} Ні про що не турбуйтесь, а в усьому нехай виявляються Богові ваші бажання молитвою й проханням з подякою.
\end{tcolorbox}
\begin{tcolorbox}
\textsubscript{7} І мир Божий, що вищий від усякого розуму, хай береже серця ваші та ваші думки у Христі Ісусі.
\end{tcolorbox}
\begin{tcolorbox}
\textsubscript{8} Наостанку, браття, що тільки правдиве, що тільки чесне, що тільки праведне, що тільки чисте, що тільки любе, що тільки гідне хвали, коли яка чеснота, коли яка похвала, думайте про це!
\end{tcolorbox}
\begin{tcolorbox}
\textsubscript{9} Чого ви від мене й навчилися, і прийняли, і чули та бачили, робіть те! І Бог миру буде з вами!
\end{tcolorbox}
\begin{tcolorbox}
\textsubscript{10} Я вельми потішився в Господі, що справді ви вже нових сил набули піклуватись про мене; ви й давніш піклувались, та часу сприятливого ви не мали.
\end{tcolorbox}
\begin{tcolorbox}
\textsubscript{11} Не за нестатком кажу, бо навчився я бути задоволеним із того, що маю.
\end{tcolorbox}
\begin{tcolorbox}
\textsubscript{12} Умію я й бути в упокоренні, умію бути й у достатку. Я привчився до всього й у всім: насищатися й голод терпіти, мати достаток і бути в недостачі.
\end{tcolorbox}
\begin{tcolorbox}
\textsubscript{13} Я все можу в Тім, Хто мене підкріпляє, в Ісусі Христі.
\end{tcolorbox}
\begin{tcolorbox}
\textsubscript{14} Тож ви добре зробили, що участь узяли в моїм горі.
\end{tcolorbox}
\begin{tcolorbox}
\textsubscript{15} І знаєте й ви, филип'яни, що на початку благовістя, коли я з Македонії вийшов, не прилучилась була жадна Церква до справи давання й приймання для мене, самі тільки ви,
\end{tcolorbox}
\begin{tcolorbox}
\textsubscript{16} що і раз, і вдруге мені на потреби мої посилали й до Солуня.
\end{tcolorbox}
\begin{tcolorbox}
\textsubscript{17} Кажу це не тому, щоб шукав я давання, я шукаю плоду, що примножується на річ вашу.
\end{tcolorbox}
\begin{tcolorbox}
\textsubscript{18} Та все я одержав, і маю достаток. Маю повно, прийнявши від Епафродита, що ви послали, як пахощі запашні, жертву приємну, Богові вгодну.
\end{tcolorbox}
\begin{tcolorbox}
\textsubscript{19} А мій Бог нехай виповнить вашу всяку потребу за Своїм багатством у Славі, у Христі Ісусі.
\end{tcolorbox}
\begin{tcolorbox}
\textsubscript{20} А Богові й нашому Отцеві слава на віки віків. Амінь.
\end{tcolorbox}
\begin{tcolorbox}
\textsubscript{21} Вітайте кожного святого у Христі Ісусі. Вітають вас браття, присутні зо мною.
\end{tcolorbox}
\begin{tcolorbox}
\textsubscript{22} Вітають вас усі святі, а найбільше ті, хто з кесаревого дому.
\end{tcolorbox}
\begin{tcolorbox}
\textsubscript{23} Благодать Господа Ісуса Христа зо всіма вами! Амінь.
\end{tcolorbox}
