\section{BOOK 102}
\subsection{CHAPTER 1}
\begin{tcolorbox}
\textsubscript{1} Пророческое видение, которое видел пророк Аввакум.
\end{tcolorbox}
\begin{tcolorbox}
\textsubscript{2} Доколе, Господи, я буду взывать, и Ты не слышишь, буду вопиять к Тебе о насилии, и Ты не спасаешь?
\end{tcolorbox}
\begin{tcolorbox}
\textsubscript{3} Для чего даешь мне видеть злодейство и смотреть на бедствия? Грабительство и насилие предо мною, и восстает вражда и поднимается раздор.
\end{tcolorbox}
\begin{tcolorbox}
\textsubscript{4} От этого закон потерял силу, и суда правильного нет: так как нечестивый одолевает праведного, то и суд происходит превратный.
\end{tcolorbox}
\begin{tcolorbox}
\textsubscript{5} Посмотрите между народами и внимательно вглядитесь, и вы сильно изумитесь; ибо Я сделаю во дни ваши такое дело, которому вы не поверили бы, если бы вам рассказывали.
\end{tcolorbox}
\begin{tcolorbox}
\textsubscript{6} Ибо вот, Я подниму Халдеев, народ жестокий и необузданный, который ходит по широтам земли, чтобы завладеть не принадлежащими ему селениями.
\end{tcolorbox}
\begin{tcolorbox}
\textsubscript{7} Страшен и грозен он; от него самого происходит суд его и власть его.
\end{tcolorbox}
\begin{tcolorbox}
\textsubscript{8} Быстрее барсов кони его и прытче вечерних волков; скачет в разные стороны конница его; издалека приходят всадники его, прилетают как орел, бросающийся на добычу.
\end{tcolorbox}
\begin{tcolorbox}
\textsubscript{9} Весь он идет для грабежа; устремив лице свое вперед, он забирает пленников, как песок.
\end{tcolorbox}
\begin{tcolorbox}
\textsubscript{10} И над царями он издевается, и князья служат ему посмешищем; над всякою крепостью он смеется: насыплет осадный вал и берет ее.
\end{tcolorbox}
\begin{tcolorbox}
\textsubscript{11} Тогда надмевается дух его, и он ходит и буйствует; сила его--бог его.
\end{tcolorbox}
\begin{tcolorbox}
\textsubscript{12} Но не Ты ли издревле Господь Бог мой, Святый мой? мы не умрем! Ты, Господи, только для суда попустил его. Скала моя! для наказания Ты назначил его.
\end{tcolorbox}
\begin{tcolorbox}
\textsubscript{13} Чистым очам Твоим не свойственно глядеть на злодеяния, и смотреть на притеснение Ты не можешь; для чего же Ты смотришь на злодеев и безмолвствуешь, когда нечестивец поглощает того, кто праведнее его,
\end{tcolorbox}
\begin{tcolorbox}
\textsubscript{14} и оставляешь людей как рыбу в море, как пресмыкающихся, у которых нет властителя?
\end{tcolorbox}
\begin{tcolorbox}
\textsubscript{15} Всех их таскает удою, захватывает в сеть свою и забирает их в неводы свои, и оттого радуется и торжествует.
\end{tcolorbox}
\begin{tcolorbox}
\textsubscript{16} За то приносит жертвы сети своей и кадит неводу своему, потому что от них тучна часть его и роскошна пища его.
\end{tcolorbox}
\begin{tcolorbox}
\textsubscript{17} Неужели для этого он должен опорожнять свою сеть и непрестанно избивать народы без пощады?
\end{tcolorbox}
\subsection{CHAPTER 2}
\begin{tcolorbox}
\textsubscript{1} На стражу мою стал я и, стоя на башне, наблюдал, чтобы узнать, что скажет Он во мне, и что мне отвечать по жалобе моей?
\end{tcolorbox}
\begin{tcolorbox}
\textsubscript{2} И отвечал мне Господь и сказал: запиши видение и начертай ясно на скрижалях, чтобы читающий легко мог прочитать,
\end{tcolorbox}
\begin{tcolorbox}
\textsubscript{3} ибо видение относится еще к определенному времени и говорит о конце и не обманет; и хотя бы и замедлило, жди его, ибо непременно сбудется, не отменится.
\end{tcolorbox}
\begin{tcolorbox}
\textsubscript{4} Вот, душа надменная не успокоится, а праведный своею верою жив будет.
\end{tcolorbox}
\begin{tcolorbox}
\textsubscript{5} Надменный человек, как бродящее вино, не успокаивается, так что расширяет душу свою как ад, и как смерть он ненасытен, и собирает к себе все народы, и захватывает себе все племена.
\end{tcolorbox}
\begin{tcolorbox}
\textsubscript{6} Но не все ли они будут произносить о нем притчу и насмешливую песнь: 'горе тому, кто без меры обогащает себя не своим, --на долго ли? --и обременяет себя залогами!'
\end{tcolorbox}
\begin{tcolorbox}
\textsubscript{7} Не восстанут ли внезапно те, которые будут терзать тебя, и не поднимутся ли против тебя грабители, и ты достанешься им на расхищение?
\end{tcolorbox}
\begin{tcolorbox}
\textsubscript{8} Так как ты ограбил многие народы, то и тебя ограбят все остальные народы за пролитие крови человеческой, за разорение страны, города и всех живущих в нем.
\end{tcolorbox}
\begin{tcolorbox}
\textsubscript{9} Горе тому, кто жаждет неправедных приобретений для дома своего, чтобы устроить гнездо свое на высоте и тем обезопасить себя от руки несчастья!
\end{tcolorbox}
\begin{tcolorbox}
\textsubscript{10} Бесславие измыслил ты для твоего дома, истребляя многие народы, и согрешил против души твоей.
\end{tcolorbox}
\begin{tcolorbox}
\textsubscript{11} Камни из стен возопиют и перекладины из дерева будут отвечать им:
\end{tcolorbox}
\begin{tcolorbox}
\textsubscript{12} 'горе строящему город на крови и созидающему крепости неправдою!'
\end{tcolorbox}
\begin{tcolorbox}
\textsubscript{13} Вот, не от Господа ли Саваофа это, что народы трудятся для огня и племена мучат себя напрасно?
\end{tcolorbox}
\begin{tcolorbox}
\textsubscript{14} Ибо земля наполнится познанием славы Господа, как воды наполняют море.
\end{tcolorbox}
\begin{tcolorbox}
\textsubscript{15} Горе тебе, который подаешь ближнему твоему питье с примесью злобы твоей и делаешь его пьяным, чтобы видеть срамоту его!
\end{tcolorbox}
\begin{tcolorbox}
\textsubscript{16} Ты пресытился стыдом вместо славы; пей же и ты и показывай срамоту, --обратится и к тебе чаша десницы Господней и посрамление на славу твою.
\end{tcolorbox}
\begin{tcolorbox}
\textsubscript{17} Ибо злодейство твое на Ливане обрушится на тебя за истребление устрашенных животных, за пролитие крови человеческой, за опустошение страны, города и всех живущих в нем.
\end{tcolorbox}
\begin{tcolorbox}
\textsubscript{18} Что за польза от истукана, сделанного художником, этого литаго лжеучителя, хотя ваятель, делая немые кумиры, полагается на свое произведение?
\end{tcolorbox}
\begin{tcolorbox}
\textsubscript{19} Горе тому, кто говорит дереву: 'встань!' и бессловесному камню: 'пробудись!' Научит ли он чему-нибудь? Вот, он обложен золотом и серебром, но дыхания в нем нет.
\end{tcolorbox}
\begin{tcolorbox}
\textsubscript{20} А Господь--во святом храме Своем: да молчит вся земля пред лицем Его!
\end{tcolorbox}
\subsection{CHAPTER 3}
\begin{tcolorbox}
\textsubscript{1} Молитва Аввакума пророка, для пения.
\end{tcolorbox}
\begin{tcolorbox}
\textsubscript{2} Господи! услышал я слух Твой и убоялся. Господи! соверши дело Твое среди лет, среди лет яви его; во гневе вспомни о милости.
\end{tcolorbox}
\begin{tcolorbox}
\textsubscript{3} Бог от Фемана грядет и Святый--от горы Фаран. Покрыло небеса величие Его, и славою Его наполнилась земля.
\end{tcolorbox}
\begin{tcolorbox}
\textsubscript{4} Блеск ее--как солнечный свет; от руки Его лучи, и здесь тайник Его силы!
\end{tcolorbox}
\begin{tcolorbox}
\textsubscript{5} Пред лицем Его идет язва, а по стопам Его--жгучий ветер.
\end{tcolorbox}
\begin{tcolorbox}
\textsubscript{6} Он стал и поколебал землю; воззрел, и в трепет привел народы; вековые горы распались, первобытные холмы опали; пути Его вечные.
\end{tcolorbox}
\begin{tcolorbox}
\textsubscript{7} Грустными видел я шатры Ефиопские; сотряслись палатки земли Мадиамской.
\end{tcolorbox}
\begin{tcolorbox}
\textsubscript{8} Разве на реки воспылал, Господи, гнев Твой? разве на реки--негодование Твое, или на море--ярость Твоя, что Ты восшел на коней Твоих, на колесницы Твои спасительные?
\end{tcolorbox}
\begin{tcolorbox}
\textsubscript{9} Ты обнажил лук Твой по клятвенному обетованию, данному коленам. Ты потоками рассек землю.
\end{tcolorbox}
\begin{tcolorbox}
\textsubscript{10} Увидев Тебя, вострепетали горы, ринулись воды; бездна дала голос свой, высоко подняла руки свои;
\end{tcolorbox}
\begin{tcolorbox}
\textsubscript{11} солнце и луна остановились на месте своем пред светом летающих стрел Твоих, пред сиянием сверкающих копьев Твоих.
\end{tcolorbox}
\begin{tcolorbox}
\textsubscript{12} Во гневе шествуешь Ты по земле и в негодовании попираешь народы.
\end{tcolorbox}
\begin{tcolorbox}
\textsubscript{13} Ты выступаешь для спасения народа Твоего, для спасения помазанного Твоего. Ты сокрушаешь главу нечестивого дома, обнажая его от основания до верха.
\end{tcolorbox}
\begin{tcolorbox}
\textsubscript{14} Ты пронзаешь копьями его главу вождей его, когда они как вихрь ринулись разбить меня, в радости, как бы думая поглотить бедного скрытно.
\end{tcolorbox}
\begin{tcolorbox}
\textsubscript{15} Ты с конями Твоими проложил путь по морю, через пучину великих вод.
\end{tcolorbox}
\begin{tcolorbox}
\textsubscript{16} Я услышал, и вострепетала внутренность моя; при вести о сем задрожали губы мои, боль проникла в кости мои, и колеблется место подо мною; а я должен быть спокоен в день бедствия, когда придет на народ мой грабитель его.
\end{tcolorbox}
\begin{tcolorbox}
\textsubscript{17} Хотя бы не расцвела смоковница и не было плода на виноградных лозах, и маслина изменила, и нива не дала пищи, хотя бы не стало овец в загоне и рогатого скота в стойлах, --
\end{tcolorbox}
\begin{tcolorbox}
\textsubscript{18} но и тогда я буду радоваться о Господе и веселиться о Боге спасения моего.
\end{tcolorbox}
\begin{tcolorbox}
\textsubscript{19} Господь Бог--сила моя: Он сделает ноги мои как у оленя и на высоты мои возведет меня! (Начальнику хора).
\end{tcolorbox}
