\section{BOOK 51}
\subsection{CHAPTER 1}
\begin{tcolorbox}
\textsubscript{1} Был человек в земле Уц, имя его Иов; и был человек этот непорочен, справедлив и богобоязнен и удалялся от зла.
\end{tcolorbox}
\begin{tcolorbox}
\textsubscript{2} И родились у него семь сыновей и три дочери.
\end{tcolorbox}
\begin{tcolorbox}
\textsubscript{3} Имения у него было: семь тысяч мелкого скота, три тысячи верблюдов, пятьсот пар волов и пятьсот ослиц и весьма много прислуги; и был человек этот знаменитее всех сынов Востока.
\end{tcolorbox}
\begin{tcolorbox}
\textsubscript{4} Сыновья его сходились, делая пиры каждый в своем доме в свой день, и посылали и приглашали трех сестер своих есть и пить с ними.
\end{tcolorbox}
\begin{tcolorbox}
\textsubscript{5} Когда круг пиршественных дней совершался, Иов посылал [за ними] и освящал их и, вставая рано утром, возносил всесожжения по числу всех их. Ибо говорил Иов: может быть, сыновья мои согрешили и похулили Бога в сердце своем. Так делал Иов во все [такие] дни.
\end{tcolorbox}
\begin{tcolorbox}
\textsubscript{6} И был день, когда пришли сыны Божии предстать пред Господа; между ними пришел и сатана.
\end{tcolorbox}
\begin{tcolorbox}
\textsubscript{7} И сказал Господь сатане: откуда ты пришел? И отвечал сатана Господу и сказал: я ходил по земле и обошел ее.
\end{tcolorbox}
\begin{tcolorbox}
\textsubscript{8} И сказал Господь сатане: обратил ли ты внимание твое на раба Моего Иова? ибо нет такого, как он, на земле: человек непорочный, справедливый, богобоязненный и удаляющийся от зла.
\end{tcolorbox}
\begin{tcolorbox}
\textsubscript{9} И отвечал сатана Господу и сказал: разве даром богобоязнен Иов?
\end{tcolorbox}
\begin{tcolorbox}
\textsubscript{10} Не Ты ли кругом оградил его и дом его и все, что у него? Дело рук его Ты благословил, и стада его распространяются по земле;
\end{tcolorbox}
\begin{tcolorbox}
\textsubscript{11} но простри руку Твою и коснись всего, что у него, --благословит ли он Тебя?
\end{tcolorbox}
\begin{tcolorbox}
\textsubscript{12} И сказал Господь сатане: вот, все, что у него, в руке твоей; только на него не простирай руки твоей. И отошел сатана от лица Господня.
\end{tcolorbox}
\begin{tcolorbox}
\textsubscript{13} И был день, когда сыновья его и дочери его ели и вино пили в доме первородного брата своего.
\end{tcolorbox}
\begin{tcolorbox}
\textsubscript{14} И [вот], приходит вестник к Иову и говорит:
\end{tcolorbox}
\begin{tcolorbox}
\textsubscript{15} волы орали, и ослицы паслись подле них, как напали Савеяне и взяли их, а отроков поразили острием меча; и спасся только я один, чтобы возвестить тебе.
\end{tcolorbox}
\begin{tcolorbox}
\textsubscript{16} Еще он говорил, как приходит другой и сказывает: огонь Божий упал с неба и опалил овец и отроков и пожрал их; и спасся только я один, чтобы возвестить тебе.
\end{tcolorbox}
\begin{tcolorbox}
\textsubscript{17} Еще он говорил, как приходит другой и сказывает: Халдеи расположились тремя отрядами и бросились на верблюдов и взяли их, а отроков поразили острием меча; и спасся только я один, чтобы возвестить тебе.
\end{tcolorbox}
\begin{tcolorbox}
\textsubscript{18} Еще этот говорил, приходит другой и сказывает: сыновья твои и дочери твои ели и вино пили в доме первородного брата своего;
\end{tcolorbox}
\begin{tcolorbox}
\textsubscript{19} и вот, большой ветер пришел от пустыни и охватил четыре угла дома, и дом упал на отроков, и они умерли; и спасся только я один, чтобы возвестить тебе.
\end{tcolorbox}
\begin{tcolorbox}
\textsubscript{20} Тогда Иов встал и разодрал верхнюю одежду свою, остриг голову свою и пал на землю и поклонился
\end{tcolorbox}
\begin{tcolorbox}
\textsubscript{21} и сказал: наг я вышел из чрева матери моей, наг и возвращусь. Господь дал, Господь и взял; да будет имя Господне благословенно!
\end{tcolorbox}
\begin{tcolorbox}
\textsubscript{22} Во всем этом не согрешил Иов и не произнес ничего неразумного о Боге.
\end{tcolorbox}
\subsection{CHAPTER 2}
\begin{tcolorbox}
\textsubscript{1} Был день, когда пришли сыны Божии предстать пред Господа; между ними пришел и сатана предстать пред Господа.
\end{tcolorbox}
\begin{tcolorbox}
\textsubscript{2} И сказал Господь сатане: откуда ты пришел? И отвечал сатана Господу и сказал: я ходил по земле и обошел ее.
\end{tcolorbox}
\begin{tcolorbox}
\textsubscript{3} И сказал Господь сатане: обратил ли ты внимание твое на раба Моего Иова? ибо нет такого, как он, на земле: человек непорочный, справедливый, богобоязненный и удаляющийся от зла, и доселе тверд в своей непорочности; а ты возбуждал Меня против него, чтобы погубить его безвинно.
\end{tcolorbox}
\begin{tcolorbox}
\textsubscript{4} И отвечал сатана Господу и сказал: кожу за кожу, а за жизнь свою отдаст человек все, что есть у него;
\end{tcolorbox}
\begin{tcolorbox}
\textsubscript{5} но простри руку Твою и коснись кости его и плоти его, --благословит ли он Тебя?
\end{tcolorbox}
\begin{tcolorbox}
\textsubscript{6} И сказал Господь сатане: вот, он в руке твоей, только душу его сбереги.
\end{tcolorbox}
\begin{tcolorbox}
\textsubscript{7} И отошел сатана от лица Господня и поразил Иова проказою лютою от подошвы ноги его по самое темя его.
\end{tcolorbox}
\begin{tcolorbox}
\textsubscript{8} И взял он себе черепицу, чтобы скоблить себя ею, и сел в пепел.
\end{tcolorbox}
\begin{tcolorbox}
\textsubscript{9} И сказала ему жена его: ты все еще тверд в непорочности твоей! похули Бога и умри. (1)
\end{tcolorbox}
\begin{tcolorbox}
\textsubscript{10} Но он сказал ей: ты говоришь как одна из безумных: неужели доброе мы будем принимать от Бога, а злого не будем принимать? Во всем этом не согрешил Иов устами своими.
\end{tcolorbox}
\begin{tcolorbox}
\textsubscript{11} И услышали трое друзей Иова о всех этих несчастьях, постигших его, и пошли каждый из своего места: Елифаз Феманитянин, Вилдад Савхеянин и Софар Наамитянин, и сошлись, чтобы идти вместе сетовать с ним и утешать его.
\end{tcolorbox}
\begin{tcolorbox}
\textsubscript{12} И подняв глаза свои издали, они не узнали его; и возвысили голос свой и зарыдали; и разодрал каждый верхнюю одежду свою, и бросали пыль над головами своими к небу.
\end{tcolorbox}
\begin{tcolorbox}
\textsubscript{13} И сидели с ним на земле семь дней и семь ночей; и никто не говорил ему ни слова, ибо видели, что страдание его весьма велико.
\end{tcolorbox}
\subsection{CHAPTER 3}
\begin{tcolorbox}
\textsubscript{1} После того открыл Иов уста свои и проклял день свой.
\end{tcolorbox}
\begin{tcolorbox}
\textsubscript{2} И начал Иов и сказал:
\end{tcolorbox}
\begin{tcolorbox}
\textsubscript{3} погибни день, в который я родился, и ночь, в которую сказано: зачался человек!
\end{tcolorbox}
\begin{tcolorbox}
\textsubscript{4} День тот да будет тьмою; да не взыщет его Бог свыше, и да не воссияет над ним свет!
\end{tcolorbox}
\begin{tcolorbox}
\textsubscript{5} Да омрачит его тьма и тень смертная, да обложит его туча, да страшатся его, как палящего зноя!
\end{tcolorbox}
\begin{tcolorbox}
\textsubscript{6} Ночь та, --да обладает ею мрак, да не сочтется она в днях года, да не войдет в число месяцев!
\end{tcolorbox}
\begin{tcolorbox}
\textsubscript{7} О! ночь та--да будет она безлюдна; да не войдет в нее веселье!
\end{tcolorbox}
\begin{tcolorbox}
\textsubscript{8} Да проклянут ее проклинающие день, способные разбудить левиафана!
\end{tcolorbox}
\begin{tcolorbox}
\textsubscript{9} Да померкнут звезды рассвета ее: пусть ждет она света, и он не приходит, и да не увидит она ресниц денницы
\end{tcolorbox}
\begin{tcolorbox}
\textsubscript{10} за то, что не затворила дверей чрева [матери] моей и не сокрыла горести от очей моих!
\end{tcolorbox}
\begin{tcolorbox}
\textsubscript{11} Для чего не умер я, выходя из утробы, и не скончался, когда вышел из чрева?
\end{tcolorbox}
\begin{tcolorbox}
\textsubscript{12} Зачем приняли меня колени? зачем было мне сосать сосцы?
\end{tcolorbox}
\begin{tcolorbox}
\textsubscript{13} Теперь бы лежал я и почивал; спал бы, и мне было бы покойно
\end{tcolorbox}
\begin{tcolorbox}
\textsubscript{14} с царями и советниками земли, которые застраивали для себя пустыни,
\end{tcolorbox}
\begin{tcolorbox}
\textsubscript{15} или с князьями, у которых было золото, и которые наполняли домы свои серебром;
\end{tcolorbox}
\begin{tcolorbox}
\textsubscript{16} или, как выкидыш сокрытый, я не существовал бы, как младенцы, не увидевшие света.
\end{tcolorbox}
\begin{tcolorbox}
\textsubscript{17} Там беззаконные перестают наводить страх, и там отдыхают истощившиеся в силах.
\end{tcolorbox}
\begin{tcolorbox}
\textsubscript{18} Там узники вместе наслаждаются покоем и не слышат криков приставника.
\end{tcolorbox}
\begin{tcolorbox}
\textsubscript{19} Малый и великий там равны, и раб свободен от господина своего.
\end{tcolorbox}
\begin{tcolorbox}
\textsubscript{20} На что дан страдальцу свет, и жизнь огорченным душею,
\end{tcolorbox}
\begin{tcolorbox}
\textsubscript{21} которые ждут смерти, и нет ее, которые вырыли бы ее охотнее, нежели клад,
\end{tcolorbox}
\begin{tcolorbox}
\textsubscript{22} обрадовались бы до восторга, восхитились бы, что нашли гроб?
\end{tcolorbox}
\begin{tcolorbox}
\textsubscript{23} [На что дан свет] человеку, которого путь закрыт, и которого Бог окружил мраком?
\end{tcolorbox}
\begin{tcolorbox}
\textsubscript{24} Вздохи мои предупреждают хлеб мой, и стоны мои льются, как вода,
\end{tcolorbox}
\begin{tcolorbox}
\textsubscript{25} ибо ужасное, чего я ужасался, то и постигло меня; и чего я боялся, то и пришло ко мне.
\end{tcolorbox}
\begin{tcolorbox}
\textsubscript{26} Нет мне мира, нет покоя, нет отрады: постигло несчастье.
\end{tcolorbox}
\subsection{CHAPTER 4}
\begin{tcolorbox}
\textsubscript{1} И отвечал Елифаз Феманитянин и сказал:
\end{tcolorbox}
\begin{tcolorbox}
\textsubscript{2} [если] попытаемся мы [сказать] к тебе слово, --не тяжело ли будет тебе? Впрочем кто может возбранить слову!
\end{tcolorbox}
\begin{tcolorbox}
\textsubscript{3} Вот, ты наставлял многих и опустившиеся руки поддерживал,
\end{tcolorbox}
\begin{tcolorbox}
\textsubscript{4} падающего восставляли слова твои, и гнущиеся колени ты укреплял.
\end{tcolorbox}
\begin{tcolorbox}
\textsubscript{5} А теперь дошло до тебя, и ты изнемог; коснулось тебя, и ты упал духом.
\end{tcolorbox}
\begin{tcolorbox}
\textsubscript{6} Богобоязненность твоя не должна ли быть твоею надеждою, и непорочность путей твоих--упованием твоим?
\end{tcolorbox}
\begin{tcolorbox}
\textsubscript{7} Вспомни же, погибал ли кто невинный, и где праведные бывали искореняемы?
\end{tcolorbox}
\begin{tcolorbox}
\textsubscript{8} Как я видал, то оравшие нечестие и сеявшие зло пожинают его;
\end{tcolorbox}
\begin{tcolorbox}
\textsubscript{9} от дуновения Божия погибают и от духа гнева Его исчезают.
\end{tcolorbox}
\begin{tcolorbox}
\textsubscript{10} Рев льва и голос рыкающего [умолкает], и зубы скимнов сокрушаются;
\end{tcolorbox}
\begin{tcolorbox}
\textsubscript{11} могучий лев погибает без добычи, и дети львицы рассеиваются.
\end{tcolorbox}
\begin{tcolorbox}
\textsubscript{12} И вот, ко мне тайно принеслось слово, и ухо мое приняло нечто от него.
\end{tcolorbox}
\begin{tcolorbox}
\textsubscript{13} Среди размышлений о ночных видениях, когда сон находит на людей,
\end{tcolorbox}
\begin{tcolorbox}
\textsubscript{14} объял меня ужас и трепет и потряс все кости мои.
\end{tcolorbox}
\begin{tcolorbox}
\textsubscript{15} И дух прошел надо мною; дыбом стали волосы на мне.
\end{tcolorbox}
\begin{tcolorbox}
\textsubscript{16} Он стал, --но я не распознал вида его, --только облик был пред глазами моими; тихое веяние, --и я слышу голос:
\end{tcolorbox}
\begin{tcolorbox}
\textsubscript{17} человек праведнее ли Бога? и муж чище ли Творца своего?
\end{tcolorbox}
\begin{tcolorbox}
\textsubscript{18} Вот, Он и слугам Своим не доверяет и в Ангелах Своих усматривает недостатки:
\end{tcolorbox}
\begin{tcolorbox}
\textsubscript{19} тем более--в обитающих в храминах из брения, которых основание прах, которые истребляются скорее моли.
\end{tcolorbox}
\begin{tcolorbox}
\textsubscript{20} Между утром и вечером они распадаются; не увидишь, как они вовсе исчезнут.
\end{tcolorbox}
\begin{tcolorbox}
\textsubscript{21} Не погибают ли с ними и достоинства их? Они умирают, не достигнув мудрости.
\end{tcolorbox}
\subsection{CHAPTER 5}
\begin{tcolorbox}
\textsubscript{1} Взывай, если есть отвечающий тебе. И к кому из святых обратишься ты?
\end{tcolorbox}
\begin{tcolorbox}
\textsubscript{2} Так, глупца убивает гневливость, и несмысленного губит раздражительность.
\end{tcolorbox}
\begin{tcolorbox}
\textsubscript{3} Видел я, как глупец укореняется, и тотчас проклял дом его.
\end{tcolorbox}
\begin{tcolorbox}
\textsubscript{4} Дети его далеки от счастья, их будут бить у ворот, и не будет заступника.
\end{tcolorbox}
\begin{tcolorbox}
\textsubscript{5} Жатву его съест голодный и из-за терна возьмет ее, и жаждущие поглотят имущество его.
\end{tcolorbox}
\begin{tcolorbox}
\textsubscript{6} Так, не из праха выходит горе, и не из земли вырастает беда;
\end{tcolorbox}
\begin{tcolorbox}
\textsubscript{7} но человек рождается на страдание, [как] искры, чтобы устремляться вверх.
\end{tcolorbox}
\begin{tcolorbox}
\textsubscript{8} Но я к Богу обратился бы, предал бы дело мое Богу,
\end{tcolorbox}
\begin{tcolorbox}
\textsubscript{9} Который творит дела великие и неисследимые, чудные без числа,
\end{tcolorbox}
\begin{tcolorbox}
\textsubscript{10} дает дождь на лице земли и посылает воды на лице полей;
\end{tcolorbox}
\begin{tcolorbox}
\textsubscript{11} униженных поставляет на высоту, и сетующие возносятся во спасение.
\end{tcolorbox}
\begin{tcolorbox}
\textsubscript{12} Он разрушает замыслы коварных, и руки их не довершают предприятия.
\end{tcolorbox}
\begin{tcolorbox}
\textsubscript{13} Он уловляет мудрецов их же лукавством, и совет хитрых становится тщетным:
\end{tcolorbox}
\begin{tcolorbox}
\textsubscript{14} днем они встречают тьму и в полдень ходят ощупью, как ночью.
\end{tcolorbox}
\begin{tcolorbox}
\textsubscript{15} Он спасает бедного от меча, от уст их и от руки сильного.
\end{tcolorbox}
\begin{tcolorbox}
\textsubscript{16} И есть несчастному надежда, и неправда затворяет уста свои.
\end{tcolorbox}
\begin{tcolorbox}
\textsubscript{17} Блажен человек, которого вразумляет Бог, и потому наказания Вседержителева не отвергай,
\end{tcolorbox}
\begin{tcolorbox}
\textsubscript{18} ибо Он причиняет раны и Сам обвязывает их; Он поражает, и Его же руки врачуют.
\end{tcolorbox}
\begin{tcolorbox}
\textsubscript{19} В шести бедах спасет тебя, и в седьмой не коснется тебя зло.
\end{tcolorbox}
\begin{tcolorbox}
\textsubscript{20} Во время голода избавит тебя от смерти, и на войне--от руки меча.
\end{tcolorbox}
\begin{tcolorbox}
\textsubscript{21} От бича языка укроешь себя и не убоишься опустошения, когда оно придет.
\end{tcolorbox}
\begin{tcolorbox}
\textsubscript{22} Опустошению и голоду посмеешься и зверей земли не убоишься,
\end{tcolorbox}
\begin{tcolorbox}
\textsubscript{23} ибо с камнями полевыми у тебя союз, и звери полевые в мире с тобою.
\end{tcolorbox}
\begin{tcolorbox}
\textsubscript{24} И узнаешь, что шатер твой в безопасности, и будешь смотреть за домом твоим, и не согрешишь.
\end{tcolorbox}
\begin{tcolorbox}
\textsubscript{25} И увидишь, что семя твое многочисленно, и отрасли твои, как трава на земле.
\end{tcolorbox}
\begin{tcolorbox}
\textsubscript{26} Войдешь во гроб в зрелости, как укладываются снопы пшеницы в свое время.
\end{tcolorbox}
\begin{tcolorbox}
\textsubscript{27} Вот, что мы дознали; так оно и есть: выслушай это и заметь для себя.
\end{tcolorbox}
\subsection{CHAPTER 6}
\begin{tcolorbox}
\textsubscript{1} И отвечал Иов и сказал:
\end{tcolorbox}
\begin{tcolorbox}
\textsubscript{2} о, если бы верно взвешены были вопли мои, и вместе с ними положили на весы страдание мое!
\end{tcolorbox}
\begin{tcolorbox}
\textsubscript{3} Оно верно перетянуло бы песок морей! Оттого слова мои неистовы.
\end{tcolorbox}
\begin{tcolorbox}
\textsubscript{4} Ибо стрелы Вседержителя во мне; яд их пьет дух мой; ужасы Божии ополчились против меня.
\end{tcolorbox}
\begin{tcolorbox}
\textsubscript{5} Ревет ли дикий осел на траве? мычит ли бык у месива своего?
\end{tcolorbox}
\begin{tcolorbox}
\textsubscript{6} Едят ли безвкусное без соли, и есть ли вкус в яичном белке?
\end{tcolorbox}
\begin{tcolorbox}
\textsubscript{7} До чего не хотела коснуться душа моя, то составляет отвратительную пищу мою.
\end{tcolorbox}
\begin{tcolorbox}
\textsubscript{8} О, когда бы сбылось желание мое и чаяние мое исполнил Бог!
\end{tcolorbox}
\begin{tcolorbox}
\textsubscript{9} О, если бы благоволил Бог сокрушить меня, простер руку Свою и сразил меня!
\end{tcolorbox}
\begin{tcolorbox}
\textsubscript{10} Это было бы еще отрадою мне, и я крепился бы в моей беспощадной болезни, ибо я не отвергся изречений Святаго.
\end{tcolorbox}
\begin{tcolorbox}
\textsubscript{11} Что за сила у меня, чтобы надеяться мне? и какой конец, чтобы длить мне жизнь мою?
\end{tcolorbox}
\begin{tcolorbox}
\textsubscript{12} Твердость ли камней твердость моя? и медь ли плоть моя?
\end{tcolorbox}
\begin{tcolorbox}
\textsubscript{13} Есть ли во мне помощь для меня, и есть ли для меня какая опора?
\end{tcolorbox}
\begin{tcolorbox}
\textsubscript{14} К страждущему должно быть сожаление от друга его, если только он не оставил страха к Вседержителю.
\end{tcolorbox}
\begin{tcolorbox}
\textsubscript{15} Но братья мои неверны, как поток, как быстро текущие ручьи,
\end{tcolorbox}
\begin{tcolorbox}
\textsubscript{16} которые черны от льда и в которых скрывается снег.
\end{tcolorbox}
\begin{tcolorbox}
\textsubscript{17} Когда становится тепло, они умаляются, а во время жары исчезают с мест своих.
\end{tcolorbox}
\begin{tcolorbox}
\textsubscript{18} Уклоняют они направление путей своих, заходят в пустыню и теряются;
\end{tcolorbox}
\begin{tcolorbox}
\textsubscript{19} смотрят на них дороги Фемайские, надеются на них пути Савейские,
\end{tcolorbox}
\begin{tcolorbox}
\textsubscript{20} но остаются пристыженными в своей надежде; приходят туда и от стыда краснеют.
\end{tcolorbox}
\begin{tcolorbox}
\textsubscript{21} Так и вы теперь ничто: увидели страшное и испугались.
\end{tcolorbox}
\begin{tcolorbox}
\textsubscript{22} Говорил ли я: дайте мне, или от достатка вашего заплатите за меня;
\end{tcolorbox}
\begin{tcolorbox}
\textsubscript{23} и избавьте меня от руки врага, и от руки мучителей выкупите меня?
\end{tcolorbox}
\begin{tcolorbox}
\textsubscript{24} Научите меня, и я замолчу; укажите, в чем я погрешил.
\end{tcolorbox}
\begin{tcolorbox}
\textsubscript{25} Как сильны слова правды! Но что доказывают обличения ваши?
\end{tcolorbox}
\begin{tcolorbox}
\textsubscript{26} Вы придумываете речи для обличения? На ветер пускаете слова ваши.
\end{tcolorbox}
\begin{tcolorbox}
\textsubscript{27} Вы нападаете на сироту и роете яму другу вашему.
\end{tcolorbox}
\begin{tcolorbox}
\textsubscript{28} Но прошу вас, взгляните на меня; буду ли я говорить ложь пред лицем вашим?
\end{tcolorbox}
\begin{tcolorbox}
\textsubscript{29} Пересмотрите, есть ли неправда? пересмотрите, --правда моя.
\end{tcolorbox}
\begin{tcolorbox}
\textsubscript{30} Есть ли на языке моем неправда? Неужели гортань моя не может различить горечи?
\end{tcolorbox}
\subsection{CHAPTER 7}
\begin{tcolorbox}
\textsubscript{1} Не определено ли человеку время на земле, и дни его не то же ли, что дни наемника?
\end{tcolorbox}
\begin{tcolorbox}
\textsubscript{2} Как раб жаждет тени, и как наемник ждет окончания работы своей,
\end{tcolorbox}
\begin{tcolorbox}
\textsubscript{3} так я получил в удел месяцы суетные, и ночи горестные отчислены мне.
\end{tcolorbox}
\begin{tcolorbox}
\textsubscript{4} Когда ложусь, то говорю: 'когда-то встану?', а вечер длится, и я ворочаюсь досыта до самого рассвета.
\end{tcolorbox}
\begin{tcolorbox}
\textsubscript{5} Тело мое одето червями и пыльными струпами; кожа моя лопается и гноится.
\end{tcolorbox}
\begin{tcolorbox}
\textsubscript{6} Дни мои бегут скорее челнока и кончаются без надежды.
\end{tcolorbox}
\begin{tcolorbox}
\textsubscript{7} Вспомни, что жизнь моя дуновение, что око мое не возвратится видеть доброе.
\end{tcolorbox}
\begin{tcolorbox}
\textsubscript{8} Не увидит меня око видевшего меня; очи Твои на меня, --и нет меня.
\end{tcolorbox}
\begin{tcolorbox}
\textsubscript{9} Редеет облако и уходит; так нисшедший в преисподнюю не выйдет,
\end{tcolorbox}
\begin{tcolorbox}
\textsubscript{10} не возвратится более в дом свой, и место его не будет уже знать его.
\end{tcolorbox}
\begin{tcolorbox}
\textsubscript{11} Не буду же я удерживать уст моих; буду говорить в стеснении духа моего; буду жаловаться в горести души моей.
\end{tcolorbox}
\begin{tcolorbox}
\textsubscript{12} Разве я море или морское чудовище, что Ты поставил надо мною стражу?
\end{tcolorbox}
\begin{tcolorbox}
\textsubscript{13} Когда подумаю: утешит меня постель моя, унесет горесть мою ложе мое,
\end{tcolorbox}
\begin{tcolorbox}
\textsubscript{14} ты страшишь меня снами и видениями пугаешь меня;
\end{tcolorbox}
\begin{tcolorbox}
\textsubscript{15} и душа моя желает лучше прекращения дыхания, лучше смерти, нежели [сбережения] костей моих.
\end{tcolorbox}
\begin{tcolorbox}
\textsubscript{16} Опротивела мне жизнь. Не вечно жить мне. Отступи от меня, ибо дни мои суета.
\end{tcolorbox}
\begin{tcolorbox}
\textsubscript{17} Что такое человек, что Ты столько ценишь его и обращаешь на него внимание Твое,
\end{tcolorbox}
\begin{tcolorbox}
\textsubscript{18} посещаешь его каждое утро, каждое мгновение испытываешь его?
\end{tcolorbox}
\begin{tcolorbox}
\textsubscript{19} Доколе же Ты не оставишь, доколе не отойдешь от меня, доколе не дашь мне проглотить слюну мою?
\end{tcolorbox}
\begin{tcolorbox}
\textsubscript{20} Если я согрешил, то что я сделаю Тебе, страж человеков! Зачем Ты поставил меня противником Себе, так что я стал самому себе в тягость?
\end{tcolorbox}
\begin{tcolorbox}
\textsubscript{21} И зачем бы не простить мне греха и не снять с меня беззакония моего? ибо, вот, я лягу в прахе; завтра поищешь меня, и меня нет.
\end{tcolorbox}
\subsection{CHAPTER 8}
\begin{tcolorbox}
\textsubscript{1} И отвечал Вилдад Савхеянин и сказал:
\end{tcolorbox}
\begin{tcolorbox}
\textsubscript{2} долго ли ты будешь говорить так? --слова уст твоих бурный ветер!
\end{tcolorbox}
\begin{tcolorbox}
\textsubscript{3} Неужели Бог извращает суд, и Вседержитель превращает правду?
\end{tcolorbox}
\begin{tcolorbox}
\textsubscript{4} Если сыновья твои согрешили пред Ним, то Он и предал их в руку беззакония их.
\end{tcolorbox}
\begin{tcolorbox}
\textsubscript{5} Если же ты взыщешь Бога и помолишься Вседержителю,
\end{tcolorbox}
\begin{tcolorbox}
\textsubscript{6} и если ты чист и прав, то Он ныне же встанет над тобою и умиротворит жилище правды твоей.
\end{tcolorbox}
\begin{tcolorbox}
\textsubscript{7} И если вначале у тебя было мало, то впоследствии будет весьма много.
\end{tcolorbox}
\begin{tcolorbox}
\textsubscript{8} Ибо спроси у прежних родов и вникни в наблюдения отцов их;
\end{tcolorbox}
\begin{tcolorbox}
\textsubscript{9} а мы--вчерашние и ничего не знаем, потому что наши дни на земле тень.
\end{tcolorbox}
\begin{tcolorbox}
\textsubscript{10} Вот они научат тебя, скажут тебе и от сердца своего произнесут слова:
\end{tcolorbox}
\begin{tcolorbox}
\textsubscript{11} поднимается ли тростник без влаги? растет ли камыш без воды?
\end{tcolorbox}
\begin{tcolorbox}
\textsubscript{12} Еще он в свежести своей и не срезан, а прежде всякой травы засыхает.
\end{tcolorbox}
\begin{tcolorbox}
\textsubscript{13} Таковы пути всех забывающих Бога, и надежда лицемера погибнет;
\end{tcolorbox}
\begin{tcolorbox}
\textsubscript{14} упование его подсечено, и уверенность его--дом паука.
\end{tcolorbox}
\begin{tcolorbox}
\textsubscript{15} Обопрется о дом свой и не устоит; ухватится за него и не удержится.
\end{tcolorbox}
\begin{tcolorbox}
\textsubscript{16} Зеленеет он пред солнцем, за сад простираются ветви его;
\end{tcolorbox}
\begin{tcolorbox}
\textsubscript{17} в кучу [камней] вплетаются корни его, между камнями врезываются.
\end{tcolorbox}
\begin{tcolorbox}
\textsubscript{18} Но когда вырвут его с места его, оно откажется от него: 'я не видало тебя!'
\end{tcolorbox}
\begin{tcolorbox}
\textsubscript{19} Вот радость пути его! а из земли вырастают другие.
\end{tcolorbox}
\begin{tcolorbox}
\textsubscript{20} Видишь, Бог не отвергает непорочного и не поддерживает руки злодеев.
\end{tcolorbox}
\begin{tcolorbox}
\textsubscript{21} Он еще наполнит смехом уста твои и губы твои радостным восклицанием.
\end{tcolorbox}
\begin{tcolorbox}
\textsubscript{22} Ненавидящие тебя облекутся в стыд, и шатра нечестивых не станет.
\end{tcolorbox}
\subsection{CHAPTER 9}
\begin{tcolorbox}
\textsubscript{1} И отвечал Иов и сказал:
\end{tcolorbox}
\begin{tcolorbox}
\textsubscript{2} правда! знаю, что так; но как оправдается человек пред Богом?
\end{tcolorbox}
\begin{tcolorbox}
\textsubscript{3} Если захочет вступить в прение с Ним, то не ответит Ему ни на одно из тысячи.
\end{tcolorbox}
\begin{tcolorbox}
\textsubscript{4} Премудр сердцем и могущ силою; кто восставал против Него и оставался в покое?
\end{tcolorbox}
\begin{tcolorbox}
\textsubscript{5} Он передвигает горы, и не узнают их: Он превращает их в гневе Своем;
\end{tcolorbox}
\begin{tcolorbox}
\textsubscript{6} сдвигает землю с места ее, и столбы ее дрожат;
\end{tcolorbox}
\begin{tcolorbox}
\textsubscript{7} скажет солнцу, --и не взойдет, и на звезды налагает печать.
\end{tcolorbox}
\begin{tcolorbox}
\textsubscript{8} Он один распростирает небеса и ходит по высотам моря;
\end{tcolorbox}
\begin{tcolorbox}
\textsubscript{9} сотворил Ас, Кесиль и Хима (1) и тайники юга;
\end{tcolorbox}
\begin{tcolorbox}
\textsubscript{10} делает великое, неисследимое и чудное без числа!
\end{tcolorbox}
\begin{tcolorbox}
\textsubscript{11} Вот, Он пройдет предо мною, и не увижу Его; пронесется и не замечу Его.
\end{tcolorbox}
\begin{tcolorbox}
\textsubscript{12} Возьмет, и кто возбранит Ему? кто скажет Ему: что Ты делаешь?
\end{tcolorbox}
\begin{tcolorbox}
\textsubscript{13} Бог не отвратит гнева Своего; пред Ним падут поборники гордыни.
\end{tcolorbox}
\begin{tcolorbox}
\textsubscript{14} Тем более могу ли я отвечать Ему и приискивать себе слова пред Ним?
\end{tcolorbox}
\begin{tcolorbox}
\textsubscript{15} Хотя бы я и прав был, но не буду отвечать, а буду умолять Судию моего.
\end{tcolorbox}
\begin{tcolorbox}
\textsubscript{16} Если бы я воззвал, и Он ответил мне, --я не поверил бы, что голос мой услышал Тот,
\end{tcolorbox}
\begin{tcolorbox}
\textsubscript{17} Кто в вихре разит меня и умножает безвинно мои раны,
\end{tcolorbox}
\begin{tcolorbox}
\textsubscript{18} не дает мне перевести духа, но пресыщает меня горестями.
\end{tcolorbox}
\begin{tcolorbox}
\textsubscript{19} Если [действовать] силою, то Он могуществен; если судом, кто сведет меня с Ним?
\end{tcolorbox}
\begin{tcolorbox}
\textsubscript{20} Если я буду оправдываться, то мои же уста обвинят меня; [если] я невинен, то Он признает меня виновным.
\end{tcolorbox}
\begin{tcolorbox}
\textsubscript{21} Невинен я; не хочу знать души моей, презираю жизнь мою.
\end{tcolorbox}
\begin{tcolorbox}
\textsubscript{22} Все одно; поэтому я сказал, что Он губит и непорочного и виновного.
\end{tcolorbox}
\begin{tcolorbox}
\textsubscript{23} Если этого поражает Он бичом вдруг, то пытке невинных посмевается.
\end{tcolorbox}
\begin{tcolorbox}
\textsubscript{24} Земля отдана в руки нечестивых; лица судей ее Он закрывает. Если не Он, то кто же?
\end{tcolorbox}
\begin{tcolorbox}
\textsubscript{25} Дни мои быстрее гонца, --бегут, не видят добра,
\end{tcolorbox}
\begin{tcolorbox}
\textsubscript{26} несутся, как легкие ладьи, как орел стремится на добычу.
\end{tcolorbox}
\begin{tcolorbox}
\textsubscript{27} Если сказать мне: забуду я жалобы мои, отложу мрачный вид свой и ободрюсь;
\end{tcolorbox}
\begin{tcolorbox}
\textsubscript{28} то трепещу всех страданий моих, зная, что Ты не объявишь меня невинным.
\end{tcolorbox}
\begin{tcolorbox}
\textsubscript{29} Если же я виновен, то для чего напрасно томлюсь?
\end{tcolorbox}
\begin{tcolorbox}
\textsubscript{30} Хотя бы я омылся и снежною водою и совершенно очистил руки мои,
\end{tcolorbox}
\begin{tcolorbox}
\textsubscript{31} то и тогда Ты погрузишь меня в грязь, и возгнушаются мною одежды мои.
\end{tcolorbox}
\begin{tcolorbox}
\textsubscript{32} Ибо Он не человек, как я, чтоб я мог отвечать Ему и идти вместе с Ним на суд!
\end{tcolorbox}
\begin{tcolorbox}
\textsubscript{33} Нет между нами посредника, который положил бы руку свою на обоих нас.
\end{tcolorbox}
\begin{tcolorbox}
\textsubscript{34} Да отстранит Он от меня жезл Свой, и страх Его да не ужасает меня, --
\end{tcolorbox}
\begin{tcolorbox}
\textsubscript{35} и тогда я буду говорить и не убоюсь Его, ибо я не таков сам в себе.
\end{tcolorbox}
\subsection{CHAPTER 10}
\begin{tcolorbox}
\textsubscript{1} Опротивела душе моей жизнь моя; предамся печали моей; буду говорить в горести души моей.
\end{tcolorbox}
\begin{tcolorbox}
\textsubscript{2} Скажу Богу: не обвиняй меня; объяви мне, за что Ты со мною борешься?
\end{tcolorbox}
\begin{tcolorbox}
\textsubscript{3} Хорошо ли для Тебя, что Ты угнетаешь, что презираешь дело рук Твоих, а на совет нечестивых посылаешь свет?
\end{tcolorbox}
\begin{tcolorbox}
\textsubscript{4} Разве у Тебя плотские очи, и Ты смотришь, как смотрит человек?
\end{tcolorbox}
\begin{tcolorbox}
\textsubscript{5} Разве дни Твои, как дни человека, или лета Твои, как дни мужа,
\end{tcolorbox}
\begin{tcolorbox}
\textsubscript{6} что Ты ищешь порока во мне и допытываешься греха во мне,
\end{tcolorbox}
\begin{tcolorbox}
\textsubscript{7} хотя знаешь, что я не беззаконник, и что некому избавить меня от руки Твоей?
\end{tcolorbox}
\begin{tcolorbox}
\textsubscript{8} Твои руки трудились надо мною и образовали всего меня кругом, --и Ты губишь меня?
\end{tcolorbox}
\begin{tcolorbox}
\textsubscript{9} Вспомни, что Ты, как глину, обделал меня, и в прах обращаешь меня?
\end{tcolorbox}
\begin{tcolorbox}
\textsubscript{10} Не Ты ли вылил меня, как молоко, и, как творог, сгустил меня,
\end{tcolorbox}
\begin{tcolorbox}
\textsubscript{11} кожею и плотью одел меня, костями и жилами скрепил меня,
\end{tcolorbox}
\begin{tcolorbox}
\textsubscript{12} жизнь и милость даровал мне, и попечение Твое хранило дух мой?
\end{tcolorbox}
\begin{tcolorbox}
\textsubscript{13} Но и то скрывал Ты в сердце Своем, --знаю, что это было у Тебя, --
\end{tcolorbox}
\begin{tcolorbox}
\textsubscript{14} что если я согрешу, Ты заметишь и не оставишь греха моего без наказания.
\end{tcolorbox}
\begin{tcolorbox}
\textsubscript{15} Если я виновен, горе мне! если и прав, то не осмелюсь поднять головы моей. Я пресыщен унижением; взгляни на бедствие мое:
\end{tcolorbox}
\begin{tcolorbox}
\textsubscript{16} оно увеличивается. Ты гонишься за мною, как лев, и снова нападаешь на меня и чудным являешься во мне.
\end{tcolorbox}
\begin{tcolorbox}
\textsubscript{17} Выводишь новых свидетелей Твоих против меня; усиливаешь гнев Твой на меня; и беды, одни за другими, ополчаются против меня.
\end{tcolorbox}
\begin{tcolorbox}
\textsubscript{18} И зачем Ты вывел меня из чрева? пусть бы я умер, когда еще ничей глаз не видел меня;
\end{tcolorbox}
\begin{tcolorbox}
\textsubscript{19} пусть бы я, как небывший, из чрева перенесен был во гроб!
\end{tcolorbox}
\begin{tcolorbox}
\textsubscript{20} Не малы ли дни мои? Оставь, отступи от меня, чтобы я немного ободрился,
\end{tcolorbox}
\begin{tcolorbox}
\textsubscript{21} прежде нежели отойду, --и уже не возвращусь, --в страну тьмы и сени смертной,
\end{tcolorbox}
\begin{tcolorbox}
\textsubscript{22} в страну мрака, каков есть мрак тени смертной, где нет устройства, [где] темно, как самая тьма.
\end{tcolorbox}
\subsection{CHAPTER 11}
\begin{tcolorbox}
\textsubscript{1} И отвечал Софар Наамитянин и сказал:
\end{tcolorbox}
\begin{tcolorbox}
\textsubscript{2} разве на множество слов нельзя дать ответа, и разве человек многоречивый прав?
\end{tcolorbox}
\begin{tcolorbox}
\textsubscript{3} Пустословие твое заставит ли молчать мужей, чтобы ты глумился, и некому было постыдить тебя?
\end{tcolorbox}
\begin{tcolorbox}
\textsubscript{4} Ты сказал: суждение мое верно, и чист я в очах Твоих.
\end{tcolorbox}
\begin{tcolorbox}
\textsubscript{5} Но если бы Бог возглаголал и отверз уста Свои к тебе
\end{tcolorbox}
\begin{tcolorbox}
\textsubscript{6} и открыл тебе тайны премудрости, что тебе вдвое больше следовало бы понести! Итак знай, что Бог для тебя некоторые из беззаконий твоих предал забвению.
\end{tcolorbox}
\begin{tcolorbox}
\textsubscript{7} Можешь ли ты исследованием найти Бога? Можешь ли совершенно постигнуть Вседержителя?
\end{tcolorbox}
\begin{tcolorbox}
\textsubscript{8} Он превыше небес, --что можешь сделать? глубже преисподней, --что можешь узнать?
\end{tcolorbox}
\begin{tcolorbox}
\textsubscript{9} Длиннее земли мера Его и шире моря.
\end{tcolorbox}
\begin{tcolorbox}
\textsubscript{10} Если Он пройдет и заключит кого в оковы и представит на суд, то кто отклонит Его?
\end{tcolorbox}
\begin{tcolorbox}
\textsubscript{11} Ибо Он знает людей лживых и видит беззаконие, и оставит ли его без внимания?
\end{tcolorbox}
\begin{tcolorbox}
\textsubscript{12} Но пустой человек мудрствует, хотя человек рождается подобно дикому осленку.
\end{tcolorbox}
\begin{tcolorbox}
\textsubscript{13} Если ты управишь сердце твое и прострешь к Нему руки твои,
\end{tcolorbox}
\begin{tcolorbox}
\textsubscript{14} и если есть порок в руке твоей, а ты удалишь его и не дашь беззаконию обитать в шатрах твоих,
\end{tcolorbox}
\begin{tcolorbox}
\textsubscript{15} то поднимешь незапятнанное лице твое и будешь тверд и не будешь бояться.
\end{tcolorbox}
\begin{tcolorbox}
\textsubscript{16} Тогда забудешь горе: как о воде протекшей, будешь вспоминать о нем.
\end{tcolorbox}
\begin{tcolorbox}
\textsubscript{17} И яснее полдня пойдет жизнь твоя; просветлеешь, как утро.
\end{tcolorbox}
\begin{tcolorbox}
\textsubscript{18} И будешь спокоен, ибо есть надежда; ты огражден, и можешь спать безопасно.
\end{tcolorbox}
\begin{tcolorbox}
\textsubscript{19} Будешь лежать, и не будет устрашающего, и многие будут заискивать у тебя.
\end{tcolorbox}
\begin{tcolorbox}
\textsubscript{20} глаза беззаконных истают, и убежище пропадет у них, и надежда их исчезнет.
\end{tcolorbox}
\subsection{CHAPTER 12}
\begin{tcolorbox}
\textsubscript{1} И отвечал Иов и сказал:
\end{tcolorbox}
\begin{tcolorbox}
\textsubscript{2} подлинно, [только] вы люди, и с вами умрет мудрость!
\end{tcolorbox}
\begin{tcolorbox}
\textsubscript{3} И у меня [есть] сердце, как у вас; не ниже я вас; и кто не знает того же?
\end{tcolorbox}
\begin{tcolorbox}
\textsubscript{4} Посмешищем стал я для друга своего, я, который взывал к Богу, и которому Он отвечал, посмешищем--[человек] праведный, непорочный.
\end{tcolorbox}
\begin{tcolorbox}
\textsubscript{5} Так презрен по мыслям сидящего в покое факел, приготовленный для спотыкающихся ногами.
\end{tcolorbox}
\begin{tcolorbox}
\textsubscript{6} Покойны шатры у грабителей и безопасны у раздражающих Бога, которые как бы Бога носят в руках своих.
\end{tcolorbox}
\begin{tcolorbox}
\textsubscript{7} И подлинно: спроси у скота, и научит тебя, у птицы небесной, и возвестит тебе;
\end{tcolorbox}
\begin{tcolorbox}
\textsubscript{8} или побеседуй с землею, и наставит тебя, и скажут тебе рыбы морские.
\end{tcolorbox}
\begin{tcolorbox}
\textsubscript{9} Кто во всем этом не узнает, что рука Господа сотворила сие?
\end{tcolorbox}
\begin{tcolorbox}
\textsubscript{10} В Его руке душа всего живущего и дух всякой человеческой плоти.
\end{tcolorbox}
\begin{tcolorbox}
\textsubscript{11} Не ухо ли разбирает слова, и не язык ли распознает вкус пищи?
\end{tcolorbox}
\begin{tcolorbox}
\textsubscript{12} В старцах--мудрость, и в долголетних--разум.
\end{tcolorbox}
\begin{tcolorbox}
\textsubscript{13} У Него премудрость и сила; Его совет и разум.
\end{tcolorbox}
\begin{tcolorbox}
\textsubscript{14} Что Он разрушит, то не построится; кого Он заключит, тот не высвободится.
\end{tcolorbox}
\begin{tcolorbox}
\textsubscript{15} Остановит воды, и все высохнет; пустит их, и превратят землю.
\end{tcolorbox}
\begin{tcolorbox}
\textsubscript{16} У Него могущество и премудрость, пред Ним заблуждающийся и вводящий в заблуждение.
\end{tcolorbox}
\begin{tcolorbox}
\textsubscript{17} Он приводит советников в необдуманность и судей делает глупыми.
\end{tcolorbox}
\begin{tcolorbox}
\textsubscript{18} Он лишает перевязей царей и поясом обвязывает чресла их;
\end{tcolorbox}
\begin{tcolorbox}
\textsubscript{19} князей лишает достоинства и низвергает храбрых;
\end{tcolorbox}
\begin{tcolorbox}
\textsubscript{20} отнимает язык у велеречивых и старцев лишает смысла;
\end{tcolorbox}
\begin{tcolorbox}
\textsubscript{21} покрывает стыдом знаменитых и силу могучих ослабляет;
\end{tcolorbox}
\begin{tcolorbox}
\textsubscript{22} открывает глубокое из среды тьмы и выводит на свет тень смертную;
\end{tcolorbox}
\begin{tcolorbox}
\textsubscript{23} умножает народы и истребляет их; рассевает народы и собирает их;
\end{tcolorbox}
\begin{tcolorbox}
\textsubscript{24} отнимает ум у глав народа земли и оставляет их блуждать в пустыне, где нет пути:
\end{tcolorbox}
\begin{tcolorbox}
\textsubscript{25} ощупью ходят они во тьме без света и шатаются, как пьяные.
\end{tcolorbox}
\subsection{CHAPTER 13}
\begin{tcolorbox}
\textsubscript{1} Вот, все [это] видело око мое, слышало ухо мое и заметило для себя.
\end{tcolorbox}
\begin{tcolorbox}
\textsubscript{2} Сколько знаете вы, знаю и я: не ниже я вас.
\end{tcolorbox}
\begin{tcolorbox}
\textsubscript{3} Но я к Вседержителю хотел бы говорить и желал бы состязаться с Богом.
\end{tcolorbox}
\begin{tcolorbox}
\textsubscript{4} А вы сплетчики лжи; все вы бесполезные врачи.
\end{tcolorbox}
\begin{tcolorbox}
\textsubscript{5} О, если бы вы только молчали! это было бы [вменено] вам в мудрость.
\end{tcolorbox}
\begin{tcolorbox}
\textsubscript{6} Выслушайте же рассуждения мои и вникните в возражение уст моих.
\end{tcolorbox}
\begin{tcolorbox}
\textsubscript{7} Надлежало ли вам ради Бога говорить неправду и для Него говорить ложь?
\end{tcolorbox}
\begin{tcolorbox}
\textsubscript{8} Надлежало ли вам быть лицеприятными к Нему и за Бога так препираться?
\end{tcolorbox}
\begin{tcolorbox}
\textsubscript{9} Хорошо ли будет, когда Он испытает вас? Обманете ли Его, как обманывают человека?
\end{tcolorbox}
\begin{tcolorbox}
\textsubscript{10} Строго накажет Он вас, хотя вы и скрытно лицемерите.
\end{tcolorbox}
\begin{tcolorbox}
\textsubscript{11} Неужели величие Его не устрашает вас, и страх Его не нападает на вас?
\end{tcolorbox}
\begin{tcolorbox}
\textsubscript{12} Напоминания ваши подобны пеплу; оплоты ваши--оплоты глиняные.
\end{tcolorbox}
\begin{tcolorbox}
\textsubscript{13} Замолчите предо мною, и я буду говорить, что бы ни постигло меня.
\end{tcolorbox}
\begin{tcolorbox}
\textsubscript{14} Для чего мне терзать тело мое зубами моими и душу мою полагать в руку мою?
\end{tcolorbox}
\begin{tcolorbox}
\textsubscript{15} Вот, Он убивает меня, но я буду надеяться; я желал бы только отстоять пути мои пред лицем Его!
\end{tcolorbox}
\begin{tcolorbox}
\textsubscript{16} И это уже в оправдание мне, потому что лицемер не пойдет пред лице Его!
\end{tcolorbox}
\begin{tcolorbox}
\textsubscript{17} Выслушайте внимательно слово мое и объяснение мое ушами вашими.
\end{tcolorbox}
\begin{tcolorbox}
\textsubscript{18} Вот, я завел судебное дело: знаю, что буду прав.
\end{tcolorbox}
\begin{tcolorbox}
\textsubscript{19} Кто в состоянии оспорить меня? Ибо я скоро умолкну и испущу дух.
\end{tcolorbox}
\begin{tcolorbox}
\textsubscript{20} Двух только [вещей] не делай со мною, и тогда я не буду укрываться от лица Твоего:
\end{tcolorbox}
\begin{tcolorbox}
\textsubscript{21} удали от меня руку Твою, и ужас Твой да не потрясает меня.
\end{tcolorbox}
\begin{tcolorbox}
\textsubscript{22} Тогда зови, и я буду отвечать, или буду говорить я, а Ты отвечай мне.
\end{tcolorbox}
\begin{tcolorbox}
\textsubscript{23} Сколько у меня пороков и грехов? покажи мне беззаконие мое и грех мой.
\end{tcolorbox}
\begin{tcolorbox}
\textsubscript{24} Для чего скрываешь лице Твое и считаешь меня врагом Тебе?
\end{tcolorbox}
\begin{tcolorbox}
\textsubscript{25} Не сорванный ли листок Ты сокрушаешь и не сухую ли соломинку преследуешь?
\end{tcolorbox}
\begin{tcolorbox}
\textsubscript{26} Ибо Ты пишешь на меня горькое и вменяешь мне грехи юности моей,
\end{tcolorbox}
\begin{tcolorbox}
\textsubscript{27} и ставишь в колоду ноги мои и подстерегаешь все стези мои, --гонишься по следам ног моих.
\end{tcolorbox}
\begin{tcolorbox}
\textsubscript{28} А он, как гниль, распадается, как одежда, изъеденная молью.
\end{tcolorbox}
\subsection{CHAPTER 14}
\begin{tcolorbox}
\textsubscript{1} Человек, рожденный женою, краткодневен и пресыщен печалями:
\end{tcolorbox}
\begin{tcolorbox}
\textsubscript{2} как цветок, он выходит и опадает; убегает, как тень, и не останавливается.
\end{tcolorbox}
\begin{tcolorbox}
\textsubscript{3} И на него-то Ты отверзаешь очи Твои, и меня ведешь на суд с Тобою?
\end{tcolorbox}
\begin{tcolorbox}
\textsubscript{4} Кто родится чистым от нечистого? Ни один.
\end{tcolorbox}
\begin{tcolorbox}
\textsubscript{5} Если дни ему определены, и число месяцев его у Тебя, если Ты положил ему предел, которого он не перейдет,
\end{tcolorbox}
\begin{tcolorbox}
\textsubscript{6} то уклонись от него: пусть он отдохнет, доколе не окончит, как наемник, дня своего.
\end{tcolorbox}
\begin{tcolorbox}
\textsubscript{7} Для дерева есть надежда, что оно, если и будет срублено, снова оживет, и отрасли от него [выходить] не перестанут:
\end{tcolorbox}
\begin{tcolorbox}
\textsubscript{8} если и устарел в земле корень его, и пень его замер в пыли,
\end{tcolorbox}
\begin{tcolorbox}
\textsubscript{9} но, лишь почуяло воду, оно дает отпрыски и пускает ветви, как бы вновь посаженное.
\end{tcolorbox}
\begin{tcolorbox}
\textsubscript{10} А человек умирает и распадается; отошел, и где он?
\end{tcolorbox}
\begin{tcolorbox}
\textsubscript{11} Уходят воды из озера, и река иссякает и высыхает:
\end{tcolorbox}
\begin{tcolorbox}
\textsubscript{12} так человек ляжет и не станет; до скончания неба он не пробудится и не воспрянет от сна своего.
\end{tcolorbox}
\begin{tcolorbox}
\textsubscript{13} О, если бы Ты в преисподней сокрыл меня и укрывал меня, пока пройдет гнев Твой, положил мне срок и потом вспомнил обо мне!
\end{tcolorbox}
\begin{tcolorbox}
\textsubscript{14} Когда умрет человек, то будет ли он опять жить? Во все дни определенного мне времени я ожидал бы, пока придет мне смена.
\end{tcolorbox}
\begin{tcolorbox}
\textsubscript{15} Воззвал бы Ты, и я дал бы Тебе ответ, и Ты явил бы благоволение творению рук Твоих;
\end{tcolorbox}
\begin{tcolorbox}
\textsubscript{16} ибо тогда Ты исчислял бы шаги мои и не подстерегал бы греха моего;
\end{tcolorbox}
\begin{tcolorbox}
\textsubscript{17} в свитке было бы запечатано беззаконие мое, и Ты закрыл бы вину мою.
\end{tcolorbox}
\begin{tcolorbox}
\textsubscript{18} Но гора падая разрушается, и скала сходит с места своего;
\end{tcolorbox}
\begin{tcolorbox}
\textsubscript{19} вода стирает камни; разлив ее смывает земную пыль: так и надежду человека Ты уничтожаешь.
\end{tcolorbox}
\begin{tcolorbox}
\textsubscript{20} Теснишь его до конца, и он уходит; изменяешь ему лице и отсылаешь его.
\end{tcolorbox}
\begin{tcolorbox}
\textsubscript{21} В чести ли дети его--он не знает, унижены ли--он не замечает;
\end{tcolorbox}
\begin{tcolorbox}
\textsubscript{22} но плоть его на нем болит, и душа его в нем страдает.
\end{tcolorbox}
\subsection{CHAPTER 15}
\begin{tcolorbox}
\textsubscript{1} И отвечал Елифаз Феманитянин и сказал:
\end{tcolorbox}
\begin{tcolorbox}
\textsubscript{2} станет ли мудрый отвечать знанием пустым и наполнять чрево свое ветром палящим,
\end{tcolorbox}
\begin{tcolorbox}
\textsubscript{3} оправдываться словами бесполезными и речью, не имеющею никакой силы?
\end{tcolorbox}
\begin{tcolorbox}
\textsubscript{4} Да ты отложил и страх и за малость считаешь речь к Богу.
\end{tcolorbox}
\begin{tcolorbox}
\textsubscript{5} Нечестие твое настроило так уста твои, и ты избрал язык лукавых.
\end{tcolorbox}
\begin{tcolorbox}
\textsubscript{6} Тебя обвиняют уста твои, а не я, и твой язык говорит против тебя.
\end{tcolorbox}
\begin{tcolorbox}
\textsubscript{7} Разве ты первым человеком родился и прежде холмов создан?
\end{tcolorbox}
\begin{tcolorbox}
\textsubscript{8} Разве совет Божий ты слышал и привлек к себе премудрость?
\end{tcolorbox}
\begin{tcolorbox}
\textsubscript{9} Что знаешь ты, чего бы не знали мы? что разумеешь ты, чего не было бы и у нас?
\end{tcolorbox}
\begin{tcolorbox}
\textsubscript{10} И седовласый и старец есть между нами, днями превышающий отца твоего.
\end{tcolorbox}
\begin{tcolorbox}
\textsubscript{11} Разве малость для тебя утешения Божии? И это неизвестно тебе?
\end{tcolorbox}
\begin{tcolorbox}
\textsubscript{12} К чему порывает тебя сердце твое, и к чему так гордо смотришь?
\end{tcolorbox}
\begin{tcolorbox}
\textsubscript{13} Что устремляешь против Бога дух твой и устами твоими произносишь такие речи?
\end{tcolorbox}
\begin{tcolorbox}
\textsubscript{14} Что такое человек, чтоб быть ему чистым, и чтобы рожденному женщиною быть праведным?
\end{tcolorbox}
\begin{tcolorbox}
\textsubscript{15} Вот, Он и святым Своим не доверяет, и небеса нечисты в очах Его:
\end{tcolorbox}
\begin{tcolorbox}
\textsubscript{16} тем больше нечист и растлен человек, пьющий беззаконие, как воду.
\end{tcolorbox}
\begin{tcolorbox}
\textsubscript{17} Я буду говорить тебе, слушай меня; я расскажу тебе, что видел,
\end{tcolorbox}
\begin{tcolorbox}
\textsubscript{18} что слышали мудрые и не скрыли слышанного от отцов своих,
\end{tcolorbox}
\begin{tcolorbox}
\textsubscript{19} которым одним отдана была земля, и среди которых чужой не ходил.
\end{tcolorbox}
\begin{tcolorbox}
\textsubscript{20} Нечестивый мучит себя во все дни свои, и число лет закрыто от притеснителя;
\end{tcolorbox}
\begin{tcolorbox}
\textsubscript{21} звук ужасов в ушах его; среди мира идет на него губитель.
\end{tcolorbox}
\begin{tcolorbox}
\textsubscript{22} Он не надеется спастись от тьмы; видит пред собою меч.
\end{tcolorbox}
\begin{tcolorbox}
\textsubscript{23} Он скитается за куском хлеба повсюду; знает, что уже готов, в руках у него день тьмы.
\end{tcolorbox}
\begin{tcolorbox}
\textsubscript{24} Устрашает его нужда и теснота; одолевает его, как царь, приготовившийся к битве,
\end{tcolorbox}
\begin{tcolorbox}
\textsubscript{25} за то, что он простирал против Бога руку свою и противился Вседержителю,
\end{tcolorbox}
\begin{tcolorbox}
\textsubscript{26} устремлялся против Него с [гордою] выею, под толстыми щитами своими;
\end{tcolorbox}
\begin{tcolorbox}
\textsubscript{27} потому что он покрыл лице свое жиром своим и обложил туком лядвеи свои.
\end{tcolorbox}
\begin{tcolorbox}
\textsubscript{28} И он селится в городах разоренных, в домах, в которых не живут, которые обречены на развалины.
\end{tcolorbox}
\begin{tcolorbox}
\textsubscript{29} Не пребудет он богатым, и не уцелеет имущество его, и не распрострется по земле приобретение его.
\end{tcolorbox}
\begin{tcolorbox}
\textsubscript{30} Не уйдет от тьмы; отрасли его иссушит пламя и дуновением уст своих увлечет его.
\end{tcolorbox}
\begin{tcolorbox}
\textsubscript{31} Пусть не доверяет суете заблудший, ибо суета будет и воздаянием ему.
\end{tcolorbox}
\begin{tcolorbox}
\textsubscript{32} Не в свой день он скончается, и ветви его не будут зеленеть.
\end{tcolorbox}
\begin{tcolorbox}
\textsubscript{33} Сбросит он, как виноградная лоза, недозрелую ягоду свою и, как маслина, стряхнет цвет свой.
\end{tcolorbox}
\begin{tcolorbox}
\textsubscript{34} Так опустеет дом нечестивого, и огонь пожрет шатры мздоимства.
\end{tcolorbox}
\begin{tcolorbox}
\textsubscript{35} Он зачал зло и родил ложь, и утроба его приготовляет обман.
\end{tcolorbox}
\subsection{CHAPTER 16}
\begin{tcolorbox}
\textsubscript{1} И отвечал Иов и сказал:
\end{tcolorbox}
\begin{tcolorbox}
\textsubscript{2} слышал я много такого; жалкие утешители все вы!
\end{tcolorbox}
\begin{tcolorbox}
\textsubscript{3} Будет ли конец ветреным словам? и что побудило тебя так отвечать?
\end{tcolorbox}
\begin{tcolorbox}
\textsubscript{4} И я мог бы так же говорить, как вы, если бы душа ваша была на месте души моей; ополчался бы на вас словами и кивал бы на вас головою моею;
\end{tcolorbox}
\begin{tcolorbox}
\textsubscript{5} подкреплял бы вас языком моим и движением губ утешал бы.
\end{tcolorbox}
\begin{tcolorbox}
\textsubscript{6} Говорю ли я, не утоляется скорбь моя; перестаю ли, что отходит от меня?
\end{tcolorbox}
\begin{tcolorbox}
\textsubscript{7} Но ныне Он изнурил меня. Ты разрушил всю семью мою.
\end{tcolorbox}
\begin{tcolorbox}
\textsubscript{8} Ты покрыл меня морщинами во свидетельство против меня; восстает на меня изможденность моя, в лицо укоряет меня.
\end{tcolorbox}
\begin{tcolorbox}
\textsubscript{9} Гнев Его терзает и враждует против меня, скрежещет на меня зубами своими; неприятель мой острит на меня глаза свои.
\end{tcolorbox}
\begin{tcolorbox}
\textsubscript{10} Разинули на меня пасть свою; ругаясь бьют меня по щекам; все сговорились против меня.
\end{tcolorbox}
\begin{tcolorbox}
\textsubscript{11} Предал меня Бог беззаконнику и в руки нечестивым бросил меня.
\end{tcolorbox}
\begin{tcolorbox}
\textsubscript{12} Я был спокоен, но Он потряс меня; взял меня за шею и избил меня и поставил меня целью для Себя.
\end{tcolorbox}
\begin{tcolorbox}
\textsubscript{13} Окружили меня стрельцы Его; Он рассекает внутренности мои и не щадит, пролил на землю желчь мою,
\end{tcolorbox}
\begin{tcolorbox}
\textsubscript{14} пробивает во мне пролом за проломом, бежит на меня, как ратоборец.
\end{tcolorbox}
\begin{tcolorbox}
\textsubscript{15} Вретище сшил я на кожу мою и в прах положил голову мою.
\end{tcolorbox}
\begin{tcolorbox}
\textsubscript{16} Лицо мое побагровело от плача, и на веждах моих тень смерти,
\end{tcolorbox}
\begin{tcolorbox}
\textsubscript{17} при всем том, что нет хищения в руках моих, и молитва моя чиста.
\end{tcolorbox}
\begin{tcolorbox}
\textsubscript{18} Земля! не закрой моей крови, и да не будет места воплю моему.
\end{tcolorbox}
\begin{tcolorbox}
\textsubscript{19} И ныне вот на небесах Свидетель мой, и Заступник мой в вышних!
\end{tcolorbox}
\begin{tcolorbox}
\textsubscript{20} Многоречивые друзья мои! К Богу слезит око мое.
\end{tcolorbox}
\begin{tcolorbox}
\textsubscript{21} О, если бы человек мог иметь состязание с Богом, как сын человеческий с ближним своим!
\end{tcolorbox}
\begin{tcolorbox}
\textsubscript{22} Ибо летам моим приходит конец, и я отхожу в путь невозвратный.
\end{tcolorbox}
\subsection{CHAPTER 17}
\begin{tcolorbox}
\textsubscript{1} Дыхание мое ослабело; дни мои угасают; гробы предо мною.
\end{tcolorbox}
\begin{tcolorbox}
\textsubscript{2} Если бы не насмешки их, то и среди споров их око мое пребывало бы спокойно.
\end{tcolorbox}
\begin{tcolorbox}
\textsubscript{3} Заступись, поручись [Сам] за меня пред Собою! иначе кто поручится за меня?
\end{tcolorbox}
\begin{tcolorbox}
\textsubscript{4} Ибо Ты закрыл сердце их от разумения, и потому не дашь восторжествовать [им].
\end{tcolorbox}
\begin{tcolorbox}
\textsubscript{5} Кто обрекает друзей своих в добычу, у детей того глаза истают.
\end{tcolorbox}
\begin{tcolorbox}
\textsubscript{6} Он поставил меня притчею для народа и посмешищем для него.
\end{tcolorbox}
\begin{tcolorbox}
\textsubscript{7} Помутилось от горести око мое, и все члены мои, как тень.
\end{tcolorbox}
\begin{tcolorbox}
\textsubscript{8} Изумятся о сем праведные, и невинный вознегодует на лицемера.
\end{tcolorbox}
\begin{tcolorbox}
\textsubscript{9} Но праведник будет крепко держаться пути своего, и чистый руками будет больше и больше утверждаться.
\end{tcolorbox}
\begin{tcolorbox}
\textsubscript{10} Выслушайте, все вы, и подойдите; не найду я мудрого между вами.
\end{tcolorbox}
\begin{tcolorbox}
\textsubscript{11} Дни мои прошли; думы мои--достояние сердца моего--разбиты.
\end{tcolorbox}
\begin{tcolorbox}
\textsubscript{12} А они ночь [хотят] превратить в день, свет приблизить к лицу тьмы.
\end{tcolorbox}
\begin{tcolorbox}
\textsubscript{13} Если бы я и ожидать стал, то преисподняя--дом мой; во тьме постелю я постель мою;
\end{tcolorbox}
\begin{tcolorbox}
\textsubscript{14} гробу скажу: ты отец мой, червю: ты мать моя и сестра моя.
\end{tcolorbox}
\begin{tcolorbox}
\textsubscript{15} Где же после этого надежда моя? и ожидаемое мною кто увидит?
\end{tcolorbox}
\begin{tcolorbox}
\textsubscript{16} В преисподнюю сойдет она и будет покоиться со мною в прахе.
\end{tcolorbox}
\subsection{CHAPTER 18}
\begin{tcolorbox}
\textsubscript{1} И отвечал Вилдад Савхеянин и сказал:
\end{tcolorbox}
\begin{tcolorbox}
\textsubscript{2} когда же положите вы конец таким речам? обдумайте, и потом будем говорить.
\end{tcolorbox}
\begin{tcolorbox}
\textsubscript{3} Зачем считаться нам за животных и быть униженными в собственных глазах ваших?
\end{tcolorbox}
\begin{tcolorbox}
\textsubscript{4} [О ты], раздирающий душу твою в гневе твоем! Неужели для тебя опустеть земле, и скале сдвинуться с места своего?
\end{tcolorbox}
\begin{tcolorbox}
\textsubscript{5} Да, свет у беззаконного потухнет, и не останется искры от огня его.
\end{tcolorbox}
\begin{tcolorbox}
\textsubscript{6} Померкнет свет в шатре его, и светильник его угаснет над ним.
\end{tcolorbox}
\begin{tcolorbox}
\textsubscript{7} Сократятся шаги могущества его, и низложит его собственный замысл его,
\end{tcolorbox}
\begin{tcolorbox}
\textsubscript{8} ибо он попадет в сеть своими ногами и по тенетам ходить будет.
\end{tcolorbox}
\begin{tcolorbox}
\textsubscript{9} Петля зацепит за ногу его, и грабитель уловит его.
\end{tcolorbox}
\begin{tcolorbox}
\textsubscript{10} Скрытно разложены по земле силки для него и западни на дороге.
\end{tcolorbox}
\begin{tcolorbox}
\textsubscript{11} Со всех сторон будут страшить его ужасы и заставят его бросаться туда и сюда.
\end{tcolorbox}
\begin{tcolorbox}
\textsubscript{12} Истощится от голода сила его, и гибель готова, сбоку у него.
\end{tcolorbox}
\begin{tcolorbox}
\textsubscript{13} Съест члены тела его, съест члены его первенец смерти.
\end{tcolorbox}
\begin{tcolorbox}
\textsubscript{14} Изгнана будет из шатра его надежда его, и это низведет его к царю ужасов.
\end{tcolorbox}
\begin{tcolorbox}
\textsubscript{15} Поселятся в шатре его, потому что он уже не его; жилище его посыпано будет серою.
\end{tcolorbox}
\begin{tcolorbox}
\textsubscript{16} Снизу подсохнут корни его, и сверху увянут ветви его.
\end{tcolorbox}
\begin{tcolorbox}
\textsubscript{17} Память о нем исчезнет с земли, и имени его не будет на площади.
\end{tcolorbox}
\begin{tcolorbox}
\textsubscript{18} Изгонят его из света во тьму и сотрут его с лица земли.
\end{tcolorbox}
\begin{tcolorbox}
\textsubscript{19} Ни сына его, ни внука не будет в народе его, и никого не останется в жилищах его.
\end{tcolorbox}
\begin{tcolorbox}
\textsubscript{20} О дне его ужаснутся потомки, и современники будут объяты трепетом.
\end{tcolorbox}
\begin{tcolorbox}
\textsubscript{21} Таковы жилища беззаконного, и таково место того, кто не знает Бога.
\end{tcolorbox}
\subsection{CHAPTER 19}
\begin{tcolorbox}
\textsubscript{1} И отвечал Иов и сказал:
\end{tcolorbox}
\begin{tcolorbox}
\textsubscript{2} доколе будете мучить душу мою и терзать меня речами?
\end{tcolorbox}
\begin{tcolorbox}
\textsubscript{3} Вот, уже раз десять вы срамили меня и не стыдитесь теснить меня.
\end{tcolorbox}
\begin{tcolorbox}
\textsubscript{4} Если я и действительно погрешил, то погрешность моя при мне остается.
\end{tcolorbox}
\begin{tcolorbox}
\textsubscript{5} Если же вы хотите повеличаться надо мною и упрекнуть меня позором моим,
\end{tcolorbox}
\begin{tcolorbox}
\textsubscript{6} то знайте, что Бог ниспроверг меня и обложил меня Своею сетью.
\end{tcolorbox}
\begin{tcolorbox}
\textsubscript{7} Вот, я кричу: обида! и никто не слушает; вопию, и нет суда.
\end{tcolorbox}
\begin{tcolorbox}
\textsubscript{8} Он преградил мне дорогу, и не могу пройти, и на стези мои положил тьму.
\end{tcolorbox}
\begin{tcolorbox}
\textsubscript{9} Совлек с меня славу мою и снял венец с головы моей.
\end{tcolorbox}
\begin{tcolorbox}
\textsubscript{10} Кругом разорил меня, и я отхожу; и, как дерево, Он исторг надежду мою.
\end{tcolorbox}
\begin{tcolorbox}
\textsubscript{11} Воспылал на меня гневом Своим и считает меня между врагами Своими.
\end{tcolorbox}
\begin{tcolorbox}
\textsubscript{12} Полки Его пришли вместе и направили путь свой ко мне и расположились вокруг шатра моего.
\end{tcolorbox}
\begin{tcolorbox}
\textsubscript{13} Братьев моих Он удалил от меня, и знающие меня чуждаются меня.
\end{tcolorbox}
\begin{tcolorbox}
\textsubscript{14} Покинули меня близкие мои, и знакомые мои забыли меня.
\end{tcolorbox}
\begin{tcolorbox}
\textsubscript{15} Пришлые в доме моем и служанки мои чужим считают меня; посторонним стал я в глазах их.
\end{tcolorbox}
\begin{tcolorbox}
\textsubscript{16} Зову слугу моего, и он не откликается; устами моими я должен умолять его.
\end{tcolorbox}
\begin{tcolorbox}
\textsubscript{17} Дыхание мое опротивело жене моей, и я должен умолять ее ради детей чрева моего.
\end{tcolorbox}
\begin{tcolorbox}
\textsubscript{18} Даже малые дети презирают меня: поднимаюсь, и они издеваются надо мною.
\end{tcolorbox}
\begin{tcolorbox}
\textsubscript{19} Гнушаются мною все наперсники мои, и те, которых я любил, обратились против меня.
\end{tcolorbox}
\begin{tcolorbox}
\textsubscript{20} Кости мои прилипли к коже моей и плоти моей, и я остался только с кожею около зубов моих.
\end{tcolorbox}
\begin{tcolorbox}
\textsubscript{21} Помилуйте меня, помилуйте меня вы, друзья мои, ибо рука Божия коснулась меня.
\end{tcolorbox}
\begin{tcolorbox}
\textsubscript{22} Зачем и вы преследуете меня, как Бог, и плотью моею не можете насытиться?
\end{tcolorbox}
\begin{tcolorbox}
\textsubscript{23} О, если бы записаны были слова мои! Если бы начертаны были они в книге
\end{tcolorbox}
\begin{tcolorbox}
\textsubscript{24} резцом железным с оловом, --на вечное время на камне вырезаны были!
\end{tcolorbox}
\begin{tcolorbox}
\textsubscript{25} А я знаю, Искупитель мой жив, и Он в последний день восставит из праха распадающуюся кожу мою сию,
\end{tcolorbox}
\begin{tcolorbox}
\textsubscript{26} и я во плоти моей узрю Бога.
\end{tcolorbox}
\begin{tcolorbox}
\textsubscript{27} Я узрю Его сам; мои глаза, не глаза другого, увидят Его. Истаевает сердце мое в груди моей!
\end{tcolorbox}
\begin{tcolorbox}
\textsubscript{28} Вам надлежало бы сказать: зачем мы преследуем его? Как будто корень зла найден во мне.
\end{tcolorbox}
\begin{tcolorbox}
\textsubscript{29} Убойтесь меча, ибо меч есть отмститель неправды, и знайте, что есть суд.
\end{tcolorbox}
\subsection{CHAPTER 20}
\begin{tcolorbox}
\textsubscript{1} И отвечал Софар Наамитянин и сказал:
\end{tcolorbox}
\begin{tcolorbox}
\textsubscript{2} размышления мои побуждают меня отвечать, и я поспешаю выразить их.
\end{tcolorbox}
\begin{tcolorbox}
\textsubscript{3} Упрек, позорный для меня, выслушал я, и дух разумения моего ответит за меня.
\end{tcolorbox}
\begin{tcolorbox}
\textsubscript{4} Разве не знаешь ты, что от века, --с того времени, как поставлен человек на земле, --
\end{tcolorbox}
\begin{tcolorbox}
\textsubscript{5} веселье беззаконных кратковременно, и радость лицемера мгновенна?
\end{tcolorbox}
\begin{tcolorbox}
\textsubscript{6} Хотя бы возросло до небес величие его, и голова его касалась облаков, --
\end{tcolorbox}
\begin{tcolorbox}
\textsubscript{7} как помет его, на веки пропадает он; видевшие его скажут: где он?
\end{tcolorbox}
\begin{tcolorbox}
\textsubscript{8} Как сон, улетит, и не найдут его; и, как ночное видение, исчезнет.
\end{tcolorbox}
\begin{tcolorbox}
\textsubscript{9} Глаз, видевший его, больше не увидит его, и уже не усмотрит его место его.
\end{tcolorbox}
\begin{tcolorbox}
\textsubscript{10} Сыновья его будут заискивать у нищих, и руки его возвратят похищенное им.
\end{tcolorbox}
\begin{tcolorbox}
\textsubscript{11} Кости его наполнены грехами юности его, и с ним лягут они в прах.
\end{tcolorbox}
\begin{tcolorbox}
\textsubscript{12} Если сладко во рту его зло, и он таит его под языком своим,
\end{tcolorbox}
\begin{tcolorbox}
\textsubscript{13} бережет и не бросает его, а держит его в устах своих,
\end{tcolorbox}
\begin{tcolorbox}
\textsubscript{14} то эта пища его в утробе его превратится в желчь аспидов внутри его.
\end{tcolorbox}
\begin{tcolorbox}
\textsubscript{15} Имение, которое он глотал, изблюет: Бог исторгнет его из чрева его.
\end{tcolorbox}
\begin{tcolorbox}
\textsubscript{16} Змеиный яд он сосет; умертвит его язык ехидны.
\end{tcolorbox}
\begin{tcolorbox}
\textsubscript{17} Не видать ему ручьев, рек, текущих медом и молоком!
\end{tcolorbox}
\begin{tcolorbox}
\textsubscript{18} Нажитое трудом возвратит, не проглотит; по мере имения его будет и расплата его, а он не порадуется.
\end{tcolorbox}
\begin{tcolorbox}
\textsubscript{19} Ибо он угнетал, отсылал бедных; захватывал домы, которых не строил;
\end{tcolorbox}
\begin{tcolorbox}
\textsubscript{20} не знал сытости во чреве своем и в жадности своей не щадил ничего.
\end{tcolorbox}
\begin{tcolorbox}
\textsubscript{21} Ничего не спаслось от обжорства его, зато не устоит счастье его.
\end{tcolorbox}
\begin{tcolorbox}
\textsubscript{22} В полноте изобилия будет тесно ему; всякая рука обиженного поднимется на него.
\end{tcolorbox}
\begin{tcolorbox}
\textsubscript{23} Когда будет чем наполнить утробу его, Он пошлет на него ярость гнева Своего и одождит на него болезни в плоти его.
\end{tcolorbox}
\begin{tcolorbox}
\textsubscript{24} Убежит ли он от оружия железного, --пронзит его лук медный;
\end{tcolorbox}
\begin{tcolorbox}
\textsubscript{25} станет вынимать [стрелу], --и она выйдет из тела, выйдет, сверкая сквозь желчь его; ужасы смерти найдут на него!
\end{tcolorbox}
\begin{tcolorbox}
\textsubscript{26} Все мрачное сокрыто внутри его; будет пожирать его огонь, никем не раздуваемый; зло постигнет и оставшееся в шатре его.
\end{tcolorbox}
\begin{tcolorbox}
\textsubscript{27} Небо откроет беззаконие его, и земля восстанет против него.
\end{tcolorbox}
\begin{tcolorbox}
\textsubscript{28} Исчезнет стяжание дома его; все расплывется в день гнева Его.
\end{tcolorbox}
\begin{tcolorbox}
\textsubscript{29} Вот удел человеку беззаконному от Бога и наследие, определенное ему Вседержителем!
\end{tcolorbox}
\subsection{CHAPTER 21}
\begin{tcolorbox}
\textsubscript{1} И отвечал Иов и сказал:
\end{tcolorbox}
\begin{tcolorbox}
\textsubscript{2} выслушайте внимательно речь мою, и это будет мне утешением от вас.
\end{tcolorbox}
\begin{tcolorbox}
\textsubscript{3} Потерпите меня, и я буду говорить; а после того, как поговорю, насмехайся.
\end{tcolorbox}
\begin{tcolorbox}
\textsubscript{4} Разве к человеку речь моя? как же мне и не малодушествовать?
\end{tcolorbox}
\begin{tcolorbox}
\textsubscript{5} Посмотрите на меня и ужаснитесь, и положите перст на уста.
\end{tcolorbox}
\begin{tcolorbox}
\textsubscript{6} Лишь только я вспомню, --содрогаюсь, и трепет объемлет тело мое.
\end{tcolorbox}
\begin{tcolorbox}
\textsubscript{7} Почему беззаконные живут, достигают старости, да и силами крепки?
\end{tcolorbox}
\begin{tcolorbox}
\textsubscript{8} Дети их с ними перед лицем их, и внуки их перед глазами их.
\end{tcolorbox}
\begin{tcolorbox}
\textsubscript{9} Домы их безопасны от страха, и нет жезла Божия на них.
\end{tcolorbox}
\begin{tcolorbox}
\textsubscript{10} Вол их оплодотворяет и не извергает, корова их зачинает и не выкидывает.
\end{tcolorbox}
\begin{tcolorbox}
\textsubscript{11} Как стадо, выпускают они малюток своих, и дети их прыгают.
\end{tcolorbox}
\begin{tcolorbox}
\textsubscript{12} Восклицают под [голос] тимпана и цитры и веселятся при [звуках] свирели;
\end{tcolorbox}
\begin{tcolorbox}
\textsubscript{13} проводят дни свои в счастьи и мгновенно нисходят в преисподнюю.
\end{tcolorbox}
\begin{tcolorbox}
\textsubscript{14} А между тем они говорят Богу: отойди от нас, не хотим мы знать путей Твоих!
\end{tcolorbox}
\begin{tcolorbox}
\textsubscript{15} Что Вседержитель, чтобы нам служить Ему? и что пользы прибегать к Нему?
\end{tcolorbox}
\begin{tcolorbox}
\textsubscript{16} Видишь, счастье их не от их рук. --Совет нечестивых будь далек от меня!
\end{tcolorbox}
\begin{tcolorbox}
\textsubscript{17} Часто ли угасает светильник у беззаконных, и находит на них беда, и Он дает им в удел страдания во гневе Своем?
\end{tcolorbox}
\begin{tcolorbox}
\textsubscript{18} Они должны быть, как соломинка пред ветром и как плева, уносимая вихрем.
\end{tcolorbox}
\begin{tcolorbox}
\textsubscript{19} [Скажешь]: Бог бережет для детей его несчастье его. --Пусть воздаст Он ему самому, чтобы он это знал.
\end{tcolorbox}
\begin{tcolorbox}
\textsubscript{20} Пусть его глаза увидят несчастье его, и пусть он сам пьет от гнева Вседержителева.
\end{tcolorbox}
\begin{tcolorbox}
\textsubscript{21} Ибо какая ему забота до дома своего после него, когда число месяцев его кончится?
\end{tcolorbox}
\begin{tcolorbox}
\textsubscript{22} Но Бога ли учить мудрости, когда Он судит и горних?
\end{tcolorbox}
\begin{tcolorbox}
\textsubscript{23} Один умирает в самой полноте сил своих, совершенно спокойный и мирный;
\end{tcolorbox}
\begin{tcolorbox}
\textsubscript{24} внутренности его полны жира, и кости его напоены мозгом.
\end{tcolorbox}
\begin{tcolorbox}
\textsubscript{25} А другой умирает с душею огорченною, не вкусив добра.
\end{tcolorbox}
\begin{tcolorbox}
\textsubscript{26} И они вместе будут лежать во прахе, и червь покроет их.
\end{tcolorbox}
\begin{tcolorbox}
\textsubscript{27} Знаю я ваши мысли и ухищрения, какие вы против меня сплетаете.
\end{tcolorbox}
\begin{tcolorbox}
\textsubscript{28} Вы скажете: где дом князя, и где шатер, в котором жили беззаконные?
\end{tcolorbox}
\begin{tcolorbox}
\textsubscript{29} Разве вы не спрашивали у путешественников и незнакомы с их наблюдениями,
\end{tcolorbox}
\begin{tcolorbox}
\textsubscript{30} что в день погибели пощажен бывает злодей, в день гнева отводится в сторону?
\end{tcolorbox}
\begin{tcolorbox}
\textsubscript{31} Кто представит ему пред лице путь его, и кто воздаст ему за то, что он делал?
\end{tcolorbox}
\begin{tcolorbox}
\textsubscript{32} Его провожают ко гробам и на его могиле ставят стражу.
\end{tcolorbox}
\begin{tcolorbox}
\textsubscript{33} Сладки для него глыбы долины, и за ним идет толпа людей, а идущим перед ним нет числа.
\end{tcolorbox}
\begin{tcolorbox}
\textsubscript{34} Как же вы хотите утешать меня пустым? В ваших ответах остается [одна] ложь.
\end{tcolorbox}
\subsection{CHAPTER 22}
\begin{tcolorbox}
\textsubscript{1} И отвечал Елифаз Феманитянин и сказал:
\end{tcolorbox}
\begin{tcolorbox}
\textsubscript{2} разве может человек доставлять пользу Богу? Разумный доставляет пользу себе самому.
\end{tcolorbox}
\begin{tcolorbox}
\textsubscript{3} Что за удовольствие Вседержителю, что ты праведен? И будет ли Ему выгода от того, что ты содержишь пути твои в непорочности?
\end{tcolorbox}
\begin{tcolorbox}
\textsubscript{4} Неужели Он, боясь тебя, вступит с тобою в состязание, пойдет судиться с тобою?
\end{tcolorbox}
\begin{tcolorbox}
\textsubscript{5} Верно, злоба твоя велика, и беззакониям твоим нет конца.
\end{tcolorbox}
\begin{tcolorbox}
\textsubscript{6} Верно, ты брал залоги от братьев твоих ни за что и с полунагих снимал одежду.
\end{tcolorbox}
\begin{tcolorbox}
\textsubscript{7} Утомленному жаждою не подавал воды напиться и голодному отказывал в хлебе;
\end{tcolorbox}
\begin{tcolorbox}
\textsubscript{8} а человеку сильному ты [давал] землю, и сановитый селился на ней.
\end{tcolorbox}
\begin{tcolorbox}
\textsubscript{9} Вдов ты отсылал ни с чем и сирот оставлял с пустыми руками.
\end{tcolorbox}
\begin{tcolorbox}
\textsubscript{10} За то вокруг тебя петли, и возмутил тебя неожиданный ужас,
\end{tcolorbox}
\begin{tcolorbox}
\textsubscript{11} или тьма, в которой ты ничего не видишь, и множество вод покрыло тебя.
\end{tcolorbox}
\begin{tcolorbox}
\textsubscript{12} Не превыше ли небес Бог? посмотри вверх на звезды, как они высоко!
\end{tcolorbox}
\begin{tcolorbox}
\textsubscript{13} И ты говоришь: что знает Бог? может ли Он судить сквозь мрак?
\end{tcolorbox}
\begin{tcolorbox}
\textsubscript{14} Облака--завеса Его, так что Он не видит, а ходит [только] по небесному кругу.
\end{tcolorbox}
\begin{tcolorbox}
\textsubscript{15} Неужели ты держишься пути древних, по которому шли люди беззаконные,
\end{tcolorbox}
\begin{tcolorbox}
\textsubscript{16} которые преждевременно были истреблены, когда вода разлилась под основание их?
\end{tcolorbox}
\begin{tcolorbox}
\textsubscript{17} Они говорили Богу: отойди от нас! и что сделает им Вседержитель?
\end{tcolorbox}
\begin{tcolorbox}
\textsubscript{18} А Он наполнял домы их добром. Но совет нечестивых будь далек от меня!
\end{tcolorbox}
\begin{tcolorbox}
\textsubscript{19} Видели праведники и радовались, и непорочный смеялся им:
\end{tcolorbox}
\begin{tcolorbox}
\textsubscript{20} враг наш истреблен, а оставшееся после них пожрал огонь.
\end{tcolorbox}
\begin{tcolorbox}
\textsubscript{21} Сблизься же с Ним--и будешь спокоен; чрез это придет к тебе добро.
\end{tcolorbox}
\begin{tcolorbox}
\textsubscript{22} Прими из уст Его закон и положи слова Его в сердце твое.
\end{tcolorbox}
\begin{tcolorbox}
\textsubscript{23} Если ты обратишься к Вседержителю, то вновь устроишься, удалишь беззаконие от шатра твоего
\end{tcolorbox}
\begin{tcolorbox}
\textsubscript{24} и будешь вменять в прах блестящий металл, и в камни потоков--[золото] Офирское.
\end{tcolorbox}
\begin{tcolorbox}
\textsubscript{25} И будет Вседержитель твоим золотом и блестящим серебром у тебя,
\end{tcolorbox}
\begin{tcolorbox}
\textsubscript{26} ибо тогда будешь радоваться о Вседержителе и поднимешь к Богу лице твое.
\end{tcolorbox}
\begin{tcolorbox}
\textsubscript{27} Помолишься Ему, и Он услышит тебя, и ты исполнишь обеты твои.
\end{tcolorbox}
\begin{tcolorbox}
\textsubscript{28} Положишь намерение, и оно состоится у тебя, и над путями твоими будет сиять свет.
\end{tcolorbox}
\begin{tcolorbox}
\textsubscript{29} Когда кто уничижен будет, ты скажешь: возвышение! и Он спасет поникшего лицем,
\end{tcolorbox}
\begin{tcolorbox}
\textsubscript{30} избавит и небезвинного, и он спасется чистотою рук твоих.
\end{tcolorbox}
\subsection{CHAPTER 23}
\begin{tcolorbox}
\textsubscript{1} И отвечал Иов и сказал:
\end{tcolorbox}
\begin{tcolorbox}
\textsubscript{2} еще и ныне горька речь моя: страдания мои тяжелее стонов моих.
\end{tcolorbox}
\begin{tcolorbox}
\textsubscript{3} О, если бы я знал, где найти Его, и мог подойти к престолу Его!
\end{tcolorbox}
\begin{tcolorbox}
\textsubscript{4} Я изложил бы пред Ним дело мое и уста мои наполнил бы оправданиями;
\end{tcolorbox}
\begin{tcolorbox}
\textsubscript{5} узнал бы слова, какими Он ответит мне, и понял бы, что Он скажет мне.
\end{tcolorbox}
\begin{tcolorbox}
\textsubscript{6} Неужели Он в полном могуществе стал бы состязаться со мною? О, нет! Пусть Он только обратил бы внимание на меня.
\end{tcolorbox}
\begin{tcolorbox}
\textsubscript{7} Тогда праведник мог бы состязаться с Ним, --и я навсегда получил бы свободу от Судии моего.
\end{tcolorbox}
\begin{tcolorbox}
\textsubscript{8} Но вот, я иду вперед--и нет Его, назад--и не нахожу Его;
\end{tcolorbox}
\begin{tcolorbox}
\textsubscript{9} делает ли Он что на левой стороне, я не вижу; скрывается ли на правой, не усматриваю.
\end{tcolorbox}
\begin{tcolorbox}
\textsubscript{10} Но Он знает путь мой; пусть испытает меня, --выйду, как золото.
\end{tcolorbox}
\begin{tcolorbox}
\textsubscript{11} Нога моя твердо держится стези Его; пути Его я хранил и не уклонялся.
\end{tcolorbox}
\begin{tcolorbox}
\textsubscript{12} От заповеди уст Его не отступал; глаголы уст Его хранил больше, нежели мои правила.
\end{tcolorbox}
\begin{tcolorbox}
\textsubscript{13} Но Он тверд; и кто отклонит Его? Он делает, чего хочет душа Его.
\end{tcolorbox}
\begin{tcolorbox}
\textsubscript{14} Так, Он выполнит положенное мне, и подобного этому много у Него.
\end{tcolorbox}
\begin{tcolorbox}
\textsubscript{15} Поэтому я трепещу пред лицем Его; размышляю--и страшусь Его.
\end{tcolorbox}
\begin{tcolorbox}
\textsubscript{16} Бог расслабил сердце мое, и Вседержитель устрашил меня.
\end{tcolorbox}
\begin{tcolorbox}
\textsubscript{17} Зачем я не уничтожен прежде этой тьмы, и Он не сокрыл мрака от лица моего!
\end{tcolorbox}
\subsection{CHAPTER 24}
\begin{tcolorbox}
\textsubscript{1} Почему не сокрыты от Вседержителя времена, и знающие Его не видят дней Его?
\end{tcolorbox}
\begin{tcolorbox}
\textsubscript{2} Межи передвигают, угоняют стада и пасут [у себя].
\end{tcolorbox}
\begin{tcolorbox}
\textsubscript{3} У сирот уводят осла, у вдовы берут в залог вола;
\end{tcolorbox}
\begin{tcolorbox}
\textsubscript{4} бедных сталкивают с дороги, все уничиженные земли принуждены скрываться.
\end{tcolorbox}
\begin{tcolorbox}
\textsubscript{5} Вот они, [как] дикие ослы в пустыне, выходят на дело свое, вставая рано на добычу; степь [дает] хлеб для них и для детей их;
\end{tcolorbox}
\begin{tcolorbox}
\textsubscript{6} жнут они на поле не своем и собирают виноград у нечестивца;
\end{tcolorbox}
\begin{tcolorbox}
\textsubscript{7} нагие ночуют без покрова и без одеяния на стуже;
\end{tcolorbox}
\begin{tcolorbox}
\textsubscript{8} мокнут от горных дождей и, не имея убежища, жмутся к скале;
\end{tcolorbox}
\begin{tcolorbox}
\textsubscript{9} отторгают от сосцов сироту и с нищего берут залог;
\end{tcolorbox}
\begin{tcolorbox}
\textsubscript{10} заставляют ходить нагими, без одеяния, и голодных кормят колосьями;
\end{tcolorbox}
\begin{tcolorbox}
\textsubscript{11} между стенами выжимают масло оливковое, топчут в точилах и жаждут.
\end{tcolorbox}
\begin{tcolorbox}
\textsubscript{12} В городе люди стонут, и душа убиваемых вопит, и Бог не воспрещает того.
\end{tcolorbox}
\begin{tcolorbox}
\textsubscript{13} Есть из них враги света, не знают путей его и не ходят по стезям его.
\end{tcolorbox}
\begin{tcolorbox}
\textsubscript{14} С рассветом встает убийца, умерщвляет бедного и нищего, а ночью бывает вором.
\end{tcolorbox}
\begin{tcolorbox}
\textsubscript{15} И око прелюбодея ждет сумерков, говоря: ничей глаз не увидит меня, --и закрывает лице.
\end{tcolorbox}
\begin{tcolorbox}
\textsubscript{16} В темноте подкапываются под домы, которые днем они заметили для себя; не знают света.
\end{tcolorbox}
\begin{tcolorbox}
\textsubscript{17} Ибо для них утро--смертная тень, так как они знакомы с ужасами смертной тени.
\end{tcolorbox}
\begin{tcolorbox}
\textsubscript{18} Легок такой на поверхности воды, проклята часть его на земле, и не смотрит он на дорогу садов виноградных.
\end{tcolorbox}
\begin{tcolorbox}
\textsubscript{19} Засуха и жара поглощают снежную воду: так преисподняя--грешников.
\end{tcolorbox}
\begin{tcolorbox}
\textsubscript{20} Пусть забудет его утроба [матери]; пусть лакомится им червь; пусть не остается о нем память; как дерево, пусть сломится беззаконник,
\end{tcolorbox}
\begin{tcolorbox}
\textsubscript{21} который угнетает бездетную, не рождавшую, и вдове не делает добра.
\end{tcolorbox}
\begin{tcolorbox}
\textsubscript{22} Он и сильных увлекает своею силою; он встает и никто не уверен за жизнь свою.
\end{tcolorbox}
\begin{tcolorbox}
\textsubscript{23} А Он дает ему [все] для безопасности, и он [на то] опирается, и очи Его видят пути их.
\end{tcolorbox}
\begin{tcolorbox}
\textsubscript{24} Поднялись высоко, --и вот, нет их; падают и умирают, как и все, и, как верхушки колосьев, срезываются.
\end{tcolorbox}
\begin{tcolorbox}
\textsubscript{25} Если это не так, --кто обличит меня во лжи и в ничто обратит речь мою?
\end{tcolorbox}
\subsection{CHAPTER 25}
\begin{tcolorbox}
\textsubscript{1} И отвечал Вилдад Савхеянин и сказал:
\end{tcolorbox}
\begin{tcolorbox}
\textsubscript{2} держава и страх у Него; Он творит мир на высотах Своих!
\end{tcolorbox}
\begin{tcolorbox}
\textsubscript{3} Есть ли счет воинствам Его? и над кем не восходит свет Его?
\end{tcolorbox}
\begin{tcolorbox}
\textsubscript{4} И как человеку быть правым пред Богом, и как быть чистым рожденному женщиною?
\end{tcolorbox}
\begin{tcolorbox}
\textsubscript{5} Вот даже луна, и та несветла, и звезды нечисты пред очами Его.
\end{tcolorbox}
\begin{tcolorbox}
\textsubscript{6} Тем менее человек, [который] есть червь, и сын человеческий, [который] есть моль.
\end{tcolorbox}
\subsection{CHAPTER 26}
\begin{tcolorbox}
\textsubscript{1} И отвечал Иов и сказал:
\end{tcolorbox}
\begin{tcolorbox}
\textsubscript{2} как ты помог бессильному, поддержал мышцу немощного!
\end{tcolorbox}
\begin{tcolorbox}
\textsubscript{3} Какой совет подал ты немудрому и как во всей полноте объяснил дело!
\end{tcolorbox}
\begin{tcolorbox}
\textsubscript{4} Кому ты говорил эти слова, и чей дух исходил из тебя?
\end{tcolorbox}
\begin{tcolorbox}
\textsubscript{5} Рефаимы трепещут под водами, и живущие в них.
\end{tcolorbox}
\begin{tcolorbox}
\textsubscript{6} Преисподняя обнажена пред Ним, и нет покрывала Аваддону.
\end{tcolorbox}
\begin{tcolorbox}
\textsubscript{7} Он распростер север над пустотою, повесил землю ни на чем.
\end{tcolorbox}
\begin{tcolorbox}
\textsubscript{8} Он заключает воды в облаках Своих, и облако не расседается под ними.
\end{tcolorbox}
\begin{tcolorbox}
\textsubscript{9} Он поставил престол Свой, распростер над ним облако Свое.
\end{tcolorbox}
\begin{tcolorbox}
\textsubscript{10} Черту провел над поверхностью воды, до границ света со тьмою.
\end{tcolorbox}
\begin{tcolorbox}
\textsubscript{11} Столпы небес дрожат и ужасаются от грозы Его.
\end{tcolorbox}
\begin{tcolorbox}
\textsubscript{12} Силою Своею волнует море и разумом Своим сражает его дерзость.
\end{tcolorbox}
\begin{tcolorbox}
\textsubscript{13} От духа Его--великолепие неба; рука Его образовала быстрого скорпиона.
\end{tcolorbox}
\begin{tcolorbox}
\textsubscript{14} Вот, это части путей Его; и как мало мы слышали о Нем! А гром могущества Его кто может уразуметь?
\end{tcolorbox}
\subsection{CHAPTER 27}
\begin{tcolorbox}
\textsubscript{1} И продолжал Иов возвышенную речь свою и сказал:
\end{tcolorbox}
\begin{tcolorbox}
\textsubscript{2} жив Бог, лишивший [меня] суда, и Вседержитель, огорчивший душу мою,
\end{tcolorbox}
\begin{tcolorbox}
\textsubscript{3} что, доколе еще дыхание мое во мне и дух Божий в ноздрях моих,
\end{tcolorbox}
\begin{tcolorbox}
\textsubscript{4} не скажут уста мои неправды, и язык мой не произнесет лжи!
\end{tcolorbox}
\begin{tcolorbox}
\textsubscript{5} Далек я от того, чтобы признать вас справедливыми; доколе не умру, не уступлю непорочности моей.
\end{tcolorbox}
\begin{tcolorbox}
\textsubscript{6} Крепко держал я правду мою и не опущу ее; не укорит меня сердце мое во все дни мои.
\end{tcolorbox}
\begin{tcolorbox}
\textsubscript{7} Враг мой будет, как нечестивец, и восстающий на меня, как беззаконник.
\end{tcolorbox}
\begin{tcolorbox}
\textsubscript{8} Ибо какая надежда лицемеру, когда возьмет, когда исторгнет Бог душу его?
\end{tcolorbox}
\begin{tcolorbox}
\textsubscript{9} Услышит ли Бог вопль его, когда придет на него беда?
\end{tcolorbox}
\begin{tcolorbox}
\textsubscript{10} Будет ли он утешаться Вседержителем и призывать Бога во всякое время?
\end{tcolorbox}
\begin{tcolorbox}
\textsubscript{11} Возвещу вам, что в руке Божией; что у Вседержителя, не скрою.
\end{tcolorbox}
\begin{tcolorbox}
\textsubscript{12} Вот, все вы и сами видели; и для чего вы столько пустословите?
\end{tcolorbox}
\begin{tcolorbox}
\textsubscript{13} Вот доля человеку беззаконному от Бога, и наследие, какое получают от Вседержителя притеснители.
\end{tcolorbox}
\begin{tcolorbox}
\textsubscript{14} Если умножаются сыновья его, то под меч; и потомки его не насытятся хлебом.
\end{tcolorbox}
\begin{tcolorbox}
\textsubscript{15} Оставшихся по нем смерть низведет во гроб, и вдовы их не будут плакать.
\end{tcolorbox}
\begin{tcolorbox}
\textsubscript{16} Если он наберет кучи серебра, как праха, и наготовит одежд, как брение,
\end{tcolorbox}
\begin{tcolorbox}
\textsubscript{17} то он наготовит, а одеваться будет праведник, и серебро получит себе на долю беспорочный.
\end{tcolorbox}
\begin{tcolorbox}
\textsubscript{18} Он строит, как моль, дом свой и, как сторож, делает себе шалаш;
\end{tcolorbox}
\begin{tcolorbox}
\textsubscript{19} ложится спать богачом и таким не встанет; открывает глаза свои, и он уже не тот.
\end{tcolorbox}
\begin{tcolorbox}
\textsubscript{20} Как воды, постигнут его ужасы; в ночи похитит его буря.
\end{tcolorbox}
\begin{tcolorbox}
\textsubscript{21} Поднимет его восточный ветер и понесет, и он быстро побежит от него.
\end{tcolorbox}
\begin{tcolorbox}
\textsubscript{22} Устремится на него и не пощадит, как бы он ни силился убежать от руки его.
\end{tcolorbox}
\begin{tcolorbox}
\textsubscript{23} Всплеснут о нем руками и посвищут над ним с места его!
\end{tcolorbox}
\subsection{CHAPTER 28}
\begin{tcolorbox}
\textsubscript{1} Так! у серебра есть источная жила, и у золота место, [где его] плавят.
\end{tcolorbox}
\begin{tcolorbox}
\textsubscript{2} Железо получается из земли; из камня выплавляется медь.
\end{tcolorbox}
\begin{tcolorbox}
\textsubscript{3} [Человек] полагает предел тьме и тщательно разыскивает камень во мраке и тени смертной.
\end{tcolorbox}
\begin{tcolorbox}
\textsubscript{4} Вырывают рудокопный колодезь в местах, забытых ногою, спускаются вглубь, висят [и] зыблются вдали от людей.
\end{tcolorbox}
\begin{tcolorbox}
\textsubscript{5} Земля, на которой вырастает хлеб, внутри изрыта как бы огнем.
\end{tcolorbox}
\begin{tcolorbox}
\textsubscript{6} Камни ее--место сапфира, и в ней песчинки золота.
\end{tcolorbox}
\begin{tcolorbox}
\textsubscript{7} Стези [туда] не знает хищная птица, и не видал ее глаз коршуна;
\end{tcolorbox}
\begin{tcolorbox}
\textsubscript{8} не попирали ее скимны, и не ходил по ней шакал.
\end{tcolorbox}
\begin{tcolorbox}
\textsubscript{9} На гранит налагает он руку свою, с корнем опрокидывает горы;
\end{tcolorbox}
\begin{tcolorbox}
\textsubscript{10} в скалах просекает каналы, и все драгоценное видит глаз его;
\end{tcolorbox}
\begin{tcolorbox}
\textsubscript{11} останавливает течение потоков и сокровенное выносит на свет.
\end{tcolorbox}
\begin{tcolorbox}
\textsubscript{12} Но где премудрость обретается? и где место разума?
\end{tcolorbox}
\begin{tcolorbox}
\textsubscript{13} Не знает человек цены ее, и она не обретается на земле живых.
\end{tcolorbox}
\begin{tcolorbox}
\textsubscript{14} Бездна говорит: не во мне она; и море говорит: не у меня.
\end{tcolorbox}
\begin{tcolorbox}
\textsubscript{15} Не дается она за золото и не приобретается она за вес серебра;
\end{tcolorbox}
\begin{tcolorbox}
\textsubscript{16} не оценивается она золотом Офирским, ни драгоценным ониксом, ни сапфиром;
\end{tcolorbox}
\begin{tcolorbox}
\textsubscript{17} не равняется с нею золото и кристалл, и не выменяешь ее на сосуды из чистого золота.
\end{tcolorbox}
\begin{tcolorbox}
\textsubscript{18} А о кораллах и жемчуге и упоминать нечего, и приобретение премудрости выше рубинов.
\end{tcolorbox}
\begin{tcolorbox}
\textsubscript{19} Не равняется с нею топаз Ефиопский; чистым золотом не оценивается она.
\end{tcolorbox}
\begin{tcolorbox}
\textsubscript{20} Откуда же исходит премудрость? и где место разума?
\end{tcolorbox}
\begin{tcolorbox}
\textsubscript{21} Сокрыта она от очей всего живущего и от птиц небесных утаена.
\end{tcolorbox}
\begin{tcolorbox}
\textsubscript{22} Аваддон и смерть говорят: ушами нашими слышали мы слух о ней.
\end{tcolorbox}
\begin{tcolorbox}
\textsubscript{23} Бог знает путь ее, и Он ведает место ее.
\end{tcolorbox}
\begin{tcolorbox}
\textsubscript{24} Ибо Он прозирает до концов земли и видит под всем небом.
\end{tcolorbox}
\begin{tcolorbox}
\textsubscript{25} Когда Он ветру полагал вес и располагал воду по мере,
\end{tcolorbox}
\begin{tcolorbox}
\textsubscript{26} когда назначал устав дождю и путь для молнии громоносной,
\end{tcolorbox}
\begin{tcolorbox}
\textsubscript{27} тогда Он видел ее и явил ее, приготовил ее и еще испытал ее
\end{tcolorbox}
\begin{tcolorbox}
\textsubscript{28} и сказал человеку: вот, страх Господень есть истинная премудрость, и удаление от зла--разум.
\end{tcolorbox}
\subsection{CHAPTER 29}
\begin{tcolorbox}
\textsubscript{1} И продолжал Иов возвышенную речь свою и сказал:
\end{tcolorbox}
\begin{tcolorbox}
\textsubscript{2} о, если бы я был, как в прежние месяцы, как в те дни, когда Бог хранил меня,
\end{tcolorbox}
\begin{tcolorbox}
\textsubscript{3} когда светильник Его светил над головою моею, и я при свете Его ходил среди тьмы;
\end{tcolorbox}
\begin{tcolorbox}
\textsubscript{4} как был я во дни молодости моей, когда милость Божия [была] над шатром моим,
\end{tcolorbox}
\begin{tcolorbox}
\textsubscript{5} когда еще Вседержитель [был] со мною, и дети мои вокруг меня,
\end{tcolorbox}
\begin{tcolorbox}
\textsubscript{6} когда пути мои обливались молоком, и скала источала для меня ручьи елея!
\end{tcolorbox}
\begin{tcolorbox}
\textsubscript{7} когда я выходил к воротам города и на площади ставил седалище свое, --
\end{tcolorbox}
\begin{tcolorbox}
\textsubscript{8} юноши, увидев меня, прятались, а старцы вставали и стояли;
\end{tcolorbox}
\begin{tcolorbox}
\textsubscript{9} князья удерживались от речи и персты полагали на уста свои;
\end{tcolorbox}
\begin{tcolorbox}
\textsubscript{10} голос знатных умолкал, и язык их прилипал к гортани их.
\end{tcolorbox}
\begin{tcolorbox}
\textsubscript{11} Ухо, слышавшее меня, ублажало меня; око видевшее восхваляло меня,
\end{tcolorbox}
\begin{tcolorbox}
\textsubscript{12} потому что я спасал страдальца вопиющего и сироту беспомощного.
\end{tcolorbox}
\begin{tcolorbox}
\textsubscript{13} Благословение погибавшего приходило на меня, и сердцу вдовы доставлял я радость.
\end{tcolorbox}
\begin{tcolorbox}
\textsubscript{14} Я облекался в правду, и суд мой одевал меня, как мантия и увясло.
\end{tcolorbox}
\begin{tcolorbox}
\textsubscript{15} Я был глазами слепому и ногами хромому;
\end{tcolorbox}
\begin{tcolorbox}
\textsubscript{16} отцом был я для нищих и тяжбу, которой я не знал, разбирал внимательно.
\end{tcolorbox}
\begin{tcolorbox}
\textsubscript{17} Сокрушал я беззаконному челюсти и из зубов его исторгал похищенное.
\end{tcolorbox}
\begin{tcolorbox}
\textsubscript{18} И говорил я: в гнезде моем скончаюсь, и дни [мои] будут многи, как песок;
\end{tcolorbox}
\begin{tcolorbox}
\textsubscript{19} корень мой открыт для воды, и роса ночует на ветвях моих;
\end{tcolorbox}
\begin{tcolorbox}
\textsubscript{20} слава моя не стареет, лук мой крепок в руке моей.
\end{tcolorbox}
\begin{tcolorbox}
\textsubscript{21} Внимали мне и ожидали, и безмолвствовали при совете моем.
\end{tcolorbox}
\begin{tcolorbox}
\textsubscript{22} После слов моих уже не рассуждали; речь моя капала на них.
\end{tcolorbox}
\begin{tcolorbox}
\textsubscript{23} Ждали меня, как дождя, и, [как] дождю позднему, открывали уста свои.
\end{tcolorbox}
\begin{tcolorbox}
\textsubscript{24} Бывало, улыбнусь им--они не верят; и света лица моего они не помрачали.
\end{tcolorbox}
\begin{tcolorbox}
\textsubscript{25} Я назначал пути им и сидел во главе и жил как царь в кругу воинов, как утешитель плачущих.
\end{tcolorbox}
\subsection{CHAPTER 30}
\begin{tcolorbox}
\textsubscript{1} А ныне смеются надо мною младшие меня летами, те, которых отцов я не согласился бы поместить с псами стад моих.
\end{tcolorbox}
\begin{tcolorbox}
\textsubscript{2} И сила рук их к чему мне? Над ними уже прошло время.
\end{tcolorbox}
\begin{tcolorbox}
\textsubscript{3} Бедностью и голодом истощенные, они убегают в степь безводную, мрачную и опустевшую;
\end{tcolorbox}
\begin{tcolorbox}
\textsubscript{4} щиплют зелень подле кустов, и ягоды можжевельника--хлеб их.
\end{tcolorbox}
\begin{tcolorbox}
\textsubscript{5} Из общества изгоняют их, кричат на них, как на воров,
\end{tcolorbox}
\begin{tcolorbox}
\textsubscript{6} чтобы жили они в рытвинах потоков, в ущельях земли и утесов.
\end{tcolorbox}
\begin{tcolorbox}
\textsubscript{7} Ревут между кустами, жмутся под терном.
\end{tcolorbox}
\begin{tcolorbox}
\textsubscript{8} Люди отверженные, люди без имени, отребье земли!
\end{tcolorbox}
\begin{tcolorbox}
\textsubscript{9} Их-то сделался я ныне песнью и пищею разговора их.
\end{tcolorbox}
\begin{tcolorbox}
\textsubscript{10} Они гнушаются мною, удаляются от меня и не удерживаются плевать пред лицем моим.
\end{tcolorbox}
\begin{tcolorbox}
\textsubscript{11} Так как Он развязал повод мой и поразил меня, то они сбросили с себя узду пред лицем моим.
\end{tcolorbox}
\begin{tcolorbox}
\textsubscript{12} С правого боку встает это исчадие, сбивает меня с ног, направляет гибельные свои пути ко мне.
\end{tcolorbox}
\begin{tcolorbox}
\textsubscript{13} А мою стезю испортили: всё успели сделать к моей погибели, не имея помощника.
\end{tcolorbox}
\begin{tcolorbox}
\textsubscript{14} Они пришли ко мне, как сквозь широкий пролом; с шумом бросились на меня.
\end{tcolorbox}
\begin{tcolorbox}
\textsubscript{15} Ужасы устремились на меня; как ветер, развеялось величие мое, и счастье мое унеслось, как облако.
\end{tcolorbox}
\begin{tcolorbox}
\textsubscript{16} И ныне изливается душа моя во мне: дни скорби объяли меня.
\end{tcolorbox}
\begin{tcolorbox}
\textsubscript{17} Ночью ноют во мне кости мои, и жилы мои не имеют покоя.
\end{tcolorbox}
\begin{tcolorbox}
\textsubscript{18} С великим трудом снимается с меня одежда моя; края хитона моего жмут меня.
\end{tcolorbox}
\begin{tcolorbox}
\textsubscript{19} Он бросил меня в грязь, и я стал, как прах и пепел.
\end{tcolorbox}
\begin{tcolorbox}
\textsubscript{20} Я взываю к Тебе, и Ты не внимаешь мне, --стою, а Ты [только] смотришь на меня.
\end{tcolorbox}
\begin{tcolorbox}
\textsubscript{21} Ты сделался жестоким ко мне, крепкою рукою враждуешь против меня.
\end{tcolorbox}
\begin{tcolorbox}
\textsubscript{22} Ты поднял меня и заставил меня носиться по ветру и сокрушаешь меня.
\end{tcolorbox}
\begin{tcolorbox}
\textsubscript{23} Так, я знаю, что Ты приведешь меня к смерти и в дом собрания всех живущих.
\end{tcolorbox}
\begin{tcolorbox}
\textsubscript{24} Верно, Он не прострет руки Своей на дом костей: будут ли они кричать при своем разрушении?
\end{tcolorbox}
\begin{tcolorbox}
\textsubscript{25} Не плакал ли я о том, кто был в горе? не скорбела ли душа моя о бедных?
\end{tcolorbox}
\begin{tcolorbox}
\textsubscript{26} Когда я чаял добра, пришло зло; когда ожидал света, пришла тьма.
\end{tcolorbox}
\begin{tcolorbox}
\textsubscript{27} Мои внутренности кипят и не перестают; встретили меня дни печали.
\end{tcolorbox}
\begin{tcolorbox}
\textsubscript{28} Я хожу почернелый, но не от солнца; встаю в собрании и кричу.
\end{tcolorbox}
\begin{tcolorbox}
\textsubscript{29} Я стал братом шакалам и другом страусам.
\end{tcolorbox}
\begin{tcolorbox}
\textsubscript{30} Моя кожа почернела на мне, и кости мои обгорели от жара.
\end{tcolorbox}
\begin{tcolorbox}
\textsubscript{31} И цитра моя сделалась унылою, и свирель моя--голосом плачевным.
\end{tcolorbox}
\subsection{CHAPTER 31}
\begin{tcolorbox}
\textsubscript{1} Завет положил я с глазами моими, чтобы не помышлять мне о девице.
\end{tcolorbox}
\begin{tcolorbox}
\textsubscript{2} Какая же участь [мне] от Бога свыше? И какое наследие от Вседержителя с небес?
\end{tcolorbox}
\begin{tcolorbox}
\textsubscript{3} Не для нечестивого ли гибель, и не для делающего ли зло напасть?
\end{tcolorbox}
\begin{tcolorbox}
\textsubscript{4} Не видел ли Он путей моих, и не считал ли всех моих шагов?
\end{tcolorbox}
\begin{tcolorbox}
\textsubscript{5} Если я ходил в суете, и если нога моя спешила на лукавство, --
\end{tcolorbox}
\begin{tcolorbox}
\textsubscript{6} пусть взвесят меня на весах правды, и Бог узнает мою непорочность.
\end{tcolorbox}
\begin{tcolorbox}
\textsubscript{7} Если стопы мои уклонялись от пути и сердце мое следовало за глазами моими, и если что-либо [нечистое] пристало к рукам моим,
\end{tcolorbox}
\begin{tcolorbox}
\textsubscript{8} то пусть я сею, а другой ест, и пусть отрасли мои искоренены будут.
\end{tcolorbox}
\begin{tcolorbox}
\textsubscript{9} Если сердце мое прельщалось женщиною и я строил ковы у дверей моего ближнего, --
\end{tcolorbox}
\begin{tcolorbox}
\textsubscript{10} пусть моя жена мелет на другого, и пусть другие издеваются над нею,
\end{tcolorbox}
\begin{tcolorbox}
\textsubscript{11} потому что это--преступление, это--беззаконие, подлежащее суду;
\end{tcolorbox}
\begin{tcolorbox}
\textsubscript{12} это--огонь, поядающий до истребления, который искоренил бы все добро мое.
\end{tcolorbox}
\begin{tcolorbox}
\textsubscript{13} Если я пренебрегал правами слуги и служанки моей, когда они имели спор со мною,
\end{tcolorbox}
\begin{tcolorbox}
\textsubscript{14} то что стал бы я делать, когда бы Бог восстал? И когда бы Он взглянул на меня, что мог бы я отвечать Ему?
\end{tcolorbox}
\begin{tcolorbox}
\textsubscript{15} Не Он ли, Который создал меня во чреве, создал и его и равно образовал нас в утробе?
\end{tcolorbox}
\begin{tcolorbox}
\textsubscript{16} Отказывал ли я нуждающимся в их просьбе и томил ли глаза вдовы?
\end{tcolorbox}
\begin{tcolorbox}
\textsubscript{17} Один ли я съедал кусок мой, и не ел ли от него и сирота?
\end{tcolorbox}
\begin{tcolorbox}
\textsubscript{18} Ибо с детства он рос со мною, как с отцом, и от чрева матери моей я руководил [вдову].
\end{tcolorbox}
\begin{tcolorbox}
\textsubscript{19} Если я видел кого погибавшим без одежды и бедного без покрова, --
\end{tcolorbox}
\begin{tcolorbox}
\textsubscript{20} не благословляли ли меня чресла его, и не был ли он согрет шерстью овец моих?
\end{tcolorbox}
\begin{tcolorbox}
\textsubscript{21} Если я поднимал руку мою на сироту, когда видел помощь себе у ворот,
\end{tcolorbox}
\begin{tcolorbox}
\textsubscript{22} то пусть плечо мое отпадет от спины, и рука моя пусть отломится от локтя,
\end{tcolorbox}
\begin{tcolorbox}
\textsubscript{23} ибо страшно для меня наказание от Бога: пред величием Его не устоял бы я.
\end{tcolorbox}
\begin{tcolorbox}
\textsubscript{24} Полагал ли я в золоте опору мою и говорил ли сокровищу: ты--надежда моя?
\end{tcolorbox}
\begin{tcolorbox}
\textsubscript{25} Радовался ли я, что богатство мое было велико, и что рука моя приобрела много?
\end{tcolorbox}
\begin{tcolorbox}
\textsubscript{26} Смотря на солнце, как оно сияет, и на луну, как она величественно шествует,
\end{tcolorbox}
\begin{tcolorbox}
\textsubscript{27} прельстился ли я в тайне сердца моего, и целовали ли уста мои руку мою?
\end{tcolorbox}
\begin{tcolorbox}
\textsubscript{28} Это также было бы преступление, подлежащее суду, потому что я отрекся бы [тогда] от Бога Всевышнего.
\end{tcolorbox}
\begin{tcolorbox}
\textsubscript{29} Радовался ли я погибели врага моего и торжествовал ли, когда несчастье постигало его?
\end{tcolorbox}
\begin{tcolorbox}
\textsubscript{30} Не позволял я устам моим грешить проклятием души его.
\end{tcolorbox}
\begin{tcolorbox}
\textsubscript{31} Не говорили ли люди шатра моего: о, если бы мы от мяс его не насытились?
\end{tcolorbox}
\begin{tcolorbox}
\textsubscript{32} Странник не ночевал на улице; двери мои я отворял прохожему.
\end{tcolorbox}
\begin{tcolorbox}
\textsubscript{33} Если бы я скрывал проступки мои, как человек, утаивая в груди моей пороки мои,
\end{tcolorbox}
\begin{tcolorbox}
\textsubscript{34} то я боялся бы большого общества, и презрение одноплеменников страшило бы меня, и я молчал бы и не выходил бы за двери.
\end{tcolorbox}
\begin{tcolorbox}
\textsubscript{35} О, если бы кто выслушал меня! Вот мое желание, чтобы Вседержитель отвечал мне, и чтобы защитник мой составил запись.
\end{tcolorbox}
\begin{tcolorbox}
\textsubscript{36} Я носил бы ее на плечах моих и возлагал бы ее, как венец;
\end{tcolorbox}
\begin{tcolorbox}
\textsubscript{37} объявил бы ему число шагов моих, сблизился бы с ним, как с князем.
\end{tcolorbox}
\begin{tcolorbox}
\textsubscript{38} Если вопияла на меня земля моя и жаловались на меня борозды ее;
\end{tcolorbox}
\begin{tcolorbox}
\textsubscript{39} если я ел плоды ее без платы и отягощал жизнь земледельцев,
\end{tcolorbox}
\begin{tcolorbox}
\textsubscript{40} то пусть вместо пшеницы вырастает волчец и вместо ячменя куколь. Слова Иова кончились.
\end{tcolorbox}
\subsection{CHAPTER 32}
\begin{tcolorbox}
\textsubscript{1} Когда те три мужа перестали отвечать Иову, потому что он был прав в глазах своих,
\end{tcolorbox}
\begin{tcolorbox}
\textsubscript{2} тогда воспылал гнев Елиуя, сына Варахиилова, Вузитянина из племени Рамова: воспылал гнев его на Иова за то, что он оправдывал себя больше, нежели Бога,
\end{tcolorbox}
\begin{tcolorbox}
\textsubscript{3} а на трех друзей его воспылал гнев его за то, что они не нашли, что отвечать, а между тем обвиняли Иова.
\end{tcolorbox}
\begin{tcolorbox}
\textsubscript{4} Елиуй ждал, пока Иов говорил, потому что они летами были старше его.
\end{tcolorbox}
\begin{tcolorbox}
\textsubscript{5} Когда же Елиуй увидел, что нет ответа в устах тех трех мужей, тогда воспылал гнев его.
\end{tcolorbox}
\begin{tcolorbox}
\textsubscript{6} И отвечал Елиуй, сын Варахиилов, Вузитянин, и сказал: я молод летами, а вы--старцы; поэтому я робел и боялся объявлять вам мое мнение.
\end{tcolorbox}
\begin{tcolorbox}
\textsubscript{7} Я говорил сам себе: пусть говорят дни, и многолетие поучает мудрости.
\end{tcolorbox}
\begin{tcolorbox}
\textsubscript{8} Но дух в человеке и дыхание Вседержителя дает ему разумение.
\end{tcolorbox}
\begin{tcolorbox}
\textsubscript{9} Не многолетние [только] мудры, и не старики разумеют правду.
\end{tcolorbox}
\begin{tcolorbox}
\textsubscript{10} Поэтому я говорю: выслушайте меня, объявлю вам мое мнение и я.
\end{tcolorbox}
\begin{tcolorbox}
\textsubscript{11} Вот, я ожидал слов ваших, --вслушивался в суждения ваши, доколе вы придумывали, что сказать.
\end{tcolorbox}
\begin{tcolorbox}
\textsubscript{12} Я пристально смотрел на вас, и вот никто из вас не обличает Иова и не отвечает на слова его.
\end{tcolorbox}
\begin{tcolorbox}
\textsubscript{13} Не скажите: мы нашли мудрость: Бог опровергнет его, а не человек.
\end{tcolorbox}
\begin{tcolorbox}
\textsubscript{14} Если бы он обращал слова свои ко мне, то я не вашими речами отвечал бы ему.
\end{tcolorbox}
\begin{tcolorbox}
\textsubscript{15} Испугались, не отвечают более; перестали говорить.
\end{tcolorbox}
\begin{tcolorbox}
\textsubscript{16} И как я ждал, а они не говорят, остановились и не отвечают более,
\end{tcolorbox}
\begin{tcolorbox}
\textsubscript{17} то и я отвечу с моей стороны, объявлю мое мнение и я,
\end{tcolorbox}
\begin{tcolorbox}
\textsubscript{18} ибо я полон речами, и дух во мне теснит меня.
\end{tcolorbox}
\begin{tcolorbox}
\textsubscript{19} Вот, утроба моя, как вино неоткрытое: она готова прорваться, подобно новым мехам.
\end{tcolorbox}
\begin{tcolorbox}
\textsubscript{20} Поговорю, и будет легче мне; открою уста мои и отвечу.
\end{tcolorbox}
\begin{tcolorbox}
\textsubscript{21} На лице человека смотреть не буду и никакому человеку льстить не стану,
\end{tcolorbox}
\begin{tcolorbox}
\textsubscript{22} потому что я не умею льстить: сейчас убей меня, Творец мой.
\end{tcolorbox}
\subsection{CHAPTER 33}
\begin{tcolorbox}
\textsubscript{1} Итак слушай, Иов, речи мои и внимай всем словам моим.
\end{tcolorbox}
\begin{tcolorbox}
\textsubscript{2} Вот, я открываю уста мои, язык мой говорит в гортани моей.
\end{tcolorbox}
\begin{tcolorbox}
\textsubscript{3} Слова мои от искренности моего сердца, и уста мои произнесут знание чистое.
\end{tcolorbox}
\begin{tcolorbox}
\textsubscript{4} Дух Божий создал меня, и дыхание Вседержителя дало мне жизнь.
\end{tcolorbox}
\begin{tcolorbox}
\textsubscript{5} Если можешь, отвечай мне и стань передо мною.
\end{tcolorbox}
\begin{tcolorbox}
\textsubscript{6} Вот я, по желанию твоему, вместо Бога. Я образован также из брения;
\end{tcolorbox}
\begin{tcolorbox}
\textsubscript{7} поэтому страх передо мною не может смутить тебя, и рука моя не будет тяжела для тебя.
\end{tcolorbox}
\begin{tcolorbox}
\textsubscript{8} Ты говорил в уши мои, и я слышал звук слов:
\end{tcolorbox}
\begin{tcolorbox}
\textsubscript{9} чист я, без порока, невинен я, и нет во мне неправды;
\end{tcolorbox}
\begin{tcolorbox}
\textsubscript{10} а Он нашел обвинение против меня и считает меня Своим противником;
\end{tcolorbox}
\begin{tcolorbox}
\textsubscript{11} поставил ноги мои в колоду, наблюдает за всеми путями моими.
\end{tcolorbox}
\begin{tcolorbox}
\textsubscript{12} Вот в этом ты неправ, отвечаю тебе, потому что Бог выше человека.
\end{tcolorbox}
\begin{tcolorbox}
\textsubscript{13} Для чего тебе состязаться с Ним? Он не дает отчета ни в каких делах Своих.
\end{tcolorbox}
\begin{tcolorbox}
\textsubscript{14} Бог говорит однажды и, если того не заметят, в другой раз:
\end{tcolorbox}
\begin{tcolorbox}
\textsubscript{15} во сне, в ночном видении, когда сон находит на людей, во время дремоты на ложе.
\end{tcolorbox}
\begin{tcolorbox}
\textsubscript{16} Тогда Он открывает у человека ухо и запечатлевает Свое наставление,
\end{tcolorbox}
\begin{tcolorbox}
\textsubscript{17} чтобы отвести человека от какого-либо предприятия и удалить от него гордость,
\end{tcolorbox}
\begin{tcolorbox}
\textsubscript{18} чтобы отвести душу его от пропасти и жизнь его от поражения мечом.
\end{tcolorbox}
\begin{tcolorbox}
\textsubscript{19} Или он вразумляется болезнью на ложе своем и жестокою болью во всех костях своих, --
\end{tcolorbox}
\begin{tcolorbox}
\textsubscript{20} и жизнь его отвращается от хлеба и душа его от любимой пищи.
\end{tcolorbox}
\begin{tcolorbox}
\textsubscript{21} Плоть на нем пропадает, так что ее не видно, и показываются кости его, которых не было видно.
\end{tcolorbox}
\begin{tcolorbox}
\textsubscript{22} И душа его приближается к могиле и жизнь его--к смерти.
\end{tcolorbox}
\begin{tcolorbox}
\textsubscript{23} Если есть у него Ангел-наставник, один из тысячи, чтобы показать человеку прямой [путь] его, --
\end{tcolorbox}
\begin{tcolorbox}
\textsubscript{24} [Бог] умилосердится над ним и скажет: освободи его от могилы; Я нашел умилостивление.
\end{tcolorbox}
\begin{tcolorbox}
\textsubscript{25} Тогда тело его сделается свежее, нежели в молодости; он возвратится к дням юности своей.
\end{tcolorbox}
\begin{tcolorbox}
\textsubscript{26} Будет молиться Богу, и Он--милостив к нему; с радостью взирает на лице его и возвращает человеку праведность его.
\end{tcolorbox}
\begin{tcolorbox}
\textsubscript{27} Он будет смотреть на людей и говорить: грешил я и превращал правду, и не воздано мне;
\end{tcolorbox}
\begin{tcolorbox}
\textsubscript{28} Он освободил душу мою от могилы, и жизнь моя видит свет.
\end{tcolorbox}
\begin{tcolorbox}
\textsubscript{29} Вот, все это делает Бог два-три раза с человеком,
\end{tcolorbox}
\begin{tcolorbox}
\textsubscript{30} чтобы отвести душу его от могилы и просветить его светом живых.
\end{tcolorbox}
\begin{tcolorbox}
\textsubscript{31} Внимай, Иов, слушай меня, молчи, и я буду говорить.
\end{tcolorbox}
\begin{tcolorbox}
\textsubscript{32} Если имеешь, что сказать, отвечай; говори, потому что я желал бы твоего оправдания;
\end{tcolorbox}
\begin{tcolorbox}
\textsubscript{33} если же нет, то слушай меня: молчи, и я научу тебя мудрости.
\end{tcolorbox}
\subsection{CHAPTER 34}
\begin{tcolorbox}
\textsubscript{1} И продолжал Елиуй и сказал:
\end{tcolorbox}
\begin{tcolorbox}
\textsubscript{2} выслушайте, мудрые, речь мою, и приклоните ко мне ухо, рассудительные!
\end{tcolorbox}
\begin{tcolorbox}
\textsubscript{3} Ибо ухо разбирает слова, как гортань различает вкус в пище.
\end{tcolorbox}
\begin{tcolorbox}
\textsubscript{4} Установим между собою рассуждение и распознаем, что хорошо.
\end{tcolorbox}
\begin{tcolorbox}
\textsubscript{5} Вот, Иов сказал: я прав, но Бог лишил меня суда.
\end{tcolorbox}
\begin{tcolorbox}
\textsubscript{6} Должен ли я лгать на правду мою? Моя рана неисцелима без вины.
\end{tcolorbox}
\begin{tcolorbox}
\textsubscript{7} Есть ли такой человек, как Иов, который пьет глумление, как воду,
\end{tcolorbox}
\begin{tcolorbox}
\textsubscript{8} вступает в сообщество с делающими беззаконие и ходит с людьми нечестивыми?
\end{tcolorbox}
\begin{tcolorbox}
\textsubscript{9} Потому что он сказал: нет пользы для человека в благоугождении Богу.
\end{tcolorbox}
\begin{tcolorbox}
\textsubscript{10} Итак послушайте меня, мужи мудрые! Не может быть у Бога неправда или у Вседержителя неправосудие,
\end{tcolorbox}
\begin{tcolorbox}
\textsubscript{11} ибо Он по делам человека поступает с ним и по путям мужа воздает ему.
\end{tcolorbox}
\begin{tcolorbox}
\textsubscript{12} Истинно, Бог не делает неправды и Вседержитель не извращает суда.
\end{tcolorbox}
\begin{tcolorbox}
\textsubscript{13} Кто кроме Его промышляет о земле? И кто управляет всею вселенною?
\end{tcolorbox}
\begin{tcolorbox}
\textsubscript{14} Если бы Он обратил сердце Свое к Себе и взял к Себе дух ее и дыхание ее, --
\end{tcolorbox}
\begin{tcolorbox}
\textsubscript{15} вдруг погибла бы всякая плоть, и человек возвратился бы в прах.
\end{tcolorbox}
\begin{tcolorbox}
\textsubscript{16} Итак, если ты имеешь разум, то слушай это и внимай словам моим.
\end{tcolorbox}
\begin{tcolorbox}
\textsubscript{17} Ненавидящий правду может ли владычествовать? И можешь ли ты обвинить Всеправедного?
\end{tcolorbox}
\begin{tcolorbox}
\textsubscript{18} Можно ли сказать царю: ты--нечестивец, и князьям: вы--беззаконники?
\end{tcolorbox}
\begin{tcolorbox}
\textsubscript{19} Но Он не смотрит и на лица князей и не предпочитает богатого бедному, потому что все они дело рук Его.
\end{tcolorbox}
\begin{tcolorbox}
\textsubscript{20} Внезапно они умирают; среди ночи народ возмутится, и они исчезают; и сильных изгоняют не силою.
\end{tcolorbox}
\begin{tcolorbox}
\textsubscript{21} Ибо очи Его над путями человека, и Он видит все шаги его.
\end{tcolorbox}
\begin{tcolorbox}
\textsubscript{22} Нет тьмы, ни тени смертной, где могли бы укрыться делающие беззаконие.
\end{tcolorbox}
\begin{tcolorbox}
\textsubscript{23} Потому Он уже не требует от человека, чтобы шел на суд с Богом.
\end{tcolorbox}
\begin{tcolorbox}
\textsubscript{24} Он сокрушает сильных без исследования и поставляет других на их места;
\end{tcolorbox}
\begin{tcolorbox}
\textsubscript{25} потому что Он делает известными дела их и низлагает их ночью, и они истребляются.
\end{tcolorbox}
\begin{tcolorbox}
\textsubscript{26} Он поражает их, как беззаконных людей, пред глазами других,
\end{tcolorbox}
\begin{tcolorbox}
\textsubscript{27} за то, что они отвратились от Него и не уразумели всех путей Его,
\end{tcolorbox}
\begin{tcolorbox}
\textsubscript{28} так что дошел до Него вопль бедных, и Он услышал стенание угнетенных.
\end{tcolorbox}
\begin{tcolorbox}
\textsubscript{29} Дарует ли Он тишину, кто может возмутить? скрывает ли Он лице Свое, кто может увидеть Его? Будет ли это для народа, или для одного человека,
\end{tcolorbox}
\begin{tcolorbox}
\textsubscript{30} чтобы не царствовал лицемер к соблазну народа.
\end{tcolorbox}
\begin{tcolorbox}
\textsubscript{31} К Богу должно говорить: я потерпел, больше не буду грешить.
\end{tcolorbox}
\begin{tcolorbox}
\textsubscript{32} А чего я не знаю, Ты научи меня; и если я сделал беззаконие, больше не буду.
\end{tcolorbox}
\begin{tcolorbox}
\textsubscript{33} По твоему ли [рассуждению] Он должен воздавать? И как ты отвергаешь, то тебе следует избирать, а не мне; говори, что знаешь.
\end{tcolorbox}
\begin{tcolorbox}
\textsubscript{34} Люди разумные скажут мне, и муж мудрый, слушающий меня:
\end{tcolorbox}
\begin{tcolorbox}
\textsubscript{35} Иов не умно говорит, и слова его не со смыслом.
\end{tcolorbox}
\begin{tcolorbox}
\textsubscript{36} Я желал бы, чтобы Иов вполне был испытан, по ответам его, свойственным людям нечестивым.
\end{tcolorbox}
\begin{tcolorbox}
\textsubscript{37} Иначе он ко греху своему прибавит отступление, будет рукоплескать между нами и еще больше наговорит против Бога.
\end{tcolorbox}
\subsection{CHAPTER 35}
\begin{tcolorbox}
\textsubscript{1} И продолжал Елиуй и сказал:
\end{tcolorbox}
\begin{tcolorbox}
\textsubscript{2} считаешь ли ты справедливым, что сказал: я правее Бога?
\end{tcolorbox}
\begin{tcolorbox}
\textsubscript{3} Ты сказал: что пользы мне? и какую прибыль я имел бы пред тем, как если бы я и грешил?
\end{tcolorbox}
\begin{tcolorbox}
\textsubscript{4} Я отвечу тебе и твоим друзьям с тобою:
\end{tcolorbox}
\begin{tcolorbox}
\textsubscript{5} взгляни на небо и смотри; воззри на облака, они выше тебя.
\end{tcolorbox}
\begin{tcolorbox}
\textsubscript{6} Если ты грешишь, что делаешь ты Ему? и если преступления твои умножаются, что причиняешь ты Ему?
\end{tcolorbox}
\begin{tcolorbox}
\textsubscript{7} Если ты праведен, что даешь Ему? или что получает Он от руки твоей?
\end{tcolorbox}
\begin{tcolorbox}
\textsubscript{8} Нечестие твое относится к человеку, как ты, и праведность твоя к сыну человеческому.
\end{tcolorbox}
\begin{tcolorbox}
\textsubscript{9} От множества притеснителей стонут притесняемые, и от руки сильных вопиют.
\end{tcolorbox}
\begin{tcolorbox}
\textsubscript{10} Но никто не говорит: где Бог, Творец мой, Который дает песни в ночи,
\end{tcolorbox}
\begin{tcolorbox}
\textsubscript{11} Который научает нас более, нежели скотов земных, и вразумляет нас более, нежели птиц небесных?
\end{tcolorbox}
\begin{tcolorbox}
\textsubscript{12} Там они вопиют, и Он не отвечает им, по причине гордости злых людей.
\end{tcolorbox}
\begin{tcolorbox}
\textsubscript{13} Но неправда, что Бог не слышит и Вседержитель не взирает на это.
\end{tcolorbox}
\begin{tcolorbox}
\textsubscript{14} Хотя ты сказал, что ты не видишь Его, но суд пред Ним, и--жди его.
\end{tcolorbox}
\begin{tcolorbox}
\textsubscript{15} Но ныне, потому что гнев Его не посетил его и он не познал его во всей строгости,
\end{tcolorbox}
\begin{tcolorbox}
\textsubscript{16} Иов и открыл легкомысленно уста свои и безрассудно расточает слова.
\end{tcolorbox}
\subsection{CHAPTER 36}
\begin{tcolorbox}
\textsubscript{1} И продолжал Елиуй и сказал:
\end{tcolorbox}
\begin{tcolorbox}
\textsubscript{2} подожди меня немного, и я покажу тебе, что я имею еще что сказать за Бога.
\end{tcolorbox}
\begin{tcolorbox}
\textsubscript{3} Начну мои рассуждения издалека и воздам Создателю моему справедливость,
\end{tcolorbox}
\begin{tcolorbox}
\textsubscript{4} потому что слова мои точно не ложь: пред тобою--совершенный в познаниях.
\end{tcolorbox}
\begin{tcolorbox}
\textsubscript{5} Вот, Бог могуществен и не презирает сильного крепостью сердца;
\end{tcolorbox}
\begin{tcolorbox}
\textsubscript{6} Он не поддерживает нечестивых и воздает должное угнетенным;
\end{tcolorbox}
\begin{tcolorbox}
\textsubscript{7} Он не отвращает очей Своих от праведников, но с царями навсегда посаждает их на престоле, и они возвышаются.
\end{tcolorbox}
\begin{tcolorbox}
\textsubscript{8} Если же они окованы цепями и содержатся в узах бедствия,
\end{tcolorbox}
\begin{tcolorbox}
\textsubscript{9} то Он указывает им на дела их и на беззакония их, потому что умножились,
\end{tcolorbox}
\begin{tcolorbox}
\textsubscript{10} и открывает их ухо для вразумления и говорит им, чтоб они отстали от нечестия.
\end{tcolorbox}
\begin{tcolorbox}
\textsubscript{11} Если послушают и будут служить Ему, то проведут дни свои в благополучии и лета свои в радости;
\end{tcolorbox}
\begin{tcolorbox}
\textsubscript{12} если же не послушают, то погибнут от стрелы и умрут в неразумии.
\end{tcolorbox}
\begin{tcolorbox}
\textsubscript{13} Но лицемеры питают в сердце гнев и не взывают к Нему, когда Он заключает их в узы;
\end{tcolorbox}
\begin{tcolorbox}
\textsubscript{14} поэтому душа их умирает в молодости и жизнь их с блудниками.
\end{tcolorbox}
\begin{tcolorbox}
\textsubscript{15} Он спасает бедного от беды его и в угнетении открывает ухо его.
\end{tcolorbox}
\begin{tcolorbox}
\textsubscript{16} И тебя вывел бы Он из тесноты на простор, где нет стеснения, и поставляемое на стол твой было бы наполнено туком;
\end{tcolorbox}
\begin{tcolorbox}
\textsubscript{17} но ты преисполнен суждениями нечестивых: суждение и осуждение--близки.
\end{tcolorbox}
\begin{tcolorbox}
\textsubscript{18} Да не поразит тебя гнев [Божий] наказанием! Большой выкуп не спасет тебя.
\end{tcolorbox}
\begin{tcolorbox}
\textsubscript{19} Даст ли Он какую цену твоему богатству? Нет, --ни золоту и никакому сокровищу.
\end{tcolorbox}
\begin{tcolorbox}
\textsubscript{20} Не желай той ночи, когда народы истребляются на своем месте.
\end{tcolorbox}
\begin{tcolorbox}
\textsubscript{21} Берегись, не склоняйся к нечестию, которое ты предпочел страданию.
\end{tcolorbox}
\begin{tcolorbox}
\textsubscript{22} Бог высок могуществом Своим, и кто такой, как Он, наставник?
\end{tcolorbox}
\begin{tcolorbox}
\textsubscript{23} Кто укажет Ему путь Его; кто может сказать: Ты поступаешь несправедливо?
\end{tcolorbox}
\begin{tcolorbox}
\textsubscript{24} Помни о том, чтобы превозносить дела его, которые люди видят.
\end{tcolorbox}
\begin{tcolorbox}
\textsubscript{25} Все люди могут видеть их; человек может усматривать их издали.
\end{tcolorbox}
\begin{tcolorbox}
\textsubscript{26} Вот, Бог велик, и мы не можем познать Его; число лет Его неисследимо.
\end{tcolorbox}
\begin{tcolorbox}
\textsubscript{27} Он собирает капли воды; они во множестве изливаются дождем:
\end{tcolorbox}
\begin{tcolorbox}
\textsubscript{28} из облаков каплют и изливаются обильно на людей.
\end{tcolorbox}
\begin{tcolorbox}
\textsubscript{29} Кто может также постигнуть протяжение облаков, треск шатра Его?
\end{tcolorbox}
\begin{tcolorbox}
\textsubscript{30} Вот, Он распространяет над ним свет Свой и покрывает дно моря.
\end{tcolorbox}
\begin{tcolorbox}
\textsubscript{31} Оттуда Он судит народы, дает пищу в изобилии.
\end{tcolorbox}
\begin{tcolorbox}
\textsubscript{32} Он сокрывает в дланях Своих молнию и повелевает ей, кого разить.
\end{tcolorbox}
\begin{tcolorbox}
\textsubscript{33} Треск ее дает знать о ней; скот также чувствует происходящее.
\end{tcolorbox}
\subsection{CHAPTER 37}
\begin{tcolorbox}
\textsubscript{1} И от сего трепещет сердце мое и подвиглось с места своего.
\end{tcolorbox}
\begin{tcolorbox}
\textsubscript{2} Слушайте, слушайте голос Его и гром, исходящий из уст Его.
\end{tcolorbox}
\begin{tcolorbox}
\textsubscript{3} Под всем небом раскат его, и блистание его--до краев земли.
\end{tcolorbox}
\begin{tcolorbox}
\textsubscript{4} За ним гремит глас; гремит Он гласом величества Своего и не останавливает его, когда голос Его услышан.
\end{tcolorbox}
\begin{tcolorbox}
\textsubscript{5} Дивно гремит Бог гласом Своим, делает дела великие, для нас непостижимые.
\end{tcolorbox}
\begin{tcolorbox}
\textsubscript{6} Ибо снегу Он говорит: будь на земле; равно мелкий дождь и большой дождь в Его власти.
\end{tcolorbox}
\begin{tcolorbox}
\textsubscript{7} Он полагает печать на руку каждого человека, чтобы все люди знали дело Его.
\end{tcolorbox}
\begin{tcolorbox}
\textsubscript{8} Тогда зверь уходит в убежище и остается в своих логовищах.
\end{tcolorbox}
\begin{tcolorbox}
\textsubscript{9} От юга приходит буря, от севера--стужа.
\end{tcolorbox}
\begin{tcolorbox}
\textsubscript{10} От дуновения Божия происходит лед, и поверхность воды сжимается.
\end{tcolorbox}
\begin{tcolorbox}
\textsubscript{11} Также влагою Он наполняет тучи, и облака сыплют свет Его,
\end{tcolorbox}
\begin{tcolorbox}
\textsubscript{12} и они направляются по намерениям Его, чтоб исполнить то, что Он повелит им на лице обитаемой земли.
\end{tcolorbox}
\begin{tcolorbox}
\textsubscript{13} Он повелевает им идти или для наказания, или в благоволение, или для помилования.
\end{tcolorbox}
\begin{tcolorbox}
\textsubscript{14} Внимай сему, Иов; стой и разумевай чудные дела Божии.
\end{tcolorbox}
\begin{tcolorbox}
\textsubscript{15} Знаешь ли, как Бог располагает ими и повелевает свету блистать из облака Своего?
\end{tcolorbox}
\begin{tcolorbox}
\textsubscript{16} Разумеешь ли равновесие облаков, чудное дело Совершеннейшего в знании?
\end{tcolorbox}
\begin{tcolorbox}
\textsubscript{17} Как нагревается твоя одежда, когда Он успокаивает землю от юга?
\end{tcolorbox}
\begin{tcolorbox}
\textsubscript{18} Ты ли с Ним распростер небеса, твердые, как литое зеркало?
\end{tcolorbox}
\begin{tcolorbox}
\textsubscript{19} Научи нас, что сказать Ему? Мы в этой тьме ничего не можем сообразить.
\end{tcolorbox}
\begin{tcolorbox}
\textsubscript{20} Будет ли возвещено Ему, что я говорю? Сказал ли кто, что сказанное доносится Ему?
\end{tcolorbox}
\begin{tcolorbox}
\textsubscript{21} Теперь не видно яркого света в облаках, но пронесется ветер и расчистит их.
\end{tcolorbox}
\begin{tcolorbox}
\textsubscript{22} Светлая погода приходит от севера, и окрест Бога страшное великолепие.
\end{tcolorbox}
\begin{tcolorbox}
\textsubscript{23} Вседержитель! мы не постигаем Его. Он велик силою, судом и полнотою правосудия. Он [никого] не угнетает.
\end{tcolorbox}
\begin{tcolorbox}
\textsubscript{24} Посему да благоговеют пред Ним люди, и да трепещут пред Ним все мудрые сердцем!
\end{tcolorbox}
\subsection{CHAPTER 38}
\begin{tcolorbox}
\textsubscript{1} Господь отвечал Иову из бури и сказал:
\end{tcolorbox}
\begin{tcolorbox}
\textsubscript{2} кто сей, омрачающий Провидение словами без смысла?
\end{tcolorbox}
\begin{tcolorbox}
\textsubscript{3} Препояшь ныне чресла твои, как муж: Я буду спрашивать тебя, и ты объясняй Мне:
\end{tcolorbox}
\begin{tcolorbox}
\textsubscript{4} где был ты, когда Я полагал основания земли? Скажи, если знаешь.
\end{tcolorbox}
\begin{tcolorbox}
\textsubscript{5} Кто положил меру ей, если знаешь? или кто протягивал по ней вервь?
\end{tcolorbox}
\begin{tcolorbox}
\textsubscript{6} На чем утверждены основания ее, или кто положил краеугольный камень ее,
\end{tcolorbox}
\begin{tcolorbox}
\textsubscript{7} при общем ликовании утренних звезд, когда все сыны Божии восклицали от радости?
\end{tcolorbox}
\begin{tcolorbox}
\textsubscript{8} Кто затворил море воротами, когда оно исторглось, вышло как бы из чрева,
\end{tcolorbox}
\begin{tcolorbox}
\textsubscript{9} когда Я облака сделал одеждою его и мглу пеленами его,
\end{tcolorbox}
\begin{tcolorbox}
\textsubscript{10} и утвердил ему Мое определение, и поставил запоры и ворота,
\end{tcolorbox}
\begin{tcolorbox}
\textsubscript{11} и сказал: доселе дойдешь и не перейдешь, и здесь предел надменным волнам твоим?
\end{tcolorbox}
\begin{tcolorbox}
\textsubscript{12} Давал ли ты когда в жизни своей приказания утру и указывал ли заре место ее,
\end{tcolorbox}
\begin{tcolorbox}
\textsubscript{13} чтобы она охватила края земли и стряхнула с нее нечестивых,
\end{tcolorbox}
\begin{tcolorbox}
\textsubscript{14} чтобы [земля] изменилась, как глина под печатью, и стала, как разноцветная одежда,
\end{tcolorbox}
\begin{tcolorbox}
\textsubscript{15} и чтобы отнялся у нечестивых свет их и дерзкая рука их сокрушилась?
\end{tcolorbox}
\begin{tcolorbox}
\textsubscript{16} Нисходил ли ты во глубину моря и входил ли в исследование бездны?
\end{tcolorbox}
\begin{tcolorbox}
\textsubscript{17} Отворялись ли для тебя врата смерти, и видел ли ты врата тени смертной?
\end{tcolorbox}
\begin{tcolorbox}
\textsubscript{18} Обозрел ли ты широту земли? Объясни, если знаешь все это.
\end{tcolorbox}
\begin{tcolorbox}
\textsubscript{19} Где путь к жилищу света, и где место тьмы?
\end{tcolorbox}
\begin{tcolorbox}
\textsubscript{20} Ты, конечно, доходил до границ ее и знаешь стези к дому ее.
\end{tcolorbox}
\begin{tcolorbox}
\textsubscript{21} Ты знаешь это, потому что ты был уже тогда рожден, и число дней твоих очень велико.
\end{tcolorbox}
\begin{tcolorbox}
\textsubscript{22} Входил ли ты в хранилища снега и видел ли сокровищницы града,
\end{tcolorbox}
\begin{tcolorbox}
\textsubscript{23} которые берегу Я на время смутное, на день битвы и войны?
\end{tcolorbox}
\begin{tcolorbox}
\textsubscript{24} По какому пути разливается свет и разносится восточный ветер по земле?
\end{tcolorbox}
\begin{tcolorbox}
\textsubscript{25} Кто проводит протоки для излияния воды и путь для громоносной молнии,
\end{tcolorbox}
\begin{tcolorbox}
\textsubscript{26} чтобы шел дождь на землю безлюдную, на пустыню, где нет человека,
\end{tcolorbox}
\begin{tcolorbox}
\textsubscript{27} чтобы насыщать пустыню и степь и возбуждать травные зародыши к возрастанию?
\end{tcolorbox}
\begin{tcolorbox}
\textsubscript{28} Есть ли у дождя отец? или кто рождает капли росы?
\end{tcolorbox}
\begin{tcolorbox}
\textsubscript{29} Из чьего чрева выходит лед, и иней небесный, --кто рождает его?
\end{tcolorbox}
\begin{tcolorbox}
\textsubscript{30} Воды, как камень, крепнут, и поверхность бездны замерзает.
\end{tcolorbox}
\begin{tcolorbox}
\textsubscript{31} Можешь ли ты связать узел Хима и разрешить узы Кесиль?
\end{tcolorbox}
\begin{tcolorbox}
\textsubscript{32} Можешь ли выводить созвездия в свое время и вести Ас с ее детьми?
\end{tcolorbox}
\begin{tcolorbox}
\textsubscript{33} Знаешь ли ты уставы неба, можешь ли установить господство его на земле?
\end{tcolorbox}
\begin{tcolorbox}
\textsubscript{34} Можешь ли возвысить голос твой к облакам, чтобы вода в обилии покрыла тебя?
\end{tcolorbox}
\begin{tcolorbox}
\textsubscript{35} Можешь ли посылать молнии, и пойдут ли они и скажут ли тебе: вот мы?
\end{tcolorbox}
\begin{tcolorbox}
\textsubscript{36} Кто вложил мудрость в сердце, или кто дал смысл разуму?
\end{tcolorbox}
\begin{tcolorbox}
\textsubscript{37} Кто может расчислить облака своею мудростью и удержать сосуды неба,
\end{tcolorbox}
\begin{tcolorbox}
\textsubscript{38} когда пыль обращается в грязь и глыбы слипаются?
\end{tcolorbox}
\begin{tcolorbox}
\textsubscript{39} Ты ли ловишь добычу львице и насыщаешь молодых львов,
\end{tcolorbox}
\begin{tcolorbox}
\textsubscript{40} когда они лежат в берлогах или покоятся под тенью в засаде?
\end{tcolorbox}
\begin{tcolorbox}
\textsubscript{41} Кто приготовляет ворону корм его, когда птенцы его кричат к Богу, бродя без пищи?
\end{tcolorbox}
\subsection{CHAPTER 39}
\begin{tcolorbox}
\textsubscript{1} Знаешь ли ты время, когда рождаются дикие козы на скалах, и замечал ли роды ланей?
\end{tcolorbox}
\begin{tcolorbox}
\textsubscript{2} можешь ли расчислить месяцы беременности их? и знаешь ли время родов их?
\end{tcolorbox}
\begin{tcolorbox}
\textsubscript{3} Они изгибаются, рождая детей своих, выбрасывая свои ноши;
\end{tcolorbox}
\begin{tcolorbox}
\textsubscript{4} дети их приходят в силу, растут на поле, уходят и не возвращаются к ним.
\end{tcolorbox}
\begin{tcolorbox}
\textsubscript{5} Кто пустил дикого осла на свободу, и кто разрешил узы онагру,
\end{tcolorbox}
\begin{tcolorbox}
\textsubscript{6} которому степь Я назначил домом и солончаки--жилищем?
\end{tcolorbox}
\begin{tcolorbox}
\textsubscript{7} Он посмевается городскому многолюдству и не слышит криков погонщика,
\end{tcolorbox}
\begin{tcolorbox}
\textsubscript{8} по горам ищет себе пищи и гоняется за всякою зеленью.
\end{tcolorbox}
\begin{tcolorbox}
\textsubscript{9} Захочет ли единорог служить тебе и переночует ли у яслей твоих?
\end{tcolorbox}
\begin{tcolorbox}
\textsubscript{10} Можешь ли веревкою привязать единорога к борозде, и станет ли он боронить за тобою поле?
\end{tcolorbox}
\begin{tcolorbox}
\textsubscript{11} Понадеешься ли на него, потому что у него сила велика, и предоставишь ли ему работу твою?
\end{tcolorbox}
\begin{tcolorbox}
\textsubscript{12} Поверишь ли ему, что он семена твои возвратит и сложит на гумно твое?
\end{tcolorbox}
\begin{tcolorbox}
\textsubscript{13} Ты ли дал красивые крылья павлину и перья и пух страусу?
\end{tcolorbox}
\begin{tcolorbox}
\textsubscript{14} Он оставляет яйца свои на земле, и на песке согревает их,
\end{tcolorbox}
\begin{tcolorbox}
\textsubscript{15} и забывает, что нога может раздавить их и полевой зверь может растоптать их;
\end{tcolorbox}
\begin{tcolorbox}
\textsubscript{16} он жесток к детям своим, как бы не своим, и не опасается, что труд его будет напрасен;
\end{tcolorbox}
\begin{tcolorbox}
\textsubscript{17} потому что Бог не дал ему мудрости и не уделил ему смысла;
\end{tcolorbox}
\begin{tcolorbox}
\textsubscript{18} а когда поднимется на высоту, посмевается коню и всаднику его.
\end{tcolorbox}
\begin{tcolorbox}
\textsubscript{19} Ты ли дал коню силу и облек шею его гривою?
\end{tcolorbox}
\begin{tcolorbox}
\textsubscript{20} Можешь ли ты испугать его, как саранчу? Храпение ноздрей его--ужас;
\end{tcolorbox}
\begin{tcolorbox}
\textsubscript{21} роет ногою землю и восхищается силою; идет навстречу оружию;
\end{tcolorbox}
\begin{tcolorbox}
\textsubscript{22} он смеется над опасностью и не робеет и не отворачивается от меча;
\end{tcolorbox}
\begin{tcolorbox}
\textsubscript{23} колчан звучит над ним, сверкает копье и дротик;
\end{tcolorbox}
\begin{tcolorbox}
\textsubscript{24} в порыве и ярости он глотает землю и не может стоять при звуке трубы;
\end{tcolorbox}
\begin{tcolorbox}
\textsubscript{25} при трубном звуке он издает голос: гу! гу! и издалека чует битву, громкие голоса вождей и крик.
\end{tcolorbox}
\begin{tcolorbox}
\textsubscript{26} Твоею ли мудростью летает ястреб и направляет крылья свои на полдень?
\end{tcolorbox}
\begin{tcolorbox}
\textsubscript{27} По твоему ли слову возносится орел и устрояет на высоте гнездо свое?
\end{tcolorbox}
\begin{tcolorbox}
\textsubscript{28} Он живет на скале и ночует на зубце утесов и на местах неприступных;
\end{tcolorbox}
\begin{tcolorbox}
\textsubscript{29} оттуда высматривает себе пищу: глаза его смотрят далеко;
\end{tcolorbox}
\begin{tcolorbox}
\textsubscript{30} птенцы его пьют кровь, и где труп, там и он.
\end{tcolorbox}
\subsection{CHAPTER 40}
\begin{tcolorbox}
\textsubscript{1} (39-31) И продолжал Господь и сказал Иову:
\end{tcolorbox}
\begin{tcolorbox}
\textsubscript{2} (39-32) будет ли состязающийся со Вседержителем еще учить? Обличающий Бога пусть отвечает Ему.
\end{tcolorbox}
\begin{tcolorbox}
\textsubscript{3} (39-33) И отвечал Иов Господу и сказал:
\end{tcolorbox}
\begin{tcolorbox}
\textsubscript{4} (39-34) вот, я ничтожен; что буду я отвечать Тебе? Руку мою полагаю на уста мои.
\end{tcolorbox}
\begin{tcolorbox}
\textsubscript{5} (39-35) Однажды я говорил, --теперь отвечать не буду, даже дважды, но более не буду.
\end{tcolorbox}
\begin{tcolorbox}
\textsubscript{6} (40-1) И отвечал Господь Иову из бури и сказал:
\end{tcolorbox}
\begin{tcolorbox}
\textsubscript{7} (40-2) препояшь, как муж, чресла твои: Я буду спрашивать тебя, а ты объясняй Мне.
\end{tcolorbox}
\begin{tcolorbox}
\textsubscript{8} (40-3) Ты хочешь ниспровергнуть суд Мой, обвинить Меня, чтобы оправдать себя?
\end{tcolorbox}
\begin{tcolorbox}
\textsubscript{9} (40-4) Такая ли у тебя мышца, как у Бога? И можешь ли возгреметь голосом, как Он?
\end{tcolorbox}
\begin{tcolorbox}
\textsubscript{10} (40-5) Укрась же себя величием и славою, облекись в блеск и великолепие;
\end{tcolorbox}
\begin{tcolorbox}
\textsubscript{11} (40-6) излей ярость гнева твоего, посмотри на все гордое и смири его;
\end{tcolorbox}
\begin{tcolorbox}
\textsubscript{12} (40-7) взгляни на всех высокомерных и унизь их, и сокруши нечестивых на местах их;
\end{tcolorbox}
\begin{tcolorbox}
\textsubscript{13} (40-8) зарой всех их в землю и лица их покрой тьмою.
\end{tcolorbox}
\begin{tcolorbox}
\textsubscript{14} (40-9) Тогда и Я признаю, что десница твоя может спасать тебя.
\end{tcolorbox}
\begin{tcolorbox}
\textsubscript{15} (40-10) Вот бегемот, которого Я создал, как и тебя; он ест траву, как вол;
\end{tcolorbox}
\begin{tcolorbox}
\textsubscript{16} (40-11) вот, его сила в чреслах его и крепость его в мускулах чрева его;
\end{tcolorbox}
\begin{tcolorbox}
\textsubscript{17} (40-12) поворачивает хвостом своим, как кедром; жилы же на бедрах его переплетены;
\end{tcolorbox}
\begin{tcolorbox}
\textsubscript{18} (40-13) ноги у него, как медные трубы; кости у него, как железные прутья;
\end{tcolorbox}
\begin{tcolorbox}
\textsubscript{19} (40-14) это--верх путей Божиих; только Сотворивший его может приблизить к нему меч Свой;
\end{tcolorbox}
\begin{tcolorbox}
\textsubscript{20} (40-15) горы приносят ему пищу, и там все звери полевые играют;
\end{tcolorbox}
\begin{tcolorbox}
\textsubscript{21} (40-16) он ложится под тенистыми деревьями, под кровом тростника и в болотах;
\end{tcolorbox}
\begin{tcolorbox}
\textsubscript{22} (40-17) тенистые дерева покрывают его своею тенью; ивы при ручьях окружают его;
\end{tcolorbox}
\begin{tcolorbox}
\textsubscript{23} (40-18) вот, он пьет из реки и не торопится; остается спокоен, хотя бы Иордан устремился ко рту его.
\end{tcolorbox}
\begin{tcolorbox}
\textsubscript{24} (40-19) Возьмет ли кто его в глазах его и проколет ли ему нос багром?
\end{tcolorbox}
\subsection{CHAPTER 41}
\begin{tcolorbox}
\textsubscript{1} (40-20) Можешь ли ты удою вытащить левиафана и веревкою схватить за язык его?
\end{tcolorbox}
\begin{tcolorbox}
\textsubscript{2} (40-21) вденешь ли кольцо в ноздри его? проколешь ли иглою челюсть его?
\end{tcolorbox}
\begin{tcolorbox}
\textsubscript{3} (40-22) будет ли он много умолять тебя и будет ли говорить с тобою кротко?
\end{tcolorbox}
\begin{tcolorbox}
\textsubscript{4} (40-23) сделает ли он договор с тобою, и возьмешь ли его навсегда себе в рабы?
\end{tcolorbox}
\begin{tcolorbox}
\textsubscript{5} (40-24) станешь ли забавляться им, как птичкою, и свяжешь ли его для девочек твоих?
\end{tcolorbox}
\begin{tcolorbox}
\textsubscript{6} (40-25) будут ли продавать его товарищи ловли, разделят ли его между Хананейскими купцами?
\end{tcolorbox}
\begin{tcolorbox}
\textsubscript{7} (40-26) можешь ли пронзить кожу его копьем и голову его рыбачьею острогою?
\end{tcolorbox}
\begin{tcolorbox}
\textsubscript{8} (40-27) Клади на него руку твою, и помни о борьбе: вперед не будешь.
\end{tcolorbox}
\begin{tcolorbox}
\textsubscript{9} (41-1) Надежда тщетна: не упадешь ли от одного взгляда его?
\end{tcolorbox}
\begin{tcolorbox}
\textsubscript{10} (41-2) Нет столь отважного, который осмелился бы потревожить его; кто же может устоять перед Моим лицем?
\end{tcolorbox}
\begin{tcolorbox}
\textsubscript{11} (41-3) Кто предварил Меня, чтобы Мне воздавать ему? под всем небом все Мое.
\end{tcolorbox}
\begin{tcolorbox}
\textsubscript{12} (41-4) Не умолчу о членах его, о силе и красивой соразмерности их.
\end{tcolorbox}
\begin{tcolorbox}
\textsubscript{13} (41-5) Кто может открыть верх одежды его, кто подойдет к двойным челюстям его?
\end{tcolorbox}
\begin{tcolorbox}
\textsubscript{14} (41-6) Кто может отворить двери лица его? круг зубов его--ужас;
\end{tcolorbox}
\begin{tcolorbox}
\textsubscript{15} (41-7) крепкие щиты его--великолепие; они скреплены как бы твердою печатью;
\end{tcolorbox}
\begin{tcolorbox}
\textsubscript{16} (41-8) один к другому прикасается близко, так что и воздух не проходит между ними;
\end{tcolorbox}
\begin{tcolorbox}
\textsubscript{17} (41-9) один с другим лежат плотно, сцепились и не раздвигаются.
\end{tcolorbox}
\begin{tcolorbox}
\textsubscript{18} (41-10) От его чихания показывается свет; глаза у него как ресницы зари;
\end{tcolorbox}
\begin{tcolorbox}
\textsubscript{19} (41-11) из пасти его выходят пламенники, выскакивают огненные искры;
\end{tcolorbox}
\begin{tcolorbox}
\textsubscript{20} (41-12) из ноздрей его выходит дым, как из кипящего горшка или котла.
\end{tcolorbox}
\begin{tcolorbox}
\textsubscript{21} (41-13) Дыхание его раскаляет угли, и из пасти его выходит пламя.
\end{tcolorbox}
\begin{tcolorbox}
\textsubscript{22} (41-14) На шее его обитает сила, и перед ним бежит ужас.
\end{tcolorbox}
\begin{tcolorbox}
\textsubscript{23} (41-15) Мясистые части тела его сплочены между собою твердо, не дрогнут.
\end{tcolorbox}
\begin{tcolorbox}
\textsubscript{24} (41-16) Сердце его твердо, как камень, и жестко, как нижний жернов.
\end{tcolorbox}
\begin{tcolorbox}
\textsubscript{25} (41-17) Когда он поднимается, силачи в страхе, совсем теряются от ужаса.
\end{tcolorbox}
\begin{tcolorbox}
\textsubscript{26} (41-18) Меч, коснувшийся его, не устоит, ни копье, ни дротик, ни латы.
\end{tcolorbox}
\begin{tcolorbox}
\textsubscript{27} (41-19) Железо он считает за солому, медь--за гнилое дерево.
\end{tcolorbox}
\begin{tcolorbox}
\textsubscript{28} (41-20) Дочь лука не обратит его в бегство; пращные камни обращаются для него в плеву.
\end{tcolorbox}
\begin{tcolorbox}
\textsubscript{29} (41-21) Булава считается у него за соломину; свисту дротика он смеется.
\end{tcolorbox}
\begin{tcolorbox}
\textsubscript{30} (41-22) Под ним острые камни, и он на острых камнях лежит в грязи.
\end{tcolorbox}
\begin{tcolorbox}
\textsubscript{31} (41-23) Он кипятит пучину, как котел, и море претворяет в кипящую мазь;
\end{tcolorbox}
\begin{tcolorbox}
\textsubscript{32} (41-24) оставляет за собою светящуюся стезю; бездна кажется сединою.
\end{tcolorbox}
\begin{tcolorbox}
\textsubscript{33} (41-25) Нет на земле подобного ему; он сотворен бесстрашным;
\end{tcolorbox}
\begin{tcolorbox}
\textsubscript{34} (41-26) на все высокое смотрит смело; он царь над всеми сынами гордости.
\end{tcolorbox}
\subsection{CHAPTER 42}
\begin{tcolorbox}
\textsubscript{1} И отвечал Иов Господу и сказал:
\end{tcolorbox}
\begin{tcolorbox}
\textsubscript{2} знаю, что Ты все можешь, и что намерение Твое не может быть остановлено.
\end{tcolorbox}
\begin{tcolorbox}
\textsubscript{3} Кто сей, омрачающий Провидение, ничего не разумея? --Так, я говорил о том, чего не разумел, о делах чудных для меня, которых я не знал.
\end{tcolorbox}
\begin{tcolorbox}
\textsubscript{4} Выслушай, [взывал я,] и я буду говорить, и что буду спрашивать у Тебя, объясни мне.
\end{tcolorbox}
\begin{tcolorbox}
\textsubscript{5} Я слышал о Тебе слухом уха; теперь же мои глаза видят Тебя;
\end{tcolorbox}
\begin{tcolorbox}
\textsubscript{6} поэтому я отрекаюсь и раскаиваюсь в прахе и пепле.
\end{tcolorbox}
\begin{tcolorbox}
\textsubscript{7} И было после того, как Господь сказал слова те Иову, сказал Господь Елифазу Феманитянину: горит гнев Мой на тебя и на двух друзей твоих за то, что вы говорили о Мне не так верно, как раб Мой Иов.
\end{tcolorbox}
\begin{tcolorbox}
\textsubscript{8} Итак возьмите себе семь тельцов и семь овнов и пойдите к рабу Моему Иову и принесите за себя жертву; и раб Мой Иов помолится за вас, ибо только лице его Я приму, дабы не отвергнуть вас за то, что вы говорили о Мне не так верно, как раб Мой Иов.
\end{tcolorbox}
\begin{tcolorbox}
\textsubscript{9} И пошли Елифаз Феманитянин и Вилдад Савхеянин и Софар Наамитянин, и сделали так, как Господь повелел им, --и Господь принял лице Иова.
\end{tcolorbox}
\begin{tcolorbox}
\textsubscript{10} И возвратил Господь потерю Иова, когда он помолился за друзей своих; и дал Господь Иову вдвое больше того, что он имел прежде.
\end{tcolorbox}
\begin{tcolorbox}
\textsubscript{11} Тогда пришли к нему все братья его и все сестры его и все прежние знакомые его, и ели с ним хлеб в доме его, и тужили с ним, и утешали его за все зло, которое Господь навел на него, и дали ему каждый по кесите и по золотому кольцу.
\end{tcolorbox}
\begin{tcolorbox}
\textsubscript{12} И благословил Бог последние дни Иова более, нежели прежние: у него было четырнадцать тысяч мелкого скота, шесть тысяч верблюдов, тысяча пар волов и тысяча ослиц.
\end{tcolorbox}
\begin{tcolorbox}
\textsubscript{13} И было у него семь сыновей и три дочери.
\end{tcolorbox}
\begin{tcolorbox}
\textsubscript{14} И нарек он имя первой Емима, имя второй--Кассия, а имя третьей--Керенгаппух.
\end{tcolorbox}
\begin{tcolorbox}
\textsubscript{15} И не было на всей земле таких прекрасных женщин, как дочери Иова, и дал им отец их наследство между братьями их.
\end{tcolorbox}
\begin{tcolorbox}
\textsubscript{16} После того Иов жил сто сорок лет, и видел сыновей своих и сыновей сыновних до четвертого рода;
\end{tcolorbox}
\begin{tcolorbox}
\textsubscript{17} и умер Иов в старости, насыщенный днями.
\end{tcolorbox}
