\section{BOOK 165}
\subsection{CHAPTER 1}
\begin{tcolorbox}
\textsubscript{1} Павел, раб Божий, Апостол же Иисуса Христа, по вере избранных Божиих и познанию истины, [относящейся] к благочестию,
\end{tcolorbox}
\begin{tcolorbox}
\textsubscript{2} в надежде вечной жизни, которую обещал неизменный в слове Бог прежде вековых времен,
\end{tcolorbox}
\begin{tcolorbox}
\textsubscript{3} а в свое время явил Свое слово в проповеди, вверенной мне по повелению Спасителя нашего, Бога, --
\end{tcolorbox}
\begin{tcolorbox}
\textsubscript{4} Титу, истинному сыну по общей вере: благодать, милость и мир от Бога Отца и Господа Иисуса Христа, Спасителя нашего.
\end{tcolorbox}
\begin{tcolorbox}
\textsubscript{5} Для того я оставил тебя в Крите, чтобы ты довершил недоконченное и поставил по всем городам пресвитеров, как я тебе приказывал:
\end{tcolorbox}
\begin{tcolorbox}
\textsubscript{6} если кто непорочен, муж одной жены, детей имеет верных, не укоряемых в распутстве или непокорности.
\end{tcolorbox}
\begin{tcolorbox}
\textsubscript{7} Ибо епископ должен быть непорочен, как Божий домостроитель, не дерзок, не гневлив, не пьяница, не бийца, не корыстолюбец,
\end{tcolorbox}
\begin{tcolorbox}
\textsubscript{8} но страннолюбив, любящий добро, целомудрен, справедлив, благочестив, воздержан,
\end{tcolorbox}
\begin{tcolorbox}
\textsubscript{9} держащийся истинного слова, согласного с учением, чтобы он был силен и наставлять в здравом учении и противящихся обличать.
\end{tcolorbox}
\begin{tcolorbox}
\textsubscript{10} Ибо есть много и непокорных, пустословов и обманщиков, особенно из обрезанных,
\end{tcolorbox}
\begin{tcolorbox}
\textsubscript{11} каковым должно заграждать уста: они развращают целые домы, уча, чему не должно, из постыдной корысти.
\end{tcolorbox}
\begin{tcolorbox}
\textsubscript{12} Из них же самих один стихотворец сказал: 'Критяне всегда лжецы, злые звери, утробы ленивые'.
\end{tcolorbox}
\begin{tcolorbox}
\textsubscript{13} Свидетельство это справедливо. По сей причине обличай их строго, дабы они были здравы в вере,
\end{tcolorbox}
\begin{tcolorbox}
\textsubscript{14} не внимая Иудейским басням и постановлениям людей, отвращающихся от истины.
\end{tcolorbox}
\begin{tcolorbox}
\textsubscript{15} Для чистых все чисто; а для оскверненных и неверных нет ничего чистого, но осквернены и ум их и совесть.
\end{tcolorbox}
\begin{tcolorbox}
\textsubscript{16} Они говорят, что знают Бога, а делами отрекаются, будучи гнусны и непокорны и не способны ни к какому доброму делу.
\end{tcolorbox}
\subsection{CHAPTER 2}
\begin{tcolorbox}
\textsubscript{1} Ты же говори то, что сообразно с здравым учением:
\end{tcolorbox}
\begin{tcolorbox}
\textsubscript{2} чтобы старцы были бдительны, степенны, целомудренны, здравы в вере, в любви, в терпении;
\end{tcolorbox}
\begin{tcolorbox}
\textsubscript{3} чтобы старицы также одевались прилично святым, не были клеветницы, не порабощались пьянству, учили добру;
\end{tcolorbox}
\begin{tcolorbox}
\textsubscript{4} чтобы вразумляли молодых любить мужей, любить детей,
\end{tcolorbox}
\begin{tcolorbox}
\textsubscript{5} быть целомудренными, чистыми, попечительными о доме, добрыми, покорными своим мужьям, да не порицается слово Божие.
\end{tcolorbox}
\begin{tcolorbox}
\textsubscript{6} Юношей также увещевай быть целомудренными.
\end{tcolorbox}
\begin{tcolorbox}
\textsubscript{7} Во всем показывай в себе образец добрых дел, в учительстве чистоту, степенность, неповрежденность,
\end{tcolorbox}
\begin{tcolorbox}
\textsubscript{8} слово здравое, неукоризненное, чтобы противник был посрамлен, не имея ничего сказать о нас худого.
\end{tcolorbox}
\begin{tcolorbox}
\textsubscript{9} Рабов [увещевай] повиноваться своим господам, угождать им во всем, не прекословить,
\end{tcolorbox}
\begin{tcolorbox}
\textsubscript{10} не красть, но оказывать всю добрую верность, дабы они во всем были украшением учению Спасителя нашего, Бога.
\end{tcolorbox}
\begin{tcolorbox}
\textsubscript{11} Ибо явилась благодать Божия, спасительная для всех человеков,
\end{tcolorbox}
\begin{tcolorbox}
\textsubscript{12} научающая нас, чтобы мы, отвергнув нечестие и мирские похоти, целомудренно, праведно и благочестиво жили в нынешнем веке,
\end{tcolorbox}
\begin{tcolorbox}
\textsubscript{13} ожидая блаженного упования и явления славы великого Бога и Спасителя нашего Иисуса Христа,
\end{tcolorbox}
\begin{tcolorbox}
\textsubscript{14} Который дал Себя за нас, чтобы избавить нас от всякого беззакония и очистить Себе народ особенный, ревностный к добрым делам.
\end{tcolorbox}
\begin{tcolorbox}
\textsubscript{15} Сие говори, увещевай и обличай со всякою властью, чтобы никто не пренебрегал тебя.
\end{tcolorbox}
\subsection{CHAPTER 3}
\begin{tcolorbox}
\textsubscript{1} Напоминай им повиноваться и покоряться начальству и властям, быть готовыми на всякое доброе дело,
\end{tcolorbox}
\begin{tcolorbox}
\textsubscript{2} никого не злословить, быть не сварливыми, но тихими, и оказывать всякую кротость ко всем человекам.
\end{tcolorbox}
\begin{tcolorbox}
\textsubscript{3} Ибо и мы были некогда несмысленны, непокорны, заблуждшие, были рабы похотей и различных удовольствий, жили в злобе и зависти, были гнусны, ненавидели друг друга.
\end{tcolorbox}
\begin{tcolorbox}
\textsubscript{4} Когда же явилась благодать и человеколюбие Спасителя нашего, Бога,
\end{tcolorbox}
\begin{tcolorbox}
\textsubscript{5} Он спас нас не по делам праведности, которые бы мы сотворили, а по Своей милости, банею возрождения и обновления Святым Духом,
\end{tcolorbox}
\begin{tcolorbox}
\textsubscript{6} Которого излил на нас обильно через Иисуса Христа, Спасителя нашего,
\end{tcolorbox}
\begin{tcolorbox}
\textsubscript{7} чтобы, оправдавшись Его благодатью, мы по упованию соделались наследниками вечной жизни.
\end{tcolorbox}
\begin{tcolorbox}
\textsubscript{8} Слово это верно; и я желаю, чтобы ты подтверждал о сем, дабы уверовавшие в Бога старались быть прилежными к добрым делам: это хорошо и полезно человекам.
\end{tcolorbox}
\begin{tcolorbox}
\textsubscript{9} Глупых же состязаний и родословий, и споров и распрей о законе удаляйся, ибо они бесполезны и суетны.
\end{tcolorbox}
\begin{tcolorbox}
\textsubscript{10} Еретика, после первого и второго вразумления, отвращайся,
\end{tcolorbox}
\begin{tcolorbox}
\textsubscript{11} зная, что таковой развратился и грешит, будучи самоосужден.
\end{tcolorbox}
\begin{tcolorbox}
\textsubscript{12} Когда пришлю к тебе Артему или Тихика, поспеши придти ко мне в Никополь, ибо я положил там провести зиму.
\end{tcolorbox}
\begin{tcolorbox}
\textsubscript{13} Зину законника и Аполлоса позаботься отправить так, чтобы у них ни в чем не было недостатка.
\end{tcolorbox}
\begin{tcolorbox}
\textsubscript{14} Пусть и наши учатся упражняться в добрых делах, [в] [удовлетворении] необходимым нуждам, дабы не были бесплодны.
\end{tcolorbox}
\begin{tcolorbox}
\textsubscript{15} Приветствуют тебя все находящиеся со мною. Приветствуй любящих нас в вере. Благодать со всеми вами. Аминь.
\end{tcolorbox}
