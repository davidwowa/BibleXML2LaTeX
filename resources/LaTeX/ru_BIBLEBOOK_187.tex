\section{BOOK 186}
\subsection{CHAPTER 1}
\begin{tcolorbox}
\textsubscript{1} Старец--избранной госпоже и детям ее, которых я люблю по истине, и не только я, но и все, познавшие истину,
\end{tcolorbox}
\begin{tcolorbox}
\textsubscript{2} ради истины, которая пребывает в нас и будет с нами вовек.
\end{tcolorbox}
\begin{tcolorbox}
\textsubscript{3} Да будет с вами благодать, милость, мир от Бога Отца и от Господа Иисуса Христа, Сына Отчего, в истине и любви.
\end{tcolorbox}
\begin{tcolorbox}
\textsubscript{4} Я весьма обрадовался, что нашел из детей твоих, ходящих в истине, как мы получили заповедь от Отца.
\end{tcolorbox}
\begin{tcolorbox}
\textsubscript{5} И ныне прошу тебя, госпожа, не как новую заповедь предписывая тебе, но ту, которую имеем от начала, чтобы мы любили друг друга.
\end{tcolorbox}
\begin{tcolorbox}
\textsubscript{6} Любовь же состоит в том, чтобы мы поступали по заповедям Его. Это та заповедь, которую вы слышали от начала, чтобы поступали по ней.
\end{tcolorbox}
\begin{tcolorbox}
\textsubscript{7} Ибо многие обольстители вошли в мир, не исповедующие Иисуса Христа, пришедшего во плоти: такой [человек] есть обольститель и антихрист.
\end{tcolorbox}
\begin{tcolorbox}
\textsubscript{8} Наблюдайте за собою, чтобы нам не потерять того, над чем мы трудились, но чтобы получить полную награду.
\end{tcolorbox}
\begin{tcolorbox}
\textsubscript{9} Всякий, преступающий учение Христово и не пребывающий в нем, не имеет Бога; пребывающий в учении Христовом имеет и Отца и Сына.
\end{tcolorbox}
\begin{tcolorbox}
\textsubscript{10} Кто приходит к вам и не приносит сего учения, того не принимайте в дом и не приветствуйте его.
\end{tcolorbox}
\begin{tcolorbox}
\textsubscript{11} Ибо приветствующий его участвует в злых делах его.
\end{tcolorbox}
\begin{tcolorbox}
\textsubscript{12} Многое имею писать вам, но не хочу на бумаге чернилами, а надеюсь придти к вам и говорить устами к устам, чтобы радость ваша была полна.
\end{tcolorbox}
\begin{tcolorbox}
\textsubscript{13} Приветствуют тебя дети сестры твоей избранной. Аминь.
\end{tcolorbox}
