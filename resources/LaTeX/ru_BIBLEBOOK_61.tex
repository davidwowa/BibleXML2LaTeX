\section{BOOK 60}
\subsection{CHAPTER 1}
\begin{tcolorbox}
\textsubscript{1} Слова Екклесиаста, сына Давидова, царя в Иерусалиме.
\end{tcolorbox}
\begin{tcolorbox}
\textsubscript{2} Суета сует, сказал Екклесиаст, суета сует, --всё суета!
\end{tcolorbox}
\begin{tcolorbox}
\textsubscript{3} Что пользы человеку от всех трудов его, которыми трудится он под солнцем?
\end{tcolorbox}
\begin{tcolorbox}
\textsubscript{4} Род проходит, и род приходит, а земля пребывает во веки.
\end{tcolorbox}
\begin{tcolorbox}
\textsubscript{5} Восходит солнце, и заходит солнце, и спешит к месту своему, где оно восходит.
\end{tcolorbox}
\begin{tcolorbox}
\textsubscript{6} Идет ветер к югу, и переходит к северу, кружится, кружится на ходу своем, и возвращается ветер на круги свои.
\end{tcolorbox}
\begin{tcolorbox}
\textsubscript{7} Все реки текут в море, но море не переполняется: к тому месту, откуда реки текут, они возвращаются, чтобы опять течь.
\end{tcolorbox}
\begin{tcolorbox}
\textsubscript{8} Все вещи--в труде: не может человек пересказать всего; не насытится око зрением, не наполнится ухо слушанием.
\end{tcolorbox}
\begin{tcolorbox}
\textsubscript{9} Что было, то и будет; и что делалось, то и будет делаться, и нет ничего нового под солнцем.
\end{tcolorbox}
\begin{tcolorbox}
\textsubscript{10} Бывает нечто, о чем говорят: 'смотри, вот это новое'; но [это] было уже в веках, бывших прежде нас.
\end{tcolorbox}
\begin{tcolorbox}
\textsubscript{11} Нет памяти о прежнем; да и о том, что будет, не останется памяти у тех, которые будут после.
\end{tcolorbox}
\begin{tcolorbox}
\textsubscript{12} Я, Екклесиаст, был царем над Израилем в Иерусалиме;
\end{tcolorbox}
\begin{tcolorbox}
\textsubscript{13} и предал я сердце мое тому, чтобы исследовать и испытать мудростью все, что делается под небом: это тяжелое занятие дал Бог сынам человеческим, чтобы они упражнялись в нем.
\end{tcolorbox}
\begin{tcolorbox}
\textsubscript{14} Видел я все дела, какие делаются под солнцем, и вот, всё--суета и томление духа!
\end{tcolorbox}
\begin{tcolorbox}
\textsubscript{15} Кривое не может сделаться прямым, и чего нет, того нельзя считать.
\end{tcolorbox}
\begin{tcolorbox}
\textsubscript{16} Говорил я с сердцем моим так: вот, я возвеличился и приобрел мудрости больше всех, которые были прежде меня над Иерусалимом, и сердце мое видело много мудрости и знания.
\end{tcolorbox}
\begin{tcolorbox}
\textsubscript{17} И предал я сердце мое тому, чтобы познать мудрость и познать безумие и глупость: узнал, что и это--томление духа;
\end{tcolorbox}
\begin{tcolorbox}
\textsubscript{18} потому что во многой мудрости много печали; и кто умножает познания, умножает скорбь.
\end{tcolorbox}
\subsection{CHAPTER 2}
\begin{tcolorbox}
\textsubscript{1} Сказал я в сердце моем: 'дай, испытаю я тебя весельем, и насладись добром'; но и это--суета!
\end{tcolorbox}
\begin{tcolorbox}
\textsubscript{2} О смехе сказал я: 'глупость!', а о веселье: 'что оно делает?'
\end{tcolorbox}
\begin{tcolorbox}
\textsubscript{3} Вздумал я в сердце моем услаждать вином тело мое и, между тем, как сердце мое руководилось мудростью, придержаться и глупости, доколе не увижу, что хорошо для сынов человеческих, что должны были бы они делать под небом в немногие дни жизни своей.
\end{tcolorbox}
\begin{tcolorbox}
\textsubscript{4} Я предпринял большие дела: построил себе домы, посадил себе виноградники,
\end{tcolorbox}
\begin{tcolorbox}
\textsubscript{5} устроил себе сады и рощи и насадил в них всякие плодовитые дерева;
\end{tcolorbox}
\begin{tcolorbox}
\textsubscript{6} сделал себе водоемы для орошения из них рощей, произращающих деревья;
\end{tcolorbox}
\begin{tcolorbox}
\textsubscript{7} приобрел себе слуг и служанок, и домочадцы были у меня; также крупного и мелкого скота было у меня больше, нежели у всех, бывших прежде меня в Иерусалиме;
\end{tcolorbox}
\begin{tcolorbox}
\textsubscript{8} собрал себе серебра и золота и драгоценностей от царей и областей; завел у себя певцов и певиц и услаждения сынов человеческих--разные музыкальные орудия.
\end{tcolorbox}
\begin{tcolorbox}
\textsubscript{9} И сделался я великим и богатым больше всех, бывших прежде меня в Иерусалиме; и мудрость моя пребыла со мною.
\end{tcolorbox}
\begin{tcolorbox}
\textsubscript{10} Чего бы глаза мои ни пожелали, я не отказывал им, не возбранял сердцу моему никакого веселья, потому что сердце мое радовалось во всех трудах моих, и это было моею долею от всех трудов моих.
\end{tcolorbox}
\begin{tcolorbox}
\textsubscript{11} И оглянулся я на все дела мои, которые сделали руки мои, и на труд, которым трудился я, делая [их]: и вот, всё--суета и томление духа, и нет [от них] пользы под солнцем!
\end{tcolorbox}
\begin{tcolorbox}
\textsubscript{12} И обратился я, чтобы взглянуть на мудрость и безумие и глупость: ибо что [может сделать] человек после царя [сверх того], что уже сделано?
\end{tcolorbox}
\begin{tcolorbox}
\textsubscript{13} И увидел я, что преимущество мудрости перед глупостью такое же, как преимущество света перед тьмою:
\end{tcolorbox}
\begin{tcolorbox}
\textsubscript{14} у мудрого глаза его--в голове его, а глупый ходит во тьме; но узнал я, что одна участь постигает их всех.
\end{tcolorbox}
\begin{tcolorbox}
\textsubscript{15} И сказал я в сердце моем: 'и меня постигнет та же участь, как и глупого: к чему же я сделался очень мудрым?' И сказал я в сердце моем, что и это--суета;
\end{tcolorbox}
\begin{tcolorbox}
\textsubscript{16} потому что мудрого не будут помнить вечно, как и глупого; в грядущие дни все будет забыто, и увы! мудрый умирает наравне с глупым.
\end{tcolorbox}
\begin{tcolorbox}
\textsubscript{17} И возненавидел я жизнь, потому что противны стали мне дела, которые делаются под солнцем; ибо всё--суета и томление духа!
\end{tcolorbox}
\begin{tcolorbox}
\textsubscript{18} И возненавидел я весь труд мой, которым трудился под солнцем, потому что должен оставить его человеку, который будет после меня.
\end{tcolorbox}
\begin{tcolorbox}
\textsubscript{19} И кто знает: мудрый ли будет он, или глупый? А он будет распоряжаться всем трудом моим, которым я трудился и которым показал себя мудрым под солнцем. И это--суета!
\end{tcolorbox}
\begin{tcolorbox}
\textsubscript{20} И обратился я, чтобы внушить сердцу моему отречься от всего труда, которым я трудился под солнцем,
\end{tcolorbox}
\begin{tcolorbox}
\textsubscript{21} потому что иной человек трудится мудро, с знанием и успехом, и должен отдать всё человеку, не трудившемуся в том, как бы часть его. И это--суета и зло великое!
\end{tcolorbox}
\begin{tcolorbox}
\textsubscript{22} Ибо что будет иметь человек от всего труда своего и заботы сердца своего, что трудится он под солнцем?
\end{tcolorbox}
\begin{tcolorbox}
\textsubscript{23} Потому что все дни его--скорби, и его труды--беспокойство; даже и ночью сердце его не знает покоя. И это--суета!
\end{tcolorbox}
\begin{tcolorbox}
\textsubscript{24} Не во власти человека и то благо, чтобы есть и пить и услаждать душу свою от труда своего. Я увидел, что и это--от руки Божией;
\end{tcolorbox}
\begin{tcolorbox}
\textsubscript{25} потому что кто может есть и кто может наслаждаться без Него?
\end{tcolorbox}
\begin{tcolorbox}
\textsubscript{26} Ибо человеку, который добр пред лицем Его, Он дает мудрость и знание и радость; а грешнику дает заботу собирать и копить, чтобы [после] отдать доброму пред лицем Божиим. И это--суета и томление духа!
\end{tcolorbox}
\subsection{CHAPTER 3}
\begin{tcolorbox}
\textsubscript{1} Всему свое время, и время всякой вещи под небом:
\end{tcolorbox}
\begin{tcolorbox}
\textsubscript{2} время рождаться, и время умирать; время насаждать, и время вырывать посаженное;
\end{tcolorbox}
\begin{tcolorbox}
\textsubscript{3} время убивать, и время врачевать; время разрушать, и время строить;
\end{tcolorbox}
\begin{tcolorbox}
\textsubscript{4} время плакать, и время смеяться; время сетовать, и время плясать;
\end{tcolorbox}
\begin{tcolorbox}
\textsubscript{5} время разбрасывать камни, и время собирать камни; время обнимать, и время уклоняться от объятий;
\end{tcolorbox}
\begin{tcolorbox}
\textsubscript{6} время искать, и время терять; время сберегать, и время бросать;
\end{tcolorbox}
\begin{tcolorbox}
\textsubscript{7} время раздирать, и время сшивать; время молчать, и время говорить;
\end{tcolorbox}
\begin{tcolorbox}
\textsubscript{8} время любить, и время ненавидеть; время войне, и время миру.
\end{tcolorbox}
\begin{tcolorbox}
\textsubscript{9} Что пользы работающему от того, над чем он трудится?
\end{tcolorbox}
\begin{tcolorbox}
\textsubscript{10} Видел я эту заботу, которую дал Бог сынам человеческим, чтобы они упражнялись в том.
\end{tcolorbox}
\begin{tcolorbox}
\textsubscript{11} Всё соделал Он прекрасным в свое время, и вложил мир в сердце их, хотя человек не может постигнуть дел, которые Бог делает, от начала до конца.
\end{tcolorbox}
\begin{tcolorbox}
\textsubscript{12} Познал я, что нет для них ничего лучшего, как веселиться и делать доброе в жизни своей.
\end{tcolorbox}
\begin{tcolorbox}
\textsubscript{13} И если какой человек ест и пьет, и видит доброе во всяком труде своем, то это--дар Божий.
\end{tcolorbox}
\begin{tcolorbox}
\textsubscript{14} Познал я, что всё, что делает Бог, пребывает вовек: к тому нечего прибавлять и от того нечего убавить, --и Бог делает так, чтобы благоговели пред лицем Его.
\end{tcolorbox}
\begin{tcolorbox}
\textsubscript{15} Что было, то и теперь есть, и что будет, то уже было, --и Бог воззовет прошедшее.
\end{tcolorbox}
\begin{tcolorbox}
\textsubscript{16} Еще видел я под солнцем: место суда, а там беззаконие; место правды, а там неправда.
\end{tcolorbox}
\begin{tcolorbox}
\textsubscript{17} И сказал я в сердце своем: 'праведного и нечестивого будет судить Бог; потому что время для всякой вещи и [суд] над всяким делом там'.
\end{tcolorbox}
\begin{tcolorbox}
\textsubscript{18} Сказал я в сердце своем о сынах человеческих, чтобы испытал их Бог, и чтобы они видели, что они сами по себе животные;
\end{tcolorbox}
\begin{tcolorbox}
\textsubscript{19} потому что участь сынов человеческих и участь животных--участь одна: как те умирают, так умирают и эти, и одно дыхание у всех, и нет у человека преимущества перед скотом, потому что всё--суета!
\end{tcolorbox}
\begin{tcolorbox}
\textsubscript{20} Все идет в одно место: все произошло из праха и все возвратится в прах.
\end{tcolorbox}
\begin{tcolorbox}
\textsubscript{21} Кто знает: дух сынов человеческих восходит ли вверх, и дух животных сходит ли вниз, в землю?
\end{tcolorbox}
\begin{tcolorbox}
\textsubscript{22} Итак увидел я, что нет ничего лучше, как наслаждаться человеку делами своими: потому что это--доля его; ибо кто приведет его посмотреть на то, что будет после него?
\end{tcolorbox}
\subsection{CHAPTER 4}
\begin{tcolorbox}
\textsubscript{1} И обратился я и увидел всякие угнетения, какие делаются под солнцем: и вот слезы угнетенных, а утешителя у них нет; и в руке угнетающих их--сила, а утешителя у них нет.
\end{tcolorbox}
\begin{tcolorbox}
\textsubscript{2} И ублажил я мертвых, которые давно умерли, более живых, которые живут доселе;
\end{tcolorbox}
\begin{tcolorbox}
\textsubscript{3} а блаженнее их обоих тот, кто еще не существовал, кто не видал злых дел, какие делаются под солнцем.
\end{tcolorbox}
\begin{tcolorbox}
\textsubscript{4} Видел я также, что всякий труд и всякий успех в делах производят взаимную между людьми зависть. И это--суета и томление духа!
\end{tcolorbox}
\begin{tcolorbox}
\textsubscript{5} Глупый [сидит], сложив свои руки, и съедает плоть свою.
\end{tcolorbox}
\begin{tcolorbox}
\textsubscript{6} Лучше горсть с покоем, нежели пригоршни с трудом и томлением духа.
\end{tcolorbox}
\begin{tcolorbox}
\textsubscript{7} И обратился я и увидел еще суету под солнцем;
\end{tcolorbox}
\begin{tcolorbox}
\textsubscript{8} [человек] одинокий, и другого нет; ни сына, ни брата нет у него; а всем трудам его нет конца, и глаз его не насыщается богатством. 'Для кого же я тружусь и лишаю душу мою блага?' И это--суета и недоброе дело!
\end{tcolorbox}
\begin{tcolorbox}
\textsubscript{9} Двоим лучше, нежели одному; потому что у них есть доброе вознаграждение в труде их:
\end{tcolorbox}
\begin{tcolorbox}
\textsubscript{10} ибо если упадет один, то другой поднимет товарища своего. Но горе одному, когда упадет, а другого нет, который поднял бы его.
\end{tcolorbox}
\begin{tcolorbox}
\textsubscript{11} Также, если лежат двое, то тепло им; а одному как согреться?
\end{tcolorbox}
\begin{tcolorbox}
\textsubscript{12} И если станет преодолевать кто-либо одного, то двое устоят против него: и нитка, втрое скрученная, нескоро порвется.
\end{tcolorbox}
\begin{tcolorbox}
\textsubscript{13} Лучше бедный, но умный юноша, нежели старый, но неразумный царь, который не умеет принимать советы;
\end{tcolorbox}
\begin{tcolorbox}
\textsubscript{14} ибо тот из темницы выйдет на царство, хотя родился в царстве своем бедным.
\end{tcolorbox}
\begin{tcolorbox}
\textsubscript{15} Видел я всех живущих, которые ходят под солнцем, с этим другим юношею, который займет место того.
\end{tcolorbox}
\begin{tcolorbox}
\textsubscript{16} Не было числа всему народу, который был перед ним, хотя позднейшие не порадуются им. И это--суета и томление духа!
\end{tcolorbox}
\subsection{CHAPTER 5}
\begin{tcolorbox}
\textsubscript{1} (4-17) Наблюдай за ногою твоею, когда идешь в дом Божий, и будь готов более к слушанию, нежели к жертвоприношению; ибо они не думают, что худо делают.
\end{tcolorbox}
\begin{tcolorbox}
\textsubscript{2} (5-1) Не торопись языком твоим, и сердце твое да не спешит произнести слово пред Богом; потому что Бог на небе, а ты на земле; поэтому слова твои да будут немноги.
\end{tcolorbox}
\begin{tcolorbox}
\textsubscript{3} (5-2) Ибо, как сновидения бывают при множестве забот, так голос глупого познается при множестве слов.
\end{tcolorbox}
\begin{tcolorbox}
\textsubscript{4} (5-3) Когда даешь обет Богу, то не медли исполнить его, потому что Он не благоволит к глупым: что обещал, исполни.
\end{tcolorbox}
\begin{tcolorbox}
\textsubscript{5} (5-4) Лучше тебе не обещать, нежели обещать и не исполнить.
\end{tcolorbox}
\begin{tcolorbox}
\textsubscript{6} (5-5) Не дозволяй устам твоим вводить в грех плоть твою, и не говори пред Ангелом [Божиим]: 'это--ошибка!' Для чего тебе [делать], чтобы Бог прогневался на слово твое и разрушил дело рук твоих?
\end{tcolorbox}
\begin{tcolorbox}
\textsubscript{7} (5-6) Ибо во множестве сновидений, как и во множестве слов, --много суеты; но ты бойся Бога.
\end{tcolorbox}
\begin{tcolorbox}
\textsubscript{8} (5-7) Если ты увидишь в какой области притеснение бедному и нарушение суда и правды, то не удивляйся этому: потому что над высоким наблюдает высший, а над ними еще высший;
\end{tcolorbox}
\begin{tcolorbox}
\textsubscript{9} (5-8) превосходство же страны в целом есть царь, заботящийся о стране.
\end{tcolorbox}
\begin{tcolorbox}
\textsubscript{10} (5-9) Кто любит серебро, тот не насытится серебром, и кто любит богатство, тому нет пользы от того. И это--суета!
\end{tcolorbox}
\begin{tcolorbox}
\textsubscript{11} (5-10) Умножается имущество, умножаются и потребляющие его; и какое благо для владеющего им: разве только смотреть своими глазами?
\end{tcolorbox}
\begin{tcolorbox}
\textsubscript{12} (5-11) Сладок сон трудящегося, мало ли, много ли он съест; но пресыщение богатого не дает ему уснуть.
\end{tcolorbox}
\begin{tcolorbox}
\textsubscript{13} (5-12) Есть мучительный недуг, который видел я под солнцем: богатство, сберегаемое владетелем его во вред ему.
\end{tcolorbox}
\begin{tcolorbox}
\textsubscript{14} (5-13) И гибнет богатство это от несчастных случаев: родил он сына, и ничего нет в руках у него.
\end{tcolorbox}
\begin{tcolorbox}
\textsubscript{15} (5-14) Как вышел он нагим из утробы матери своей, таким и отходит, каким пришел, и ничего не возьмет от труда своего, что мог бы он понести в руке своей.
\end{tcolorbox}
\begin{tcolorbox}
\textsubscript{16} (5-15) И это тяжкий недуг: каким пришел он, таким и отходит. Какая же польза ему, что он трудился на ветер?
\end{tcolorbox}
\begin{tcolorbox}
\textsubscript{17} (5-16) А он во все дни свои ел впотьмах, в большом раздражении, в огорчении и досаде.
\end{tcolorbox}
\begin{tcolorbox}
\textsubscript{18} (5-17) Вот еще, что я нашел доброго и приятного: есть и пить и наслаждаться добром во всех трудах своих, какими кто трудится под солнцем во все дни жизни своей, которые дал ему Бог; потому что это его доля.
\end{tcolorbox}
\begin{tcolorbox}
\textsubscript{19} (5-18) И если какому человеку Бог дал богатство и имущество, и дал ему власть пользоваться от них и брать свою долю и наслаждаться от трудов своих, то это дар Божий.
\end{tcolorbox}
\begin{tcolorbox}
\textsubscript{20} (5-19) Недолго будут у него в памяти дни жизни его; потому Бог и вознаграждает его радостью сердца его.
\end{tcolorbox}
\subsection{CHAPTER 6}
\begin{tcolorbox}
\textsubscript{1} Есть зло, которое видел я под солнцем, и оно часто бывает между людьми:
\end{tcolorbox}
\begin{tcolorbox}
\textsubscript{2} Бог дает человеку богатство и имущество и славу, и нет для души его недостатка ни в чем, чего не пожелал бы он; но не дает ему Бог пользоваться этим, а пользуется тем чужой человек: это--суета и тяжкий недуг!
\end{tcolorbox}
\begin{tcolorbox}
\textsubscript{3} Если бы какой человек родил сто [детей], и прожил многие годы, и еще умножились дни жизни его, но душа его не наслаждалась бы добром и не было бы ему и погребения, то я сказал бы: выкидыш счастливее его,
\end{tcolorbox}
\begin{tcolorbox}
\textsubscript{4} потому что он напрасно пришел и отошел во тьму, и его имя покрыто мраком.
\end{tcolorbox}
\begin{tcolorbox}
\textsubscript{5} Он даже не видел и не знал солнца: ему покойнее, нежели тому.
\end{tcolorbox}
\begin{tcolorbox}
\textsubscript{6} А тот, хотя бы прожил две тысячи лет и не наслаждался добром, не все ли пойдет в одно место?
\end{tcolorbox}
\begin{tcolorbox}
\textsubscript{7} Все труды человека--для рта его, а душа его не насыщается.
\end{tcolorbox}
\begin{tcolorbox}
\textsubscript{8} Какое же преимущество мудрого перед глупым, какое--бедняка, умеющего ходить перед живущими?
\end{tcolorbox}
\begin{tcolorbox}
\textsubscript{9} Лучше видеть глазами, нежели бродить душею. И это--также суета и томление духа!
\end{tcolorbox}
\begin{tcolorbox}
\textsubscript{10} Что существует, тому уже наречено имя, и известно, что это--человек, и что он не может препираться с тем, кто сильнее его.
\end{tcolorbox}
\begin{tcolorbox}
\textsubscript{11} Много таких вещей, которые умножают суету: что же для человека лучше?
\end{tcolorbox}
\begin{tcolorbox}
\textsubscript{12} Ибо кто знает, что хорошо для человека в жизни, во все дни суетной жизни его, которые он проводит как тень? И кто скажет человеку, что будет после него под солнцем?
\end{tcolorbox}
\subsection{CHAPTER 7}
\begin{tcolorbox}
\textsubscript{1} Доброе имя лучше дорогой масти, и день смерти--дня рождения.
\end{tcolorbox}
\begin{tcolorbox}
\textsubscript{2} Лучше ходить в дом плача об умершем, нежели ходить в дом пира; ибо таков конец всякого человека, и живой приложит [это] к своему сердцу.
\end{tcolorbox}
\begin{tcolorbox}
\textsubscript{3} Сетование лучше смеха; потому что при печали лица сердце делается лучше.
\end{tcolorbox}
\begin{tcolorbox}
\textsubscript{4} Сердце мудрых--в доме плача, а сердце глупых--в доме веселья.
\end{tcolorbox}
\begin{tcolorbox}
\textsubscript{5} Лучше слушать обличения от мудрого, нежели слушать песни глупых;
\end{tcolorbox}
\begin{tcolorbox}
\textsubscript{6} потому что смех глупых то же, что треск тернового хвороста под котлом. И это--суета!
\end{tcolorbox}
\begin{tcolorbox}
\textsubscript{7} Притесняя других, мудрый делается глупым, и подарки портят сердце.
\end{tcolorbox}
\begin{tcolorbox}
\textsubscript{8} Конец дела лучше начала его; терпеливый лучше высокомерного.
\end{tcolorbox}
\begin{tcolorbox}
\textsubscript{9} Не будь духом твоим поспешен на гнев, потому что гнев гнездится в сердце глупых.
\end{tcolorbox}
\begin{tcolorbox}
\textsubscript{10} Не говори: 'отчего это прежние дни были лучше нынешних?', потому что не от мудрости ты спрашиваешь об этом.
\end{tcolorbox}
\begin{tcolorbox}
\textsubscript{11} Хороша мудрость с наследством, и особенно для видящих солнце:
\end{tcolorbox}
\begin{tcolorbox}
\textsubscript{12} потому что под сенью ее [то же, что] под сенью серебра; но превосходство знания в [том, что] мудрость дает жизнь владеющему ею.
\end{tcolorbox}
\begin{tcolorbox}
\textsubscript{13} Смотри на действование Божие: ибо кто может выпрямить то, что Он сделал кривым?
\end{tcolorbox}
\begin{tcolorbox}
\textsubscript{14} Во дни благополучия пользуйся благом, а во дни несчастья размышляй: то и другое соделал Бог для того, чтобы человек ничего не мог сказать против Него.
\end{tcolorbox}
\begin{tcolorbox}
\textsubscript{15} Всего насмотрелся я в суетные дни мои: праведник гибнет в праведности своей; нечестивый живет долго в нечестии своем.
\end{tcolorbox}
\begin{tcolorbox}
\textsubscript{16} Не будь слишком строг, и не выставляй себя слишком мудрым; зачем тебе губить себя?
\end{tcolorbox}
\begin{tcolorbox}
\textsubscript{17} Не предавайся греху, и не будь безумен: зачем тебе умирать не в свое время?
\end{tcolorbox}
\begin{tcolorbox}
\textsubscript{18} Хорошо, если ты будешь держаться одного и не отнимать руки от другого; потому что кто боится Бога, тот избежит всего того.
\end{tcolorbox}
\begin{tcolorbox}
\textsubscript{19} Мудрость делает мудрого сильнее десяти властителей, которые в городе.
\end{tcolorbox}
\begin{tcolorbox}
\textsubscript{20} Нет человека праведного на земле, который делал бы добро и не грешил бы;
\end{tcolorbox}
\begin{tcolorbox}
\textsubscript{21} поэтому не на всякое слово, которое говорят, обращай внимание, чтобы не услышать тебе раба твоего, когда он злословит тебя;
\end{tcolorbox}
\begin{tcolorbox}
\textsubscript{22} ибо сердце твое знает много случаев, когда и сам ты злословил других.
\end{tcolorbox}
\begin{tcolorbox}
\textsubscript{23} Все это испытал я мудростью; я сказал: 'буду я мудрым'; но мудрость далека от меня.
\end{tcolorbox}
\begin{tcolorbox}
\textsubscript{24} Далеко то, что было, и глубоко--глубоко: кто постигнет его?
\end{tcolorbox}
\begin{tcolorbox}
\textsubscript{25} Обратился я сердцем моим к тому, чтобы узнать, исследовать и изыскать мудрость и разум, и познать нечестие глупости, невежества и безумия, --
\end{tcolorbox}
\begin{tcolorbox}
\textsubscript{26} и нашел я, что горче смерти женщина, потому что она--сеть, и сердце ее--силки, руки ее--оковы; добрый пред Богом спасется от нее, а грешник уловлен будет ею.
\end{tcolorbox}
\begin{tcolorbox}
\textsubscript{27} Вот это нашел я, сказал Екклесиаст, испытывая одно за другим.
\end{tcolorbox}
\begin{tcolorbox}
\textsubscript{28} Чего еще искала душа моя, и я не нашел? --Мужчину одного из тысячи я нашел, а женщину между всеми ими не нашел.
\end{tcolorbox}
\begin{tcolorbox}
\textsubscript{29} Только это я нашел, что Бог сотворил человека правым, а люди пустились во многие помыслы.
\end{tcolorbox}
\subsection{CHAPTER 8}
\begin{tcolorbox}
\textsubscript{1} Кто--как мудрый, и кто понимает значение вещей? Мудрость человека просветляет лице его, и суровость лица его изменяется.
\end{tcolorbox}
\begin{tcolorbox}
\textsubscript{2} [Я говорю]: слово царское храни, и [это] ради клятвы пред Богом.
\end{tcolorbox}
\begin{tcolorbox}
\textsubscript{3} Не спеши уходить от лица его, и не упорствуй в худом деле; потому что он, что захочет, все может сделать.
\end{tcolorbox}
\begin{tcolorbox}
\textsubscript{4} Где слово царя, там власть; и кто скажет ему: 'что ты делаешь?'
\end{tcolorbox}
\begin{tcolorbox}
\textsubscript{5} Соблюдающий заповедь не испытает никакого зла: сердце мудрого знает и время и устав;
\end{tcolorbox}
\begin{tcolorbox}
\textsubscript{6} потому что для всякой вещи есть свое время и устав; а человеку великое зло оттого,
\end{tcolorbox}
\begin{tcolorbox}
\textsubscript{7} что он не знает, что будет; и как это будет--кто скажет ему?
\end{tcolorbox}
\begin{tcolorbox}
\textsubscript{8} Человек не властен над духом, чтобы удержать дух, и нет власти у него над днем смерти, и нет избавления в этой борьбе, и не спасет нечестие нечестивого.
\end{tcolorbox}
\begin{tcolorbox}
\textsubscript{9} Все это я видел, и обращал сердце мое на всякое дело, какое делается под солнцем. Бывает время, когда человек властвует над человеком во вред ему.
\end{tcolorbox}
\begin{tcolorbox}
\textsubscript{10} Видел я тогда, что хоронили нечестивых, и приходили и отходили от святого места, и они забываемы были в городе, где они так поступали. И это--суета!
\end{tcolorbox}
\begin{tcolorbox}
\textsubscript{11} Не скоро совершается суд над худыми делами; от этого и не страшится сердце сынов человеческих делать зло.
\end{tcolorbox}
\begin{tcolorbox}
\textsubscript{12} Хотя грешник сто раз делает зло и коснеет в нем, но я знаю, что благо будет боящимся Бога, которые благоговеют пред лицем Его;
\end{tcolorbox}
\begin{tcolorbox}
\textsubscript{13} а нечестивому не будет добра, и, подобно тени, недолго продержится тот, кто не благоговеет пред Богом.
\end{tcolorbox}
\begin{tcolorbox}
\textsubscript{14} Есть и такая суета на земле: праведников постигает то, чего заслуживали бы дела нечестивых, а с нечестивыми бывает то, чего заслуживали бы дела праведников. И сказал я: и это--суета!
\end{tcolorbox}
\begin{tcolorbox}
\textsubscript{15} И похвалил я веселье; потому что нет лучшего для человека под солнцем, как есть, пить и веселиться: это сопровождает его в трудах во дни жизни его, которые дал ему Бог под солнцем.
\end{tcolorbox}
\begin{tcolorbox}
\textsubscript{16} Когда я обратил сердце мое на то, чтобы постигнуть мудрость и обозреть дела, которые делаются на земле, и среди которых [человек] ни днем, ни ночью не знает сна, --
\end{tcolorbox}
\begin{tcolorbox}
\textsubscript{17} тогда я увидел все дела Божии и [нашел], что человек не может постигнуть дел, которые делаются под солнцем. Сколько бы человек ни трудился в исследовании, он все-таки не постигнет этого; и если бы какой мудрец сказал, что он знает, он не может постигнуть [этого].
\end{tcolorbox}
\subsection{CHAPTER 9}
\begin{tcolorbox}
\textsubscript{1} На все это я обратил сердце мое для исследования, что праведные и мудрые и деяния их--в руке Божией, и что человек ни любви, ни ненависти не знает во всем том, что перед ним.
\end{tcolorbox}
\begin{tcolorbox}
\textsubscript{2} Всему и всем--одно: одна участь праведнику и нечестивому, доброму и [злому], чистому и нечистому, приносящему жертву и не приносящему жертвы; как добродетельному, так и грешнику; как клянущемуся, так и боящемуся клятвы.
\end{tcolorbox}
\begin{tcolorbox}
\textsubscript{3} Это-то и худо во всем, что делается под солнцем, что одна участь всем, и сердце сынов человеческих исполнено зла, и безумие в сердце их, в жизни их; а после того они [отходят] к умершим.
\end{tcolorbox}
\begin{tcolorbox}
\textsubscript{4} Кто находится между живыми, тому есть еще надежда, так как и псу живому лучше, нежели мертвому льву.
\end{tcolorbox}
\begin{tcolorbox}
\textsubscript{5} Живые знают, что умрут, а мертвые ничего не знают, и уже нет им воздаяния, потому что и память о них предана забвению,
\end{tcolorbox}
\begin{tcolorbox}
\textsubscript{6} и любовь их и ненависть их и ревность их уже исчезли, и нет им более части во веки ни в чем, что делается под солнцем.
\end{tcolorbox}
\begin{tcolorbox}
\textsubscript{7} [Итак] иди, ешь с весельем хлеб твой, и пей в радости сердца вино твое, когда Бог благоволит к делам твоим.
\end{tcolorbox}
\begin{tcolorbox}
\textsubscript{8} Да будут во всякое время одежды твои светлы, и да не оскудевает елей на голове твоей.
\end{tcolorbox}
\begin{tcolorbox}
\textsubscript{9} Наслаждайся жизнью с женою, которую любишь, во все дни суетной жизни твоей, и которую дал тебе Бог под солнцем на все суетные дни твои; потому что это--доля твоя в жизни и в трудах твоих, какими ты трудишься под солнцем.
\end{tcolorbox}
\begin{tcolorbox}
\textsubscript{10} Все, что может рука твоя делать, по силам делай; потому что в могиле, куда ты пойдешь, нет ни работы, ни размышления, ни знания, ни мудрости.
\end{tcolorbox}
\begin{tcolorbox}
\textsubscript{11} И обратился я, и видел под солнцем, что не проворным достается успешный бег, не храбрым--победа, не мудрым--хлеб, и не у разумных--богатство, и не искусным--благорасположение, но время и случай для всех их.
\end{tcolorbox}
\begin{tcolorbox}
\textsubscript{12} Ибо человек не знает своего времени. Как рыбы попадаются в пагубную сеть, и как птицы запутываются в силках, так сыны человеческие уловляются в бедственное время, когда оно неожиданно находит на них.
\end{tcolorbox}
\begin{tcolorbox}
\textsubscript{13} Вот еще какую мудрость видел я под солнцем, и она показалась мне важною:
\end{tcolorbox}
\begin{tcolorbox}
\textsubscript{14} город небольшой, и людей в нем немного; к нему подступил великий царь и обложил его и произвел против него большие осадные работы;
\end{tcolorbox}
\begin{tcolorbox}
\textsubscript{15} но в нем нашелся мудрый бедняк, и он спас своею мудростью этот город; и однако же никто не вспоминал об этом бедном человеке.
\end{tcolorbox}
\begin{tcolorbox}
\textsubscript{16} И сказал я: мудрость лучше силы, и однако же мудрость бедняка пренебрегается, и слов его не слушают.
\end{tcolorbox}
\begin{tcolorbox}
\textsubscript{17} Слова мудрых, [высказанные] спокойно, выслушиваются [лучше], нежели крик властелина между глупыми.
\end{tcolorbox}
\begin{tcolorbox}
\textsubscript{18} Мудрость лучше воинских орудий; но один погрешивший погубит много доброго.
\end{tcolorbox}
\subsection{CHAPTER 10}
\begin{tcolorbox}
\textsubscript{1} Мертвые мухи портят и делают зловонною благовонную масть мироварника: то же делает небольшая глупость уважаемого человека с его мудростью и честью.
\end{tcolorbox}
\begin{tcolorbox}
\textsubscript{2} Сердце мудрого--на правую сторону, а сердце глупого--на левую.
\end{tcolorbox}
\begin{tcolorbox}
\textsubscript{3} По какой бы дороге ни шел глупый, у него [всегда] недостает смысла, и всякому он выскажет, что он глуп.
\end{tcolorbox}
\begin{tcolorbox}
\textsubscript{4} Если гнев начальника вспыхнет на тебя, то не оставляй места твоего; потому что кротость покрывает и большие проступки.
\end{tcolorbox}
\begin{tcolorbox}
\textsubscript{5} Есть зло, которое видел я под солнцем, это--как бы погрешность, происходящая от властелина;
\end{tcolorbox}
\begin{tcolorbox}
\textsubscript{6} невежество поставляется на большой высоте, а богатые сидят низко.
\end{tcolorbox}
\begin{tcolorbox}
\textsubscript{7} Видел я рабов на конях, а князей ходящих, подобно рабам, пешком.
\end{tcolorbox}
\begin{tcolorbox}
\textsubscript{8} Кто копает яму, тот упадет в нее, и кто разрушает ограду, того ужалит змей.
\end{tcolorbox}
\begin{tcolorbox}
\textsubscript{9} Кто передвигает камни, тот может надсадить себя, и кто колет дрова, тот может подвергнуться опасности от них.
\end{tcolorbox}
\begin{tcolorbox}
\textsubscript{10} Если притупится топор, и если лезвие его не будет отточено, то надобно будет напрягать силы; мудрость умеет это исправить.
\end{tcolorbox}
\begin{tcolorbox}
\textsubscript{11} Если змей ужалит без заговаривания, то не лучше его и злоязычный.
\end{tcolorbox}
\begin{tcolorbox}
\textsubscript{12} Слова из уст мудрого--благодать, а уста глупого губят его же:
\end{tcolorbox}
\begin{tcolorbox}
\textsubscript{13} начало слов из уст его--глупость, [а] конец речи из уст его--безумие.
\end{tcolorbox}
\begin{tcolorbox}
\textsubscript{14} Глупый наговорит много, [хотя] человек не знает, что будет, и кто скажет ему, что будет после него?
\end{tcolorbox}
\begin{tcolorbox}
\textsubscript{15} Труд глупого утомляет его, потому что не знает [даже] дороги в город.
\end{tcolorbox}
\begin{tcolorbox}
\textsubscript{16} Горе тебе, земля, когда царь твой отрок, и когда князья твои едят рано!
\end{tcolorbox}
\begin{tcolorbox}
\textsubscript{17} Благо тебе, земля, когда царь у тебя из благородного рода, и князья твои едят вовремя, для подкрепления, а не для пресыщения!
\end{tcolorbox}
\begin{tcolorbox}
\textsubscript{18} От лености обвиснет потолок, и когда опустятся руки, то протечет дом.
\end{tcolorbox}
\begin{tcolorbox}
\textsubscript{19} Пиры устраиваются для удовольствия, и вино веселит жизнь; а за все отвечает серебро.
\end{tcolorbox}
\begin{tcolorbox}
\textsubscript{20} Даже и в мыслях твоих не злословь царя, и в спальной комнате твоей не злословь богатого; потому что птица небесная может перенести слово [твое], и крылатая--пересказать речь [твою].
\end{tcolorbox}
\subsection{CHAPTER 11}
\begin{tcolorbox}
\textsubscript{1} Отпускай хлеб твой по водам, потому что по прошествии многих дней опять найдешь его.
\end{tcolorbox}
\begin{tcolorbox}
\textsubscript{2} Давай часть семи и даже восьми, потому что не знаешь, какая беда будет на земле.
\end{tcolorbox}
\begin{tcolorbox}
\textsubscript{3} Когда облака будут полны, то они прольют на землю дождь; и если упадет дерево на юг или на север, то оно там и останется, куда упадет.
\end{tcolorbox}
\begin{tcolorbox}
\textsubscript{4} Кто наблюдает ветер, тому не сеять; и кто смотрит на облака, тому не жать.
\end{tcolorbox}
\begin{tcolorbox}
\textsubscript{5} Как ты не знаешь путей ветра и того, как [образуются] кости во чреве беременной, так не можешь знать дело Бога, Который делает все.
\end{tcolorbox}
\begin{tcolorbox}
\textsubscript{6} Утром сей семя твое, и вечером не давай отдыха руке твоей, потому что ты не знаешь, то или другое будет удачнее, или то и другое равно хорошо будет.
\end{tcolorbox}
\begin{tcolorbox}
\textsubscript{7} Сладок свет, и приятно для глаз видеть солнце.
\end{tcolorbox}
\begin{tcolorbox}
\textsubscript{8} Если человек проживет [и] много лет, то пусть веселится он в продолжение всех их, и пусть помнит о днях темных, которых будет много: все, что будет, --суета!
\end{tcolorbox}
\begin{tcolorbox}
\textsubscript{9} Веселись, юноша, в юности твоей, и да вкушает сердце твое радости во дни юности твоей, и ходи по путям сердца твоего и по видению очей твоих; только знай, что за все это Бог приведет тебя на суд.
\end{tcolorbox}
\begin{tcolorbox}
\textsubscript{10} И удаляй печаль от сердца твоего, и уклоняй злое от тела твоего, потому что детство и юность--суета.
\end{tcolorbox}
\subsection{CHAPTER 12}
\begin{tcolorbox}
\textsubscript{1} И помни Создателя твоего в дни юности твоей, доколе не пришли тяжелые дни и не наступили годы, о которых ты будешь говорить: 'нет мне удовольствия в них!'
\end{tcolorbox}
\begin{tcolorbox}
\textsubscript{2} доколе не померкли солнце и свет и луна и звезды, и не нашли новые тучи вслед за дождем.
\end{tcolorbox}
\begin{tcolorbox}
\textsubscript{3} В тот день, когда задрожат стерегущие дом и согнутся мужи силы; и перестанут молоть мелющие, потому что их немного осталось; и помрачатся смотрящие в окно;
\end{tcolorbox}
\begin{tcolorbox}
\textsubscript{4} и запираться будут двери на улицу; когда замолкнет звук жернова, и будет вставать [человек] по крику петуха и замолкнут дщери пения;
\end{tcolorbox}
\begin{tcolorbox}
\textsubscript{5} и высоты будут им страшны, и на дороге ужасы; и зацветет миндаль, и отяжелеет кузнечик, и рассыплется каперс. Ибо отходит человек в вечный дом свой, и готовы окружить его по улице плакальщицы; --
\end{tcolorbox}
\begin{tcolorbox}
\textsubscript{6} доколе не порвалась серебряная цепочка, и не разорвалась золотая повязка, и не разбился кувшин у источника, и не обрушилось колесо над колодезем.
\end{tcolorbox}
\begin{tcolorbox}
\textsubscript{7} И возвратится прах в землю, чем он и был; а дух возвратился к Богу, Который дал его.
\end{tcolorbox}
\begin{tcolorbox}
\textsubscript{8} Суета сует, сказал Екклесиаст, всё--суета!
\end{tcolorbox}
\begin{tcolorbox}
\textsubscript{9} Кроме того, что Екклесиаст был мудр, он учил еще народ знанию. Он [все] испытывал, исследовал, [и] составил много притчей.
\end{tcolorbox}
\begin{tcolorbox}
\textsubscript{10} Старался Екклесиаст приискивать изящные изречения, и слова истины написаны [им] верно.
\end{tcolorbox}
\begin{tcolorbox}
\textsubscript{11} Слова мудрых--как иглы и как вбитые гвозди, и составители их--от единого пастыря.
\end{tcolorbox}
\begin{tcolorbox}
\textsubscript{12} А что сверх всего этого, сын мой, того берегись: составлять много книг--конца не будет, и много читать--утомительно для тела.
\end{tcolorbox}
\begin{tcolorbox}
\textsubscript{13} Выслушаем сущность всего: бойся Бога и заповеди Его соблюдай, потому что в этом всё для человека;
\end{tcolorbox}
\begin{tcolorbox}
\textsubscript{14} ибо всякое дело Бог приведет на суд, и все тайное, хорошо ли оно, или худо.
\end{tcolorbox}
