\section{BOOK 62}
\subsection{CHAPTER 1}
\begin{tcolorbox}
\textsubscript{1} Старець вибраній пані та дітям її, яких я поправді люблю, і не тільки я, але й усі, хто правду пізнав,
\end{tcolorbox}
\begin{tcolorbox}
\textsubscript{2} за правду, що в нас пробуває й повік буде з нами:
\end{tcolorbox}
\begin{tcolorbox}
\textsubscript{3} нехай буде з вами благодать, милість, мир від Бога Отця та Ісуса Христа, Сина Отцевого, у правді та в любові!
\end{tcolorbox}
\begin{tcolorbox}
\textsubscript{4} Я дуже зрадів, що між дітьми твоїми знайшов таких, що ходять у правді, як заповідь ми прийняли від Отця.
\end{tcolorbox}
\begin{tcolorbox}
\textsubscript{5} І тепер я благаю тебе, пані, не так, ніби пишу тобі нову заповідь, але ту, яку маємо від початку, щоб ми любили один одного!
\end{tcolorbox}
\begin{tcolorbox}
\textsubscript{6} А любов ця щоб ми жили згідно з Його заповідями. Це та заповідь, яку ви чули від початку, щоб ви згідно з нею жили.
\end{tcolorbox}
\begin{tcolorbox}
\textsubscript{7} Бо в світ увійшло багато обманців, які не визнають Ісуса Христа, що прийшов був у тілі. Такий то обманець та антихрист!
\end{tcolorbox}
\begin{tcolorbox}
\textsubscript{8} Пильнуйте себе, щоб ви не згубили того, над чим працювали, але щоб прийняли повну нагороду.
\end{tcolorbox}
\begin{tcolorbox}
\textsubscript{9} Кожен, хто робить переступ та не пробуває в науці Христовій, той Бога не має. А хто пробуває в науці Його, той має і Отця, і Сина.
\end{tcolorbox}
\begin{tcolorbox}
\textsubscript{10} Коли хто приходить до вас, але не приносить науки цієї, не приймайте до дому його, і не вітайте його!
\end{tcolorbox}
\begin{tcolorbox}
\textsubscript{11} Хто бо вітає його, той участь бере в лихих учинках його.
\end{tcolorbox}
\begin{tcolorbox}
\textsubscript{12} Багато я мав написати до вас, але не схотів на папері й чорнилом. Та маю надію прибути до вас, і говорити устами до уст, щоб повна була ваша радість!
\end{tcolorbox}
\begin{tcolorbox}
\textsubscript{13} Вітають тебе діти вибраної сестри твоєї. Амінь.
\end{tcolorbox}
