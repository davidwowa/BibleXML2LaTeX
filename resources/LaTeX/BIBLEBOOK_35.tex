Book 34
\subsection{CHAPTER 1}
\begin{tcolorbox}
\textsubscript{1} Пророцтво, яке бачив пророк Авакум.
\end{tcolorbox}
\begin{tcolorbox}
\textsubscript{2} Аж доки я, Господи, кликати буду, а Ти не почуєш? До тебе я кличу: Насильство! та Ти не спасаєш!
\end{tcolorbox}
\begin{tcolorbox}
\textsubscript{3} Для чого неправість мені Ти показуєш та позираєш на муку? А передо мною грабіж та насильство, і суперечка стається, і носиться сварка.
\end{tcolorbox}
\begin{tcolorbox}
\textsubscript{4} Тому то Закон припиняється, і не виходить до чину назавсіди право, бо несправедливий вигублює праведного, тому правосуддя виходить покривленим.
\end{tcolorbox}
\begin{tcolorbox}
\textsubscript{5} Пригляньтеся ви до народів, і дивіться, і дуже здивуйтесь, бо вчиню Я за ваших днів діло, про яке не повірите ви, коли буде розказане.
\end{tcolorbox}
\begin{tcolorbox}
\textsubscript{6} Бо оце Я поставлю халдеїв, народ лютий та скорий, що ґрасує по широкій землі, щоб захопити оселі, які не його.
\end{tcolorbox}
\begin{tcolorbox}
\textsubscript{7} Страшний та грізний він, від нього самого виходить і право його, і великість його.
\end{tcolorbox}
\begin{tcolorbox}
\textsubscript{8} І від пантер його коні швидші, і від вовків вечерових лютіші.
\end{tcolorbox}
\begin{tcolorbox}
\textsubscript{9} Він приходить увесь на насильство, а ціль їх обличчя вперед, і набере полонених, як того піску.
\end{tcolorbox}
\begin{tcolorbox}
\textsubscript{10} І він глузує з царів, а князі сміх для нього. Він сміється з твердині усякої, бо на вал насипає землі, і її здобуває!
\end{tcolorbox}
\begin{tcolorbox}
\textsubscript{11} Тоді він несеться, як вітер, і перейде, і згрішить, бо зробить за бога свого оцю силу свою.
\end{tcolorbox}
\begin{tcolorbox}
\textsubscript{12} Хіба ж Ти не віддавна, о Господи? Боже Ти мій, мій Святий, не помремо! Господи, Ти для суду поставив його, і, о Скеле, призначив його на карання!
\end{tcolorbox}
\begin{tcolorbox}
\textsubscript{13} Твої очі занадто пречисті, щоб міг Ти дивитись на зло, і на насильство дивитись не можеш. Чому ж дивишся Ти на грабіжників, мовчиш, коли несправедливий винищує справедливішого від себе?
\end{tcolorbox}
\begin{tcolorbox}
\textsubscript{14} Ти ж маєш людей, як у морі тих риб, немов ту черву, що пана над нею нема.
\end{tcolorbox}
\begin{tcolorbox}
\textsubscript{15} Усе це грабіжник витягує вудкою, своїм неводом тягне оце, та збирає оце в свою сітку, тому тішиться він та радіє.
\end{tcolorbox}
\begin{tcolorbox}
\textsubscript{16} Тому жертву приносить він неводові, і кадить для сітки своєї, бо від них ситий уділ його та добірна пожива його!
\end{tcolorbox}
\begin{tcolorbox}
\textsubscript{17} Чи на це випорожнює він свого невода, і завжди готов убивати народи без милости?
\end{tcolorbox}
\subsection{CHAPTER 2}
\begin{tcolorbox}
\textsubscript{1} Нехай я стою на сторожі своїй, і нехай на облозі я стану, і хай виглядаю, щоб бачити, що він буде казати мені, і що відповість на жалобу мою.
\end{tcolorbox}
\begin{tcolorbox}
\textsubscript{2} А Господь відповів та й сказав: Напиши це видіння і поясни на таблицях, щоб читач його легко читав.
\end{tcolorbox}
\begin{tcolorbox}
\textsubscript{3} Бо ще на умовлений час це видіння, і приспішає кінець, і не обмане. Якщо б протягнулось, чекай ти його, бо воно конче прийде, не спізниться.
\end{tcolorbox}
\begin{tcolorbox}
\textsubscript{4} Ось надута, не проста душа його в ньому, а праведний житиме вірою своєю.
\end{tcolorbox}
\begin{tcolorbox}
\textsubscript{5} І що ж, як зрадливе вино, так горда людина спокою не знає: він роззявлює пащу свою, як шеол, і не насичується, як та смерть, і всіх людей він до себе збирає, і всі народи до себе згромаджує.
\end{tcolorbox}
\begin{tcolorbox}
\textsubscript{6} Чи ж усі вони не складуть приповістки на нього та загадки насмішливої йому не прокажуть: Горе тому, хто для себе розмножує те, що не його! Аж доки це буде? Горе тому, хто чинить тяжкою заставу на себе!
\end{tcolorbox}
\begin{tcolorbox}
\textsubscript{7} Хіба нагло не встануть оті, хто тебе буде гризти, і збудяться ті, хто тебе попихає, і за здобич ти станеш для них?
\end{tcolorbox}
\begin{tcolorbox}
\textsubscript{8} За те, що ти грабував був багато народів, вся решта народів тебе пограбує за ту людську кров, і за насильство над краєм, над містом та над усіма, хто мешкає в ньому.
\end{tcolorbox}
\begin{tcolorbox}
\textsubscript{9} Горе тому, хто неправедний зиск побирає для дому свого, щоб покласти гніздо своє на висоті, і тим із рук злого врятованим бути!
\end{tcolorbox}
\begin{tcolorbox}
\textsubscript{10} Нарадив ти сором для дому свого, щоб кінець учинити численним народам, і ти прогрішився за душу свою.
\end{tcolorbox}
\begin{tcolorbox}
\textsubscript{11} Бо камінь з стіни буде кликати, і йому відповість сволок із дерева.
\end{tcolorbox}
\begin{tcolorbox}
\textsubscript{12} Горе тому, хто кров'ю місто будує, хто беззаконням встановлює город!
\end{tcolorbox}
\begin{tcolorbox}
\textsubscript{13} Чи ж оце не від Господа Саваота, що народи трудяться для огню, і мучаться люди на марність?
\end{tcolorbox}
\begin{tcolorbox}
\textsubscript{14} Бо пізнанням Господньої слави наповнена буде земля, як море вода покриває.
\end{tcolorbox}
\begin{tcolorbox}
\textsubscript{15} Горе тому, хто свого ближнього напоює з келіху гніву свого, і поїть, щоб бачити сором його!
\end{tcolorbox}
\begin{tcolorbox}
\textsubscript{16} Ти наситишся ганьбою більше, як славою. Пий також ти, та показуй свій сором! На тебе обернеться келіх правиці Господньої, ганьба ж на славу твою!
\end{tcolorbox}
\begin{tcolorbox}
\textsubscript{17} Бо насилля твоє над Ліваном на тебе спаде, а грабунок худоби зламає тебе за кров людську, та за насилля над краєм, над містом та над усіма, хто мешкає в ньому.
\end{tcolorbox}
\begin{tcolorbox}
\textsubscript{18} Який дасть пожиток бовван, що його вирізьбив творець його, і відлив, і вчитель неправди, що творець його мав охоту чинити богів цих німих?
\end{tcolorbox}
\begin{tcolorbox}
\textsubscript{19} Горе тому, хто дереву каже: Збудись, мовчазливому каменю: Зрушся! Чи він буде навчати? Ось він сріблом та золотом викладений, але жодного духу в ньому нема!
\end{tcolorbox}
\begin{tcolorbox}
\textsubscript{20} А Господь у Своїм храмі святім, мовчи перед обличчям Його, уся земле!
\end{tcolorbox}
\subsection{CHAPTER 3}
\begin{tcolorbox}
\textsubscript{1} Молитва пророка Авакума, для співу на струнному приладі:
\end{tcolorbox}
\begin{tcolorbox}
\textsubscript{2} Господи, звістку Твою я почув та й злякався! Господи, оживи Своє діло в середині років, у середині літ об'яви, у гніві про милість згадай!
\end{tcolorbox}
\begin{tcolorbox}
\textsubscript{3} Бог іде від Теману, і Святий від Парану гори. Села. Велич Його вкрила небо, і слави Його стала повна земля!
\end{tcolorbox}
\begin{tcolorbox}
\textsubscript{4} А сяйво було, наче соняшне світло, проміння при боці у Нього, і там укриття Його потуги.
\end{tcolorbox}
\begin{tcolorbox}
\textsubscript{5} Перед обличчям Його моровиця іде, а по стопах Його пнеться полум'я.
\end{tcolorbox}
\begin{tcolorbox}
\textsubscript{6} Став, і землю Він зміряв, поглянув і народи затряс, і попадали гори довічні, вікові похилились узгір'я. Путі Його вічні.
\end{tcolorbox}
\begin{tcolorbox}
\textsubscript{7} Я бачив намети Кушана під кривдою, тремтять покривала мідіянського краю.
\end{tcolorbox}
\begin{tcolorbox}
\textsubscript{8} Чи на ріки, о Господи, Ти запалав, чи на ріки Твій гнів? Чи Твоє пересердя на море, що їздиш на конях Своїх, на спасенних Своїх колесницях?
\end{tcolorbox}
\begin{tcolorbox}
\textsubscript{9} Лук твій голий, нагий, наповнений стріл сагайдак. Села. Ти річками землю розсік.
\end{tcolorbox}
\begin{tcolorbox}
\textsubscript{10} Тебе вгледівши, гори дрижали, водяна течія потекла, безодня свій голос дала, зняла високо руки свої.
\end{tcolorbox}
\begin{tcolorbox}
\textsubscript{11} Сонце й місяць спинилися в мешканні своєму при світлі Твоїх стріл, що літають при сяйві блискучого списа Твого.
\end{tcolorbox}
\begin{tcolorbox}
\textsubscript{12} У люті ступав Ти землею, у гніві людей молотив.
\end{tcolorbox}
\begin{tcolorbox}
\textsubscript{13} Ти вийшов спасти Свій народ, спасти Помазанця Свого. Ти з дому безбожного голову збив, обнажив Ти основу по шию. Села.
\end{tcolorbox}
\begin{tcolorbox}
\textsubscript{14} Ти пробив його списами голову князя його, як вони піднялись, щоб мене розпорошити; вони тішилися, немов мали пожерти таємно убогого.
\end{tcolorbox}
\begin{tcolorbox}
\textsubscript{15} Ти кіньми Своїми по морю топтав, по водній великій громаді.
\end{tcolorbox}
\begin{tcolorbox}
\textsubscript{16} Я почув і затремтіла утроба моя, задзвеніли на голос цей губи мої, гнилизна ввійшла в мої кості, і тремчу я на місці своїм, бо маю чекати в спокої день утиску, коли прийде народ, який має на вас наступати.
\end{tcolorbox}
\begin{tcolorbox}
\textsubscript{17} Коли б фіґове дерево не зацвіло, і не було б урожаю в виноградниках, обманило зайняття оливкою, а поле їжі не вродило б, позникала отара з кошари і не стало б в оборах худоби,
\end{tcolorbox}
\begin{tcolorbox}
\textsubscript{18} то я Господом тішитись буду й тоді, радітиму Богом спасіння свого!
\end{tcolorbox}
\begin{tcolorbox}
\textsubscript{19} Бог Господь моя сила, і чинить Він ноги мої, як у лані, і водить мене по висотах! Для дириґента хору на моїх струнних знаряддях.
\end{tcolorbox}
