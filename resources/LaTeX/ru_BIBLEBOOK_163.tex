\section{BOOK 162}
\subsection{CHAPTER 1}
\begin{tcolorbox}
\textsubscript{1} Павел, волею Божиею Апостол Иисуса Христа, по обетованию жизни во Христе Иисусе,
\end{tcolorbox}
\begin{tcolorbox}
\textsubscript{2} Тимофею, возлюбленному сыну: благодать, милость, мир от Бога Отца и Христа Иисуса, Господа нашего.
\end{tcolorbox}
\begin{tcolorbox}
\textsubscript{3} Благодарю Бога, Которому служу от прародителей с чистою совестью, что непрестанно вспоминаю о тебе в молитвах моих днем и ночью,
\end{tcolorbox}
\begin{tcolorbox}
\textsubscript{4} и желаю видеть тебя, вспоминая о слезах твоих, дабы мне исполниться радости,
\end{tcolorbox}
\begin{tcolorbox}
\textsubscript{5} приводя на память нелицемерную веру твою, которая прежде обитала в бабке твоей Лоиде и матери твоей Евнике; уверен, что она и в тебе.
\end{tcolorbox}
\begin{tcolorbox}
\textsubscript{6} По сей причине напоминаю тебе возгревать дар Божий, который в тебе через мое рукоположение;
\end{tcolorbox}
\begin{tcolorbox}
\textsubscript{7} ибо дал нам Бог духа не боязни, но силы и любви и целомудрия.
\end{tcolorbox}
\begin{tcolorbox}
\textsubscript{8} Итак, не стыдись свидетельства Господа нашего Иисуса Христа, ни меня, узника Его; но страдай с благовестием Христовым силою Бога,
\end{tcolorbox}
\begin{tcolorbox}
\textsubscript{9} спасшего нас и призвавшего званием святым, не по делам нашим, но по Своему изволению и благодати, данной нам во Христе Иисусе прежде вековых времен,
\end{tcolorbox}
\begin{tcolorbox}
\textsubscript{10} открывшейся же ныне явлением Спасителя нашего Иисуса Христа, разрушившего смерть и явившего жизнь и нетление через благовестие,
\end{tcolorbox}
\begin{tcolorbox}
\textsubscript{11} для которого я поставлен проповедником и Апостолом и учителем язычников.
\end{tcolorbox}
\begin{tcolorbox}
\textsubscript{12} По сей причине я и страдаю так; но не стыжусь. Ибо я знаю, в Кого уверовал, и уверен, что Он силен сохранить залог мой на оный день.
\end{tcolorbox}
\begin{tcolorbox}
\textsubscript{13} Держись образца здравого учения, которое ты слышал от меня, с верою и любовью во Христе Иисусе.
\end{tcolorbox}
\begin{tcolorbox}
\textsubscript{14} Храни добрый залог Духом Святым, живущим в нас.
\end{tcolorbox}
\begin{tcolorbox}
\textsubscript{15} Ты знаешь, что все Асийские оставили меня; в числе их Фигелл и Ермоген.
\end{tcolorbox}
\begin{tcolorbox}
\textsubscript{16} Да даст Господь милость дому Онисифора за то, что он многократно покоил меня и не стыдился уз моих,
\end{tcolorbox}
\begin{tcolorbox}
\textsubscript{17} но, быв в Риме, с великим тщанием искал меня и нашел.
\end{tcolorbox}
\begin{tcolorbox}
\textsubscript{18} Да даст ему Господь обрести милость у Господа в оный день; а сколько он служил мне в Ефесе, ты лучше знаешь.
\end{tcolorbox}
\subsection{CHAPTER 2}
\begin{tcolorbox}
\textsubscript{1} Итак укрепляйся, сын мой, в благодати Христом Иисусом,
\end{tcolorbox}
\begin{tcolorbox}
\textsubscript{2} и что слышал от меня при многих свидетелях, то передай верным людям, которые были бы способны и других научить.
\end{tcolorbox}
\begin{tcolorbox}
\textsubscript{3} Итак переноси страдания, как добрый воин Иисуса Христа.
\end{tcolorbox}
\begin{tcolorbox}
\textsubscript{4} Никакой воин не связывает себя делами житейскими, чтобы угодить военачальнику.
\end{tcolorbox}
\begin{tcolorbox}
\textsubscript{5} Если же кто и подвизается, не увенчивается, если незаконно будет подвизаться.
\end{tcolorbox}
\begin{tcolorbox}
\textsubscript{6} Трудящемуся земледельцу первому должно вкусить от плодов.
\end{tcolorbox}
\begin{tcolorbox}
\textsubscript{7} Разумей, что я говорю. Да даст тебе Господь разумение во всем.
\end{tcolorbox}
\begin{tcolorbox}
\textsubscript{8} Помни Господа Иисуса Христа от семени Давидова, воскресшего из мертвых, по благовествованию моему,
\end{tcolorbox}
\begin{tcolorbox}
\textsubscript{9} за которое я страдаю даже до уз, как злодей; но для слова Божия нет уз.
\end{tcolorbox}
\begin{tcolorbox}
\textsubscript{10} Посему я все терплю ради избранных, дабы и они получили спасение во Христе Иисусе с вечною славою.
\end{tcolorbox}
\begin{tcolorbox}
\textsubscript{11} Верно слово: если мы с Ним умерли, то с Ним и оживем;
\end{tcolorbox}
\begin{tcolorbox}
\textsubscript{12} если терпим, то с Ним и царствовать будем; если отречемся, и Он отречется от нас;
\end{tcolorbox}
\begin{tcolorbox}
\textsubscript{13} если мы неверны, Он пребывает верен, ибо Себя отречься не может.
\end{tcolorbox}
\begin{tcolorbox}
\textsubscript{14} Сие напоминай, заклиная пред Господом не вступать в словопрения, что нимало не служит к пользе, а к расстройству слушающих.
\end{tcolorbox}
\begin{tcolorbox}
\textsubscript{15} Старайся представить себя Богу достойным, делателем неукоризненным, верно преподающим слово истины.
\end{tcolorbox}
\begin{tcolorbox}
\textsubscript{16} А непотребного пустословия удаляйся; ибо они еще более будут преуспевать в нечестии,
\end{tcolorbox}
\begin{tcolorbox}
\textsubscript{17} и слово их, как рак, будет распространяться. Таковы Именей и Филит,
\end{tcolorbox}
\begin{tcolorbox}
\textsubscript{18} которые отступили от истины, говоря, что воскресение уже было, и разрушают в некоторых веру.
\end{tcolorbox}
\begin{tcolorbox}
\textsubscript{19} Но твердое основание Божие стоит, имея печать сию: 'познал Господь Своих'; и: 'да отступит от неправды всякий, исповедующий имя Господа'.
\end{tcolorbox}
\begin{tcolorbox}
\textsubscript{20} А в большом доме есть сосуды не только золотые и серебряные, но и деревянные и глиняные; и одни в почетном, а другие в низком употреблении.
\end{tcolorbox}
\begin{tcolorbox}
\textsubscript{21} Итак, кто будет чист от сего, тот будет сосудом в чести, освященным и благопотребным Владыке, годным на всякое доброе дело.
\end{tcolorbox}
\begin{tcolorbox}
\textsubscript{22} Юношеских похотей убегай, а держись правды, веры, любви, мира со всеми призывающими Господа от чистого сердца.
\end{tcolorbox}
\begin{tcolorbox}
\textsubscript{23} От глупых и невежественных состязаний уклоняйся, зная, что они рождают ссоры;
\end{tcolorbox}
\begin{tcolorbox}
\textsubscript{24} рабу же Господа не должно ссориться, но быть приветливым ко всем, учительным, незлобивым,
\end{tcolorbox}
\begin{tcolorbox}
\textsubscript{25} с кротостью наставлять противников, не даст ли им Бог покаяния к познанию истины,
\end{tcolorbox}
\begin{tcolorbox}
\textsubscript{26} чтобы они освободились от сети диавола, который уловил их в свою волю.
\end{tcolorbox}
\subsection{CHAPTER 3}
\begin{tcolorbox}
\textsubscript{1} Знай же, что в последние дни наступят времена тяжкие.
\end{tcolorbox}
\begin{tcolorbox}
\textsubscript{2} Ибо люди будут самолюбивы, сребролюбивы, горды, надменны, злоречивы, родителям непокорны, неблагодарны, нечестивы, недружелюбны,
\end{tcolorbox}
\begin{tcolorbox}
\textsubscript{3} непримирительны, клеветники, невоздержны, жестоки, не любящие добра,
\end{tcolorbox}
\begin{tcolorbox}
\textsubscript{4} предатели, наглы, напыщенны, более сластолюбивы, нежели боголюбивы,
\end{tcolorbox}
\begin{tcolorbox}
\textsubscript{5} имеющие вид благочестия, силы же его отрекшиеся. Таковых удаляйся.
\end{tcolorbox}
\begin{tcolorbox}
\textsubscript{6} К сим принадлежат те, которые вкрадываются в домы и обольщают женщин, утопающих во грехах, водимых различными похотями,
\end{tcolorbox}
\begin{tcolorbox}
\textsubscript{7} всегда учащихся и никогда не могущих дойти до познания истины.
\end{tcolorbox}
\begin{tcolorbox}
\textsubscript{8} Как Ианний и Иамврий противились Моисею, так и сии противятся истине, люди, развращенные умом, невежды в вере.
\end{tcolorbox}
\begin{tcolorbox}
\textsubscript{9} Но они не много успеют; ибо их безумие обнаружится перед всеми, как и с теми случилось.
\end{tcolorbox}
\begin{tcolorbox}
\textsubscript{10} А ты последовал мне в учении, житии, расположении, вере, великодушии, любви, терпении,
\end{tcolorbox}
\begin{tcolorbox}
\textsubscript{11} в гонениях, страданиях, постигших меня в Антиохии, Иконии, Листрах; каковые гонения я перенес, и от всех избавил меня Господь.
\end{tcolorbox}
\begin{tcolorbox}
\textsubscript{12} Да и все, желающие жить благочестиво во Христе Иисусе, будут гонимы.
\end{tcolorbox}
\begin{tcolorbox}
\textsubscript{13} Злые же люди и обманщики будут преуспевать во зле, вводя в заблуждение и заблуждаясь.
\end{tcolorbox}
\begin{tcolorbox}
\textsubscript{14} А ты пребывай в том, чему научен и что тебе вверено, зная, кем ты научен.
\end{tcolorbox}
\begin{tcolorbox}
\textsubscript{15} Притом же ты из детства знаешь священные писания, которые могут умудрить тебя во спасение верою во Христа Иисуса.
\end{tcolorbox}
\begin{tcolorbox}
\textsubscript{16} Все Писание богодухновенно и полезно для научения, для обличения, для исправления, для наставления в праведности,
\end{tcolorbox}
\begin{tcolorbox}
\textsubscript{17} да будет совершен Божий человек, ко всякому доброму делу приготовлен.
\end{tcolorbox}
\subsection{CHAPTER 4}
\begin{tcolorbox}
\textsubscript{1} Итак заклинаю тебя пред Богом и Господом нашим Иисусом Христом, Который будет судить живых и мертвых в явление Его и Царствие Его:
\end{tcolorbox}
\begin{tcolorbox}
\textsubscript{2} проповедуй слово, настой во время и не во время, обличай, запрещай, увещевай со всяким долготерпением и назиданием.
\end{tcolorbox}
\begin{tcolorbox}
\textsubscript{3} Ибо будет время, когда здравого учения принимать не будут, но по своим прихотям будут избирать себе учителей, которые льстили бы слуху;
\end{tcolorbox}
\begin{tcolorbox}
\textsubscript{4} и от истины отвратят слух и обратятся к басням.
\end{tcolorbox}
\begin{tcolorbox}
\textsubscript{5} Но ты будь бдителен во всем, переноси скорби, совершай дело благовестника, исполняй служение твое.
\end{tcolorbox}
\begin{tcolorbox}
\textsubscript{6} Ибо я уже становлюсь жертвою, и время моего отшествия настало.
\end{tcolorbox}
\begin{tcolorbox}
\textsubscript{7} Подвигом добрым я подвизался, течение совершил, веру сохранил;
\end{tcolorbox}
\begin{tcolorbox}
\textsubscript{8} а теперь готовится мне венец правды, который даст мне Господь, праведный Судия, в день оный; и не только мне, но и всем, возлюбившим явление Его.
\end{tcolorbox}
\begin{tcolorbox}
\textsubscript{9} Постарайся придти ко мне скоро.
\end{tcolorbox}
\begin{tcolorbox}
\textsubscript{10} Ибо Димас оставил меня, возлюбив нынешний век, и пошел в Фессалонику, Крискент в Галатию, Тит в Далматию; один Лука со мною.
\end{tcolorbox}
\begin{tcolorbox}
\textsubscript{11} Марка возьми и приведи с собою, ибо он мне нужен для служения.
\end{tcolorbox}
\begin{tcolorbox}
\textsubscript{12} Тихика я послал в Ефес.
\end{tcolorbox}
\begin{tcolorbox}
\textsubscript{13} Когда пойдешь, принеси фелонь, который я оставил в Троаде у Карпа, и книги, особенно кожаные.
\end{tcolorbox}
\begin{tcolorbox}
\textsubscript{14} Александр медник много сделал мне зла. Да воздаст ему Господь по делам его!
\end{tcolorbox}
\begin{tcolorbox}
\textsubscript{15} Берегись его и ты, ибо он сильно противился нашим словам.
\end{tcolorbox}
\begin{tcolorbox}
\textsubscript{16} При первом моем ответе никого не было со мною, но все меня оставили. Да не вменится им!
\end{tcolorbox}
\begin{tcolorbox}
\textsubscript{17} Господь же предстал мне и укрепил меня, дабы через меня утвердилось благовестие и услышали все язычники; и я избавился из львиных челюстей.
\end{tcolorbox}
\begin{tcolorbox}
\textsubscript{18} И избавит меня Господь от всякого злого дела и сохранит для Своего Небесного Царства, Ему слава во веки веков. Аминь.
\end{tcolorbox}
\begin{tcolorbox}
\textsubscript{19} Приветствуй Прискиллу и Акилу и дом Онисифоров.
\end{tcolorbox}
\begin{tcolorbox}
\textsubscript{20} Ераст остался в Коринфе; Трофима же я оставил больного в Милите.
\end{tcolorbox}
\begin{tcolorbox}
\textsubscript{21} Постарайся придти до зимы. Приветствуют тебя Еввул, и Пуд, и Лин, и Клавдия, и все братия.
\end{tcolorbox}
\begin{tcolorbox}
\textsubscript{22} Господь Иисус Христос со духом твоим. Благодать с вами. Аминь.
\end{tcolorbox}
