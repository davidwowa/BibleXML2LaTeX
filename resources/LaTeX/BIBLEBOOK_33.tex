Book 32
\subsection{CHAPTER 1}
\begin{tcolorbox}
\textsubscript{1} Слово Господнє, що було до морешетського Михея за днів Йотама, Ахаза та Єзекії, Юдиних царів, яке він бачив на Самарію та Єрусалим.
\end{tcolorbox}
\begin{tcolorbox}
\textsubscript{2} Почуйте оце, всі народи, послухай, ти земле та все, що на ній! і хай буде за свідка на вас Господь Бог, Господь з храму святого Свого!
\end{tcolorbox}
\begin{tcolorbox}
\textsubscript{3} Бо Господь ось виходить із місця Свого, і Він сходить і ступає по висотах землі.
\end{tcolorbox}
\begin{tcolorbox}
\textsubscript{4} І топляться гори під Ним, і тануть долини, мов віск від огню, мов ті води, що ллються з узбіччя.
\end{tcolorbox}
\begin{tcolorbox}
\textsubscript{5} Усе це за провинення Якова, за гріхи дому Ізраїля. Хто провинення Якова, чи ж не Самарія? А хто гріх дому Юдиного, чи ж не Єрусалим?
\end{tcolorbox}
\begin{tcolorbox}
\textsubscript{6} І зроблю Самарію руїною в полі, за місце садити виноград, і повикидаю в долину каміння її, і відкрию основи її.
\end{tcolorbox}
\begin{tcolorbox}
\textsubscript{7} І потовчені будуть усі її ідоли, всі ж дарунки її за розпусту попалені будуть в огні, і всіх бовванів її Я віддам на спустошення. Бо зібрала вона від дарунків за блуд, і на подарунки за блуд це повернеться.
\end{tcolorbox}
\begin{tcolorbox}
\textsubscript{8} Над оцим голоситиму я та ридатиму, ходитиму босий й нагий, заводити буду, немов ті шакали, і буду тужити, як струсі!
\end{tcolorbox}
\begin{tcolorbox}
\textsubscript{9} Бо рани її невигойні, бо це аж до Юди прийшло, воно досягло аж до брами народу Мого, аж до Єрусалиму.
\end{tcolorbox}
\begin{tcolorbox}
\textsubscript{10} Цього не оголошуйте в Ґаті, і плакати не плачте, качайтесь по поросі в Бет-Леафрі.
\end{tcolorbox}
\begin{tcolorbox}
\textsubscript{11} Переходь собі ти, о мешканко Шафіру, нага, осоромлена, вже бо не вийде мешканка Цаанану, голосіння Бет-Гаецелу не дасть вам спинитися в ньому.
\end{tcolorbox}
\begin{tcolorbox}
\textsubscript{12} Бо мешканка Мароту чекала добра, та до єрусалимської брами зійшло оце лихо від Господа.
\end{tcolorbox}
\begin{tcolorbox}
\textsubscript{13} Запряжи баскі коні до воза, мешканко Лахішу! Ти початок гріха для сіонської доньки, бо знайшлись серед тебе провини Ізраїлеві,
\end{tcolorbox}
\begin{tcolorbox}
\textsubscript{14} тому то даси розводові листи на Морешет-Ґат. Доми Ахзіва омана для Ізраїлевих царів.
\end{tcolorbox}
\begin{tcolorbox}
\textsubscript{15} Спроваджу тобі ще спадкоємця Я, о мешканко Мареші, Аж по Адуллам прийде слава Ізраїля.
\end{tcolorbox}
\begin{tcolorbox}
\textsubscript{16} Зроби собі лисину та острижися за синів своїх любих, пошир свою лисину, мов ув орла, бо пішли на вигнання від тебе вони!
\end{tcolorbox}
\subsection{CHAPTER 2}
\begin{tcolorbox}
\textsubscript{1} Горе тим, що задумують кривду і на ложах своїх учиняють лихе! За світла поранку виконують це, бо їхня рука має силу.
\end{tcolorbox}
\begin{tcolorbox}
\textsubscript{2} Якщо піль жадають, то грабують вони, а домів то хапають. І вони переслідують мужа та дома його, і чоловіка й спадки його.
\end{tcolorbox}
\begin{tcolorbox}
\textsubscript{3} Тому так промовляє Господь: Ось Я замишляю на цей рід лихе, що ший своїх з нього не визволите, і ходити не будете гордо, бо це час лихий.
\end{tcolorbox}
\begin{tcolorbox}
\textsubscript{4} Того дня проголосять на вас приповістку, і співатимуть пісню жалобну, говорячи: Сталось! До пня опустошені ми, уділ народу мого змінився, як це діткнуло мене! Наше поле поділене буде чужинцями,
\end{tcolorbox}
\begin{tcolorbox}
\textsubscript{5} тому в тебе не буде нікого, хто кидав би шнура мірничого, як жеребка на Господнім зібранні.
\end{tcolorbox}
\begin{tcolorbox}
\textsubscript{6} Не проповідуйте, та вони проповідують! Хай нам не проповідують, не досягне нас ганьба.
\end{tcolorbox}
\begin{tcolorbox}
\textsubscript{7} О ти, що звешся Яковів дім, чи змалів Дух Господній? Чи ці чини Його? Хіба добре не роблять слова Мої тому, хто ходить правдиво?
\end{tcolorbox}
\begin{tcolorbox}
\textsubscript{8} Ще вчора були ви народом Моїм, тепер же стаєте за ворога, з одежі верхньої плаща ви стягаєте з тих, хто проходить безпечно, як здобич війни.
\end{tcolorbox}
\begin{tcolorbox}
\textsubscript{9} Жінок Мого народу з приємного дому її виганяєте кожну, з дітей славу Мою ви берете навіки.
\end{tcolorbox}
\begin{tcolorbox}
\textsubscript{10} Устаньте й ідіть, бо тут не спочинок, це за занечищення ваше, що загладу для вас принесе, вирішальну загладу.
\end{tcolorbox}
\begin{tcolorbox}
\textsubscript{11} Коли б чоловік, який ходить за вітром, і брехню набрехав, говорячи: Буду тобі проповідувати про вино та про напій п'янкий, то був би він любим пророком оцьому народові.
\end{tcolorbox}
\begin{tcolorbox}
\textsubscript{12} Невідмінно зберу тебе всього, о Якове, невідмінно згромаджу останок Ізраїлів, разом поставлю його, як отару в Боцрі, як ту череду на пасовищі, і будуть гомоніти вони від многолюдства!
\end{tcolorbox}
\begin{tcolorbox}
\textsubscript{13} Перед ними виламувач піде, вони продеруться та браму перейдуть і вийдуть із неї. І піде їхній цар перед ними, а Господь на чолі їх!
\end{tcolorbox}
\subsection{CHAPTER 3}
\begin{tcolorbox}
\textsubscript{1} А я відказав: Послухайте ж, голови Якова та начальники дому Ізраїля, чи ж не вам знати право?
\end{tcolorbox}
\begin{tcolorbox}
\textsubscript{2} Добро ви ненавидите та кохаєте зло, шкіру їхню здираєте з них, а їхнє тіло з костей їхніх.
\end{tcolorbox}
\begin{tcolorbox}
\textsubscript{3} Ви останок народу Мого їсте та стягаєте з них їхню шкіру, а їхні кості ламаєте, і січете, немов до горняти, і мов м'ясо в котел.
\end{tcolorbox}
\begin{tcolorbox}
\textsubscript{4} Вони тоді кликати будуть до Господа, та Він відповіді їм не дасть, і заховає обличчя Своє того часу від них, коли будуть робити лихі свої вчинки.
\end{tcolorbox}
\begin{tcolorbox}
\textsubscript{5} Так говорить Господь на пророків, що вводять народ Мій у блуд, що зубами своїми гризуть та покликують: Мир! А на того, хто їм не дає що до рота, на нього святую війну оголошують.
\end{tcolorbox}
\begin{tcolorbox}
\textsubscript{6} Тому буде вам ніч, щоб не стало видіння, і стемніє вам, щоб не чарувати. І над тими пророками сонце закотиться, і над ними потемніє день.
\end{tcolorbox}
\begin{tcolorbox}
\textsubscript{7} І посоромлені будуть такі прозорливці, і будуть застиджені чарівники, і всі вони свої уста закриють, бо не буде їм Божої відповіді.
\end{tcolorbox}
\begin{tcolorbox}
\textsubscript{8} А я повний сили й Господнього Духа, і правди й відваги, щоб представити Якову прогріх його, а Ізраїлеві його гріх.
\end{tcolorbox}
\begin{tcolorbox}
\textsubscript{9} Почуйте ж це, голови дому Якового та начальники дому Ізраїля, які нехтують справедливість, а все просте викривлюють,
\end{tcolorbox}
\begin{tcolorbox}
\textsubscript{10} вони кров'ю будують Сіона, а кривдою Єрусалима.
\end{tcolorbox}
\begin{tcolorbox}
\textsubscript{11} Його голови судять за хабара, і навчають за плату його ті священики, і за срібло ворожать пророки його, хоч на Господа вони опираються, кажучи: Хіба не Господь серед нас? Зло не прийде на нас!
\end{tcolorbox}
\begin{tcolorbox}
\textsubscript{12} О так, через вас Сіон буде на поле заораний, а Єрусалим на руїни обернеться, а гора храмова стане взгір'ями лісу...
\end{tcolorbox}
\subsection{CHAPTER 4}
\begin{tcolorbox}
\textsubscript{1} Та буде наприкінці днів, гора дому Господнього міцно поставлена буде вершиною гір, і піднесена буде вона понад узгір'я, і будуть народи до неї пливсти.
\end{tcolorbox}
\begin{tcolorbox}
\textsubscript{2} І підуть численні народи та й скажуть: Ходімо, і вийдім на гору Господню та до дому Якового, і Він буде навчати доріг Своїх нас, і ми будемо ходити стежками Його. Бо вийде Закон із Сіону, а слово Господнє із Єрусалиму.
\end{tcolorbox}
\begin{tcolorbox}
\textsubscript{3} І Він буде судити численні племена, і розсуджувати буде народи міцні аж у далечині. І вони перекують мечі свої на лемеші, а списи свої на серпи. Не підійме меча народ на народ, і більше не будуть навчатись війни!
\end{tcolorbox}
\begin{tcolorbox}
\textsubscript{4} І буде кожен сидіти під своїм виноградником, і під своєю фіґовницею, і не буде того, хто б страшив, бо уста Господа Саваота оце прорекли.
\end{tcolorbox}
\begin{tcolorbox}
\textsubscript{5} Усі бо народи ходитимуть кожен ім'ям свого бога, а ми будем ходити Ім'ям Господа, нашого Бога, на віки віків!
\end{tcolorbox}
\begin{tcolorbox}
\textsubscript{6} Того дня промовляє Господь позбираю кульгаве й згромаджу розігнане, і те, що на нього навів коли лихо.
\end{tcolorbox}
\begin{tcolorbox}
\textsubscript{7} І зроблю Я кульгаве останком, а віддалене потужним народом, і зацарює над ними Господь на Сіонській горі відтепер й аж навіки!
\end{tcolorbox}
\begin{tcolorbox}
\textsubscript{8} А ти, башто Черідна, підгірку Сіонської доньки, прийде до тебе і дійде старе панування, царювання для донечки Єрусалиму.
\end{tcolorbox}
\begin{tcolorbox}
\textsubscript{9} Тепер нащо здіймаєш ти окрик? Чи в тебе немає царя? Чи ж загинув твій радник, що ти корчишся, мов породілля?
\end{tcolorbox}
\begin{tcolorbox}
\textsubscript{10} Вийся та корчся, о дочко Сіону, немов породілля, бо тепер вийдеш із міста та перебуватимеш у полі, і прийдеш аж до Вавилону. Та будеш ти там урятована, там Господь тебе викупить з рук твоїх ворогів!
\end{tcolorbox}
\begin{tcolorbox}
\textsubscript{11} А зараз зібрались на тебе численні народи, говорячи: Нехай він зневажений буде, і нехай наше око побачить нещастя Сіону!
\end{tcolorbox}
\begin{tcolorbox}
\textsubscript{12} Та не знають вони Господніх думок, і не розуміють поради Його, бо Він їх позбирав, як до клуні снопи.
\end{tcolorbox}
\begin{tcolorbox}
\textsubscript{13} Ставай та молоти, дочко Сіону, бо Я ріг твій залізом учиню, а копита твої вчиню міддю, і ти розпорошиш численні народи та вчиниш закляттям для Господа несправедливий їхній зиск, а їхнє багатство Владиці всієї землі.
\end{tcolorbox}
\subsection{CHAPTER 5}
\begin{tcolorbox}
\textsubscript{1} (4-14) І згромаджуйсь тепер, дочко товпищ! Облогу вчинили на нас, тростиною б'ють по щоці Ізраїлевого суддю...
\end{tcolorbox}
\begin{tcolorbox}
\textsubscript{2} (5-1) А ти, Віфлеєме-Єфрате, хоч малий ти у тисячах Юди, із тебе Мені вийде Той, що буде Владика в Ізраїлі, і віддавна постання Його, від днів віковічних.
\end{tcolorbox}
\begin{tcolorbox}
\textsubscript{3} (5-2) Тому Він їх видасть до часу, аж поки ота не породить, що має родити, а останок братів Його вернеться до Ізраїлевих синів.
\end{tcolorbox}
\begin{tcolorbox}
\textsubscript{4} (5-3) І стане, і буде Він пасти Господньою силою, величністю Ймення Господа Бога Свого. І осядуть вони, бо Він стане великий тепер аж до кінців землі!
\end{tcolorbox}
\begin{tcolorbox}
\textsubscript{5} (5-4) І Він буде миром. Як прийде до нашого краю Ашшур, і буде топтатись по наших палатах, то поставимо на нього сім пастирів та восьмеро людських княжат.
\end{tcolorbox}
\begin{tcolorbox}
\textsubscript{6} (5-5) І вони будуть пасти мечем край Ашшура, край же Німрода у воротях його. Та Він від Ашшура врятує, як той прийде в наш Край, і коли буде топтатись по наших границях.
\end{tcolorbox}
\begin{tcolorbox}
\textsubscript{7} (5-6) І Яковів залишок буде посеред численних народів, як роса та від Господа, як той дощ на траві, і він надії не кластиме на чоловіка, і не буде надії складати на людських синів.
\end{tcolorbox}
\begin{tcolorbox}
\textsubscript{8} (5-7) І Яковів залишок буде між людами, серед численних народів, як лев між лісною худобою, як левчук між отарами овець, що як він переходить, то топче й шматує, і немає нікого, хто б зміг урятувати.
\end{tcolorbox}
\begin{tcolorbox}
\textsubscript{9} (5-8) Хай зведеться рука твоя на твоїх ненависників, і хай всі вороги твої витяті будуть!
\end{tcolorbox}
\begin{tcolorbox}
\textsubscript{10} (5-9) І станеться в день той, говорить Господь, і витну Я коні твої з-серед тебе, і колесниці твої повигублюю.
\end{tcolorbox}
\begin{tcolorbox}
\textsubscript{11} (5-10) І понищу міста твого Краю, і всі твердині твої порозвалюю.
\end{tcolorbox}
\begin{tcolorbox}
\textsubscript{12} (5-11) І повиполюю чари з твоєї руки, і ворожбитів у тебе не буде.
\end{tcolorbox}
\begin{tcolorbox}
\textsubscript{13} (5-12) І понищу боввани твої та жертовні стовпи твої з-посеред тебе, і ти чинові рук своїх більше не будеш вклонятися.
\end{tcolorbox}
\begin{tcolorbox}
\textsubscript{14} (5-13) І повитинаю дерева жертовні твої з-серед тебе, і міста твої вигублю.
\end{tcolorbox}
\begin{tcolorbox}
\textsubscript{15} (5-14) І в гніві та в лютості помсту вчиню над народами, що Мене не послухались!
\end{tcolorbox}
\subsection{CHAPTER 6}
\begin{tcolorbox}
\textsubscript{1} Послухайте, що промовляє Господь: Устань, сперечайсь перед горами, і хай узгір'я почують твій голос!
\end{tcolorbox}
\begin{tcolorbox}
\textsubscript{2} Послухайте, гори, Господнього суду, і візьміть до вух, ви, основи землі, бо в Господа пря із народом Своїм, і з Ізраїлем буде судитися Він!
\end{tcolorbox}
\begin{tcolorbox}
\textsubscript{3} Народе ти Мій, що тобі Я зробив і чим мучив тебе, свідчи на Мене!
\end{tcolorbox}
\begin{tcolorbox}
\textsubscript{4} Бо Я з краю єгипетського тебе вивів, і тебе викупив з дому рабів, і перед тобою послав Я Мойсея, Аарона та Маріям.
\end{tcolorbox}
\begin{tcolorbox}
\textsubscript{5} Мій народе, згадай, що Балак, цар моавський, задумував був, і що йому відповів Валаам, син Беорів, від Шіттіму аж по Ґілґал, щоб пізнати тобі справедливості Господа.
\end{tcolorbox}
\begin{tcolorbox}
\textsubscript{6} З чим піду перед Господа, схилюсь перед Богом Високости? Чи піду перед Нього з цілопаленнями, з річними телятами?
\end{tcolorbox}
\begin{tcolorbox}
\textsubscript{7} Чи Господь уподобає тисячі баранів, десятитисячки потоків оливи? Чи дам за свій гріх свого первенця, плід утроби моєї за гріх моєї душі?
\end{tcolorbox}
\begin{tcolorbox}
\textsubscript{8} Було тобі виявлено, о людино, що добре, і чого пожадає від тебе Господь, нічого, а тільки чинити правосуддя, і милосердя любити, і з твоїм Богом ходити сумирно.
\end{tcolorbox}
\begin{tcolorbox}
\textsubscript{9} Голос Господній кличе до міста, а хто мудрий, боїться той Ймення Твого: Послухайте жезла й Того, Хто призначив його:
\end{tcolorbox}
\begin{tcolorbox}
\textsubscript{10} Чи є ще у домі безбожного коштовності несправедливі, та неповна й неправна ефа?
\end{tcolorbox}
\begin{tcolorbox}
\textsubscript{11} Чи Я всправедливлю вагу неправдиву, і калитку з важками обманними?
\end{tcolorbox}
\begin{tcolorbox}
\textsubscript{12} Що повні насильства його багачі, мешканці ж його говорили неправду, а їхній язик в їхніх устах омана,
\end{tcolorbox}
\begin{tcolorbox}
\textsubscript{13} то теж бити зачну Я тебе, пустошити тебе за гріхи твої.
\end{tcolorbox}
\begin{tcolorbox}
\textsubscript{14} Будеш ти їсти, але не наситишся, і буде голод у нутрі твоїм, і станеш ховати, але не врятуєш, а що ти врятуєш мечеві віддам.
\end{tcolorbox}
\begin{tcolorbox}
\textsubscript{15} Ти сіяти будеш, але не пожнеш, ти будеш оливку топтати, та не будеш маститись оливою, і молодий виноград, та вина ти не питимеш!
\end{tcolorbox}
\begin{tcolorbox}
\textsubscript{16} Бо ще переховуються всі устави Омрі та всі вчинки дому Ахава, і за їхніми радами ходите ви, тому то Я видам тебе на спустошення, і на посміх мешканців його, і ви ганьбу народу Мого понесете!
\end{tcolorbox}
\subsection{CHAPTER 7}
\begin{tcolorbox}
\textsubscript{1} Горе мені, бо я став, мов недобірки літні, як залишки по винобранні; нема грона на їжу, немає доспілої фіґи, якої жадає душа моя!
\end{tcolorbox}
\begin{tcolorbox}
\textsubscript{2} Згинув побожний з землі, і нема поміж людьми правдивого. Вони всі чатують на кров, один одного ловлять у сітку.
\end{tcolorbox}
\begin{tcolorbox}
\textsubscript{3} Наставлені руки на зло, щоб вправно чинити його, начальник жадає дарунків, суддя ж судить за плату, а великий говорить жадання своєї душі, і викривлюють все.
\end{tcolorbox}
\begin{tcolorbox}
\textsubscript{4} Найліпший із них як будяк, найправдивіший гірший від терену. Настає день Твоїх сторожів, Твоїх відвідин, тепер буде збентеження їхнє!
\end{tcolorbox}
\begin{tcolorbox}
\textsubscript{5} І другові не довіряйте, не надійтесь на приятеля, від тієї, що при лоні твоєму лежить, пильнуй двері уст своїх!
\end{tcolorbox}
\begin{tcolorbox}
\textsubscript{6} Бо гордує син батьком своїм, дочка повстає проти неньки своєї, невістка проти свекрухи своєї, вороги чоловіку домашні його!
\end{tcolorbox}
\begin{tcolorbox}
\textsubscript{7} А я виглядаю на Господа, надіюсь на Бога спасіння мого, Бог мій почує мене!
\end{tcolorbox}
\begin{tcolorbox}
\textsubscript{8} Не тішся, моя супротивнице, з мене, хоч я впала, Сіонська дочка, проте встану, хоч сиджу в темноті, та Господь мені світло!
\end{tcolorbox}
\begin{tcolorbox}
\textsubscript{9} Буду зносити я гнів Господній, бо згрішила Йому, аж поки не вирішить справи моєї, та суду не вчинить мені. Він на світло мене попровадить, побачу Його справедливість!
\end{tcolorbox}
\begin{tcolorbox}
\textsubscript{10} І побачить оце все моя супротивниця, і сором покриє її, бо казала мені: Де Він, Господь, Бог твій? Приглядатимуться мої очі до неї, її топчуть тепер, як болото на вулицях.
\end{tcolorbox}
\begin{tcolorbox}
\textsubscript{11} Настане той день, щоб мури твої будувати, тоді віддалиться границя твоя цього дня!
\end{tcolorbox}
\begin{tcolorbox}
\textsubscript{12} Це той день, коли прийдуть до тебе з Асирії та аж до Єгипту, і від Єгипту та аж до Ріки, і від моря до моря, і від гори до гори.
\end{tcolorbox}
\begin{tcolorbox}
\textsubscript{13} І спустошенням стане земля на мешканців її, через плід їхніх учинків.
\end{tcolorbox}
\begin{tcolorbox}
\textsubscript{14} Паси Мій народ своїм берлом, отару спадку Твого; що пробуває в лісі самотно, у середині саду, хай пасуться вони на Башані й Ґілеаді, як за днів стародавніх.
\end{tcolorbox}
\begin{tcolorbox}
\textsubscript{15} Як за днів твого виходу з краю єгипетського, покажу йому чуда.
\end{tcolorbox}
\begin{tcolorbox}
\textsubscript{16} Народи побачать оце, і посоромлені будуть при всій своїй силі, руку покладуть на уста, їхні вуха оглухнуть.
\end{tcolorbox}
\begin{tcolorbox}
\textsubscript{17} Будуть порох лизати вони, як той гад, як плазюче землі, повилазять з дрижанням з укріплень своїх, вони будуть тремтіти перед Господом, Богом нашим, і будуть боятись Тебе!
\end{tcolorbox}
\begin{tcolorbox}
\textsubscript{18} Хто Бог інший, як Ти, що прощає провину і пробачує прогріх останку спадку Свого, Свого гніву не держить назавжди, бо кохається в милості?
\end{tcolorbox}
\begin{tcolorbox}
\textsubscript{19} Знов над нами Він змилується, наші провини потопче, Ти кинеш у морську глибочінь усі наші гріхи.
\end{tcolorbox}
\begin{tcolorbox}
\textsubscript{20} Ти даси правду Яковові, Авраамові милість, яку присягнув Він для наших батьків від днів стародавніх.
\end{tcolorbox}
