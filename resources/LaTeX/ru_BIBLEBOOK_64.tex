\section{BOOK 63}
\subsection{CHAPTER 1}
\begin{tcolorbox}
\textsubscript{1} Да лобзает он меня лобзанием уст своих! Ибо ласки твои лучше вина.
\end{tcolorbox}
\begin{tcolorbox}
\textsubscript{2} От благовония мастей твоих имя твое--как разлитое миро; поэтому девицы любят тебя.
\end{tcolorbox}
\begin{tcolorbox}
\textsubscript{3} Влеки меня, мы побежим за тобою; --царь ввел меня в чертоги свои, --будем восхищаться и радоваться тобою, превозносить ласки твои больше, нежели вино; достойно любят тебя!
\end{tcolorbox}
\begin{tcolorbox}
\textsubscript{4} Дщери Иерусалимские! черна я, но красива, как шатры Кидарские, как завесы Соломоновы.
\end{tcolorbox}
\begin{tcolorbox}
\textsubscript{5} Не смотрите на меня, что я смугла, ибо солнце опалило меня: сыновья матери моей разгневались на меня, поставили меня стеречь виноградники, --моего собственного виноградника я не стерегла.
\end{tcolorbox}
\begin{tcolorbox}
\textsubscript{6} Скажи мне, ты, которого любит душа моя: где пасешь ты? где отдыхаешь в полдень? к чему мне быть скиталицею возле стад товарищей твоих?
\end{tcolorbox}
\begin{tcolorbox}
\textsubscript{7} Если ты не знаешь этого, прекраснейшая из женщин, то иди себе по следам овец и паси козлят твоих подле шатров пастушеских.
\end{tcolorbox}
\begin{tcolorbox}
\textsubscript{8} Кобылице моей в колеснице фараоновой я уподобил тебя, возлюбленная моя.
\end{tcolorbox}
\begin{tcolorbox}
\textsubscript{9} Прекрасны ланиты твои под подвесками, шея твоя в ожерельях;
\end{tcolorbox}
\begin{tcolorbox}
\textsubscript{10} золотые подвески мы сделаем тебе с серебряными блестками.
\end{tcolorbox}
\begin{tcolorbox}
\textsubscript{11} Доколе царь был за столом своим, нард мой издавал благовоние свое.
\end{tcolorbox}
\begin{tcolorbox}
\textsubscript{12} Мирровый пучок--возлюбленный мой у меня, у грудей моих пребывает.
\end{tcolorbox}
\begin{tcolorbox}
\textsubscript{13} Как кисть кипера, возлюбленный мой у меня в виноградниках Енгедских.
\end{tcolorbox}
\begin{tcolorbox}
\textsubscript{14} О, ты прекрасна, возлюбленная моя, ты прекрасна! глаза твои голубиные.
\end{tcolorbox}
\begin{tcolorbox}
\textsubscript{15} О, ты прекрасен, возлюбленный мой, и любезен! и ложе у нас--зелень;
\end{tcolorbox}
\begin{tcolorbox}
\textsubscript{16} кровли домов наших--кедры,
\end{tcolorbox}
\begin{tcolorbox}
\textsubscript{17} потолки наши--кипарисы.
\end{tcolorbox}
\subsection{CHAPTER 2}
\begin{tcolorbox}
\textsubscript{1} Я нарцисс Саронский, лилия долин!
\end{tcolorbox}
\begin{tcolorbox}
\textsubscript{2} Что лилия между тернами, то возлюбленная моя между девицами.
\end{tcolorbox}
\begin{tcolorbox}
\textsubscript{3} Что яблоня между лесными деревьями, то возлюбленный мой между юношами. В тени ее люблю я сидеть, и плоды ее сладки для гортани моей.
\end{tcolorbox}
\begin{tcolorbox}
\textsubscript{4} Он ввел меня в дом пира, и знамя его надо мною--любовь.
\end{tcolorbox}
\begin{tcolorbox}
\textsubscript{5} Подкрепите меня вином, освежите меня яблоками, ибо я изнемогаю от любви.
\end{tcolorbox}
\begin{tcolorbox}
\textsubscript{6} Левая рука его у меня под головою, а правая обнимает меня.
\end{tcolorbox}
\begin{tcolorbox}
\textsubscript{7} Заклинаю вас, дщери Иерусалимские, сернами или полевыми ланями: не будите и не тревожьте возлюбленной, доколе ей угодно.
\end{tcolorbox}
\begin{tcolorbox}
\textsubscript{8} Голос возлюбленного моего! вот, он идет, скачет по горам, прыгает по холмам.
\end{tcolorbox}
\begin{tcolorbox}
\textsubscript{9} Друг мой похож на серну или на молодого оленя. Вот, он стоит у нас за стеною, заглядывает в окно, мелькает сквозь решетку.
\end{tcolorbox}
\begin{tcolorbox}
\textsubscript{10} Возлюбленный мой начал говорить мне: встань, возлюбленная моя, прекрасная моя, выйди!
\end{tcolorbox}
\begin{tcolorbox}
\textsubscript{11} Вот, зима уже прошла; дождь миновал, перестал;
\end{tcolorbox}
\begin{tcolorbox}
\textsubscript{12} цветы показались на земле; время пения настало, и голос горлицы слышен в стране нашей;
\end{tcolorbox}
\begin{tcolorbox}
\textsubscript{13} смоковницы распустили свои почки, и виноградные лозы, расцветая, издают благовоние. Встань, возлюбленная моя, прекрасная моя, выйди!
\end{tcolorbox}
\begin{tcolorbox}
\textsubscript{14} Голубица моя в ущелье скалы под кровом утеса! покажи мне лице твое, дай мне услышать голос твой, потому что голос твой сладок и лице твое приятно.
\end{tcolorbox}
\begin{tcolorbox}
\textsubscript{15} Ловите нам лисиц, лисенят, которые портят виноградники, а виноградники наши в цвете.
\end{tcolorbox}
\begin{tcolorbox}
\textsubscript{16} Возлюбленный мой принадлежит мне, а я ему; он пасет между лилиями.
\end{tcolorbox}
\begin{tcolorbox}
\textsubscript{17} Доколе день дышит [прохладою], и убегают тени, возвратись, будь подобен серне или молодому оленю на расселинах гор.
\end{tcolorbox}
\subsection{CHAPTER 3}
\begin{tcolorbox}
\textsubscript{1} На ложе моем ночью искала я того, которого любит душа моя, искала его и не нашла его.
\end{tcolorbox}
\begin{tcolorbox}
\textsubscript{2} Встану же я, пойду по городу, по улицам и площадям, и буду искать того, которого любит душа моя; искала я его и не нашла его.
\end{tcolorbox}
\begin{tcolorbox}
\textsubscript{3} Встретили меня стражи, обходящие город: 'не видали ли вы того, которого любит душа моя?'
\end{tcolorbox}
\begin{tcolorbox}
\textsubscript{4} Но едва я отошла от них, как нашла того, которого любит душа моя, ухватилась за него, и не отпустила его, доколе не привела его в дом матери моей и во внутренние комнаты родительницы моей.
\end{tcolorbox}
\begin{tcolorbox}
\textsubscript{5} Заклинаю вас, дщери Иерусалимские, сернами или полевыми ланями: не будите и не тревожьте возлюбленной, доколе ей угодно.
\end{tcolorbox}
\begin{tcolorbox}
\textsubscript{6} Кто эта, восходящая от пустыни как бы столбы дыма, окуриваемая миррою и фимиамом, всякими порошками мироварника?
\end{tcolorbox}
\begin{tcolorbox}
\textsubscript{7} Вот одр его--Соломона: шестьдесят сильных вокруг него, из сильных Израилевых.
\end{tcolorbox}
\begin{tcolorbox}
\textsubscript{8} Все они держат по мечу, опытны в бою; у каждого меч при бедре его ради страха ночного.
\end{tcolorbox}
\begin{tcolorbox}
\textsubscript{9} Носильный одр сделал себе царь Соломон из дерев Ливанских;
\end{tcolorbox}
\begin{tcolorbox}
\textsubscript{10} столпцы его сделал из серебра, локотники его из золота, седалище его из пурпуровой ткани; внутренность его убрана с любовью дщерями Иерусалимскими.
\end{tcolorbox}
\begin{tcolorbox}
\textsubscript{11} Пойдите и посмотрите, дщери Сионские, на царя Соломона в венце, которым увенчала его мать его в день бракосочетания его, в день, радостный для сердца его.
\end{tcolorbox}
\subsection{CHAPTER 4}
\begin{tcolorbox}
\textsubscript{1} О, ты прекрасна, возлюбленная моя, ты прекрасна! глаза твои голубиные под кудрями твоими; волосы твои--как стадо коз, сходящих с горы Галаадской;
\end{tcolorbox}
\begin{tcolorbox}
\textsubscript{2} зубы твои--как стадо выстриженных овец, выходящих из купальни, из которых у каждой пара ягнят, и бесплодной нет между ними;
\end{tcolorbox}
\begin{tcolorbox}
\textsubscript{3} как лента алая губы твои, и уста твои любезны; как половинки гранатового яблока--ланиты твои под кудрями твоими;
\end{tcolorbox}
\begin{tcolorbox}
\textsubscript{4} шея твоя--как столп Давидов, сооруженный для оружий, тысяча щитов висит на нем--все щиты сильных;
\end{tcolorbox}
\begin{tcolorbox}
\textsubscript{5} два сосца твои--как двойни молодой серны, пасущиеся между лилиями.
\end{tcolorbox}
\begin{tcolorbox}
\textsubscript{6} Доколе день дышит [прохладою], и убегают тени, пойду я на гору мирровую и на холм фимиама.
\end{tcolorbox}
\begin{tcolorbox}
\textsubscript{7} Вся ты прекрасна, возлюбленная моя, и пятна нет на тебе!
\end{tcolorbox}
\begin{tcolorbox}
\textsubscript{8} Со мною с Ливана, невеста! со мною иди с Ливана! спеши с вершины Аманы, с вершины Сенира и Ермона, от логовищ львиных, от гор барсовых!
\end{tcolorbox}
\begin{tcolorbox}
\textsubscript{9} Пленила ты сердце мое, сестра моя, невеста! пленила ты сердце мое одним взглядом очей твоих, одним ожерельем на шее твоей.
\end{tcolorbox}
\begin{tcolorbox}
\textsubscript{10} О, как любезны ласки твои, сестра моя, невеста! о, как много ласки твои лучше вина, и благовоние мастей твоих лучше всех ароматов!
\end{tcolorbox}
\begin{tcolorbox}
\textsubscript{11} Сотовый мед каплет из уст твоих, невеста; мед и молоко под языком твоим, и благоухание одежды твоей подобно благоуханию Ливана!
\end{tcolorbox}
\begin{tcolorbox}
\textsubscript{12} Запертый сад--сестра моя, невеста, заключенный колодезь, запечатанный источник:
\end{tcolorbox}
\begin{tcolorbox}
\textsubscript{13} рассадники твои--сад с гранатовыми яблоками, с превосходными плодами, киперы с нардами,
\end{tcolorbox}
\begin{tcolorbox}
\textsubscript{14} нард и шафран, аир и корица со всякими благовонными деревами, мирра и алой со всякими лучшими ароматами;
\end{tcolorbox}
\begin{tcolorbox}
\textsubscript{15} садовый источник--колодезь живых вод и потоки с Ливана.
\end{tcolorbox}
\begin{tcolorbox}
\textsubscript{16} Поднимись [ветер] с севера и принесись с юга, повей на сад мой, --и польются ароматы его! --Пусть придет возлюбленный мой в сад свой и вкушает сладкие плоды его.
\end{tcolorbox}
\subsection{CHAPTER 5}
\begin{tcolorbox}
\textsubscript{1} Пришел я в сад мой, сестра моя, невеста; набрал мирры моей с ароматами моими, поел сотов моих с медом моим, напился вина моего с молоком моим. Ешьте, друзья, пейте и насыщайтесь, возлюбленные!
\end{tcolorbox}
\begin{tcolorbox}
\textsubscript{2} Я сплю, а сердце мое бодрствует; [вот], голос моего возлюбленного, который стучится: 'отвори мне, сестра моя, возлюбленная моя, голубица моя, чистая моя! потому что голова моя вся покрыта росою, кудри мои--ночною влагою'.
\end{tcolorbox}
\begin{tcolorbox}
\textsubscript{3} Я скинула хитон мой; как же мне опять надевать его? Я вымыла ноги мои; как же мне марать их?
\end{tcolorbox}
\begin{tcolorbox}
\textsubscript{4} Возлюбленный мой протянул руку свою сквозь скважину, и внутренность моя взволновалась от него.
\end{tcolorbox}
\begin{tcolorbox}
\textsubscript{5} Я встала, чтобы отпереть возлюбленному моему, и с рук моих капала мирра, и с перстов моих мирра капала на ручки замка.
\end{tcolorbox}
\begin{tcolorbox}
\textsubscript{6} Отперла я возлюбленному моему, а возлюбленный мой повернулся и ушел. Души во мне не стало, когда он говорил; я искала его и не находила его; звала его, и он не отзывался мне.
\end{tcolorbox}
\begin{tcolorbox}
\textsubscript{7} Встретили меня стражи, обходящие город, избили меня, изранили меня; сняли с меня покрывало стерегущие стены.
\end{tcolorbox}
\begin{tcolorbox}
\textsubscript{8} Заклинаю вас, дщери Иерусалимские: если вы встретите возлюбленного моего, что скажете вы ему? что я изнемогаю от любви.
\end{tcolorbox}
\begin{tcolorbox}
\textsubscript{9} 'Чем возлюбленный твой лучше других возлюбленных, прекраснейшая из женщин? Чем возлюбленный твой лучше других, что ты так заклинаешь нас?'
\end{tcolorbox}
\begin{tcolorbox}
\textsubscript{10} Возлюбленный мой бел и румян, лучше десяти тысяч других:
\end{tcolorbox}
\begin{tcolorbox}
\textsubscript{11} голова его--чистое золото; кудри его волнистые, черные, как ворон;
\end{tcolorbox}
\begin{tcolorbox}
\textsubscript{12} глаза его--как голуби при потоках вод, купающиеся в молоке, сидящие в довольстве;
\end{tcolorbox}
\begin{tcolorbox}
\textsubscript{13} щеки его--цветник ароматный, гряды благовонных растений; губы его--лилии, источают текучую мирру;
\end{tcolorbox}
\begin{tcolorbox}
\textsubscript{14} руки его--золотые кругляки, усаженные топазами; живот его--как изваяние из слоновой кости, обложенное сапфирами;
\end{tcolorbox}
\begin{tcolorbox}
\textsubscript{15} голени его--мраморные столбы, поставленные на золотых подножиях; вид его подобен Ливану, величествен, как кедры;
\end{tcolorbox}
\begin{tcolorbox}
\textsubscript{16} уста его--сладость, и весь он--любезность. Вот кто возлюбленный мой, и вот кто друг мой, дщери Иерусалимские!
\end{tcolorbox}
\subsection{CHAPTER 6}
\begin{tcolorbox}
\textsubscript{1} 'Куда пошел возлюбленный твой, прекраснейшая из женщин? куда обратился возлюбленный твой? мы поищем его с тобою'.
\end{tcolorbox}
\begin{tcolorbox}
\textsubscript{2} Мой возлюбленный пошел в сад свой, в цветники ароматные, чтобы пасти в садах и собирать лилии.
\end{tcolorbox}
\begin{tcolorbox}
\textsubscript{3} Я принадлежу возлюбленному моему, а возлюбленный мой--мне; он пасет между лилиями.
\end{tcolorbox}
\begin{tcolorbox}
\textsubscript{4} Прекрасна ты, возлюбленная моя, как Фирца, любезна, как Иерусалим, грозна, как полки со знаменами.
\end{tcolorbox}
\begin{tcolorbox}
\textsubscript{5} Уклони очи твои от меня, потому что они волнуют меня.
\end{tcolorbox}
\begin{tcolorbox}
\textsubscript{6} Волосы твои--как стадо коз, сходящих с Галаада; зубы твои--как стадо овец, выходящих из купальни, из которых у каждой пара ягнят, и бесплодной нет между ними;
\end{tcolorbox}
\begin{tcolorbox}
\textsubscript{7} как половинки гранатового яблока--ланиты твои под кудрями твоими.
\end{tcolorbox}
\begin{tcolorbox}
\textsubscript{8} Есть шестьдесят цариц и восемьдесят наложниц и девиц без числа,
\end{tcolorbox}
\begin{tcolorbox}
\textsubscript{9} но единственная--она, голубица моя, чистая моя; единственная она у матери своей, отличенная у родительницы своей. Увидели ее девицы, и--превознесли ее, царицы и наложницы, и--восхвалили ее.
\end{tcolorbox}
\begin{tcolorbox}
\textsubscript{10} Кто эта, блистающая, как заря, прекрасная, как луна, светлая, как солнце, грозная, как полки со знаменами?
\end{tcolorbox}
\begin{tcolorbox}
\textsubscript{11} Я сошла в ореховый сад посмотреть на зелень долины, поглядеть, распустилась ли виноградная лоза, расцвели ли гранатовые яблоки?
\end{tcolorbox}
\begin{tcolorbox}
\textsubscript{12} Не знаю, как душа моя влекла меня к колесницам знатных народа моего.
\end{tcolorbox}
\begin{tcolorbox}
\textsubscript{13} (7-1) 'Оглянись, оглянись, Суламита! оглянись, оглянись, --и мы посмотрим на тебя'. Что вам смотреть на Суламиту, как на хоровод Манаимский?
\end{tcolorbox}
\subsection{CHAPTER 7}
\begin{tcolorbox}
\textsubscript{1} (7-2) О, как прекрасны ноги твои в сандалиях, дщерь именитая! Округление бедр твоих, как ожерелье, дело рук искусного художника;
\end{tcolorbox}
\begin{tcolorbox}
\textsubscript{2} (7-3) живот твой--круглая чаша, [в которой] не истощается ароматное вино; чрево твое--ворох пшеницы, обставленный лилиями;
\end{tcolorbox}
\begin{tcolorbox}
\textsubscript{3} (7-4) два сосца твои--как два козленка, двойни серны;
\end{tcolorbox}
\begin{tcolorbox}
\textsubscript{4} (7-5) шея твоя--как столп из слоновой кости; глаза твои--озерки Есевонские, что у ворот Батраббима; нос твой--башня Ливанская, обращенная к Дамаску;
\end{tcolorbox}
\begin{tcolorbox}
\textsubscript{5} (7-6) голова твоя на тебе, как Кармил, и волосы на голове твоей, как пурпур; царь увлечен [твоими] кудрями.
\end{tcolorbox}
\begin{tcolorbox}
\textsubscript{6} (7-7) Как ты прекрасна, как привлекательна, возлюбленная, твоею миловидностью!
\end{tcolorbox}
\begin{tcolorbox}
\textsubscript{7} (7-8) Этот стан твой похож на пальму, и груди твои на виноградные кисти.
\end{tcolorbox}
\begin{tcolorbox}
\textsubscript{8} (7-9) Подумал я: влез бы я на пальму, ухватился бы за ветви ее; и груди твои были бы вместо кистей винограда, и запах от ноздрей твоих, как от яблоков;
\end{tcolorbox}
\begin{tcolorbox}
\textsubscript{9} (7-10) уста твои--как отличное вино. Оно течет прямо к другу моему, услаждает уста утомленных.
\end{tcolorbox}
\begin{tcolorbox}
\textsubscript{10} (7-11) Я принадлежу другу моему, и ко мне [обращено] желание его.
\end{tcolorbox}
\begin{tcolorbox}
\textsubscript{11} (7-12) Приди, возлюбленный мой, выйдем в поле, побудем в селах;
\end{tcolorbox}
\begin{tcolorbox}
\textsubscript{12} (7-13) поутру пойдем в виноградники, посмотрим, распустилась ли виноградная лоза, раскрылись ли почки, расцвели ли гранатовые яблоки; там я окажу ласки мои тебе.
\end{tcolorbox}
\begin{tcolorbox}
\textsubscript{13} (7-14) Мандрагоры уже пустили благовоние, и у дверей наших всякие превосходные плоды, новые и старые: [это] сберегла я для тебя, мой возлюбленный!
\end{tcolorbox}
\subsection{CHAPTER 8}
\begin{tcolorbox}
\textsubscript{1} О, если бы ты был мне брат, сосавший груди матери моей! тогда я, встретив тебя на улице, целовала бы тебя, и меня не осуждали бы.
\end{tcolorbox}
\begin{tcolorbox}
\textsubscript{2} Повела бы я тебя, привела бы тебя в дом матери моей. Ты учил бы меня, а я поила бы тебя ароматным вином, соком гранатовых яблоков моих.
\end{tcolorbox}
\begin{tcolorbox}
\textsubscript{3} Левая рука его у меня под головою, а правая обнимает меня.
\end{tcolorbox}
\begin{tcolorbox}
\textsubscript{4} Заклинаю вас, дщери Иерусалимские, --не будите и не тревожьте возлюбленной, доколе ей угодно.
\end{tcolorbox}
\begin{tcolorbox}
\textsubscript{5} Кто это восходит от пустыни, опираясь на своего возлюбленного? Под яблоней разбудила я тебя: там родила тебя мать твоя, там родила тебя родительница твоя.
\end{tcolorbox}
\begin{tcolorbox}
\textsubscript{6} Положи меня, как печать, на сердце твое, как перстень, на руку твою: ибо крепка, как смерть, любовь; люта, как преисподняя, ревность; стрелы ее--стрелы огненные; она пламень весьма сильный.
\end{tcolorbox}
\begin{tcolorbox}
\textsubscript{7} Большие воды не могут потушить любви, и реки не зальют ее. Если бы кто давал все богатство дома своего за любовь, то он был бы отвергнут с презреньем.
\end{tcolorbox}
\begin{tcolorbox}
\textsubscript{8} Есть у нас сестра, которая еще мала, и сосцов нет у нее; что нам будет делать с сестрою нашею, когда будут свататься за нее?
\end{tcolorbox}
\begin{tcolorbox}
\textsubscript{9} Если бы она была стена, то мы построили бы на ней палаты из серебра; если бы она была дверь, то мы обложили бы ее кедровыми досками.
\end{tcolorbox}
\begin{tcolorbox}
\textsubscript{10} Я--стена, и сосцы у меня, как башни; потому я буду в глазах его, как достигшая полноты.
\end{tcolorbox}
\begin{tcolorbox}
\textsubscript{11} Виноградник был у Соломона в Ваал-Гамоне; он отдал этот виноградник сторожам; каждый должен был доставлять за плоды его тысячу сребренников.
\end{tcolorbox}
\begin{tcolorbox}
\textsubscript{12} А мой виноградник у меня при себе. Тысяча пусть тебе, Соломон, а двести--стерегущим плоды его.
\end{tcolorbox}
\begin{tcolorbox}
\textsubscript{13} Жительница садов! товарищи внимают голосу твоему, дай и мне послушать его.
\end{tcolorbox}
\begin{tcolorbox}
\textsubscript{14} Беги, возлюбленный мой; будь подобен серне или молодому оленю на горах бальзамических!
\end{tcolorbox}
