\section{BOOK 81}
\subsection{CHAPTER 1}
\begin{tcolorbox}
\textsubscript{1} Слово Господне, которое было к Осии, сыну Беериину, во дни Озии, Иоафама, Ахаза, Езекии, царей Иудейских, и во дни Иеровоама, сына Иоасова, царя Израильского.
\end{tcolorbox}
\begin{tcolorbox}
\textsubscript{2} Начало слова Господня к Осии. И сказал Господь Осии: иди, возьми себе жену блудницу и детей блуда; ибо сильно блудодействует земля сия, отступив от Господа.
\end{tcolorbox}
\begin{tcolorbox}
\textsubscript{3} И пошел он и взял Гомерь, дочь Дивлаима; и она зачала и родила ему сына.
\end{tcolorbox}
\begin{tcolorbox}
\textsubscript{4} И Господь сказал ему: нареки ему имя Изреель, потому что еще немного пройдет, и Я взыщу кровь Изрееля с дома Ииуева, и положу конец царству дома Израилева,
\end{tcolorbox}
\begin{tcolorbox}
\textsubscript{5} и будет в тот день, Я сокрушу лук Израилев в долине Изреель.
\end{tcolorbox}
\begin{tcolorbox}
\textsubscript{6} И зачала еще, и родила дочь, и Он сказал ему: нареки ей имя Лорухама; ибо Я уже не буду более миловать дома Израилева, чтобы прощать им.
\end{tcolorbox}
\begin{tcolorbox}
\textsubscript{7} А дом Иудин помилую и спасу их в Господе Боге их, спасу их ни луком, ни мечом, ни войною, ни конями и всадниками.
\end{tcolorbox}
\begin{tcolorbox}
\textsubscript{8} И, откормив грудью Непомилованную, она зачала, и родила сына.
\end{tcolorbox}
\begin{tcolorbox}
\textsubscript{9} И сказал Он: нареки ему имя Лоамми, потому что вы не Мой народ, и Я не буду вашим [Богом].
\end{tcolorbox}
\begin{tcolorbox}
\textsubscript{10} Но будет число сынов Израилевых как песок морской, которого нельзя ни измерить, ни исчислить; и там, где говорили им: 'вы не Мой народ', будут говорить им: 'вы сыны Бога живаго'.
\end{tcolorbox}
\begin{tcolorbox}
\textsubscript{11} И соберутся сыны Иудины и сыны Израилевы вместе, и поставят себе одну главу, и выйдут из земли [переселения]; ибо велик день Изрееля!
\end{tcolorbox}
\subsection{CHAPTER 2}
\begin{tcolorbox}
\textsubscript{1} Говорите братьям вашим: 'Мой народ', и сестрам вашим: 'Помилованная'.
\end{tcolorbox}
\begin{tcolorbox}
\textsubscript{2} Судитесь с вашею матерью, судитесь; ибо она не жена Моя, и Я не муж ее; пусть она удалит блуд от лица своего и прелюбодеяние от грудей своих,
\end{tcolorbox}
\begin{tcolorbox}
\textsubscript{3} дабы Я не разоблачил ее донага и не выставил ее, как в день рождения ее, не сделал ее пустынею, не обратил ее в землю сухую и не уморил ее жаждою.
\end{tcolorbox}
\begin{tcolorbox}
\textsubscript{4} И детей ее не помилую, потому что они дети блуда.
\end{tcolorbox}
\begin{tcolorbox}
\textsubscript{5} Ибо блудодействовала мать их и осрамила себя зачавшая их; ибо говорила: 'пойду за любовниками моими, которые дают мне хлеб и воду, шерсть и лен, елей и напитки'.
\end{tcolorbox}
\begin{tcolorbox}
\textsubscript{6} За то вот, Я загорожу путь ее тернами и обнесу ее оградою, и она не найдет стезей своих,
\end{tcolorbox}
\begin{tcolorbox}
\textsubscript{7} и погонится за любовниками своими, но не догонит их, и будет искать их, но не найдет, и скажет: 'пойду я, и возвращусь к первому мужу моему; ибо тогда лучше было мне, нежели теперь'.
\end{tcolorbox}
\begin{tcolorbox}
\textsubscript{8} А не знала она, что Я, Я давал ей хлеб и вино и елей и умножил у нее серебро и золото, из которого сделали [истукана] Ваала.
\end{tcolorbox}
\begin{tcolorbox}
\textsubscript{9} За то Я возьму назад хлеб Мой в его время и вино Мое в его пору и отниму шерсть и лен Мой, чем покрывается нагота ее.
\end{tcolorbox}
\begin{tcolorbox}
\textsubscript{10} И ныне открою срамоту ее пред глазами любовников ее, и никто не исторгнет ее из руки Моей.
\end{tcolorbox}
\begin{tcolorbox}
\textsubscript{11} И прекращу у нее всякое веселье, праздники ее и новомесячия ее, и субботы ее, и все торжества ее.
\end{tcolorbox}
\begin{tcolorbox}
\textsubscript{12} И опустошу виноградные лозы ее и смоковницы ее, о которых она говорит: 'это у меня подарки, которые надарили мне любовники мои'; и Я превращу их в лес, и полевые звери поедят их.
\end{tcolorbox}
\begin{tcolorbox}
\textsubscript{13} И накажу ее за дни служения Ваалам, когда она кадила им и, украсив себя серьгами и ожерельями, ходила за любовниками своими, а Меня забывала, говорит Господь.
\end{tcolorbox}
\begin{tcolorbox}
\textsubscript{14} Посему вот, и Я увлеку ее, приведу ее в пустыню, и буду говорить к сердцу ее.
\end{tcolorbox}
\begin{tcolorbox}
\textsubscript{15} И дам ей оттуда виноградники ее и долину Ахор, в преддверие надежды; и она будет петь там, как во дни юности своей и как в день выхода своего из земли Египетской.
\end{tcolorbox}
\begin{tcolorbox}
\textsubscript{16} И будет в тот день, говорит Господь, ты будешь звать Меня: 'муж мой', и не будешь более звать Меня: 'Ваали'.
\end{tcolorbox}
\begin{tcolorbox}
\textsubscript{17} И удалю имена Ваалов от уст ее, и не будут более вспоминаемы имена их.
\end{tcolorbox}
\begin{tcolorbox}
\textsubscript{18} И заключу в то время для них союз с полевыми зверями и с птицами небесными, и с пресмыкающимися по земле; и лук, и меч, и войну истреблю от земли той, и дам им жить в безопасности.
\end{tcolorbox}
\begin{tcolorbox}
\textsubscript{19} И обручу тебя Мне навек, и обручу тебя Мне в правде и суде, в благости и милосердии.
\end{tcolorbox}
\begin{tcolorbox}
\textsubscript{20} И обручу тебя Мне в верности, и ты познаешь Господа.
\end{tcolorbox}
\begin{tcolorbox}
\textsubscript{21} И будет в тот день, Я услышу, говорит Господь, услышу небо, и оно услышит землю,
\end{tcolorbox}
\begin{tcolorbox}
\textsubscript{22} и земля услышит хлеб и вино и елей; а сии услышат Изреель.
\end{tcolorbox}
\begin{tcolorbox}
\textsubscript{23} И посею ее для Себя на земле, и помилую Непомилованную, и скажу не Моему народу: 'ты Мой народ', а он скажет: 'Ты мой Бог!'
\end{tcolorbox}
\subsection{CHAPTER 3}
\begin{tcolorbox}
\textsubscript{1} И сказал мне Господь: иди еще, и полюби женщину, любимую мужем, но прелюбодействующую, подобно тому, как любит Господь сынов Израилевых, а они обращаются к другим богам и любят виноградные лепешки их.
\end{tcolorbox}
\begin{tcolorbox}
\textsubscript{2} И приобрел я ее себе за пятнадцать сребренников и за хомер ячменя и полхомера ячменя
\end{tcolorbox}
\begin{tcolorbox}
\textsubscript{3} и сказал ей: много дней оставайся у меня; не блуди, и не будь с другим; так же и я буду для тебя.
\end{tcolorbox}
\begin{tcolorbox}
\textsubscript{4} Ибо долгое время сыны Израилевы будут оставаться без царя и без князя и без жертвы, без жертвенника, без ефода и терафима.
\end{tcolorbox}
\begin{tcolorbox}
\textsubscript{5} После того обратятся сыны Израилевы и взыщут Господа Бога своего и Давида, царя своего, и будут благоговеть пред Господом и благостью Его в последние дни.
\end{tcolorbox}
\subsection{CHAPTER 4}
\begin{tcolorbox}
\textsubscript{1} Слушайте слово Господне, сыны Израилевы; ибо суд у Господа с жителями сей земли, потому что нет ни истины, ни милосердия, ни Богопознания на земле.
\end{tcolorbox}
\begin{tcolorbox}
\textsubscript{2} Клятва и обман, убийство и воровство, и прелюбодейство крайне распространились, и кровопролитие следует за кровопролитием.
\end{tcolorbox}
\begin{tcolorbox}
\textsubscript{3} За то восплачет земля сия, и изнемогут все, живущие на ней, со зверями полевыми и птицами небесными, даже и рыбы морские погибнут.
\end{tcolorbox}
\begin{tcolorbox}
\textsubscript{4} Но никто не спорь, никто не обличай другого; и твой народ--как спорящие со священником.
\end{tcolorbox}
\begin{tcolorbox}
\textsubscript{5} И ты падешь днем, и пророк падет с тобою ночью, и истреблю матерь твою.
\end{tcolorbox}
\begin{tcolorbox}
\textsubscript{6} Истреблен будет народ Мой за недостаток ведения: так как ты отверг ведение, то и Я отвергну тебя от священнодействия предо Мною; и как ты забыл закон Бога твоего то и Я забуду детей твоих.
\end{tcolorbox}
\begin{tcolorbox}
\textsubscript{7} Чем больше они умножаются, тем больше грешат против Меня; славу их обращу в бесславие.
\end{tcolorbox}
\begin{tcolorbox}
\textsubscript{8} Грехами народа Моего кормятся они, и к беззаконию его стремится душа их.
\end{tcolorbox}
\begin{tcolorbox}
\textsubscript{9} И что будет с народом, то и со священником; и накажу его по путям его, и воздам ему по делам его.
\end{tcolorbox}
\begin{tcolorbox}
\textsubscript{10} Будут есть, и не насытятся; будут блудить, и не размножатся; ибо оставили служение Господу.
\end{tcolorbox}
\begin{tcolorbox}
\textsubscript{11} Блуд, вино и напитки завладели сердцем их.
\end{tcolorbox}
\begin{tcolorbox}
\textsubscript{12} Народ Мой вопрошает свое дерево и жезл его дает ему ответ; ибо дух блуда ввел их в заблуждение, и, блудодействуя, они отступили от Бога своего.
\end{tcolorbox}
\begin{tcolorbox}
\textsubscript{13} На вершинах гор они приносят жертвы и на холмах совершают каждение под дубом и тополем и теревинфом, потому что хороша от них тень; поэтому любодействуют дочери ваши и прелюбодействуют невестки ваши.
\end{tcolorbox}
\begin{tcolorbox}
\textsubscript{14} Я оставлю наказывать дочерей ваших, когда они блудодействуют, и невесток ваших, когда они прелюбодействуют, потому что вы сами на стороне блудниц и с любодейцами приносите жертвы, а невежественный народ гибнет.
\end{tcolorbox}
\begin{tcolorbox}
\textsubscript{15} Если ты, Израиль, блудодействуешь, то пусть не грешил бы Иуда; и не ходите в Галгал, и не восходите в Беф-Авен, и не клянитесь: 'жив Господь!'
\end{tcolorbox}
\begin{tcolorbox}
\textsubscript{16} Ибо как упрямая телица, упорен стал Израиль; посему будет ли теперь Господь пасти их, как агнцев на пространном пастбище?
\end{tcolorbox}
\begin{tcolorbox}
\textsubscript{17} Привязался к идолам Ефрем; оставь его!
\end{tcolorbox}
\begin{tcolorbox}
\textsubscript{18} Отвратительно пьянство их, совершенно предались блудодеянию; князья их любят постыдное.
\end{tcolorbox}
\begin{tcolorbox}
\textsubscript{19} Охватит их ветер своими крыльями, и устыдятся они жертв своих.
\end{tcolorbox}
\subsection{CHAPTER 5}
\begin{tcolorbox}
\textsubscript{1} Слушайте это, священники, и внимайте, дом Израилев, и приклоните ухо, дом царя; ибо вам будет суд, потому что вы были западнею в Массифе и сетью, раскинутою на Фаворе.
\end{tcolorbox}
\begin{tcolorbox}
\textsubscript{2} Глубоко погрязли они в распутстве; но Я накажу всех их.
\end{tcolorbox}
\begin{tcolorbox}
\textsubscript{3} Ефрема Я знаю, и Израиль не сокрыт от Меня; ибо ты блудодействуешь, Ефрем, и Израиль осквернился.
\end{tcolorbox}
\begin{tcolorbox}
\textsubscript{4} Дела их не допускают их обратиться к Богу своему, ибо дух блуда внутри них, и Господа они не познали.
\end{tcolorbox}
\begin{tcolorbox}
\textsubscript{5} И гордость Израиля унижена в глазах их; и Израиль и Ефрем падут от нечестия своего; падет и Иуда с ними.
\end{tcolorbox}
\begin{tcolorbox}
\textsubscript{6} С овцами своими и волами своими пойдут искать Господа и не найдут Его: Он удалился от них.
\end{tcolorbox}
\begin{tcolorbox}
\textsubscript{7} Господу они изменили, потому что родили чужих детей; ныне новый месяц поест их с их имуществом.
\end{tcolorbox}
\begin{tcolorbox}
\textsubscript{8} Вострубите рогом в Гиве, трубою в Раме; возглашайте в Беф-- Авене: 'за тобою, Вениамин!'
\end{tcolorbox}
\begin{tcolorbox}
\textsubscript{9} Ефрем сделается пустынею в день наказания; между коленами Израилевыми Я возвестил это.
\end{tcolorbox}
\begin{tcolorbox}
\textsubscript{10} Вожди Иудины стали подобны передвигающим межи: изолью на них гнев Мой, как воду.
\end{tcolorbox}
\begin{tcolorbox}
\textsubscript{11} Угнетен Ефрем, поражен судом; ибо захотел ходить вслед суетных.
\end{tcolorbox}
\begin{tcolorbox}
\textsubscript{12} И буду как моль для Ефрема и как червь для дома Иудина.
\end{tcolorbox}
\begin{tcolorbox}
\textsubscript{13} И увидел Ефрем болезнь свою, и Иуда--свою рану, и пошел Ефрем к Ассуру, и послал к царю Иареву; но он не может исцелить вас, и не излечит вас от раны.
\end{tcolorbox}
\begin{tcolorbox}
\textsubscript{14} Ибо Я как лев для Ефрема и как скимен для дома Иудина; Я, Я растерзаю, и уйду; унесу, и никто не спасет.
\end{tcolorbox}
\begin{tcolorbox}
\textsubscript{15} Пойду, возвращусь в Мое место, доколе они не признают себя виновными и не взыщут лица Моего.
\end{tcolorbox}
\subsection{CHAPTER 6}
\begin{tcolorbox}
\textsubscript{1} В скорби своей они с раннего утра будут искать Меня и говорить: 'пойдем и возвратимся к Господу! ибо Он уязвил--и Он исцелит нас, поразил--и перевяжет наши раны;
\end{tcolorbox}
\begin{tcolorbox}
\textsubscript{2} оживит нас через два дня, в третий день восставит нас, и мы будем жить пред лицем Его.
\end{tcolorbox}
\begin{tcolorbox}
\textsubscript{3} Итак познаем, будем стремиться познать Господа; как утренняя заря--явление Его, и Он придет к нам, как дождь, как поздний дождь оросит землю'.
\end{tcolorbox}
\begin{tcolorbox}
\textsubscript{4} Что сделаю тебе, Ефрем? что сделаю тебе, Иуда? благочестие ваше, как утренний туман и как роса, скоро исчезающая.
\end{tcolorbox}
\begin{tcolorbox}
\textsubscript{5} Посему Я поражал через пророков и бил их словами уст Моих, и суд Мой, как восходящий свет.
\end{tcolorbox}
\begin{tcolorbox}
\textsubscript{6} Ибо Я милости хочу, а не жертвы, и Боговедения более, нежели всесожжений.
\end{tcolorbox}
\begin{tcolorbox}
\textsubscript{7} Они же, подобно Адаму, нарушили завет и там изменили Мне.
\end{tcolorbox}
\begin{tcolorbox}
\textsubscript{8} Галаад--город нечестивцев, запятнанный кровью.
\end{tcolorbox}
\begin{tcolorbox}
\textsubscript{9} Как разбойники подстерегают человека, так сборище священников убивают на пути в Сихем и совершают мерзости.
\end{tcolorbox}
\begin{tcolorbox}
\textsubscript{10} В доме Израиля Я вижу ужасное; там блудодеяние у Ефрема, осквернился Израиль.
\end{tcolorbox}
\begin{tcolorbox}
\textsubscript{11} И тебе, Иуда, назначена жатва, когда Я возвращу плен народа Моего.
\end{tcolorbox}
\subsection{CHAPTER 7}
\begin{tcolorbox}
\textsubscript{1} Когда Я врачевал Израиля, открылась неправда Ефрема и злодейство Самарии: ибо они поступают лживо; и входит вор, и разбойник грабит по улицам.
\end{tcolorbox}
\begin{tcolorbox}
\textsubscript{2} Не помышляют они в сердце своем, что Я помню все злодеяния их; теперь окружают их дела их; они пред лицем Моим.
\end{tcolorbox}
\begin{tcolorbox}
\textsubscript{3} Злодейством своим они увеселяют царя и обманами своими--князей.
\end{tcolorbox}
\begin{tcolorbox}
\textsubscript{4} Все они пылают прелюбодейством, как печь, растопленная пекарем, который перестает поджигать ее, когда замесит тесто и оно вскиснет.
\end{tcolorbox}
\begin{tcolorbox}
\textsubscript{5} 'День нашего царя!' [говорят] князья, разгоряченные до болезни вином, а он протягивает руку свою к кощунам.
\end{tcolorbox}
\begin{tcolorbox}
\textsubscript{6} Ибо они коварством своим делают сердце свое подобным печи: пекарь их спит всю ночь, а утром она горит, как пылающий огонь.
\end{tcolorbox}
\begin{tcolorbox}
\textsubscript{7} Все они распалены, как печь, и пожирают судей своих; все цари их падают, и никто из них не взывает ко Мне.
\end{tcolorbox}
\begin{tcolorbox}
\textsubscript{8} Ефрем смешался с народами, Ефрем стал, как неповороченный хлеб.
\end{tcolorbox}
\begin{tcolorbox}
\textsubscript{9} Чужие пожирали силу его и он не замечал; седина покрыла его, а он не знает.
\end{tcolorbox}
\begin{tcolorbox}
\textsubscript{10} И гордость Израиля унижена в глазах их и при всем том они не обратились к Господу Богу своему и не взыскали Его.
\end{tcolorbox}
\begin{tcolorbox}
\textsubscript{11} И стал Ефрем, как глупый голубь, без сердца: зовут Египтян, идут в Ассирию.
\end{tcolorbox}
\begin{tcolorbox}
\textsubscript{12} Когда они пойдут, Я закину на них сеть Мою; как птиц небесных низвергну их; накажу их, как слышало собрание их.
\end{tcolorbox}
\begin{tcolorbox}
\textsubscript{13} Горе им, что они удалились от Меня; гибель им, что они отпали от Меня! Я спасал их, а они ложь говорили на Меня.
\end{tcolorbox}
\begin{tcolorbox}
\textsubscript{14} И не взывали ко Мне сердцем своим, когда вопили на ложах своих; собираются из-за хлеба и вина, а от Меня удаляются.
\end{tcolorbox}
\begin{tcolorbox}
\textsubscript{15} Я вразумлял [их] и укреплял мышцы их, а они умышляли злое против Меня.
\end{tcolorbox}
\begin{tcolorbox}
\textsubscript{16} Они обращались, но не к Всевышнему, стали--как неверный лук; падут от меча князья их за дерзость языка своего; это будет посмеянием над ними в земле Египетской.
\end{tcolorbox}
\subsection{CHAPTER 8}
\begin{tcolorbox}
\textsubscript{1} Трубу к устам твоим! Как орел [налетит] на дом Господень за то, что они нарушили завет Мой и преступили закон Мой!
\end{tcolorbox}
\begin{tcolorbox}
\textsubscript{2} Ко Мне будут взывать: 'Боже мой! мы познали Тебя, мы--Израиль'.
\end{tcolorbox}
\begin{tcolorbox}
\textsubscript{3} Отверг Израиль доброе; враг будет преследовать его.
\end{tcolorbox}
\begin{tcolorbox}
\textsubscript{4} Поставляли царей сами, без Меня; ставили князей, но без Моего ведома; из серебра своего и золота своего сделали для себя идолов: оттуда гибель.
\end{tcolorbox}
\begin{tcolorbox}
\textsubscript{5} Оставил тебя телец твой, Самария! воспылал гнев Мой на них; доколе не могут они очиститься?
\end{tcolorbox}
\begin{tcolorbox}
\textsubscript{6} Ибо и он--дело Израиля: художник сделал его, и потому он не бог; в куски обратится телец Самарийский!
\end{tcolorbox}
\begin{tcolorbox}
\textsubscript{7} Так как они сеяли ветер, то и пожнут бурю: хлеба на корню не будет у него; зерно не даст муки; а если и даст, то чужие проглотят ее.
\end{tcolorbox}
\begin{tcolorbox}
\textsubscript{8} Поглощен Израиль; теперь они будут среди народов, как негодный сосуд.
\end{tcolorbox}
\begin{tcolorbox}
\textsubscript{9} Они пошли к Ассуру, как дикий осел, одиноко бродящий; Ефрем приобретал подарками расположение к себе.
\end{tcolorbox}
\begin{tcolorbox}
\textsubscript{10} Хотя они и посылали дары к народам, но скоро Я соберу их, и они начнут страдать от бремени царя князей;
\end{tcolorbox}
\begin{tcolorbox}
\textsubscript{11} ибо много жертвенников настроил Ефрем для греха, --ко греху послужили ему эти жертвенники.
\end{tcolorbox}
\begin{tcolorbox}
\textsubscript{12} Написал Я ему важные законы Мои, но они сочтены им как бы чужие.
\end{tcolorbox}
\begin{tcolorbox}
\textsubscript{13} В жертвоприношениях Мне они приносят мясо и едят его; Господу неугодны они; ныне Он вспомнит нечестие их и накажет их за грехи их: они возвратятся в Египет.
\end{tcolorbox}
\begin{tcolorbox}
\textsubscript{14} Забыл Израиль Создателя своего и устроил капища, и Иуда настроил много укрепленных городов; но Я пошлю огонь на города его, и пожрет чертоги его.
\end{tcolorbox}
\subsection{CHAPTER 9}
\begin{tcolorbox}
\textsubscript{1} Не радуйся, Израиль, до восторга, как [другие] народы, ибо ты блудодействуешь, удалившись от Бога твоего: любишь блудодейные дары на всех гумнах.
\end{tcolorbox}
\begin{tcolorbox}
\textsubscript{2} Гумно и точило не будут питать их, и [надежда] на виноградный сок обманет их.
\end{tcolorbox}
\begin{tcolorbox}
\textsubscript{3} Не будут они жить на земле Господней: Ефрем возвратится в Египет, и в Ассирии будут есть нечистое.
\end{tcolorbox}
\begin{tcolorbox}
\textsubscript{4} Не будут возливать Господу вина, и неугодны Ему будут жертвы их; они будут для них, как хлеб похоронный: все, которые будут есть его, осквернятся, ибо хлеб их--для души их, а в дом Господень он не войдет.
\end{tcolorbox}
\begin{tcolorbox}
\textsubscript{5} Что будете делать в день торжества и в день праздника Господня?
\end{tcolorbox}
\begin{tcolorbox}
\textsubscript{6} Ибо вот, они уйдут по причине опустошения; Египет соберет их, Мемфис похоронит их; драгоценностями их из серебра завладеет крапива, колючий терн будет в шатрах их.
\end{tcolorbox}
\begin{tcolorbox}
\textsubscript{7} Пришли дни посещения, пришли дни воздаяния; да узнает Израиль, что глуп прорицатель, безумен выдающий себя за вдохновенного, по причине множества беззаконий твоих и великой враждебности.
\end{tcolorbox}
\begin{tcolorbox}
\textsubscript{8} Ефрем--страж подле Бога моего; пророк--сеть птицелова на всех путях его; соблазн в доме Бога его.
\end{tcolorbox}
\begin{tcolorbox}
\textsubscript{9} Глубоко упали они, развратились, как во дни Гивы; Он вспомнит нечестие их, накажет их за грехи их.
\end{tcolorbox}
\begin{tcolorbox}
\textsubscript{10} Как виноград в пустыне, Я нашел Израиля; как первую ягоду на смоковнице, в первое время ее, увидел Я отцов ваших, --но они пошли к Ваал-Фегору и предались постыдному, и сами стали мерзкими, как те, которых возлюбили.
\end{tcolorbox}
\begin{tcolorbox}
\textsubscript{11} У Ефремлян, как птица улетит слава: ни рождения, ни беременности, ни зачатия [не будет].
\end{tcolorbox}
\begin{tcolorbox}
\textsubscript{12} А хотя бы они и воспитали детей своих, отниму их; ибо горе им, когда удалюсь от них!
\end{tcolorbox}
\begin{tcolorbox}
\textsubscript{13} Ефрем, как Я видел его до Тира, насажден на прекрасной местности; однако Ефрем выведет детей своих к убийце.
\end{tcolorbox}
\begin{tcolorbox}
\textsubscript{14} Дай им, Господи: что Ты дашь им? дай им утробу нерождающую и сухие сосцы.
\end{tcolorbox}
\begin{tcolorbox}
\textsubscript{15} Все зло их в Галгале: там Я возненавидел их за злые дела их; изгоню их из дома Моего, не буду больше любить их; все князья их--отступники.
\end{tcolorbox}
\begin{tcolorbox}
\textsubscript{16} Поражен Ефрем; иссох корень их, --не будут приносить они плода, а если и будут рождать, Я умерщвлю вожделенный плод утробы их.
\end{tcolorbox}
\begin{tcolorbox}
\textsubscript{17} Отвергнет их Бог мой, потому что они не послушались Его, и будут скитальцами между народами.
\end{tcolorbox}
\subsection{CHAPTER 10}
\begin{tcolorbox}
\textsubscript{1} Израиль--ветвистый виноград, умножает для себя плод: чем более у него плодов, тем более умножает жертвенники; чем лучше земля у него, тем более украшают они кумиры.
\end{tcolorbox}
\begin{tcolorbox}
\textsubscript{2} Разделилось сердце их, за то они и будут наказаны: Он разрушит жертвенники их, сокрушит кумиры их.
\end{tcolorbox}
\begin{tcolorbox}
\textsubscript{3} Теперь они говорят: 'нет у нас царя, ибо мы не убоялись Господа; а царь, --что он нам сделает?'
\end{tcolorbox}
\begin{tcolorbox}
\textsubscript{4} Говорят слова [пустые], клянутся ложно, заключают союзы; за то явится суд над ними, как ядовитая трава на бороздах поля.
\end{tcolorbox}
\begin{tcolorbox}
\textsubscript{5} За тельца Беф-Авена вострепещут жители Самарии; восплачет о нем народ его, и жрецы его, радовавшиеся о нем, будут плакать о славе его, потому что она отойдет от него.
\end{tcolorbox}
\begin{tcolorbox}
\textsubscript{6} И сам он отнесен будет в Ассирию, в дар царю Иареву; постыжен будет Ефрем, и посрамится Израиль от замысла своего.
\end{tcolorbox}
\begin{tcolorbox}
\textsubscript{7} Исчезнет в Самарии царь ее, как пена на поверхности воды.
\end{tcolorbox}
\begin{tcolorbox}
\textsubscript{8} И истреблены будут высоты Авена, грех Израиля; терние и волчцы вырастут на жертвенниках их, и скажут они горам: 'покройте нас', и холмам: 'падите на нас'.
\end{tcolorbox}
\begin{tcolorbox}
\textsubscript{9} Больше, нежели во дни Гивы, грешил ты, Израиль; там они устояли; война в Гаваоне против сынов нечестия не постигла их.
\end{tcolorbox}
\begin{tcolorbox}
\textsubscript{10} По желанию Моему накажу их, и соберутся против них народы, и они будут связаны за двойное преступление их.
\end{tcolorbox}
\begin{tcolorbox}
\textsubscript{11} Ефрем--обученная телица, привычная к молотьбе, и Я Сам возложу ярмо на тучную шею его; на Ефреме будут верхом ездить, Иуда будет пахать, Иаков будет боронить.
\end{tcolorbox}
\begin{tcolorbox}
\textsubscript{12} Сейте себе в правду, и пожнете милость; распахивайте у себя новину, ибо время взыскать Господа, чтобы Он, когда придет, дождем пролил на вас правду.
\end{tcolorbox}
\begin{tcolorbox}
\textsubscript{13} Вы возделывали нечестие, пожинаете беззаконие, едите плод лжи, потому что ты надеялся на путь твой, на множество ратников твоих.
\end{tcolorbox}
\begin{tcolorbox}
\textsubscript{14} И произойдет смятение в народе твоем, и все твердыни твои будут разрушены, как Салман разрушил Бет-Арбел в день брани: мать была убита с детьми.
\end{tcolorbox}
\begin{tcolorbox}
\textsubscript{15} Вот что причинит вам Вефиль за крайнее нечестие ваше.
\end{tcolorbox}
\subsection{CHAPTER 11}
\begin{tcolorbox}
\textsubscript{1} На заре погибнет царь Израилев! Когда Израиль был юн, Я любил его и из Египта вызвал сына Моего.
\end{tcolorbox}
\begin{tcolorbox}
\textsubscript{2} Звали их, а они уходили прочь от лица их: приносили жертву Ваалам и кадили истуканам.
\end{tcolorbox}
\begin{tcolorbox}
\textsubscript{3} Я Сам приучал Ефрема ходить, носил его на руках Своих, а они не сознавали, что Я врачевал их.
\end{tcolorbox}
\begin{tcolorbox}
\textsubscript{4} Узами человеческими влек Я их, узами любви, и был для них как бы поднимающий ярмо с челюстей их, и ласково подкладывал пищу им.
\end{tcolorbox}
\begin{tcolorbox}
\textsubscript{5} Не возвратится он в Египет, но Ассур--он будет царем его, потому что они не захотели обратиться [ко Мне].
\end{tcolorbox}
\begin{tcolorbox}
\textsubscript{6} И падет меч на города его, и истребит затворы его, и пожрет их за умыслы их.
\end{tcolorbox}
\begin{tcolorbox}
\textsubscript{7} Народ Мой закоснел в отпадении от Меня, и хотя призывают его к горнему, он не возвышается единодушно.
\end{tcolorbox}
\begin{tcolorbox}
\textsubscript{8} Как поступлю с тобою, Ефрем? как предам тебя, Израиль? Поступлю ли с тобою, как с Адамою, сделаю ли тебе, что Севоиму? Повернулось во Мне сердце Мое, возгорелась вся жалость Моя!
\end{tcolorbox}
\begin{tcolorbox}
\textsubscript{9} Не сделаю по ярости гнева Моего, не истреблю Ефрема, ибо Я Бог, а не человек; среди тебя Святый; Я не войду в город.
\end{tcolorbox}
\begin{tcolorbox}
\textsubscript{10} Вслед Господа пойдут они; как лев, Он даст глас Свой, даст глас Свой, и встрепенутся к Нему сыны с запада,
\end{tcolorbox}
\begin{tcolorbox}
\textsubscript{11} встрепенутся из Египта, как птицы, и из земли Ассирийской, как голуби, и вселю их в домы их, говорит Господь.
\end{tcolorbox}
\begin{tcolorbox}
\textsubscript{12} Окружил Меня Ефрем ложью и дом Израилев лукавством; Иуда держался еще Бога и верен был со святыми.
\end{tcolorbox}
\subsection{CHAPTER 12}
\begin{tcolorbox}
\textsubscript{1} Ефрем пасет ветер и гоняется за восточным ветром, каждый день умножает ложь и разорение; заключают они союз с Ассуром, и в Египет отвозится елей.
\end{tcolorbox}
\begin{tcolorbox}
\textsubscript{2} Но и с Иудою у Господа суд и Он посетит Иакова по путям его, воздаст ему по делам его.
\end{tcolorbox}
\begin{tcolorbox}
\textsubscript{3} Еще во чреве матери запинал он брата своего, а возмужав боролся с Богом.
\end{tcolorbox}
\begin{tcolorbox}
\textsubscript{4} Он боролся с Ангелом--и превозмог; плакал и умолял Его; в Вефиле Он нашел нас и там говорил с нами.
\end{tcolorbox}
\begin{tcolorbox}
\textsubscript{5} А Господь есть Бог Саваоф; Сущий (Иегова) --имя Его.
\end{tcolorbox}
\begin{tcolorbox}
\textsubscript{6} Обратись и ты к Богу твоему; наблюдай милость и суд и уповай на Бога твоего всегда.
\end{tcolorbox}
\begin{tcolorbox}
\textsubscript{7} Хананеянин с неверными весами в руке любит обижать;
\end{tcolorbox}
\begin{tcolorbox}
\textsubscript{8} и Ефрем говорит: 'однако я разбогател; накопил себе имущества, хотя во всех моих трудах не найдут ничего незаконного, что было бы грехом'.
\end{tcolorbox}
\begin{tcolorbox}
\textsubscript{9} А Я, Господь Бог твой от самой земли Египетской, опять поселю тебя в кущах, как во дни праздника.
\end{tcolorbox}
\begin{tcolorbox}
\textsubscript{10} Я говорил к пророкам, и умножал видения, и чрез пророков употреблял притчи.
\end{tcolorbox}
\begin{tcolorbox}
\textsubscript{11} Если Галаад сделался Авеном, то они стали суетны, в Галгалах заколали в жертву тельцов, и жертвенники их стояли как груды камней на межах поля.
\end{tcolorbox}
\begin{tcolorbox}
\textsubscript{12} Убежал Иаков на поля Сирийские, и служил Израиль за жену, и за жену стерег [овец].
\end{tcolorbox}
\begin{tcolorbox}
\textsubscript{13} Чрез пророка вывел Господь Израиля из Египта, и чрез пророка Он охранял его.
\end{tcolorbox}
\begin{tcolorbox}
\textsubscript{14} Сильно раздражил Ефрем [Господа] и за то кровь его оставит на нем, и поношение его обратит Господь на него.
\end{tcolorbox}
\subsection{CHAPTER 13}
\begin{tcolorbox}
\textsubscript{1} Когда Ефрем говорил, все трепетали. Он был высок в Израиле; но сделался виновным через Ваала, и погиб.
\end{tcolorbox}
\begin{tcolorbox}
\textsubscript{2} И ныне прибавили они ко греху: сделали для себя литых истуканов из серебра своего, по понятию своему, --полная работа художников, --и говорят они приносящим жертву людям: 'целуйте тельцов!'
\end{tcolorbox}
\begin{tcolorbox}
\textsubscript{3} За то они будут как утренний туман, как роса, скоро исчезающая, как мякина, свеваемая с гумна, и как дым из трубы.
\end{tcolorbox}
\begin{tcolorbox}
\textsubscript{4} Но Я--Господь Бог твой от земли Египетской, --и ты не должен знать другого бога, кроме Меня, и нет спасителя, кроме Меня.
\end{tcolorbox}
\begin{tcolorbox}
\textsubscript{5} Я признал тебя в пустыне, в земле жаждущей.
\end{tcolorbox}
\begin{tcolorbox}
\textsubscript{6} Имея пажити, они были сыты; а когда насыщались, то превозносилось сердце их, и потому они забывали Меня.
\end{tcolorbox}
\begin{tcolorbox}
\textsubscript{7} И Я буду для них как лев, как скимен буду подстерегать при дороге.
\end{tcolorbox}
\begin{tcolorbox}
\textsubscript{8} Буду нападать на них, как лишенная детей медведица, и раздирать вместилище сердца их, и поедать их там, как львица; полевые звери будут терзать их.
\end{tcolorbox}
\begin{tcolorbox}
\textsubscript{9} Погубил ты себя, Израиль, ибо только во Мне опора твоя.
\end{tcolorbox}
\begin{tcolorbox}
\textsubscript{10} Где царь твой теперь? Пусть он спасет тебя во всех городах твоих! Где судьи твои, о которых говорил ты: 'дай нам царя и начальников'?
\end{tcolorbox}
\begin{tcolorbox}
\textsubscript{11} И Я дал тебе царя во гневе Моем, и отнял в негодовании Моем.
\end{tcolorbox}
\begin{tcolorbox}
\textsubscript{12} Связано в узел беззаконие Ефрема, сбережен его грех.
\end{tcolorbox}
\begin{tcolorbox}
\textsubscript{13} Муки родильницы постигнут его; он--сын неразумный, иначе не стоял бы долго в положении рождающихся детей.
\end{tcolorbox}
\begin{tcolorbox}
\textsubscript{14} От власти ада Я искуплю их, от смерти избавлю их. Смерть! где твое жало? ад! где твоя победа? Раскаяния в том не будет у Меня.
\end{tcolorbox}
\begin{tcolorbox}
\textsubscript{15} Хотя [Ефрем] плодовит между братьями, но придет восточный ветер, поднимется ветер Господень из пустыни, и иссохнет родник его, и иссякнет источник его; он опустошит сокровищницу всех драгоценных сосудов.
\end{tcolorbox}
\begin{tcolorbox}
\textsubscript{16} (14-1) Опустошена будет Самария, потому что восстала против Бога своего; от меча падут они; младенцы их будут разбиты, и беременные их будут рассечены.
\end{tcolorbox}
\subsection{CHAPTER 14}
\begin{tcolorbox}
\textsubscript{1} (14-2) Обратись, Израиль, к Господу Богу твоему; ибо ты упал от нечестия твоего.
\end{tcolorbox}
\begin{tcolorbox}
\textsubscript{2} (14-3) Возьмите с собою [молитвенные] слова и обратитесь к Господу; говорите Ему: 'отними всякое беззаконие и прими во благо, и мы принесем жертву уст наших.
\end{tcolorbox}
\begin{tcolorbox}
\textsubscript{3} (14-4) Ассур не будет уже спасать нас; не станем садиться на коня и не будем более говорить изделию рук наших: боги наши; потому что у Тебя милосердие для сирот'.
\end{tcolorbox}
\begin{tcolorbox}
\textsubscript{4} (14-5) Уврачую отпадение их, возлюблю их по благоволению; ибо гнев Мой отвратился от них.
\end{tcolorbox}
\begin{tcolorbox}
\textsubscript{5} (14-6) Я буду росою для Израиля; он расцветет, как лилия, и пустит корни свои, как Ливан.
\end{tcolorbox}
\begin{tcolorbox}
\textsubscript{6} (14-7) Расширятся ветви его, и будет красота его, как маслины, и благоухание от него, как от Ливана.
\end{tcolorbox}
\begin{tcolorbox}
\textsubscript{7} (14-8) Возвратятся сидевшие под тенью его, будут изобиловать хлебом, и расцветут, как виноградная лоза, славны будут, как вино Ливанское.
\end{tcolorbox}
\begin{tcolorbox}
\textsubscript{8} (14-9) 'Что мне еще за дело до идолов?' --скажет Ефрем. --Я услышу его и призрю на него; Я буду как зеленеющий кипарис; от Меня будут тебе плоды.
\end{tcolorbox}
\begin{tcolorbox}
\textsubscript{9} (14-10) Кто мудр, чтобы разуметь это? кто разумен, чтобы познать это? Ибо правы пути Господни, и праведники ходят по ним, а беззаконные падут на них.
\end{tcolorbox}
