\section{BOOK 168}
\subsection{CHAPTER 1}
\begin{tcolorbox}
\textsubscript{1} Павел, узник Иисуса Христа, и Тимофей брат, Филимону возлюбленному и сотруднику нашему,
\end{tcolorbox}
\begin{tcolorbox}
\textsubscript{2} и Апфии, (сестре) возлюбленной, и Архиппу, сподвижнику нашему, и домашней твоей церкви:
\end{tcolorbox}
\begin{tcolorbox}
\textsubscript{3} благодать вам и мир от Бога Отца нашего и Господа Иисуса Христа.
\end{tcolorbox}
\begin{tcolorbox}
\textsubscript{4} Благодарю Бога моего, всегда вспоминая о тебе в молитвах моих,
\end{tcolorbox}
\begin{tcolorbox}
\textsubscript{5} слыша о твоей любви и вере, которую имеешь к Господу Иисусу и ко всем святым,
\end{tcolorbox}
\begin{tcolorbox}
\textsubscript{6} дабы общение веры твоей оказалось деятельным в познании всякого у вас добра во Христе Иисусе.
\end{tcolorbox}
\begin{tcolorbox}
\textsubscript{7} Ибо мы имеем великую радость и утешение в любви твоей, потому что тобою, брат, успокоены сердца святых.
\end{tcolorbox}
\begin{tcolorbox}
\textsubscript{8} Посему, имея великое во Христе дерзновение приказывать тебе, что должно,
\end{tcolorbox}
\begin{tcolorbox}
\textsubscript{9} по любви лучше прошу, не иной кто, как я, Павел старец, а теперь и узник Иисуса Христа;
\end{tcolorbox}
\begin{tcolorbox}
\textsubscript{10} прошу тебя о сыне моем Онисиме, которого родил я в узах моих:
\end{tcolorbox}
\begin{tcolorbox}
\textsubscript{11} он был некогда негоден для тебя, а теперь годен тебе и мне; я возвращаю его;
\end{tcolorbox}
\begin{tcolorbox}
\textsubscript{12} ты же прими его, как мое сердце.
\end{tcolorbox}
\begin{tcolorbox}
\textsubscript{13} Я хотел при себе удержать его, дабы он вместо тебя послужил мне в узах [за] благовествование;
\end{tcolorbox}
\begin{tcolorbox}
\textsubscript{14} но без твоего согласия ничего не хотел сделать, чтобы доброе дело твое было не вынужденно, а добровольно.
\end{tcolorbox}
\begin{tcolorbox}
\textsubscript{15} Ибо, может быть, он для того на время отлучился, чтобы тебе принять его навсегда,
\end{tcolorbox}
\begin{tcolorbox}
\textsubscript{16} не как уже раба, но выше раба, брата возлюбленного, особенно мне, а тем больше тебе, и по плоти и в Господе.
\end{tcolorbox}
\begin{tcolorbox}
\textsubscript{17} Итак, если ты имеешь общение со мною, то прими его, как меня.
\end{tcolorbox}
\begin{tcolorbox}
\textsubscript{18} Если же он чем обидел тебя, или должен, считай это на мне.
\end{tcolorbox}
\begin{tcolorbox}
\textsubscript{19} Я, Павел, написал моею рукою: я заплачу; не говорю тебе о том, что ты и самим собою мне должен.
\end{tcolorbox}
\begin{tcolorbox}
\textsubscript{20} Так, брат, дай мне воспользоваться от тебя в Господе; успокой мое сердце в Господе.
\end{tcolorbox}
\begin{tcolorbox}
\textsubscript{21} Надеясь на послушание твое, я написал к тебе, зная, что ты сделаешь и более, нежели говорю.
\end{tcolorbox}
\begin{tcolorbox}
\textsubscript{22} А вместе приготовь для меня и помещение; ибо надеюсь, что по молитвам вашим я буду дарован вам.
\end{tcolorbox}
\begin{tcolorbox}
\textsubscript{23} Приветствует тебя Епафрас, узник вместе со мною ради Христа Иисуса,
\end{tcolorbox}
\begin{tcolorbox}
\textsubscript{24} Марк, Аристарх, Димас, Лука, сотрудники мои.
\end{tcolorbox}
\begin{tcolorbox}
\textsubscript{25} Благодать Господа нашего Иисуса Христа со духом вашим. Аминь.
\end{tcolorbox}
