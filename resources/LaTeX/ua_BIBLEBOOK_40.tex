\section{BOOK 39}
\subsection{CHAPTER 1}
\begin{tcolorbox}
\textsubscript{1} Книга родоводу Ісуса Христа, Сина Давидового, Сина Авраамового:
\end{tcolorbox}
\begin{tcolorbox}
\textsubscript{2} Авраам породив Ісака, а Ісак породив Якова, а Яків породив Юду й братів його.
\end{tcolorbox}
\begin{tcolorbox}
\textsubscript{3} Юда ж породив Фареса та Зару від Тамари. Фарес же породив Есрома, а Есром породив Арама.
\end{tcolorbox}
\begin{tcolorbox}
\textsubscript{4} А Арам породив Амінадава, Амінадав же породив Наассона, а Наассон породив Салмона.
\end{tcolorbox}
\begin{tcolorbox}
\textsubscript{5} Салмон же породив Вооза від Рахави, а Вооз породив Йовіда від Рути, Йовід же породив Єссея.
\end{tcolorbox}
\begin{tcolorbox}
\textsubscript{6} А Єссей породив царя Давида, Давид же породив Соломона від Урієвої.
\end{tcolorbox}
\begin{tcolorbox}
\textsubscript{7} Соломон же породив Ровоама, а Ровоам породив Авію, а Авія породив Асафа.
\end{tcolorbox}
\begin{tcolorbox}
\textsubscript{8} Асаф же породив Йосафата, а Йосафат породив Йорама, Йорам же породив Озію.
\end{tcolorbox}
\begin{tcolorbox}
\textsubscript{9} Озія ж породив Йоатама, а Йоатам породив Ахаза, Ахаз же породив Єзекію.
\end{tcolorbox}
\begin{tcolorbox}
\textsubscript{10} А Єзекія породив Манасію, Манасія ж породив Амоса, а Амос породив Йосію.
\end{tcolorbox}
\begin{tcolorbox}
\textsubscript{11} Йосія ж породив Йоякима, Йояким породив Єхонію й братів його за вавилонського переселення.
\end{tcolorbox}
\begin{tcolorbox}
\textsubscript{12} А по вавилонськім переселенні Єхонія породив Салатіїля, а Салатіїль породив Зоровавеля.
\end{tcolorbox}
\begin{tcolorbox}
\textsubscript{13} Зоровавель же породив Авіюда, а Авіюд породив Еліякима, а Еліяким породив Азора.
\end{tcolorbox}
\begin{tcolorbox}
\textsubscript{14} Азор же породив Садока, а Садок породив Ахіма, а Ахім породив Еліюда.
\end{tcolorbox}
\begin{tcolorbox}
\textsubscript{15} Еліюд же породив Елеазара, а Елеазар породив Маттана, а Маттан породив Якова.
\end{tcolorbox}
\begin{tcolorbox}
\textsubscript{16} А Яків породив Йосипа, мужа Марії, що з неї родився Ісус, званий Христос.
\end{tcolorbox}
\begin{tcolorbox}
\textsubscript{17} А всіх поколінь від Авраама аж до Давида чотирнадцять поколінь, і від Давида аж до вавилонського переселення чотирнадцять поколінь, і від вавилонського переселення до Христа поколінь чотирнадцять.
\end{tcolorbox}
\begin{tcolorbox}
\textsubscript{18} Народження ж Ісуса Христа сталося так. Коли Його матір Марію заручено з Йосипом, то перш, ніж зійшлися вони, виявилося, що вона має в утробі від Духа Святого.
\end{tcolorbox}
\begin{tcolorbox}
\textsubscript{19} А Йосип, муж її, бувши праведний, і не бажавши ославити її, хотів тайкома відпустити її.
\end{tcolorbox}
\begin{tcolorbox}
\textsubscript{20} Коли ж він те подумав, ось з'явивсь йому Ангол Господній у сні, промовляючи: Йосипе, сину Давидів, не бійся прийняти Марію, дружину свою, бо зачате в ній то від Духа Святого.
\end{tcolorbox}
\begin{tcolorbox}
\textsubscript{21} І вона вродить Сина, ти ж даси Йому йменна Ісус, бо спасе Він людей Своїх від їхніх гріхів.
\end{tcolorbox}
\begin{tcolorbox}
\textsubscript{22} А все оце сталось, щоб збулося сказане пророком від Господа, який провіщає:
\end{tcolorbox}
\begin{tcolorbox}
\textsubscript{23} Ось діва в утробі зачне, і Сина породить, і назвуть Йому Ймення Еммануїл, що в перекладі є: З нами Бог.
\end{tcolorbox}
\begin{tcolorbox}
\textsubscript{24} Як прокинувся ж Йосип зо сну, то зробив, як звелів йому Ангол Господній, і прийняв він дружину свою.
\end{tcolorbox}
\begin{tcolorbox}
\textsubscript{25} І не знав він її, аж Сина свого первородженого вона породила, а він дав Йому ймення Ісус.
\end{tcolorbox}
\subsection{CHAPTER 2}
\begin{tcolorbox}
\textsubscript{1} Коли ж народився Ісус у Віфлеємі Юдейськім, за днів царя Ірода, то ось мудреці прибули до Єрусалиму зо сходу,
\end{tcolorbox}
\begin{tcolorbox}
\textsubscript{2} і питали: Де народжений Цар Юдейський? Бо на сході ми бачили зорю Його, і прибули поклонитись Йому.
\end{tcolorbox}
\begin{tcolorbox}
\textsubscript{3} І, як зачув це цар Ірод, занепокоївся, і з ним увесь Єрусалим.
\end{tcolorbox}
\begin{tcolorbox}
\textsubscript{4} І, зібравши всіх первосвящеників і книжників людських, він випитував у них, де має Христос народитись?
\end{tcolorbox}
\begin{tcolorbox}
\textsubscript{5} Вони ж відказали йому: У Віфлеємі Юдейськім, бо в пророка написано так:
\end{tcolorbox}
\begin{tcolorbox}
\textsubscript{6} І ти, Віфлеєме, земле Юдина, не менший нічим між осадами Юдиними, бо з тебе з'явиться Вождь, що буде Він пасти народ Мій ізраїльський.
\end{tcolorbox}
\begin{tcolorbox}
\textsubscript{7} Тоді Ірод покликав таємно отих мудреців, і докладно випитував їх про час, коли з'явилась зоря.
\end{tcolorbox}
\begin{tcolorbox}
\textsubscript{8} І він відіслав їх до Віфлеєму, говорячи: Ідіть, і пильно розвідайтеся про Дитятко; а як знайдете, сповістіть мене, щоб і я міг піти й поклонитись Йому.
\end{tcolorbox}
\begin{tcolorbox}
\textsubscript{9} Вони ж царя вислухали й відійшли. І ось зоря, що на сході вони її бачили, ішла перед ними, аж прийшла й стала зверху, де Дитятко було.
\end{tcolorbox}
\begin{tcolorbox}
\textsubscript{10} А бачивши зорю, вони надзвичайно зраділи.
\end{tcolorbox}
\begin{tcolorbox}
\textsubscript{11} І, ввійшовши до дому, знайшли там Дитятко з Марією, Його матір'ю. І вони впали ницьма, і вклонились Йому. І, відчинивши скарбниці свої, піднесли Йому свої дари: золото, ладан та смирну.
\end{tcolorbox}
\begin{tcolorbox}
\textsubscript{12} А вві сні остережені, щоб не вертатись до Ірода, відійшли вони іншим шляхом до своєї землі.
\end{tcolorbox}
\begin{tcolorbox}
\textsubscript{13} Як вони ж відійшли, ось Ангол Господній з'явивсь у сні Йосипові та й сказав: Уставай, візьми Дитятко та матір Його, і втікай до Єгипту, і там зоставайся, аж поки скажу тобі, бо Дитятка шукатиме Ірод, щоб Його погубити.
\end{tcolorbox}
\begin{tcolorbox}
\textsubscript{14} І він устав, узяв Дитятко та матір Його вночі, та й пішов до Єгипту.
\end{tcolorbox}
\begin{tcolorbox}
\textsubscript{15} І він там зоставався аж до смерти Іродової, щоб збулося сказане від Господа пророком, який провіщає: Із Єгипту покликав Я Сина Свого.
\end{tcolorbox}
\begin{tcolorbox}
\textsubscript{16} Спостеріг тоді Ірод, що ті мудреці насміялися з нього, та й розгнівався дуже, і послав повбивати в Віфлеємі й по всій тій околиці всіх дітей від двох років і менше, за часом, що його в мудреців він був випитав.
\end{tcolorbox}
\begin{tcolorbox}
\textsubscript{17} Тоді справдилось те, що сказав Єремія пророк, промовляючи:
\end{tcolorbox}
\begin{tcolorbox}
\textsubscript{18} Чути голос у Рамі, плач і ридання та голосіння велике: Рахиль плаче за дітьми своїми, і не дається розважити себе, бо нема їх...
\end{tcolorbox}
\begin{tcolorbox}
\textsubscript{19} Коли ж Ірод умер, ось Ангол Господній з'явився в Єгипті вві сні Йосипові, та й промовив:
\end{tcolorbox}
\begin{tcolorbox}
\textsubscript{20} Уставай, візьми Дитятко та матір Його, та йди в землю Ізраїлеву, бо вимерли ті, хто шукав був душу Дитини.
\end{tcolorbox}
\begin{tcolorbox}
\textsubscript{21} І він устав, узяв Дитятко та матір Його, і прийшов у землю Ізраїлеву.
\end{tcolorbox}
\begin{tcolorbox}
\textsubscript{22} Та прочувши, що царює в Юдеї Архелай, замість Ірода, батька свого, побоявся піти туди він. А вві сні остережений, відійшов до країв галілейських.
\end{tcolorbox}
\begin{tcolorbox}
\textsubscript{23} А прибувши, оселився у місті, на ім'я Назарет, щоб збулося пророками сказане, що Він Назарянин буде званий.
\end{tcolorbox}
\subsection{CHAPTER 3}
\begin{tcolorbox}
\textsubscript{1} Тими ж днями приходить Іван Христитель, і проповідує в пустині юдейській,
\end{tcolorbox}
\begin{tcolorbox}
\textsubscript{2} та й каже: Покайтесь, бо наблизилось Царство Небесне!
\end{tcolorbox}
\begin{tcolorbox}
\textsubscript{3} Бо він той, що про нього сказав був Ісая пророк, промовляючи: Голос того, хто кличе: В пустині готуйте дорогу для Господа, рівняйте стежки Йому!
\end{tcolorbox}
\begin{tcolorbox}
\textsubscript{4} Сам же Іван мав одежу собі з верблюжого волосу, і пояс ремінний на стегнах своїх; а пожива для нього була сарана та мед польовий.
\end{tcolorbox}
\begin{tcolorbox}
\textsubscript{5} Тоді до нього виходив Єрусалим, і вся Юдея, і вся йорданська околиця,
\end{tcolorbox}
\begin{tcolorbox}
\textsubscript{6} і в річці Йордані христились від нього, і визнавали гріхи свої.
\end{tcolorbox}
\begin{tcolorbox}
\textsubscript{7} Як побачив же він багатьох фарисеїв та саддукеїв, що приходять на хрищення, то промовив до них: Роде зміїний, хто вас надоумив утікати від гніву майбутнього?
\end{tcolorbox}
\begin{tcolorbox}
\textsubscript{8} Отож, учиніть гідний плід покаяння!
\end{tcolorbox}
\begin{tcolorbox}
\textsubscript{9} І не думайте говорити в собі: Ми маємо отця Авраама. Кажу бо я вам, що Бог може піднести дітей Авраамові з цього каміння!
\end{tcolorbox}
\begin{tcolorbox}
\textsubscript{10} Бо вже до коріння дерев і сокира прикладена: кожне ж дерево, що доброго плоду не родить, буде зрубане та й в огонь буде вкинене.
\end{tcolorbox}
\begin{tcolorbox}
\textsubscript{11} Я хрищу вас водою на покаяння, але Той, Хто йде по мені, потужніший від мене: я недостойний понести взуття Йому! Він христитиме вас Святим Духом й огнем.
\end{tcolorbox}
\begin{tcolorbox}
\textsubscript{12} У руці Своїй має Він віячку, і перечистить Свій тік: пшеницю Свою Він збере до засіків, а полову попалить ув огні невгасимім.
\end{tcolorbox}
\begin{tcolorbox}
\textsubscript{13} Тоді прибуває Ісус із Галілеї понад Йордан до Івана, щоб христитись від нього.
\end{tcolorbox}
\begin{tcolorbox}
\textsubscript{14} Але перешкоджав він Йому й говорив: Я повинен христитись від Тебе, і чи Тобі йти до мене?
\end{tcolorbox}
\begin{tcolorbox}
\textsubscript{15} А Ісус відповів і сказав йому: Допусти це тепер, бо так годиться нам виповнити усю правду. Тоді допустив він Його.
\end{tcolorbox}
\begin{tcolorbox}
\textsubscript{16} І охристившись Ісус, зараз вийшов із води. І ось небо розкрилось, і побачив Іван Духа Божого, що спускався, як голуб, і сходив на Нього.
\end{tcolorbox}
\begin{tcolorbox}
\textsubscript{17} І ось голос почувся із неба: Це Син Мій Улюблений, що Його Я вподобав!
\end{tcolorbox}
\subsection{CHAPTER 4}
\begin{tcolorbox}
\textsubscript{1} Потому Ісус був поведений Духом у пустиню, щоб диявол Його спокушав.
\end{tcolorbox}
\begin{tcolorbox}
\textsubscript{2} І постив Він сорок день і сорок ночей, а вкінці зголоднів.
\end{tcolorbox}
\begin{tcolorbox}
\textsubscript{3} І ось приступив до Нього спокусник, і сказав: Коли Ти Син Божий, скажи, щоб каміння це стало хлібами!
\end{tcolorbox}
\begin{tcolorbox}
\textsubscript{4} А Він відповів і промовив: Написано: Не хлібом самим буде жити людина, але кожним словом, що походить із уст Божих.
\end{tcolorbox}
\begin{tcolorbox}
\textsubscript{5} Тоді забирає диявол Його в святе місто, і ставить Його на наріжника храму,
\end{tcolorbox}
\begin{tcolorbox}
\textsubscript{6} та й каже Йому: Коли Ти Син Божий, то кинься додолу, бож написано: Він накаже про Тебе Своїм Анголам, і вони на руках понесуть Тебе, щоб об камінь коли не спіткнув Ти Своєї ноги.
\end{tcolorbox}
\begin{tcolorbox}
\textsubscript{7} Ісус відказав йому: Ще написано: Не спокушуй Господа Бога свого!
\end{tcolorbox}
\begin{tcolorbox}
\textsubscript{8} Знов диявол бере Його на височезную гору, і показує Йому всі царства на світі та їхнюю славу,
\end{tcolorbox}
\begin{tcolorbox}
\textsubscript{9} та й каже до Нього: Це все Тобі дам, якщо впадеш і мені Ти поклонишся!
\end{tcolorbox}
\begin{tcolorbox}
\textsubscript{10} Тоді каже до нього Ісус: Відійди, сатано! Бож написано: Господеві Богові своєму вклоняйся, і служи Одному Йому!
\end{tcolorbox}
\begin{tcolorbox}
\textsubscript{11} Тоді позоставив диявол Його. І ось Анголи приступили, і служили Йому.
\end{tcolorbox}
\begin{tcolorbox}
\textsubscript{12} Як довідавсь Ісус, що Івана ув'язнено, перейшов у Галілею.
\end{tcolorbox}
\begin{tcolorbox}
\textsubscript{13} І, покинувши Він Назарета, прийшов й оселився в Капернаумі приморськім, на границі країн Завулонової й Нефталимової,
\end{tcolorbox}
\begin{tcolorbox}
\textsubscript{14} щоб справдилось те, що сказав був Ісая пророк, промовляючи:
\end{tcolorbox}
\begin{tcolorbox}
\textsubscript{15} Завулонова земле, і Нефталимова земле, за Йорданом при морській дорозі, Галілеє поганська!
\end{tcolorbox}
\begin{tcolorbox}
\textsubscript{16} Народ, що в темноті сидів, світло велике побачив, а тим, хто сидів у країні смертельної тіні, засяяло світло.
\end{tcolorbox}
\begin{tcolorbox}
\textsubscript{17} Із того часу Ісус розпочав проповідувати й промовляти: Покайтеся, бо наблизилось Царство Небесне!
\end{tcolorbox}
\begin{tcolorbox}
\textsubscript{18} Як проходив же Він поблизу Галілейського моря, то побачив двох братів: Симона, що зветься Петром, та Андрія, його брата, що невода в море закидали, бо рибалки були.
\end{tcolorbox}
\begin{tcolorbox}
\textsubscript{19} І Він каже до них: Ідіть за Мною, Я зроблю вас ловцями людей!
\end{tcolorbox}
\begin{tcolorbox}
\textsubscript{20} І вони зараз покинули сіті, та й пішли вслід за Ним.
\end{tcolorbox}
\begin{tcolorbox}
\textsubscript{21} І, далі пішовши звідти, Він побачив двох інших братів, Зеведеєвого сина Якова та Івана, його брата, із Зеведеєм, їхнім батьком, що лагодили свого невода в човні, і покликав Він їх.
\end{tcolorbox}
\begin{tcolorbox}
\textsubscript{22} Вони зараз залишили човна та батька свого, та й пішли вслід за Ним.
\end{tcolorbox}
\begin{tcolorbox}
\textsubscript{23} І ходив Він по всій Галілеї, по їхніх синагогах навчаючи, та Євангелію Царства проповідуючи, і вздоровлюючи всяку недугу, і всяку неміч між людьми.
\end{tcolorbox}
\begin{tcolorbox}
\textsubscript{24} А чутка про Нього пішла по всій Сирії. І водили до Нього недужих усіх, хто терпів на різні хвороби та муки, і біснуватих, і сновид, і розслаблених, і Він їх уздоровляв.
\end{tcolorbox}
\begin{tcolorbox}
\textsubscript{25} І багато людей ішло за Ним і з Галілеї, і з Десятимістя, і з Єрусалиму, і з Юдеї, і з Зайордання.
\end{tcolorbox}
\subsection{CHAPTER 5}
\begin{tcolorbox}
\textsubscript{1} І, побачивши натовп, Він вийшов на гору. А як сів, підійшли Його учні до Нього.
\end{tcolorbox}
\begin{tcolorbox}
\textsubscript{2} І, відкривши уста Свої, Він навчати їх став, промовляючи:
\end{tcolorbox}
\begin{tcolorbox}
\textsubscript{3} Блаженні вбогі духом, бо їхнєє Царство Небесне.
\end{tcolorbox}
\begin{tcolorbox}
\textsubscript{4} Блаженні засмучені, бо вони будуть утішені.
\end{tcolorbox}
\begin{tcolorbox}
\textsubscript{5} Блаженні лагідні, бо землю вспадкують вони.
\end{tcolorbox}
\begin{tcolorbox}
\textsubscript{6} Блаженні голодні та спрагнені правди, бо вони нагодовані будуть.
\end{tcolorbox}
\begin{tcolorbox}
\textsubscript{7} Блаженні милостиві, бо помилувані вони будуть.
\end{tcolorbox}
\begin{tcolorbox}
\textsubscript{8} Блаженні чисті серцем, бо вони будуть бачити Бога.
\end{tcolorbox}
\begin{tcolorbox}
\textsubscript{9} Блаженні миротворці, бо вони синами Божими стануть.
\end{tcolorbox}
\begin{tcolorbox}
\textsubscript{10} Блаженні вигнані за правду, бо їхнє Царство Небесне.
\end{tcolorbox}
\begin{tcolorbox}
\textsubscript{11} Блаженні ви, як ганьбити та гнати вас будуть, і будуть облудно на вас наговорювати всяке слово лихе ради Мене.
\end{tcolorbox}
\begin{tcolorbox}
\textsubscript{12} Радійте та веселіться, нагорода бо ваша велика на небесах! Бо так гнали й пророків, що були перед вами.
\end{tcolorbox}
\begin{tcolorbox}
\textsubscript{13} Ви сіль землі. Коли сіль ізвітріє, то чим насолити її? Не придасться вона вже нінащо, хіба щоб надвір була висипана та потоптана людьми.
\end{tcolorbox}
\begin{tcolorbox}
\textsubscript{14} Ви світло для світу. Не може сховатися місто, що стоїть на верховині гори.
\end{tcolorbox}
\begin{tcolorbox}
\textsubscript{15} І не запалюють світильника, щоб поставити його під посудину, але на свічник, і світить воно всім у домі.
\end{tcolorbox}
\begin{tcolorbox}
\textsubscript{16} Отак ваше світло нехай світить перед людьми, щоб вони бачили ваші добрі діла, та прославляли Отця вашого, що на небі.
\end{tcolorbox}
\begin{tcolorbox}
\textsubscript{17} Не подумайте, ніби Я руйнувати Закон чи Пророків прийшов, Я не руйнувати прийшов, але виконати.
\end{tcolorbox}
\begin{tcolorbox}
\textsubscript{18} Поправді ж кажу вам: доки небо й земля не минеться, ані йота єдина, ані жаден значок із Закону не минеться, аж поки не збудеться все.
\end{tcolorbox}
\begin{tcolorbox}
\textsubscript{19} Хто ж порушить одну з найменших цих заповідей, та й людей так навчить, той буде найменшим у Царстві Небеснім; а хто виконає та й навчить, той стане великим у Царстві Небеснім.
\end{tcolorbox}
\begin{tcolorbox}
\textsubscript{20} Кажу бо Я вам: коли праведність ваша не буде рясніша, як книжників та фарисеїв, то не ввійдете в Царство Небесне!
\end{tcolorbox}
\begin{tcolorbox}
\textsubscript{21} Ви чули, що було стародавнім наказане: Не вбивай, а хто вб'є, підпадає він судові.
\end{tcolorbox}
\begin{tcolorbox}
\textsubscript{22} А Я вам кажу, що кожен, хто гнівається на брата свого, підпадає вже судові. А хто скаже на брата свого: рака, підпадає верховному судові, а хто скаже дурний, підпадає геєнні огненній.
\end{tcolorbox}
\begin{tcolorbox}
\textsubscript{23} Тому, коли принесеш ти до жертівника свого дара, та тут ізгадаєш, що брат твій щось має на тебе,
\end{tcolorbox}
\begin{tcolorbox}
\textsubscript{24} залиши отут дара свого перед жертівником, і піди, примирись перше з братом своїм, і тоді повертайся, і принось свого дара.
\end{tcolorbox}
\begin{tcolorbox}
\textsubscript{25} Зо своїм супротивником швидко мирися, доки з ним на дорозі ще ти, щоб тебе супротивник судді не віддав, а суддя щоб прислужникові тебе не передав, і щоб тебе до в'язниці не вкинули.
\end{tcolorbox}
\begin{tcolorbox}
\textsubscript{26} Поправді кажу тобі: Не вийдеш ізвідти, поки не віддаси ти й останнього шеляга!
\end{tcolorbox}
\begin{tcolorbox}
\textsubscript{27} Ви чули, що сказано: Не чини перелюбу.
\end{tcolorbox}
\begin{tcolorbox}
\textsubscript{28} А Я вам кажу, що кожен, хто на жінку подивиться із пожадливістю, той уже вчинив із нею перелюб у серці своїм.
\end{tcolorbox}
\begin{tcolorbox}
\textsubscript{29} Коли праве око твоє спокушає тебе, його вибери, і кинь від себе: бо краще тобі, щоб загинув один із твоїх членів, аніж до геєнни все тіло твоє було вкинене.
\end{tcolorbox}
\begin{tcolorbox}
\textsubscript{30} І як правиця твоя спокушає тебе, відітни її й кинь від себе: бо краще тобі, щоб загинув один із твоїх членів, аніж до геєнни все тіло твоє було вкинене.
\end{tcolorbox}
\begin{tcolorbox}
\textsubscript{31} Також сказано: Хто дружину свою відпускає, нехай дасть їй листа розводового.
\end{tcolorbox}
\begin{tcolorbox}
\textsubscript{32} А Я вам кажу, що кожен, хто пускає дружину свою, крім провини розпусти, той доводить її до перелюбу. І хто з відпущеною побереться, той чинить перелюб.
\end{tcolorbox}
\begin{tcolorbox}
\textsubscript{33} Ще ви чули, що було стародавнім наказане: Не клянись неправдиво, але виконуй клятви свої перед Господом.
\end{tcolorbox}
\begin{tcolorbox}
\textsubscript{34} А Я вам кажу не клястися зовсім: ані небом, бо воно престол Божий;
\end{tcolorbox}
\begin{tcolorbox}
\textsubscript{35} ні землею, бо підніжок для ніг Його це; ані Єрусалимом, бо він місто Царя Великого;
\end{tcolorbox}
\begin{tcolorbox}
\textsubscript{36} не клянись головою своєю, бо навіть однієї волосинки ти не можеш учинити білою чи чорною.
\end{tcolorbox}
\begin{tcolorbox}
\textsubscript{37} Ваше ж слово хай буде: так-так, ні-ні. А що більше над це, то те від лукавого.
\end{tcolorbox}
\begin{tcolorbox}
\textsubscript{38} Ви чули, що сказано: Око за око, і зуб за зуба.
\end{tcolorbox}
\begin{tcolorbox}
\textsubscript{39} А Я вам кажу не противитись злому. І коли вдарить тебе хто у праву щоку твою, підстав йому й другу.
\end{tcolorbox}
\begin{tcolorbox}
\textsubscript{40} А хто хоче тебе позивати й забрати сорочку твою, віддай і плаща йому.
\end{tcolorbox}
\begin{tcolorbox}
\textsubscript{41} А хто силувати тебе буде відбути подорожнє на милю одну, іди з ним навіть дві.
\end{tcolorbox}
\begin{tcolorbox}
\textsubscript{42} Хто просить у тебе то дай, а хто хоче позичити в тебе не відвертайсь від нього.
\end{tcolorbox}
\begin{tcolorbox}
\textsubscript{43} Ви чули, що сказано: Люби свого ближнього, і ненавидь свого ворога.
\end{tcolorbox}
\begin{tcolorbox}
\textsubscript{44} А Я вам кажу: Любіть ворогів своїх, благословляйте тих, хто вас проклинає, творіть добро тим, хто ненавидить вас, і моліться за тих, хто вас переслідує,
\end{tcolorbox}
\begin{tcolorbox}
\textsubscript{45} щоб вам бути синами Отця вашого, що на небі, що наказує сходити сонцю Своєму над злими й над добрими, і дощ посилає на праведних і на неправедних.
\end{tcolorbox}
\begin{tcolorbox}
\textsubscript{46} Коли бо ви любите тих, хто вас любить, то яку нагороду ви маєте? Хіба не те саме й митники роблять?
\end{tcolorbox}
\begin{tcolorbox}
\textsubscript{47} І коли ви вітаєте тільки братів своїх, то що ж особливого робите? Чи й погани не чинять отак?
\end{tcolorbox}
\begin{tcolorbox}
\textsubscript{48} Отож, будьте досконалі, як досконалий Отець ваш Небесний!
\end{tcolorbox}
\subsection{CHAPTER 6}
\begin{tcolorbox}
\textsubscript{1} Стережіться виставляти свою милостиню перед людьми, щоб бачили вас; а як ні, то не матимете нагороди від Отця вашого, що на небі.
\end{tcolorbox}
\begin{tcolorbox}
\textsubscript{2} Отож, коли чиниш ти милостиню, не сурми перед себе, як то роблять оті лицеміри по синагогах та вулицях, щоб хвалили їх люди. Поправді кажу вам: вони мають уже нагороду свою!
\end{tcolorbox}
\begin{tcolorbox}
\textsubscript{3} А як ти чиниш милостиню, хай не знатиме ліва рука твоя, що робить правиця твоя,
\end{tcolorbox}
\begin{tcolorbox}
\textsubscript{4} щоб таємна була твоя милостиня, а Отець твій, що бачить таємне, віддасть тобі явно.
\end{tcolorbox}
\begin{tcolorbox}
\textsubscript{5} А як молитеся, то не будьте, як ті лицеміри, що люблять ставати й молитися по синагогах та на перехрестях, щоб їх бачили люди. Поправді кажу вам: вони мають уже нагороду свою!
\end{tcolorbox}
\begin{tcolorbox}
\textsubscript{6} А ти, коли молишся, увійди до своєї комірчини, зачини свої двері, і помолися Отцеві своєму, що в таїні; а Отець твій, що бачить таємне, віддасть тобі явно.
\end{tcolorbox}
\begin{tcolorbox}
\textsubscript{7} А як молитеся, не проказуйте зайвого, як ті погани, бо думають, ніби вони будуть вислухані за своє велемовство.
\end{tcolorbox}
\begin{tcolorbox}
\textsubscript{8} Отож, не вподобляйтеся їм, бо знає Отець ваш, чого потребуєте, ще раніше за ваше прохання!
\end{tcolorbox}
\begin{tcolorbox}
\textsubscript{9} Ви ж моліться отак: Отче наш, що єси на небесах! Нехай святиться Ім'я Твоє,
\end{tcolorbox}
\begin{tcolorbox}
\textsubscript{10} нехай прийде Царство Твоє, нехай буде воля Твоя, як на небі, так і на землі.
\end{tcolorbox}
\begin{tcolorbox}
\textsubscript{11} Хліба нашого насущного дай нам сьогодні.
\end{tcolorbox}
\begin{tcolorbox}
\textsubscript{12} І прости нам довги наші, як і ми прощаємо винуватцям нашим.
\end{tcolorbox}
\begin{tcolorbox}
\textsubscript{13} І не введи нас у випробовування, але визволи нас від лукавого. Бо Твоє є царство, і сила, і слава навіки. Амінь.
\end{tcolorbox}
\begin{tcolorbox}
\textsubscript{14} Бо як людям ви простите прогріхи їхні, то простить і вам ваш Небесний Отець.
\end{tcolorbox}
\begin{tcolorbox}
\textsubscript{15} А коли ви не будете людям прощати, то й Отець ваш не простить вам прогріхів ваших.
\end{tcolorbox}
\begin{tcolorbox}
\textsubscript{16} А як постите, то не будьте сумні, як оті лицеміри: вони бо зміняють обличчя свої, щоб бачили люди, що постять вони. Поправді кажу вам: вони мають уже нагороду свою!
\end{tcolorbox}
\begin{tcolorbox}
\textsubscript{17} А ти, коли постиш, намасти свою голову, і лице своє вмий,
\end{tcolorbox}
\begin{tcolorbox}
\textsubscript{18} щоб ти посту свого не виявив людям, а Отцеві своєму, що в таїні; і Отець твій, що бачить таємне, віддасть тобі явно.
\end{tcolorbox}
\begin{tcolorbox}
\textsubscript{19} Не складайте скарбів собі на землі, де нищить їх міль та іржа, і де злодії підкопуються й викрадають.
\end{tcolorbox}
\begin{tcolorbox}
\textsubscript{20} Складайте ж собі скарби на небі, де ні міль, ні іржа їх не нищить, і де злодії до них не підкопуються та не крадуть.
\end{tcolorbox}
\begin{tcolorbox}
\textsubscript{21} Бо де скарб твій, там буде й серце твоє!
\end{tcolorbox}
\begin{tcolorbox}
\textsubscript{22} Око то світильник для тіла. Тож як око твоє буде здорове, то й усе тіло твоє буде світле.
\end{tcolorbox}
\begin{tcolorbox}
\textsubscript{23} А коли б твоє око лихе було, то й усе тіло твоє буде темне. Отож, коли світло, що в тобі, є темрява, то яка ж то велика та темрява!
\end{tcolorbox}
\begin{tcolorbox}
\textsubscript{24} Ніхто двом панам служити не може, бо або одного зненавидить, а другого буде любити, або буде триматись одного, а другого знехтує. Не можете Богові служити й мамоні.
\end{tcolorbox}
\begin{tcolorbox}
\textsubscript{25} Через те вам кажу: Не журіться про життя своє що будете їсти та що будете пити, ні про тіло своє, у що зодягнетеся. Чи ж не більше від їжі життя, а від одягу тіло?
\end{tcolorbox}
\begin{tcolorbox}
\textsubscript{26} Погляньте на птахів небесних, що не сіють, не жнуть, не збирають у клуні, та проте ваш Небесний Отець їх годує. Чи ж ви не багато вартніші за них?
\end{tcolorbox}
\begin{tcolorbox}
\textsubscript{27} Хто ж із вас, коли журиться, зможе додати до зросту свого бодай ліктя одного?
\end{tcolorbox}
\begin{tcolorbox}
\textsubscript{28} І про одяг чого ви клопочетесь? Погляньте на польові лілеї, як зростають вони, не працюють, ані не прядуть.
\end{tcolorbox}
\begin{tcolorbox}
\textsubscript{29} А Я вам кажу, що й сам Соломон у всій славі своїй не вдягався отак, як одна з них.
\end{tcolorbox}
\begin{tcolorbox}
\textsubscript{30} І коли польову ту траву, що сьогодні ось є, а взавтра до печі вкидається, Бог отак зодягає, скільки ж краще зодягне Він вас, маловірні!
\end{tcolorbox}
\begin{tcolorbox}
\textsubscript{31} Отож, не журіться, кажучи: Що ми будемо їсти, чи: Що будемо пити, або: У що ми зодягнемось?
\end{tcolorbox}
\begin{tcolorbox}
\textsubscript{32} Бож усього того погани шукають; але знає Отець ваш Небесний, що всього того вам потрібно.
\end{tcolorbox}
\begin{tcolorbox}
\textsubscript{33} Шукайте ж найперш Царства Божого й правди Його, а все це вам додасться.
\end{tcolorbox}
\begin{tcolorbox}
\textsubscript{34} Отож, не журіться про завтрашній день, бо завтра за себе само поклопочеться. Кожний день має досить своєї турботи!
\end{tcolorbox}
\subsection{CHAPTER 7}
\begin{tcolorbox}
\textsubscript{1} Не судіть, щоб і вас не судили;
\end{tcolorbox}
\begin{tcolorbox}
\textsubscript{2} бо яким судом судити будете, таким же осудять і вас, і якою мірою будете міряти, такою відміряють вам.
\end{tcolorbox}
\begin{tcolorbox}
\textsubscript{3} І чого в оці брата свого ти заскалку бачиш, колоди ж у власному оці не чуєш?
\end{tcolorbox}
\begin{tcolorbox}
\textsubscript{4} Або як ти скажеш до брата свого: Давай вийму я заскалку з ока твого, коли он колода у власному оці?
\end{tcolorbox}
\begin{tcolorbox}
\textsubscript{5} Лицеміре, вийми перше колоду із власного ока, а потім побачиш, як вийняти заскалку з ока брата твого.
\end{tcolorbox}
\begin{tcolorbox}
\textsubscript{6} Не давайте святого псам, і не розсипайте перел своїх перед свиньми, щоб вони не потоптали їх ногами своїми, і, обернувшись, щоб не розшматували й вас...
\end{tcolorbox}
\begin{tcolorbox}
\textsubscript{7} Просіть і буде вам дано, шукайте і знайдете, стукайте і відчинять вам;
\end{tcolorbox}
\begin{tcolorbox}
\textsubscript{8} бо кожен, хто просить одержує, хто шукає знаходить, а хто стукає відчинять йому.
\end{tcolorbox}
\begin{tcolorbox}
\textsubscript{9} Чи ж то серед вас є людина, що подасть своєму синові каменя, коли хліба проситиме він?
\end{tcolorbox}
\begin{tcolorbox}
\textsubscript{10} Або коли риби проситиме, то подасть йому гадину?
\end{tcolorbox}
\begin{tcolorbox}
\textsubscript{11} Тож як ви, бувши злі, потрапите добрі дари своїм дітям давати, скільки ж більше Отець ваш Небесний подасть добра тим, хто проситиме в Нього!
\end{tcolorbox}
\begin{tcolorbox}
\textsubscript{12} Тож усе, чого тільки бажаєте, щоб чинили вам люди, те саме чиніть їм і ви. Бо в цьому Закон і Пророки.
\end{tcolorbox}
\begin{tcolorbox}
\textsubscript{13} Увіходьте тісними ворітьми, бо просторі ворота й широка дорога, що веде до погибелі, і нею багато-хто ходять.
\end{tcolorbox}
\begin{tcolorbox}
\textsubscript{14} Бо тісні ті ворота, і вузька та дорога, що веде до життя, і мало таких, що знаходять її!
\end{tcolorbox}
\begin{tcolorbox}
\textsubscript{15} Стережіться фальшивих пророків, що приходять до вас ув одежі овечій, а всередині хижі вовки.
\end{tcolorbox}
\begin{tcolorbox}
\textsubscript{16} По їхніх плодах ви пізнаєте їх. Бо хіба ж виноград на тернині збирають, або фіґи із будяків?
\end{tcolorbox}
\begin{tcolorbox}
\textsubscript{17} Так ото родить добрі плоди кожне дерево добре, а дерево зле плоди родить лихі.
\end{tcolorbox}
\begin{tcolorbox}
\textsubscript{18} Не може родить добре дерево плоду лихого, ані дерево зле плодів добрих родити.
\end{tcolorbox}
\begin{tcolorbox}
\textsubscript{19} Усяке ж дерево, що доброго плоду не родить, зрубується та в огонь укидається.
\end{tcolorbox}
\begin{tcolorbox}
\textsubscript{20} Ото ж бо, по їхніх плодах ви пізнаєте їх!
\end{tcolorbox}
\begin{tcolorbox}
\textsubscript{21} Не кожен, хто каже до Мене: Господи, Господи! увійде в Царство Небесне, але той, хто виконує волю Мого Отця, що на небі.
\end{tcolorbox}
\begin{tcolorbox}
\textsubscript{22} Багато-хто скажуть Мені того дня: Господи, Господи, хіба ми не Ім'ям Твоїм пророкували, хіба не Ім'ям Твоїм демонів ми виганяли, або не Ім'ям Твоїм чуда великі творили?
\end{tcolorbox}
\begin{tcolorbox}
\textsubscript{23} І їм оголошу Я тоді: Я ніколи не знав вас... Відійдіть від Мене, хто чинить беззаконня!
\end{tcolorbox}
\begin{tcolorbox}
\textsubscript{24} Отож, кожен, хто слухає цих Моїх слів і виконує їх, подібний до чоловіка розумного, що свій дім збудував на камені.
\end{tcolorbox}
\begin{tcolorbox}
\textsubscript{25} І линула злива, і розлилися річки, і буря знялася, і на дім отой кинулась, та не впав, бо на камені був він заснований.
\end{tcolorbox}
\begin{tcolorbox}
\textsubscript{26} А кожен, хто слухає цих Моїх слів, та їх не виконує, подібний до чоловіка того необачного, що свій дім збудував на піску.
\end{tcolorbox}
\begin{tcolorbox}
\textsubscript{27} І линула злива, і розлилися річки, і буря знялася й на дім отой кинулась, і він упав. І велика була та руїна його!
\end{tcolorbox}
\begin{tcolorbox}
\textsubscript{28} І ото, як Ісус закінчив ці слова, то народ дивувався з науки Його.
\end{tcolorbox}
\begin{tcolorbox}
\textsubscript{29} Бо навчав Він їх, як можновладний, а не як ті книжники їхні.
\end{tcolorbox}
\subsection{CHAPTER 8}
\begin{tcolorbox}
\textsubscript{1} А коли Він зійшов із гори, услід за Ним ішов натовп великий.
\end{tcolorbox}
\begin{tcolorbox}
\textsubscript{2} І ось підійшов прокажений, уклонився Йому та й сказав: Коли, Господи, хочеш, Ти можеш очистити мене!
\end{tcolorbox}
\begin{tcolorbox}
\textsubscript{3} А Ісус простяг руку, і доторкнувся до нього, говорячи: Хочу, будь чистий! І тієї хвилини очистився той від своєї прокази.
\end{tcolorbox}
\begin{tcolorbox}
\textsubscript{4} І говорить до нього Ісус: Гляди, не розповідай нікому. Але йди, покажися священикові, та дар принеси, якого Мойсей заповів, їм на свідоцтво.
\end{tcolorbox}
\begin{tcolorbox}
\textsubscript{5} А коли Він до Капернауму ввійшов, то до Нього наблизився сотник, та й благати зачав Його,
\end{tcolorbox}
\begin{tcolorbox}
\textsubscript{6} кажучи: Господи, мій слуга лежить удома розслаблений, і тяжко страждає.
\end{tcolorbox}
\begin{tcolorbox}
\textsubscript{7} Він говорить йому: Я прийду й уздоровлю його.
\end{tcolorbox}
\begin{tcolorbox}
\textsubscript{8} А сотник Йому відповів: Недостойний я, Господи, щоб зайшов Ти під стріху мою... Та промов тільки слово, і видужає мій слуга!
\end{tcolorbox}
\begin{tcolorbox}
\textsubscript{9} Бо й я людина підвладна, і вояків під собою я маю; і одному кажу: піди то йде він, а тому: прийди і приходить, або рабові своєму: зроби те і він зробить.
\end{tcolorbox}
\begin{tcolorbox}
\textsubscript{10} Почувши таке, Ісус здивувався, і промовив до тих, хто йшов услід за Ним: Поправді кажу вам: навіть серед Ізраїля Я не знайшов був такої великої віри!
\end{tcolorbox}
\begin{tcolorbox}
\textsubscript{11} Кажу ж вам, що багато-хто прийдуть від сходу та заходу, і засядуть у Царстві Небеснім із Авраамом, Ісаком та Яковом.
\end{tcolorbox}
\begin{tcolorbox}
\textsubscript{12} Сини ж Царства повкидані будуть до темряви зовнішньої буде там плач і скрегіт зубів!...
\end{tcolorbox}
\begin{tcolorbox}
\textsubscript{13} І сказав Ісус сотникові: Іди, і як повірив ти, нехай так тобі й станеться! І тієї ж години одужав слуга його.
\end{tcolorbox}
\begin{tcolorbox}
\textsubscript{14} Як прийшов же Ісус до Петрового дому, то побачив тещу його, що лежала в гарячці.
\end{tcolorbox}
\begin{tcolorbox}
\textsubscript{15} І Він доторкнувся руки її, і гарячка покинула ту... І встала вона, та й Йому прислуговувала!
\end{tcolorbox}
\begin{tcolorbox}
\textsubscript{16} А коли настав вечір, привели багатьох біснуватих до Нього, і Він словом Своїм вигнав духів, а недужих усіх уздоровив,
\end{tcolorbox}
\begin{tcolorbox}
\textsubscript{17} щоб справдилося, що сказав був Ісая пророк, промовляючи: Він узяв наші немочі, і недуги поніс.
\end{tcolorbox}
\begin{tcolorbox}
\textsubscript{18} А як угледів Ісус навколо Себе багато народу, наказав переплинути на той бік.
\end{tcolorbox}
\begin{tcolorbox}
\textsubscript{19} І приступив один книжник та й до Нього сказав: Учителю, я піду за Тобою, хоч би куди ти пішов!
\end{tcolorbox}
\begin{tcolorbox}
\textsubscript{20} Промовляє до нього Ісус: Мають нори лисиці, а гнізда небесні пташки, Син же Людський не має де й голови прихилити...
\end{tcolorbox}
\begin{tcolorbox}
\textsubscript{21} А інший із учнів промовив до Нього: Дозволь мені, Господи, перше піти та батька свого поховати.
\end{tcolorbox}
\begin{tcolorbox}
\textsubscript{22} А Ісус йому каже: Іди за Мною, і зостав мертвим ховати мерців своїх!
\end{tcolorbox}
\begin{tcolorbox}
\textsubscript{23} І коли Він до човна вступив, за Ним увійшли Його учні.
\end{tcolorbox}
\begin{tcolorbox}
\textsubscript{24} І ось буря велика зірвалась на морі, аж човен зачав заливатися хвилями. А Він спав...
\end{tcolorbox}
\begin{tcolorbox}
\textsubscript{25} І кинулись учні, і збудили Його та й благали: Рятуй, Господи, гинемо!
\end{tcolorbox}
\begin{tcolorbox}
\textsubscript{26} А Він відповів їм: Чого полохливі ви, маловірні? Тоді встав, заказав бурі й морю, і тиша велика настала...
\end{tcolorbox}
\begin{tcolorbox}
\textsubscript{27} А народ дивувався й казав: Хто ж це такий, що вітри та море слухняні Йому?
\end{tcolorbox}
\begin{tcolorbox}
\textsubscript{28} І, як прибув Він на той бік, до землі Гадаринської, перестріли Його два біснуваті, що вийшли з могильних печер, дуже люті, так що ніхто не міг переходити тією дорогою.
\end{tcolorbox}
\begin{tcolorbox}
\textsubscript{29} І ось, вони стали кричати, говорячи: Що Тобі, Сину Божий, до нас? Прийшов Ти сюди передчасно нас мучити?
\end{tcolorbox}
\begin{tcolorbox}
\textsubscript{30} А оподаль від них пасся гурт великий свиней.
\end{tcolorbox}
\begin{tcolorbox}
\textsubscript{31} І просилися демони, кажучи: Коли виженеш нас, то пошли нас у той гурт свиней.
\end{tcolorbox}
\begin{tcolorbox}
\textsubscript{32} А Він відповів їм: Ідіть. І вийшли вони, і пішли в гурт свиней. І ось кинувся з кручі до моря ввесь гурт, і потопився в воді.
\end{tcolorbox}
\begin{tcolorbox}
\textsubscript{33} Пастухи ж повтікали; а коли прибули вони в місто, то про все розповіли, і про біснуватих.
\end{tcolorbox}
\begin{tcolorbox}
\textsubscript{34} І ось, усе місто вийшло назустріч Ісусові. Як Його ж угледіли, то стали благати, щоб пішов Собі з їхнього краю!..
\end{tcolorbox}
\subsection{CHAPTER 9}
\begin{tcolorbox}
\textsubscript{1} І, сівши до човна, Він переплинув, і до міста Свого прибув.
\end{tcolorbox}
\begin{tcolorbox}
\textsubscript{2} І ото, принесли до Нього розслабленого, що на ложі лежав. І, як побачив Ісус їхню віру, сказав розслабленому: Будь бадьорий, сину! Прощаються тобі гріхи твої!
\end{tcolorbox}
\begin{tcolorbox}
\textsubscript{3} І ось, дехто із книжників стали казати про себе: Він богозневажає.
\end{tcolorbox}
\begin{tcolorbox}
\textsubscript{4} Ісус же думки їхні знав і сказав: Чого думаєте ви лукаве в серцях своїх?
\end{tcolorbox}
\begin{tcolorbox}
\textsubscript{5} Що легше, сказати: Прощаються тобі гріхи, чи сказати: Уставай та й ходи?
\end{tcolorbox}
\begin{tcolorbox}
\textsubscript{6} Але щоб ви знали, що прощати гріхи на землі має владу Син Людський, тож каже Він розслабленому: Уставай, візьми ложе своє, та й іди у свій дім!
\end{tcolorbox}
\begin{tcolorbox}
\textsubscript{7} Той устав і пішов у свій дім.
\end{tcolorbox}
\begin{tcolorbox}
\textsubscript{8} А натовп, побачивши це, налякався, і славив Бога, що людям Він дав таку владу!...
\end{tcolorbox}
\begin{tcolorbox}
\textsubscript{9} А коли Ісус звідти проходив, побачив чоловіка, на ймення Матвія, що сидів на митниці, та й каже йому: Іди за Мною! Той устав, і пішов услід за Ним.
\end{tcolorbox}
\begin{tcolorbox}
\textsubscript{10} І сталось, як Ісус сидів при столі у домі, ось зійшлося багато митників і грішників, і вони посідали з Ним та з Його учнями.
\end{tcolorbox}
\begin{tcolorbox}
\textsubscript{11} Як побачили ж те фарисеї, то сказали до учнів Його: Чому то Вчитель ваш їсть із митниками та із грішниками?
\end{tcolorbox}
\begin{tcolorbox}
\textsubscript{12} А Він це почув та й сказав: Лікаря не потребують здорові, а слабі!
\end{tcolorbox}
\begin{tcolorbox}
\textsubscript{13} Ідіть же, і навчіться, що то є: Милости хочу, а не жертви. Бо Я не прийшов кликати праведних, але грішників до покаяння.
\end{tcolorbox}
\begin{tcolorbox}
\textsubscript{14} Тоді приступили до Нього Іванові учні та й кажуть: Чому постимо ми й фарисеї, а учні Твої не постять?
\end{tcolorbox}
\begin{tcolorbox}
\textsubscript{15} Ісус же промовив до них: Хіба можуть гості весільні сумувати, поки з ними ще є молодий? Але прийдуть ті дні, коли заберуть молодого від них, тоді й постити будуть вони.
\end{tcolorbox}
\begin{tcolorbox}
\textsubscript{16} До одежі ж старої ніхто не вставляє латки з сукна сирового, бо збіжиться воно, і дірка стане ще гірша.
\end{tcolorbox}
\begin{tcolorbox}
\textsubscript{17} І не вливають вина молодого в старі бурдюки, а то бурдюки розірвуться, і вино розіллється, і бурдюки пропадуть; а вливають вино молоде до нових бурдюків, і одне й друге збережено буде.
\end{tcolorbox}
\begin{tcolorbox}
\textsubscript{18} Коли Він говорив це до них, підійшов ось один із старших, уклонився Йому та й говорить: Дочка моя хвилі цієї померла. Та прийди, поклади Свою руку на неї, і вона оживе.
\end{tcolorbox}
\begin{tcolorbox}
\textsubscript{19} І підвівся Ісус, і пішов услід за ним, також учні Його.
\end{tcolorbox}
\begin{tcolorbox}
\textsubscript{20} І ото одна жінка, що дванадцять літ хворою на кровотечу була, приступила ззаду, і доторкнулась до краю одежі Його.
\end{tcolorbox}
\begin{tcolorbox}
\textsubscript{21} Бо вона говорила про себе: Коли хоч доторкнуся одежі Його, то одужаю.
\end{tcolorbox}
\begin{tcolorbox}
\textsubscript{22} Ісус, обернувшись, побачив її та й сказав: Будь бадьорою, дочко, твоя віра спасла тебе! І одужала жінка з тієї години.
\end{tcolorbox}
\begin{tcolorbox}
\textsubscript{23} А Ісус, як прибув до господи старшого, і вздрів дударів та юрбу голосільників,
\end{tcolorbox}
\begin{tcolorbox}
\textsubscript{24} то сказав: Відійдіть, бо не вмерло дівча, але спить. І насміхалися з Нього.
\end{tcolorbox}
\begin{tcolorbox}
\textsubscript{25} А коли народ випроваджено, Він увійшов, узяв за руку її, і дівчина встала!
\end{tcolorbox}
\begin{tcolorbox}
\textsubscript{26} І вістка про це розійшлася по всій тій країні.
\end{tcolorbox}
\begin{tcolorbox}
\textsubscript{27} Коли ж Ісус звідти вертався, ішли за Ним два сліпці, що кричали й казали: Змилуйсь над нами, Сину Давидів!
\end{tcolorbox}
\begin{tcolorbox}
\textsubscript{28} І коли Він додому прийшов, приступили до Нього сліпці. А Ісус до них каже: Чи ж вірите ви, що Я можу вчинити оце? Говорять до Нього вони: Так, Господи.
\end{tcolorbox}
\begin{tcolorbox}
\textsubscript{29} Тоді Він доторкнувся до їхніх очей і сказав: Нехай станеться вам згідно з вашою вірою!
\end{tcolorbox}
\begin{tcolorbox}
\textsubscript{30} І очі відкрилися їм. А Ісус наказав їм суворо, говорячи: Глядіть, щоб ніхто не довідавсь про це!
\end{tcolorbox}
\begin{tcolorbox}
\textsubscript{31} А вони відійшли, та й розголосили про Нього по всій тій країні.
\end{tcolorbox}
\begin{tcolorbox}
\textsubscript{32} Коли ж ті виходили, то ось привели до Нього чоловіка німого, що був біснуватий.
\end{tcolorbox}
\begin{tcolorbox}
\textsubscript{33} І як демон був вигнаний, німий заговорив. І дивувався народ і казав: Ніколи таке не траплялося серед Ізраїля!
\end{tcolorbox}
\begin{tcolorbox}
\textsubscript{34} Фарисеї ж казали: Виганяє Він демонів силою князя демонів.
\end{tcolorbox}
\begin{tcolorbox}
\textsubscript{35} І обходив Ісус всі міста та оселі, навчаючи в їхніх синагогах, та Євангелію Царства проповідуючи, і вздоровлюючи всяку недугу та неміч усяку.
\end{tcolorbox}
\begin{tcolorbox}
\textsubscript{36} А як бачив людей, змилосерджувався Він над ними, бо були вони змучені та розпорошені, як ті вівці, що не мають пастуха.
\end{tcolorbox}
\begin{tcolorbox}
\textsubscript{37} Тоді Він казав Своїм учням: Жниво справді велике, та робітників мало;
\end{tcolorbox}
\begin{tcolorbox}
\textsubscript{38} тож благайте Господаря жнива, щоб на жниво Своє Він робітників вислав.
\end{tcolorbox}
\subsection{CHAPTER 10}
\begin{tcolorbox}
\textsubscript{1} І закликав Він дванадцятьох Своїх учнів, і владу їм дав над нечистими духами, щоб їх виганяли вони, і щоб уздоровляли всіляку недугу та неміч всіляку.
\end{tcolorbox}
\begin{tcolorbox}
\textsubscript{2} А ймення апостолів дванадцятьох отакі: перший Симон, що Петром прозивається, і Андрій, брат його; Яків, син Зеведеїв, та Іван, брат його;
\end{tcolorbox}
\begin{tcolorbox}
\textsubscript{3} Пилип і Варфоломій, Хома й митник Матвій; Яків, син Алфеїв, і Тадей;
\end{tcolorbox}
\begin{tcolorbox}
\textsubscript{4} Симон Кананіт, та Юда Іскаріотський, що й видав Його.
\end{tcolorbox}
\begin{tcolorbox}
\textsubscript{5} Цих Дванадцятьох Ісус вислав, і їм наказав, промовляючи: На путь до поган не ходіть, і до самарянського міста не входьте,
\end{tcolorbox}
\begin{tcolorbox}
\textsubscript{6} але йдіть радніш до овечок загинулих дому Ізраїлевого.
\end{tcolorbox}
\begin{tcolorbox}
\textsubscript{7} А ходячи, проповідуйте та говоріть, що наблизилось Царство Небесне.
\end{tcolorbox}
\begin{tcolorbox}
\textsubscript{8} Уздоровляйте недужих, воскрешайте померлих, очищайте прокажених, виганяйте демонів. Ви дармо дістали, дармо й давайте.
\end{tcolorbox}
\begin{tcolorbox}
\textsubscript{9} Не беріть ані золота, ані срібла, ані мідяків до своїх поясів,
\end{tcolorbox}
\begin{tcolorbox}
\textsubscript{10} ані торби в дорогу, ані двох одеж, ні сандаль, ані палиці. Бо вартий робітник своєї поживи.
\end{tcolorbox}
\begin{tcolorbox}
\textsubscript{11} А як зайдете в місто якесь чи в село, то розвідайте, хто там достойний, і там перебудьте, аж поки не вийдете.
\end{tcolorbox}
\begin{tcolorbox}
\textsubscript{12} А входячи в дім, вітайте його, промовляючи: Мир дому цьому!
\end{tcolorbox}
\begin{tcolorbox}
\textsubscript{13} І коли буде достойний той дім, нехай зійде на нього ваш мир; а як недостойний він буде, то мир ваш нехай до вас вернеться.
\end{tcolorbox}
\begin{tcolorbox}
\textsubscript{14} А як хто вас не прийме, і ваших слів не послухає, то, виходячи з дому чи з міста того, обтрусіть порох із ніг своїх.
\end{tcolorbox}
\begin{tcolorbox}
\textsubscript{15} Поправді кажу вам: легше буде країні содомській й гоморській дня судного, аніж місту тому!
\end{tcolorbox}
\begin{tcolorbox}
\textsubscript{16} Оце посилаю Я вас, як овець між вовки. Будьте ж мудрі, як змії, і невинні, як голубки.
\end{tcolorbox}
\begin{tcolorbox}
\textsubscript{17} Стережіться ж людей, бо вони на суди видаватимуть вас, та по синагогах своїх бичувати вас будуть.
\end{tcolorbox}
\begin{tcolorbox}
\textsubscript{18} І до правителів та до царів поведуть вас за Мене, на свідчення їм і поганам.
\end{tcolorbox}
\begin{tcolorbox}
\textsubscript{19} А коли видаватимуть вас, не журіться, як або що говорити: тієї години буде вам дане, що маєте ви говорити,
\end{tcolorbox}
\begin{tcolorbox}
\textsubscript{20} бо не ви промовлятимете, але Дух Отця вашого в вас промовлятиме.
\end{tcolorbox}
\begin{tcolorbox}
\textsubscript{21} І видасть на смерть брата брат, а батько дитину. І діти повстануть супроти батьків, і їх повбивають.
\end{tcolorbox}
\begin{tcolorbox}
\textsubscript{22} І за Ім'я Моє будуть усі вас ненавидіти. А хто витерпить аж до кінця, той буде спасений.
\end{tcolorbox}
\begin{tcolorbox}
\textsubscript{23} А коли будуть вас переслідувати в однім місті, утікайте до іншого. Поправді кажу вам, не встигнете ви обійти міст Ізраїлевих, як прийде Син Людський.
\end{tcolorbox}
\begin{tcolorbox}
\textsubscript{24} Учень не більший за вчителя, а раб понад пана свого.
\end{tcolorbox}
\begin{tcolorbox}
\textsubscript{25} Доволі для учня, коли буде він, як учитель його, а раб як господар його. Коли Вельзевулом назвали господаря дому, скільки ж більше назвуть так домашніх його!
\end{tcolorbox}
\begin{tcolorbox}
\textsubscript{26} Але не лякайтеся їх. Немає нічого захованого, що воно не відкриється, ані потаємного, що не виявиться.
\end{tcolorbox}
\begin{tcolorbox}
\textsubscript{27} Що кажу Я вам потемки, говоріть те при світлі, що ж на вухо ви чуєте проповідуйте те на дахах.
\end{tcolorbox}
\begin{tcolorbox}
\textsubscript{28} І не лякайтеся тих, хто тіло вбиває, а душі вбити не може; але бійтеся більше того, хто може й душу, і тіло вам занапастити в геєнні.
\end{tcolorbox}
\begin{tcolorbox}
\textsubscript{29} Чи не два горобці продаються за гріш? А на землю із них ні один не впаде без волі Отця вашого.
\end{tcolorbox}
\begin{tcolorbox}
\textsubscript{30} А вам і волосся все на голові пораховано.
\end{tcolorbox}
\begin{tcolorbox}
\textsubscript{31} Отож, не лякайтесь, бо вартніші ви за багатьох горобців.
\end{tcolorbox}
\begin{tcolorbox}
\textsubscript{32} Отже, кожного, хто Мене визнає перед людьми, того перед Небесним Отцем Моїм визнаю й Я.
\end{tcolorbox}
\begin{tcolorbox}
\textsubscript{33} Хто ж Мене відцурається перед людьми, того й Я відцураюся перед Небесним Отцем Моїм.
\end{tcolorbox}
\begin{tcolorbox}
\textsubscript{34} Не думайте, що Я прийшов, щоб мир на землю принести, Я не мир принести прийшов, а меча.
\end{tcolorbox}
\begin{tcolorbox}
\textsubscript{35} Я ж прийшов порізнити чоловіка з батьком його, дочку з її матір'ю, і невістку з свекрухою її.
\end{tcolorbox}
\begin{tcolorbox}
\textsubscript{36} І: вороги чоловікові домашні його!
\end{tcolorbox}
\begin{tcolorbox}
\textsubscript{37} Хто більш, як Мене, любить батька чи матір, той Мене недостойний. І хто більш, як Мене, любить сина чи дочку, той Мене недостойний.
\end{tcolorbox}
\begin{tcolorbox}
\textsubscript{38} І хто не візьме свого хреста, і не піде за Мною слідом, той Мене недостойний.
\end{tcolorbox}
\begin{tcolorbox}
\textsubscript{39} Хто душу свою зберігає, той погубить її, хто ж за Мене погубить душу свою, той знайде її.
\end{tcolorbox}
\begin{tcolorbox}
\textsubscript{40} Хто вас приймає приймає Мене, хто ж приймає Мене, приймає Того, Хто послав Мене.
\end{tcolorbox}
\begin{tcolorbox}
\textsubscript{41} Хто приймає пророка, як пророка, той дістане нагороду пророчу, хто ж приймає праведника, як праведника, той дістане нагороду праведничу.
\end{tcolorbox}
\begin{tcolorbox}
\textsubscript{42} І хто напоїть, як учня, кого з малих цих бодай кухлем водиці холодної, поправді кажу вам, той не згубить нагороди своєї.
\end{tcolorbox}
\subsection{CHAPTER 11}
\begin{tcolorbox}
\textsubscript{1} І сталось, коли Ісус перестав навчати дванадцятьох Своїх учнів, Він звідти пішов, щоб учити, і по їхніх містах проповідувати.
\end{tcolorbox}
\begin{tcolorbox}
\textsubscript{2} Прочувши ж Іван у в'язниці про дії Христові, послав через учнів своїх,
\end{tcolorbox}
\begin{tcolorbox}
\textsubscript{3} щоб Його запитати: Чи Ти Той, Хто має прийти, чи чекати нам Іншого?
\end{tcolorbox}
\begin{tcolorbox}
\textsubscript{4} Ісус же промовив у відповідь їм: Ідіть, і перекажіть Іванові, що ви чуєте й бачите:
\end{tcolorbox}
\begin{tcolorbox}
\textsubscript{5} Сліпі прозрівають, і криві ходять, стають чистими прокажені, і чують глухі, і померлі встають, а вбогим звіщається Добра Новина...
\end{tcolorbox}
\begin{tcolorbox}
\textsubscript{6} І блаженний, хто через Мене спокуси не матиме!
\end{tcolorbox}
\begin{tcolorbox}
\textsubscript{7} Як вони ж відійшли, Ісус про Івана почав говорити народові: На що ви ходили в пустиню дивитися? Чи на очерет, що вітер гойдає його?
\end{tcolorbox}
\begin{tcolorbox}
\textsubscript{8} Та на що ви дивитись ходили? Може на чоловіка, у м'які шати одягненого? Аджеж ті, хто носить м'яке, по палатах царських.
\end{tcolorbox}
\begin{tcolorbox}
\textsubscript{9} По що ж ви ходили? Може бачити пророка? Так, кажу вам, навіть більш, як пророка.
\end{tcolorbox}
\begin{tcolorbox}
\textsubscript{10} Бо це ж той, що про нього написано: Ось перед обличчя Твоє посилаю Свого посланця, який перед Тобою дорогу Твою приготує!
\end{tcolorbox}
\begin{tcolorbox}
\textsubscript{11} Поправді кажу вам: Між народженими від жінок не було більшого над Івана Христителя! Та найменший у Царстві Небеснім той більший від нього!
\end{tcolorbox}
\begin{tcolorbox}
\textsubscript{12} Від днів же Івана Христителя й досі Царство Небесне здобувається силою, і ті, хто вживає зусилля, хапають його.
\end{tcolorbox}
\begin{tcolorbox}
\textsubscript{13} Усі бо Пророки й Закон до Івана провіщували.
\end{tcolorbox}
\begin{tcolorbox}
\textsubscript{14} Коли ж хочете знати, то Ілля він, що має прийти.
\end{tcolorbox}
\begin{tcolorbox}
\textsubscript{15} Хто має вуха, нехай слухає!
\end{tcolorbox}
\begin{tcolorbox}
\textsubscript{16} До кого ж цей рід прирівняю? До хлоп'ят він подібний, що на ринку сидять та вигукують іншим,
\end{tcolorbox}
\begin{tcolorbox}
\textsubscript{17} і кажуть: Ми вам грали, а ви не танцювали, ми співали вам жалібно, та не плакали ви...
\end{tcolorbox}
\begin{tcolorbox}
\textsubscript{18} Бо прийшов був Іван, що не їв і не пив, вони ж кажуть: Він демона має.
\end{tcolorbox}
\begin{tcolorbox}
\textsubscript{19} Прийшов же Син Людський, що їсть і п'є, вони ж кажуть: Чоловік ось, ласун і п'яниця, Він приятель митників і грішників. І виправдалася мудрість своїми ділами.
\end{tcolorbox}
\begin{tcolorbox}
\textsubscript{20} Ісус тоді став докоряти містам, де відбулося найбільш Його чуд, що вони не покаялись:
\end{tcolorbox}
\begin{tcolorbox}
\textsubscript{21} Горе тобі, Хоразіне, горе тобі, Віфсаїдо! Бо коли б то в Тирі й Сидоні були відбулися ті чуда, що сталися в вас, то давно б вони каялися в волосяниці та в попелі.
\end{tcolorbox}
\begin{tcolorbox}
\textsubscript{22} Але кажу вам: Легше буде дня судного Тиру й Сидону, ніж вам!
\end{tcolorbox}
\begin{tcolorbox}
\textsubscript{23} А ти, Капернауме, що до неба піднісся, аж до аду ти зійдеш. Бо коли б у Содомі були відбулися ті чуда, що в тобі вони стались, то лишився б він був по сьогоднішній день.
\end{tcolorbox}
\begin{tcolorbox}
\textsubscript{24} Але кажу вам, що содомській землі буде легше дня судного, аніж тобі!...
\end{tcolorbox}
\begin{tcolorbox}
\textsubscript{25} Того часу, навчаючи, промовив Ісус: Прославляю Тебе, Отче, Господи неба й землі, що втаїв Ти оце від премудрих і розумних, та його немовлятам відкрив.
\end{tcolorbox}
\begin{tcolorbox}
\textsubscript{26} Так, Отче, бо Тобі так було до вподоби!
\end{tcolorbox}
\begin{tcolorbox}
\textsubscript{27} Передав Мені все Мій Отець. І Сина не знає ніхто, крім Отця, і Отця не знає ніхто, окрім Сина, та кому Син захоче відкрити.
\end{tcolorbox}
\begin{tcolorbox}
\textsubscript{28} Прийдіть до Мене, усі струджені та обтяжені, і Я вас заспокою!
\end{tcolorbox}
\begin{tcolorbox}
\textsubscript{29} Візьміть на себе ярмо Моє, і навчіться від Мене, бо Я тихий і серцем покірливий, і знайдете спокій душам своїм.
\end{tcolorbox}
\begin{tcolorbox}
\textsubscript{30} Бож ярмо Моє любе, а тягар Мій легкий!
\end{tcolorbox}
\subsection{CHAPTER 12}
\begin{tcolorbox}
\textsubscript{1} Того часу Ісус переходив ланами в суботу. А учні Його зголодніли були, і стали зривати колосся та їсти.
\end{tcolorbox}
\begin{tcolorbox}
\textsubscript{2} Побачили ж це фарисеї, та й кажуть Йому: Он учні Твої роблять те, чого не годиться робити в суботу...
\end{tcolorbox}
\begin{tcolorbox}
\textsubscript{3} А Він відповів їм: Чи ж ви не читали, що зробив був Давид, коли сам зголоднів і ті, хто був із ним?
\end{tcolorbox}
\begin{tcolorbox}
\textsubscript{4} Як він увійшов до Божого дому, і спожив хліби показні, яких їсти не можна було ні йому, ані тим, хто був із ним, а тільки самим священикам?
\end{tcolorbox}
\begin{tcolorbox}
\textsubscript{5} Або ви не читали в Законі що в суботу священики порушують суботу у храмі, і невинні вони?
\end{tcolorbox}
\begin{tcolorbox}
\textsubscript{6} А Я вам кажу, що тут Більший, як храм!
\end{tcolorbox}
\begin{tcolorbox}
\textsubscript{7} Коли б знали ви, що то є: Милости хочу, а не жертви, то ви не судили б невинних...
\end{tcolorbox}
\begin{tcolorbox}
\textsubscript{8} Бо Син Людський Господь і суботі!
\end{tcolorbox}
\begin{tcolorbox}
\textsubscript{9} І, вийшовши звідти, прибув Він до їхньої синагоги.
\end{tcolorbox}
\begin{tcolorbox}
\textsubscript{10} І ото, був там чоловік, що мав суху руку. І, щоб обвинити Ісуса, запитали Його: Чи вздоровляти годиться в суботу?
\end{tcolorbox}
\begin{tcolorbox}
\textsubscript{11} А Він їм сказав: Чи знайдеться між вами людина, яка, одну мавши вівцю, не піде по неї, і не врятує її, як вона впаде в яму в суботу?
\end{tcolorbox}
\begin{tcolorbox}
\textsubscript{12} А скільки ж людина вартніша за тую овечку! Тому можна чинити добро й у суботу!
\end{tcolorbox}
\begin{tcolorbox}
\textsubscript{13} І каже тоді чоловікові: Простягни свою руку! Той простяг, і стала здорова вона, як і друга...
\end{tcolorbox}
\begin{tcolorbox}
\textsubscript{14} Фарисеї ж пішли, і зібрали нараду на Нього, як би Його погубити?...
\end{tcolorbox}
\begin{tcolorbox}
\textsubscript{15} А Ісус, розізнавши, пішов Собі звідти. І багато пішло вслід за Ним, і Він їх уздоровив усіх.
\end{tcolorbox}
\begin{tcolorbox}
\textsubscript{16} А Він наказав їм суворо Його не виявляти,
\end{tcolorbox}
\begin{tcolorbox}
\textsubscript{17} щоб справдилось те, що сказав був Ісая пророк, промовляючи:
\end{tcolorbox}
\begin{tcolorbox}
\textsubscript{18} Ото Мій Отрок, що Я вибрав Його, Мій Улюблений, що Його полюбила душа Моя! Вкладу Свого Духа в Нього, і Він суд проголосить поганам.
\end{tcolorbox}
\begin{tcolorbox}
\textsubscript{19} Він не буде змагатися, ані кричати, і на вулицях чути не буде ніхто Його голосу.
\end{tcolorbox}
\begin{tcolorbox}
\textsubscript{20} Він очеретини надломленої не доломить, і ґнота догасаючого не погасить, поки не допровадить присуду до перемоги...
\end{tcolorbox}
\begin{tcolorbox}
\textsubscript{21} І погани надіятись будуть на Ймення Його!
\end{tcolorbox}
\begin{tcolorbox}
\textsubscript{22} Тоді привели до Нього німого сліпця, що був біснуватий, і Він уздоровив його, так що німий став говорити та бачити.
\end{tcolorbox}
\begin{tcolorbox}
\textsubscript{23} І дивувались усі люди й казали: Чи ж не Син це Давидів?
\end{tcolorbox}
\begin{tcolorbox}
\textsubscript{24} Фарисеї ж, почувши, сказали: Він демонів не виганяє інакше, тільки як Вельзевулом, князем демонів.
\end{tcolorbox}
\begin{tcolorbox}
\textsubscript{25} А Він знав думки їхні, і промовив до них: Кожне царство, поділене супроти себе, запустіє. І кожне місто чи дім, поділені супроти себе, не втримаються.
\end{tcolorbox}
\begin{tcolorbox}
\textsubscript{26} І коли сатана сатану виганяє, то ділиться супроти себе; як же втримається царство його?
\end{tcolorbox}
\begin{tcolorbox}
\textsubscript{27} І коли Вельзевулом виганяю Я демонів, то ким виганяють сини ваші? Тому вони стануть вам суддями.
\end{tcolorbox}
\begin{tcolorbox}
\textsubscript{28} А коли ж Духом Божим вигоню Я демонів, то настало для вас Царство Боже.
\end{tcolorbox}
\begin{tcolorbox}
\textsubscript{29} Або як то хто може вдертися в дім дужого, та пограбувати добро його, якщо перше не зв'яже дужого? І аж тоді він господу його пограбує.
\end{tcolorbox}
\begin{tcolorbox}
\textsubscript{30} Хто не зо Мною, той супроти Мене; і хто не збирає зо Мною, той розкидає.
\end{tcolorbox}
\begin{tcolorbox}
\textsubscript{31} Тому то кажу вам: усякий гріх, навіть богозневага, проститься людям, але богозневага на Духа не проститься!
\end{tcolorbox}
\begin{tcolorbox}
\textsubscript{32} І як скаже хто слово на Людського Сина, то йому проститься те; а коли скаже проти Духа Святого, не проститься того йому ані в цім віці, ані в майбутнім!
\end{tcolorbox}
\begin{tcolorbox}
\textsubscript{33} Або виростіть дерево добре, то й плід його добрий, або виростіть дерево зле, то й плід його злий. Пізнається бо дерево з плоду!
\end{tcolorbox}
\begin{tcolorbox}
\textsubscript{34} Роде зміїний! Як ви можете мовити добре, бувши злі? Бо чим серце наповнене, те говорять уста.
\end{tcolorbox}
\begin{tcolorbox}
\textsubscript{35} Добра людина з доброго скарбу добре виносить, а лукава людина зо скарбу лихого виносить лихе.
\end{tcolorbox}
\begin{tcolorbox}
\textsubscript{36} Кажу ж вам, що за кожне слово пусте, яке скажуть люди, дадуть вони відповідь судного дня!
\end{tcolorbox}
\begin{tcolorbox}
\textsubscript{37} Бо зо слів своїх будеш виправданий, і зо слів своїх будеш засуджений.
\end{tcolorbox}
\begin{tcolorbox}
\textsubscript{38} Тоді дехто із книжників та фарисеїв озвались до Нього й сказали: Учителю, хочемо побачити ознаку від Тебе.
\end{tcolorbox}
\begin{tcolorbox}
\textsubscript{39} А Ісус відповів їм: Рід лукавий і перелюбний шукає ознаки, та ознаки йому не дадуть, окрім ознаки пророка Йони.
\end{tcolorbox}
\begin{tcolorbox}
\textsubscript{40} Як Йона перебув у середині китовій три дні і три ночі, так перебуде три дні та три ночі й Син Людський у серці землі.
\end{tcolorbox}
\begin{tcolorbox}
\textsubscript{41} Ніневітяни стануть на суд із цим родом, і осудять його, вони бо покаялися через Йонину проповідь. А тут ото Більший, ніж Йона!
\end{tcolorbox}
\begin{tcolorbox}
\textsubscript{42} Цариця з півдня на суд стане з родом оцим, і засудить його, бо вона з кінця світу прийшла Соломонову мудрість послухати. А тут ото Більший, аніж Соломон!
\end{tcolorbox}
\begin{tcolorbox}
\textsubscript{43} А коли дух нечистий виходить із людини, то блукає місцями безвідними, відпочинку шукаючи, та не знаходить.
\end{tcolorbox}
\begin{tcolorbox}
\textsubscript{44} Тоді він говорить: Вернуся до дому свого, звідки вийшов. А як вернеться він, то хату знаходить порожню, заметену й прибрану.
\end{tcolorbox}
\begin{tcolorbox}
\textsubscript{45} Тоді він іде, та й приводить сімох духів інших, лютіших за себе, і входять вони та й живуть тут. І буде останнє людині тій гірше за перше... Так буде й лукавому родові цьому!
\end{tcolorbox}
\begin{tcolorbox}
\textsubscript{46} Коли Він іще промовляв до народу, аж ось мати й брати Його осторонь стали, бажаючи з Ним говорити.
\end{tcolorbox}
\begin{tcolorbox}
\textsubscript{47} І сказав хтось Йому: Ото мати Твоя й Твої браття стоять онде осторонь, і говорити з Тобою бажають.
\end{tcolorbox}
\begin{tcolorbox}
\textsubscript{48} А Він відповів тому, хто Йому говорив, і сказав: Хто мати Моя? І хто браття Мої?
\end{tcolorbox}
\begin{tcolorbox}
\textsubscript{49} І, показавши рукою Своєю на учнів Своїх, Він промовив: Ото Моя мати та браття Мої!
\end{tcolorbox}
\begin{tcolorbox}
\textsubscript{50} Бо хто волю Мого Отця, що на небі, чинитиме, той Мені брат, і сестра, і мати!
\end{tcolorbox}
\subsection{CHAPTER 13}
\begin{tcolorbox}
\textsubscript{1} Того ж дня Ісус вийшов із дому, та й сів біля моря.
\end{tcolorbox}
\begin{tcolorbox}
\textsubscript{2} І безліч народу зібралась до Нього, так що Він увійшов був до човна та й сів, а ввесь натовп стояв понад берегом.
\end{tcolorbox}
\begin{tcolorbox}
\textsubscript{3} І багато навчав Він їх притчами, кажучи: Ось вийшов сіяч, щоб посіяти.
\end{tcolorbox}
\begin{tcolorbox}
\textsubscript{4} І як сіяв він зерна, упали одні край дороги, і пташки налетіли, та їх повидзьобували.
\end{tcolorbox}
\begin{tcolorbox}
\textsubscript{5} Другі ж упали на ґрунт кам'янистий, де не мали багато землі, і негайно посходили, бо земля неглибока була;
\end{tcolorbox}
\begin{tcolorbox}
\textsubscript{6} а як сонце зійшло, то зів'яли, і коріння не мавши, посохли.
\end{tcolorbox}
\begin{tcolorbox}
\textsubscript{7} А інші попадали в терен, і вигнався терен, і їх поглушив.
\end{tcolorbox}
\begin{tcolorbox}
\textsubscript{8} Інші ж упали на добрую землю і зродили: одне в сто раз, друге в шістдесят, а те втридцятеро.
\end{tcolorbox}
\begin{tcolorbox}
\textsubscript{9} Хто має вуха, щоб слухати, нехай слухає!
\end{tcolorbox}
\begin{tcolorbox}
\textsubscript{10} І учні Його приступили й сказали до Нього: Чому притчами Ти промовляєш до них?
\end{tcolorbox}
\begin{tcolorbox}
\textsubscript{11} А Він відповів і промовив: Тому, що вам дано пізнати таємниці Царства Небесного, їм же не дано.
\end{tcolorbox}
\begin{tcolorbox}
\textsubscript{12} Бо хто має, то дасться йому та додасться, хто ж не має, забереться від нього й те, що він має.
\end{tcolorbox}
\begin{tcolorbox}
\textsubscript{13} Я тому говорю до них притчами, що вони, дивлячися, не бачать, і слухаючи, не чують, і не розуміють.
\end{tcolorbox}
\begin{tcolorbox}
\textsubscript{14} І над ними збувається пророцтво Ісаї, яке промовляє: Почуєте слухом, і не зрозумієте, дивитися будете оком, і не побачите...
\end{tcolorbox}
\begin{tcolorbox}
\textsubscript{15} Затовстіло бо серце людей цих, тяжко чують вухами вони, і зажмурили очі свої, щоб коли не побачити очима й не почути вухами, і не зрозуміти їм серцем, і не навернутись, щоб Я їх уздоровив!
\end{tcolorbox}
\begin{tcolorbox}
\textsubscript{16} Очі ж ваші блаженні, що бачать, і вуха ваші, що чують.
\end{tcolorbox}
\begin{tcolorbox}
\textsubscript{17} Бо поправді кажу вам, що багато пророків і праведних бажали побачити, що бачите ви, та не бачили, і почути, що чуєте ви, і не чули.
\end{tcolorbox}
\begin{tcolorbox}
\textsubscript{18} Послухайте ж притчу про сіяча.
\end{tcolorbox}
\begin{tcolorbox}
\textsubscript{19} До кожного, хто слухає слово про Царство, але не розуміє, приходить лукавий, і краде посіяне в серці його; це те, що посіяне понад дорогою.
\end{tcolorbox}
\begin{tcolorbox}
\textsubscript{20} А посіяне на кам'янистому ґрунті, це той, хто слухає слово, і з радістю зараз приймає його;
\end{tcolorbox}
\begin{tcolorbox}
\textsubscript{21} але кореня в ньому нема, тому він непостійний; коли ж утиск або переслідування настають за слово, то він зараз спокушується.
\end{tcolorbox}
\begin{tcolorbox}
\textsubscript{22} А між терен посіяне, це той, хто слухає слово, але клопоти віку цього та омана багатства заглушують слово, і воно зостається без плоду.
\end{tcolorbox}
\begin{tcolorbox}
\textsubscript{23} А посіяне в добрій землі, це той, хто слухає слово й його розуміє, і плід він приносить, і дає один у сто раз, другий у шістдесят, а той утридцятеро.
\end{tcolorbox}
\begin{tcolorbox}
\textsubscript{24} Іншу притчу подав Він їм, кажучи: Царство Небесне подібне до чоловіка, що посіяв був добре насіння на полі своїм.
\end{tcolorbox}
\begin{tcolorbox}
\textsubscript{25} А коли люди спали, прийшов ворог його, і куколю між пшеницю насіяв, та й пішов.
\end{tcolorbox}
\begin{tcolorbox}
\textsubscript{26} А як виросло збіжжя та кинуло колос, тоді показався і кукіль.
\end{tcolorbox}
\begin{tcolorbox}
\textsubscript{27} І прийшли господареві раби, та й кажуть йому: Пане, чи ж не добре насіння ти сіяв на полі своїм? Звідки ж узявся кукіль?
\end{tcolorbox}
\begin{tcolorbox}
\textsubscript{28} А він їм відказав: Чоловік супротивник накоїв оце. А раби відказали йому: Отож, чи не хочеш, щоб пішли ми і його повиполювали?
\end{tcolorbox}
\begin{tcolorbox}
\textsubscript{29} Але він відказав: Ні, щоб, виполюючи той кукіль, ви не вирвали разом із ним і пшеницю.
\end{tcolorbox}
\begin{tcolorbox}
\textsubscript{30} Залишіть, хай разом обоє ростуть аж до жнив; а в жнива накажу я женцям: Зберіть перше кукіль і його пов'яжіть у снопки, щоб їх попалити; пшеницю ж спровадьте до клуні моєї.
\end{tcolorbox}
\begin{tcolorbox}
\textsubscript{31} Іншу притчу подав Він їм, кажучи: Царство Небесне подібне до зерна гірчичного, що взяв чоловік і посіяв на полі своїм.
\end{tcolorbox}
\begin{tcolorbox}
\textsubscript{32} Воно найдрібніше з увсього насіння, але, коли виросте, більше воно за зілля, і стає деревом, так що птаство небесне злітається, і кублиться в віттях його.
\end{tcolorbox}
\begin{tcolorbox}
\textsubscript{33} Іншу притчу Він їм розповів: Царство Небесне подібне до розчини, що її бере жінка, і кладе на три мірі муки, аж поки все вкисне.
\end{tcolorbox}
\begin{tcolorbox}
\textsubscript{34} Це все в притчах Ісус говорив до людей, і без притчі нічого Він їм не казав,
\end{tcolorbox}
\begin{tcolorbox}
\textsubscript{35} щоб справдилось те, що сказав був пророк, промовляючи: Відкрию у притчах уста Свої, розповім таємниці від почину світу!
\end{tcolorbox}
\begin{tcolorbox}
\textsubscript{36} Тоді відпустив Він народ і додому прийшов. І підійшли Його учні до Нього й сказали: Поясни нам притчу про кукіль польовий.
\end{tcolorbox}
\begin{tcolorbox}
\textsubscript{37} А Він відповів і промовив до них: Хто добре насіння посіяв був, це Син Людський,
\end{tcolorbox}
\begin{tcolorbox}
\textsubscript{38} а поле це світ, добре ж насіння це сини Царства, а кукіль сини лукавого;
\end{tcolorbox}
\begin{tcolorbox}
\textsubscript{39} а ворог, що всіяв його це диявол, жнива кінець віку, а женці Анголи.
\end{tcolorbox}
\begin{tcolorbox}
\textsubscript{40} І як збирають кукіль, і як палять в огні, так буде й наприкінці віку цього.
\end{tcolorbox}
\begin{tcolorbox}
\textsubscript{41} Пошле Людський Син Своїх Анголів, і вони позбирають із Царства Його всі спокуси, і тих, хто чинить беззаконня,
\end{tcolorbox}
\begin{tcolorbox}
\textsubscript{42} і їх повкидають до печі огненної, буде там плач і скрегіт зубів!
\end{tcolorbox}
\begin{tcolorbox}
\textsubscript{43} Тоді праведники, немов сонце, засяють у Царстві свого Отця. Хто має вуха, нехай слухає!
\end{tcolorbox}
\begin{tcolorbox}
\textsubscript{44} Царство Небесне подібне ще до захованого в полі скарбу, що людина, знайшовши, ховає його, і з радости з того йде, та й усе, що має, продає та купує те поле.
\end{tcolorbox}
\begin{tcolorbox}
\textsubscript{45} Подібне ще Царство Небесне до того купця, що пошукує перел добрих,
\end{tcolorbox}
\begin{tcolorbox}
\textsubscript{46} а як знайде одну дорогоцінну перлину, то йде, і все продає, що має, і купує її.
\end{tcolorbox}
\begin{tcolorbox}
\textsubscript{47} Подібне ще Царство Небесне до невода, у море закиненого, що зібрав він усячину.
\end{tcolorbox}
\begin{tcolorbox}
\textsubscript{48} Коли він наповниться, тягнуть на берег його, і, сівши, вибирають до посуду добре, непотріб же геть викидають.
\end{tcolorbox}
\begin{tcolorbox}
\textsubscript{49} Так буде й наприкінці віку: Анголи повиходять, і вилучать злих з-поміж праведних,
\end{tcolorbox}
\begin{tcolorbox}
\textsubscript{50} і їх повкидають до печі огненної, буде там плач і скрегіт зубів!
\end{tcolorbox}
\begin{tcolorbox}
\textsubscript{51} Чи ви зрозуміли це все? Так! відказали Йому.
\end{tcolorbox}
\begin{tcolorbox}
\textsubscript{52} І Він їм сказав: Тому кожен книжник, що навчений про Царство Небесне, подібний до того господаря, що з скарбниці своєї виносить нове та старе.
\end{tcolorbox}
\begin{tcolorbox}
\textsubscript{53} І сталось, як скінчив Ісус притчі оці, Він звідти пішов.
\end{tcolorbox}
\begin{tcolorbox}
\textsubscript{54} І прийшов Він до Своєї батьківщини, і навчав їх у їхній синагозі, так що стали вони дивуватися й питати: Звідки в Нього ця мудрість та сили чудодійні?
\end{tcolorbox}
\begin{tcolorbox}
\textsubscript{55} Чи ж Він не син теслі? Чи ж мати Його не Марією зветься, а брати Його Яків, і Йосип, і Симон та Юда?
\end{tcolorbox}
\begin{tcolorbox}
\textsubscript{56} І чи ж сестри Його не всі з нами? Звідки ж Йому все оте?
\end{tcolorbox}
\begin{tcolorbox}
\textsubscript{57} І вони спокушалися Ним. А Ісус їм сказав: Пророка нема без пошани, хіба тільки в вітчизні своїй та в домі своїм!
\end{tcolorbox}
\begin{tcolorbox}
\textsubscript{58} І Він не вчинив тут чуд багатьох через їхню невіру.
\end{tcolorbox}
\subsection{CHAPTER 14}
\begin{tcolorbox}
\textsubscript{1} Того часу прочув Ірод чотиривласник чутки про Ісуса,
\end{tcolorbox}
\begin{tcolorbox}
\textsubscript{2} і сказав своїм слугам: Це Іван Христитель, він із мертвих воскрес, і тому чуда творяться ним...
\end{tcolorbox}
\begin{tcolorbox}
\textsubscript{3} Бо Ірод схопив був Івана, і зв'язав його, і посадив у в'язницю через Іродіяду, дружину брата свого Пилипа.
\end{tcolorbox}
\begin{tcolorbox}
\textsubscript{4} Бо до нього Іван говорив: Не годиться тобі її мати!
\end{tcolorbox}
\begin{tcolorbox}
\textsubscript{5} І хотів Ірод смерть заподіяти йому, та боявся народу, бо того за пророка вважали.
\end{tcolorbox}
\begin{tcolorbox}
\textsubscript{6} А як був день народження Ірода, танцювала посеред гостей дочка Іродіядина, та й Іродові догодила.
\end{tcolorbox}
\begin{tcolorbox}
\textsubscript{7} Тому під присягою він обіцявся їй дати, чого тільки попросить вона.
\end{tcolorbox}
\begin{tcolorbox}
\textsubscript{8} А вона, за намовою матері своєї: Дай мені проказала отут на полумиску голову Івана Христителя!...
\end{tcolorbox}
\begin{tcolorbox}
\textsubscript{9} І цар засмутився, але через клятву та тих, хто сидів при столі з ним, звелів дати.
\end{tcolorbox}
\begin{tcolorbox}
\textsubscript{10} І послав стяти Івана в в'язниці.
\end{tcolorbox}
\begin{tcolorbox}
\textsubscript{11} І принесли на полумискові його голову, та й дали дівчині, а та віднесла її своїй матері...
\end{tcolorbox}
\begin{tcolorbox}
\textsubscript{12} А учні його прибули, взяли тіло, і поховали його, та прийшли й сповістили Ісуса.
\end{tcolorbox}
\begin{tcolorbox}
\textsubscript{13} Як Ісус те почув, Він відплив звідти човном у місце пустинне й самотнє. І, прочувши, народ із міст пішов пішки за Ним.
\end{tcolorbox}
\begin{tcolorbox}
\textsubscript{14} І, як вийшов Ісус, Він побачив багато народу, і змилосердивсь над ними, і їхніх слабих уздоровив.
\end{tcolorbox}
\begin{tcolorbox}
\textsubscript{15} А коли настав вечір, підійшли Його учні до Нього й сказали: Тут місце пустинне, і година вже пізня; відпусти народ, хай по селах розійдуться, і куплять поживи собі.
\end{tcolorbox}
\begin{tcolorbox}
\textsubscript{16} А Ісус їм сказав: Непотрібно відходити їм, нагодуйте їх ви!
\end{tcolorbox}
\begin{tcolorbox}
\textsubscript{17} Вони ж кажуть Йому: Не маємо чим тут, тільки п'ятеро хліба й дві рибі.
\end{tcolorbox}
\begin{tcolorbox}
\textsubscript{18} А Він відказав: Принесіть Мені їх сюди.
\end{tcolorbox}
\begin{tcolorbox}
\textsubscript{19} І, звелівши натовпові посідати на траві, Він узяв п'ятеро хліба й дві рибі, споглянув на небо, поблагословив й поламав ті хліби, і дав учням, а учні народові.
\end{tcolorbox}
\begin{tcolorbox}
\textsubscript{20} І всі їли й наситились, а з кусків позосталих назбирали дванадцятеро повних кошів...
\end{tcolorbox}
\begin{tcolorbox}
\textsubscript{21} Їдців же було мужа тисяч із п'ять, крім жінок і дітей.
\end{tcolorbox}
\begin{tcolorbox}
\textsubscript{22} І зараз звелів Ісус учням до човна сідати, і переплисти на той бік раніше Його, аж поки народ Він відпустить.
\end{tcolorbox}
\begin{tcolorbox}
\textsubscript{23} Відпустивши ж народ, Він на гору пішов помолитися насамоті; і як вечір настав, був там Сам.
\end{tcolorbox}
\begin{tcolorbox}
\textsubscript{24} А човен вже був на середині моря, і кидали хвилі його, бо вітер зірвавсь супротивний.
\end{tcolorbox}
\begin{tcolorbox}
\textsubscript{25} А о четвертій сторожі нічній Ісус підійшов до них, ідучи по морю.
\end{tcolorbox}
\begin{tcolorbox}
\textsubscript{26} Як побачили ж учні, що йде Він по морю, то настрашилися та й казали: Мара! І від страху вони закричали...
\end{tcolorbox}
\begin{tcolorbox}
\textsubscript{27} А Ісус до них зараз озвався й сказав: Заспокойтесь, це Я, не лякайтесь!
\end{tcolorbox}
\begin{tcolorbox}
\textsubscript{28} Петро ж відповів і сказав: Коли, Господи, Ти це, то звели, щоб прийшов я до Тебе по воді.
\end{tcolorbox}
\begin{tcolorbox}
\textsubscript{29} А Він відказав йому: Іди. І, вилізши з човна, Петро став іти по воді, і пішов до Ісуса.
\end{tcolorbox}
\begin{tcolorbox}
\textsubscript{30} Але, бачачи велику бурю, злякався, і зачав потопати, і скричав: Рятуй мене, Господи!...
\end{tcolorbox}
\begin{tcolorbox}
\textsubscript{31} І зараз Ісус простяг руку й схопив його, і каже до нього: Маловірний, чого усумнився?
\end{tcolorbox}
\begin{tcolorbox}
\textsubscript{32} Як до човна ж вони ввійшли, буря вщухнула.
\end{tcolorbox}
\begin{tcolorbox}
\textsubscript{33} А приявні в човні вклонились Йому та сказали: Ти справді Син Божий!
\end{tcolorbox}
\begin{tcolorbox}
\textsubscript{34} Перепливши ж вони, прибули в землю Генісаретську.
\end{tcolorbox}
\begin{tcolorbox}
\textsubscript{35} А люди тієї місцевости, пізнавши Його, сповістили по всій тій околиці, і до Нього принесли всіх хворих.
\end{tcolorbox}
\begin{tcolorbox}
\textsubscript{36} І благали Його, щоб бодай доторкнутися краю одежі Його. А хто доторкавсь, уздоровлений був.
\end{tcolorbox}
\subsection{CHAPTER 15}
\begin{tcolorbox}
\textsubscript{1} Тоді до Ісуса прийшли фарисеї та книжники з Єрусалиму й сказали:
\end{tcolorbox}
\begin{tcolorbox}
\textsubscript{2} Чого Твої учні ламають передання старших? Бо не миють вони своїх рук, коли хліб споживають.
\end{tcolorbox}
\begin{tcolorbox}
\textsubscript{3} А Він відповів і промовив до них: А чого й ви порушуєте Божу заповідь ради передання вашого?
\end{tcolorbox}
\begin{tcolorbox}
\textsubscript{4} Бо Бог заповів: Шануй батька та матір, та: Хто злорічить на батька чи матір, хай смертю помре.
\end{tcolorbox}
\begin{tcolorbox}
\textsubscript{5} А ви кажете: Коли скаже хто батьку чи матері: Те, чим би ви скористатись від мене хотіли, то дар Богові,
\end{tcolorbox}
\begin{tcolorbox}
\textsubscript{6} то може вже й не шанувати той батька свого або матір свою. Так ви ради передання вашого знівечили Боже Слово.
\end{tcolorbox}
\begin{tcolorbox}
\textsubscript{7} Лицеміри! Про вас добре Ісая пророкував був, говорячи:
\end{tcolorbox}
\begin{tcolorbox}
\textsubscript{8} Оці люди устами шанують Мене, серце ж їхнє далеко від Мене!
\end{tcolorbox}
\begin{tcolorbox}
\textsubscript{9} Та однак надаремне шанують Мене, бо навчають наук людських заповідей...
\end{tcolorbox}
\begin{tcolorbox}
\textsubscript{10} І Він покликав народ, і промовив до нього: Послухайте та зрозумійте!
\end{tcolorbox}
\begin{tcolorbox}
\textsubscript{11} Не те, що входить до уст, людину сквернить, але те, що виходить із уст, те людину сквернить.
\end{tcolorbox}
\begin{tcolorbox}
\textsubscript{12} Тоді учні Його приступили й сказали Йому: Чи Ти знаєш, що фарисеї, почувши це слово, спокусилися?
\end{tcolorbox}
\begin{tcolorbox}
\textsubscript{13} А Він відповів і сказав: Усяка рослина, яку насадив не Отець Мій Небесний, буде вирвана з коренем.
\end{tcolorbox}
\begin{tcolorbox}
\textsubscript{14} Залишіть ви їх: це сліпі поводатарі для сліпих. А коли сліпий водить сліпого, обоє до ями впадуть...
\end{tcolorbox}
\begin{tcolorbox}
\textsubscript{15} А Петро відповів і до Нього сказав: Поясни нам цю притчу.
\end{tcolorbox}
\begin{tcolorbox}
\textsubscript{16} А Він відказав: Чи ж і ви розуміння не маєте?
\end{tcolorbox}
\begin{tcolorbox}
\textsubscript{17} Чи ж ви не розумієте, що все те, що входить до уст, вступає в живіт, та й назовні виходить?
\end{tcolorbox}
\begin{tcolorbox}
\textsubscript{18} Що ж виходить із уст, те походить із серця, і воно опоганює людину.
\end{tcolorbox}
\begin{tcolorbox}
\textsubscript{19} Бо з серця виходять лихі думки, душогубства, перелюби, розпуста, крадіж, неправдиві засвідчення, богозневаги.
\end{tcolorbox}
\begin{tcolorbox}
\textsubscript{20} Оце те, що людину опоганює. А їсти руками невмитими, не опоганює це людини!
\end{tcolorbox}
\begin{tcolorbox}
\textsubscript{21} І, вийшовши звідти, Ісус відійшов у землі тирські й сидонські.
\end{tcolorbox}
\begin{tcolorbox}
\textsubscript{22} І ось жінка одна хананеянка, із тих околиць прийшовши, заголосила до Нього й сказала: Змилуйся надо мною, Господи, Сину Давидів, демон тяжко дочку мою мучить!
\end{tcolorbox}
\begin{tcolorbox}
\textsubscript{23} А Він їй не казав ані слова. Тоді учні Його, підійшовши, благали Його та казали: Відпусти її, бо кричить услід за нами!
\end{tcolorbox}
\begin{tcolorbox}
\textsubscript{24} А Він відповів і сказав: Я посланий тільки до овечок загинулих дому Ізраїлевого...
\end{tcolorbox}
\begin{tcolorbox}
\textsubscript{25} А вона, підійшовши, уклонилась Йому та й сказала: Господи, допоможи мені!
\end{tcolorbox}
\begin{tcolorbox}
\textsubscript{26} А Він відповів і сказав: Не годиться взяти хліб у дітей, і кинути щенятам...
\end{tcolorbox}
\begin{tcolorbox}
\textsubscript{27} Вона ж відказала: Так, Господи! Але ж і щенята їдять ті кришки, що спадають зо столу їхніх панів.
\end{tcolorbox}
\begin{tcolorbox}
\textsubscript{28} Тоді відповів і сказав їй Ісус: О жінко, твоя віра велика, нехай буде тобі, як ти хочеш! І тієї години дочка її видужала.
\end{tcolorbox}
\begin{tcolorbox}
\textsubscript{29} І, відійшовши звідти, Ісус прибув до Галілейського моря, і, зійшовши на гору, сів там.
\end{tcolorbox}
\begin{tcolorbox}
\textsubscript{30} І приступило до Нього багато народу, що мали з собою кривих, калік, сліпих, німих і інших багато, і клали їх до Ісусових ніг. І Він уздоровлював їх.
\end{tcolorbox}
\begin{tcolorbox}
\textsubscript{31} А народ не виходив із дива, бо бачив, що говорять німі, каліки стають здорові, криві ходять, і бачать сліпі, і славив він Бога Ізраїлевого!
\end{tcolorbox}
\begin{tcolorbox}
\textsubscript{32} А Ісус Своїх учнів покликав і сказав: Жаль Мені цих людей, що вже три дні зо Мною знаходяться, але їсти не мають чого; відпустити їх без їжі не хочу, щоб вони не ослабли в дорозі.
\end{tcolorbox}
\begin{tcolorbox}
\textsubscript{33} А учні Йому відказали: Де нам узяти стільки хліба в пустині, щоб нагодувати стільки народу?
\end{tcolorbox}
\begin{tcolorbox}
\textsubscript{34} А Ісус запитав їх: Скільки маєте хліба? Вони ж відказали: Семеро, та трохи рибок.
\end{tcolorbox}
\begin{tcolorbox}
\textsubscript{35} І Він ізвелів на землі посідати народові.
\end{tcolorbox}
\begin{tcolorbox}
\textsubscript{36} І, взявши сім хлібів і риби, віддавши Богу подяку, поламав і дав учням Своїм, а учні народові.
\end{tcolorbox}
\begin{tcolorbox}
\textsubscript{37} І всі їли й наситилися, а з позосталих кусків назбирали сім кошиків повних...
\end{tcolorbox}
\begin{tcolorbox}
\textsubscript{38} Їдців же було чотири тисячі мужа, окрім жінок та дітей.
\end{tcolorbox}
\begin{tcolorbox}
\textsubscript{39} І, відпустивши народ, усів Він до човна, і прибув до землі Магдалинської.
\end{tcolorbox}
\subsection{CHAPTER 16}
\begin{tcolorbox}
\textsubscript{1} І підійшли фарисеї та саддукеї, і, випробовуючи, просили Його показати ознаку їм із неба.
\end{tcolorbox}
\begin{tcolorbox}
\textsubscript{2} А Він відповів і промовив до них: Ви звечора кажете: Буде погода, червоніє бо небо.
\end{tcolorbox}
\begin{tcolorbox}
\textsubscript{3} А ранком: Сьогодні негода, червоніє бо небо похмуре. Розпізнати небесне обличчя ви вмієте, ознак часу ж не можете!
\end{tcolorbox}
\begin{tcolorbox}
\textsubscript{4} Рід лукавий і перелюбний шукає ознаки, та ознаки йому не дадуть, окрім ознаки пророка Йони. І, їх полишивши, Він відійшов.
\end{tcolorbox}
\begin{tcolorbox}
\textsubscript{5} А учні Його, перейшовши на той бік, забули взяти хліба.
\end{tcolorbox}
\begin{tcolorbox}
\textsubscript{6} Ісус же промовив до них: Стережіться уважливо фарисейської та саддукейської розчини!
\end{tcolorbox}
\begin{tcolorbox}
\textsubscript{7} Вони ж міркували собі й говорили: Ми ж хлібів не взяли.
\end{tcolorbox}
\begin{tcolorbox}
\textsubscript{8} А Ісус, знавши те, запитав: Чого між собою міркуєте ви, маловірні, що хлібів не взяли?
\end{tcolorbox}
\begin{tcolorbox}
\textsubscript{9} Чи ж ви ще не розумієте й не пам'ятаєте про п'ять хлібів на п'ять тисяч, і скільки кошів ви зібрали?
\end{tcolorbox}
\begin{tcolorbox}
\textsubscript{10} Ані про сім хлібів на чотири тисячі, і скільки кошиків ви назбирали?
\end{tcolorbox}
\begin{tcolorbox}
\textsubscript{11} Як ви не розумієте, що Я не про хліб вам сказав? Стережіться но розчини фарисейської та саддукейської!
\end{tcolorbox}
\begin{tcolorbox}
\textsubscript{12} Тоді зрозуміли вони, що Він не казав стерегтися їм розчини хлібної, але фарисейської та саддукейської науки.
\end{tcolorbox}
\begin{tcolorbox}
\textsubscript{13} Прийшовши ж Ісус до землі Кесарії Пилипової, питав Своїх учнів і казав: За кого народ уважає Мене, Сина Людського?
\end{tcolorbox}
\begin{tcolorbox}
\textsubscript{14} Вони ж відповіли: Одні за Івана Христителя, одні за Іллю, інші ж за Єремію або за одного з пророків.
\end{tcolorbox}
\begin{tcolorbox}
\textsubscript{15} Він каже до них: А ви за кого Мене маєте?
\end{tcolorbox}
\begin{tcolorbox}
\textsubscript{16} А Симон Петро відповів і сказав: Ти Христос, Син Бога Живого!
\end{tcolorbox}
\begin{tcolorbox}
\textsubscript{17} А Ісус відповів і до нього промовив: Блаженний ти, Симоне, сину Йонин, бо не тіло і кров тобі оце виявили, але Мій Небесний Отець.
\end{tcolorbox}
\begin{tcolorbox}
\textsubscript{18} І кажу Я тобі, що ти скеля, і на скелі оцій побудую Я Церкву Свою, і сили адові не переможуть її.
\end{tcolorbox}
\begin{tcolorbox}
\textsubscript{19} І ключі тобі дам від Царства Небесного, і що на землі ти зв'яжеш, те зв'язане буде на небі, а що на землі ти розв'яжеш, те розв'язане буде на небі!
\end{tcolorbox}
\begin{tcolorbox}
\textsubscript{20} Тоді наказав Своїм учням, щоб нікому не казали, що Він Христос.
\end{tcolorbox}
\begin{tcolorbox}
\textsubscript{21} Із того часу Ісус став виказувати Своїм учням, що Він мусить іти до Єрусалиму, і постраждати багато від старших, і первосвящеників, і книжників, і вбитому бути, і воскреснути третього дня.
\end{tcolorbox}
\begin{tcolorbox}
\textsubscript{22} І, набік відвівши Його, Петро став Йому докоряти й казати: Змилуйся, Господи, такого Тобі хай не буде!
\end{tcolorbox}
\begin{tcolorbox}
\textsubscript{23} А Він обернувся й промовив Петрові: Відступися від Мене, сатано, ти спокуса Мені, бо думаєш не про Боже, а про людське!
\end{tcolorbox}
\begin{tcolorbox}
\textsubscript{24} Промовив тоді Ісус учням Своїм: Коли хоче хто йти вслід за Мною, хай зречеться самого себе, і хай візьме свого хреста, та й іде вслід за Мною.
\end{tcolorbox}
\begin{tcolorbox}
\textsubscript{25} Бо хто хоче спасти свою душу, той погубить її, хто ж за Мене свою душу погубить, той знайде її.
\end{tcolorbox}
\begin{tcolorbox}
\textsubscript{26} Яка ж користь людині, що здобуде ввесь світ, але душу свою занапастить? Або що дасть людина взамін за душу свою?
\end{tcolorbox}
\begin{tcolorbox}
\textsubscript{27} Бо прийде Син Людський у славі Свого Отця з Анголами Своїми, і тоді віддасть кожному згідно з ділами його.
\end{tcolorbox}
\begin{tcolorbox}
\textsubscript{28} Поправді кажу вам, що деякі з тут-о приявних не скуштують смерти, аж поки не побачать Сина Людського, що йде в Царстві Своїм.
\end{tcolorbox}
\subsection{CHAPTER 17}
\begin{tcolorbox}
\textsubscript{1} А через шість день забирає Ісус Петра, і Якова, і Івана, брата його, та й веде їх осібно на гору високу.
\end{tcolorbox}
\begin{tcolorbox}
\textsubscript{2} І Він перед ними переобразився: обличчя Його, як те сонце, засяло, а одежа Його стала біла, як світло.
\end{tcolorbox}
\begin{tcolorbox}
\textsubscript{3} І ось з'явились до них Мойсей та Ілля, і розмовляли із Ним.
\end{tcolorbox}
\begin{tcolorbox}
\textsubscript{4} І озвався Петро та й сказав до Ісуса: Господи, добре бути нам тут! Коли хочеш, поставлю отут три шатрі: для Тебе одне, і одне для Мойсея, і одне для Іллі.
\end{tcolorbox}
\begin{tcolorbox}
\textsubscript{5} Як він ще говорив, ось хмара ясна заслонила їх, і ось голос із хмари почувсь, що казав: Це Син Мій Улюблений, що Його Я вподобав. Його слухайтеся!
\end{tcolorbox}
\begin{tcolorbox}
\textsubscript{6} А почувши, попадали учні долілиць, і полякалися сильно...
\end{tcolorbox}
\begin{tcolorbox}
\textsubscript{7} А Ісус підійшов, доторкнувся до них і промовив: Уставайте й не бійтесь!
\end{tcolorbox}
\begin{tcolorbox}
\textsubscript{8} Звівши ж очі свої, нікого вони не побачили, окрім Самого Ісуса.
\end{tcolorbox}
\begin{tcolorbox}
\textsubscript{9} А коли з гори сходили, заповів їм Ісус і сказав: Не кажіть нікому про цеє видіння, аж поки Син Людський із мертвих воскресне.
\end{tcolorbox}
\begin{tcolorbox}
\textsubscript{10} І запитали Його учні, говорячи: Що це книжники кажуть, ніби треба Іллі перш прийти?
\end{tcolorbox}
\begin{tcolorbox}
\textsubscript{11} А Він відповів і сказав: Ілля, правда, прийде, і все приготує.
\end{tcolorbox}
\begin{tcolorbox}
\textsubscript{12} Але кажу вам, що Ілля вже прийшов був, та його не пізнали, але з ним зробили, що тільки хотіли... Так і Син Людський має страждати від них.
\end{tcolorbox}
\begin{tcolorbox}
\textsubscript{13} Учні тоді зрозуміли, що Він їм говорив про Івана Христителя.
\end{tcolorbox}
\begin{tcolorbox}
\textsubscript{14} І як вони до народу прийшли, то до Нього один чоловік приступив, і навколішки впав перед Ним,
\end{tcolorbox}
\begin{tcolorbox}
\textsubscript{15} і сказав: Господи, змилуйсь над сином моїм, що біснується у новомісяччі, і мучиться тяжко, бо почасту падає він ув огонь, і почасту в воду.
\end{tcolorbox}
\begin{tcolorbox}
\textsubscript{16} Я його був привів до учнів Твоїх, та вони не могли вздоровити його.
\end{tcolorbox}
\begin{tcolorbox}
\textsubscript{17} А Ісус відповів і сказав: О роде невірний й розбещений, доки буду Я з вами? Доки вас Я терпітиму? Приведіть до Мене сюди його!
\end{tcolorbox}
\begin{tcolorbox}
\textsubscript{18} Потому Ісус погрозив йому, і демон вийшов із нього. І видужав хлопець тієї години!
\end{tcolorbox}
\begin{tcolorbox}
\textsubscript{19} Тоді підійшли учні насамоті до Ісуса й сказали: Чому ми не могли його вигнати?
\end{tcolorbox}
\begin{tcolorbox}
\textsubscript{20} А Він їм відповів: Через ваше невірство. Бо поправді кажу вам: коли будете ви мати віру, хоч як зерно гірчичне, і горі оцій скажете: Перейди звідси туди, то й перейде вона, і нічого не матимете неможливого!
\end{tcolorbox}
\begin{tcolorbox}
\textsubscript{21} Цей же рід не виходить інакше, як тільки молитвою й постом.
\end{tcolorbox}
\begin{tcolorbox}
\textsubscript{22} Коли пробували вони в Галілеї, то сказав їм Ісус: Людський Син буде виданий людям до рук,
\end{tcolorbox}
\begin{tcolorbox}
\textsubscript{23} і вони Його вб'ють, але третього дня Він воскресне. І тяжко вони зажурились...
\end{tcolorbox}
\begin{tcolorbox}
\textsubscript{24} Як прийшли ж вони в Капернаум, до Петра підійшли збирачі дидрахм на храм, та й сказали: Чи не заплатить ваш учитель дидрахми?
\end{tcolorbox}
\begin{tcolorbox}
\textsubscript{25} Він відказує: Так. І як він увійшов до дому, то Ісус попередив його та сказав: Як ти думаєш, Симоне: царі земні з кого беруть мито або податки: від синів своїх, чи чужих?
\end{tcolorbox}
\begin{tcolorbox}
\textsubscript{26} А як той відказав: Від чужих, то промовив до нього Ісус: Тож вільні сини!
\end{tcolorbox}
\begin{tcolorbox}
\textsubscript{27} Та щоб їх не спокусити, піди над море, та вудку закинь, і яку першу рибу ізловиш, візьми, і рота відкрий їй, і знайдеш статира; візьми ти його, і віддай їм за Мене й за себе...
\end{tcolorbox}
\subsection{CHAPTER 18}
\begin{tcolorbox}
\textsubscript{1} Підійшли до Ісуса тоді Його учні, питаючи: Хто найбільший у Царстві Небеснім?
\end{tcolorbox}
\begin{tcolorbox}
\textsubscript{2} Він же дитину покликав, і поставив її серед них,
\end{tcolorbox}
\begin{tcolorbox}
\textsubscript{3} та й сказав: Поправді кажу вам: коли не навернетесь, і не станете, як ті діти, не ввійдете в Царство Небесне!
\end{tcolorbox}
\begin{tcolorbox}
\textsubscript{4} Отже, хто впокориться, як дитина оця, той найбільший у Царстві Небеснім.
\end{tcolorbox}
\begin{tcolorbox}
\textsubscript{5} І хто прийме таку дитину одну в Моє Ймення, той приймає Мене.
\end{tcolorbox}
\begin{tcolorbox}
\textsubscript{6} Хто ж спокусить одне з цих малих, що вірують в Мене, то краще б такому було, коли б жорно млинове на шию йому почепити, і його потопити в морській глибині...
\end{tcolorbox}
\begin{tcolorbox}
\textsubscript{7} Від спокус горе світові, бо мусять спокуси прийти; надто горе людині, що від неї приходить спокуса!
\end{tcolorbox}
\begin{tcolorbox}
\textsubscript{8} Коли тільки рука твоя, чи нога твоя спокушає тебе, відітни її й кинь від себе: краще тобі увійти в життя одноруким або одноногим, ніж з обома руками чи з обома ногами бути вкиненому в огонь вічний.
\end{tcolorbox}
\begin{tcolorbox}
\textsubscript{9} І коли твоє око тебе спокушає його вибери й кинь від себе: краще тобі однооким ввійти в життя, ніж з обома очима бути вкиненому до геєнни огненної.
\end{tcolorbox}
\begin{tcolorbox}
\textsubscript{10} Стережіться, щоб ви не погордували ані одним із малих цих; кажу бо Я вам, що їхні Анголи повсякчасно бачать у небі обличчя Мого Отця, що на небі.
\end{tcolorbox}
\begin{tcolorbox}
\textsubscript{11} Син бо Людський прийшов, щоб спасти загинуле.
\end{tcolorbox}
\begin{tcolorbox}
\textsubscript{12} Як вам здається: коли має який чоловік сто овець, а одна з них заблудить, то чи він не покине дев'ятдесятьох і дев'ятьох у горах, і не піде шукати заблудлої?
\end{tcolorbox}
\begin{tcolorbox}
\textsubscript{13} І коли пощастить відшукати її, поправді кажу вам, що радіє за неї він більше, аніж за дев'ятдесятьох і дев'ятьох незаблудлих.
\end{tcolorbox}
\begin{tcolorbox}
\textsubscript{14} Так волі нема Отця вашого, що на небі, щоб загинув один із цих малих.
\end{tcolorbox}
\begin{tcolorbox}
\textsubscript{15} А коли прогрішиться твій брат проти тебе, іди й йому викажи поміж тобою та ним самим; як тебе він послухає, ти придбав свого брата.
\end{tcolorbox}
\begin{tcolorbox}
\textsubscript{16} А коли не послухає він, то візьми з собою ще одного чи двох, щоб справа всіляка ствердилась устами двох чи трьох свідків.
\end{tcolorbox}
\begin{tcolorbox}
\textsubscript{17} А коли не послухає їх, скажи Церкві; коли ж не послухає й Церкви, хай буде тобі, як поганин і митник!
\end{tcolorbox}
\begin{tcolorbox}
\textsubscript{18} Поправді кажу вам: Що тільки зв'яжете на землі, зв'язане буде на небі, і що тільки розв'яжете на землі, розв'язане буде на небі.
\end{tcolorbox}
\begin{tcolorbox}
\textsubscript{19} Ще поправді кажу вам, що коли б двоє з вас на землі погодились про всяку річ, то коли вони будуть просити за неї, станеться їм від Мого Отця, що на небі!
\end{tcolorbox}
\begin{tcolorbox}
\textsubscript{20} Бо де двоє чи троє в Ім'я Моє зібрані, там Я серед них.
\end{tcolorbox}
\begin{tcolorbox}
\textsubscript{21} Петро приступив тоді та запитався Його: Господи, скільки разів брат мій може згрішити проти мене, а я маю прощати йому? Чи до семи раз?
\end{tcolorbox}
\begin{tcolorbox}
\textsubscript{22} Ісус промовляє до нього: Не кажу тобі до семи раз, але аж до семидесяти раз по семи!
\end{tcolorbox}
\begin{tcolorbox}
\textsubscript{23} Тим то Царство Небесне подібне одному цареві, що захотів обрахунок зробити з своїми рабами.
\end{tcolorbox}
\begin{tcolorbox}
\textsubscript{24} Коли ж він почав обраховувати, то йому привели одного, що винен був десять тисяч талантів.
\end{tcolorbox}
\begin{tcolorbox}
\textsubscript{25} А що він не мав із чого віддати, наказав пан продати його, і його дружину та діти, і все, що він мав, і заплатити.
\end{tcolorbox}
\begin{tcolorbox}
\textsubscript{26} Тоді раб той упав до ніг, і вклонявся йому та благав: Потерпи мені, я віддам тобі все!
\end{tcolorbox}
\begin{tcolorbox}
\textsubscript{27} І змилосердився пан над рабом тим, і звільнив його, і простив йому борг.
\end{tcolorbox}
\begin{tcolorbox}
\textsubscript{28} А як вийшов той раб, то спіткав він одного з своїх співтоваришів, що був винен йому сто динаріїв. І, схопивши його, він душив та казав: Віддай, що ти винен!
\end{tcolorbox}
\begin{tcolorbox}
\textsubscript{29} А товариш його впав у ноги йому, і благав його, кажучи: Потерпи мені, і я віддам тобі!
\end{tcolorbox}
\begin{tcolorbox}
\textsubscript{30} Та той не схотів, а пішов і всадив до в'язниці його, аж поки він боргу не верне.
\end{tcolorbox}
\begin{tcolorbox}
\textsubscript{31} Як побачили ж товариші його те, що сталося, то засмутилися дуже, і прийшли й розповіли своєму панові все, що було.
\end{tcolorbox}
\begin{tcolorbox}
\textsubscript{32} Тоді пан його кличе його, та й говорить до нього: Рабе лукавий, я простив був тобі ввесь той борг, бо просив ти мене.
\end{tcolorbox}
\begin{tcolorbox}
\textsubscript{33} Чи й тобі не належало змилуватись над своїм співтоваришем, як і я над тобою був змилувався?
\end{tcolorbox}
\begin{tcolorbox}
\textsubscript{34} І прогнівався пан його, і катам його видав, аж поки йому не віддасть всього боргу.
\end{tcolorbox}
\begin{tcolorbox}
\textsubscript{35} Так само й Отець Мій Небесний учинить із вами, коли кожен із вас не простить своєму братові з серця свого їхніх прогріхів.
\end{tcolorbox}
\subsection{CHAPTER 19}
\begin{tcolorbox}
\textsubscript{1} І сталось, як Ісус закінчив ці слова, то Він вирушив із Галілеї, і прибув до країни Юдейської, на той бік Йордану.
\end{tcolorbox}
\begin{tcolorbox}
\textsubscript{2} А за Ним ішла безліч народу, і Він уздоровив їх тут.
\end{tcolorbox}
\begin{tcolorbox}
\textsubscript{3} І підійшли фарисеї до Нього, і, випробовуючи, запитали Його: Чи дозволено дружину свою відпускати з причини всякої?
\end{tcolorbox}
\begin{tcolorbox}
\textsubscript{4} А Він відповів і сказав: Чи ви не читали, що Той, Хто створив споконвіку людей, створив їх чоловіком і жінкою?
\end{tcolorbox}
\begin{tcolorbox}
\textsubscript{5} І сказав: Покине тому чоловік батька й матір, і пристане до дружини своєї, і стануть обоє вони одним тілом,
\end{tcolorbox}
\begin{tcolorbox}
\textsubscript{6} тому то немає вже двох, але одне тіло. Тож, що Бог спарував, людина нехай не розлучує!
\end{tcolorbox}
\begin{tcolorbox}
\textsubscript{7} Вони кажуть Йому: А чому ж Мойсей заповів дати листа розводового, та й відпускати?
\end{tcolorbox}
\begin{tcolorbox}
\textsubscript{8} Він говорить до них: То за ваше жорстокосердя дозволив Мойсей відпускати дружин ваших, спочатку ж так не було.
\end{tcolorbox}
\begin{tcolorbox}
\textsubscript{9} А Я вам кажу: Хто дружину відпустить свою не з причини перелюбу, і одружиться з іншою, той чинить перелюб. І хто одружиться з розведеною, той чинить перелюб.
\end{tcolorbox}
\begin{tcolorbox}
\textsubscript{10} Учні говорять Йому: Коли справа така чоловіка із дружиною, то не добре одружуватись.
\end{tcolorbox}
\begin{tcolorbox}
\textsubscript{11} А Він їм відказав: Це слово вміщають не всі, але ті, кому дано.
\end{tcolorbox}
\begin{tcolorbox}
\textsubscript{12} Бо бувають скопці, що з утроби ще матерньої народилися так; є й скопці, що їх люди оскопили, і є скопці, що самі оскопили себе ради Царства Небесного. Хто може вмістити, нехай вмістить.
\end{tcolorbox}
\begin{tcolorbox}
\textsubscript{13} Тоді привели Йому діток, щоб поклав на них руки, і за них помолився, учні ж їм докоряли.
\end{tcolorbox}
\begin{tcolorbox}
\textsubscript{14} Ісус же сказав: Пустіть діток, і не бороніть їм приходити до Мене, бо Царство Небесне належить таким.
\end{tcolorbox}
\begin{tcolorbox}
\textsubscript{15} І Він руки на них поклав, та й пішов звідтіля.
\end{tcolorbox}
\begin{tcolorbox}
\textsubscript{16} І підійшов ось один, і до Нього сказав: Учителю Добрий, що маю зробити я доброго, щоб мати життя вічне?
\end{tcolorbox}
\begin{tcolorbox}
\textsubscript{17} Він же йому відказав: Чого звеш Мене Добрим? Ніхто не є Добрий, крім Бога Самого. Коли ж хочеш ввійти до життя, то виконай заповіді.
\end{tcolorbox}
\begin{tcolorbox}
\textsubscript{18} Той питає Його: Які саме? А Ісус відказав: Не вбивай, не чини перелюбу, не кради, не свідкуй неправдиво.
\end{tcolorbox}
\begin{tcolorbox}
\textsubscript{19} Шануй батька та матір, і: Люби свого ближнього, як самого себе.
\end{tcolorbox}
\begin{tcolorbox}
\textsubscript{20} Говорить до Нього юнак: Це я виконав все. Чого ще бракує мені?
\end{tcolorbox}
\begin{tcolorbox}
\textsubscript{21} Ісус каже йому: Коли хочеш бути досконалим, піди, продай добра свої та й убогим роздай, і матимеш скарб ти на небі. Потому приходь та й іди вслід за Мною.
\end{tcolorbox}
\begin{tcolorbox}
\textsubscript{22} Почувши ж юнак таке слово, відійшов, зажурившись, бо великі маєтки він мав.
\end{tcolorbox}
\begin{tcolorbox}
\textsubscript{23} Ісус же сказав Своїм учням: Поправді кажу вам, що багатому трудно ввійти в Царство Небесне.
\end{tcolorbox}
\begin{tcolorbox}
\textsubscript{24} Іще вам кажу: Верблюдові легше пройти через голчине вушко, ніж багатому в Боже Царство ввійти!
\end{tcolorbox}
\begin{tcolorbox}
\textsubscript{25} Як учні ж Його це зачули, здивувалися дуже й сказали: Хто ж тоді може спастися?
\end{tcolorbox}
\begin{tcolorbox}
\textsubscript{26} А Ісус позирнув і сказав їм: Неможливе це людям, та можливе все Богові.
\end{tcolorbox}
\begin{tcolorbox}
\textsubscript{27} Тоді відізвався Петро та до Нього сказав: От усе ми покинули, та й пішли за Тобою слідом; що ж нам буде за це?
\end{tcolorbox}
\begin{tcolorbox}
\textsubscript{28} А Ісус відказав їм: Поправді кажу вам, що коли, при відновленні світу, Син Людський засяде на престолі слави Своєї, тоді сядете й ви, що за Мною пішли, на дванадцять престолів, щоб судити дванадцять племен Ізраїлевих.
\end{tcolorbox}
\begin{tcolorbox}
\textsubscript{29} І кожен, хто за Ймення Моє кине дім, чи братів, чи сестер, або батька, чи матір, чи діти, чи землі, той багатокротно одержить і успадкує вічне життя.
\end{tcolorbox}
\begin{tcolorbox}
\textsubscript{30} І багато-хто з перших останніми стануть, а останні першими.
\end{tcolorbox}
\subsection{CHAPTER 20}
\begin{tcolorbox}
\textsubscript{1} Бо Царство Небесне подібне одному господареві, що вдосвіта вийшов згодити робітників у свій виноградник.
\end{tcolorbox}
\begin{tcolorbox}
\textsubscript{2} Згодившися ж він із робітниками по динарію за день, послав їх до свого виноградника.
\end{tcolorbox}
\begin{tcolorbox}
\textsubscript{3} А вийшовши коло години десь третьої, побачив він інших, що стояли без праці на ринку,
\end{tcolorbox}
\begin{tcolorbox}
\textsubscript{4} та й каже до них: Ідіть і ви до мого виноградника, і що буде належати, дам вам.
\end{tcolorbox}
\begin{tcolorbox}
\textsubscript{5} Вони ж відійшли. І вийшов він знов о годині десь шостій й дев'ятій, і те саме зробив.
\end{tcolorbox}
\begin{tcolorbox}
\textsubscript{6} А вийшовши коло години одинадцятої, знайшов інших, що стояли без праці, та й каже до них: Чого тут стоїте цілий день безробітні?
\end{tcolorbox}
\begin{tcolorbox}
\textsubscript{7} Вони кажуть до нього: Бо ніхто не найняв нас. Відказує їм: Ідіть і ви в виноградник.
\end{tcolorbox}
\begin{tcolorbox}
\textsubscript{8} Коли ж вечір настав, то говорить тоді до свого управителя пан виноградника: Поклич робітників, і дай їм заплату, почавши з останніх до перших.
\end{tcolorbox}
\begin{tcolorbox}
\textsubscript{9} І прийшли ті, що з години одинадцятої, і взяли по динарію.
\end{tcolorbox}
\begin{tcolorbox}
\textsubscript{10} Коли ж прийшли перші, то думали, що вони візьмуть більше. Та й вони по динару взяли.
\end{tcolorbox}
\begin{tcolorbox}
\textsubscript{11} А взявши, вони почали нарікати на господаря,
\end{tcolorbox}
\begin{tcolorbox}
\textsubscript{12} кажучи: Ці останні годину одну працювали, а ти прирівняв їх до нас, що витерпіли тягар дня та спекоту...
\end{tcolorbox}
\begin{tcolorbox}
\textsubscript{13} А він відповів і сказав до одного із них: Не кривджу я, друже, тебе, хіба не за динарія згодився зо мною?
\end{tcolorbox}
\begin{tcolorbox}
\textsubscript{14} Візьми ти своє та й іди. Але я хочу дати й цьому ось останньому, як і тобі.
\end{tcolorbox}
\begin{tcolorbox}
\textsubscript{15} Чи ж не вільно мені зо своїм, що я хочу, зробити? Хіба око твоє заздре від того, що я добрий?
\end{tcolorbox}
\begin{tcolorbox}
\textsubscript{16} Отак будуть останні першими, а перші останніми!
\end{tcolorbox}
\begin{tcolorbox}
\textsubscript{17} Побажавши ж піти до Єрусалиму, Ісус взяв осібно Дванадцятьох, і на дорозі їм сповістив:
\end{tcolorbox}
\begin{tcolorbox}
\textsubscript{18} Оце в Єрусалим ми йдемо, і первосвященикам і книжникам виданий буде Син Людський, і засудять на смерть Його...
\end{tcolorbox}
\begin{tcolorbox}
\textsubscript{19} І посганам Його вони видадуть на наругу та на катування, і на розп'яття, але третього дня Він воскресне!
\end{tcolorbox}
\begin{tcolorbox}
\textsubscript{20} Тоді приступила до Нього мати синів Зеведеєвих, і вклонилась, і просила від Нього чогось.
\end{tcolorbox}
\begin{tcolorbox}
\textsubscript{21} А Він їй сказав: Чого хочеш? Вона каже Йому: Скажи, щоб обидва сини мої ці сіли в Царстві Твоїм праворуч один, і ліворуч від Тебе один.
\end{tcolorbox}
\begin{tcolorbox}
\textsubscript{22} А Ісус відповів і сказав: Не знаєте, чого просите. Чи ж можете ви пити чашу, що Я її питиму або христитися хрищенням, що я ним хрищуся? Вони кажуть Йому: Можемо.
\end{tcolorbox}
\begin{tcolorbox}
\textsubscript{23} Він говорить до них: Ви питимете Мою чашу і будете христитися хрищенням, що Я ним хрищуся. А сидіти праворуч Мене та ліворуч не Моє це давати, а кому від Мого Отця те вготовано.
\end{tcolorbox}
\begin{tcolorbox}
\textsubscript{24} Як почули це десятеро, стали гніватися на обох тих братів.
\end{tcolorbox}
\begin{tcolorbox}
\textsubscript{25} А Ісус їх покликав і промовив: Ви знаєте, що князі народів панують над ними, а вельможі їх тиснуть.
\end{tcolorbox}
\begin{tcolorbox}
\textsubscript{26} Не так буде між вами, але хто великим із вас хоче бути, хай буде слугою він вам.
\end{tcolorbox}
\begin{tcolorbox}
\textsubscript{27} А хто з вас бути першим бажає, нехай буде він вам за раба.
\end{tcolorbox}
\begin{tcolorbox}
\textsubscript{28} Так само й Син Людський прийшов не на те, щоб служили Йому, а щоб послужити, і душу Свою дати на викуп за багатьох!
\end{tcolorbox}
\begin{tcolorbox}
\textsubscript{29} Як вони ж з Єрихону виходили, за Ним ішов натовп великий.
\end{tcolorbox}
\begin{tcolorbox}
\textsubscript{30} І ось двоє сліпих, що сиділи при дорозі, почувши, що переходить Ісус, стали кричати, благаючи: Змилуйсь над нами, Господи, Сину Давидів!
\end{tcolorbox}
\begin{tcolorbox}
\textsubscript{31} Народ же сварився на них, щоб мовчали, вони ж іще більше кричали, благаючи: Змилуйсь над нами, Господи, Сину Давидів!
\end{tcolorbox}
\begin{tcolorbox}
\textsubscript{32} Ісус же спинився, покликав їх та й сказав: Що хочете, щоб Я вам зробив?
\end{tcolorbox}
\begin{tcolorbox}
\textsubscript{33} Вони Йому кажуть: Господи, нехай нам розкриються очі!
\end{tcolorbox}
\begin{tcolorbox}
\textsubscript{34} І змилосердивсь Ісус, доторкнувся до їхніх очей, і зараз прозріли їм очі, і вони подалися за Ним.
\end{tcolorbox}
\subsection{CHAPTER 21}
\begin{tcolorbox}
\textsubscript{1} А коли вони наблизились до Єрусалиму, і прийшли до Вітфагії, до гори до Оливної, тоді Ісус вислав двох учнів,
\end{tcolorbox}
\begin{tcolorbox}
\textsubscript{2} до них, кажучи: Ідіть у село, яке перед вами, і знайдете зараз ослицю прив'язану та з нею осля; відв'яжіть, і Мені приведіть їх.
\end{tcolorbox}
\begin{tcolorbox}
\textsubscript{3} А як хто вам що скаже, відкажіть, що їх потребує Господь, і він зараз пошле їх.
\end{tcolorbox}
\begin{tcolorbox}
\textsubscript{4} А це сталось, щоб справдилось те, що сказав був пророк, промовляючи:
\end{tcolorbox}
\begin{tcolorbox}
\textsubscript{5} Скажіте Сіонській доньці: Ось до тебе йде Цар твій! Він покірливий, і всів на осла, на осля, під'яремної сина.
\end{tcolorbox}
\begin{tcolorbox}
\textsubscript{6} А учні пішли та й зробили, як звелів їм Ісус.
\end{tcolorbox}
\begin{tcolorbox}
\textsubscript{7} Вони привели до Ісуса ослицю й осля, і одежу поклали на них, і Він сів на них.
\end{tcolorbox}
\begin{tcolorbox}
\textsubscript{8} І багато народу стелили одежу свою по дорозі, інші ж різали віття з дерев і стелили дорогою.
\end{tcolorbox}
\begin{tcolorbox}
\textsubscript{9} А народ, що йшов перед Ним і позаду, викрикував, кажучи: Осанна Сину Давидовому! Благословенний, хто йде у Господнє Ім'я! Осанна на висоті!
\end{tcolorbox}
\begin{tcolorbox}
\textsubscript{10} А коли увійшов Він до Єрусалиму, то здвигнулося ціле місто, питаючи: Хто це такий?
\end{tcolorbox}
\begin{tcolorbox}
\textsubscript{11} А народ говорив: Це Пророк, Ісус із Назарету Галілейського!
\end{tcolorbox}
\begin{tcolorbox}
\textsubscript{12} Потому Ісус увійшов у храм Божий, і вигнав усіх продавців і покупців у храмі, і поперевертав грошомінам столи, та ослони продавцям голубів.
\end{tcolorbox}
\begin{tcolorbox}
\textsubscript{13} І сказав їм: Написано: Дім Мій буде домом молитви, а ви робите з нього печеру розбійників.
\end{tcolorbox}
\begin{tcolorbox}
\textsubscript{14} І приступили у храмі до Нього сліпі та криві, і Він їх уздоровив.
\end{tcolorbox}
\begin{tcolorbox}
\textsubscript{15} А первосвященики й книжники, бачивши чуда, що Він учинив, і дітей, що в храмі викрикували: Осанна Сину Давидовому, обурилися,
\end{tcolorbox}
\begin{tcolorbox}
\textsubscript{16} та й сказали Йому: Чи ти чуєш, що кажуть вони? А Ісус відказав їм: Так. Чи ж ви не читали ніколи: Із уст немовлят, і тих, що ссуть, учинив Ти хвалу?
\end{tcolorbox}
\begin{tcolorbox}
\textsubscript{17} І покинувши їх, Він вийшов за місто в Віфанію, і там ніч перебув.
\end{tcolorbox}
\begin{tcolorbox}
\textsubscript{18} А вранці, до міста вертаючись, Він зголоднів.
\end{tcolorbox}
\begin{tcolorbox}
\textsubscript{19} І побачив Він при дорозі одне фіґове дерево, і до нього прийшов, та нічого, крім листя самого, на нім не знайшов. І до нього Він каже: Нехай плоду із тебе не буде ніколи повіки! І фіґове дерево зараз усохло.
\end{tcolorbox}
\begin{tcolorbox}
\textsubscript{20} А учні, побачивши це, дивувалися та говорили: Як швидко всохло це фіґове дерево!...
\end{tcolorbox}
\begin{tcolorbox}
\textsubscript{21} Ісус же промовив у відповідь їм: Поправді кажу вам: Коли б мали ви віру, і не мали сумніву, то вчинили б не тільки як із фіґовим деревом, а якби й цій горі ви сказали: Порушся та кинься до моря, то й станеться те!
\end{tcolorbox}
\begin{tcolorbox}
\textsubscript{22} І все, чого ви в молитві попросите з вірою, то одержите.
\end{tcolorbox}
\begin{tcolorbox}
\textsubscript{23} А коли Він прийшов у храм і навчав, поприходили первосвященики й старші народу до Нього й сказали: Якою Ти владою чиниш оце? І хто Тобі владу цю дав?
\end{tcolorbox}
\begin{tcolorbox}
\textsubscript{24} Ісус же промовив у відповідь їм: Запитаю й Я вас одне слово. Як про нього дасте Мені відповідь, то й Я вам скажу, якою владою Я це чиню.
\end{tcolorbox}
\begin{tcolorbox}
\textsubscript{25} Іванове хрищення звідки було: із неба, чи від людей? Вони ж міркували собі й говорили: Коли скажемо: Із неба, відкаже Він нам: Чого ж ви йому не повірили?
\end{tcolorbox}
\begin{tcolorbox}
\textsubscript{26} А як скажемо: Від людей, боїмося народу, бо Івана вважають усі за пророка.
\end{tcolorbox}
\begin{tcolorbox}
\textsubscript{27} І сказали Ісусові в відповідь: Ми не знаємо. Відказав їм і Він: То й Я вам не скажу, якою владою Я це чиню.
\end{tcolorbox}
\begin{tcolorbox}
\textsubscript{28} А як вам здається? Один чоловік мав двох синів. Прийшовши до першого, він сказав: Піди но, дитино, сьогодні, працюй у винограднику!
\end{tcolorbox}
\begin{tcolorbox}
\textsubscript{29} А той відповів і сказав: Готовий, панотче, і не пішов.
\end{tcolorbox}
\begin{tcolorbox}
\textsubscript{30} І, прийшовши до другого, так само сказав. А той відповів і сказав: Я не хочу. А потім покаявся, і пішов.
\end{tcolorbox}
\begin{tcolorbox}
\textsubscript{31} Котрий же з двох учинив волю батькову? Вони кажуть: Останній. Ісус промовляє до них: Поправді кажу вам, що митники та блудодійки випереджують вас у Боже Царство.
\end{tcolorbox}
\begin{tcolorbox}
\textsubscript{32} Бо прийшов був до вас дорогою праведности Іван, та йому не повірили ви, а митники та блудодійки йняли йому віри. А ви бачили, та проте не покаялися й опісля, щоб повірити йому.
\end{tcolorbox}
\begin{tcolorbox}
\textsubscript{33} Послухайте іншої притчі. Був господар один. Насадив виноградника він, обгородив його муром, видовбав у ньому чавило, башту поставив, і віддав його винарям, та й пішов.
\end{tcolorbox}
\begin{tcolorbox}
\textsubscript{34} Коли ж надійшов час плодів, він до винарів послав рабів своїх, щоб прийняти плоди свої.
\end{tcolorbox}
\begin{tcolorbox}
\textsubscript{35} Винарі ж рабів його похапали, і одного побили, а другого замордували, а того вкаменували.
\end{tcolorbox}
\begin{tcolorbox}
\textsubscript{36} Знов послав він інших рабів, більш як перше, та й їм учинили те саме.
\end{tcolorbox}
\begin{tcolorbox}
\textsubscript{37} Нарешті послав до них сина свого і сказав: Посоромляться сина мого.
\end{tcolorbox}
\begin{tcolorbox}
\textsubscript{38} Але винарі, як побачили сина, міркувати собі стали: Це спадкоємець; ходім, замордуймо його, і заберемо його спадщину!
\end{tcolorbox}
\begin{tcolorbox}
\textsubscript{39} І, схопивши його, вони вивели за виноградник його, та й убили.
\end{tcolorbox}
\begin{tcolorbox}
\textsubscript{40} Отож, як прибуде той пан виноградника, що зробить він тим винарям?
\end{tcolorbox}
\begin{tcolorbox}
\textsubscript{41} Вони кажуть Йому: Злочинців погубить жорстоко, виноградника ж віддасть іншим винарям, що будуть плоди віддавати йому своєчасно.
\end{tcolorbox}
\begin{tcolorbox}
\textsubscript{42} Ісус промовляє до них: Чи ви не читали ніколи в Писанні: Камінь, що його будівничі відкинули, той наріжним став каменем; від Господа сталося це, і дивне воно в очах наших!
\end{tcolorbox}
\begin{tcolorbox}
\textsubscript{43} Тому кажу вам, що від вас Царство Боже відійметься, і дасться народові, що плоди його буде приносити.
\end{tcolorbox}
\begin{tcolorbox}
\textsubscript{44} І хто впаде на цей камінь розіб'ється, а на кого він сам упаде то розчавить його.
\end{tcolorbox}
\begin{tcolorbox}
\textsubscript{45} А як первосвященики та фарисеї почули ці притчі Його, то вони зрозуміли, що про них Він говорить.
\end{tcolorbox}
\begin{tcolorbox}
\textsubscript{46} І намагались схопити Його, але побоялись людей, бо вважали Його за Пророка.
\end{tcolorbox}
\subsection{CHAPTER 22}
\begin{tcolorbox}
\textsubscript{1} А Ісус, відповідаючи, знов почав говорити їм притчами, кажучи:
\end{tcolorbox}
\begin{tcolorbox}
\textsubscript{2} Царство Небесне подібне одному цареві, що весілля справляв був для сина свого.
\end{tcolorbox}
\begin{tcolorbox}
\textsubscript{3} І послав він своїх рабів покликати тих, хто був на весілля запрошений, та ті не хотіли прийти.
\end{tcolorbox}
\begin{tcolorbox}
\textsubscript{4} Знову послав він інших рабів, наказуючи: Скажіть запрошеним: Ось я приготував обід свій, закололи бики й відгодоване, і все готове. Ідіть на весілля!
\end{tcolorbox}
\begin{tcolorbox}
\textsubscript{5} Та вони злегковажили та порозходились, той на поле своє, а той на свій торг.
\end{tcolorbox}
\begin{tcolorbox}
\textsubscript{6} А останні, похапавши рабів його, знущалися, та й повбивали їх.
\end{tcolorbox}
\begin{tcolorbox}
\textsubscript{7} І розгнівався цар, і послав своє військо, і вигубив тих убійників, а їхнє місто спалив.
\end{tcolorbox}
\begin{tcolorbox}
\textsubscript{8} Тоді каже рабам своїм: Весілля готове, але недостойні були ті покликані.
\end{tcolorbox}
\begin{tcolorbox}
\textsubscript{9} Тож підіть на роздоріжжя, і кого тільки спіткаєте, кличте їх на весілля.
\end{tcolorbox}
\begin{tcolorbox}
\textsubscript{10} І вийшовши раби ті на роздоріжжя, зібрали всіх, кого тільки спіткали, злих і добрих. І весільна кімната гістьми переповнилась.
\end{tcolorbox}
\begin{tcolorbox}
\textsubscript{11} Як прийшов же той цар на гостей подивитись, побачив там чоловіка, в одежу весільну не вбраного,
\end{tcolorbox}
\begin{tcolorbox}
\textsubscript{12} та й каже йому: Як ти, друже, ввійшов сюди, не мавши одежі весільної? Той же мовчав.
\end{tcolorbox}
\begin{tcolorbox}
\textsubscript{13} Тоді цар сказав своїм слугам: Зв'яжіть йому ноги та руки, та й киньте до зовнішньої темряви, буде плач там і скрегіт зубів...
\end{tcolorbox}
\begin{tcolorbox}
\textsubscript{14} Бо багато покликаних, та вибраних мало.
\end{tcolorbox}
\begin{tcolorbox}
\textsubscript{15} Тоді фарисеї пішли й умовлялись, як зловити на слові Його.
\end{tcolorbox}
\begin{tcolorbox}
\textsubscript{16} І посилають до Нього своїх учнів із іродіянами, і кажуть: Учителю, знаємо ми, що Ти справедливий, і наставляєш на Божу дорогу правдиво, і не зважаєш ні на кого, бо на людське обличчя не дивишся Ти.
\end{tcolorbox}
\begin{tcolorbox}
\textsubscript{17} Скажи ж нам, як здається Тобі: чи годиться давати податок для кесаря, чи ні?
\end{tcolorbox}
\begin{tcolorbox}
\textsubscript{18} А Ісус, знавши їхнє лукавство, сказав: Чого ви, лицеміри, Мене випробовуєте?
\end{tcolorbox}
\begin{tcolorbox}
\textsubscript{19} Покажіть Мені гріш податковий. І принесли динарія Йому.
\end{tcolorbox}
\begin{tcolorbox}
\textsubscript{20} А Він каже до них: Чий це образ і напис?
\end{tcolorbox}
\begin{tcolorbox}
\textsubscript{21} Ті відказують: Кесарів. Тоді каже Він їм: Тож віддайте кесареве кесареві, а Богові Боже.
\end{tcolorbox}
\begin{tcolorbox}
\textsubscript{22} А почувши таке, вони диву далися. І, лишивши Його, відійшли.
\end{tcolorbox}
\begin{tcolorbox}
\textsubscript{23} Того дня приступили до Нього саддукеї, що твердять, ніби нема воскресення, і запитали Його,
\end{tcolorbox}
\begin{tcolorbox}
\textsubscript{24} та й сказали: Учителю, Мойсей наказав: Коли хто помре, не мавши дітей, то нехай його брат візьме вдову його, і відновить насіння для брата свого.
\end{tcolorbox}
\begin{tcolorbox}
\textsubscript{25} Було ж у нас сім братів. І перший, одружившись, умер, і, не мавши насіння, зоставив дружину свою братові своєму.
\end{tcolorbox}
\begin{tcolorbox}
\textsubscript{26} Так само і другий, і третій, аж до сьомого.
\end{tcolorbox}
\begin{tcolorbox}
\textsubscript{27} А по всіх вмерла й жінка.
\end{tcolorbox}
\begin{tcolorbox}
\textsubscript{28} Отож, у воскресенні котрому з сімох вона дружиною буде? Бо всі мали її.
\end{tcolorbox}
\begin{tcolorbox}
\textsubscript{29} Ісус же промовив у відповідь їм: Помиляєтесь ви, не знавши писання, ні Божої сили.
\end{tcolorbox}
\begin{tcolorbox}
\textsubscript{30} Бо в воскресенні ні женяться, ані заміж виходять, але як Анголи ті на небі.
\end{tcolorbox}
\begin{tcolorbox}
\textsubscript{31} А про воскресення померлих хіба не читали прореченого вам від Бога, що каже:
\end{tcolorbox}
\begin{tcolorbox}
\textsubscript{32} Я Бог Авраамів, і Бог Ісаків, і Бог Яковів; Бог не є Богом мертвих, а живих.
\end{tcolorbox}
\begin{tcolorbox}
\textsubscript{33} А народ, чувши це, дивувався науці Його.
\end{tcolorbox}
\begin{tcolorbox}
\textsubscript{34} Фарисеї ж, почувши, що Він уста замкнув саддукеям, зібралися разом.
\end{tcolorbox}
\begin{tcolorbox}
\textsubscript{35} І спитався один із них, учитель Закону, Його випробовуючи й кажучи:
\end{tcolorbox}
\begin{tcolorbox}
\textsubscript{36} Учителю, котра заповідь найбільша в Законі?
\end{tcolorbox}
\begin{tcolorbox}
\textsubscript{37} Він же промовив йому: Люби Господа Бога свого всім серцем своїм, і всією душею своєю, і всією своєю думкою.
\end{tcolorbox}
\begin{tcolorbox}
\textsubscript{38} Це найбільша й найперша заповідь.
\end{tcolorbox}
\begin{tcolorbox}
\textsubscript{39} А друга однакова з нею: Люби свого ближнього, як самого себе.
\end{tcolorbox}
\begin{tcolorbox}
\textsubscript{40} На двох оцих заповідях увесь Закон і Пророки стоять.
\end{tcolorbox}
\begin{tcolorbox}
\textsubscript{41} Коли ж фарисеї зібрались, Ісус їх запитав,
\end{tcolorbox}
\begin{tcolorbox}
\textsubscript{42} і сказав: Що ви думаєте про Христа? Чий Він син? Вони Йому кажуть: Давидів.
\end{tcolorbox}
\begin{tcolorbox}
\textsubscript{43} Він до них промовляє: Як же то силою Духа Давид Його Господом зве, коли каже:
\end{tcolorbox}
\begin{tcolorbox}
\textsubscript{44} Промовив Господь Господеві моєму: сядь праворуч Мене, доки не покладу Я Твоїх ворогів підніжком ногам Твоїм.
\end{tcolorbox}
\begin{tcolorbox}
\textsubscript{45} Тож, коли Давид зве Його Господом, як же Він йому син?
\end{tcolorbox}
\begin{tcolorbox}
\textsubscript{46} І ніхто не спромігся відповісти Йому ані слова... І ніхто з того дня не наважувався більш питати Його.
\end{tcolorbox}
\subsection{CHAPTER 23}
\begin{tcolorbox}
\textsubscript{1} Тоді промовив Ісус до народу й до учнів Своїх,
\end{tcolorbox}
\begin{tcolorbox}
\textsubscript{2} і сказав: На сидінні Мойсеєвім усілися книжники та фарисеї.
\end{tcolorbox}
\begin{tcolorbox}
\textsubscript{3} Тож усе, що вони скажуть вам, робіть і виконуйте; та за вчинками їхніми не робіть, бо говорять вони та не роблять того!
\end{tcolorbox}
\begin{tcolorbox}
\textsubscript{4} Вони ж в'яжуть тяжкі тягарі, і кладуть їх на людські рамена, самі ж навіть пальцем своїм не хотять їх порушити...
\end{tcolorbox}
\begin{tcolorbox}
\textsubscript{5} Усі ж учинки свої вони роблять, щоб їх бачили люди, і богомілля свої розширяють, і здовжують китиці.
\end{tcolorbox}
\begin{tcolorbox}
\textsubscript{6} І люблять вони передніші місця на бенкетах, і передніші лавки в синагогах,
\end{tcolorbox}
\begin{tcolorbox}
\textsubscript{7} і привіти на ринках, і щоб звали їх люди: Учителю!
\end{tcolorbox}
\begin{tcolorbox}
\textsubscript{8} А ви вчителями не звіться, бо один вам Учитель, а ви всі брати.
\end{tcolorbox}
\begin{tcolorbox}
\textsubscript{9} І не називайте нікого отцем на землі, бо один вам Отець, що на небі.
\end{tcolorbox}
\begin{tcolorbox}
\textsubscript{10} І не звіться наставниками, бо один вам Наставник, Христос.
\end{tcolorbox}
\begin{tcolorbox}
\textsubscript{11} Хто між вами найбільший, хай слугою вам буде!
\end{tcolorbox}
\begin{tcolorbox}
\textsubscript{12} Хто бо підноситься, буде понижений, хто ж понижується, той піднесеться.
\end{tcolorbox}
\begin{tcolorbox}
\textsubscript{13} Горе ж вам, книжники та фарисеї, лицеміри, що перед людьми зачиняєте Царство Небесне, бо й самі ви не входите, ані тих, хто хоче ввійти, увійти не пускаєте!
\end{tcolorbox}
\begin{tcolorbox}
\textsubscript{14} Горе ж вам, книжники та фарисеї, лицеміри, що вдовині хати поїдаєте, і напоказ молитесь довго, через те осуд тяжчий ви приймете!
\end{tcolorbox}
\begin{tcolorbox}
\textsubscript{15} Горе вам, книжники та фарисеї, лицеміри, що обходите море та землю, щоб придбати нововірця одного; а коли те стається, то робите його сином геєнни, вдвоє гіршим від вас!
\end{tcolorbox}
\begin{tcolorbox}
\textsubscript{16} Горе вам, проводирі ви сліпі, що говорите: Коли хто поклянеться храмом, то нічого; а хто поклянеться золотом храму, то той винуватий.
\end{tcolorbox}
\begin{tcolorbox}
\textsubscript{17} Нерозумні й сліпі, що бо більше: чи золото, чи той храм, що освячує золото?
\end{tcolorbox}
\begin{tcolorbox}
\textsubscript{18} І: Коли хто поклянеться жертівником, то нічого, а хто поклянеться жертвою, що на нім, то він винуватий.
\end{tcolorbox}
\begin{tcolorbox}
\textsubscript{19} Нерозумні й сліпі, що бо більше: чи жертва, чи той жертівник, що освячує жертву?
\end{tcolorbox}
\begin{tcolorbox}
\textsubscript{20} Отож, хто клянеться жертівником, клянеться ним та всім, що на ньому.
\end{tcolorbox}
\begin{tcolorbox}
\textsubscript{21} І хто храмом клянеться, клянеться ним та Тим, Хто живе в нім.
\end{tcolorbox}
\begin{tcolorbox}
\textsubscript{22} І хто небом клянеться, клянеться Божим престолом і Тим, Хто на ньому сидить.
\end{tcolorbox}
\begin{tcolorbox}
\textsubscript{23} Горе вам, книжники та фарисеї, лицеміри, що даєте десятину із м'яти, і ганусу й кмину, але найважливіше в Законі покинули: суд, милосердя та віру; це треба робити, і того не кидати.
\end{tcolorbox}
\begin{tcolorbox}
\textsubscript{24} Проводирі ви сліпі, що відціджуєте комаря, а верблюда ковтаєте!
\end{tcolorbox}
\begin{tcolorbox}
\textsubscript{25} Горе вам, книжники та фарисеї, лицеміри, що чистите зовнішність кухля та миски, а всередині повні вони здирства й кривди!
\end{tcolorbox}
\begin{tcolorbox}
\textsubscript{26} Фарисею сліпий, очисти перше середину кухля, щоб чистий він був і назовні!
\end{tcolorbox}
\begin{tcolorbox}
\textsubscript{27} Горе вам, книжники та фарисеї, лицеміри, що подібні до гробів побілених, які гарними зверху здаються, а всередині повні трупних кісток та всякої нечистости!
\end{tcolorbox}
\begin{tcolorbox}
\textsubscript{28} Так і ви, назовні здаєтеся людям за праведних, а всередині повні лицемірства та беззаконня!
\end{tcolorbox}
\begin{tcolorbox}
\textsubscript{29} Горе вам, книжники та фарисеї, лицеміри, що пророкам надгробники ставите, і праведникам прикрашаєте пам'ятники,
\end{tcolorbox}
\begin{tcolorbox}
\textsubscript{30} та говорите: Якби ми жили за днів наших батьків, то ми не були б спільниками їхніми в крові пророків.
\end{tcolorbox}
\begin{tcolorbox}
\textsubscript{31} Тим самим на себе свідкуєте, що сини ви убивців пророків.
\end{tcolorbox}
\begin{tcolorbox}
\textsubscript{32} Доповніть і ви міру провини ваших батьків!
\end{tcolorbox}
\begin{tcolorbox}
\textsubscript{33} О змії, о роде гадючий, як ви втечете від засуду до геєнни?
\end{tcolorbox}
\begin{tcolorbox}
\textsubscript{34} І ось тому посилаю до вас Я пророків, і мудрих, і книжників; частину їх ви повбиваєте та розіпнете, а частину їх ви бичуватимете в синагогах своїх, і будете гнати з міста до міста.
\end{tcolorbox}
\begin{tcolorbox}
\textsubscript{35} Щоб спала на вас уся праведна кров, що пролита була на землі, від крови Авеля праведного, аж до крови Захарія, Варахіїного сина, що ви замордували його між храмом і жертівником!
\end{tcolorbox}
\begin{tcolorbox}
\textsubscript{36} Поправді кажу вам: Оце все спаде на рід цей!
\end{tcolorbox}
\begin{tcolorbox}
\textsubscript{37} Єрусалиме, Єрусалиме, що вбиваєш пророків та каменуєш посланих до тебе! Скільки разів Я хотів зібрати діти твої, як та квочка збирає під крила курчаток своїх, та ви не захотіли!
\end{tcolorbox}
\begin{tcolorbox}
\textsubscript{38} Ось ваш дім залишається порожній для вас!
\end{tcolorbox}
\begin{tcolorbox}
\textsubscript{39} Говорю бо Я вам: Відтепер ви Мене не побачите, аж поки не скажете: Благословенний, Хто йде у Господнє Ім'я!
\end{tcolorbox}
\subsection{CHAPTER 24}
\begin{tcolorbox}
\textsubscript{1} І вийшов Ісус і від храму пішов. І підійшли Його учні, щоб Йому показати будинки храмові.
\end{tcolorbox}
\begin{tcolorbox}
\textsubscript{2} Він же промовив у відповідь їм: Чи бачите ви все оце? Поправді кажу вам: Не залишиться тут навіть камінь на камені, який не зруйнується!...
\end{tcolorbox}
\begin{tcolorbox}
\textsubscript{3} Коли ж Він сидів на Оливній горі, підійшли Його учні до Нього самотньо й спитали: Скажи нам, коли станеться це? І яка буде ознака приходу Твого й кінця віку?
\end{tcolorbox}
\begin{tcolorbox}
\textsubscript{4} Ісус же промовив у відповідь їм: Стережіться, щоб вас хто не звів!
\end{tcolorbox}
\begin{tcolorbox}
\textsubscript{5} Бо багато хто прийде в Ім'я Моє, кажучи: Я Христос. І зведуть багатьох.
\end{tcolorbox}
\begin{tcolorbox}
\textsubscript{6} Ви ж про війни почуєте, і про воєнні чутки, глядіть, не лякайтесь, бо статись належить тому. Але це не кінець ще.
\end{tcolorbox}
\begin{tcolorbox}
\textsubscript{7} Бо повстане народ на народ, і царство на царство, і голод, мор та землетруси настануть місцями.
\end{tcolorbox}
\begin{tcolorbox}
\textsubscript{8} А все це початок терпінь породільних.
\end{tcolorbox}
\begin{tcolorbox}
\textsubscript{9} На муки тоді видаватимуть вас, і вбиватимуть вас, і вас будуть ненавидіти всі народи за Ймення Моє.
\end{tcolorbox}
\begin{tcolorbox}
\textsubscript{10} І багато-хто в той час спокусяться, і видавати один одного будуть, і один одного будуть ненавидіти.
\end{tcolorbox}
\begin{tcolorbox}
\textsubscript{11} Постане багато фальшивих пророків, і зведуть багатьох.
\end{tcolorbox}
\begin{tcolorbox}
\textsubscript{12} І через розріст беззаконства любов багатьох охолоне.
\end{tcolorbox}
\begin{tcolorbox}
\textsubscript{13} А хто витерпить аж до кінця, той буде спасений!
\end{tcolorbox}
\begin{tcolorbox}
\textsubscript{14} І проповідана буде ця Євангелія Царства по цілому світові, на свідоцтво народам усім. І тоді прийде кінець!
\end{tcolorbox}
\begin{tcolorbox}
\textsubscript{15} Тож, коли ви побачите ту гидоту спустошення, що про неї звіщав був пророк Даниїл, на місці святому, хто читає, нехай розуміє,
\end{tcolorbox}
\begin{tcolorbox}
\textsubscript{16} тоді ті, хто в Юдеї, нехай в гори втікають.
\end{tcolorbox}
\begin{tcolorbox}
\textsubscript{17} Хто на покрівлі, нехай той не сходить узяти речі з дому свого.
\end{tcolorbox}
\begin{tcolorbox}
\textsubscript{18} І хто на полі, хай назад не вертається взяти одежу свою.
\end{tcolorbox}
\begin{tcolorbox}
\textsubscript{19} Горе ж вагітним і тим, хто годує грудьми, за днів тих!
\end{tcolorbox}
\begin{tcolorbox}
\textsubscript{20} Моліться ж, щоб ваша втеча не сталась зимою, ані в суботу.
\end{tcolorbox}
\begin{tcolorbox}
\textsubscript{21} Бо скорбота велика настане тоді, якої не було з первопочину світу аж досі й не буде.
\end{tcolorbox}
\begin{tcolorbox}
\textsubscript{22} І коли б не вкоротились ті дні, не спаслася б ніяка людина; але через вибраних дні ті вкоротяться.
\end{tcolorbox}
\begin{tcolorbox}
\textsubscript{23} Тоді, як хто скаже до вас: Ото, Христос тут чи Отам, не йміть віри.
\end{tcolorbox}
\begin{tcolorbox}
\textsubscript{24} Бо постануть христи неправдиві, і неправдиві пророки, і будуть чинити великі ознаки та чуда, що звели б, коли б можна, і вибраних.
\end{tcolorbox}
\begin{tcolorbox}
\textsubscript{25} Оце Я наперед вам сказав.
\end{tcolorbox}
\begin{tcolorbox}
\textsubscript{26} А коли скажуть вам: Ось Він у пустині не виходьте, Ось Він у криївках не вірте!
\end{tcolorbox}
\begin{tcolorbox}
\textsubscript{27} Бо як блискавка та вибігає зо сходу, і з'являється аж до заходу, так буде і прихід Сина Людського.
\end{tcolorbox}
\begin{tcolorbox}
\textsubscript{28} Бо де труп, там зберуться орли.
\end{tcolorbox}
\begin{tcolorbox}
\textsubscript{29} І зараз, по скорботі тих днів, сонце затьмиться, і місяць не дасть свого світла, і зорі попадають з неба, і сили небесні порушаться.
\end{tcolorbox}
\begin{tcolorbox}
\textsubscript{30} І того часу на небі з'явиться знак Сина Людського, і тоді заголосять всі земні племена, і побачать вони Сина Людського, що йтиме на хмарах небесних із великою потугою й славою.
\end{tcolorbox}
\begin{tcolorbox}
\textsubscript{31} І пошле Анголів Своїх Він із голосним сурмовим гуком, і зберуть Його вибраних від вітрів чотирьох, від кінців неба аж до кінців його.
\end{tcolorbox}
\begin{tcolorbox}
\textsubscript{32} Від дерева ж фіґового навчіться прикладу: коли віття його вже розпукується, і кинеться листя, то ви знаєте, що близько вже літо.
\end{tcolorbox}
\begin{tcolorbox}
\textsubscript{33} Так і ви: коли все це побачите, знайте, що близько, під дверима!
\end{tcolorbox}
\begin{tcolorbox}
\textsubscript{34} Поправді кажу вам: не перейде цей рід, аж усе оце станеться.
\end{tcolorbox}
\begin{tcolorbox}
\textsubscript{35} Небо й земля проминеться, але не минуться слова Мої!
\end{tcolorbox}
\begin{tcolorbox}
\textsubscript{36} А про день той й годину не знає ніхто: ані Анголи небесні, ані Син, лише Сам Отець.
\end{tcolorbox}
\begin{tcolorbox}
\textsubscript{37} Як було за днів Ноєвих, так буде і прихід Сина Людського.
\end{tcolorbox}
\begin{tcolorbox}
\textsubscript{38} Бо так само, як за днів до потопу всі їли й пили, женилися й заміж виходили, аж до дня, коли Ной увійшов до ковчегу,
\end{tcolorbox}
\begin{tcolorbox}
\textsubscript{39} і не знали, аж поки потоп не прийшов та й усіх не забрав, так буде і прихід Сина Людського.
\end{tcolorbox}
\begin{tcolorbox}
\textsubscript{40} Будуть двоє на полі тоді, один візьметься, а другий полишиться.
\end{tcolorbox}
\begin{tcolorbox}
\textsubscript{41} Дві будуть молоти на жорнах, одна візьметься, а друга полишиться.
\end{tcolorbox}
\begin{tcolorbox}
\textsubscript{42} Тож пильнуйте, бо не знаєте, котрого дня прийде Господь ваш.
\end{tcolorbox}
\begin{tcolorbox}
\textsubscript{43} Знайте ж це, що коли б знав господар, о котрій сторожі прийде злодій, то він пильнував би, і підкопати свого дому не дав би.
\end{tcolorbox}
\begin{tcolorbox}
\textsubscript{44} Тому будьте готові й ви, бо прийде Син Людський тієї години, коли ви не думаєте!
\end{tcolorbox}
\begin{tcolorbox}
\textsubscript{45} Хто ж вірний і мудрий раб, якого пан поставив над своїми челядниками давати своєчасно поживу для них?
\end{tcolorbox}
\begin{tcolorbox}
\textsubscript{46} Блаженний той раб, що пан його прийде та знайде, що робить він так!
\end{tcolorbox}
\begin{tcolorbox}
\textsubscript{47} Поправді кажу вам, що над цілим маєтком своїм він поставить його.
\end{tcolorbox}
\begin{tcolorbox}
\textsubscript{48} А як той злий раб скаже у серці своїм: Забариться пан мій прийти,
\end{tcolorbox}
\begin{tcolorbox}
\textsubscript{49} і зачне бити товаришів своїх, а їсти та пити з п'яницями,
\end{tcolorbox}
\begin{tcolorbox}
\textsubscript{50} то пан того раба прийде дня, якого він не сподівається, і о годині, якої не знає.
\end{tcolorbox}
\begin{tcolorbox}
\textsubscript{51} І він пополовині розітне його, і визначить долю йому з лицемірами, буде плач там і скрегіт зубів!
\end{tcolorbox}
\subsection{CHAPTER 25}
\begin{tcolorbox}
\textsubscript{1} Тоді Царство Небесне буде подібне до десяти дів, що побрали каганці свої, та й пішли зустрічати молодого.
\end{tcolorbox}
\begin{tcolorbox}
\textsubscript{2} П'ять же з них нерозумні були, а п'ять мудрі.
\end{tcolorbox}
\begin{tcolorbox}
\textsubscript{3} Нерозумні ж, узявши каганці, не взяли із собою оливи.
\end{tcolorbox}
\begin{tcolorbox}
\textsubscript{4} А мудрі набрали оливи в посудинки разом із своїми каганцями.
\end{tcolorbox}
\begin{tcolorbox}
\textsubscript{5} А коли забаривсь молодий, то всі задрімали й поснули.
\end{tcolorbox}
\begin{tcolorbox}
\textsubscript{6} А опівночі крик залунав: Ось молодий, виходьте назустріч!
\end{tcolorbox}
\begin{tcolorbox}
\textsubscript{7} Схопились тоді всі ті діви, і каганці свої наготували.
\end{tcolorbox}
\begin{tcolorbox}
\textsubscript{8} Нерозумні ж сказали до мудрих: Дайте нам із своєї оливи, бо наші каганці ось гаснуть.
\end{tcolorbox}
\begin{tcolorbox}
\textsubscript{9} Мудрі ж відповіли та сказали: Щоб, бува, нам і вам не забракло, краще вдайтеся до продавців, і купіть собі.
\end{tcolorbox}
\begin{tcolorbox}
\textsubscript{10} І як вони купувати пішли, то прибув молодий; і готові ввійшли на весілля з ним, і замкнені двері були.
\end{tcolorbox}
\begin{tcolorbox}
\textsubscript{11} А потім прийшла й решта дів і казала: Пане, пане, відчини нам!
\end{tcolorbox}
\begin{tcolorbox}
\textsubscript{12} Він же в відповідь їм проказав: Поправді кажу вам, не знаю я вас!
\end{tcolorbox}
\begin{tcolorbox}
\textsubscript{13} Тож пильнуйте, бо не знаєте ні дня, ні години, коли прийде Син Людський!
\end{tcolorbox}
\begin{tcolorbox}
\textsubscript{14} Так само ж один чоловік, як відходив, покликав своїх рабів і передав їм добро своє.
\end{tcolorbox}
\begin{tcolorbox}
\textsubscript{15} І одному він дав п'ять талантів, а другому два, а тому один, кожному за спроможністю його. І відійшов.
\end{tcolorbox}
\begin{tcolorbox}
\textsubscript{16} А той, що взяв п'ять талантів, негайно пішов і орудував ними, і набув він п'ять інших талантів.
\end{tcolorbox}
\begin{tcolorbox}
\textsubscript{17} Так само ж і той, що взяв два і він ще два інших набув.
\end{tcolorbox}
\begin{tcolorbox}
\textsubscript{18} А той, що одного взяв, пішов та й закопав його в землю, і сховав срібло пана свого.
\end{tcolorbox}
\begin{tcolorbox}
\textsubscript{19} По довгому ж часі вернувся пан тих рабів, та й від них зажадав обрахунку.
\end{tcolorbox}
\begin{tcolorbox}
\textsubscript{20} І прийшов той, що взяв п'ять талантів, приніс іще п'ять талантів і сказав: Пане мій, п'ять талантів мені передав ти, ось я здобув інші п'ять талантів.
\end{tcolorbox}
\begin{tcolorbox}
\textsubscript{21} Сказав же йому його пан: Гаразд, рабе добрий і вірний! Ти в малому був вірний, над великим поставлю тебе, увійди до радощів пана свого!
\end{tcolorbox}
\begin{tcolorbox}
\textsubscript{22} Підійшов же й той, що взяв два таланти, і сказав: Два таланти мені передав ти, ось іще два таланти здобув я.
\end{tcolorbox}
\begin{tcolorbox}
\textsubscript{23} казав йому пан його: Гаразд, рабе добрий і вірний! Ти в малому був вірний, над великим поставлю тебе, увійди до радощів пана свого!
\end{tcolorbox}
\begin{tcolorbox}
\textsubscript{24} Підійшов же і той, що одного таланта взяв, і сказав: Я знав тебе, пане, що тверда ти людина, ти жнеш, де не сіяв, і збираєш, де не розсипав.
\end{tcolorbox}
\begin{tcolorbox}
\textsubscript{25} І я побоявся, пішов і таланта твого сховав у землю. Ото маєш своє...
\end{tcolorbox}
\begin{tcolorbox}
\textsubscript{26} І відповів його пан і сказав йому: Рабе лукавий і лінивий! Ти знав, що я жну, де не сіяв, і збираю, де не розсипав?
\end{tcolorbox}
\begin{tcolorbox}
\textsubscript{27} Тож тобі було треба віддати гроші мої грошомінам, і, вернувшись, я взяв би з прибутком своє.
\end{tcolorbox}
\begin{tcolorbox}
\textsubscript{28} Візьміть же від нього таланта, і віддайте тому, що десять талантів він має.
\end{tcolorbox}
\begin{tcolorbox}
\textsubscript{29} Бо кожному, хто має, дасться йому та й додасться, хто ж не має, забереться від нього й те, що він має.
\end{tcolorbox}
\begin{tcolorbox}
\textsubscript{30} А раба непотрібного вкиньте до зовнішньої темряви, буде плач там і скрегіт зубів!
\end{tcolorbox}
\begin{tcolorbox}
\textsubscript{31} Коли ж прийде Син Людський у славі Своїй, і всі Анголи з Ним, тоді Він засяде на престолі слави Своєї.
\end{tcolorbox}
\begin{tcolorbox}
\textsubscript{32} І перед Ним усі народи зберуться, і Він відділить одного від одного їх, як відділяє вівчар овець від козлів.
\end{tcolorbox}
\begin{tcolorbox}
\textsubscript{33} І поставить Він вівці праворуч Себе, а козлята ліворуч.
\end{tcolorbox}
\begin{tcolorbox}
\textsubscript{34} Тоді скаже Цар тим, хто праворуч Його: Прийдіть, благословенні Мого Отця, посядьте Царство, уготоване вам від закладин світу.
\end{tcolorbox}
\begin{tcolorbox}
\textsubscript{35} Бо Я голодував був і ви нагодували Мене, прагнув і ви напоїли Мене, мандрівником Я був і Мене прийняли ви.
\end{tcolorbox}
\begin{tcolorbox}
\textsubscript{36} Був нагий і Мене зодягли ви, слабував і Мене ви відвідали, у в'язниці Я був і прийшли ви до Мене.
\end{tcolorbox}
\begin{tcolorbox}
\textsubscript{37} Тоді відповідять Йому праведні й скажуть: Господи, коли то Тебе ми голодного бачили і нагодували, або спрагненого і напоїли?
\end{tcolorbox}
\begin{tcolorbox}
\textsubscript{38} Коли то Тебе мандрівником ми бачили і прийняли, чи нагим і зодягли?
\end{tcolorbox}
\begin{tcolorbox}
\textsubscript{39} Коли то Тебе ми недужого бачили, чи в в'язниці і до Тебе прийшли?
\end{tcolorbox}
\begin{tcolorbox}
\textsubscript{40} Цар відповість і промовить до них: Поправді кажу вам: що тільки вчинили ви одному з найменших братів Моїх цих, те Мені ви вчинили.
\end{tcolorbox}
\begin{tcolorbox}
\textsubscript{41} Тоді скаже й тим, хто ліворуч: Ідіть ви від Мене, прокляті, у вічний огонь, що дияволові та його посланцям приготований.
\end{tcolorbox}
\begin{tcolorbox}
\textsubscript{42} Бо Я голодував був і не нагодували Мене, прагнув і ви не напоїли Мене,
\end{tcolorbox}
\begin{tcolorbox}
\textsubscript{43} мандрівником Я був і не прийняли ви Мене, був нагий і не зодягли ви Мене, слабий і в в'язниці і Мене не відвідали ви.
\end{tcolorbox}
\begin{tcolorbox}
\textsubscript{44} Тоді відповідять і вони, промовляючи: Господи, коли то Тебе ми голодного бачили, або спрагненого, або мандрівником, чи нагого, чи недужого, чи в в'язниці і не послужили Тобі?
\end{tcolorbox}
\begin{tcolorbox}
\textsubscript{45} Тоді Він відповість їм і скаже: Поправді кажу вам: чого тільки одному з найменших цих ви не вчинили, Мені не вчинили!
\end{tcolorbox}
\begin{tcolorbox}
\textsubscript{46} І ці підуть на вічную муку, а праведники на вічне життя.
\end{tcolorbox}
\subsection{CHAPTER 26}
\begin{tcolorbox}
\textsubscript{1} І сталось, коли закінчив Ісус усі ці слова, Він сказав Своїм учням:
\end{tcolorbox}
\begin{tcolorbox}
\textsubscript{2} Ви знаєте, що через два дні буде Пасха, і Людський Син буде виданий на розп'яття.
\end{tcolorbox}
\begin{tcolorbox}
\textsubscript{3} Тоді первосвященики, і книжники, і старші народу зібралися в домі первосвященика, званого Кайяфою,
\end{tcolorbox}
\begin{tcolorbox}
\textsubscript{4} і радилися, щоб підступом взяти Ісуса й забити.
\end{tcolorbox}
\begin{tcolorbox}
\textsubscript{5} І вони говорили: Та не в свято, щоб бува колотнеча в народі не сталась.
\end{tcolorbox}
\begin{tcolorbox}
\textsubscript{6} Коли ж Ісус був у Віфанії, у домі Симона прокаженого,
\end{tcolorbox}
\begin{tcolorbox}
\textsubscript{7} підійшла одна жінка до Нього, маючи алябастрову пляшечку дорогоцінного мира, і вилила на Його голову, як сидів при столі Він.
\end{tcolorbox}
\begin{tcolorbox}
\textsubscript{8} Як побачили ж учні це, то обурилися та й сказали: Нащо таке марнотратство?
\end{tcolorbox}
\begin{tcolorbox}
\textsubscript{9} Бо дорого можна було б це продати, і віддати убогим.
\end{tcolorbox}
\begin{tcolorbox}
\textsubscript{10} Зрозумівши Ісус, промовив до них: Чого прикрість ви робите жінці? Вона ж добрий учинок зробила Мені.
\end{tcolorbox}
\begin{tcolorbox}
\textsubscript{11} Бо вбогих ви маєте завжди з собою, а Мене не постійно ви маєте.
\end{tcolorbox}
\begin{tcolorbox}
\textsubscript{12} Бо, виливши миро оце на тіло Моє, вона те вчинила на похорон Мій.
\end{tcolorbox}
\begin{tcolorbox}
\textsubscript{13} Поправді кажу вам: де тільки оця Євангелія проповідувана буде в цілому світі, на пам'ятку їй буде сказане й те, що зробила вона!
\end{tcolorbox}
\begin{tcolorbox}
\textsubscript{14} Тоді один із Дванадцятьох, званий Юдою Іскаріотським, подався до первосвящеників,
\end{tcolorbox}
\begin{tcolorbox}
\textsubscript{15} і сказав: Що хочете дати мені, і я вам Його видам? І вони йому виплатили тридцять срібняків.
\end{tcolorbox}
\begin{tcolorbox}
\textsubscript{16} І він відтоді шукав слушного часу, щоб видати Його.
\end{tcolorbox}
\begin{tcolorbox}
\textsubscript{17} А першого дня Опрісноків учні підійшли до Ісуса й сказали Йому: Де хочеш, щоб ми приготували пасху спожити Тобі?
\end{tcolorbox}
\begin{tcolorbox}
\textsubscript{18} А Він відказав: Ідіть до такого то в місто, і перекажіть йому: каже Вчитель: час Мій близький, справлю Пасху з Своїми учнями в тебе.
\end{tcolorbox}
\begin{tcolorbox}
\textsubscript{19} І учні зробили, як звелів їм Ісус, і зачали пасху готувати.
\end{tcolorbox}
\begin{tcolorbox}
\textsubscript{20} А коли настав вечір, Він із дванадцятьма учнями сів за стіл.
\end{tcolorbox}
\begin{tcolorbox}
\textsubscript{21} І, як вони споживали, Він сказав: Поправді кажу вам, що один із вас видасть Мене...
\end{tcolorbox}
\begin{tcolorbox}
\textsubscript{22} А вони засмутилися тяжко, і кожен із них став питати Його: Чи не я то, о Господи?
\end{tcolorbox}
\begin{tcolorbox}
\textsubscript{23} А Він відповів і промовив: Хто руку свою вмочить у миску зо Мною, той видасть Мене.
\end{tcolorbox}
\begin{tcolorbox}
\textsubscript{24} Людський Син справді йде, як про Нього написано; але горе тому чоловікові, що видасть Людського Сина! Було б краще йому, коли б той чоловік не родився!
\end{tcolorbox}
\begin{tcolorbox}
\textsubscript{25} Юда ж, зрадник Його, відповів і сказав: Чи не я то, Учителю? Відказав Він йому: Ти сказав...
\end{tcolorbox}
\begin{tcolorbox}
\textsubscript{26} Як вони ж споживали, Ісус узяв хліб, і поблагословив, поламав, і давав Своїм учням, і сказав: Прийміть, споживайте, це тіло Моє.
\end{tcolorbox}
\begin{tcolorbox}
\textsubscript{27} А взявши чашу, і подяку вчинивши, Він подав їм і сказав: Пийте з неї всі,
\end{tcolorbox}
\begin{tcolorbox}
\textsubscript{28} бо це кров Моя Нового Заповіту, що за багатьох проливається на відпущення гріхів!
\end{tcolorbox}
\begin{tcolorbox}
\textsubscript{29} Кажу ж вам, що віднині не питиму Я від оцього плоду виноградного аж до дня, коли з вами його новим питиму в Царстві Мого Отця.
\end{tcolorbox}
\begin{tcolorbox}
\textsubscript{30} А коли відспівали вони, то на гору Оливну пішли.
\end{tcolorbox}
\begin{tcolorbox}
\textsubscript{31} Промовляє тоді їм Ісус: Усі ви через Мене спокуситеся ночі цієї. Бо написано: Уражу пастиря, і розпорошаться вівці отари.
\end{tcolorbox}
\begin{tcolorbox}
\textsubscript{32} По воскресенні ж Своїм Я вас випереджу в Галілеї.
\end{tcolorbox}
\begin{tcolorbox}
\textsubscript{33} А Петро відповів і сказав Йому: Якби й усі спокусились про Тебе, я не спокушуся ніколи.
\end{tcolorbox}
\begin{tcolorbox}
\textsubscript{34} Промовив до нього Ісус: Поправді кажу тобі, що ночі цієї, перше ніж заспіває півень, відречешся ти тричі від Мене...
\end{tcolorbox}
\begin{tcolorbox}
\textsubscript{35} Говорить до Нього Петро: Коли б мені навіть умерти з Тобою, я не відречуся від Тебе! Так сказали й усі учні.
\end{tcolorbox}
\begin{tcolorbox}
\textsubscript{36} Тоді з ними приходить Ісус до місцевости, званої Гефсиманія, і промовляє до учнів: Посидьте ви тут, аж поки піду й помолюся отам.
\end{tcolorbox}
\begin{tcolorbox}
\textsubscript{37} І, взявши Петра й двох синів Зеведеєвих, зачав сумувати й тужити.
\end{tcolorbox}
\begin{tcolorbox}
\textsubscript{38} Тоді промовляє до них: Обгорнена сумом смертельним душа Моя! Залишіться тут, і попильнуйте зо Мною...
\end{tcolorbox}
\begin{tcolorbox}
\textsubscript{39} І, трохи далі пройшовши, упав Він долілиць, та молився й благав: Отче Мій, коли можна, нехай обмине ця чаша Мене... Та проте, не як Я хочу, а як Ти...
\end{tcolorbox}
\begin{tcolorbox}
\textsubscript{40} І, вернувшись до учнів, знайшов їх, що спали, і промовив Петрові: Отак, не змогли ви й однієї години попильнувати зо Мною?...
\end{tcolorbox}
\begin{tcolorbox}
\textsubscript{41} Пильнуйте й моліться, щоб не впасти на спробу, бадьорий бо дух, але немічне тіло.
\end{tcolorbox}
\begin{tcolorbox}
\textsubscript{42} Відійшовши ще вдруге, Він молився й благав: Отче Мій, як ця чаша не може минути Мене, щоб не пити її, нехай станеться воля Твоя!
\end{tcolorbox}
\begin{tcolorbox}
\textsubscript{43} І, прийшовши, ізнову знайшов їх, що спали, бо зважніли їм очі були.
\end{tcolorbox}
\begin{tcolorbox}
\textsubscript{44} І, залишивши їх, знов пішов, і помолився втретє, те саме слово промовивши.
\end{tcolorbox}
\begin{tcolorbox}
\textsubscript{45} Потому приходить до учнів і їм промовляє: Ви ще далі спите й спочиваєте? Ось година наблизилась, і до рук грішникам виданий буде Син Людський...
\end{tcolorbox}
\begin{tcolorbox}
\textsubscript{46} Уставайте, ходім, ось наблизився Мій зрадник!
\end{tcolorbox}
\begin{tcolorbox}
\textsubscript{47} І коли Він іще говорив, аж ось прийшов Юда, один із Дванадцятьох, а з ним люду багато від первосвящеників і старших народу з мечами та киями.
\end{tcolorbox}
\begin{tcolorbox}
\textsubscript{48} А зрадник Його дав був знака їм, кажучи: Кого поцілую, то Він, беріть Його.
\end{tcolorbox}
\begin{tcolorbox}
\textsubscript{49} І зараз Він підійшов до Ісуса й сказав: Радій, Учителю! І поцілував Його.
\end{tcolorbox}
\begin{tcolorbox}
\textsubscript{50} Ісус же йому відказав: Чого, друже, прийшов ти? Тоді приступили та руки наклали на Ісуса, і схопили Його.
\end{tcolorbox}
\begin{tcolorbox}
\textsubscript{51} А ось один із тих, що з Ісусом були, витягнув руку, і меча свого вихопив та й рубонув раба первосвященика, і відтяв йому вухо.
\end{tcolorbox}
\begin{tcolorbox}
\textsubscript{52} Тоді промовляє до нього Ісус: Сховай свого меча в його місце, бо всі, хто візьме меча, від меча і загинуть.
\end{tcolorbox}
\begin{tcolorbox}
\textsubscript{53} Чи ти думаєш, що не можу тепер упросити Свого Отця, і Він дасть Мені зараз більше дванадцяти леґіонів Анголів?
\end{tcolorbox}
\begin{tcolorbox}
\textsubscript{54} Але як має збутись Писання, що так статися мусить?
\end{tcolorbox}
\begin{tcolorbox}
\textsubscript{55} Тієї години промовив Ісус до народу: Немов на розбійника вийшли з мечами та киями, щоб узяти Мене! Я щоденно у храмі сидів і навчав, і Мене не взяли ви.
\end{tcolorbox}
\begin{tcolorbox}
\textsubscript{56} Це ж сталось усе, щоб збулися писання пророків. Усі учні тоді залишили Його й повтікали...
\end{tcolorbox}
\begin{tcolorbox}
\textsubscript{57} А вони схопили Ісуса, і повели до первосвященика Кайяфи, де зібралися книжники й старші.
\end{tcolorbox}
\begin{tcolorbox}
\textsubscript{58} Петро ж здалека йшов услід за Ним аж до двору первосвященика, і, ввійшовши всередину, сів із службою, щоб бачити кінець.
\end{tcolorbox}
\begin{tcolorbox}
\textsubscript{59} А первосвященики та ввесь синедріон шукали на Ісуса неправдивого свідчення, щоб смерть заподіяти Йому,
\end{tcolorbox}
\begin{tcolorbox}
\textsubscript{60} і не знаходили, хоч кривосвідків багато підходило. Аж ось накінець з'явилися двоє,
\end{tcolorbox}
\begin{tcolorbox}
\textsubscript{61} і сказали: Він говорив: Я можу зруйнувати храм Божий, і за три дні збудувати його.
\end{tcolorbox}
\begin{tcolorbox}
\textsubscript{62} Тоді первосвященик устав і до Нього сказав: Ти нічого не відповідаєш на те, що свідчать супроти Тебе?
\end{tcolorbox}
\begin{tcolorbox}
\textsubscript{63} Ісус же мовчав. І первосвященик сказав Йому: Заприсягаю Тебе Живим Богом, щоб нам Ти сказав, чи Христос Ти, Син Божий?
\end{tcolorbox}
\begin{tcolorbox}
\textsubscript{64} Промовляє до нього Ісус: Ти сказав... А навіть повім вам: відтепер ви побачите Людського Сина, що сидітиме праворуч сили Божої, і на хмарах небесних приходитиме!
\end{tcolorbox}
\begin{tcolorbox}
\textsubscript{65} Тоді первосвященик роздер одежу свою та й сказав: Він богозневажив! Нащо нам іще свідки потрібні? Ось ви чули тепер Його богозневагу!
\end{tcolorbox}
\begin{tcolorbox}
\textsubscript{66} Як вам іздається? Вони ж відповіли та сказали: Повинен умерти!
\end{tcolorbox}
\begin{tcolorbox}
\textsubscript{67} Тоді стали плювати на обличчя Йому, та бити по щоках Його, інші ж киями били,
\end{tcolorbox}
\begin{tcolorbox}
\textsubscript{68} і казали: Пророкуй нам, Христе, хто то вдарив Тебе?...
\end{tcolorbox}
\begin{tcolorbox}
\textsubscript{69} А Петро перед домом сидів на подвір'ї. І приступила до нього служниця одна та й сказала: І ти був з Ісусом Галілеянином!
\end{tcolorbox}
\begin{tcolorbox}
\textsubscript{70} А він перед всіма відрікся, сказавши: Не відаю я, що ти кажеш...
\end{tcolorbox}
\begin{tcolorbox}
\textsubscript{71} А коли до воріт він підходив, побачила інша його та й сказала приявним там людям: Оцей був з Ісусом Назарянином!
\end{tcolorbox}
\begin{tcolorbox}
\textsubscript{72} І він знову відрікся та став присягатись: Не знаю Цього Чоловіка!...
\end{tcolorbox}
\begin{tcolorbox}
\textsubscript{73} Підійшли ж трохи згодом присутні й сказали Петрові: І ти справді з отих, та й мова твоя виявляє тебе.
\end{tcolorbox}
\begin{tcolorbox}
\textsubscript{74} Тоді він став клястись та божитись: Не знаю Цього Чоловіка! І заспівав півень хвилі тієї...
\end{tcolorbox}
\begin{tcolorbox}
\textsubscript{75} І згадав Петро сказане слово Ісусове: Перше ніж заспіває півень, відречешся ти тричі від Мене. І, вийшовши звідти, він гірко заплакав...
\end{tcolorbox}
\subsection{CHAPTER 27}
\begin{tcolorbox}
\textsubscript{1} А коли настав ранок, усі первосвященики й старші народу зібрали нараду супроти Ісуса, щоб Йому заподіяти смерть.
\end{tcolorbox}
\begin{tcolorbox}
\textsubscript{2} І, зв'язавши Його, повели, та й Понтію Пилату намісникові віддали.
\end{tcolorbox}
\begin{tcolorbox}
\textsubscript{3} Тоді Юда, що видав Його, як побачив, що Його засудили, розкаявся, і вернув тридцять срібняків первосвященикам і старшим,
\end{tcolorbox}
\begin{tcolorbox}
\textsubscript{4} та й сказав: Я згрішив, невинну кров видавши. Вони ж відказали: А нам що до того? Дивись собі сам...
\end{tcolorbox}
\begin{tcolorbox}
\textsubscript{5} І, кинувши в храм срібняки, відійшов, а потому пішов, та й повісився...
\end{tcolorbox}
\begin{tcolorbox}
\textsubscript{6} А первосвященики, як взяли срібняки, то сказали: Цього не годиться покласти до сховку церковного, це ж бо заплата за кров.
\end{tcolorbox}
\begin{tcolorbox}
\textsubscript{7} А порадившись, купили на них поле ганчарське, щоб мандрівників ховати,
\end{tcolorbox}
\begin{tcolorbox}
\textsubscript{8} чому й зветься те поле полем крови аж до сьогодні.
\end{tcolorbox}
\begin{tcolorbox}
\textsubscript{9} Тоді справдилось те, що сказав був пророк Єремія, промовляючи: І взяли вони тридцять срібняків, заплату Оціненого, що Його оцінили сини Ізраїлеві,
\end{tcolorbox}
\begin{tcolorbox}
\textsubscript{10} і дали їх за поле ганчарське, як Господь наказав був мені.
\end{tcolorbox}
\begin{tcolorbox}
\textsubscript{11} Ісус же став перед намісником. І намісник Його запитав і сказав: Чи Ти Цар Юдейський? Ісус же йому відказав: Ти кажеш.
\end{tcolorbox}
\begin{tcolorbox}
\textsubscript{12} Коли ж первосвященики й старші Його винуватили, Він нічого на те не відказував.
\end{tcolorbox}
\begin{tcolorbox}
\textsubscript{13} Тоді каже до Нього Пилат: Чи не чуєш, як багато на Тебе свідкують?
\end{tcolorbox}
\begin{tcolorbox}
\textsubscript{14} А Він ні на одне слово йому не відказував, так що намісник був дуже здивований.
\end{tcolorbox}
\begin{tcolorbox}
\textsubscript{15} Мав же намісник звичай відпускати на свято народові в'язня одного, котрого хотіли вони.
\end{tcolorbox}
\begin{tcolorbox}
\textsubscript{16} Був тоді в'язень відомий, що звався Варавва.
\end{tcolorbox}
\begin{tcolorbox}
\textsubscript{17} І, як зібрались вони, то сказав їм Пилат: Котрого бажаєте, щоб я вам відпустив: Варавву, чи Ісуса, що зветься Христос?
\end{tcolorbox}
\begin{tcolorbox}
\textsubscript{18} Бо він знав, що Його через заздрощі видали.
\end{tcolorbox}
\begin{tcolorbox}
\textsubscript{19} Коли ж він сидів на суддевім сидінні, його дружина прислала сказати йому: Нічого не май з отим Праведником, бо сьогодні вві сні я багато терпіла з-за Нього...
\end{tcolorbox}
\begin{tcolorbox}
\textsubscript{20} А первосвященики й старші попідмовляли народ, щоб просити за Варавву, а Ісусові смерть заподіяти.
\end{tcolorbox}
\begin{tcolorbox}
\textsubscript{21} Намісник тоді відповів і сказав їм: Котрого ж із двох ви бажаєте, щоб я вам відпустив? Вони ж відказали: Варавву.
\end{tcolorbox}
\begin{tcolorbox}
\textsubscript{22} Пилат каже до них: А що ж маю зробити з Ісусом, що зветься Христос? Усі закричали: Нехай розп'ятий буде!...
\end{tcolorbox}
\begin{tcolorbox}
\textsubscript{23} А намісник спитав: Яке ж зло Він зробив? Вони ж зачали ще сильніше кричати й казати: Нехай розп'ятий буде!
\end{tcolorbox}
\begin{tcolorbox}
\textsubscript{24} І, як побачив Пилат, що нічого не вдіє, а неспокій ще більший стається, набрав він води, та й перед народом умив свої руки й сказав: Я невинний у крові Його! Самі ви побачите...
\end{tcolorbox}
\begin{tcolorbox}
\textsubscript{25} А ввесь народ відповів і сказав: На нас Його кров і на наших дітей!...
\end{tcolorbox}
\begin{tcolorbox}
\textsubscript{26} Тоді відпустив їм Варавву, а Ісуса, збичувавши, він видав, щоб розп'ятий був.
\end{tcolorbox}
\begin{tcolorbox}
\textsubscript{27} Тоді то намісникові вояки, до преторія взявши Ісуса, зібрали на Нього ввесь відділ.
\end{tcolorbox}
\begin{tcolorbox}
\textsubscript{28} І, роздягнувши Його, багряницю наділи на Нього.
\end{tcolorbox}
\begin{tcolorbox}
\textsubscript{29} І, сплівши з тернини вінка, поклали Йому на голову, а тростину в правицю Його. І, навколішки падаючи перед Ним, сміялися з Нього й казали: Радій, Царю Юдейський!
\end{tcolorbox}
\begin{tcolorbox}
\textsubscript{30} І, плювавши на Нього, хапали тростину, та й по голові Його били...
\end{tcolorbox}
\begin{tcolorbox}
\textsubscript{31} А коли назнущалися з Нього, зняли з Нього плаща, і зодягнули в одежу Його. І повели Його на розп'яття.
\end{tcolorbox}
\begin{tcolorbox}
\textsubscript{32} А виходячи, стріли одного кірінеянина, Симон на ймення, його змусили нести для Нього хреста.
\end{tcolorbox}
\begin{tcolorbox}
\textsubscript{33} І, прибувши на місце, що зветься Голгофа, цебто сказати Череповище,
\end{tcolorbox}
\begin{tcolorbox}
\textsubscript{34} дали Йому пити вина, із гіркотою змішаного, та, покуштувавши, Він пити не схотів.
\end{tcolorbox}
\begin{tcolorbox}
\textsubscript{35} А розп'явши Його, вони поділили одежу Його, кинувши жереба.
\end{tcolorbox}
\begin{tcolorbox}
\textsubscript{36} І, посідавши, стерегли Його там.
\end{tcolorbox}
\begin{tcolorbox}
\textsubscript{37} І напис провини Його помістили над Його головою: Це Ісус, Цар Юдейський.
\end{tcolorbox}
\begin{tcolorbox}
\textsubscript{38} Тоді розп'ято з Ним двох розбійників: одного праворуч, а одного ліворуч.
\end{tcolorbox}
\begin{tcolorbox}
\textsubscript{39} А хто побіч проходив, Його лихословили та головами своїми хитали,
\end{tcolorbox}
\begin{tcolorbox}
\textsubscript{40} і казали: Ти, що храма руйнуєш та за три дні будуєш, спаси Самого Себе! Коли Ти Божий Син, то зійди з хреста!
\end{tcolorbox}
\begin{tcolorbox}
\textsubscript{41} Так само ж і первосвященики з книжниками та старшими, насміхаючися, говорили:
\end{tcolorbox}
\begin{tcolorbox}
\textsubscript{42} Він інших спасав, а Самого Себе не може спасти! Коли Цар Він Ізраїлів, нехай зійде тепер із хреста, і ми повіримо Йому!
\end{tcolorbox}
\begin{tcolorbox}
\textsubscript{43} Покладав Він надію на Бога, нехай Той Його тепер визволить, якщо Він угодний Йому. Бо Він говорив: Я Син Божий...
\end{tcolorbox}
\begin{tcolorbox}
\textsubscript{44} Також насміхалися з Нього й розбійники, що з Ним були розп'яті.
\end{tcolorbox}
\begin{tcolorbox}
\textsubscript{45} А від години шостої аж до години дев'ятої темрява сталась по цілій землі!
\end{tcolorbox}
\begin{tcolorbox}
\textsubscript{46} А коло години дев'ятої скрикнув Ісус гучним голосом, кажучи: Елі, Елі, лама савахтані? цебто: Боже Мій, Боже Мій, нащо Мене Ти покинув?...
\end{tcolorbox}
\begin{tcolorbox}
\textsubscript{47} Дехто ж із тих, що стояли там, це почули й казали, що Він кличе Іллю.
\end{tcolorbox}
\begin{tcolorbox}
\textsubscript{48} А один із них зараз побіг і взяв губку та, оцтом її наповнивши, настромив на тростину й давав Йому пити.
\end{tcolorbox}
\begin{tcolorbox}
\textsubscript{49} Інші казали: Чекай но, побачмо, чи прийде Ілля визволяти Його.
\end{tcolorbox}
\begin{tcolorbox}
\textsubscript{50} А Ісус знову голосом гучним іскрикнув, і духа віддав...
\end{tcolorbox}
\begin{tcolorbox}
\textsubscript{51} І ось завіса у храмі роздерлась надвоє від верху аж додолу, і земля потряслася, і зачали розпадатися скелі,
\end{tcolorbox}
\begin{tcolorbox}
\textsubscript{52} і повідкривались гроби, і повставало багато тіл спочилих святих,
\end{tcolorbox}
\begin{tcolorbox}
\textsubscript{53} а з гробів повиходивши, по Його воскресенні, до міста святого ввійшли, і багатьом із'явились.
\end{tcolorbox}
\begin{tcolorbox}
\textsubscript{54} А сотник та ті, що Ісуса з ним стерегли, як землетруса побачили, і те, що там сталося, налякалися дуже й казали: Він був справді Син Божий!
\end{tcolorbox}
\begin{tcolorbox}
\textsubscript{55} Було там багато й жінок, що дивилися здалека, і що за Ісусом прийшли з Галілеї, і Йому прислуговували.
\end{tcolorbox}
\begin{tcolorbox}
\textsubscript{56} Між ними була Марія Магдалина, і Марія, мати Якова й Йосипа, і мати синів Зеведеєвих.
\end{tcolorbox}
\begin{tcolorbox}
\textsubscript{57} А коли настав вечір, то прийшов муж багатий із Ариматеї, на ім'я Йосип, що й сам був навчався в Ісуса.
\end{tcolorbox}
\begin{tcolorbox}
\textsubscript{58} Він прийшов до Пилата й просив тіла Ісусового. Пилат ізвелів тоді видати.
\end{tcolorbox}
\begin{tcolorbox}
\textsubscript{59} І взяв Йосип Ісусове тіло, обгорнув його плащаницею чистою,
\end{tcolorbox}
\begin{tcolorbox}
\textsubscript{60} і поклав його в гробі новому своїм, що був висік у скелі. До дверей гробових привалив він великого каменя, та й відійшов.
\end{tcolorbox}
\begin{tcolorbox}
\textsubscript{61} Була ж там Марія Магдалина та інша Марія, що сиділи насупроти гробу.
\end{tcolorbox}
\begin{tcolorbox}
\textsubscript{62} А наступного дня, що за п'ятницею, до Пилата зібралися первосвященики та фарисеї,
\end{tcolorbox}
\begin{tcolorbox}
\textsubscript{63} і сказали: Пригадали ми, пане, собі, що обманець отой, як живий іще був, то сказав: По трьох днях Я воскресну.
\end{tcolorbox}
\begin{tcolorbox}
\textsubscript{64} Звели ж гріб стерегти аж до третього дня, щоб учні Його не прийшли, та й не вкрали Його, і не сказали народові: Він із мертвих воскрес! І буде остання обмана гірша за першу...
\end{tcolorbox}
\begin{tcolorbox}
\textsubscript{65} Відказав їм Пилат: Сторожу ви маєте, ідіть, забезпечте, як знаєте.
\end{tcolorbox}
\begin{tcolorbox}
\textsubscript{66} І вони відійшли, і, запечатавши каменя, біля гробу сторожу поставили.
\end{tcolorbox}
\subsection{CHAPTER 28}
\begin{tcolorbox}
\textsubscript{1} Як минула ж субота, на світанку дня першого в тижні, прийшла Марія Магдалина та інша Марія побачити гріб.
\end{tcolorbox}
\begin{tcolorbox}
\textsubscript{2} І великий ось ставсь землетрус, бо зійшов із неба Ангол Господній, і, приступивши, відвалив від гробу каменя, та й сів на ньому.
\end{tcolorbox}
\begin{tcolorbox}
\textsubscript{3} Його ж постать була, як та блискавка, а шати його були білі, як сніг.
\end{tcolorbox}
\begin{tcolorbox}
\textsubscript{4} І від страху перед ним затряслася сторожа, та й стала, як мертва.
\end{tcolorbox}
\begin{tcolorbox}
\textsubscript{5} А Ангол озвався й промовив жінкам: Не лякайтеся, бо я знаю, що Ісуса розп'ятого це ви шукаєте.
\end{tcolorbox}
\begin{tcolorbox}
\textsubscript{6} Нема Його тут, бо воскрес, як сказав. Підійдіть, подивіться на місце, де знаходився Він.
\end{tcolorbox}
\begin{tcolorbox}
\textsubscript{7} Ідіть же хутко, і скажіть Його учням, що воскрес Він із мертвих, і ото випереджує вас в Галілеї, там Його ви побачите. Ось, вам я звістив!
\end{tcolorbox}
\begin{tcolorbox}
\textsubscript{8} І пішли вони хутко від гробу, зо страхом і великою радістю, і побігли, щоб учнів Його сповістити.
\end{tcolorbox}
\begin{tcolorbox}
\textsubscript{9} Аж ось перестрів їх Ісус і сказав: Радійте! Вони ж підійшли, обняли Його ноги і вклонились Йому до землі.
\end{tcolorbox}
\begin{tcolorbox}
\textsubscript{10} Промовляє тоді їм Ісус: Не лякайтесь! Ідіть, повідомте братів Моїх, нехай вони йдуть у Галілею, там побачать Мене!
\end{tcolorbox}
\begin{tcolorbox}
\textsubscript{11} Коли ж вони йшли, ось дехто зо сторожі до міста прийшли та й первосвященикам розповіли все, що сталось.
\end{tcolorbox}
\begin{tcolorbox}
\textsubscript{12} І, зібравшись зо старшими, вони врадили раду, і дали сторожі чимало срібняків,
\end{tcolorbox}
\begin{tcolorbox}
\textsubscript{13} і сказали: Розповідайте: Його учні вночі прибули, і вкрали Його, як ми спали.
\end{tcolorbox}
\begin{tcolorbox}
\textsubscript{14} Як почує ж намісник про це, то його ми переконаємо, і від клопоту визволимо вас.
\end{tcolorbox}
\begin{tcolorbox}
\textsubscript{15} І, взявши вони срібняки, зробили, як навчено їх. І пронеслося слово оце між юдеями, і тримається аж до сьогодні.
\end{tcolorbox}
\begin{tcolorbox}
\textsubscript{16} Одинадцять же учнів пішли в Галілею на гору, куди звелів їм Ісус.
\end{tcolorbox}
\begin{tcolorbox}
\textsubscript{17} І як вони Його вгледіли, поклонились Йому до землі, а дехто вагався.
\end{tcolorbox}
\begin{tcolorbox}
\textsubscript{18} А Ісус підійшов і промовив до них та й сказав: Дана Мені всяка влада на небі й на землі.
\end{tcolorbox}
\begin{tcolorbox}
\textsubscript{19} Тож ідіть, і навчіть всі народи, христячи їх в Ім'я Отця, і Сина, і Святого Духа,
\end{tcolorbox}
\begin{tcolorbox}
\textsubscript{20} навчаючи їх зберігати все те, що Я вам заповів. І ото, Я перебуватиму з вами повсякденно аж до кінця віку! Амінь
\end{tcolorbox}
