\section{BOOK 63}
\subsection{CHAPTER 1}
\begin{tcolorbox}
\textsubscript{1} Старець улюбленому Гаєві, якого я направду люблю.
\end{tcolorbox}
\begin{tcolorbox}
\textsubscript{2} Улюблений, я молюся, щоб добре велося в усьому тобі, і щоб був ти здоровий, як добре ведеться душі твоїй.
\end{tcolorbox}
\begin{tcolorbox}
\textsubscript{3} Бо я дуже зрадів, як прийшли були браття, і засвідчили правду твою, як ти живеш у правді.
\end{tcolorbox}
\begin{tcolorbox}
\textsubscript{4} Я не маю більшої радости від цієї, щоб чути, що діти мої живуть у правді.
\end{tcolorbox}
\begin{tcolorbox}
\textsubscript{5} Улюблений, вірно ти чиниш, як що робиш для братті та для чужинців,
\end{tcolorbox}
\begin{tcolorbox}
\textsubscript{6} вони про любов твою свідчили Церкві; добре ти зробиш, як їх випровадиш, як достойно для Бога,
\end{tcolorbox}
\begin{tcolorbox}
\textsubscript{7} бо вийшли вони ради Ймення Його, нічого не взявши від поган.
\end{tcolorbox}
\begin{tcolorbox}
\textsubscript{8} Отож, ми повинні приймати таких, щоб бути співробітниками правді.
\end{tcolorbox}
\begin{tcolorbox}
\textsubscript{9} Я до Церкви писав був, але Діотреф, що любить бути першим у них, нас не приймає.
\end{tcolorbox}
\begin{tcolorbox}
\textsubscript{10} Тому то, коли я прийду, то згадаю про вчинки його, що їх робить, словами лихими обмовляючи нас. І він тим не задовольнюється, а й сам не приймає братів, і тим, що бажають приймати, боронить, і вигонить із Церкви.
\end{tcolorbox}
\begin{tcolorbox}
\textsubscript{11} Улюблений, не робися подібним до лихого, а до доброго: доброчинець від Бога, а злочинець Бога не бачив.
\end{tcolorbox}
\begin{tcolorbox}
\textsubscript{12} Про Димитрія свідчили всі й сама правда. І свідчимо й ми, а ви знаєте, що свідчення наше правдиве.
\end{tcolorbox}
\begin{tcolorbox}
\textsubscript{13} Багато хотів я писати, та не хочу писати до тебе чорнилом та очеретинкою,
\end{tcolorbox}
\begin{tcolorbox}
\textsubscript{14} але маю надію побачити тебе незабаром, і говорити устами до уст. (1-15) Мир тобі! Друзі вітають тебе. Привітай друзів пойменно! Амінь.
\end{tcolorbox}
