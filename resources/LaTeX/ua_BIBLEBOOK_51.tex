\section{BOOK 50}
\subsection{CHAPTER 1}
\begin{tcolorbox}
\textsubscript{1} Павло, із волі Божої апостол Христа Ісуса, і брат Тимофій
\end{tcolorbox}
\begin{tcolorbox}
\textsubscript{2} до святих і вірних братів у Христі, що в Колосах: благодать вам і мир від Бога, Отця нашого!
\end{tcolorbox}
\begin{tcolorbox}
\textsubscript{3} Ми дякуємо Богові, Отцеві Господа нашого Ісуса Христа, завжди за вас молячись,
\end{tcolorbox}
\begin{tcolorbox}
\textsubscript{4} прочувши про вашу віру в Христа Ісуса та про любов, яку маєте до всіх святих
\end{tcolorbox}
\begin{tcolorbox}
\textsubscript{5} через надію, приготовану в небі для вас, що про неї давніше ви чули в слові істини Євангелії,
\end{tcolorbox}
\begin{tcolorbox}
\textsubscript{6} що до вас прибула, і на цілому світі плодоносна й росте, як і в вас, з того дня, коли ви почули й пізнали благодать Божу в правді.
\end{tcolorbox}
\begin{tcolorbox}
\textsubscript{7} Отак ви і навчилися від Епафра, улюбленого співробітника нашого, що за вас він вірний служитель Христа,
\end{tcolorbox}
\begin{tcolorbox}
\textsubscript{8} що й виявив нам про вашу духовну любов.
\end{tcolorbox}
\begin{tcolorbox}
\textsubscript{9} Через це то й ми з того дня, як почули, не перестаємо молитись за вас та просити, щоб для пізнання волі Його були ви наповнені всякою мудрістю й розумом духовним,
\end{tcolorbox}
\begin{tcolorbox}
\textsubscript{10} щоб ви поводилися належно щодо Господа в усякому догодженні, в усякому доброму ділі приносячи плід і зростаючи в пізнанні Бога,
\end{tcolorbox}
\begin{tcolorbox}
\textsubscript{11} зміцняючись усякою силою за могучістю слави Його для всякої витривалости й довготерпіння з радістю,
\end{tcolorbox}
\begin{tcolorbox}
\textsubscript{12} дякуючи Отцеві, що вчинив нас достойними участи в спадщині святих у світлі,
\end{tcolorbox}
\begin{tcolorbox}
\textsubscript{13} що визволив нас із влади темряви й переставив нас до Царства Свого улюбленого Сина,
\end{tcolorbox}
\begin{tcolorbox}
\textsubscript{14} в Якім маємо відкуплення і прощення гріхів.
\end{tcolorbox}
\begin{tcolorbox}
\textsubscript{15} Він є образ невидимого Бога, роджений перш усякого творива.
\end{tcolorbox}
\begin{tcolorbox}
\textsubscript{16} Бо то Ним створено все на небі й на землі, видиме й невидиме, чи то престоли, чи то господства, чи то влади, чи то начальства, усе через Нього й для Нього створено!
\end{tcolorbox}
\begin{tcolorbox}
\textsubscript{17} А Він є перший від усього, і все Ним стоїть.
\end{tcolorbox}
\begin{tcolorbox}
\textsubscript{18} І Він Голова тіла, Церкви. Він початок, первороджений з мертвих, щоб у всьому Він мав першенство.
\end{tcolorbox}
\begin{tcolorbox}
\textsubscript{19} Бо вгодно було, щоб у Нім перебувала вся повнота,
\end{tcolorbox}
\begin{tcolorbox}
\textsubscript{20} і щоб Ним поєднати з Собою все, примиривши кров'ю хреста Його, через Нього, чи то земне, чи то небесне.
\end{tcolorbox}
\begin{tcolorbox}
\textsubscript{21} І вас, що були колись відчужені й вороги думкою в злих учинках,
\end{tcolorbox}
\begin{tcolorbox}
\textsubscript{22} тепер же примирив смертю в людськім тілі Його, щоб учинити вас святими, і непорочними, і неповинними перед Собою,
\end{tcolorbox}
\begin{tcolorbox}
\textsubscript{23} якщо тільки пробуваєте в вірі тверді та сталі, і не відпадаєте від надії Євангелії, що ви чули її, яка проповідана всьому створінню під небом, якій я, Павло, став служителем.
\end{tcolorbox}
\begin{tcolorbox}
\textsubscript{24} Тепер я радію в стражданнях своїх за вас, і доповнюю недостачу скорботи Христової в тілі своїм за тіло Його, що воно Церква;
\end{tcolorbox}
\begin{tcolorbox}
\textsubscript{25} якій я став служителем за Божим зарядженням, що для вас мені дане, щоб виконати Слово Боже,
\end{tcolorbox}
\begin{tcolorbox}
\textsubscript{26} Таємницю, заховану від віків і поколінь, а тепер виявлену Його святим,
\end{tcolorbox}
\begin{tcolorbox}
\textsubscript{27} що їм Бог захотів показати, яке багатство слави цієї таємниці між поганами, а вона Христос у вас, надія слави!
\end{tcolorbox}
\begin{tcolorbox}
\textsubscript{28} Його ми проповідуємо, нагадуючи кожній людині й навчаючи кожну людину всякої мудрости, щоб учинити кожну людину досконалою в Христі.
\end{tcolorbox}
\begin{tcolorbox}
\textsubscript{29} У тому й працюю я, борючися силою Його, яка сильно діє в мені.
\end{tcolorbox}
\subsection{CHAPTER 2}
\begin{tcolorbox}
\textsubscript{1} Я хочу, щоб ви знали, яку велику боротьбу я маю за вас і за тих, хто в Лаодикії, і за всіх, хто не бачив мого тілесного обличчя.
\end{tcolorbox}
\begin{tcolorbox}
\textsubscript{2} Хай потішаться їхні серця, у любові поєднані, для всякого багатства повного розуміння, для пізнання таємниці Бога, Христа,
\end{tcolorbox}
\begin{tcolorbox}
\textsubscript{3} в Якому всі скарби премудрости й пізнання заховані.
\end{tcolorbox}
\begin{tcolorbox}
\textsubscript{4} А це говорю, щоб ніхто вас не звів фальшивими доводами при суперечці.
\end{tcolorbox}
\begin{tcolorbox}
\textsubscript{5} Бо хоч тілом я й неприсутній, та духом я з вами, і з радістю бачу ваш порядок та твердість вашої віри в Христа.
\end{tcolorbox}
\begin{tcolorbox}
\textsubscript{6} Отже, як ви прийняли були Христа Ісуса Господа, так і в Ньому ходіть,
\end{tcolorbox}
\begin{tcolorbox}
\textsubscript{7} бувши вкорінені й збудовані на Ньому, та зміцнені в вірі, як вас навчено, збагачуючись у ній з подякою.
\end{tcolorbox}
\begin{tcolorbox}
\textsubscript{8} Стережіться, щоб ніхто вас не звів філософією та марною оманою за переданням людським, за стихіями світу, а не за Христом,
\end{tcolorbox}
\begin{tcolorbox}
\textsubscript{9} бо в Ньому тілесно живе вся повнота Божества.
\end{tcolorbox}
\begin{tcolorbox}
\textsubscript{10} І ви маєте в Нім повноту, а Він Голова всякої влади й начальства.
\end{tcolorbox}
\begin{tcolorbox}
\textsubscript{11} Ви в Ньому були й обрізані нерукотворним обрізанням, скинувши людське тіло гріховне в Христовім обрізанні.
\end{tcolorbox}
\begin{tcolorbox}
\textsubscript{12} Ви були з Ним поховані у хрищенні, у Ньому ви й разом воскресли через віру в силу Бога, що Він з мертвих Його воскресив.
\end{tcolorbox}
\begin{tcolorbox}
\textsubscript{13} І вас, що мертві були в гріхах та в необрізанні вашого тіла, Він оживив разом із Ним, простивши усі гріхи,
\end{tcolorbox}
\begin{tcolorbox}
\textsubscript{14} знищивши рукописання на нас, що наказами було проти нас, Він із середини взяв його та й прибив його на хресті,
\end{tcolorbox}
\begin{tcolorbox}
\textsubscript{15} роззброївши влади й начальства, сміливо їх вивів на посміховисько, перемігши їх на хресті!
\end{tcolorbox}
\begin{tcolorbox}
\textsubscript{16} Тож, хай ніхто вас не судить за їжу, чи за питво, чи за чергове свято, чи за новомісяччя, чи за суботи,
\end{tcolorbox}
\begin{tcolorbox}
\textsubscript{17} бо це тінь майбутнього, а тіло Христове.
\end{tcolorbox}
\begin{tcolorbox}
\textsubscript{18} Нехай вас не зводить ніхто удаваною покорою та службою Анголам, вдаючися до того, чого не бачив, нерозважно надимаючись своїм тілесним розумом,
\end{tcolorbox}
\begin{tcolorbox}
\textsubscript{19} а не тримачись Голови, від Якої все тіло, суглобами й зв'язями з'єднане й зміцнене, росте зростом Божим.
\end{tcolorbox}
\begin{tcolorbox}
\textsubscript{20} Отож, як ви вмерли з Христом для стихій світу, то чого ви, немов ті, хто в світі живе, пристаєте на постанови:
\end{tcolorbox}
\begin{tcolorbox}
\textsubscript{21} не дотикайся, ані їж, ані рухай,
\end{tcolorbox}
\begin{tcolorbox}
\textsubscript{22} бо то все знищиться, як уживати його, за приказами та наукою людською.
\end{tcolorbox}
\begin{tcolorbox}
\textsubscript{23} Воно ж має вид мудрости в самовільній службі й покорі та в знесилюванні тіла, та не має якогось значення, хіба щодо насичення тіла.
\end{tcolorbox}
\subsection{CHAPTER 3}
\begin{tcolorbox}
\textsubscript{1} Отож, коли ви воскресли з Христом, то шукайте того, що вгорі, де сидить Христос по Божій правиці.
\end{tcolorbox}
\begin{tcolorbox}
\textsubscript{2} Думайте про те, що вгорі, а не про те, що на землі.
\end{tcolorbox}
\begin{tcolorbox}
\textsubscript{3} Бож ви вмерли, а життя ваше сховане в Бозі з Христом.
\end{tcolorbox}
\begin{tcolorbox}
\textsubscript{4} Коли з'явиться Христос, наше життя, тоді з'явитеся з Ним у славі і ви.
\end{tcolorbox}
\begin{tcolorbox}
\textsubscript{5} Отож, умертвіть ваші земні члени: розпусту, нечисть, пристрасть, лиху пожадливість та зажерливість, що вона ідолослуження,
\end{tcolorbox}
\begin{tcolorbox}
\textsubscript{6} бо гнів Божий приходить за них на неслухняних.
\end{tcolorbox}
\begin{tcolorbox}
\textsubscript{7} І ви поміж ними ходили колись, як жили поміж ними.
\end{tcolorbox}
\begin{tcolorbox}
\textsubscript{8} Тепер же відкиньте і ви все оте: гнів, лютість, злобу, богозневагу, безсоромні слова з ваших уст.
\end{tcolorbox}
\begin{tcolorbox}
\textsubscript{9} Не кажіть неправди один на одного, якщо скинули з себе людину стародавню з її вчинками,
\end{tcolorbox}
\begin{tcolorbox}
\textsubscript{10} та зодягнулися в нову, що відновлюється для пізнання за образом Створителя її,
\end{tcolorbox}
\begin{tcolorbox}
\textsubscript{11} де нема ані геллена, ані юдея, обрізання та необрізання, варвара, скита, раба, вільного, але все та в усьому Христос!
\end{tcolorbox}
\begin{tcolorbox}
\textsubscript{12} Отож, зодягніться, як Божі вибранці, святі та улюблені, у щире милосердя, добротливість, покору, лагідність, довготерпіння.
\end{tcolorbox}
\begin{tcolorbox}
\textsubscript{13} Терпіть один одного, і прощайте собі, коли б мав хто на кого оскарження. Як і Христос вам простив, робіть так і ви!
\end{tcolorbox}
\begin{tcolorbox}
\textsubscript{14} А над усім тим зодягніться в любов, що вона союз досконалости!
\end{tcolorbox}
\begin{tcolorbox}
\textsubscript{15} І нехай мир Божий панує у ваших серцях, до якого й були ви покликані в одному тілі. І вдячними будьте!
\end{tcolorbox}
\begin{tcolorbox}
\textsubscript{16} Слово Христове нехай пробуває в вас рясно, у всякій премудрості. Навчайте та напоумляйте самих себе! Вдячно співайте у ваших серцях Господеві псалми, гімни, духовні пісні!
\end{tcolorbox}
\begin{tcolorbox}
\textsubscript{17} І все, що тільки робите словом чи ділом, усе робіть у Ім'я Господа Ісуса, дякуючи через Нього Богові й Отцеві.
\end{tcolorbox}
\begin{tcolorbox}
\textsubscript{18} Дружини, слухайтеся чоловіків своїх, як лицює то в Господі!
\end{tcolorbox}
\begin{tcolorbox}
\textsubscript{19} Чоловіки, любіть дружин своїх, і не будьте суворі до них!
\end{tcolorbox}
\begin{tcolorbox}
\textsubscript{20} Діти, будьте слухняні в усьому батькам, бо це Господеві приємне!
\end{tcolorbox}
\begin{tcolorbox}
\textsubscript{21} Батьки, не дратуйте дітей своїх, щоб на дусі не впали вони!
\end{tcolorbox}
\begin{tcolorbox}
\textsubscript{22} Раби, слухайтеся в усьому тілесних панів, і не працюйте тільки про людське око, немов підлещуючись, але в простоті серця, боячися Бога!
\end{tcolorbox}
\begin{tcolorbox}
\textsubscript{23} І все, що тільки чините, робіть від душі, немов Господеві, а не людям!
\end{tcolorbox}
\begin{tcolorbox}
\textsubscript{24} Знайте, що від Господа приймете в нагороду спадщину, бо служите ви Господеві Христові.
\end{tcolorbox}
\begin{tcolorbox}
\textsubscript{25} А хто кривдить, той одержить за свою кривду. Бо не дивиться Бог на особу!
\end{tcolorbox}
\subsection{CHAPTER 4}
\begin{tcolorbox}
\textsubscript{1} Пани, виявляйте до рабів справедливість та рівність, і знайте, що й для вас є на небі Господь!
\end{tcolorbox}
\begin{tcolorbox}
\textsubscript{2} Будьте тривалі в молитві, і пильнуйте з подякою в ній!
\end{tcolorbox}
\begin{tcolorbox}
\textsubscript{3} Моліться разом і за нас, щоб Бог нам відчинив двері слова, звіщати таємницю Христову, що за неї я й зв'язаний,
\end{tcolorbox}
\begin{tcolorbox}
\textsubscript{4} щоб з'явив я її, як звіщати належить мені.
\end{tcolorbox}
\begin{tcolorbox}
\textsubscript{5} Поводьтеся мудро з чужими, використовуючи час.
\end{tcolorbox}
\begin{tcolorbox}
\textsubscript{6} Слово ваше нехай буде завжди ласкаве, приправлене сіллю, щоб ви знали, як ви маєте кожному відповідати.
\end{tcolorbox}
\begin{tcolorbox}
\textsubscript{7} Що зо мною, то все вам розповість Тихик, улюблений брат і вірний служитель і співробітник у Господі.
\end{tcolorbox}
\begin{tcolorbox}
\textsubscript{8} Я саме на те його вислав до вас, щоб довідались ви про нас, і щоб ваші серця він потішив,
\end{tcolorbox}
\begin{tcolorbox}
\textsubscript{9} із Онисимом, вірним та улюбленим братом, який з-поміж вас. Вони все вам розповідять, що діється тут.
\end{tcolorbox}
\begin{tcolorbox}
\textsubscript{10} Поздоровлює вас Аристарх, ув'язнений разом зо мною, і Марко, небіж Варнавин, що про нього ви дістали накази; як прийде до вас, то прийміть його,
\end{tcolorbox}
\begin{tcolorbox}
\textsubscript{11} теж Ісус, на прізвище Юст, вони із обрізаних. Для Божого Царства єдині вони співробітники, що були мені втіхою.
\end{tcolorbox}
\begin{tcolorbox}
\textsubscript{12} Поздоровлює вас Епафрас, що з ваших, раб Христа Ісуса. Він завжди обстоює вас у молитвах, щоб ви досконалі були та наповнені всякою Божою волею.
\end{tcolorbox}
\begin{tcolorbox}
\textsubscript{13} І я свідчу за нього, що він має велику горливість про вас та про тих, що знаходяться в Лаодикії та в Гієраполі.
\end{tcolorbox}
\begin{tcolorbox}
\textsubscript{14} Вітає вас Лука, улюблений лікар, та Димас.
\end{tcolorbox}
\begin{tcolorbox}
\textsubscript{15} Привітайте братів, що в Лаодикії, і Німфана, і Церкву домашню його.
\end{tcolorbox}
\begin{tcolorbox}
\textsubscript{16} І як буде прочитаний лист цей у вас, то зробіть, щоб прочитаний був він також у Церкві Лаодикійській, а того, що написаний з Лаодикії, прочитайте і ви.
\end{tcolorbox}
\begin{tcolorbox}
\textsubscript{17} Та скажіть Архіпові: Доглядай того служіння, що прийняв його в Господі, щоб ти його виконав!
\end{tcolorbox}
\begin{tcolorbox}
\textsubscript{18} Привітання моєю рукою Павловою. Пам'ятайте про пута мої! Благодать Божа нехай буде з вами! Амінь.
\end{tcolorbox}
