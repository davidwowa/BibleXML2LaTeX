Book 7
\subsection{CHAPTER 1}
\begin{tcolorbox}
\textsubscript{1} І сталось за часу, коли судді судили, то був голод у Краю. І пішов був чоловік з Юдиного Віфлеєму мешкати в моавських полях, він і жінка та двоє синів його.
\end{tcolorbox}
\begin{tcolorbox}
\textsubscript{2} А ім'я тому чоловікові Елімелех, а ім'я жінці його Ноомі; і ім'я двох синів його Махлон і Кілйон, ефратяни з Віфлеєму Юдиного. І прийшли вони на моавські поля, та й залишилися.
\end{tcolorbox}
\begin{tcolorbox}
\textsubscript{3} І помер Елімелех, муж Ноомі, і зосталася вона та два їхні сини.
\end{tcolorbox}
\begin{tcolorbox}
\textsubscript{4} І взяли вони собі за жінок моавітянок, ім'я одній Орпа, а ім'я другій Рут. І сиділи вони там близько десяти літ.
\end{tcolorbox}
\begin{tcolorbox}
\textsubscript{5} І повмирали й вони обоє, Махлон та Кілйон. І позосталася та жінка по двох дітях своїх та по чоловікові своєму.
\end{tcolorbox}
\begin{tcolorbox}
\textsubscript{6} І встала вона та невістки її, і вернулися з моавських піль, бо почула на моавському полі, що Господь згадав про народ Свій, даючи їм хліба.
\end{tcolorbox}
\begin{tcolorbox}
\textsubscript{7} І вийшла вона з того місця, де була там, та обидві невістки з нею, та й пішли дорогою, щоб вернутися до Юдиного краю.
\end{tcolorbox}
\begin{tcolorbox}
\textsubscript{8} І сказала Ноомі до двох своїх невісток: Ідіть, верніться кожна до дому своєї матері. І нехай Господь зробить із вами милість, як ви зробили з померлими та зо мною.
\end{tcolorbox}
\begin{tcolorbox}
\textsubscript{9} Нехай Господь дасть вам, і ви знайдете відпочинок кожна в домі свого мужа! І вона поцілувала їх, а вони підняли свій голос та плакали.
\end{tcolorbox}
\begin{tcolorbox}
\textsubscript{10} І вони сказали до неї: Ні, з тобою ми вернемось до народу твого!
\end{tcolorbox}
\begin{tcolorbox}
\textsubscript{11} А Ноомі сказала: Вертайтеся, дочки мої, чого ви підете зо мною? Чи я маю ще в утробі своїй синів, а вони стануть вам за чоловіків?
\end{tcolorbox}
\begin{tcolorbox}
\textsubscript{12} Верніться, дочки мої, ідіть, бо я занадто стара, щоб бути для мужа. А коли б я й сказала: Маю надію, і коли б цієї ночі була з мужем, і також породила синів,
\end{tcolorbox}
\begin{tcolorbox}
\textsubscript{13} чи ж ви чекали б їх, аж поки повиростають? Чи ж ви зв'язалися б з ними, щоб не бути замужем? Ні, дочки мої, бо мені значно гірше, як вам, бо Господня рука знайшла мене.
\end{tcolorbox}
\begin{tcolorbox}
\textsubscript{14} І підняли вони голос свій, і заплакали ще. І поцілувала Орпа свою свекруху, а Рут пригорнулася до неї.
\end{tcolorbox}
\begin{tcolorbox}
\textsubscript{15} І сказала Ноомі: Ось зовиця твоя вернулася до народу свого та до богів своїх, вернися й ти за зовицею своєю!
\end{tcolorbox}
\begin{tcolorbox}
\textsubscript{16} А Рут відказала: Не силуй мене, щоб я покинула тебе, щоб я вернулася від тебе, бо куди підеш ти, туди піду й я, а де житимеш ти, там житиму й я. Народ твій буде мій народ, а Бог твій мій Бог.
\end{tcolorbox}
\begin{tcolorbox}
\textsubscript{17} Де помреш ти, там помру й я, і там буду похована. Нехай Господь зробить мені так, і так нехай додасть, і тільки смерть розлучить мене з тобою.
\end{tcolorbox}
\begin{tcolorbox}
\textsubscript{18} І побачила Ноомі, що вона настоює йти за нею, і перестала вговорювати її.
\end{tcolorbox}
\begin{tcolorbox}
\textsubscript{19} І пішли вони вдвох, аж прийшли до Віфлеєму. І сталося, коли вони входили до Віфлеєму, то зашуміло все місто про них, і говорили: Чи це Ноомі?
\end{tcolorbox}
\begin{tcolorbox}
\textsubscript{20} А вона сказала їм: Не кличте мене: Ноомі, кличте мене: Мара, бо велику гіркоту зробив мені Всемогутній.
\end{tcolorbox}
\begin{tcolorbox}
\textsubscript{21} Я заможною пішла була, та порожньою вернув мене Господь. Чого кличете мене: Ноомі, коли Господь свідчив проти мене, а Всемогутній послав мені горе?
\end{tcolorbox}
\begin{tcolorbox}
\textsubscript{22} І вернулася Ноомі та з нею моавітянка Рут, невістка її, що верталася з моавських піль. І прийшли вони до Віфлеєму на початку жнив ячменю.
\end{tcolorbox}
\subsection{CHAPTER 2}
\begin{tcolorbox}
\textsubscript{1} А Ноомі мала родича свого чоловіка, мужа багатого, з Елімелехового роду, а ім'я йому Боаз.
\end{tcolorbox}
\begin{tcolorbox}
\textsubscript{2} І сказала моавітянка Рут до Ноомі: Піду но я на поле, і назбираю колосся за тим, у кого в очах знайду милість. А та їй сказала: Іди, моя дочко!
\end{tcolorbox}
\begin{tcolorbox}
\textsubscript{3} І пішла вона, і прийшла та й збирала за женцями. А припадок навів її на ділянку поля Боаза, що з Елімелехового роду.
\end{tcolorbox}
\begin{tcolorbox}
\textsubscript{4} Аж ось прийшов із Віфлеєму Боаз, та й сказав до женців: Господь з вами! А вони відказали йому: Нехай поблагословить тебе Господь!
\end{tcolorbox}
\begin{tcolorbox}
\textsubscript{5} І сказав Боаз до слуги свого, поставленого над женцями: Чия це дівчина?
\end{tcolorbox}
\begin{tcolorbox}
\textsubscript{6} І відповів той слуга, поставлений над женцями, і сказав: Дівчина моавітянка вона, що вернулася з Ноомі з моавських піль.
\end{tcolorbox}
\begin{tcolorbox}
\textsubscript{7} А вона сказала: Нехай я збиратиму, та назбираю між снопами за женцями! І прийшла вона, і стала від самого ранку й аж дотепер; а вдома вона була мало.
\end{tcolorbox}
\begin{tcolorbox}
\textsubscript{8} І сказав Боаз до Рут: Ото чуєш, дочко моя, не ходи збирати на іншому полі, і не йди звідси, і так пристань до моїх дівчат.
\end{tcolorbox}
\begin{tcolorbox}
\textsubscript{9} Доглядай цього поля, де будуть жати, і ти підеш за ними. Ось я наказав слугам не займати тебе. А як спрагнеш, то підеш до начинь, та й нап'єшся з того, що поначерпують слуги!
\end{tcolorbox}
\begin{tcolorbox}
\textsubscript{10} І впала вона на обличчя своє, та й вклонилася до землі, і сказала йому: Чому знайшла я милість в очах твоїх, що ти прихилився до мене, хоч я чужа?
\end{tcolorbox}
\begin{tcolorbox}
\textsubscript{11} І відповів Боаз і сказав їй: Докладно розповіджено мені все, що зробила ти з своєю свекрухою по смерті твого чоловіка, і ти кинула батька свого й матір свою та край свого народження, і пішла до народу, якого не знала вчора-позавчора.
\end{tcolorbox}
\begin{tcolorbox}
\textsubscript{12} Нехай Господь заплатить за чин твій, і нехай буде нагорода твоя повна від Господа, Бога Ізраїлевого, що ти прийшла сховатися під крильми Його!
\end{tcolorbox}
\begin{tcolorbox}
\textsubscript{13} А вона сказала: Нехай я знайду милість в очах твоїх, пане мій, бо ти потішив мене, і говорив до серця своєї невільниці. А я не є навіть як одна з твоїх невільниць!
\end{tcolorbox}
\begin{tcolorbox}
\textsubscript{14} І сказав їй Боаз у час їди: Підійди сюди, та з'їж хліба й замочи у квасі шматок свій. І сіла вона збоку женців, а він подав їй праженого зерна. І їла вона й наситилася, і ще й позоставила.
\end{tcolorbox}
\begin{tcolorbox}
\textsubscript{15} І встала вона збирати. А Боаз наказав слугам своїм, говорячи: І між снопами нехай збирає, і не кривдьте її.
\end{tcolorbox}
\begin{tcolorbox}
\textsubscript{16} І також конче киньте їй зо снопів, і позоставте, і буде вона збирати, а ви не лайте її.
\end{tcolorbox}
\begin{tcolorbox}
\textsubscript{17} І збирала вона на полі аж до вечора, і вимолотила те, що назбирала, і було близько ефи ячменю.
\end{tcolorbox}
\begin{tcolorbox}
\textsubscript{18} І понесла вона, і ввійшла до міста, і її свекруха побачила, що вона назбирала. А вона вийняла, і дала їй, що позоставила по своїй їжі.
\end{tcolorbox}
\begin{tcolorbox}
\textsubscript{19} І сказала їй свекруха її: Де ти збирала сьогодні, і де ти робила? Нехай буде благословенний, хто прийняв тебе! І вона розповіла своїй свекрусі, у кого працювала, та й сказала: Ім'я того чоловіка, що я сьогодні робила в нього, Боаз.
\end{tcolorbox}
\begin{tcolorbox}
\textsubscript{20} І сказала Ноомі до невістки своєї: Благословенний він у Господа, що не позбавив милости своєї ані живих, ані померлих. І сказала їй Ноомі: Близький нам той чоловік, він із наших родичів.
\end{tcolorbox}
\begin{tcolorbox}
\textsubscript{21} І сказала моавітянка Рут: Він також сказав мені: Пристань до моїх слуг, аж поки не скінчать моїх жнив.
\end{tcolorbox}
\begin{tcolorbox}
\textsubscript{22} І сказала Ноомі до своєї невістки Рут: Добре, дочко моя, що ти вийдеш з його служницями, щоб не чіпали тебе на іншому полі.
\end{tcolorbox}
\begin{tcolorbox}
\textsubscript{23} І вона пристала до Боазових служниць, щоб збирати аж до закінчення жнив ячменю та жнив пшениці. І вона жила з своєю свекрухою.
\end{tcolorbox}
\subsection{CHAPTER 3}
\begin{tcolorbox}
\textsubscript{1} І сказала їй свекруха її Ноомі: Дочко моя, ось я пошукаю для тебе місця спочинку, що буде добре тобі.
\end{tcolorbox}
\begin{tcolorbox}
\textsubscript{2} А тепер ось Боаз, наш родич, що була ти з його служницями, ось він цієї ночі буде віяти ячмінь на току.
\end{tcolorbox}
\begin{tcolorbox}
\textsubscript{3} А ти вмийся, і намастися, і надягни на себе кращу одежу свою, та й зійди на тік. Але не показуйся на очі тому чоловікові, аж поки він не скінчить їсти та пити.
\end{tcolorbox}
\begin{tcolorbox}
\textsubscript{4} І станеться, коли він ляже, то ти зауваж те місце, де він лежить. І ти прийдеш, і відкриєш приніжжя його та й ляжеш, а він скаже тобі, що маєш робити.
\end{tcolorbox}
\begin{tcolorbox}
\textsubscript{5} А та відказала до неї: Усе, що ти кажеш мені, я зроблю.
\end{tcolorbox}
\begin{tcolorbox}
\textsubscript{6} І зійшла вона на тік, і зробила все, як наказала їй свекруха її.
\end{tcolorbox}
\begin{tcolorbox}
\textsubscript{7} А Боаз з'їв та випив, та й стало весело йому на серці, і прийшов він покластися біля копиці. А вона тихо прийшла, і відкрила його приніжжя та й лягла.
\end{tcolorbox}
\begin{tcolorbox}
\textsubscript{8} І сталося опівночі, і затремтів той чоловік, та й звівся, аж ось жінка лежить у приніжжі його!
\end{tcolorbox}
\begin{tcolorbox}
\textsubscript{9} І він сказав: Хто ти? А вона відказала: Я невільниця твоя Рут. Простягни ж крило над своєю невільницею, бо ти мій родич.
\end{tcolorbox}
\begin{tcolorbox}
\textsubscript{10} А він сказав: Благословенна ти в Господа, дочко моя! Твоя остання ласка до мене ліпша від першої, що не пішла ти за юнаками, чи вони бідні, чи вони багаті.
\end{tcolorbox}
\begin{tcolorbox}
\textsubscript{11} А тепер, дочко моя, не бійся! Усе, що скажеш, я зроблю тобі, бо все місто народу мого знає, що ти жінка чеснотна!
\end{tcolorbox}
\begin{tcolorbox}
\textsubscript{12} А тепер справді, що я родич, та є родич ще, ближчий від мене.
\end{tcolorbox}
\begin{tcolorbox}
\textsubscript{13} Ночуй цю ніч, а ранком, якщо він викупить тебе добре, нехай викупить. А якщо він не схоче викупити тебе, то викуплю тебе я, як живий Господь! Лежи тут аж до ранку.
\end{tcolorbox}
\begin{tcolorbox}
\textsubscript{14} І лежала вона у приніжжі його аж до ранку, і встала, перше ніж можна розпізнати один одного. А він сказав: Нехай не пізнають, що жінка приходила на тік.
\end{tcolorbox}
\begin{tcolorbox}
\textsubscript{15} І він сказав: Дай хустку, що на тобі, і подерж її. І держала вона її, а він відміряв шість мір ячменю, і поклав на неї, та й пішов до міста.
\end{tcolorbox}
\begin{tcolorbox}
\textsubscript{16} А вона прийшла до своєї свекрухи. А та сказала: Як справа, дочко моя? А вона розповіла їй усе, що зробив їй той чоловік.
\end{tcolorbox}
\begin{tcolorbox}
\textsubscript{17} І сказала: Ці шість мір ячменю він дав мені, бо сказав: Не приходь порожньо до своєї свекрухи.
\end{tcolorbox}
\begin{tcolorbox}
\textsubscript{18} А та сказала: Почекай, моя дочко, аж поки довідаєшся, як випаде справа, бо той чоловік не заспокоїться, доки не викінчить цієї справи сьогодні.
\end{tcolorbox}
\subsection{CHAPTER 4}
\begin{tcolorbox}
\textsubscript{1} А Боаз прийшов до брами, та й сів там. Аж ось проходить родич, про якого говорив був Боаз. І він сказав йому: Зайди сюди, послухай, і сядь отут! І той зайшов і сів.
\end{tcolorbox}
\begin{tcolorbox}
\textsubscript{2} А Боаз узяв десять мужа зо старших того міста та й сказав: Сідайте тут! І вони посідали.
\end{tcolorbox}
\begin{tcolorbox}
\textsubscript{3} І сказав він до родича: Ділянку поля, що нашого брата Елімелеха, продала Ноомі, яка вернулася з моавського поля.
\end{tcolorbox}
\begin{tcolorbox}
\textsubscript{4} А я постановив: Подам тобі до ушей твоїх, говорячи: Купи при тих, що сидять тут, та при старших мого народу. Якщо викупиш викупи, а якщо не викупиш скажи мені, і нехай я знаю, бо окрім тебе нема кому викупити, а я за тобою. А той сказав: Я викуплю.
\end{tcolorbox}
\begin{tcolorbox}
\textsubscript{5} І сказав Боаз: Того дня, коли набудеш поле з руки Ноомі, то набудеш також моавитянку Рут, жінку померлого, щоб поставити ім'я померлому на наділі його.
\end{tcolorbox}
\begin{tcolorbox}
\textsubscript{6} А родич сказав: Не можу я викупити собі, щоб не понищити свого наділу. Викупи собі мого викупа, бо я не можу викупити.
\end{tcolorbox}
\begin{tcolorbox}
\textsubscript{7} А оце було колись серед Ізраїля на викуп, і на заміну, і на ствердження кожної справи: чоловік здіймав сандалю свою, і давав своєму ближньому, і це було свідоцтвом серед Ізраїля.
\end{tcolorbox}
\begin{tcolorbox}
\textsubscript{8} І сказав родич до Боаза: Купи собі! І зняв свою сандалю.
\end{tcolorbox}
\begin{tcolorbox}
\textsubscript{9} І сказав Боаз до старших та до всього народу: Ви свідки сьогодні, що я набув усе, що Елімелехове, і все, що Кілйонове та Махлонове з руки Ноомі.
\end{tcolorbox}
\begin{tcolorbox}
\textsubscript{10} А також моавітянку Рут, Махлонову жінку, набув я собі за жінку, щоб поставити ім'я померлому на спадкові його, і не буде знищене ім'я померлого між братами його та з брами його місця. Ви сьогодні свідки на це!
\end{tcolorbox}
\begin{tcolorbox}
\textsubscript{11} І сказав увесь народ, що були в брамі, та старші, свідки: Нехай дасть Господь цю жінку, що входить до дому твого, як Рахиль та як Лію, що вони обидві збудували Ізраїлів дім. І розбагатій в Ефраті, і здобуть собі славне ім'я в Віфлеємі.
\end{tcolorbox}
\begin{tcolorbox}
\textsubscript{12} А з насіння, що Господь дасть тобі від цієї молодої жінки, нехай стане дім твій, як дім Переца, що Тамар породила була Юді.
\end{tcolorbox}
\begin{tcolorbox}
\textsubscript{13} І взяв Боаз Рут, і вона стала йому за жінку. І він увійшов до неї, а Господь дав їй вагітність, і вона породила сина.
\end{tcolorbox}
\begin{tcolorbox}
\textsubscript{14} І сказали жінки до Ноомі: Благословенний Господь, що не позбавив тебе сьогодні родича! І буде славним ім'я його серед Ізраїля.
\end{tcolorbox}
\begin{tcolorbox}
\textsubscript{15} І він буде тобі потішителем душі та на виживлення твоєї сивини, бо породила його твоя невістка, що любить тебе, що ліпша тобі за сімох синів.
\end{tcolorbox}
\begin{tcolorbox}
\textsubscript{16} І взяла Ноомі ту дитину, і поклала її на коліна свої, і була їй за няньку.
\end{tcolorbox}
\begin{tcolorbox}
\textsubscript{17} А сусідки назвали ім'я йому, говорячи: Народився син для Ноомі! І назвали ім'я йому: Овед. А він батько Єссея, Давидового батька.
\end{tcolorbox}
\begin{tcolorbox}
\textsubscript{18} А оце Перецові нащадки: Перец породив Гецрона,
\end{tcolorbox}
\begin{tcolorbox}
\textsubscript{19} а Гецрон породив Рама, а Рам породив Аммінадава,
\end{tcolorbox}
\begin{tcolorbox}
\textsubscript{20} а Аммінадав породив Нахшона, а Нахшон породив Салмона;
\end{tcolorbox}
\begin{tcolorbox}
\textsubscript{21} а Салмон породив Боаза, а Боаз породив Оведа;
\end{tcolorbox}
\begin{tcolorbox}
\textsubscript{22} а Овед породив Єссея, а Єссей породив Давида.
\end{tcolorbox}
