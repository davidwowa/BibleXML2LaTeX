\section{BOOK 54}
\subsection{CHAPTER 1}
\begin{tcolorbox}
\textsubscript{1} Павло, з волі Божої апостол Христа Ісуса, за обітницею життя, що в Христі Ісусі,
\end{tcolorbox}
\begin{tcolorbox}
\textsubscript{2} до Тимофія, сина улюбленого: благодать, милість, мир від Бога Отця й Христа Ісуса, Господа нашого!
\end{tcolorbox}
\begin{tcolorbox}
\textsubscript{3} Дякую Богові, Якому служу від предків чистим сумлінням, що тебе пам'ятаю я завжди в молитвах своїх день і ніч.
\end{tcolorbox}
\begin{tcolorbox}
\textsubscript{4} Я бажаю побачити тебе, пам'ятаючи сльози твої, щоб наповнитись радістю.
\end{tcolorbox}
\begin{tcolorbox}
\textsubscript{5} Я приводжу на пам'ять собі твою нелицемірну віру, що перше була оселилася в бабі твоїй Лоіді та в твоїй матері Евнікії; певен же я, що й у тобі вона оселилась.
\end{tcolorbox}
\begin{tcolorbox}
\textsubscript{6} З цієї причини я нагадую тобі, що ти розгрівав Божого дара, який у тобі через покладання рук моїх.
\end{tcolorbox}
\begin{tcolorbox}
\textsubscript{7} Бо не дав нам Бог духа страху, але сили, і любови, і здорового розуму.
\end{tcolorbox}
\begin{tcolorbox}
\textsubscript{8} Тож, не соромся засвідчення Господа нашого, ні мене, Його в'язня, але страждай з Євангелією за силою Бога,
\end{tcolorbox}
\begin{tcolorbox}
\textsubscript{9} що нас спас і покликав святим покликом, не за наші діла, але з волі Своєї та з благодаті, що нам дана в Христі Ісусі попереду вічних часів.
\end{tcolorbox}
\begin{tcolorbox}
\textsubscript{10} А тепер об'явилась через з'явлення Спасителя нашого Христа Ісуса, що й смерть зруйнував, і вивів на світло життя та нетління Євангелією,
\end{tcolorbox}
\begin{tcolorbox}
\textsubscript{11} що для неї я був настановлений за проповідника, апостола й учителя.
\end{tcolorbox}
\begin{tcolorbox}
\textsubscript{12} З цієї причини й терплю я оце, але не соромлюсь, бо знаю, в Кого я ввірував та впевнився, що має Він силу заховати на той день заставу мою.
\end{tcolorbox}
\begin{tcolorbox}
\textsubscript{13} Май же за взір здорових слів ті, які від мене почув ти у вірі й любові, що в Христі Ісусі вона.
\end{tcolorbox}
\begin{tcolorbox}
\textsubscript{14} Добро припоручене стережи Святим Духом, що в нас пробуває.
\end{tcolorbox}
\begin{tcolorbox}
\textsubscript{15} Ти знаєш оце, що відвернулись від мене всі, хто в Азії, а між ними Фігел та Гермоген.
\end{tcolorbox}
\begin{tcolorbox}
\textsubscript{16} Хай Господь подасть милосердя Онисифоровому дому, бо він часто мене підкріпляв і кайданів моїх не соромився.
\end{tcolorbox}
\begin{tcolorbox}
\textsubscript{17} А коли він до Риму прибув, шукав мене пильно й знайшов,
\end{tcolorbox}
\begin{tcolorbox}
\textsubscript{18} хай Господь йому дасть знайти милість від Господа в день той, скільки ж він послужив був в Ефесі мені, ти відаєш краще!
\end{tcolorbox}
\subsection{CHAPTER 2}
\begin{tcolorbox}
\textsubscript{1} Отож, сину мій, зміцняйся в благодаті, що в Христі Ісусі вона!
\end{tcolorbox}
\begin{tcolorbox}
\textsubscript{2} А що чув ти від мене при багатьох свідках, те передай вірним людям, що будуть спроможні й інших навчити.
\end{tcolorbox}
\begin{tcolorbox}
\textsubscript{3} А ти терпи лихо, як добрий вояк Христа Ісуса!
\end{tcolorbox}
\begin{tcolorbox}
\textsubscript{4} Бо жаден вояк не в'яжеться в справи життя, аби догодити тому, хто військо збирає.
\end{tcolorbox}
\begin{tcolorbox}
\textsubscript{5} А як хто йде на змаги, то вінка не одержує, якщо незаконно змагається.
\end{tcolorbox}
\begin{tcolorbox}
\textsubscript{6} Трудящому хліборобові належиться першому покуштувати з плоду.
\end{tcolorbox}
\begin{tcolorbox}
\textsubscript{7} Розумій, що я говорю. А Господь нехай дасть тобі розум у всьому.
\end{tcolorbox}
\begin{tcolorbox}
\textsubscript{8} Пам'ятай про Ісуса Христа з насіння Давидового, що воскрес із мертвих, за моєю Євангелією,
\end{tcolorbox}
\begin{tcolorbox}
\textsubscript{9} за яку я терплю муки аж до ув'язнення, як той злочинець. Але Слова Божого не ув'язнити!
\end{tcolorbox}
\begin{tcolorbox}
\textsubscript{10} Через це переношу я все ради вибраних, щоб і вони доступили спасіння, що в Христі Ісусі, зо славою вічною.
\end{tcolorbox}
\begin{tcolorbox}
\textsubscript{11} Вірне слово: коли разом із Ним ми померли, то й житимемо разом із Ним!
\end{tcolorbox}
\begin{tcolorbox}
\textsubscript{12} А коли терпимо, то будемо разом також царювати. А коли відцураємось, то й Він відцурається нас!
\end{tcolorbox}
\begin{tcolorbox}
\textsubscript{13} А коли ми невірні, зостається Він вірним, бо не може зректися Самого Себе!
\end{tcolorbox}
\begin{tcolorbox}
\textsubscript{14} Нагадуй про це й заклинай перед Богом, щоб не сперечались словами, бо нінащо воно, хіба слухачам на руїну.
\end{tcolorbox}
\begin{tcolorbox}
\textsubscript{15} Силкуйся поставити себе перед Богом гідним, працівником бездоганним, що вірно навчає науки правди.
\end{tcolorbox}
\begin{tcolorbox}
\textsubscript{16} Стережися ж базікань марних, бо вони ще більше провадять до безбожности,
\end{tcolorbox}
\begin{tcolorbox}
\textsubscript{17} а їхнє слово, як рак, буде ширитися. Від таких Гіменей і Філіт,
\end{tcolorbox}
\begin{tcolorbox}
\textsubscript{18} що вони погрішилися в правді, казавши, що воскресіння було вже, і віру деяких руйнують.
\end{tcolorbox}
\begin{tcolorbox}
\textsubscript{19} Та однако стоїть міцна Божа основа та має печатку оцю: Господь знає тих, хто Його, та: Нехай від неправди відступиться всякий, хто Господнє Ім'я називає!
\end{tcolorbox}
\begin{tcolorbox}
\textsubscript{20} А в великому домі знаходиться посуд не тільки золотий та срібний, але й дерев'яний та глиняний, і одні посудини на честь, а другі на нечесть.
\end{tcolorbox}
\begin{tcolorbox}
\textsubscript{21} Отож, хто від цього очистить себе, буде посуд на честь, освячений, потрібний Володареві, приготований на всяке добре діло.
\end{tcolorbox}
\begin{tcolorbox}
\textsubscript{22} Стережися молодечих пожадливостей, тримайся правди, віри, любови, миру з тими, хто Господа кличе від чистого серця.
\end{tcolorbox}
\begin{tcolorbox}
\textsubscript{23} А від нерозумних та від невчених змагань ухиляйся, знавши, що вони родять сварки.
\end{tcolorbox}
\begin{tcolorbox}
\textsubscript{24} А раб Господній не повинен сваритись, але бути привітним до всіх, навчальним, до лиха терплячим,
\end{tcolorbox}
\begin{tcolorbox}
\textsubscript{25} що навчав би противників із лагідністю, чи Бог їм не дасть покаяння, щоб правду пізнати,
\end{tcolorbox}
\begin{tcolorbox}
\textsubscript{26} щоб визволитися від сітки диявола, що він уловив їх для роблення волі своєї.
\end{tcolorbox}
\subsection{CHAPTER 3}
\begin{tcolorbox}
\textsubscript{1} Знай же ти це, що останніми днями настануть тяжкі часи.
\end{tcolorbox}
\begin{tcolorbox}
\textsubscript{2} Будуть бо люди тоді самолюбні, грошолюбні, зарозумілі, горді, богозневажники, батькам неслухняні, невдячні, непобожні,
\end{tcolorbox}
\begin{tcolorbox}
\textsubscript{3} нелюбовні, запеклі, осудливі, нестримливі, жорстокі, ненависники добра,
\end{tcolorbox}
\begin{tcolorbox}
\textsubscript{4} зрадники, нахабні, бундючні, що більше люблять розкоші, аніж люблять Бога,
\end{tcolorbox}
\begin{tcolorbox}
\textsubscript{5} вони мають вигляд благочестя, але сили його відреклися. Відвертайсь від таких!
\end{tcolorbox}
\begin{tcolorbox}
\textsubscript{6} До них бо належать і ті, хто пролазить до хат та зводить жінок, гріхами обтяжених, ведених усякими пожадливостями,
\end{tcolorbox}
\begin{tcolorbox}
\textsubscript{7} що вони завжди вчаться, та ніколи не можуть прийти до пізнання правди.
\end{tcolorbox}
\begin{tcolorbox}
\textsubscript{8} Як Янній та Ямврій протиставилися були Мойсеєві, так і ці протиставляться правді, люди зіпсутого розуму, неуки щодо віри.
\end{tcolorbox}
\begin{tcolorbox}
\textsubscript{9} Та більше не матимуть успіху, бо всім виявиться їхній безум, як і з тими було.
\end{tcolorbox}
\begin{tcolorbox}
\textsubscript{10} Ти ж пішов услід за мною наукою, поступованням, заміром, вірою, витривалістю, любов'ю, терпеливістю,
\end{tcolorbox}
\begin{tcolorbox}
\textsubscript{11} переслідуваннями та стражданнями, що спіткали були мене в Антіохії, в Іконії, у Лістрах, такі переслідування переніс я, та Господь від усіх мене визволив.
\end{tcolorbox}
\begin{tcolorbox}
\textsubscript{12} Та й усі, хто хоче жити побожно у Христі Ісусі, будуть переслідувані.
\end{tcolorbox}
\begin{tcolorbox}
\textsubscript{13} А люди лихі та дурисвіти матимуть успіх у злому, зводячи й зведені бувши.
\end{tcolorbox}
\begin{tcolorbox}
\textsubscript{14} А ти в тім пробувай, чого тебе навчено, і що тобі звірено, відаючи тих, від кого навчився був ти.
\end{tcolorbox}
\begin{tcolorbox}
\textsubscript{15} І ти знаєш з дитинства Писання святе, що може зробити тебе мудрим на спасіння вірою в Христа Ісуса.
\end{tcolorbox}
\begin{tcolorbox}
\textsubscript{16} Усе Писання Богом надхнене, і корисне до навчання, до докору, до направи, до виховання в праведності,
\end{tcolorbox}
\begin{tcolorbox}
\textsubscript{17} щоб Божа людина була досконала, до всякого доброго діла готова.
\end{tcolorbox}
\subsection{CHAPTER 4}
\begin{tcolorbox}
\textsubscript{1} Отже, я свідкую тобі перед Богом і Христом Ісусом, що Він має судити живих і мертвих за Свого приходу та за Свого Царства.
\end{tcolorbox}
\begin{tcolorbox}
\textsubscript{2} Проповідуй Слово, допоминайся вчасно-невчасно, докоряй, забороняй, переконуй з терпеливістю та з наукою.
\end{tcolorbox}
\begin{tcolorbox}
\textsubscript{3} Настане бо час, коли здорової науки не будуть триматись, але за своїми пожадливостями виберуть собі вчителів, щоб вони їхні вуха влещували.
\end{tcolorbox}
\begin{tcolorbox}
\textsubscript{4} Вони слух свій від правди відвернуть та до байок нахиляться.
\end{tcolorbox}
\begin{tcolorbox}
\textsubscript{5} Але ти будь пильний у всьому, терпи лихо, виконуй працю благовісника, сповняй свою службу.
\end{tcolorbox}
\begin{tcolorbox}
\textsubscript{6} Бо я вже за жертву стаю, і час відходу мого вже настав.
\end{tcolorbox}
\begin{tcolorbox}
\textsubscript{7} Я змагався добрим змагом, свій біг закінчив, віру зберіг.
\end{tcolorbox}
\begin{tcolorbox}
\textsubscript{8} Наостанку мені призначається вінок праведности, якого мені того дня дасть Господь, Суддя праведний; і не тільки мені, але й усім, хто прихід Його полюбив.
\end{tcolorbox}
\begin{tcolorbox}
\textsubscript{9} Подбай незабаром прибути до мене.
\end{tcolorbox}
\begin{tcolorbox}
\textsubscript{10} Бо Димас мене кинув, цей вік полюбивши, і пішов до Солуня, Крискент до Галатії, Тит до Далматії.
\end{tcolorbox}
\begin{tcolorbox}
\textsubscript{11} Зо мною сам тільки Лука. Візьми Марка, і приведи з собою, бо мені він потрібний для служби.
\end{tcolorbox}
\begin{tcolorbox}
\textsubscript{12} А Тихика послав я в Ефес.
\end{tcolorbox}
\begin{tcolorbox}
\textsubscript{13} Як будеш іти, то плаща принеси, що його я в Троаді зоставив у Карпа, і книжки, особливо пергаменові.
\end{tcolorbox}
\begin{tcolorbox}
\textsubscript{14} Котляр Олександер накоїв був лиха чимало мені... Нехай Господь йому віддасть за його вчинками!
\end{tcolorbox}
\begin{tcolorbox}
\textsubscript{15} Стережись його й ти, бо він міцно противився нашим словам!
\end{tcolorbox}
\begin{tcolorbox}
\textsubscript{16} При першій моїй обороні жаден не був при мені, але всі покинули мене... Хай Господь їм того не полічить!
\end{tcolorbox}
\begin{tcolorbox}
\textsubscript{17} Але Господь став при мені та й мене підкріпив, щоб проповідь виконалась через мене, та щоб усі погани почули її. І я визволився з пащі лев'ячої...
\end{tcolorbox}
\begin{tcolorbox}
\textsubscript{18} А від усякого вчинку лихого Господь мене визволить та збереже для Свого Небесного Царства. Йому слава на віки вічні, амінь!
\end{tcolorbox}
\begin{tcolorbox}
\textsubscript{19} Поздоров Прискиллу й Акилу та дім Онисифора.
\end{tcolorbox}
\begin{tcolorbox}
\textsubscript{20} Ераст позостався в Коринті, а Трохима лишив я слабого в Мілеті.
\end{tcolorbox}
\begin{tcolorbox}
\textsubscript{21} Попильнуй прийти до зими. Вітає тебе Еввул, і Пуд, і Лин, і Клавдія, і вся браття.
\end{tcolorbox}
\begin{tcolorbox}
\textsubscript{22} Господь з твоїм духом! Благодать з вами! Амінь.
\end{tcolorbox}
