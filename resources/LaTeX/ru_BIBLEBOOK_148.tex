\section{BOOK 147}
\subsection{CHAPTER 1}
\begin{tcolorbox}
\textsubscript{1} Павел и Тимофей, рабы Иисуса Христа, всем святым во Христе Иисусе, находящимся в Филиппах, с епископами и диаконами:
\end{tcolorbox}
\begin{tcolorbox}
\textsubscript{2} благодать вам и мир от Бога Отца нашего и Господа Иисуса Христа.
\end{tcolorbox}
\begin{tcolorbox}
\textsubscript{3} Благодарю Бога моего при всяком воспоминании о вас,
\end{tcolorbox}
\begin{tcolorbox}
\textsubscript{4} всегда во всякой молитве моей за всех вас принося с радостью молитву мою,
\end{tcolorbox}
\begin{tcolorbox}
\textsubscript{5} за ваше участие в благовествовании от первого дня даже доныне,
\end{tcolorbox}
\begin{tcolorbox}
\textsubscript{6} будучи уверен в том, что начавший в вас доброе дело будет совершать его даже до дня Иисуса Христа,
\end{tcolorbox}
\begin{tcolorbox}
\textsubscript{7} как и должно мне помышлять о всех вас, потому что я имею вас в сердце в узах моих, при защищении и утверждении благовествования, вас всех, как соучастников моих в благодати.
\end{tcolorbox}
\begin{tcolorbox}
\textsubscript{8} Бог--свидетель, что я люблю всех вас любовью Иисуса Христа;
\end{tcolorbox}
\begin{tcolorbox}
\textsubscript{9} и молюсь о том, чтобы любовь ваша еще более и более возрастала в познании и всяком чувстве,
\end{tcolorbox}
\begin{tcolorbox}
\textsubscript{10} чтобы, познавая лучшее, вы были чисты и непреткновенны в день Христов,
\end{tcolorbox}
\begin{tcolorbox}
\textsubscript{11} исполнены плодов праведности Иисусом Христом, в славу и похвалу Божию.
\end{tcolorbox}
\begin{tcolorbox}
\textsubscript{12} Желаю, братия, чтобы вы знали, что обстоятельства мои послужили к большему успеху благовествования,
\end{tcolorbox}
\begin{tcolorbox}
\textsubscript{13} так что узы мои о Христе сделались известными всей претории и всем прочим,
\end{tcolorbox}
\begin{tcolorbox}
\textsubscript{14} и большая часть из братьев в Господе, ободрившись узами моими, начали с большею смелостью, безбоязненно проповедывать слово Божие.
\end{tcolorbox}
\begin{tcolorbox}
\textsubscript{15} Некоторые, правда, по зависти и любопрению, а другие с добрым расположением проповедуют Христа.
\end{tcolorbox}
\begin{tcolorbox}
\textsubscript{16} Одни по любопрению проповедуют Христа не чисто, думая увеличить тяжесть уз моих;
\end{tcolorbox}
\begin{tcolorbox}
\textsubscript{17} а другие--из любви, зная, что я поставлен защищать благовествование.
\end{tcolorbox}
\begin{tcolorbox}
\textsubscript{18} Но что до того? Как бы ни проповедали Христа, притворно или искренно, я и тому радуюсь и буду радоваться,
\end{tcolorbox}
\begin{tcolorbox}
\textsubscript{19} ибо знаю, что это послужит мне во спасение по вашей молитве и содействием Духа Иисуса Христа,
\end{tcolorbox}
\begin{tcolorbox}
\textsubscript{20} при уверенности и надежде моей, что я ни в чем посрамлен не буду, но при всяком дерзновении, и ныне, как и всегда, возвеличится Христос в теле моем, жизнью ли то, или смертью.
\end{tcolorbox}
\begin{tcolorbox}
\textsubscript{21} Ибо для меня жизнь--Христос, и смерть--приобретение.
\end{tcolorbox}
\begin{tcolorbox}
\textsubscript{22} Если же жизнь во плоти [доставляет] плод моему делу, то не знаю, что избрать.
\end{tcolorbox}
\begin{tcolorbox}
\textsubscript{23} Влечет меня то и другое: имею желание разрешиться и быть со Христом, потому что это несравненно лучше;
\end{tcolorbox}
\begin{tcolorbox}
\textsubscript{24} а оставаться во плоти нужнее для вас.
\end{tcolorbox}
\begin{tcolorbox}
\textsubscript{25} И я верно знаю, что останусь и пребуду со всеми вами для вашего успеха и радости в вере,
\end{tcolorbox}
\begin{tcolorbox}
\textsubscript{26} дабы похвала ваша во Христе Иисусе умножилась через меня, при моем вторичном к вам пришествии.
\end{tcolorbox}
\begin{tcolorbox}
\textsubscript{27} Только живите достойно благовествования Христова, чтобы мне, приду ли я и увижу вас, или не приду, слышать о вас, что вы стоите в одном духе, подвизаясь единодушно за веру Евангельскую,
\end{tcolorbox}
\begin{tcolorbox}
\textsubscript{28} и не страшитесь ни в чем противников: это для них есть предзнаменование погибели, а для вас--спасения. И сие от Бога,
\end{tcolorbox}
\begin{tcolorbox}
\textsubscript{29} потому что вам дано ради Христа не только веровать в Него, но и страдать за Него
\end{tcolorbox}
\begin{tcolorbox}
\textsubscript{30} таким же подвигом, какой вы видели во мне и ныне слышите о мне.
\end{tcolorbox}
\subsection{CHAPTER 2}
\begin{tcolorbox}
\textsubscript{1} Итак, если [есть] какое утешение во Христе, если [есть] какая отрада любви, если [есть] какое общение духа, если [есть] какое милосердие и сострадательность,
\end{tcolorbox}
\begin{tcolorbox}
\textsubscript{2} то дополните мою радость: имейте одни мысли, имейте ту же любовь, будьте единодушны и единомысленны;
\end{tcolorbox}
\begin{tcolorbox}
\textsubscript{3} ничего [не делайте] по любопрению или по тщеславию, но по смиренномудрию почитайте один другого высшим себя.
\end{tcolorbox}
\begin{tcolorbox}
\textsubscript{4} Не о себе [только] каждый заботься, но каждый и о других.
\end{tcolorbox}
\begin{tcolorbox}
\textsubscript{5} Ибо в вас должны быть те же чувствования, какие и во Христе Иисусе:
\end{tcolorbox}
\begin{tcolorbox}
\textsubscript{6} Он, будучи образом Божиим, не почитал хищением быть равным Богу;
\end{tcolorbox}
\begin{tcolorbox}
\textsubscript{7} но уничижил Себя Самого, приняв образ раба, сделавшись подобным человекам и по виду став как человек;
\end{tcolorbox}
\begin{tcolorbox}
\textsubscript{8} смирил Себя, быв послушным даже до смерти, и смерти крестной.
\end{tcolorbox}
\begin{tcolorbox}
\textsubscript{9} Посему и Бог превознес Его и дал Ему имя выше всякого имени,
\end{tcolorbox}
\begin{tcolorbox}
\textsubscript{10} дабы пред именем Иисуса преклонилось всякое колено небесных, земных и преисподних,
\end{tcolorbox}
\begin{tcolorbox}
\textsubscript{11} и всякий язык исповедал, что Господь Иисус Христос в славу Бога Отца.
\end{tcolorbox}
\begin{tcolorbox}
\textsubscript{12} Итак, возлюбленные мои, как вы всегда были послушны, не только в присутствии моем, но гораздо более ныне во время отсутствия моего, со страхом и трепетом совершайте свое спасение,
\end{tcolorbox}
\begin{tcolorbox}
\textsubscript{13} потому что Бог производит в вас и хотение и действие по [Своему] благоволению.
\end{tcolorbox}
\begin{tcolorbox}
\textsubscript{14} Всё делайте без ропота и сомнения,
\end{tcolorbox}
\begin{tcolorbox}
\textsubscript{15} чтобы вам быть неукоризненными и чистыми, чадами Божиими непорочными среди строптивого и развращенного рода, в котором вы сияете, как светила в мире,
\end{tcolorbox}
\begin{tcolorbox}
\textsubscript{16} содержа слово жизни, к похвале моей в день Христов, что я не тщетно подвизался и не тщетно трудился.
\end{tcolorbox}
\begin{tcolorbox}
\textsubscript{17} Но если я и соделываюсь жертвою за жертву и служение веры вашей, то радуюсь и сорадуюсь всем вам.
\end{tcolorbox}
\begin{tcolorbox}
\textsubscript{18} О сем самом и вы радуйтесь и сорадуйтесь мне.
\end{tcolorbox}
\begin{tcolorbox}
\textsubscript{19} Надеюсь же в Господе Иисусе вскоре послать к вам Тимофея, дабы и я, узнав о ваших обстоятельствах, утешился духом.
\end{tcolorbox}
\begin{tcolorbox}
\textsubscript{20} Ибо я не имею никого равно усердного, кто бы столь искренно заботился о вас,
\end{tcolorbox}
\begin{tcolorbox}
\textsubscript{21} потому что все ищут своего, а не того, что [угодно] Иисусу Христу.
\end{tcolorbox}
\begin{tcolorbox}
\textsubscript{22} А его верность вам известна, потому что он, как сын отцу, служил мне в благовествовании.
\end{tcolorbox}
\begin{tcolorbox}
\textsubscript{23} Итак я надеюсь послать его тотчас же, как скоро узнаю, что будет со мною.
\end{tcolorbox}
\begin{tcolorbox}
\textsubscript{24} Я уверен в Господе, что и сам скоро приду к вам.
\end{tcolorbox}
\begin{tcolorbox}
\textsubscript{25} Впрочем я почел нужным послать к вам Епафродита, брата и сотрудника и сподвижника моего, а вашего посланника и служителя в нужде моей,
\end{tcolorbox}
\begin{tcolorbox}
\textsubscript{26} потому что он сильно желал видеть всех вас и тяжко скорбел о том, что до вас дошел слух о его болезни.
\end{tcolorbox}
\begin{tcolorbox}
\textsubscript{27} Ибо он был болен при смерти; но Бог помиловал его, и не его только, но и меня, чтобы не прибавилась мне печаль к печали.
\end{tcolorbox}
\begin{tcolorbox}
\textsubscript{28} Посему я скорее послал его, чтобы вы, увидев его снова, возрадовались, и я был менее печален.
\end{tcolorbox}
\begin{tcolorbox}
\textsubscript{29} Примите же его в Господе со всякою радостью, и таких имейте в уважении,
\end{tcolorbox}
\begin{tcolorbox}
\textsubscript{30} ибо он за дело Христово был близок к смерти, подвергая опасности жизнь, дабы восполнить недостаток ваших услуг мне.
\end{tcolorbox}
\subsection{CHAPTER 3}
\begin{tcolorbox}
\textsubscript{1} Впрочем, братия мои, радуйтесь о Господе. Писать вам о том же для меня не тягостно, а для вас назидательно.
\end{tcolorbox}
\begin{tcolorbox}
\textsubscript{2} Берегитесь псов, берегитесь злых делателей, берегитесь обрезания,
\end{tcolorbox}
\begin{tcolorbox}
\textsubscript{3} потому что обрезание--мы, служащие Богу духом и хвалящиеся Христом Иисусом, и не на плоть надеющиеся,
\end{tcolorbox}
\begin{tcolorbox}
\textsubscript{4} хотя я могу надеяться и на плоть. Если кто другой думает надеяться на плоть, то более я,
\end{tcolorbox}
\begin{tcolorbox}
\textsubscript{5} обрезанный в восьмой день, из рода Израилева, колена Вениаминова, Еврей от Евреев, по учению фарисей,
\end{tcolorbox}
\begin{tcolorbox}
\textsubscript{6} по ревности--гонитель Церкви Божией, по правде законной--непорочный.
\end{tcolorbox}
\begin{tcolorbox}
\textsubscript{7} Но что для меня было преимуществом, то ради Христа я почел тщетою.
\end{tcolorbox}
\begin{tcolorbox}
\textsubscript{8} Да и все почитаю тщетою ради превосходства познания Христа Иисуса, Господа моего: для Него я от всего отказался, и все почитаю за сор, чтобы приобрести Христа
\end{tcolorbox}
\begin{tcolorbox}
\textsubscript{9} и найтись в Нем не со своею праведностью, которая от закона, но с тою, которая через веру во Христа, с праведностью от Бога по вере;
\end{tcolorbox}
\begin{tcolorbox}
\textsubscript{10} чтобы познать Его, и силу воскресения Его, и участие в страданиях Его, сообразуясь смерти Его,
\end{tcolorbox}
\begin{tcolorbox}
\textsubscript{11} чтобы достигнуть воскресения мертвых.
\end{tcolorbox}
\begin{tcolorbox}
\textsubscript{12} [Говорю так] не потому, чтобы я уже достиг, или усовершился; но стремлюсь, не достигну ли я, как достиг меня Христос Иисус.
\end{tcolorbox}
\begin{tcolorbox}
\textsubscript{13} Братия, я не почитаю себя достигшим; а только, забывая заднее и простираясь вперед,
\end{tcolorbox}
\begin{tcolorbox}
\textsubscript{14} стремлюсь к цели, к почести вышнего звания Божия во Христе Иисусе.
\end{tcolorbox}
\begin{tcolorbox}
\textsubscript{15} Итак, кто из нас совершен, так должен мыслить; если же вы о чем иначе мыслите, то и это Бог вам откроет.
\end{tcolorbox}
\begin{tcolorbox}
\textsubscript{16} Впрочем, до чего мы достигли, так и должны мыслить и по тому правилу жить.
\end{tcolorbox}
\begin{tcolorbox}
\textsubscript{17} Подражайте, братия, мне и смотрите на тех, которые поступают по образу, какой имеете в нас.
\end{tcolorbox}
\begin{tcolorbox}
\textsubscript{18} Ибо многие, о которых я часто говорил вам, а теперь даже со слезами говорю, поступают как враги креста Христова.
\end{tcolorbox}
\begin{tcolorbox}
\textsubscript{19} Их конец--погибель, их бог--чрево, и слава их--в сраме, они мыслят о земном.
\end{tcolorbox}
\begin{tcolorbox}
\textsubscript{20} Наше же жительство--на небесах, откуда мы ожидаем и Спасителя, Господа нашего Иисуса Христа,
\end{tcolorbox}
\begin{tcolorbox}
\textsubscript{21} Который уничиженное тело наше преобразит так, что оно будет сообразно славному телу Его, силою, [которою] Он действует и покоряет Себе всё.
\end{tcolorbox}
\subsection{CHAPTER 4}
\begin{tcolorbox}
\textsubscript{1} Итак, братия мои возлюбленные и вожделенные, радость и венец мой, стойте так в Господе, возлюбленные.
\end{tcolorbox}
\begin{tcolorbox}
\textsubscript{2} Умоляю Еводию, умоляю Синтихию мыслить то же о Господе.
\end{tcolorbox}
\begin{tcolorbox}
\textsubscript{3} Ей, прошу и тебя, искренний сотрудник, помогай им, подвизавшимся в благовествовании вместе со мною и с Климентом и с прочими сотрудниками моими, которых имена--в книге жизни.
\end{tcolorbox}
\begin{tcolorbox}
\textsubscript{4} Радуйтесь всегда в Господе; и еще говорю: радуйтесь.
\end{tcolorbox}
\begin{tcolorbox}
\textsubscript{5} Кротость ваша да будет известна всем человекам. Господь близко.
\end{tcolorbox}
\begin{tcolorbox}
\textsubscript{6} Не заботьтесь ни о чем, но всегда в молитве и прошении с благодарением открывайте свои желания пред Богом,
\end{tcolorbox}
\begin{tcolorbox}
\textsubscript{7} и мир Божий, который превыше всякого ума, соблюдет сердца ваши и помышления ваши во Христе Иисусе.
\end{tcolorbox}
\begin{tcolorbox}
\textsubscript{8} Наконец, братия мои, что только истинно, что честно, что справедливо, что чисто, что любезно, что достославно, что только добродетель и похвала, о том помышляйте.
\end{tcolorbox}
\begin{tcolorbox}
\textsubscript{9} Чему вы научились, что приняли и слышали и видели во мне, то исполняйте, --и Бог мира будет с вами.
\end{tcolorbox}
\begin{tcolorbox}
\textsubscript{10} Я весьма возрадовался в Господе, что вы уже вновь начали заботиться о мне; вы и прежде заботились, но вам не благоприятствовали обстоятельства.
\end{tcolorbox}
\begin{tcolorbox}
\textsubscript{11} Говорю это не потому, что нуждаюсь, ибо я научился быть довольным тем, что у меня есть.
\end{tcolorbox}
\begin{tcolorbox}
\textsubscript{12} Умею жить и в скудости, умею жить и в изобилии; научился всему и во всем, насыщаться и терпеть голод, быть и в обилии и в недостатке.
\end{tcolorbox}
\begin{tcolorbox}
\textsubscript{13} Все могу в укрепляющем меня Иисусе Христе.
\end{tcolorbox}
\begin{tcolorbox}
\textsubscript{14} Впрочем вы хорошо поступили, приняв участие в моей скорби.
\end{tcolorbox}
\begin{tcolorbox}
\textsubscript{15} Вы знаете, Филиппийцы, что в начале благовествования, когда я вышел из Македонии, ни одна церковь не оказала мне участия подаянием и принятием, кроме вас одних;
\end{tcolorbox}
\begin{tcolorbox}
\textsubscript{16} вы и в Фессалонику и раз и два присылали мне на нужду.
\end{tcolorbox}
\begin{tcolorbox}
\textsubscript{17} [Говорю это] не потому, чтобы я искал даяния; но ищу плода, умножающегося в пользу вашу.
\end{tcolorbox}
\begin{tcolorbox}
\textsubscript{18} Я получил все, и избыточествую; я доволен, получив от Епафродита посланное вами, [как] благовонное курение, жертву приятную, благоугодную Богу.
\end{tcolorbox}
\begin{tcolorbox}
\textsubscript{19} Бог мой да восполнит всякую нужду вашу, по богатству Своему в славе, Христом Иисусом.
\end{tcolorbox}
\begin{tcolorbox}
\textsubscript{20} Богу же и Отцу нашему слава во веки веков! Аминь.
\end{tcolorbox}
\begin{tcolorbox}
\textsubscript{21} Приветствуйте всякого святого во Христе Иисусе. Приветствуют вас находящиеся со мною братия.
\end{tcolorbox}
\begin{tcolorbox}
\textsubscript{22} Приветствуют вас все святые, а наипаче из кесарева дома.
\end{tcolorbox}
\begin{tcolorbox}
\textsubscript{23} Благодать Господа нашего Иисуса Христа со всеми вами. Аминь.
\end{tcolorbox}
