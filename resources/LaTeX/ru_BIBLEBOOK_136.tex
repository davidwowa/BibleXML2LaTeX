\section{BOOK 135}
\subsection{CHAPTER 1}
\begin{tcolorbox}
\textsubscript{1} Павел, волею Божиею призванный Апостол Иисуса Христа, и Сосфен брат,
\end{tcolorbox}
\begin{tcolorbox}
\textsubscript{2} церкви Божией, находящейся в Коринфе, освященным во Христе Иисусе, призванным святым, со всеми призывающими имя Господа нашего Иисуса Христа, во всяком месте, у них и у нас:
\end{tcolorbox}
\begin{tcolorbox}
\textsubscript{3} благодать вам и мир от Бога Отца нашего и Господа Иисуса Христа.
\end{tcolorbox}
\begin{tcolorbox}
\textsubscript{4} Непрестанно благодарю Бога моего за вас, ради благодати Божией, дарованной вам во Христе Иисусе,
\end{tcolorbox}
\begin{tcolorbox}
\textsubscript{5} потому что в Нем вы обогатились всем, всяким словом и всяким познанием, --
\end{tcolorbox}
\begin{tcolorbox}
\textsubscript{6} ибо свидетельство Христово утвердилось в вас, --
\end{tcolorbox}
\begin{tcolorbox}
\textsubscript{7} так что вы не имеете недостатка ни в каком даровании, ожидая явления Господа нашего Иисуса Христа,
\end{tcolorbox}
\begin{tcolorbox}
\textsubscript{8} Который и утвердит вас до конца, [чтобы вам быть] неповинными в день Господа нашего Иисуса Христа.
\end{tcolorbox}
\begin{tcolorbox}
\textsubscript{9} Верен Бог, Которым вы призваны в общение Сына Его Иисуса Христа, Господа нашего.
\end{tcolorbox}
\begin{tcolorbox}
\textsubscript{10} Умоляю вас, братия, именем Господа нашего Иисуса Христа, чтобы все вы говорили одно, и не было между вами разделений, но чтобы вы соединены были в одном духе и в одних мыслях.
\end{tcolorbox}
\begin{tcolorbox}
\textsubscript{11} Ибо от [домашних] Хлоиных сделалось мне известным о вас, братия мои, что между вами есть споры.
\end{tcolorbox}
\begin{tcolorbox}
\textsubscript{12} Я разумею то, что у вас говорят: 'я Павлов'; 'я Аполлосов'; 'я Кифин'; 'а я Христов'.
\end{tcolorbox}
\begin{tcolorbox}
\textsubscript{13} Разве разделился Христос? разве Павел распялся за вас? или во имя Павла вы крестились?
\end{tcolorbox}
\begin{tcolorbox}
\textsubscript{14} Благодарю Бога, что я никого из вас не крестил, кроме Криспа и Гаия,
\end{tcolorbox}
\begin{tcolorbox}
\textsubscript{15} дабы не сказал кто, что я крестил в мое имя.
\end{tcolorbox}
\begin{tcolorbox}
\textsubscript{16} Крестил я также Стефанов дом; а крестил ли еще кого, не знаю.
\end{tcolorbox}
\begin{tcolorbox}
\textsubscript{17} Ибо Христос послал меня не крестить, а благовествовать, не в премудрости слова, чтобы не упразднить креста Христова.
\end{tcolorbox}
\begin{tcolorbox}
\textsubscript{18} Ибо слово о кресте для погибающих юродство есть, а для нас, спасаемых, --сила Божия.
\end{tcolorbox}
\begin{tcolorbox}
\textsubscript{19} Ибо написано: погублю мудрость мудрецов, и разум разумных отвергну.
\end{tcolorbox}
\begin{tcolorbox}
\textsubscript{20} Где мудрец? где книжник? где совопросник века сего? Не обратил ли Бог мудрость мира сего в безумие?
\end{tcolorbox}
\begin{tcolorbox}
\textsubscript{21} Ибо когда мир [своею] мудростью не познал Бога в премудрости Божией, то благоугодно было Богу юродством проповеди спасти верующих.
\end{tcolorbox}
\begin{tcolorbox}
\textsubscript{22} Ибо и Иудеи требуют чудес, и Еллины ищут мудрости;
\end{tcolorbox}
\begin{tcolorbox}
\textsubscript{23} а мы проповедуем Христа распятого, для Иудеев соблазн, а для Еллинов безумие,
\end{tcolorbox}
\begin{tcolorbox}
\textsubscript{24} для самих же призванных, Иудеев и Еллинов, Христа, Божию силу и Божию премудрость;
\end{tcolorbox}
\begin{tcolorbox}
\textsubscript{25} потому что немудрое Божие премудрее человеков, и немощное Божие сильнее человеков.
\end{tcolorbox}
\begin{tcolorbox}
\textsubscript{26} Посмотрите, братия, кто вы, призванные: не много [из вас] мудрых по плоти, не много сильных, не много благородных;
\end{tcolorbox}
\begin{tcolorbox}
\textsubscript{27} но Бог избрал немудрое мира, чтобы посрамить мудрых, и немощное мира избрал Бог, чтобы посрамить сильное;
\end{tcolorbox}
\begin{tcolorbox}
\textsubscript{28} и незнатное мира и уничиженное и ничего не значащее избрал Бог, чтобы упразднить значащее, --
\end{tcolorbox}
\begin{tcolorbox}
\textsubscript{29} для того, чтобы никакая плоть не хвалилась пред Богом.
\end{tcolorbox}
\begin{tcolorbox}
\textsubscript{30} От Него и вы во Христе Иисусе, Который сделался для нас премудростью от Бога, праведностью и освящением и искуплением,
\end{tcolorbox}
\begin{tcolorbox}
\textsubscript{31} чтобы [было], как написано: хвалящийся хвались Господом.
\end{tcolorbox}
\subsection{CHAPTER 2}
\begin{tcolorbox}
\textsubscript{1} И когда я приходил к вам, братия, приходил возвещать вам свидетельство Божие не в превосходстве слова или мудрости,
\end{tcolorbox}
\begin{tcolorbox}
\textsubscript{2} ибо я рассудил быть у вас незнающим ничего, кроме Иисуса Христа, и притом распятого,
\end{tcolorbox}
\begin{tcolorbox}
\textsubscript{3} и был я у вас в немощи и в страхе и в великом трепете.
\end{tcolorbox}
\begin{tcolorbox}
\textsubscript{4} И слово мое и проповедь моя не в убедительных словах человеческой мудрости, но в явлении духа и силы,
\end{tcolorbox}
\begin{tcolorbox}
\textsubscript{5} чтобы вера ваша [утверждалась] не на мудрости человеческой, но на силе Божией.
\end{tcolorbox}
\begin{tcolorbox}
\textsubscript{6} Мудрость же мы проповедуем между совершенными, но мудрость не века сего и не властей века сего преходящих,
\end{tcolorbox}
\begin{tcolorbox}
\textsubscript{7} но проповедуем премудрость Божию, тайную, сокровенную, которую предназначил Бог прежде веков к славе нашей,
\end{tcolorbox}
\begin{tcolorbox}
\textsubscript{8} которой никто из властей века сего не познал; ибо если бы познали, то не распяли бы Господа славы.
\end{tcolorbox}
\begin{tcolorbox}
\textsubscript{9} Но, как написано: не видел того глаз, не слышало ухо, и не приходило то на сердце человеку, что приготовил Бог любящим Его.
\end{tcolorbox}
\begin{tcolorbox}
\textsubscript{10} А нам Бог открыл [это] Духом Своим; ибо Дух все проницает, и глубины Божии.
\end{tcolorbox}
\begin{tcolorbox}
\textsubscript{11} Ибо кто из человеков знает, что в человеке, кроме духа человеческого, живущего в нем? Так и Божьего никто не знает, кроме Духа Божия.
\end{tcolorbox}
\begin{tcolorbox}
\textsubscript{12} Но мы приняли не духа мира сего, а Духа от Бога, дабы знать дарованное нам от Бога,
\end{tcolorbox}
\begin{tcolorbox}
\textsubscript{13} что и возвещаем не от человеческой мудрости изученными словами, но изученными от Духа Святаго, соображая духовное с духовным.
\end{tcolorbox}
\begin{tcolorbox}
\textsubscript{14} Душевный человек не принимает того, что от Духа Божия, потому что он почитает это безумием; и не может разуметь, потому что о сем [надобно] судить духовно.
\end{tcolorbox}
\begin{tcolorbox}
\textsubscript{15} Но духовный судит о всем, а о нем судить никто не может.
\end{tcolorbox}
\begin{tcolorbox}
\textsubscript{16} Ибо кто познал ум Господень, чтобы [мог] судить его? А мы имеем ум Христов.
\end{tcolorbox}
\subsection{CHAPTER 3}
\begin{tcolorbox}
\textsubscript{1} И я не мог говорить с вами, братия, как с духовными, но как с плотскими, как с младенцами во Христе.
\end{tcolorbox}
\begin{tcolorbox}
\textsubscript{2} Я питал вас молоком, а не [твердою] пищею, ибо вы были еще не в силах, да и теперь не в силах,
\end{tcolorbox}
\begin{tcolorbox}
\textsubscript{3} потому что вы еще плотские. Ибо если между вами зависть, споры и разногласия, то не плотские ли вы? и не по человеческому ли [обычаю] поступаете?
\end{tcolorbox}
\begin{tcolorbox}
\textsubscript{4} Ибо когда один говорит: 'я Павлов', а другой: 'я Аполлосов', то не плотские ли вы?
\end{tcolorbox}
\begin{tcolorbox}
\textsubscript{5} Кто Павел? кто Аполлос? Они только служители, через которых вы уверовали, и притом поскольку каждому дал Господь.
\end{tcolorbox}
\begin{tcolorbox}
\textsubscript{6} Я насадил, Аполлос поливал, но возрастил Бог;
\end{tcolorbox}
\begin{tcolorbox}
\textsubscript{7} посему и насаждающий и поливающий есть ничто, а [все] Бог возращающий.
\end{tcolorbox}
\begin{tcolorbox}
\textsubscript{8} Насаждающий же и поливающий суть одно; но каждый получит свою награду по своему труду.
\end{tcolorbox}
\begin{tcolorbox}
\textsubscript{9} Ибо мы соработники у Бога, [а] вы Божия нива, Божие строение.
\end{tcolorbox}
\begin{tcolorbox}
\textsubscript{10} Я, по данной мне от Бога благодати, как мудрый строитель, положил основание, а другой строит на [нем]; но каждый смотри, как строит.
\end{tcolorbox}
\begin{tcolorbox}
\textsubscript{11} Ибо никто не может положить другого основания, кроме положенного, которое есть Иисус Христос.
\end{tcolorbox}
\begin{tcolorbox}
\textsubscript{12} Строит ли кто на этом основании из золота, серебра, драгоценных камней, дерева, сена, соломы, --
\end{tcolorbox}
\begin{tcolorbox}
\textsubscript{13} каждого дело обнаружится; ибо день покажет, потому что в огне открывается, и огонь испытает дело каждого, каково оно есть.
\end{tcolorbox}
\begin{tcolorbox}
\textsubscript{14} У кого дело, которое он строил, устоит, тот получит награду.
\end{tcolorbox}
\begin{tcolorbox}
\textsubscript{15} А у кого дело сгорит, тот потерпит урон; впрочем сам спасется, но так, как бы из огня.
\end{tcolorbox}
\begin{tcolorbox}
\textsubscript{16} Разве не знаете, что вы храм Божий, и Дух Божий живет в вас?
\end{tcolorbox}
\begin{tcolorbox}
\textsubscript{17} Если кто разорит храм Божий, того покарает Бог: ибо храм Божий свят; а этот [храм] --вы.
\end{tcolorbox}
\begin{tcolorbox}
\textsubscript{18} Никто не обольщай самого себя. Если кто из вас думает быть мудрым в веке сем, тот будь безумным, чтобы быть мудрым.
\end{tcolorbox}
\begin{tcolorbox}
\textsubscript{19} Ибо мудрость мира сего есть безумие пред Богом, как написано: уловляет мудрых в лукавстве их.
\end{tcolorbox}
\begin{tcolorbox}
\textsubscript{20} И еще: Господь знает умствования мудрецов, что они суетны.
\end{tcolorbox}
\begin{tcolorbox}
\textsubscript{21} Итак никто не хвались человеками, ибо все ваше:
\end{tcolorbox}
\begin{tcolorbox}
\textsubscript{22} Павел ли, или Аполлос, или Кифа, или мир, или жизнь, или смерть, или настоящее, или будущее, --все ваше;
\end{tcolorbox}
\begin{tcolorbox}
\textsubscript{23} вы же--Христовы, а Христос--Божий.
\end{tcolorbox}
\subsection{CHAPTER 4}
\begin{tcolorbox}
\textsubscript{1} Итак каждый должен разуметь нас, как служителей Христовых и домостроителей таин Божиих.
\end{tcolorbox}
\begin{tcolorbox}
\textsubscript{2} От домостроителей же требуется, чтобы каждый оказался верным.
\end{tcolorbox}
\begin{tcolorbox}
\textsubscript{3} Для меня очень мало значит, как судите обо мне вы или [как] [судят] другие люди; я и сам не сужу о себе.
\end{tcolorbox}
\begin{tcolorbox}
\textsubscript{4} Ибо [хотя] я ничего не знаю за собою, но тем не оправдываюсь; судия же мне Господь.
\end{tcolorbox}
\begin{tcolorbox}
\textsubscript{5} Посему не судите никак прежде времени, пока не придет Господь, Который и осветит скрытое во мраке и обнаружит сердечные намерения, и тогда каждому будет похвала от Бога.
\end{tcolorbox}
\begin{tcolorbox}
\textsubscript{6} Это, братия, приложил я к себе и Аполлосу ради вас, чтобы вы научились от нас не мудрствовать сверх того, что написано, и не превозносились один перед другим.
\end{tcolorbox}
\begin{tcolorbox}
\textsubscript{7} Ибо кто отличает тебя? Что ты имеешь, чего бы не получил? А если получил, что хвалишься, как будто не получил?
\end{tcolorbox}
\begin{tcolorbox}
\textsubscript{8} Вы уже пресытились, вы уже обогатились, вы стали царствовать без нас. О, если бы вы [и в самом деле] царствовали, чтобы и нам с вами царствовать!
\end{tcolorbox}
\begin{tcolorbox}
\textsubscript{9} Ибо я думаю, что нам, последним посланникам, Бог судил быть как бы приговоренными к смерти, потому что мы сделались позорищем для мира, для Ангелов и человеков.
\end{tcolorbox}
\begin{tcolorbox}
\textsubscript{10} Мы безумны Христа ради, а вы мудры во Христе; мы немощны, а вы крепки; вы в славе, а мы в бесчестии.
\end{tcolorbox}
\begin{tcolorbox}
\textsubscript{11} Даже доныне терпим голод и жажду, и наготу и побои, и скитаемся,
\end{tcolorbox}
\begin{tcolorbox}
\textsubscript{12} и трудимся, работая своими руками. Злословят нас, мы благословляем; гонят нас, мы терпим;
\end{tcolorbox}
\begin{tcolorbox}
\textsubscript{13} хулят нас, мы молим; мы как сор для мира, [как] прах, всеми [попираемый] доныне.
\end{tcolorbox}
\begin{tcolorbox}
\textsubscript{14} Не к постыжению вашему пишу сие, но вразумляю вас, как возлюбленных детей моих.
\end{tcolorbox}
\begin{tcolorbox}
\textsubscript{15} Ибо, хотя у вас тысячи наставников во Христе, но не много отцов; я родил вас во Христе Иисусе благовествованием.
\end{tcolorbox}
\begin{tcolorbox}
\textsubscript{16} Посему умоляю вас: подражайте мне, как я Христу.
\end{tcolorbox}
\begin{tcolorbox}
\textsubscript{17} Для сего я послал к вам Тимофея, моего возлюбленного и верного в Господе сына, который напомнит вам о путях моих во Христе, как я учу везде во всякой церкви.
\end{tcolorbox}
\begin{tcolorbox}
\textsubscript{18} Как я не иду к вам, то некоторые [у вас] возгордились;
\end{tcolorbox}
\begin{tcolorbox}
\textsubscript{19} но я скоро приду к вам, если угодно будет Господу, и испытаю не слова возгордившихся, а силу,
\end{tcolorbox}
\begin{tcolorbox}
\textsubscript{20} ибо Царство Божие не в слове, а в силе.
\end{tcolorbox}
\begin{tcolorbox}
\textsubscript{21} Чего вы хотите? с жезлом придти к вам, или с любовью и духом кротости?
\end{tcolorbox}
\subsection{CHAPTER 5}
\begin{tcolorbox}
\textsubscript{1} Есть верный слух, что у вас [появилось] блудодеяние, и притом такое блудодеяние, какого не слышно даже у язычников, что некто [вместо] [жены] имеет жену отца своего.
\end{tcolorbox}
\begin{tcolorbox}
\textsubscript{2} И вы возгордились, вместо того, чтобы лучше плакать, дабы изъят был из среды вас сделавший такое дело.
\end{tcolorbox}
\begin{tcolorbox}
\textsubscript{3} А я, отсутствуя телом, но присутствуя [у вас] духом, уже решил, как бы находясь у вас: сделавшего такое дело,
\end{tcolorbox}
\begin{tcolorbox}
\textsubscript{4} в собрании вашем во имя Господа нашего Иисуса Христа, обще с моим духом, силою Господа нашего Иисуса Христа,
\end{tcolorbox}
\begin{tcolorbox}
\textsubscript{5} предать сатане во измождение плоти, чтобы дух был спасен в день Господа нашего Иисуса Христа.
\end{tcolorbox}
\begin{tcolorbox}
\textsubscript{6} Нечем вам хвалиться. Разве не знаете, что малая закваска квасит все тесто?
\end{tcolorbox}
\begin{tcolorbox}
\textsubscript{7} Итак очистите старую закваску, чтобы быть вам новым тестом, так как вы бесквасны, ибо Пасха наша, Христос, заклан за нас.
\end{tcolorbox}
\begin{tcolorbox}
\textsubscript{8} Посему станем праздновать не со старою закваскою, не с закваскою порока и лукавства, но с опресноками чистоты и истины.
\end{tcolorbox}
\begin{tcolorbox}
\textsubscript{9} Я писал вам в послании--не сообщаться с блудниками;
\end{tcolorbox}
\begin{tcolorbox}
\textsubscript{10} впрочем не вообще с блудниками мира сего, или лихоимцами, или хищниками, или идолослужителями, ибо иначе надлежало бы вам выйти из мира [сего].
\end{tcolorbox}
\begin{tcolorbox}
\textsubscript{11} Но я писал вам не сообщаться с тем, кто, называясь братом, остается блудником, или лихоимцем, или идолослужителем, или злоречивым, или пьяницею, или хищником; с таким даже и не есть вместе.
\end{tcolorbox}
\begin{tcolorbox}
\textsubscript{12} Ибо что мне судить и внешних? Не внутренних ли вы судите?
\end{tcolorbox}
\begin{tcolorbox}
\textsubscript{13} Внешних же судит Бог. Итак, извергните развращенного из среды вас.
\end{tcolorbox}
\subsection{CHAPTER 6}
\begin{tcolorbox}
\textsubscript{1} Как смеет кто у вас, имея дело с другим, судиться у нечестивых, а не у святых?
\end{tcolorbox}
\begin{tcolorbox}
\textsubscript{2} Разве не знаете, что святые будут судить мир? Если же вами будет судим мир, то неужели вы недостойны судить маловажные [дела]?
\end{tcolorbox}
\begin{tcolorbox}
\textsubscript{3} Разве не знаете, что мы будем судить ангелов, не тем ли более [дела] житейские?
\end{tcolorbox}
\begin{tcolorbox}
\textsubscript{4} А вы, когда имеете житейские тяжбы, поставляете [своими судьями] ничего не значащих в церкви.
\end{tcolorbox}
\begin{tcolorbox}
\textsubscript{5} К стыду вашему говорю: неужели нет между вами ни одного разумного, который мог бы рассудить между братьями своими?
\end{tcolorbox}
\begin{tcolorbox}
\textsubscript{6} Но брат с братом судится, и притом перед неверными.
\end{tcolorbox}
\begin{tcolorbox}
\textsubscript{7} И то уже весьма унизительно для вас, что вы имеете тяжбы между собою. Для чего бы вам лучше не оставаться обиженными? для чего бы вам лучше не терпеть лишения?
\end{tcolorbox}
\begin{tcolorbox}
\textsubscript{8} Но вы [сами] обижаете и отнимаете, и притом у братьев.
\end{tcolorbox}
\begin{tcolorbox}
\textsubscript{9} Или не знаете, что неправедные Царства Божия не наследуют? Не обманывайтесь: ни блудники, ни идолослужители, ни прелюбодеи, ни малакии, ни мужеложники,
\end{tcolorbox}
\begin{tcolorbox}
\textsubscript{10} ни воры, ни лихоимцы, ни пьяницы, ни злоречивые, ни хищники--Царства Божия не наследуют.
\end{tcolorbox}
\begin{tcolorbox}
\textsubscript{11} И такими были некоторые из вас; но омылись, но освятились, но оправдались именем Господа нашего Иисуса Христа и Духом Бога нашего.
\end{tcolorbox}
\begin{tcolorbox}
\textsubscript{12} Все мне позволительно, но не все полезно; все мне позволительно, но ничто не должно обладать мною.
\end{tcolorbox}
\begin{tcolorbox}
\textsubscript{13} Пища для чрева, и чрево для пищи; но Бог уничтожит и то и другое. Тело же не для блуда, но для Господа, и Господь для тела.
\end{tcolorbox}
\begin{tcolorbox}
\textsubscript{14} Бог воскресил Господа, воскресит и нас силою Своею.
\end{tcolorbox}
\begin{tcolorbox}
\textsubscript{15} Разве не знаете, что тела ваши суть члены Христовы? Итак отниму ли члены у Христа, чтобы сделать [их] членами блудницы? Да не будет!
\end{tcolorbox}
\begin{tcolorbox}
\textsubscript{16} Или не знаете, что совокупляющийся с блудницею становится одно тело [с нею]? ибо сказано: два будут одна плоть.
\end{tcolorbox}
\begin{tcolorbox}
\textsubscript{17} А соединяющийся с Господом есть один дух с Господом.
\end{tcolorbox}
\begin{tcolorbox}
\textsubscript{18} Бегайте блуда; всякий грех, какой делает человек, есть вне тела, а блудник грешит против собственного тела.
\end{tcolorbox}
\begin{tcolorbox}
\textsubscript{19} Не знаете ли, что тела ваши суть храм живущего в вас Святаго Духа, Которого имеете вы от Бога, и вы не свои?
\end{tcolorbox}
\begin{tcolorbox}
\textsubscript{20} Ибо вы куплены [дорогою] ценою. Посему прославляйте Бога и в телах ваших и в душах ваших, которые суть Божии.
\end{tcolorbox}
\subsection{CHAPTER 7}
\begin{tcolorbox}
\textsubscript{1} А о чем вы писали ко мне, то хорошо человеку не касаться женщины.
\end{tcolorbox}
\begin{tcolorbox}
\textsubscript{2} Но, [во избежание] блуда, каждый имей свою жену, и каждая имей своего мужа.
\end{tcolorbox}
\begin{tcolorbox}
\textsubscript{3} Муж оказывай жене должное благорасположение; подобно и жена мужу.
\end{tcolorbox}
\begin{tcolorbox}
\textsubscript{4} Жена не властна над своим телом, но муж; равно и муж не властен над своим телом, но жена.
\end{tcolorbox}
\begin{tcolorbox}
\textsubscript{5} Не уклоняйтесь друг от друга, разве по согласию, на время, для упражнения в посте и молитве, а [потом] опять будьте вместе, чтобы не искушал вас сатана невоздержанием вашим.
\end{tcolorbox}
\begin{tcolorbox}
\textsubscript{6} Впрочем это сказано мною как позволение, а не как повеление.
\end{tcolorbox}
\begin{tcolorbox}
\textsubscript{7} Ибо желаю, чтобы все люди были, как и я; но каждый имеет свое дарование от Бога, один так, другой иначе.
\end{tcolorbox}
\begin{tcolorbox}
\textsubscript{8} Безбрачным же и вдовам говорю: хорошо им оставаться, как я.
\end{tcolorbox}
\begin{tcolorbox}
\textsubscript{9} Но если не [могут] воздержаться, пусть вступают в брак; ибо лучше вступить в брак, нежели разжигаться.
\end{tcolorbox}
\begin{tcolorbox}
\textsubscript{10} А вступившим в брак не я повелеваю, а Господь: жене не разводиться с мужем, --
\end{tcolorbox}
\begin{tcolorbox}
\textsubscript{11} если же разведется, то должна оставаться безбрачною, или примириться с мужем своим, --и мужу не оставлять жены [своей].
\end{tcolorbox}
\begin{tcolorbox}
\textsubscript{12} Прочим же я говорю, а не Господь: если какой брат имеет жену неверующую, и она согласна жить с ним, то он не должен оставлять ее;
\end{tcolorbox}
\begin{tcolorbox}
\textsubscript{13} и жена, которая имеет мужа неверующего, и он согласен жить с нею, не должна оставлять его.
\end{tcolorbox}
\begin{tcolorbox}
\textsubscript{14} Ибо неверующий муж освящается женою верующею, и жена неверующая освящается мужем верующим. Иначе дети ваши были бы нечисты, а теперь святы.
\end{tcolorbox}
\begin{tcolorbox}
\textsubscript{15} Если же неверующий [хочет] развестись, пусть разводится; брат или сестра в таких [случаях] не связаны; к миру призвал нас Господь.
\end{tcolorbox}
\begin{tcolorbox}
\textsubscript{16} Почему ты знаешь, жена, не спасешь ли мужа? Или ты, муж, почему знаешь, не спасешь ли жены?
\end{tcolorbox}
\begin{tcolorbox}
\textsubscript{17} Только каждый поступай так, как Бог ему определил, и каждый, как Господь призвал. Так я повелеваю по всем церквам.
\end{tcolorbox}
\begin{tcolorbox}
\textsubscript{18} Призван ли кто обрезанным, не скрывайся; призван ли кто необрезанным, не обрезывайся.
\end{tcolorbox}
\begin{tcolorbox}
\textsubscript{19} Обрезание ничто и необрезание ничто, но [всё] в соблюдении заповедей Божиих.
\end{tcolorbox}
\begin{tcolorbox}
\textsubscript{20} Каждый оставайся в том звании, в котором призван.
\end{tcolorbox}
\begin{tcolorbox}
\textsubscript{21} Рабом ли ты призван, не смущайся; но если и можешь сделаться свободным, то лучшим воспользуйся.
\end{tcolorbox}
\begin{tcolorbox}
\textsubscript{22} Ибо раб, призванный в Господе, есть свободный Господа; равно и призванный свободным есть раб Христов.
\end{tcolorbox}
\begin{tcolorbox}
\textsubscript{23} Вы куплены [дорогою] ценою; не делайтесь рабами человеков.
\end{tcolorbox}
\begin{tcolorbox}
\textsubscript{24} В каком [звании] кто призван, братия, в том каждый и оставайся пред Богом.
\end{tcolorbox}
\begin{tcolorbox}
\textsubscript{25} Относительно девства я не имею повеления Господня, а даю совет, как получивший от Господа милость быть [Ему] верным.
\end{tcolorbox}
\begin{tcolorbox}
\textsubscript{26} По настоящей нужде за лучшее признаю, что хорошо человеку оставаться так.
\end{tcolorbox}
\begin{tcolorbox}
\textsubscript{27} Соединен ли ты с женой? не ищи развода. Остался ли без жены? не ищи жены.
\end{tcolorbox}
\begin{tcolorbox}
\textsubscript{28} Впрочем, если и женишься, не согрешишь; и если девица выйдет замуж, не согрешит. Но таковые будут иметь скорби по плоти; а мне вас жаль.
\end{tcolorbox}
\begin{tcolorbox}
\textsubscript{29} Я вам сказываю, братия: время уже коротко, так что имеющие жен должны быть, как не имеющие;
\end{tcolorbox}
\begin{tcolorbox}
\textsubscript{30} и плачущие, как не плачущие; и радующиеся, как не радующиеся; и покупающие, как не приобретающие;
\end{tcolorbox}
\begin{tcolorbox}
\textsubscript{31} и пользующиеся миром сим, как не пользующиеся; ибо проходит образ мира сего.
\end{tcolorbox}
\begin{tcolorbox}
\textsubscript{32} А я хочу, чтобы вы были без забот. Неженатый заботится о Господнем, как угодить Господу;
\end{tcolorbox}
\begin{tcolorbox}
\textsubscript{33} а женатый заботится о мирском, как угодить жене. Есть разность между замужнею и девицею:
\end{tcolorbox}
\begin{tcolorbox}
\textsubscript{34} незамужняя заботится о Господнем, как угодить Господу, чтобы быть святою и телом и духом; а замужняя заботится о мирском, как угодить мужу.
\end{tcolorbox}
\begin{tcolorbox}
\textsubscript{35} Говорю это для вашей же пользы, не с тем, чтобы наложить на вас узы, но чтобы вы благочинно и непрестанно [служили] Господу без развлечения.
\end{tcolorbox}
\begin{tcolorbox}
\textsubscript{36} Если же кто почитает неприличным для своей девицы то, чтобы она, будучи в зрелом возрасте, оставалась так, тот пусть делает, как хочет: не согрешит; пусть [таковые] выходят замуж.
\end{tcolorbox}
\begin{tcolorbox}
\textsubscript{37} Но кто непоколебимо тверд в сердце своем и, не будучи стесняем нуждою, но будучи властен в своей воле, решился в сердце своем соблюдать свою деву, тот хорошо поступает.
\end{tcolorbox}
\begin{tcolorbox}
\textsubscript{38} Посему выдающий замуж свою девицу поступает хорошо; а не выдающий поступает лучше.
\end{tcolorbox}
\begin{tcolorbox}
\textsubscript{39} Жена связана законом, доколе жив муж ее; если же муж ее умрет, свободна выйти, за кого хочет, только в Господе.
\end{tcolorbox}
\begin{tcolorbox}
\textsubscript{40} Но она блаженнее, если останется так, по моему совету; а думаю, и я имею Духа Божия.
\end{tcolorbox}
\subsection{CHAPTER 8}
\begin{tcolorbox}
\textsubscript{1} О идоложертвенных [яствах] мы знаем, потому что мы все имеем знание; но знание надмевает, а любовь назидает.
\end{tcolorbox}
\begin{tcolorbox}
\textsubscript{2} Кто думает, что он знает что-нибудь, тот ничего еще не знает так, как должно знать.
\end{tcolorbox}
\begin{tcolorbox}
\textsubscript{3} Но кто любит Бога, тому дано знание от Него.
\end{tcolorbox}
\begin{tcolorbox}
\textsubscript{4} Итак об употреблении в пищу идоложертвенного мы знаем, что идол в мире ничто, и что нет иного Бога, кроме Единого.
\end{tcolorbox}
\begin{tcolorbox}
\textsubscript{5} Ибо хотя и есть так называемые боги, или на небе, или на земле, так как есть много богов и господ много, --
\end{tcolorbox}
\begin{tcolorbox}
\textsubscript{6} но у нас один Бог Отец, из Которого все, и мы для Него, и один Господь Иисус Христос, Которым все, и мы Им.
\end{tcolorbox}
\begin{tcolorbox}
\textsubscript{7} Но не у всех [такое] знание: некоторые и доныне с совестью, [признающею] идолов, едят [идоложертвенное] как жертвы идольские, и совесть их, будучи немощна, оскверняется.
\end{tcolorbox}
\begin{tcolorbox}
\textsubscript{8} Пища не приближает нас к Богу: ибо, едим ли мы, ничего не приобретаем; не едим ли, ничего не теряем.
\end{tcolorbox}
\begin{tcolorbox}
\textsubscript{9} Берегитесь однако же, чтобы эта свобода ваша не послужила соблазном для немощных.
\end{tcolorbox}
\begin{tcolorbox}
\textsubscript{10} Ибо если кто-нибудь увидит, что ты, имея знание, сидишь за столом в капище, то совесть его, как немощного, не расположит ли и его есть идоложертвенное?
\end{tcolorbox}
\begin{tcolorbox}
\textsubscript{11} И от знания твоего погибнет немощный брат, за которого умер Христос.
\end{tcolorbox}
\begin{tcolorbox}
\textsubscript{12} А согрешая таким образом против братьев и уязвляя немощную совесть их, вы согрешаете против Христа.
\end{tcolorbox}
\begin{tcolorbox}
\textsubscript{13} И потому, если пища соблазняет брата моего, не буду есть мяса вовек, чтобы не соблазнить брата моего.
\end{tcolorbox}
\subsection{CHAPTER 9}
\begin{tcolorbox}
\textsubscript{1} Не Апостол ли я? Не свободен ли я? Не видел ли я Иисуса Христа, Господа нашего? Не мое ли дело вы в Господе?
\end{tcolorbox}
\begin{tcolorbox}
\textsubscript{2} Если для других я не Апостол, то для вас [Апостол]; ибо печать моего апостольства--вы в Господе.
\end{tcolorbox}
\begin{tcolorbox}
\textsubscript{3} Вот мое защищение против осуждающих меня.
\end{tcolorbox}
\begin{tcolorbox}
\textsubscript{4} Или мы не имеем власти есть и пить?
\end{tcolorbox}
\begin{tcolorbox}
\textsubscript{5} Или не имеем власти иметь спутницею сестру жену, как и прочие Апостолы, и братья Господни, и Кифа?
\end{tcolorbox}
\begin{tcolorbox}
\textsubscript{6} Или один я и Варнава не имеем власти не работать?
\end{tcolorbox}
\begin{tcolorbox}
\textsubscript{7} Какой воин служит когда-либо на своем содержании? Кто, насадив виноград, не ест плодов его? Кто, пася стадо, не ест молока от стада?
\end{tcolorbox}
\begin{tcolorbox}
\textsubscript{8} По человеческому ли только [рассуждению] я это говорю? Не то же ли говорит и закон?
\end{tcolorbox}
\begin{tcolorbox}
\textsubscript{9} Ибо в Моисеевом законе написано: не заграждай рта у вола молотящего. О волах ли печется Бог?
\end{tcolorbox}
\begin{tcolorbox}
\textsubscript{10} Или, конечно, для нас говорится? Так, для нас это написано; ибо, кто пашет, должен пахать с надеждою, и кто молотит, [должен молотить] с надеждою получить ожидаемое.
\end{tcolorbox}
\begin{tcolorbox}
\textsubscript{11} Если мы посеяли в вас духовное, велико ли то, если пожнем у вас телесное?
\end{tcolorbox}
\begin{tcolorbox}
\textsubscript{12} Если другие имеют у вас власть, не паче ли мы? Однако мы не пользовались сею властью, но все переносим, дабы не поставить какой преграды благовествованию Христову.
\end{tcolorbox}
\begin{tcolorbox}
\textsubscript{13} Разве не знаете, что священнодействующие питаются от святилища? что служащие жертвеннику берут долю от жертвенника?
\end{tcolorbox}
\begin{tcolorbox}
\textsubscript{14} Так и Господь повелел проповедующим Евангелие жить от благовествования.
\end{tcolorbox}
\begin{tcolorbox}
\textsubscript{15} Но я не пользовался ничем таковым. И написал это не для того, чтобы так было для меня. Ибо для меня лучше умереть, нежели чтобы кто уничтожил похвалу мою.
\end{tcolorbox}
\begin{tcolorbox}
\textsubscript{16} Ибо если я благовествую, то нечем мне хвалиться, потому что это необходимая [обязанность] моя, и горе мне, если не благовествую!
\end{tcolorbox}
\begin{tcolorbox}
\textsubscript{17} Ибо если делаю это добровольно, то [буду] иметь награду; а если недобровольно, то [исполняю только] вверенное мне служение.
\end{tcolorbox}
\begin{tcolorbox}
\textsubscript{18} За что же мне награда? За то, что, проповедуя Евангелие, благовествую о Христе безмездно, не пользуясь моею властью в благовествовании.
\end{tcolorbox}
\begin{tcolorbox}
\textsubscript{19} Ибо, будучи свободен от всех, я всем поработил себя, дабы больше приобрести:
\end{tcolorbox}
\begin{tcolorbox}
\textsubscript{20} для Иудеев я был как Иудей, чтобы приобрести Иудеев; для подзаконных был как подзаконный, чтобы приобрести подзаконных;
\end{tcolorbox}
\begin{tcolorbox}
\textsubscript{21} для чуждых закона--как чуждый закона, --не будучи чужд закона пред Богом, но подзаконен Христу, --чтобы приобрести чуждых закона;
\end{tcolorbox}
\begin{tcolorbox}
\textsubscript{22} для немощных был как немощный, чтобы приобрести немощных. Для всех я сделался всем, чтобы спасти по крайней мере некоторых.
\end{tcolorbox}
\begin{tcolorbox}
\textsubscript{23} Сие же делаю для Евангелия, чтобы быть соучастником его.
\end{tcolorbox}
\begin{tcolorbox}
\textsubscript{24} Не знаете ли, что бегущие на ристалище бегут все, но один получает награду? Так бегите, чтобы получить.
\end{tcolorbox}
\begin{tcolorbox}
\textsubscript{25} Все подвижники воздерживаются от всего: те для получения венца тленного, а мы--нетленного.
\end{tcolorbox}
\begin{tcolorbox}
\textsubscript{26} И потому я бегу не так, как на неверное, бьюсь не так, чтобы только бить воздух;
\end{tcolorbox}
\begin{tcolorbox}
\textsubscript{27} но усмиряю и порабощаю тело мое, дабы, проповедуя другим, самому не остаться недостойным.
\end{tcolorbox}
\subsection{CHAPTER 10}
\begin{tcolorbox}
\textsubscript{1} Не хочу оставить вас, братия, в неведении, что отцы наши все были под облаком, и все прошли сквозь море;
\end{tcolorbox}
\begin{tcolorbox}
\textsubscript{2} и все крестились в Моисея в облаке и в море;
\end{tcolorbox}
\begin{tcolorbox}
\textsubscript{3} и все ели одну и ту же духовную пищу;
\end{tcolorbox}
\begin{tcolorbox}
\textsubscript{4} и все пили одно и то же духовное питие: ибо пили из духовного последующего камня; камень же был Христос.
\end{tcolorbox}
\begin{tcolorbox}
\textsubscript{5} Но не о многих из них благоволил Бог, ибо они поражены были в пустыне.
\end{tcolorbox}
\begin{tcolorbox}
\textsubscript{6} А это были образы для нас, чтобы мы не были похотливы на злое, как они были похотливы.
\end{tcolorbox}
\begin{tcolorbox}
\textsubscript{7} Не будьте также идолопоклонниками, как некоторые из них, о которых написано: народ сел есть и пить, и встал играть.
\end{tcolorbox}
\begin{tcolorbox}
\textsubscript{8} Не станем блудодействовать, как некоторые из них блудодействовали, и в один день погибло их двадцать три тысячи.
\end{tcolorbox}
\begin{tcolorbox}
\textsubscript{9} Не станем искушать Христа, как некоторые из них искушали и погибли от змей.
\end{tcolorbox}
\begin{tcolorbox}
\textsubscript{10} Не ропщите, как некоторые из них роптали и погибли от истребителя.
\end{tcolorbox}
\begin{tcolorbox}
\textsubscript{11} Все это происходило с ними, [как] образы; а описано в наставление нам, достигшим последних веков.
\end{tcolorbox}
\begin{tcolorbox}
\textsubscript{12} Посему, кто думает, что он стоит, берегись, чтобы не упасть.
\end{tcolorbox}
\begin{tcolorbox}
\textsubscript{13} Вас постигло искушение не иное, как человеческое; и верен Бог, Который не попустит вам быть искушаемыми сверх сил, но при искушении даст и облегчение, так чтобы вы могли перенести.
\end{tcolorbox}
\begin{tcolorbox}
\textsubscript{14} Итак, возлюбленные мои, убегайте идолослужения.
\end{tcolorbox}
\begin{tcolorbox}
\textsubscript{15} Я говорю [вам] как рассудительным; сами рассудите о том, что говорю.
\end{tcolorbox}
\begin{tcolorbox}
\textsubscript{16} Чаша благословения, которую благословляем, не есть ли приобщение Крови Христовой? Хлеб, который преломляем, не есть ли приобщение Тела Христова?
\end{tcolorbox}
\begin{tcolorbox}
\textsubscript{17} Один хлеб, и мы многие одно тело; ибо все причащаемся от одного хлеба.
\end{tcolorbox}
\begin{tcolorbox}
\textsubscript{18} Посмотрите на Израиля по плоти: те, которые едят жертвы, не участники ли жертвенника?
\end{tcolorbox}
\begin{tcolorbox}
\textsubscript{19} Что же я говорю? То ли, что идол есть что-нибудь, или идоложертвенное значит что-нибудь?
\end{tcolorbox}
\begin{tcolorbox}
\textsubscript{20} [Нет], но что язычники, принося жертвы, приносят бесам, а не Богу. Но я не хочу, чтобы вы были в общении с бесами.
\end{tcolorbox}
\begin{tcolorbox}
\textsubscript{21} Не можете пить чашу Господню и чашу бесовскую; не можете быть участниками в трапезе Господней и в трапезе бесовской.
\end{tcolorbox}
\begin{tcolorbox}
\textsubscript{22} Неужели мы [решимся] раздражать Господа? Разве мы сильнее Его?
\end{tcolorbox}
\begin{tcolorbox}
\textsubscript{23} Все мне позволительно, но не все полезно; все мне позволительно, но не все назидает.
\end{tcolorbox}
\begin{tcolorbox}
\textsubscript{24} Никто не ищи своего, но каждый [пользы] другого.
\end{tcolorbox}
\begin{tcolorbox}
\textsubscript{25} Все, что продается на торгу, ешьте без всякого исследования, для [спокойствия] совести;
\end{tcolorbox}
\begin{tcolorbox}
\textsubscript{26} ибо Господня земля, и что наполняет ее.
\end{tcolorbox}
\begin{tcolorbox}
\textsubscript{27} Если кто из неверных позовет вас, и вы захотите пойти, то все, предлагаемое вам, ешьте без всякого исследования, для [спокойствия] совести.
\end{tcolorbox}
\begin{tcolorbox}
\textsubscript{28} Но если кто скажет вам: это идоложертвенное, --то не ешьте ради того, кто объявил вам, и ради совести. Ибо Господня земля, и что наполняет ее.
\end{tcolorbox}
\begin{tcolorbox}
\textsubscript{29} Совесть же разумею не свою, а другого: ибо для чего моей свободе быть судимой чужою совестью?
\end{tcolorbox}
\begin{tcolorbox}
\textsubscript{30} Если я с благодарением принимаю [пищу], то для чего порицать меня за то, за что я благодарю?
\end{tcolorbox}
\begin{tcolorbox}
\textsubscript{31} Итак, едите ли, пьете ли, или иное что делаете, все делайте в славу Божию.
\end{tcolorbox}
\begin{tcolorbox}
\textsubscript{32} Не подавайте соблазна ни Иудеям, ни Еллинам, ни церкви Божией,
\end{tcolorbox}
\begin{tcolorbox}
\textsubscript{33} так, как и я угождаю всем во всем, ища не своей пользы, но [пользы] многих, чтобы они спаслись.
\end{tcolorbox}
\subsection{CHAPTER 11}
\begin{tcolorbox}
\textsubscript{1} Будьте подражателями мне, как я Христу.
\end{tcolorbox}
\begin{tcolorbox}
\textsubscript{2} Хвалю вас, братия, что вы все мое помните и держите предания так, как я передал вам.
\end{tcolorbox}
\begin{tcolorbox}
\textsubscript{3} Хочу также, чтобы вы знали, что всякому мужу глава Христос, жене глава--муж, а Христу глава--Бог.
\end{tcolorbox}
\begin{tcolorbox}
\textsubscript{4} Всякий муж, молящийся или пророчествующий с покрытою головою, постыжает свою голову.
\end{tcolorbox}
\begin{tcolorbox}
\textsubscript{5} И всякая жена, молящаяся или пророчествующая с открытою головою, постыжает свою голову, ибо [это] то же, как если бы она была обритая.
\end{tcolorbox}
\begin{tcolorbox}
\textsubscript{6} Ибо если жена не хочет покрываться, то пусть и стрижется; а если жене стыдно быть остриженной или обритой, пусть покрывается.
\end{tcolorbox}
\begin{tcolorbox}
\textsubscript{7} Итак муж не должен покрывать голову, потому что он есть образ и слава Божия; а жена есть слава мужа.
\end{tcolorbox}
\begin{tcolorbox}
\textsubscript{8} Ибо не муж от жены, но жена от мужа;
\end{tcolorbox}
\begin{tcolorbox}
\textsubscript{9} и не муж создан для жены, но жена для мужа.
\end{tcolorbox}
\begin{tcolorbox}
\textsubscript{10} Посему жена и должна иметь на голове своей [знак] власти [над] [нею], для Ангелов.
\end{tcolorbox}
\begin{tcolorbox}
\textsubscript{11} Впрочем ни муж без жены, ни жена без мужа, в Господе.
\end{tcolorbox}
\begin{tcolorbox}
\textsubscript{12} Ибо как жена от мужа, так и муж через жену; все же--от Бога.
\end{tcolorbox}
\begin{tcolorbox}
\textsubscript{13} Рассудите сами, прилично ли жене молиться Богу с непокрытою [головою]?
\end{tcolorbox}
\begin{tcolorbox}
\textsubscript{14} Не сама ли природа учит вас, что если муж растит волосы, то это бесчестье для него,
\end{tcolorbox}
\begin{tcolorbox}
\textsubscript{15} но если жена растит волосы, для нее это честь, так как волосы даны ей вместо покрывала?
\end{tcolorbox}
\begin{tcolorbox}
\textsubscript{16} А если бы кто захотел спорить, то мы не имеем такого обычая, ни церкви Божии.
\end{tcolorbox}
\begin{tcolorbox}
\textsubscript{17} Но, предлагая сие, не хвалю [вас], что вы собираетесь не на лучшее, а на худшее.
\end{tcolorbox}
\begin{tcolorbox}
\textsubscript{18} Ибо, во-первых, слышу, что, когда вы собираетесь в церковь, между вами бывают разделения, чему отчасти и верю.
\end{tcolorbox}
\begin{tcolorbox}
\textsubscript{19} Ибо надлежит быть и разномыслиям между вами, дабы открылись между вами искусные.
\end{tcolorbox}
\begin{tcolorbox}
\textsubscript{20} Далее, вы собираетесь, [так, что это] не значит вкушать вечерю Господню;
\end{tcolorbox}
\begin{tcolorbox}
\textsubscript{21} ибо всякий поспешает прежде [других] есть свою пищу, [так] [что] иной бывает голоден, а иной упивается.
\end{tcolorbox}
\begin{tcolorbox}
\textsubscript{22} Разве у вас нет домов на то, чтобы есть и пить? Или пренебрегаете церковь Божию и унижаете неимущих? Что сказать вам? похвалить ли вас за это? Не похвалю.
\end{tcolorbox}
\begin{tcolorbox}
\textsubscript{23} Ибо я от [Самого] Господа принял то, что и вам передал, что Господь Иисус в ту ночь, в которую предан был, взял хлеб
\end{tcolorbox}
\begin{tcolorbox}
\textsubscript{24} и, возблагодарив, преломил и сказал: приимите, ядите, сие есть Тело Мое, за вас ломимое; сие творите в Мое воспоминание.
\end{tcolorbox}
\begin{tcolorbox}
\textsubscript{25} Также и чашу после вечери, и сказал: сия чаша есть новый завет в Моей Крови; сие творите, когда только будете пить, в Мое воспоминание.
\end{tcolorbox}
\begin{tcolorbox}
\textsubscript{26} Ибо всякий раз, когда вы едите хлеб сей и пьете чашу сию, смерть Господню возвещаете, доколе Он придет.
\end{tcolorbox}
\begin{tcolorbox}
\textsubscript{27} Посему, кто будет есть хлеб сей или пить чашу Господню недостойно, виновен будет против Тела и Крови Господней.
\end{tcolorbox}
\begin{tcolorbox}
\textsubscript{28} Да испытывает же себя человек, и таким образом пусть ест от хлеба сего и пьет из чаши сей.
\end{tcolorbox}
\begin{tcolorbox}
\textsubscript{29} Ибо, кто ест и пьет недостойно, тот ест и пьет осуждение себе, не рассуждая о Теле Господнем.
\end{tcolorbox}
\begin{tcolorbox}
\textsubscript{30} Оттого многие из вас немощны и больны и немало умирает.
\end{tcolorbox}
\begin{tcolorbox}
\textsubscript{31} Ибо если бы мы судили сами себя, то не были бы судимы.
\end{tcolorbox}
\begin{tcolorbox}
\textsubscript{32} Будучи же судимы, наказываемся от Господа, чтобы не быть осужденными с миром.
\end{tcolorbox}
\begin{tcolorbox}
\textsubscript{33} Посему, братия мои, собираясь на вечерю, друг друга ждите.
\end{tcolorbox}
\begin{tcolorbox}
\textsubscript{34} А если кто голоден, пусть ест дома, чтобы собираться вам не на осуждение. Прочее устрою, когда приду.
\end{tcolorbox}
\subsection{CHAPTER 12}
\begin{tcolorbox}
\textsubscript{1} Не хочу оставить вас, братия, в неведении и о [дарах] духовных.
\end{tcolorbox}
\begin{tcolorbox}
\textsubscript{2} Знаете, что когда вы были язычниками, то ходили к безгласным идолам, так, как бы вели вас.
\end{tcolorbox}
\begin{tcolorbox}
\textsubscript{3} Потому сказываю вам, что никто, говорящий Духом Божиим, не произнесет анафемы на Иисуса, и никто не может назвать Иисуса Господом, как только Духом Святым.
\end{tcolorbox}
\begin{tcolorbox}
\textsubscript{4} Дары различны, но Дух один и тот же;
\end{tcolorbox}
\begin{tcolorbox}
\textsubscript{5} и служения различны, а Господь один и тот же;
\end{tcolorbox}
\begin{tcolorbox}
\textsubscript{6} и действия различны, а Бог один и тот же, производящий все во всех.
\end{tcolorbox}
\begin{tcolorbox}
\textsubscript{7} Но каждому дается проявление Духа на пользу.
\end{tcolorbox}
\begin{tcolorbox}
\textsubscript{8} Одному дается Духом слово мудрости, другому слово знания, тем же Духом;
\end{tcolorbox}
\begin{tcolorbox}
\textsubscript{9} иному вера, тем же Духом; иному дары исцелений, тем же Духом;
\end{tcolorbox}
\begin{tcolorbox}
\textsubscript{10} иному чудотворения, иному пророчество, иному различение духов, иному разные языки, иному истолкование языков.
\end{tcolorbox}
\begin{tcolorbox}
\textsubscript{11} Все же сие производит один и тот же Дух, разделяя каждому особо, как Ему угодно.
\end{tcolorbox}
\begin{tcolorbox}
\textsubscript{12} Ибо, как тело одно, но имеет многие члены, и все члены одного тела, хотя их и много, составляют одно тело, --так и Христос.
\end{tcolorbox}
\begin{tcolorbox}
\textsubscript{13} Ибо все мы одним Духом крестились в одно тело, Иудеи или Еллины, рабы или свободные, и все напоены одним Духом.
\end{tcolorbox}
\begin{tcolorbox}
\textsubscript{14} Тело же не из одного члена, но из многих.
\end{tcolorbox}
\begin{tcolorbox}
\textsubscript{15} Если нога скажет: я не принадлежу к телу, потому что я не рука, то неужели она потому не принадлежит к телу?
\end{tcolorbox}
\begin{tcolorbox}
\textsubscript{16} И если ухо скажет: я не принадлежу к телу, потому что я не глаз, то неужели оно потому не принадлежит к телу?
\end{tcolorbox}
\begin{tcolorbox}
\textsubscript{17} Если все тело глаз, то где слух? Если все слух, то где обоняние?
\end{tcolorbox}
\begin{tcolorbox}
\textsubscript{18} Но Бог расположил члены, каждый в [составе] тела, как Ему было угодно.
\end{tcolorbox}
\begin{tcolorbox}
\textsubscript{19} А если бы все были один член, то где [было бы] тело?
\end{tcolorbox}
\begin{tcolorbox}
\textsubscript{20} Но теперь членов много, а тело одно.
\end{tcolorbox}
\begin{tcolorbox}
\textsubscript{21} Не может глаз сказать руке: ты мне не надобна; или также голова ногам: вы мне не нужны.
\end{tcolorbox}
\begin{tcolorbox}
\textsubscript{22} Напротив, члены тела, которые кажутся слабейшими, гораздо нужнее,
\end{tcolorbox}
\begin{tcolorbox}
\textsubscript{23} и которые нам кажутся менее благородными в теле, о тех более прилагаем попечения;
\end{tcolorbox}
\begin{tcolorbox}
\textsubscript{24} и неблагообразные наши более благовидно покрываются, а благообразные наши не имеют [в том] нужды. Но Бог соразмерил тело, внушив о менее совершенном большее попечение,
\end{tcolorbox}
\begin{tcolorbox}
\textsubscript{25} дабы не было разделения в теле, а все члены одинаково заботились друг о друге.
\end{tcolorbox}
\begin{tcolorbox}
\textsubscript{26} Посему, страдает ли один член, страдают с ним все члены; славится ли один член, с ним радуются все члены.
\end{tcolorbox}
\begin{tcolorbox}
\textsubscript{27} И вы--тело Христово, а порознь--члены.
\end{tcolorbox}
\begin{tcolorbox}
\textsubscript{28} И иных Бог поставил в Церкви, во-первых, Апостолами, во-- вторых, пророками, в-третьих, учителями; далее, [иным дал] силы [чудодейственные], также дары исцелений, вспоможения, управления, разные языки.
\end{tcolorbox}
\begin{tcolorbox}
\textsubscript{29} Все ли Апостолы? Все ли пророки? Все ли учители? Все ли чудотворцы?
\end{tcolorbox}
\begin{tcolorbox}
\textsubscript{30} Все ли имеют дары исцелений? Все ли говорят языками? Все ли истолкователи?
\end{tcolorbox}
\begin{tcolorbox}
\textsubscript{31} Ревнуйте о дарах больших, и я покажу вам путь еще превосходнейший.
\end{tcolorbox}
\subsection{CHAPTER 13}
\begin{tcolorbox}
\textsubscript{1} Если я говорю языками человеческими и ангельскими, а любви не имею, то я--медь звенящая или кимвал звучащий.
\end{tcolorbox}
\begin{tcolorbox}
\textsubscript{2} Если имею [дар] пророчества, и знаю все тайны, и имею всякое познание и всю веру, так что [могу] и горы переставлять, а не имею любви, --то я ничто.
\end{tcolorbox}
\begin{tcolorbox}
\textsubscript{3} И если я раздам все имение мое и отдам тело мое на сожжение, а любви не имею, нет мне в том никакой пользы.
\end{tcolorbox}
\begin{tcolorbox}
\textsubscript{4} Любовь долготерпит, милосердствует, любовь не завидует, любовь не превозносится, не гордится,
\end{tcolorbox}
\begin{tcolorbox}
\textsubscript{5} не бесчинствует, не ищет своего, не раздражается, не мыслит зла,
\end{tcolorbox}
\begin{tcolorbox}
\textsubscript{6} не радуется неправде, а сорадуется истине;
\end{tcolorbox}
\begin{tcolorbox}
\textsubscript{7} все покрывает, всему верит, всего надеется, все переносит.
\end{tcolorbox}
\begin{tcolorbox}
\textsubscript{8} Любовь никогда не перестает, хотя и пророчества прекратятся, и языки умолкнут, и знание упразднится.
\end{tcolorbox}
\begin{tcolorbox}
\textsubscript{9} Ибо мы отчасти знаем, и отчасти пророчествуем;
\end{tcolorbox}
\begin{tcolorbox}
\textsubscript{10} когда же настанет совершенное, тогда то, что отчасти, прекратится.
\end{tcolorbox}
\begin{tcolorbox}
\textsubscript{11} Когда я был младенцем, то по-младенчески говорил, по-- младенчески мыслил, по-младенчески рассуждал; а как стал мужем, то оставил младенческое.
\end{tcolorbox}
\begin{tcolorbox}
\textsubscript{12} Теперь мы видим как бы сквозь [тусклое] стекло, гадательно, тогда же лицем к лицу; теперь знаю я отчасти, а тогда познаю, подобно как я познан.
\end{tcolorbox}
\begin{tcolorbox}
\textsubscript{13} А теперь пребывают сии три: вера, надежда, любовь; но любовь из них больше.
\end{tcolorbox}
\subsection{CHAPTER 14}
\begin{tcolorbox}
\textsubscript{1} Достигайте любви; ревнуйте о [дарах] духовных, особенно же о том, чтобы пророчествовать.
\end{tcolorbox}
\begin{tcolorbox}
\textsubscript{2} Ибо кто говорит на [незнакомом] языке, тот говорит не людям, а Богу; потому что никто не понимает [его], он тайны говорит духом;
\end{tcolorbox}
\begin{tcolorbox}
\textsubscript{3} а кто пророчествует, тот говорит людям в назидание, увещание и утешение.
\end{tcolorbox}
\begin{tcolorbox}
\textsubscript{4} Кто говорит на [незнакомом] языке, тот назидает себя; а кто пророчествует, тот назидает церковь.
\end{tcolorbox}
\begin{tcolorbox}
\textsubscript{5} Желаю, чтобы вы все говорили языками; но лучше, чтобы вы пророчествовали; ибо пророчествующий превосходнее того, кто говорит языками, разве он притом будет и изъяснять, чтобы церковь получила назидание.
\end{tcolorbox}
\begin{tcolorbox}
\textsubscript{6} Теперь, если я приду к вам, братия, и стану говорить на [незнакомых] языках, то какую принесу вам пользу, когда не изъяснюсь вам или откровением, или познанием, или пророчеством, или учением?
\end{tcolorbox}
\begin{tcolorbox}
\textsubscript{7} И бездушные [вещи], издающие звук, свирель или гусли, если не производят раздельных тонов, как распознать то, что играют на свирели или на гуслях?
\end{tcolorbox}
\begin{tcolorbox}
\textsubscript{8} И если труба будет издавать неопределенный звук, кто станет готовиться к сражению?
\end{tcolorbox}
\begin{tcolorbox}
\textsubscript{9} Так если и вы языком произносите невразумительные слова, то как узнают, что вы говорите? Вы будете говорить на ветер.
\end{tcolorbox}
\begin{tcolorbox}
\textsubscript{10} Сколько, например, различных слов в мире, и ни одного из них нет без значения.
\end{tcolorbox}
\begin{tcolorbox}
\textsubscript{11} Но если я не разумею значения слов, то я для говорящего чужестранец, и говорящий для меня чужестранец.
\end{tcolorbox}
\begin{tcolorbox}
\textsubscript{12} Так и вы, ревнуя о [дарах] духовных, старайтесь обогатиться [ими] к назиданию церкви.
\end{tcolorbox}
\begin{tcolorbox}
\textsubscript{13} А потому, говорящий на [незнакомом] языке, молись о даре истолкования.
\end{tcolorbox}
\begin{tcolorbox}
\textsubscript{14} Ибо когда я молюсь на [незнакомом] языке, то хотя дух мой и молится, но ум мой остается без плода.
\end{tcolorbox}
\begin{tcolorbox}
\textsubscript{15} Что же делать? Стану молиться духом, стану молиться и умом; буду петь духом, буду петь и умом.
\end{tcolorbox}
\begin{tcolorbox}
\textsubscript{16} Ибо если ты будешь благословлять духом, то стоящий на месте простолюдина как скажет: 'аминь' при твоем благодарении? Ибо он не понимает, что ты говоришь.
\end{tcolorbox}
\begin{tcolorbox}
\textsubscript{17} Ты хорошо благодаришь, но другой не назидается.
\end{tcolorbox}
\begin{tcolorbox}
\textsubscript{18} Благодарю Бога моего: я более всех вас говорю языками;
\end{tcolorbox}
\begin{tcolorbox}
\textsubscript{19} но в церкви хочу лучше пять слов сказать умом моим, чтобы и других наставить, нежели тьму слов на [незнакомом] языке.
\end{tcolorbox}
\begin{tcolorbox}
\textsubscript{20} Братия! не будьте дети умом: на злое будьте младенцы, а по уму будьте совершеннолетни.
\end{tcolorbox}
\begin{tcolorbox}
\textsubscript{21} В законе написано: иными языками и иными устами буду говорить народу сему; но и тогда не послушают Меня, говорит Господь.
\end{tcolorbox}
\begin{tcolorbox}
\textsubscript{22} Итак языки суть знамение не для верующих, а для неверующих; пророчество же не для неверующих, а для верующих.
\end{tcolorbox}
\begin{tcolorbox}
\textsubscript{23} Если вся церковь сойдется вместе, и все станут говорить [незнакомыми] языками, и войдут к вам незнающие или неверующие, то не скажут ли, что вы беснуетесь?
\end{tcolorbox}
\begin{tcolorbox}
\textsubscript{24} Но когда все пророчествуют, и войдет кто неверующий или незнающий, то он всеми обличается, всеми судится.
\end{tcolorbox}
\begin{tcolorbox}
\textsubscript{25} И таким образом тайны сердца его обнаруживаются, и он падет ниц, поклонится Богу и скажет: истинно с вами Бог.
\end{tcolorbox}
\begin{tcolorbox}
\textsubscript{26} Итак что же, братия? Когда вы сходитесь, и у каждого из вас есть псалом, есть поучение, есть язык, есть откровение, есть истолкование, --все сие да будет к назиданию.
\end{tcolorbox}
\begin{tcolorbox}
\textsubscript{27} Если кто говорит на [незнакомом] языке, [говорите] двое, или много трое, и то порознь, а один изъясняй.
\end{tcolorbox}
\begin{tcolorbox}
\textsubscript{28} Если же не будет истолкователя, то молчи в церкви, а говори себе и Богу.
\end{tcolorbox}
\begin{tcolorbox}
\textsubscript{29} И пророки пусть говорят двое или трое, а прочие пусть рассуждают.
\end{tcolorbox}
\begin{tcolorbox}
\textsubscript{30} Если же другому из сидящих будет откровение, то первый молчи.
\end{tcolorbox}
\begin{tcolorbox}
\textsubscript{31} Ибо все один за другим можете пророчествовать, чтобы всем поучаться и всем получать утешение.
\end{tcolorbox}
\begin{tcolorbox}
\textsubscript{32} И духи пророческие послушны пророкам,
\end{tcolorbox}
\begin{tcolorbox}
\textsubscript{33} потому что Бог не есть [Бог] неустройства, но мира. Так [бывает] во всех церквах у святых.
\end{tcolorbox}
\begin{tcolorbox}
\textsubscript{34} Жены ваши в церквах да молчат, ибо не позволено им говорить, а быть в подчинении, как и закон говорит.
\end{tcolorbox}
\begin{tcolorbox}
\textsubscript{35} Если же они хотят чему научиться, пусть спрашивают [о том] дома у мужей своих; ибо неприлично жене говорить в церкви.
\end{tcolorbox}
\begin{tcolorbox}
\textsubscript{36} Разве от вас вышло слово Божие? Или до вас одних достигло?
\end{tcolorbox}
\begin{tcolorbox}
\textsubscript{37} Если кто почитает себя пророком или духовным, тот да разумеет, что я пишу вам, ибо это заповеди Господни.
\end{tcolorbox}
\begin{tcolorbox}
\textsubscript{38} А кто не разумеет, пусть не разумеет.
\end{tcolorbox}
\begin{tcolorbox}
\textsubscript{39} Итак, братия, ревнуйте о том, чтобы пророчествовать, но не запрещайте говорить и языками;
\end{tcolorbox}
\begin{tcolorbox}
\textsubscript{40} только всё должно быть благопристойно и чинно.
\end{tcolorbox}
\subsection{CHAPTER 15}
\begin{tcolorbox}
\textsubscript{1} Напоминаю вам, братия, Евангелие, которое я благовествовал вам, которое вы и приняли, в котором и утвердились,
\end{tcolorbox}
\begin{tcolorbox}
\textsubscript{2} которым и спасаетесь, если преподанное удерживаете так, как я благовествовал вам, если только не тщетно уверовали.
\end{tcolorbox}
\begin{tcolorbox}
\textsubscript{3} Ибо я первоначально преподал вам, что и [сам] принял, [то] [есть], что Христос умер за грехи наши, по Писанию,
\end{tcolorbox}
\begin{tcolorbox}
\textsubscript{4} и что Он погребен был, и что воскрес в третий день, по Писанию,
\end{tcolorbox}
\begin{tcolorbox}
\textsubscript{5} и что явился Кифе, потом двенадцати;
\end{tcolorbox}
\begin{tcolorbox}
\textsubscript{6} потом явился более нежели пятистам братий в одно время, из которых большая часть доныне в живых, а некоторые и почили;
\end{tcolorbox}
\begin{tcolorbox}
\textsubscript{7} потом явился Иакову, также всем Апостолам;
\end{tcolorbox}
\begin{tcolorbox}
\textsubscript{8} а после всех явился и мне, как некоему извергу.
\end{tcolorbox}
\begin{tcolorbox}
\textsubscript{9} Ибо я наименьший из Апостолов, и недостоин называться Апостолом, потому что гнал церковь Божию.
\end{tcolorbox}
\begin{tcolorbox}
\textsubscript{10} Но благодатию Божиею есмь то, что есмь; и благодать Его во мне не была тщетна, но я более всех их потрудился: не я, впрочем, а благодать Божия, которая со мною.
\end{tcolorbox}
\begin{tcolorbox}
\textsubscript{11} Итак я ли, они ли, мы так проповедуем, и вы так уверовали.
\end{tcolorbox}
\begin{tcolorbox}
\textsubscript{12} Если же о Христе проповедуется, что Он воскрес из мертвых, то как некоторые из вас говорят, что нет воскресения мертвых?
\end{tcolorbox}
\begin{tcolorbox}
\textsubscript{13} Если нет воскресения мертвых, то и Христос не воскрес;
\end{tcolorbox}
\begin{tcolorbox}
\textsubscript{14} а если Христос не воскрес, то и проповедь наша тщетна, тщетна и вера ваша.
\end{tcolorbox}
\begin{tcolorbox}
\textsubscript{15} Притом мы оказались бы и лжесвидетелями о Боге, потому что свидетельствовали бы о Боге, что Он воскресил Христа, Которого Он не воскрешал, если, [то есть], мертвые не воскресают;
\end{tcolorbox}
\begin{tcolorbox}
\textsubscript{16} ибо если мертвые не воскресают, то и Христос не воскрес.
\end{tcolorbox}
\begin{tcolorbox}
\textsubscript{17} А если Христос не воскрес, то вера ваша тщетна: вы еще во грехах ваших.
\end{tcolorbox}
\begin{tcolorbox}
\textsubscript{18} Поэтому и умершие во Христе погибли.
\end{tcolorbox}
\begin{tcolorbox}
\textsubscript{19} И если мы в этой только жизни надеемся на Христа, то мы несчастнее всех человеков.
\end{tcolorbox}
\begin{tcolorbox}
\textsubscript{20} Но Христос воскрес из мертвых, первенец из умерших.
\end{tcolorbox}
\begin{tcolorbox}
\textsubscript{21} Ибо, как смерть через человека, [так] через человека и воскресение мертвых.
\end{tcolorbox}
\begin{tcolorbox}
\textsubscript{22} Как в Адаме все умирают, так во Христе все оживут,
\end{tcolorbox}
\begin{tcolorbox}
\textsubscript{23} каждый в своем порядке: первенец Христос, потом Христовы, в пришествие Его.
\end{tcolorbox}
\begin{tcolorbox}
\textsubscript{24} А затем конец, когда Он предаст Царство Богу и Отцу, когда упразднит всякое начальство и всякую власть и силу.
\end{tcolorbox}
\begin{tcolorbox}
\textsubscript{25} Ибо Ему надлежит царствовать, доколе низложит всех врагов под ноги Свои.
\end{tcolorbox}
\begin{tcolorbox}
\textsubscript{26} Последний же враг истребится--смерть,
\end{tcolorbox}
\begin{tcolorbox}
\textsubscript{27} потому что все покорил под ноги Его. Когда же сказано, что [Ему] все покорено, то ясно, что кроме Того, Который покорил Ему все.
\end{tcolorbox}
\begin{tcolorbox}
\textsubscript{28} Когда же все покорит Ему, тогда и Сам Сын покорится Покорившему все Ему, да будет Бог все во всем.
\end{tcolorbox}
\begin{tcolorbox}
\textsubscript{29} Иначе, что делают крестящиеся для мертвых? Если мертвые совсем не воскресают, то для чего и крестятся для мертвых?
\end{tcolorbox}
\begin{tcolorbox}
\textsubscript{30} Для чего и мы ежечасно подвергаемся бедствиям?
\end{tcolorbox}
\begin{tcolorbox}
\textsubscript{31} Я каждый день умираю: свидетельствуюсь в том похвалою вашею, братия, которую я имею во Христе Иисусе, Господе нашем.
\end{tcolorbox}
\begin{tcolorbox}
\textsubscript{32} По [рассуждению] человеческому, когда я боролся со зверями в Ефесе, какая мне польза, если мертвые не воскресают? Станем есть и пить, ибо завтра умрем!
\end{tcolorbox}
\begin{tcolorbox}
\textsubscript{33} Не обманывайтесь: худые сообщества развращают добрые нравы.
\end{tcolorbox}
\begin{tcolorbox}
\textsubscript{34} Отрезвитесь, как должно, и не грешите; ибо, к стыду вашему скажу, некоторые из вас не знают Бога.
\end{tcolorbox}
\begin{tcolorbox}
\textsubscript{35} Но скажет кто-нибудь: как воскреснут мертвые? и в каком теле придут?
\end{tcolorbox}
\begin{tcolorbox}
\textsubscript{36} Безрассудный! то, что ты сеешь, не оживет, если не умрет.
\end{tcolorbox}
\begin{tcolorbox}
\textsubscript{37} И когда ты сеешь, то сеешь не тело будущее, а голое зерно, какое случится, пшеничное или другое какое;
\end{tcolorbox}
\begin{tcolorbox}
\textsubscript{38} но Бог дает ему тело, как хочет, и каждому семени свое тело.
\end{tcolorbox}
\begin{tcolorbox}
\textsubscript{39} Не всякая плоть такая же плоть; но иная плоть у человеков, иная плоть у скотов, иная у рыб, иная у птиц.
\end{tcolorbox}
\begin{tcolorbox}
\textsubscript{40} Есть тела небесные и тела земные; но иная слава небесных, иная земных.
\end{tcolorbox}
\begin{tcolorbox}
\textsubscript{41} Иная слава солнца, иная слава луны, иная звезд; и звезда от звезды разнится в славе.
\end{tcolorbox}
\begin{tcolorbox}
\textsubscript{42} Так и при воскресении мертвых: сеется в тлении, восстает в нетлении;
\end{tcolorbox}
\begin{tcolorbox}
\textsubscript{43} сеется в уничижении, восстает в славе; сеется в немощи, восстает в силе;
\end{tcolorbox}
\begin{tcolorbox}
\textsubscript{44} сеется тело душевное, восстает тело духовное. Есть тело душевное, есть тело и духовное.
\end{tcolorbox}
\begin{tcolorbox}
\textsubscript{45} Так и написано: первый человек Адам стал душею живущею; а последний Адам есть дух животворящий.
\end{tcolorbox}
\begin{tcolorbox}
\textsubscript{46} Но не духовное прежде, а душевное, потом духовное.
\end{tcolorbox}
\begin{tcolorbox}
\textsubscript{47} Первый человек--из земли, перстный; второй человек--Господь с неба.
\end{tcolorbox}
\begin{tcolorbox}
\textsubscript{48} Каков перстный, таковы и перстные; и каков небесный, таковы и небесные.
\end{tcolorbox}
\begin{tcolorbox}
\textsubscript{49} И как мы носили образ перстного, будем носить и образ небесного.
\end{tcolorbox}
\begin{tcolorbox}
\textsubscript{50} Но то скажу [вам], братия, что плоть и кровь не могут наследовать Царствия Божия, и тление не наследует нетления.
\end{tcolorbox}
\begin{tcolorbox}
\textsubscript{51} Говорю вам тайну: не все мы умрем, но все изменимся
\end{tcolorbox}
\begin{tcolorbox}
\textsubscript{52} вдруг, во мгновение ока, при последней трубе; ибо вострубит, и мертвые воскреснут нетленными, а мы изменимся.
\end{tcolorbox}
\begin{tcolorbox}
\textsubscript{53} Ибо тленному сему надлежит облечься в нетление, и смертному сему облечься в бессмертие.
\end{tcolorbox}
\begin{tcolorbox}
\textsubscript{54} Когда же тленное сие облечется в нетление и смертное сие облечется в бессмертие, тогда сбудется слово написанное: поглощена смерть победою.
\end{tcolorbox}
\begin{tcolorbox}
\textsubscript{55} Смерть! где твое жало? ад! где твоя победа?
\end{tcolorbox}
\begin{tcolorbox}
\textsubscript{56} Жало же смерти--грех; а сила греха--закон.
\end{tcolorbox}
\begin{tcolorbox}
\textsubscript{57} Благодарение Богу, даровавшему нам победу Господом нашим Иисусом Христом!
\end{tcolorbox}
\begin{tcolorbox}
\textsubscript{58} Итак, братия мои возлюбленные, будьте тверды, непоколебимы, всегда преуспевайте в деле Господнем, зная, что труд ваш не тщетен пред Господом.
\end{tcolorbox}
\subsection{CHAPTER 16}
\begin{tcolorbox}
\textsubscript{1} При сборе же для святых поступайте так, как я установил в церквах Галатийских.
\end{tcolorbox}
\begin{tcolorbox}
\textsubscript{2} В первый день недели каждый из вас пусть отлагает у себя и сберегает, сколько позволит ему состояние, чтобы не делать сборов, когда я приду.
\end{tcolorbox}
\begin{tcolorbox}
\textsubscript{3} Когда же приду, то, которых вы изберете, тех отправлю с письмами, для доставления вашего подаяния в Иерусалим.
\end{tcolorbox}
\begin{tcolorbox}
\textsubscript{4} А если прилично будет и мне отправиться, то они со мной пойдут.
\end{tcolorbox}
\begin{tcolorbox}
\textsubscript{5} Я приду к вам, когда пройду Македонию; ибо я иду через Македонию.
\end{tcolorbox}
\begin{tcolorbox}
\textsubscript{6} У вас же, может быть, поживу, или и перезимую, чтобы вы меня проводили, куда пойду.
\end{tcolorbox}
\begin{tcolorbox}
\textsubscript{7} Ибо я не хочу видеться с вами теперь мимоходом, а надеюсь пробыть у вас несколько времени, если Господь позволит.
\end{tcolorbox}
\begin{tcolorbox}
\textsubscript{8} В Ефесе же я пробуду до Пятидесятницы,
\end{tcolorbox}
\begin{tcolorbox}
\textsubscript{9} ибо для меня отверста великая и широкая дверь, и противников много.
\end{tcolorbox}
\begin{tcolorbox}
\textsubscript{10} Если же придет к вам Тимофей, смотрите, чтобы он был у вас безопасен; ибо он делает дело Господне, как и я.
\end{tcolorbox}
\begin{tcolorbox}
\textsubscript{11} Посему никто не пренебрегай его, но проводите его с миром, чтобы он пришел ко мне, ибо я жду его с братиями.
\end{tcolorbox}
\begin{tcolorbox}
\textsubscript{12} А что до брата Аполлоса, я очень просил его, чтобы он с братиями пошел к вам; но он никак не хотел идти ныне, а придет, когда ему будет удобно.
\end{tcolorbox}
\begin{tcolorbox}
\textsubscript{13} Бодрствуйте, стойте в вере, будьте мужественны, тверды.
\end{tcolorbox}
\begin{tcolorbox}
\textsubscript{14} Все у вас да будет с любовью.
\end{tcolorbox}
\begin{tcolorbox}
\textsubscript{15} Прошу вас, братия (вы знаете семейство Стефаново, что оно есть начаток Ахаии и что они посвятили себя на служение святым),
\end{tcolorbox}
\begin{tcolorbox}
\textsubscript{16} будьте и вы почтительны к таковым и ко всякому содействующему и трудящемуся.
\end{tcolorbox}
\begin{tcolorbox}
\textsubscript{17} Я рад прибытию Стефана, Фортуната и Ахаика: они восполнили для меня отсутствие ваше,
\end{tcolorbox}
\begin{tcolorbox}
\textsubscript{18} ибо они мой и ваш дух успокоили. Почитайте таковых.
\end{tcolorbox}
\begin{tcolorbox}
\textsubscript{19} Приветствуют вас церкви Асийские; приветствуют вас усердно в Господе Акила и Прискилла с домашнею их церковью.
\end{tcolorbox}
\begin{tcolorbox}
\textsubscript{20} Приветствуют вас все братия. Приветствуйте друг друга святым целованием.
\end{tcolorbox}
\begin{tcolorbox}
\textsubscript{21} Мое, Павлово, приветствие собственноручно.
\end{tcolorbox}
\begin{tcolorbox}
\textsubscript{22} Кто не любит Господа Иисуса Христа, анафема, маран-афа.
\end{tcolorbox}
\begin{tcolorbox}
\textsubscript{23} Благодать Господа нашего Иисуса Христа с вами,
\end{tcolorbox}
\begin{tcolorbox}
\textsubscript{24} и любовь моя со всеми вами во Христе Иисусе. Аминь.
\end{tcolorbox}
