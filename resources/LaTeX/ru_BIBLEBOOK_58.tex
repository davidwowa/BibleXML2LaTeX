\section{BOOK 57}
\subsection{CHAPTER 1}
\begin{tcolorbox}
\textsubscript{1} Притчи Соломона, сына Давидова, царя Израильского,
\end{tcolorbox}
\begin{tcolorbox}
\textsubscript{2} чтобы познать мудрость и наставление, понять изречения разума;
\end{tcolorbox}
\begin{tcolorbox}
\textsubscript{3} усвоить правила благоразумия, правосудия, суда и правоты;
\end{tcolorbox}
\begin{tcolorbox}
\textsubscript{4} простым дать смышленость, юноше--знание и рассудительность;
\end{tcolorbox}
\begin{tcolorbox}
\textsubscript{5} послушает мудрый--и умножит познания, и разумный найдет мудрые советы;
\end{tcolorbox}
\begin{tcolorbox}
\textsubscript{6} чтобы разуметь притчу и замысловатую речь, слова мудрецов и загадки их.
\end{tcolorbox}
\begin{tcolorbox}
\textsubscript{7} Начало мудрости--страх Господень; глупцы только презирают мудрость и наставление.
\end{tcolorbox}
\begin{tcolorbox}
\textsubscript{8} Слушай, сын мой, наставление отца твоего и не отвергай завета матери твоей,
\end{tcolorbox}
\begin{tcolorbox}
\textsubscript{9} потому что это--прекрасный венок для головы твоей и украшение для шеи твоей.
\end{tcolorbox}
\begin{tcolorbox}
\textsubscript{10} Сын мой! если будут склонять тебя грешники, не соглашайся;
\end{tcolorbox}
\begin{tcolorbox}
\textsubscript{11} если будут говорить: 'иди с нами, сделаем засаду для убийства, подстережем непорочного без вины,
\end{tcolorbox}
\begin{tcolorbox}
\textsubscript{12} живых проглотим их, как преисподняя, и--целых, как нисходящих в могилу;
\end{tcolorbox}
\begin{tcolorbox}
\textsubscript{13} наберем всякого драгоценного имущества, наполним домы наши добычею;
\end{tcolorbox}
\begin{tcolorbox}
\textsubscript{14} жребий твой ты будешь бросать вместе с нами, склад один будет у всех нас', --
\end{tcolorbox}
\begin{tcolorbox}
\textsubscript{15} сын мой! не ходи в путь с ними, удержи ногу твою от стези их,
\end{tcolorbox}
\begin{tcolorbox}
\textsubscript{16} потому что ноги их бегут ко злу и спешат на пролитие крови.
\end{tcolorbox}
\begin{tcolorbox}
\textsubscript{17} В глазах всех птиц напрасно расставляется сеть,
\end{tcolorbox}
\begin{tcolorbox}
\textsubscript{18} а делают засаду для их крови и подстерегают их души.
\end{tcolorbox}
\begin{tcolorbox}
\textsubscript{19} Таковы пути всякого, кто алчет чужого добра: оно отнимает жизнь у завладевшего им.
\end{tcolorbox}
\begin{tcolorbox}
\textsubscript{20} Премудрость возглашает на улице, на площадях возвышает голос свой,
\end{tcolorbox}
\begin{tcolorbox}
\textsubscript{21} в главных местах собраний проповедует, при входах в городские ворота говорит речь свою:
\end{tcolorbox}
\begin{tcolorbox}
\textsubscript{22} 'доколе, невежды, будете любить невежество? [доколе] буйные будут услаждаться буйством? доколе глупцы будут ненавидеть знание?
\end{tcolorbox}
\begin{tcolorbox}
\textsubscript{23} Обратитесь к моему обличению: вот, я изолью на вас дух мой, возвещу вам слова мои.
\end{tcolorbox}
\begin{tcolorbox}
\textsubscript{24} Я звала, и вы не послушались; простирала руку мою, и не было внимающего;
\end{tcolorbox}
\begin{tcolorbox}
\textsubscript{25} и вы отвергли все мои советы, и обличений моих не приняли.
\end{tcolorbox}
\begin{tcolorbox}
\textsubscript{26} За то и я посмеюсь вашей погибели; порадуюсь, когда придет на вас ужас;
\end{tcolorbox}
\begin{tcolorbox}
\textsubscript{27} когда придет на вас ужас, как буря, и беда, как вихрь, принесется на вас; когда постигнет вас скорбь и теснота.
\end{tcolorbox}
\begin{tcolorbox}
\textsubscript{28} Тогда будут звать меня, и я не услышу; с утра будут искать меня, и не найдут меня.
\end{tcolorbox}
\begin{tcolorbox}
\textsubscript{29} За то, что они возненавидели знание и не избрали [для себя] страха Господня,
\end{tcolorbox}
\begin{tcolorbox}
\textsubscript{30} не приняли совета моего, презрели все обличения мои;
\end{tcolorbox}
\begin{tcolorbox}
\textsubscript{31} за то и будут они вкушать от плодов путей своих и насыщаться от помыслов их.
\end{tcolorbox}
\begin{tcolorbox}
\textsubscript{32} Потому что упорство невежд убьет их, и беспечность глупцов погубит их,
\end{tcolorbox}
\begin{tcolorbox}
\textsubscript{33} а слушающий меня будет жить безопасно и спокойно, не страшась зла'.
\end{tcolorbox}
\subsection{CHAPTER 2}
\begin{tcolorbox}
\textsubscript{1} Сын мой! если ты примешь слова мои и сохранишь при себе заповеди мои,
\end{tcolorbox}
\begin{tcolorbox}
\textsubscript{2} так что ухо твое сделаешь внимательным к мудрости и наклонишь сердце твое к размышлению;
\end{tcolorbox}
\begin{tcolorbox}
\textsubscript{3} если будешь призывать знание и взывать к разуму;
\end{tcolorbox}
\begin{tcolorbox}
\textsubscript{4} если будешь искать его, как серебра, и отыскивать его, как сокровище,
\end{tcolorbox}
\begin{tcolorbox}
\textsubscript{5} то уразумеешь страх Господень и найдешь познание о Боге.
\end{tcolorbox}
\begin{tcolorbox}
\textsubscript{6} Ибо Господь дает мудрость; из уст Его--знание и разум;
\end{tcolorbox}
\begin{tcolorbox}
\textsubscript{7} Он сохраняет для праведных спасение; Он--щит для ходящих непорочно;
\end{tcolorbox}
\begin{tcolorbox}
\textsubscript{8} Он охраняет пути правды и оберегает стезю святых Своих.
\end{tcolorbox}
\begin{tcolorbox}
\textsubscript{9} Тогда ты уразумеешь правду и правосудие и прямоту, всякую добрую стезю.
\end{tcolorbox}
\begin{tcolorbox}
\textsubscript{10} Когда мудрость войдет в сердце твое, и знание будет приятно душе твоей,
\end{tcolorbox}
\begin{tcolorbox}
\textsubscript{11} тогда рассудительность будет оберегать тебя, разум будет охранять тебя,
\end{tcolorbox}
\begin{tcolorbox}
\textsubscript{12} дабы спасти тебя от пути злого, от человека, говорящего ложь,
\end{tcolorbox}
\begin{tcolorbox}
\textsubscript{13} от тех, которые оставляют стези прямые, чтобы ходить путями тьмы;
\end{tcolorbox}
\begin{tcolorbox}
\textsubscript{14} от тех, которые радуются, делая зло, восхищаются злым развратом,
\end{tcolorbox}
\begin{tcolorbox}
\textsubscript{15} которых пути кривы, и которые блуждают на стезях своих;
\end{tcolorbox}
\begin{tcolorbox}
\textsubscript{16} дабы спасти тебя от жены другого, от чужой, которая умягчает речи свои,
\end{tcolorbox}
\begin{tcolorbox}
\textsubscript{17} которая оставила руководителя юности своей и забыла завет Бога своего.
\end{tcolorbox}
\begin{tcolorbox}
\textsubscript{18} Дом ее ведет к смерти, и стези ее--к мертвецам;
\end{tcolorbox}
\begin{tcolorbox}
\textsubscript{19} никто из вошедших к ней не возвращается и не вступает на путь жизни.
\end{tcolorbox}
\begin{tcolorbox}
\textsubscript{20} Посему ходи путем добрых и держись стезей праведников,
\end{tcolorbox}
\begin{tcolorbox}
\textsubscript{21} потому что праведные будут жить на земле, и непорочные пребудут на ней;
\end{tcolorbox}
\begin{tcolorbox}
\textsubscript{22} а беззаконные будут истреблены с земли, и вероломные искоренены из нее.
\end{tcolorbox}
\subsection{CHAPTER 3}
\begin{tcolorbox}
\textsubscript{1} Сын мой! наставления моего не забывай, и заповеди мои да хранит сердце твое;
\end{tcolorbox}
\begin{tcolorbox}
\textsubscript{2} ибо долготы дней, лет жизни и мира они приложат тебе.
\end{tcolorbox}
\begin{tcolorbox}
\textsubscript{3} Милость и истина да не оставляют тебя: обвяжи ими шею твою, напиши их на скрижали сердца твоего,
\end{tcolorbox}
\begin{tcolorbox}
\textsubscript{4} и обретешь милость и благоволение в очах Бога и людей.
\end{tcolorbox}
\begin{tcolorbox}
\textsubscript{5} Надейся на Господа всем сердцем твоим, и не полагайся на разум твой.
\end{tcolorbox}
\begin{tcolorbox}
\textsubscript{6} Во всех путях твоих познавай Его, и Он направит стези твои.
\end{tcolorbox}
\begin{tcolorbox}
\textsubscript{7} Не будь мудрецом в глазах твоих; бойся Господа и удаляйся от зла:
\end{tcolorbox}
\begin{tcolorbox}
\textsubscript{8} это будет здравием для тела твоего и питанием для костей твоих.
\end{tcolorbox}
\begin{tcolorbox}
\textsubscript{9} Чти Господа от имения твоего и от начатков всех прибытков твоих,
\end{tcolorbox}
\begin{tcolorbox}
\textsubscript{10} и наполнятся житницы твои до избытка, и точила твои будут переливаться новым вином.
\end{tcolorbox}
\begin{tcolorbox}
\textsubscript{11} Наказания Господня, сын мой, не отвергай, и не тяготись обличением Его;
\end{tcolorbox}
\begin{tcolorbox}
\textsubscript{12} ибо кого любит Господь, того наказывает и благоволит к тому, как отец к сыну своему.
\end{tcolorbox}
\begin{tcolorbox}
\textsubscript{13} Блажен человек, который снискал мудрость, и человек, который приобрел разум, --
\end{tcolorbox}
\begin{tcolorbox}
\textsubscript{14} потому что приобретение ее лучше приобретения серебра, и прибыли от нее больше, нежели от золота:
\end{tcolorbox}
\begin{tcolorbox}
\textsubscript{15} она дороже драгоценных камней; и ничто из желаемого тобою не сравнится с нею.
\end{tcolorbox}
\begin{tcolorbox}
\textsubscript{16} Долгоденствие--в правой руке ее, а в левой у нее--богатство и слава;
\end{tcolorbox}
\begin{tcolorbox}
\textsubscript{17} пути ее--пути приятные, и все стези ее--мирные.
\end{tcolorbox}
\begin{tcolorbox}
\textsubscript{18} Она--древо жизни для тех, которые приобретают ее, --и блаженны, которые сохраняют ее!
\end{tcolorbox}
\begin{tcolorbox}
\textsubscript{19} Господь премудростью основал землю, небеса утвердил разумом;
\end{tcolorbox}
\begin{tcolorbox}
\textsubscript{20} Его премудростью разверзлись бездны, и облака кропят росою.
\end{tcolorbox}
\begin{tcolorbox}
\textsubscript{21} Сын мой! не упускай их из глаз твоих; храни здравомыслие и рассудительность,
\end{tcolorbox}
\begin{tcolorbox}
\textsubscript{22} и они будут жизнью для души твоей и украшением для шеи твоей.
\end{tcolorbox}
\begin{tcolorbox}
\textsubscript{23} Тогда безопасно пойдешь по пути твоему, и нога твоя не споткнется.
\end{tcolorbox}
\begin{tcolorbox}
\textsubscript{24} Когда ляжешь спать, --не будешь бояться; и когда уснешь, --сон твой приятен будет.
\end{tcolorbox}
\begin{tcolorbox}
\textsubscript{25} Не убоишься внезапного страха и пагубы от нечестивых, когда она придет;
\end{tcolorbox}
\begin{tcolorbox}
\textsubscript{26} потому что Господь будет упованием твоим и сохранит ногу твою от уловления.
\end{tcolorbox}
\begin{tcolorbox}
\textsubscript{27} Не отказывай в благодеянии нуждающемуся, когда рука твоя в силе сделать его.
\end{tcolorbox}
\begin{tcolorbox}
\textsubscript{28} Не говори другу твоему: 'пойди и приди опять, и завтра я дам', когда ты имеешь при себе.
\end{tcolorbox}
\begin{tcolorbox}
\textsubscript{29} Не замышляй против ближнего твоего зла, когда он без опасения живет с тобою.
\end{tcolorbox}
\begin{tcolorbox}
\textsubscript{30} Не ссорься с человеком без причины, когда он не сделал зла тебе.
\end{tcolorbox}
\begin{tcolorbox}
\textsubscript{31} Не соревнуй человеку, поступающему насильственно, и не избирай ни одного из путей его;
\end{tcolorbox}
\begin{tcolorbox}
\textsubscript{32} потому что мерзость пред Господом развратный, а с праведными у Него общение.
\end{tcolorbox}
\begin{tcolorbox}
\textsubscript{33} Проклятие Господне на доме нечестивого, а жилище благочестивых Он благословляет.
\end{tcolorbox}
\begin{tcolorbox}
\textsubscript{34} Если над кощунниками Он посмевается, то смиренным дает благодать.
\end{tcolorbox}
\begin{tcolorbox}
\textsubscript{35} Мудрые наследуют славу, а глупые--бесславие.
\end{tcolorbox}
\subsection{CHAPTER 4}
\begin{tcolorbox}
\textsubscript{1} Слушайте, дети, наставление отца, и внимайте, чтобы научиться разуму,
\end{tcolorbox}
\begin{tcolorbox}
\textsubscript{2} потому что я преподал вам доброе учение. Не оставляйте заповеди моей.
\end{tcolorbox}
\begin{tcolorbox}
\textsubscript{3} Ибо и я был сын у отца моего, нежно любимый и единственный у матери моей,
\end{tcolorbox}
\begin{tcolorbox}
\textsubscript{4} и он учил меня и говорил мне: да удержит сердце твое слова мои; храни заповеди мои, и живи.
\end{tcolorbox}
\begin{tcolorbox}
\textsubscript{5} Приобретай мудрость, приобретай разум: не забывай этого и не уклоняйся от слов уст моих.
\end{tcolorbox}
\begin{tcolorbox}
\textsubscript{6} Не оставляй ее, и она будет охранять тебя; люби ее, и она будет оберегать тебя.
\end{tcolorbox}
\begin{tcolorbox}
\textsubscript{7} Главное--мудрость: приобретай мудрость, и всем имением твоим приобретай разум.
\end{tcolorbox}
\begin{tcolorbox}
\textsubscript{8} Высоко цени ее, и она возвысит тебя; она прославит тебя, если ты прилепишься к ней;
\end{tcolorbox}
\begin{tcolorbox}
\textsubscript{9} возложит на голову твою прекрасный венок, доставит тебе великолепный венец.
\end{tcolorbox}
\begin{tcolorbox}
\textsubscript{10} Слушай, сын мой, и прими слова мои, --и умножатся тебе лета жизни.
\end{tcolorbox}
\begin{tcolorbox}
\textsubscript{11} Я указываю тебе путь мудрости, веду тебя по стезям прямым.
\end{tcolorbox}
\begin{tcolorbox}
\textsubscript{12} Когда пойдешь, не будет стеснен ход твой, и когда побежишь, не споткнешься.
\end{tcolorbox}
\begin{tcolorbox}
\textsubscript{13} Крепко держись наставления, не оставляй, храни его, потому что оно--жизнь твоя.
\end{tcolorbox}
\begin{tcolorbox}
\textsubscript{14} Не вступай на стезю нечестивых и не ходи по пути злых;
\end{tcolorbox}
\begin{tcolorbox}
\textsubscript{15} оставь его, не ходи по нему, уклонись от него и пройди мимо;
\end{tcolorbox}
\begin{tcolorbox}
\textsubscript{16} потому что они не заснут, если не сделают зла; пропадает сон у них, если они не доведут кого до падения;
\end{tcolorbox}
\begin{tcolorbox}
\textsubscript{17} ибо они едят хлеб беззакония и пьют вино хищения.
\end{tcolorbox}
\begin{tcolorbox}
\textsubscript{18} Стезя праведных--как светило лучезарное, которое более и более светлеет до полного дня.
\end{tcolorbox}
\begin{tcolorbox}
\textsubscript{19} Путь же беззаконных--как тьма; они не знают, обо что споткнутся.
\end{tcolorbox}
\begin{tcolorbox}
\textsubscript{20} Сын мой! словам моим внимай, и к речам моим приклони ухо твое;
\end{tcolorbox}
\begin{tcolorbox}
\textsubscript{21} да не отходят они от глаз твоих; храни их внутри сердца твоего:
\end{tcolorbox}
\begin{tcolorbox}
\textsubscript{22} потому что они жизнь для того, кто нашел их, и здравие для всего тела его.
\end{tcolorbox}
\begin{tcolorbox}
\textsubscript{23} Больше всего хранимого храни сердце твое, потому что из него источники жизни.
\end{tcolorbox}
\begin{tcolorbox}
\textsubscript{24} Отвергни от себя лживость уст, и лукавство языка удали от себя.
\end{tcolorbox}
\begin{tcolorbox}
\textsubscript{25} Глаза твои пусть прямо смотрят, и ресницы твои да направлены будут прямо пред тобою.
\end{tcolorbox}
\begin{tcolorbox}
\textsubscript{26} Обдумай стезю для ноги твоей, и все пути твои да будут тверды.
\end{tcolorbox}
\begin{tcolorbox}
\textsubscript{27} Не уклоняйся ни направо, ни налево; удали ногу твою от зла,
\end{tcolorbox}
\subsection{CHAPTER 5}
\begin{tcolorbox}
\textsubscript{1} Сын мой! внимай мудрости моей, и приклони ухо твое к разуму моему,
\end{tcolorbox}
\begin{tcolorbox}
\textsubscript{2} чтобы соблюсти рассудительность, и чтобы уста твои сохранили знание.
\end{tcolorbox}
\begin{tcolorbox}
\textsubscript{3} ибо мед источают уста чужой жены, и мягче елея речь ее;
\end{tcolorbox}
\begin{tcolorbox}
\textsubscript{4} но последствия от нее горьки, как полынь, остры, как меч обоюдоострый;
\end{tcolorbox}
\begin{tcolorbox}
\textsubscript{5} ноги ее нисходят к смерти, стопы ее достигают преисподней.
\end{tcolorbox}
\begin{tcolorbox}
\textsubscript{6} Если бы ты захотел постигнуть стезю жизни ее, то пути ее непостоянны, и ты не узнаешь их.
\end{tcolorbox}
\begin{tcolorbox}
\textsubscript{7} Итак, дети, слушайте меня и не отступайте от слов уст моих.
\end{tcolorbox}
\begin{tcolorbox}
\textsubscript{8} Держи дальше от нее путь твой и не подходи близко к дверям дома ее,
\end{tcolorbox}
\begin{tcolorbox}
\textsubscript{9} чтобы здоровья твоего не отдать другим и лет твоих мучителю;
\end{tcolorbox}
\begin{tcolorbox}
\textsubscript{10} чтобы не насыщались силою твоею чужие, и труды твои не были для чужого дома.
\end{tcolorbox}
\begin{tcolorbox}
\textsubscript{11} И ты будешь стонать после, когда плоть твоя и тело твое будут истощены, --
\end{tcolorbox}
\begin{tcolorbox}
\textsubscript{12} и скажешь: 'зачем я ненавидел наставление, и сердце мое пренебрегало обличением,
\end{tcolorbox}
\begin{tcolorbox}
\textsubscript{13} и я не слушал голоса учителей моих, не приклонял уха моего к наставникам моим:
\end{tcolorbox}
\begin{tcolorbox}
\textsubscript{14} едва не впал я во всякое зло среди собрания и общества!'
\end{tcolorbox}
\begin{tcolorbox}
\textsubscript{15} Пей воду из твоего водоема и текущую из твоего колодезя.
\end{tcolorbox}
\begin{tcolorbox}
\textsubscript{16} Пусть [не] разливаются источники твои по улице, потоки вод--по площадям;
\end{tcolorbox}
\begin{tcolorbox}
\textsubscript{17} пусть они будут принадлежать тебе одному, а не чужим с тобою.
\end{tcolorbox}
\begin{tcolorbox}
\textsubscript{18} Источник твой да будет благословен; и утешайся женою юности твоей,
\end{tcolorbox}
\begin{tcolorbox}
\textsubscript{19} любезною ланью и прекрасною серною: груди ее да упоявают тебя во всякое время, любовью ее услаждайся постоянно.
\end{tcolorbox}
\begin{tcolorbox}
\textsubscript{20} И для чего тебе, сын мой, увлекаться постороннею и обнимать груди чужой?
\end{tcolorbox}
\begin{tcolorbox}
\textsubscript{21} Ибо пред очами Господа пути человека, и Он измеряет все стези его.
\end{tcolorbox}
\begin{tcolorbox}
\textsubscript{22} Беззаконного уловляют собственные беззакония его, и в узах греха своего он содержится:
\end{tcolorbox}
\begin{tcolorbox}
\textsubscript{23} он умирает без наставления, и от множества безумия своего теряется.
\end{tcolorbox}
\subsection{CHAPTER 6}
\begin{tcolorbox}
\textsubscript{1} Сын мой! если ты поручился за ближнего твоего и дал руку твою за другого, --
\end{tcolorbox}
\begin{tcolorbox}
\textsubscript{2} ты опутал себя словами уст твоих, пойман словами уст твоих.
\end{tcolorbox}
\begin{tcolorbox}
\textsubscript{3} Сделай же, сын мой, вот что, и избавь себя, так как ты попался в руки ближнего твоего: пойди, пади к ногам и умоляй ближнего твоего;
\end{tcolorbox}
\begin{tcolorbox}
\textsubscript{4} не давай сна глазам твоим и дремания веждам твоим;
\end{tcolorbox}
\begin{tcolorbox}
\textsubscript{5} спасайся, как серна из руки и как птица из руки птицелова.
\end{tcolorbox}
\begin{tcolorbox}
\textsubscript{6} Пойди к муравью, ленивец, посмотри на действия его, и будь мудрым.
\end{tcolorbox}
\begin{tcolorbox}
\textsubscript{7} Нет у него ни начальника, ни приставника, ни повелителя;
\end{tcolorbox}
\begin{tcolorbox}
\textsubscript{8} но он заготовляет летом хлеб свой, собирает во время жатвы пищу свою.
\end{tcolorbox}
\begin{tcolorbox}
\textsubscript{9} Доколе ты, ленивец, будешь спать? когда ты встанешь от сна твоего?
\end{tcolorbox}
\begin{tcolorbox}
\textsubscript{10} Немного поспишь, немного подремлешь, немного, сложив руки, полежишь:
\end{tcolorbox}
\begin{tcolorbox}
\textsubscript{11} и придет, как прохожий, бедность твоя, и нужда твоя, как разбойник.
\end{tcolorbox}
\begin{tcolorbox}
\textsubscript{12} Человек лукавый, человек нечестивый ходит со лживыми устами,
\end{tcolorbox}
\begin{tcolorbox}
\textsubscript{13} мигает глазами своими, говорит ногами своими, дает знаки пальцами своими;
\end{tcolorbox}
\begin{tcolorbox}
\textsubscript{14} коварство в сердце его: он умышляет зло во всякое время, сеет раздоры.
\end{tcolorbox}
\begin{tcolorbox}
\textsubscript{15} Зато внезапно придет погибель его, вдруг будет разбит--без исцеления.
\end{tcolorbox}
\begin{tcolorbox}
\textsubscript{16} Вот шесть, что ненавидит Господь, даже семь, что мерзость душе Его:
\end{tcolorbox}
\begin{tcolorbox}
\textsubscript{17} глаза гордые, язык лживый и руки, проливающие кровь невинную,
\end{tcolorbox}
\begin{tcolorbox}
\textsubscript{18} сердце, кующее злые замыслы, ноги, быстро бегущие к злодейству,
\end{tcolorbox}
\begin{tcolorbox}
\textsubscript{19} лжесвидетель, наговаривающий ложь и сеющий раздор между братьями.
\end{tcolorbox}
\begin{tcolorbox}
\textsubscript{20} Сын мой! храни заповедь отца твоего и не отвергай наставления матери твоей;
\end{tcolorbox}
\begin{tcolorbox}
\textsubscript{21} навяжи их навсегда на сердце твое, обвяжи ими шею твою.
\end{tcolorbox}
\begin{tcolorbox}
\textsubscript{22} Когда ты пойдешь, они будут руководить тебя; когда ляжешь спать, будут охранять тебя; когда пробудишься, будут беседовать с тобою:
\end{tcolorbox}
\begin{tcolorbox}
\textsubscript{23} ибо заповедь есть светильник, и наставление--свет, и назидательные поучения--путь к жизни,
\end{tcolorbox}
\begin{tcolorbox}
\textsubscript{24} чтобы остерегать тебя от негодной женщины, от льстивого языка чужой.
\end{tcolorbox}
\begin{tcolorbox}
\textsubscript{25} Не пожелай красоты ее в сердце твоем, и да не увлечет она тебя ресницами своими;
\end{tcolorbox}
\begin{tcolorbox}
\textsubscript{26} потому что из-за жены блудной [обнищевают] до куска хлеба, а замужняя жена уловляет дорогую душу.
\end{tcolorbox}
\begin{tcolorbox}
\textsubscript{27} Может ли кто взять себе огонь в пазуху, чтобы не прогорело платье его?
\end{tcolorbox}
\begin{tcolorbox}
\textsubscript{28} Может ли кто ходить по горящим угольям, чтобы не обжечь ног своих?
\end{tcolorbox}
\begin{tcolorbox}
\textsubscript{29} То же бывает и с тем, кто входит к жене ближнего своего: кто прикоснется к ней, не останется без вины.
\end{tcolorbox}
\begin{tcolorbox}
\textsubscript{30} Не спускают вору, если он крадет, чтобы насытить душу свою, когда он голоден;
\end{tcolorbox}
\begin{tcolorbox}
\textsubscript{31} но, будучи пойман, он заплатит всемеро, отдаст все имущество дома своего.
\end{tcolorbox}
\begin{tcolorbox}
\textsubscript{32} Кто же прелюбодействует с женщиною, у того нет ума; тот губит душу свою, кто делает это:
\end{tcolorbox}
\begin{tcolorbox}
\textsubscript{33} побои и позор найдет он, и бесчестие его не изгладится,
\end{tcolorbox}
\begin{tcolorbox}
\textsubscript{34} потому что ревность--ярость мужа, и не пощадит он в день мщения,
\end{tcolorbox}
\begin{tcolorbox}
\textsubscript{35} не примет никакого выкупа и не удовольствуется, сколько бы ты ни умножал даров.
\end{tcolorbox}
\subsection{CHAPTER 7}
\begin{tcolorbox}
\textsubscript{1} Сын мой! храни слова мои и заповеди мои сокрой у себя.
\end{tcolorbox}
\begin{tcolorbox}
\textsubscript{2} Храни заповеди мои и живи, и учение мое, как зрачок глаз твоих.
\end{tcolorbox}
\begin{tcolorbox}
\textsubscript{3} Навяжи их на персты твои, напиши их на скрижали сердца твоего.
\end{tcolorbox}
\begin{tcolorbox}
\textsubscript{4} Скажи мудрости: 'Ты сестра моя!' и разум назови родным твоим,
\end{tcolorbox}
\begin{tcolorbox}
\textsubscript{5} чтобы они охраняли тебя от жены другого, от чужой, которая умягчает слова свои.
\end{tcolorbox}
\begin{tcolorbox}
\textsubscript{6} Вот, однажды смотрел я в окно дома моего, сквозь решетку мою,
\end{tcolorbox}
\begin{tcolorbox}
\textsubscript{7} и увидел среди неопытных, заметил между молодыми людьми неразумного юношу,
\end{tcolorbox}
\begin{tcolorbox}
\textsubscript{8} переходившего площадь близ угла ее и шедшего по дороге к дому ее,
\end{tcolorbox}
\begin{tcolorbox}
\textsubscript{9} в сумерки в вечер дня, в ночной темноте и во мраке.
\end{tcolorbox}
\begin{tcolorbox}
\textsubscript{10} И вот--навстречу к нему женщина, в наряде блудницы, с коварным сердцем,
\end{tcolorbox}
\begin{tcolorbox}
\textsubscript{11} шумливая и необузданная; ноги ее не живут в доме ее:
\end{tcolorbox}
\begin{tcolorbox}
\textsubscript{12} то на улице, то на площадях, и у каждого угла строит она ковы.
\end{tcolorbox}
\begin{tcolorbox}
\textsubscript{13} Она схватила его, целовала его, и с бесстыдным лицом говорила ему:
\end{tcolorbox}
\begin{tcolorbox}
\textsubscript{14} 'мирная жертва у меня: сегодня я совершила обеты мои;
\end{tcolorbox}
\begin{tcolorbox}
\textsubscript{15} поэтому и вышла навстречу тебе, чтобы отыскать тебя, и--нашла тебя;
\end{tcolorbox}
\begin{tcolorbox}
\textsubscript{16} коврами я убрала постель мою, разноцветными тканями Египетскими;
\end{tcolorbox}
\begin{tcolorbox}
\textsubscript{17} спальню мою надушила смирною, алоем и корицею;
\end{tcolorbox}
\begin{tcolorbox}
\textsubscript{18} зайди, будем упиваться нежностями до утра, насладимся любовью,
\end{tcolorbox}
\begin{tcolorbox}
\textsubscript{19} потому что мужа нет дома: он отправился в дальнюю дорогу;
\end{tcolorbox}
\begin{tcolorbox}
\textsubscript{20} кошелек серебра взял с собою; придет домой ко дню полнолуния'.
\end{tcolorbox}
\begin{tcolorbox}
\textsubscript{21} Множеством ласковых слов она увлекла его, мягкостью уст своих овладела им.
\end{tcolorbox}
\begin{tcolorbox}
\textsubscript{22} Тотчас он пошел за нею, как вол идет на убой, и как олень--на выстрел,
\end{tcolorbox}
\begin{tcolorbox}
\textsubscript{23} доколе стрела не пронзит печени его; как птичка кидается в силки, и не знает, что они--на погибель ее.
\end{tcolorbox}
\begin{tcolorbox}
\textsubscript{24} Итак, дети, слушайте меня и внимайте словам уст моих.
\end{tcolorbox}
\begin{tcolorbox}
\textsubscript{25} Да не уклоняется сердце твое на пути ее, не блуждай по стезям ее,
\end{tcolorbox}
\begin{tcolorbox}
\textsubscript{26} потому что многих повергла она ранеными, и много сильных убиты ею:
\end{tcolorbox}
\begin{tcolorbox}
\textsubscript{27} дом ее--пути в преисподнюю, нисходящие во внутренние жилища смерти.
\end{tcolorbox}
\subsection{CHAPTER 8}
\begin{tcolorbox}
\textsubscript{1} Не премудрость ли взывает? и не разум ли возвышает голос свой?
\end{tcolorbox}
\begin{tcolorbox}
\textsubscript{2} Она становится на возвышенных местах, при дороге, на распутиях;
\end{tcolorbox}
\begin{tcolorbox}
\textsubscript{3} она взывает у ворот при входе в город, при входе в двери:
\end{tcolorbox}
\begin{tcolorbox}
\textsubscript{4} 'к вам, люди, взываю я, и к сынам человеческим голос мой!
\end{tcolorbox}
\begin{tcolorbox}
\textsubscript{5} Научитесь, неразумные, благоразумию, и глупые--разуму.
\end{tcolorbox}
\begin{tcolorbox}
\textsubscript{6} Слушайте, потому что я буду говорить важное, и изречение уст моих--правда;
\end{tcolorbox}
\begin{tcolorbox}
\textsubscript{7} ибо истину произнесет язык мой, и нечестие--мерзость для уст моих;
\end{tcolorbox}
\begin{tcolorbox}
\textsubscript{8} все слова уст моих справедливы; нет в них коварства и лукавства;
\end{tcolorbox}
\begin{tcolorbox}
\textsubscript{9} все они ясны для разумного и справедливы для приобретших знание.
\end{tcolorbox}
\begin{tcolorbox}
\textsubscript{10} Примите учение мое, а не серебро; лучше знание, нежели отборное золото;
\end{tcolorbox}
\begin{tcolorbox}
\textsubscript{11} потому что мудрость лучше жемчуга, и ничто из желаемого не сравнится с нею.
\end{tcolorbox}
\begin{tcolorbox}
\textsubscript{12} Я, премудрость, обитаю с разумом и ищу рассудительного знания.
\end{tcolorbox}
\begin{tcolorbox}
\textsubscript{13} Страх Господень--ненавидеть зло; гордость и высокомерие и злой путь и коварные уста я ненавижу.
\end{tcolorbox}
\begin{tcolorbox}
\textsubscript{14} У меня совет и правда; я разум, у меня сила.
\end{tcolorbox}
\begin{tcolorbox}
\textsubscript{15} Мною цари царствуют и повелители узаконяют правду;
\end{tcolorbox}
\begin{tcolorbox}
\textsubscript{16} мною начальствуют начальники и вельможи и все судьи земли.
\end{tcolorbox}
\begin{tcolorbox}
\textsubscript{17} Любящих меня я люблю, и ищущие меня найдут меня;
\end{tcolorbox}
\begin{tcolorbox}
\textsubscript{18} богатство и слава у меня, сокровище непогибающее и правда;
\end{tcolorbox}
\begin{tcolorbox}
\textsubscript{19} плоды мои лучше золота, и золота самого чистого, и пользы от меня больше, нежели от отборного серебра.
\end{tcolorbox}
\begin{tcolorbox}
\textsubscript{20} Я хожу по пути правды, по стезям правосудия,
\end{tcolorbox}
\begin{tcolorbox}
\textsubscript{21} чтобы доставить любящим меня существенное благо, и сокровищницы их я наполняю.
\end{tcolorbox}
\begin{tcolorbox}
\textsubscript{22} Господь имел меня началом пути Своего, прежде созданий Своих, искони;
\end{tcolorbox}
\begin{tcolorbox}
\textsubscript{23} от века я помазана, от начала, прежде бытия земли.
\end{tcolorbox}
\begin{tcolorbox}
\textsubscript{24} Я родилась, когда еще не существовали бездны, когда еще не было источников, обильных водою.
\end{tcolorbox}
\begin{tcolorbox}
\textsubscript{25} Я родилась прежде, нежели водружены были горы, прежде холмов,
\end{tcolorbox}
\begin{tcolorbox}
\textsubscript{26} когда еще Он не сотворил ни земли, ни полей, ни начальных пылинок вселенной.
\end{tcolorbox}
\begin{tcolorbox}
\textsubscript{27} Когда Он уготовлял небеса, [я была] там. Когда Он проводил круговую черту по лицу бездны,
\end{tcolorbox}
\begin{tcolorbox}
\textsubscript{28} когда утверждал вверху облака, когда укреплял источники бездны,
\end{tcolorbox}
\begin{tcolorbox}
\textsubscript{29} когда давал морю устав, чтобы воды не переступали пределов его, когда полагал основания земли:
\end{tcolorbox}
\begin{tcolorbox}
\textsubscript{30} тогда я была при Нем художницею, и была радостью всякий день, веселясь пред лицем Его во все время,
\end{tcolorbox}
\begin{tcolorbox}
\textsubscript{31} веселясь на земном кругу Его, и радость моя [была] с сынами человеческими.
\end{tcolorbox}
\begin{tcolorbox}
\textsubscript{32} Итак, дети, послушайте меня; и блаженны те, которые хранят пути мои!
\end{tcolorbox}
\begin{tcolorbox}
\textsubscript{33} Послушайте наставления и будьте мудры, и не отступайте [от] [него].
\end{tcolorbox}
\begin{tcolorbox}
\textsubscript{34} Блажен человек, который слушает меня, бодрствуя каждый день у ворот моих и стоя на страже у дверей моих!
\end{tcolorbox}
\begin{tcolorbox}
\textsubscript{35} потому что, кто нашел меня, тот нашел жизнь, и получит благодать от Господа;
\end{tcolorbox}
\begin{tcolorbox}
\textsubscript{36} а согрешающий против меня наносит вред душе своей: все ненавидящие меня любят смерть'.
\end{tcolorbox}
\subsection{CHAPTER 9}
\begin{tcolorbox}
\textsubscript{1} Премудрость построила себе дом, вытесала семь столбов его,
\end{tcolorbox}
\begin{tcolorbox}
\textsubscript{2} заколола жертву, растворила вино свое и приготовила у себя трапезу;
\end{tcolorbox}
\begin{tcolorbox}
\textsubscript{3} послала слуг своих провозгласить с возвышенностей городских:
\end{tcolorbox}
\begin{tcolorbox}
\textsubscript{4} 'кто неразумен, обратись сюда!' И скудоумному она сказала:
\end{tcolorbox}
\begin{tcolorbox}
\textsubscript{5} 'идите, ешьте хлеб мой и пейте вино, мною растворенное;
\end{tcolorbox}
\begin{tcolorbox}
\textsubscript{6} оставьте неразумие, и живите, и ходите путем разума'.
\end{tcolorbox}
\begin{tcolorbox}
\textsubscript{7} Поучающий кощунника наживет себе бесславие, и обличающий нечестивого--пятно себе.
\end{tcolorbox}
\begin{tcolorbox}
\textsubscript{8} Не обличай кощунника, чтобы он не возненавидел тебя; обличай мудрого, и он возлюбит тебя;
\end{tcolorbox}
\begin{tcolorbox}
\textsubscript{9} дай [наставление] мудрому, и он будет еще мудрее; научи правдивого, и он приумножит знание.
\end{tcolorbox}
\begin{tcolorbox}
\textsubscript{10} Начало мудрости--страх Господень, и познание Святаго--разум;
\end{tcolorbox}
\begin{tcolorbox}
\textsubscript{11} потому что чрез меня умножатся дни твои, и прибавится тебе лет жизни.
\end{tcolorbox}
\begin{tcolorbox}
\textsubscript{12} если ты мудр, то мудр для себя; и если буен, то один потерпишь.
\end{tcolorbox}
\begin{tcolorbox}
\textsubscript{13} Женщина безрассудная, шумливая, глупая и ничего не знающая
\end{tcolorbox}
\begin{tcolorbox}
\textsubscript{14} садится у дверей дома своего на стуле, на возвышенных местах города,
\end{tcolorbox}
\begin{tcolorbox}
\textsubscript{15} чтобы звать проходящих дорогою, идущих прямо своими путями:
\end{tcolorbox}
\begin{tcolorbox}
\textsubscript{16} 'кто глуп, обратись сюда!' и скудоумному сказала она:
\end{tcolorbox}
\begin{tcolorbox}
\textsubscript{17} 'воды краденые сладки, и утаенный хлеб приятен'.
\end{tcolorbox}
\begin{tcolorbox}
\textsubscript{18} И он не знает, что мертвецы там, и что в глубине преисподней зазванные ею.
\end{tcolorbox}
\subsection{CHAPTER 10}
\begin{tcolorbox}
\textsubscript{1} Притчи Соломона. Сын мудрый радует отца, а сын глупый--огорчение для его матери.
\end{tcolorbox}
\begin{tcolorbox}
\textsubscript{2} Не доставляют пользы сокровища неправедные, правда же избавляет от смерти.
\end{tcolorbox}
\begin{tcolorbox}
\textsubscript{3} Не допустит Господь терпеть голод душе праведного, стяжание же нечестивых исторгнет.
\end{tcolorbox}
\begin{tcolorbox}
\textsubscript{4} Ленивая рука делает бедным, а рука прилежных обогащает.
\end{tcolorbox}
\begin{tcolorbox}
\textsubscript{5} Собирающий во время лета--сын разумный, спящий же во время жатвы--сын беспутный.
\end{tcolorbox}
\begin{tcolorbox}
\textsubscript{6} Благословения--на голове праведника, уста же беззаконных заградит насилие.
\end{tcolorbox}
\begin{tcolorbox}
\textsubscript{7} Память праведника пребудет благословенна, а имя нечестивых омерзеет.
\end{tcolorbox}
\begin{tcolorbox}
\textsubscript{8} Мудрый сердцем принимает заповеди, а глупый устами преткнется.
\end{tcolorbox}
\begin{tcolorbox}
\textsubscript{9} Кто ходит в непорочности, тот ходит безопасно; а кто превращает пути свои, тот будет наказан.
\end{tcolorbox}
\begin{tcolorbox}
\textsubscript{10} Кто мигает глазами, тот причиняет досаду, а глупый устами преткнется.
\end{tcolorbox}
\begin{tcolorbox}
\textsubscript{11} Уста праведника--источник жизни, уста же беззаконных заградит насилие.
\end{tcolorbox}
\begin{tcolorbox}
\textsubscript{12} Ненависть возбуждает раздоры, но любовь покрывает все грехи.
\end{tcolorbox}
\begin{tcolorbox}
\textsubscript{13} В устах разумного находится мудрость, но на теле глупого--розга.
\end{tcolorbox}
\begin{tcolorbox}
\textsubscript{14} Мудрые сберегают знание, но уста глупого--близкая погибель.
\end{tcolorbox}
\begin{tcolorbox}
\textsubscript{15} Имущество богатого--крепкий город его, беда для бедных--скудость их.
\end{tcolorbox}
\begin{tcolorbox}
\textsubscript{16} Труды праведного--к жизни, успех нечестивого--ко греху.
\end{tcolorbox}
\begin{tcolorbox}
\textsubscript{17} Кто хранит наставление, тот на пути к жизни; а отвергающий обличение--блуждает.
\end{tcolorbox}
\begin{tcolorbox}
\textsubscript{18} Кто скрывает ненависть, у того уста лживые; и кто разглашает клевету, тот глуп.
\end{tcolorbox}
\begin{tcolorbox}
\textsubscript{19} При многословии не миновать греха, а сдерживающий уста свои--разумен.
\end{tcolorbox}
\begin{tcolorbox}
\textsubscript{20} Отборное серебро--язык праведного, сердце же нечестивых--ничтожество.
\end{tcolorbox}
\begin{tcolorbox}
\textsubscript{21} Уста праведного пасут многих, а глупые умирают от недостатка разума.
\end{tcolorbox}
\begin{tcolorbox}
\textsubscript{22} Благословение Господне--оно обогащает и печали с собою не приносит.
\end{tcolorbox}
\begin{tcolorbox}
\textsubscript{23} Для глупого преступное деяние как бы забава, а человеку разумному свойственна мудрость.
\end{tcolorbox}
\begin{tcolorbox}
\textsubscript{24} Чего страшится нечестивый, то и постигнет его, а желание праведников исполнится.
\end{tcolorbox}
\begin{tcolorbox}
\textsubscript{25} Как проносится вихрь, [так] нет более нечестивого; а праведник--на вечном основании.
\end{tcolorbox}
\begin{tcolorbox}
\textsubscript{26} Что уксус для зубов и дым для глаз, то ленивый для посылающих его.
\end{tcolorbox}
\begin{tcolorbox}
\textsubscript{27} Страх Господень прибавляет дней, лета же нечестивых сократятся.
\end{tcolorbox}
\begin{tcolorbox}
\textsubscript{28} Ожидание праведников--радость, а надежда нечестивых погибнет.
\end{tcolorbox}
\begin{tcolorbox}
\textsubscript{29} Путь Господень--твердыня для непорочного и страх для делающих беззаконие.
\end{tcolorbox}
\begin{tcolorbox}
\textsubscript{30} Праведник во веки не поколеблется, нечестивые же не поживут на земле.
\end{tcolorbox}
\begin{tcolorbox}
\textsubscript{31} Уста праведника источают мудрость, а язык зловредный отсечется.
\end{tcolorbox}
\begin{tcolorbox}
\textsubscript{32} Уста праведного знают благоприятное, а уста нечестивых--развращенное.
\end{tcolorbox}
\subsection{CHAPTER 11}
\begin{tcolorbox}
\textsubscript{1} Неверные весы--мерзость пред Господом, но правильный вес угоден Ему.
\end{tcolorbox}
\begin{tcolorbox}
\textsubscript{2} Придет гордость, придет и посрамление; но со смиренными--мудрость.
\end{tcolorbox}
\begin{tcolorbox}
\textsubscript{3} Непорочность прямодушных будет руководить их, а лукавство коварных погубит их.
\end{tcolorbox}
\begin{tcolorbox}
\textsubscript{4} Не поможет богатство в день гнева, правда же спасет от смерти.
\end{tcolorbox}
\begin{tcolorbox}
\textsubscript{5} Правда непорочного уравнивает путь его, а нечестивый падет от нечестия своего.
\end{tcolorbox}
\begin{tcolorbox}
\textsubscript{6} Правда прямодушных спасет их, а беззаконники будут уловлены беззаконием своим.
\end{tcolorbox}
\begin{tcolorbox}
\textsubscript{7} Со смертью человека нечестивого исчезает надежда, и ожидание беззаконных погибает.
\end{tcolorbox}
\begin{tcolorbox}
\textsubscript{8} Праведник спасается от беды, а вместо него попадает [в нее] нечестивый.
\end{tcolorbox}
\begin{tcolorbox}
\textsubscript{9} Устами лицемер губит ближнего своего, но праведники прозорливостью спасаются.
\end{tcolorbox}
\begin{tcolorbox}
\textsubscript{10} При благоденствии праведников веселится город, и при погибели нечестивых [бывает] торжество.
\end{tcolorbox}
\begin{tcolorbox}
\textsubscript{11} Благословением праведных возвышается город, а устами нечестивых разрушается.
\end{tcolorbox}
\begin{tcolorbox}
\textsubscript{12} Скудоумный высказывает презрение к ближнему своему; но разумный человек молчит.
\end{tcolorbox}
\begin{tcolorbox}
\textsubscript{13} Кто ходит переносчиком, тот открывает тайну; но верный человек таит дело.
\end{tcolorbox}
\begin{tcolorbox}
\textsubscript{14} При недостатке попечения падает народ, а при многих советниках благоденствует.
\end{tcolorbox}
\begin{tcolorbox}
\textsubscript{15} Зло причиняет себе, кто ручается за постороннего; а кто ненавидит ручательство, тот безопасен.
\end{tcolorbox}
\begin{tcolorbox}
\textsubscript{16} Благонравная жена приобретает славу, а трудолюбивые приобретают богатство.
\end{tcolorbox}
\begin{tcolorbox}
\textsubscript{17} Человек милосердый благотворит душе своей, а жестокосердый разрушает плоть свою.
\end{tcolorbox}
\begin{tcolorbox}
\textsubscript{18} Нечестивый делает дело ненадежное, а сеющему правду--награда верная.
\end{tcolorbox}
\begin{tcolorbox}
\textsubscript{19} Праведность [ведет] к жизни, а стремящийся к злу [стремится] к смерти своей.
\end{tcolorbox}
\begin{tcolorbox}
\textsubscript{20} Мерзость пред Господом--коварные сердцем; но благоугодны Ему непорочные в пути.
\end{tcolorbox}
\begin{tcolorbox}
\textsubscript{21} Можно поручиться, что порочный не останется ненаказанным; семя же праведных спасется.
\end{tcolorbox}
\begin{tcolorbox}
\textsubscript{22} Что золотое кольцо в носу у свиньи, то женщина красивая и--безрассудная.
\end{tcolorbox}
\begin{tcolorbox}
\textsubscript{23} Желание праведных [есть] одно добро, ожидание нечестивых--гнев.
\end{tcolorbox}
\begin{tcolorbox}
\textsubscript{24} Иной сыплет щедро, и [ему] еще прибавляется; а другой сверх меры бережлив, и однако же беднеет.
\end{tcolorbox}
\begin{tcolorbox}
\textsubscript{25} Благотворительная душа будет насыщена, и кто напояет [других], тот и сам напоен будет.
\end{tcolorbox}
\begin{tcolorbox}
\textsubscript{26} Кто удерживает у себя хлеб, того клянет народ; а на голове продающего--благословение.
\end{tcolorbox}
\begin{tcolorbox}
\textsubscript{27} Кто стремится к добру, тот ищет благоволения; а кто ищет зла, к тому оно и приходит.
\end{tcolorbox}
\begin{tcolorbox}
\textsubscript{28} Надеющийся на богатство свое упадет; а праведники, как лист, будут зеленеть.
\end{tcolorbox}
\begin{tcolorbox}
\textsubscript{29} Расстроивающий дом свой получит в удел ветер, и глупый будет рабом мудрого сердцем.
\end{tcolorbox}
\begin{tcolorbox}
\textsubscript{30} Плод праведника--древо жизни, и мудрый привлекает души.
\end{tcolorbox}
\begin{tcolorbox}
\textsubscript{31} Так праведнику воздается на земле, тем паче нечестивому и грешнику.
\end{tcolorbox}
\subsection{CHAPTER 12}
\begin{tcolorbox}
\textsubscript{1} Кто любит наставление, тот любит знание; а кто ненавидит обличение, тот невежда.
\end{tcolorbox}
\begin{tcolorbox}
\textsubscript{2} Добрый приобретает благоволение от Господа; а человека коварного Он осудит.
\end{tcolorbox}
\begin{tcolorbox}
\textsubscript{3} Не утвердит себя человек беззаконием; корень же праведников неподвижен.
\end{tcolorbox}
\begin{tcolorbox}
\textsubscript{4} Добродетельная жена--венец для мужа своего; а позорная--как гниль в костях его.
\end{tcolorbox}
\begin{tcolorbox}
\textsubscript{5} Промышления праведных--правда, а замыслы нечестивых--коварство.
\end{tcolorbox}
\begin{tcolorbox}
\textsubscript{6} Речи нечестивых--засада для пролития крови, уста же праведных спасают их.
\end{tcolorbox}
\begin{tcolorbox}
\textsubscript{7} Коснись нечестивых несчастие--и нет их, а дом праведных стоит.
\end{tcolorbox}
\begin{tcolorbox}
\textsubscript{8} Хвалят человека по мере разума его, а развращенный сердцем будет в презрении.
\end{tcolorbox}
\begin{tcolorbox}
\textsubscript{9} Лучше простой, но работающий на себя, нежели выдающий себя за знатного, но нуждающийся в хлебе.
\end{tcolorbox}
\begin{tcolorbox}
\textsubscript{10} Праведный печется и о жизни скота своего, сердце же нечестивых жестоко.
\end{tcolorbox}
\begin{tcolorbox}
\textsubscript{11} Кто возделывает землю свою, тот будет насыщаться хлебом; а кто идет по следам празднолюбцев, тот скудоумен.
\end{tcolorbox}
\begin{tcolorbox}
\textsubscript{12} Нечестивый желает уловить в сеть зла; но корень праведных тверд.
\end{tcolorbox}
\begin{tcolorbox}
\textsubscript{13} Нечестивый уловляется грехами уст своих; но праведник выйдет из беды.
\end{tcolorbox}
\begin{tcolorbox}
\textsubscript{14} От плода уст [своих] человек насыщается добром, и воздаяние человеку--по делам рук его.
\end{tcolorbox}
\begin{tcolorbox}
\textsubscript{15} Путь глупого прямой в его глазах; но кто слушает совета, тот мудр.
\end{tcolorbox}
\begin{tcolorbox}
\textsubscript{16} У глупого тотчас же выкажется гнев его, а благоразумный скрывает оскорбление.
\end{tcolorbox}
\begin{tcolorbox}
\textsubscript{17} Кто говорит то, что знает, тот говорит правду; а у свидетеля ложного--обман.
\end{tcolorbox}
\begin{tcolorbox}
\textsubscript{18} Иной пустослов уязвляет как мечом, а язык мудрых--врачует.
\end{tcolorbox}
\begin{tcolorbox}
\textsubscript{19} Уста правдивые вечно пребывают, а лживый язык--только на мгновение.
\end{tcolorbox}
\begin{tcolorbox}
\textsubscript{20} Коварство--в сердце злоумышленников, радость--у миротворцев.
\end{tcolorbox}
\begin{tcolorbox}
\textsubscript{21} Не приключится праведнику никакого зла, нечестивые же будут преисполнены зол.
\end{tcolorbox}
\begin{tcolorbox}
\textsubscript{22} Мерзость пред Господом--уста лживые, а говорящие истину благоугодны Ему.
\end{tcolorbox}
\begin{tcolorbox}
\textsubscript{23} Человек рассудительный скрывает знание, а сердце глупых высказывает глупость.
\end{tcolorbox}
\begin{tcolorbox}
\textsubscript{24} Рука прилежных будет господствовать, а ленивая будет под данью.
\end{tcolorbox}
\begin{tcolorbox}
\textsubscript{25} Тоска на сердце человека подавляет его, а доброе слово развеселяет его.
\end{tcolorbox}
\begin{tcolorbox}
\textsubscript{26} Праведник указывает ближнему своему путь, а путь нечестивых вводит их в заблуждение.
\end{tcolorbox}
\begin{tcolorbox}
\textsubscript{27} Ленивый не жарит своей дичи; а имущество человека прилежного многоценно.
\end{tcolorbox}
\begin{tcolorbox}
\textsubscript{28} На пути правды--жизнь, и на стезе ее нет смерти.
\end{tcolorbox}
\subsection{CHAPTER 13}
\begin{tcolorbox}
\textsubscript{1} Мудрый сын [слушает] наставление отца, а буйный не слушает обличения.
\end{tcolorbox}
\begin{tcolorbox}
\textsubscript{2} От плода уст [своих] человек вкусит добро, душа же законопреступников--зло.
\end{tcolorbox}
\begin{tcolorbox}
\textsubscript{3} Кто хранит уста свои, тот бережет душу свою; а кто широко раскрывает свой рот, тому беда.
\end{tcolorbox}
\begin{tcolorbox}
\textsubscript{4} Душа ленивого желает, но тщетно; а душа прилежных насытится.
\end{tcolorbox}
\begin{tcolorbox}
\textsubscript{5} Праведник ненавидит ложное слово, а нечестивый срамит и бесчестит [себя].
\end{tcolorbox}
\begin{tcolorbox}
\textsubscript{6} Правда хранит непорочного в пути, а нечестие губит грешника.
\end{tcolorbox}
\begin{tcolorbox}
\textsubscript{7} Иной выдает себя за богатого, а у него ничего нет; другой выдает себя за бедного, а у него богатства много.
\end{tcolorbox}
\begin{tcolorbox}
\textsubscript{8} Богатством своим человек выкупает жизнь [свою], а бедный и угрозы не слышит.
\end{tcolorbox}
\begin{tcolorbox}
\textsubscript{9} Свет праведных весело горит, светильник же нечестивых угасает.
\end{tcolorbox}
\begin{tcolorbox}
\textsubscript{10} От высокомерия происходит раздор, а у советующихся--мудрость.
\end{tcolorbox}
\begin{tcolorbox}
\textsubscript{11} Богатство от суетности истощается, а собирающий трудами умножает его.
\end{tcolorbox}
\begin{tcolorbox}
\textsubscript{12} Надежда, долго не сбывающаяся, томит сердце, а исполнившееся желание--[как] древо жизни.
\end{tcolorbox}
\begin{tcolorbox}
\textsubscript{13} Кто пренебрегает словом, тот причиняет вред себе; а кто боится заповеди, тому воздается.
\end{tcolorbox}
\begin{tcolorbox}
\textsubscript{14} Учение мудрого--источник жизни, удаляющий от сетей смерти.
\end{tcolorbox}
\begin{tcolorbox}
\textsubscript{15} Добрый разум доставляет приятность, путь же беззаконных жесток.
\end{tcolorbox}
\begin{tcolorbox}
\textsubscript{16} Всякий благоразумный действует с знанием, а глупый выставляет напоказ глупость.
\end{tcolorbox}
\begin{tcolorbox}
\textsubscript{17} Худой посол попадает в беду, а верный посланник--спасение.
\end{tcolorbox}
\begin{tcolorbox}
\textsubscript{18} Нищета и посрамление отвергающему учение; а кто соблюдает наставление, будет в чести.
\end{tcolorbox}
\begin{tcolorbox}
\textsubscript{19} Желание исполнившееся--приятно для души; но несносно для глупых уклоняться от зла.
\end{tcolorbox}
\begin{tcolorbox}
\textsubscript{20} Общающийся с мудрыми будет мудр, а кто дружит с глупыми, развратится.
\end{tcolorbox}
\begin{tcolorbox}
\textsubscript{21} Грешников преследует зло, а праведникам воздается добром.
\end{tcolorbox}
\begin{tcolorbox}
\textsubscript{22} Добрый оставляет наследство [и] внукам, а богатство грешника сберегается для праведного.
\end{tcolorbox}
\begin{tcolorbox}
\textsubscript{23} Много хлеба [бывает] и на ниве бедных; но некоторые гибнут от беспорядка.
\end{tcolorbox}
\begin{tcolorbox}
\textsubscript{24} Кто жалеет розги своей, тот ненавидит сына; а кто любит, тот с детства наказывает его.
\end{tcolorbox}
\begin{tcolorbox}
\textsubscript{25} Праведник ест до сытости, а чрево беззаконных терпит лишение.
\end{tcolorbox}
\subsection{CHAPTER 14}
\begin{tcolorbox}
\textsubscript{1} Мудрая жена устроит дом свой, а глупая разрушит его своими руками.
\end{tcolorbox}
\begin{tcolorbox}
\textsubscript{2} Идущий прямым путем боится Господа; но чьи пути кривы, тот небрежет о Нем.
\end{tcolorbox}
\begin{tcolorbox}
\textsubscript{3} В устах глупого--бич гордости; уста же мудрых охраняют их.
\end{tcolorbox}
\begin{tcolorbox}
\textsubscript{4} Где нет волов, [там] ясли пусты; а много прибыли от силы волов.
\end{tcolorbox}
\begin{tcolorbox}
\textsubscript{5} Верный свидетель не лжет, а свидетель ложный наговорит много лжи.
\end{tcolorbox}
\begin{tcolorbox}
\textsubscript{6} Распутный ищет мудрости, и не находит; а для разумного знание легко.
\end{tcolorbox}
\begin{tcolorbox}
\textsubscript{7} Отойди от человека глупого, у которого ты не замечаешь разумных уст.
\end{tcolorbox}
\begin{tcolorbox}
\textsubscript{8} Мудрость разумного--знание пути своего, глупость же безрассудных--заблуждение.
\end{tcolorbox}
\begin{tcolorbox}
\textsubscript{9} Глупые смеются над грехом, а посреди праведных--благоволение.
\end{tcolorbox}
\begin{tcolorbox}
\textsubscript{10} Сердце знает горе души своей, и в радость его не вмешается чужой.
\end{tcolorbox}
\begin{tcolorbox}
\textsubscript{11} Дом беззаконных разорится, а жилище праведных процветет.
\end{tcolorbox}
\begin{tcolorbox}
\textsubscript{12} Есть пути, которые кажутся человеку прямыми; но конец их--путь к смерти.
\end{tcolorbox}
\begin{tcolorbox}
\textsubscript{13} И при смехе [иногда] болит сердце, и концом радости бывает печаль.
\end{tcolorbox}
\begin{tcolorbox}
\textsubscript{14} Человек с развращенным сердцем насытится от путей своих, и добрый--от своих.
\end{tcolorbox}
\begin{tcolorbox}
\textsubscript{15} Глупый верит всякому слову, благоразумный же внимателен к путям своим.
\end{tcolorbox}
\begin{tcolorbox}
\textsubscript{16} Мудрый боится и удаляется от зла, а глупый раздражителен и самонадеян.
\end{tcolorbox}
\begin{tcolorbox}
\textsubscript{17} Вспыльчивый может сделать глупость; но человек, умышленно делающий зло, ненавистен.
\end{tcolorbox}
\begin{tcolorbox}
\textsubscript{18} Невежды получают в удел себе глупость, а благоразумные увенчаются знанием.
\end{tcolorbox}
\begin{tcolorbox}
\textsubscript{19} Преклонятся злые пред добрыми и нечестивые--у ворот праведника.
\end{tcolorbox}
\begin{tcolorbox}
\textsubscript{20} Бедный ненавидим бывает даже близким своим, а у богатого много друзей.
\end{tcolorbox}
\begin{tcolorbox}
\textsubscript{21} Кто презирает ближнего своего, тот грешит; а кто милосерд к бедным, тот блажен.
\end{tcolorbox}
\begin{tcolorbox}
\textsubscript{22} Не заблуждаются ли умышляющие зло? но милость и верность у благомыслящих.
\end{tcolorbox}
\begin{tcolorbox}
\textsubscript{23} От всякого труда есть прибыль, а от пустословия только ущерб.
\end{tcolorbox}
\begin{tcolorbox}
\textsubscript{24} Венец мудрых--богатство их, а глупость невежд глупость [и] [есть].
\end{tcolorbox}
\begin{tcolorbox}
\textsubscript{25} Верный свидетель спасает души, а лживый наговорит много лжи.
\end{tcolorbox}
\begin{tcolorbox}
\textsubscript{26} В страхе пред Господом--надежда твердая, и сынам Своим Он прибежище.
\end{tcolorbox}
\begin{tcolorbox}
\textsubscript{27} Страх Господень--источник жизни, удаляющий от сетей смерти.
\end{tcolorbox}
\begin{tcolorbox}
\textsubscript{28} Во множестве народа--величие царя, а при малолюдстве народа беда государю.
\end{tcolorbox}
\begin{tcolorbox}
\textsubscript{29} У терпеливого человека много разума, а раздражительный выказывает глупость.
\end{tcolorbox}
\begin{tcolorbox}
\textsubscript{30} Кроткое сердце--жизнь для тела, а зависть--гниль для костей.
\end{tcolorbox}
\begin{tcolorbox}
\textsubscript{31} Кто теснит бедного, тот хулит Творца его; чтущий же Его благотворит нуждающемуся.
\end{tcolorbox}
\begin{tcolorbox}
\textsubscript{32} За зло свое нечестивый будет отвергнут, а праведный и при смерти своей имеет надежду.
\end{tcolorbox}
\begin{tcolorbox}
\textsubscript{33} Мудрость почиет в сердце разумного, и среди глупых дает знать о себе.
\end{tcolorbox}
\begin{tcolorbox}
\textsubscript{34} Праведность возвышает народ, а беззаконие--бесчестие народов.
\end{tcolorbox}
\begin{tcolorbox}
\textsubscript{35} Благоволение царя--к рабу разумному, а гнев его--против того, кто позорит его.
\end{tcolorbox}
\subsection{CHAPTER 15}
\begin{tcolorbox}
\textsubscript{1} Кроткий ответ отвращает гнев, а оскорбительное слово возбуждает ярость.
\end{tcolorbox}
\begin{tcolorbox}
\textsubscript{2} Язык мудрых сообщает добрые знания, а уста глупых изрыгают глупость.
\end{tcolorbox}
\begin{tcolorbox}
\textsubscript{3} На всяком месте очи Господни: они видят злых и добрых.
\end{tcolorbox}
\begin{tcolorbox}
\textsubscript{4} Кроткий язык--древо жизни, но необузданный--сокрушение духа.
\end{tcolorbox}
\begin{tcolorbox}
\textsubscript{5} Глупый пренебрегает наставлением отца своего; а кто внимает обличениям, тот благоразумен.
\end{tcolorbox}
\begin{tcolorbox}
\textsubscript{6} В доме праведника--обилие сокровищ, а в прибытке нечестивого--расстройство.
\end{tcolorbox}
\begin{tcolorbox}
\textsubscript{7} Уста мудрых распространяют знание, а сердце глупых не так.
\end{tcolorbox}
\begin{tcolorbox}
\textsubscript{8} Жертва нечестивых--мерзость пред Господом, а молитва праведных благоугодна Ему.
\end{tcolorbox}
\begin{tcolorbox}
\textsubscript{9} Мерзость пред Господом--путь нечестивого, а идущего путем правды Он любит.
\end{tcolorbox}
\begin{tcolorbox}
\textsubscript{10} Злое наказание--уклоняющемуся от пути, и ненавидящий обличение погибнет.
\end{tcolorbox}
\begin{tcolorbox}
\textsubscript{11} Преисподняя и Аваддон [открыты] пред Господом, тем более сердца сынов человеческих.
\end{tcolorbox}
\begin{tcolorbox}
\textsubscript{12} Не любит распутный обличающих его, и к мудрым не пойдет.
\end{tcolorbox}
\begin{tcolorbox}
\textsubscript{13} Веселое сердце делает лице веселым, а при сердечной скорби дух унывает.
\end{tcolorbox}
\begin{tcolorbox}
\textsubscript{14} Сердце разумного ищет знания, уста же глупых питаются глупостью.
\end{tcolorbox}
\begin{tcolorbox}
\textsubscript{15} Все дни несчастного печальны; а у кого сердце весело, у того всегда пир.
\end{tcolorbox}
\begin{tcolorbox}
\textsubscript{16} Лучше немногое при страхе Господнем, нежели большое сокровище, и при нем тревога.
\end{tcolorbox}
\begin{tcolorbox}
\textsubscript{17} Лучше блюдо зелени, и при нем любовь, нежели откормленный бык, и при нем ненависть.
\end{tcolorbox}
\begin{tcolorbox}
\textsubscript{18} Вспыльчивый человек возбуждает раздор, а терпеливый утишает распрю.
\end{tcolorbox}
\begin{tcolorbox}
\textsubscript{19} Путь ленивого--как терновый плетень, а путь праведных--гладкий.
\end{tcolorbox}
\begin{tcolorbox}
\textsubscript{20} Мудрый сын радует отца, а глупый человек пренебрегает мать свою.
\end{tcolorbox}
\begin{tcolorbox}
\textsubscript{21} Глупость--радость для малоумного, а человек разумный идет прямою дорогою.
\end{tcolorbox}
\begin{tcolorbox}
\textsubscript{22} Без совета предприятия расстроятся, а при множестве советников они состоятся.
\end{tcolorbox}
\begin{tcolorbox}
\textsubscript{23} Радость человеку в ответе уст его, и как хорошо слово вовремя!
\end{tcolorbox}
\begin{tcolorbox}
\textsubscript{24} Путь жизни мудрого вверх, чтобы уклониться от преисподней внизу.
\end{tcolorbox}
\begin{tcolorbox}
\textsubscript{25} Дом надменных разорит Господь, а межу вдовы укрепит.
\end{tcolorbox}
\begin{tcolorbox}
\textsubscript{26} Мерзость пред Господом--помышления злых, слова же непорочных угодны Ему.
\end{tcolorbox}
\begin{tcolorbox}
\textsubscript{27} Корыстолюбивый расстроит дом свой, а ненавидящий подарки будет жить.
\end{tcolorbox}
\begin{tcolorbox}
\textsubscript{28} Сердце праведного обдумывает ответ, а уста нечестивых изрыгают зло.
\end{tcolorbox}
\begin{tcolorbox}
\textsubscript{29} Далек Господь от нечестивых, а молитву праведников слышит.
\end{tcolorbox}
\begin{tcolorbox}
\textsubscript{30} Светлый взгляд радует сердце, добрая весть утучняет кости.
\end{tcolorbox}
\begin{tcolorbox}
\textsubscript{31} Ухо, внимательное к учению жизни, пребывает между мудрыми.
\end{tcolorbox}
\begin{tcolorbox}
\textsubscript{32} Отвергающий наставление не радеет о своей душе; а кто внимает обличению, тот приобретает разум.
\end{tcolorbox}
\begin{tcolorbox}
\textsubscript{33} Страх Господень научает мудрости, и славе предшествует смирение.
\end{tcolorbox}
\subsection{CHAPTER 16}
\begin{tcolorbox}
\textsubscript{1} Человеку [принадлежат] предположения сердца, но от Господа ответ языка.
\end{tcolorbox}
\begin{tcolorbox}
\textsubscript{2} Все пути человека чисты в его глазах, но Господь взвешивает души.
\end{tcolorbox}
\begin{tcolorbox}
\textsubscript{3} Предай Господу дела твои, и предприятия твои совершатся.
\end{tcolorbox}
\begin{tcolorbox}
\textsubscript{4} Все сделал Господь ради Себя; и даже нечестивого [блюдет] на день бедствия.
\end{tcolorbox}
\begin{tcolorbox}
\textsubscript{5} Мерзость пред Господом всякий надменный сердцем; можно поручиться, что он не останется ненаказанным.
\end{tcolorbox}
\begin{tcolorbox}
\textsubscript{6} Милосердием и правдою очищается грех, и страх Господень отводит от зла.
\end{tcolorbox}
\begin{tcolorbox}
\textsubscript{7} Когда Господу угодны пути человека, Он и врагов его примиряет с ним.
\end{tcolorbox}
\begin{tcolorbox}
\textsubscript{8} Лучше немногое с правдою, нежели множество прибытков с неправдою.
\end{tcolorbox}
\begin{tcolorbox}
\textsubscript{9} Сердце человека обдумывает свой путь, но Господь управляет шествием его.
\end{tcolorbox}
\begin{tcolorbox}
\textsubscript{10} В устах царя--слово вдохновенное; уста его не должны погрешать на суде.
\end{tcolorbox}
\begin{tcolorbox}
\textsubscript{11} Верные весы и весовые чаши--от Господа; от Него же все гири в суме.
\end{tcolorbox}
\begin{tcolorbox}
\textsubscript{12} Мерзость для царей--дело беззаконное, потому что правдою утверждается престол.
\end{tcolorbox}
\begin{tcolorbox}
\textsubscript{13} Приятны царю уста правдивые, и говорящего истину он любит.
\end{tcolorbox}
\begin{tcolorbox}
\textsubscript{14} Царский гнев--вестник смерти; но мудрый человек умилостивит его.
\end{tcolorbox}
\begin{tcolorbox}
\textsubscript{15} В светлом взоре царя--жизнь, и благоволение его--как облако с поздним дождем.
\end{tcolorbox}
\begin{tcolorbox}
\textsubscript{16} Приобретение мудрости гораздо лучше золота, и приобретение разума предпочтительнее отборного серебра.
\end{tcolorbox}
\begin{tcolorbox}
\textsubscript{17} Путь праведных--уклонение от зла: тот бережет душу свою, кто хранит путь свой.
\end{tcolorbox}
\begin{tcolorbox}
\textsubscript{18} Погибели предшествует гордость, и падению--надменность.
\end{tcolorbox}
\begin{tcolorbox}
\textsubscript{19} Лучше смиряться духом с кроткими, нежели разделять добычу с гордыми.
\end{tcolorbox}
\begin{tcolorbox}
\textsubscript{20} Кто ведет дело разумно, тот найдет благо, и кто надеется на Господа, тот блажен.
\end{tcolorbox}
\begin{tcolorbox}
\textsubscript{21} Мудрый сердцем прозовется благоразумным, и сладкая речь прибавит к учению.
\end{tcolorbox}
\begin{tcolorbox}
\textsubscript{22} Разум для имеющих его--источник жизни, а ученость глупых--глупость.
\end{tcolorbox}
\begin{tcolorbox}
\textsubscript{23} Сердце мудрого делает язык его мудрым и умножает знание в устах его.
\end{tcolorbox}
\begin{tcolorbox}
\textsubscript{24} Приятная речь--сотовый мед, сладка для души и целебна для костей.
\end{tcolorbox}
\begin{tcolorbox}
\textsubscript{25} Есть пути, которые кажутся человеку прямыми, но конец их путь к смерти.
\end{tcolorbox}
\begin{tcolorbox}
\textsubscript{26} Трудящийся трудится для себя, потому что понуждает его [к] [тому] рот его.
\end{tcolorbox}
\begin{tcolorbox}
\textsubscript{27} Человек лукавый замышляет зло, и на устах его как бы огонь палящий.
\end{tcolorbox}
\begin{tcolorbox}
\textsubscript{28} Человек коварный сеет раздор, и наушник разлучает друзей.
\end{tcolorbox}
\begin{tcolorbox}
\textsubscript{29} Человек неблагонамеренный развращает ближнего своего и ведет его на путь недобрый;
\end{tcolorbox}
\begin{tcolorbox}
\textsubscript{30} прищуривает глаза свои, чтобы придумать коварство; закусывая себе губы, совершает злодейство.
\end{tcolorbox}
\begin{tcolorbox}
\textsubscript{31} Венец славы--седина, которая находится на пути правды.
\end{tcolorbox}
\begin{tcolorbox}
\textsubscript{32} Долготерпеливый лучше храброго, и владеющий собою [лучше] завоевателя города.
\end{tcolorbox}
\begin{tcolorbox}
\textsubscript{33} В полу бросается жребий, но все решение его--от Господа.
\end{tcolorbox}
\subsection{CHAPTER 17}
\begin{tcolorbox}
\textsubscript{1} Лучше кусок сухого хлеба, и с ним мир, нежели дом, полный заколотого скота, с раздором.
\end{tcolorbox}
\begin{tcolorbox}
\textsubscript{2} Разумный раб господствует над беспутным сыном и между братьями разделит наследство.
\end{tcolorbox}
\begin{tcolorbox}
\textsubscript{3} Плавильня--для серебра, и горнило--для золота, а сердца испытывает Господь.
\end{tcolorbox}
\begin{tcolorbox}
\textsubscript{4} Злодей внимает устам беззаконным, лжец слушается языка пагубного.
\end{tcolorbox}
\begin{tcolorbox}
\textsubscript{5} Кто ругается над нищим, тот хулит Творца его; кто радуется несчастью, тот не останется ненаказанным.
\end{tcolorbox}
\begin{tcolorbox}
\textsubscript{6} Венец стариков--сыновья сыновей, и слава детей--родители их.
\end{tcolorbox}
\begin{tcolorbox}
\textsubscript{7} Неприлична глупому важная речь, тем паче знатному--уста лживые.
\end{tcolorbox}
\begin{tcolorbox}
\textsubscript{8} Подарок--драгоценный камень в глазах владеющего им: куда ни обратится он, успеет.
\end{tcolorbox}
\begin{tcolorbox}
\textsubscript{9} Прикрывающий проступок ищет любви; а кто снова напоминает о нем, тот удаляет друга.
\end{tcolorbox}
\begin{tcolorbox}
\textsubscript{10} На разумного сильнее действует выговор, нежели на глупого сто ударов.
\end{tcolorbox}
\begin{tcolorbox}
\textsubscript{11} Возмутитель ищет только зла; поэтому жестокий ангел будет послан против него.
\end{tcolorbox}
\begin{tcolorbox}
\textsubscript{12} Лучше встретить человеку медведицу, лишенную детей, нежели глупца с его глупостью.
\end{tcolorbox}
\begin{tcolorbox}
\textsubscript{13} Кто за добро воздает злом, от дома того не отойдет зло.
\end{tcolorbox}
\begin{tcolorbox}
\textsubscript{14} Начало ссоры--как прорыв воды; оставь ссору прежде, нежели разгорелась она.
\end{tcolorbox}
\begin{tcolorbox}
\textsubscript{15} Оправдывающий нечестивого и обвиняющий праведного--оба мерзость пред Господом.
\end{tcolorbox}
\begin{tcolorbox}
\textsubscript{16} К чему сокровище в руках глупца? Для приобретения мудрости [у] [него] нет разума.
\end{tcolorbox}
\begin{tcolorbox}
\textsubscript{17} Друг любит во всякое время и, как брат, явится во время несчастья.
\end{tcolorbox}
\begin{tcolorbox}
\textsubscript{18} Человек малоумный дает руку и ручается за ближнего своего.
\end{tcolorbox}
\begin{tcolorbox}
\textsubscript{19} Кто любит ссоры, любит грех, и кто высоко поднимает ворота свои, тот ищет падения.
\end{tcolorbox}
\begin{tcolorbox}
\textsubscript{20} Коварное сердце не найдет добра, и лукавый язык попадет в беду.
\end{tcolorbox}
\begin{tcolorbox}
\textsubscript{21} Родил кто глупого, --себе на горе, и отец глупого не порадуется.
\end{tcolorbox}
\begin{tcolorbox}
\textsubscript{22} Веселое сердце благотворно, как врачевство, а унылый дух сушит кости.
\end{tcolorbox}
\begin{tcolorbox}
\textsubscript{23} Нечестивый берет подарок из пазухи, чтобы извратить пути правосудия.
\end{tcolorbox}
\begin{tcolorbox}
\textsubscript{24} Мудрость--пред лицем у разумного, а глаза глупца--на конце земли.
\end{tcolorbox}
\begin{tcolorbox}
\textsubscript{25} Глупый сын--досада отцу своему и огорчение для матери своей.
\end{tcolorbox}
\begin{tcolorbox}
\textsubscript{26} Нехорошо и обвинять правого, [и] бить вельмож за правду.
\end{tcolorbox}
\begin{tcolorbox}
\textsubscript{27} Разумный воздержан в словах своих, и благоразумный хладнокровен.
\end{tcolorbox}
\begin{tcolorbox}
\textsubscript{28} И глупец, когда молчит, может показаться мудрым, и затворяющий уста свои--благоразумным.
\end{tcolorbox}
\subsection{CHAPTER 18}
\begin{tcolorbox}
\textsubscript{1} Прихоти ищет своенравный, восстает против всего умного.
\end{tcolorbox}
\begin{tcolorbox}
\textsubscript{2} Глупый не любит знания, а только бы выказать свой ум.
\end{tcolorbox}
\begin{tcolorbox}
\textsubscript{3} С приходом нечестивого приходит и презрение, а с бесславием--поношение.
\end{tcolorbox}
\begin{tcolorbox}
\textsubscript{4} Слова уст человеческих--глубокие воды; источник мудрости--струящийся поток.
\end{tcolorbox}
\begin{tcolorbox}
\textsubscript{5} Нехорошо быть лицеприятным к нечестивому, чтобы ниспровергнуть праведного на суде.
\end{tcolorbox}
\begin{tcolorbox}
\textsubscript{6} Уста глупого идут в ссору, и слова его вызывают побои.
\end{tcolorbox}
\begin{tcolorbox}
\textsubscript{7} Язык глупого--гибель для него, и уста его--сеть для души его.
\end{tcolorbox}
\begin{tcolorbox}
\textsubscript{8} Слова наушника--как лакомства, и они входят во внутренность чрева.
\end{tcolorbox}
\begin{tcolorbox}
\textsubscript{9} Нерадивый в работе своей--брат расточителю.
\end{tcolorbox}
\begin{tcolorbox}
\textsubscript{10} Имя Господа--крепкая башня: убегает в нее праведник--и безопасен.
\end{tcolorbox}
\begin{tcolorbox}
\textsubscript{11} Имение богатого--крепкий город его, и как высокая ограда в его воображении.
\end{tcolorbox}
\begin{tcolorbox}
\textsubscript{12} Перед падением возносится сердце человека, а смирение предшествует славе.
\end{tcolorbox}
\begin{tcolorbox}
\textsubscript{13} Кто дает ответ не выслушав, тот глуп, и стыд ему.
\end{tcolorbox}
\begin{tcolorbox}
\textsubscript{14} Дух человека переносит его немощи; а пораженный дух--кто может подкрепить его?
\end{tcolorbox}
\begin{tcolorbox}
\textsubscript{15} Сердце разумного приобретает знание, и ухо мудрых ищет знания.
\end{tcolorbox}
\begin{tcolorbox}
\textsubscript{16} Подарок у человека дает ему простор и до вельмож доведет его.
\end{tcolorbox}
\begin{tcolorbox}
\textsubscript{17} Первый в тяжбе своей прав, но приходит соперник его и исследывает его.
\end{tcolorbox}
\begin{tcolorbox}
\textsubscript{18} Жребий прекращает споры и решает между сильными.
\end{tcolorbox}
\begin{tcolorbox}
\textsubscript{19} Озлобившийся брат [неприступнее] крепкого города, и ссоры подобны запорам замка.
\end{tcolorbox}
\begin{tcolorbox}
\textsubscript{20} От плода уст человека наполняется чрево его; произведением уст своих он насыщается.
\end{tcolorbox}
\begin{tcolorbox}
\textsubscript{21} Смерть и жизнь--во власти языка, и любящие его вкусят от плодов его.
\end{tcolorbox}
\begin{tcolorbox}
\textsubscript{22} Кто нашел [добрую] жену, тот нашел благо и получил благодать от Господа.
\end{tcolorbox}
\begin{tcolorbox}
\textsubscript{23} С мольбою говорит нищий, а богатый отвечает грубо.
\end{tcolorbox}
\begin{tcolorbox}
\textsubscript{24} Кто хочет иметь друзей, тот и сам должен быть дружелюбным; и бывает друг, более привязанный, нежели брат.
\end{tcolorbox}
\subsection{CHAPTER 19}
\begin{tcolorbox}
\textsubscript{1} Лучше бедный, ходящий в своей непорочности, нежели [богатый] со лживыми устами, и притом глупый.
\end{tcolorbox}
\begin{tcolorbox}
\textsubscript{2} Нехорошо душе без знания, и торопливый ногами оступится.
\end{tcolorbox}
\begin{tcolorbox}
\textsubscript{3} Глупость человека извращает путь его, а сердце его негодует на Господа.
\end{tcolorbox}
\begin{tcolorbox}
\textsubscript{4} Богатство прибавляет много друзей, а бедный оставляется и другом своим.
\end{tcolorbox}
\begin{tcolorbox}
\textsubscript{5} Лжесвидетель не останется ненаказанным, и кто говорит ложь, не спасется.
\end{tcolorbox}
\begin{tcolorbox}
\textsubscript{6} Многие заискивают у знатных, и всякий--друг человеку, делающему подарки.
\end{tcolorbox}
\begin{tcolorbox}
\textsubscript{7} Бедного ненавидят все братья его, тем паче друзья его удаляются от него: гонится за ними, чтобы поговорить, но и этого нет.
\end{tcolorbox}
\begin{tcolorbox}
\textsubscript{8} Кто приобретает разум, тот любит душу свою; кто наблюдает благоразумие, тот находит благо.
\end{tcolorbox}
\begin{tcolorbox}
\textsubscript{9} Лжесвидетель не останется ненаказанным, и кто говорит ложь, погибнет.
\end{tcolorbox}
\begin{tcolorbox}
\textsubscript{10} Неприлична глупцу пышность, тем паче рабу господство над князьями.
\end{tcolorbox}
\begin{tcolorbox}
\textsubscript{11} Благоразумие делает человека медленным на гнев, и слава для него--быть снисходительным к проступкам.
\end{tcolorbox}
\begin{tcolorbox}
\textsubscript{12} Гнев царя--как рев льва, а благоволение его--как роса на траву.
\end{tcolorbox}
\begin{tcolorbox}
\textsubscript{13} Глупый сын--сокрушение для отца своего, и сварливая жена--сточная труба.
\end{tcolorbox}
\begin{tcolorbox}
\textsubscript{14} Дом и имение--наследство от родителей, а разумная жена--от Господа.
\end{tcolorbox}
\begin{tcolorbox}
\textsubscript{15} Леность погружает в сонливость, и нерадивая душа будет терпеть голод.
\end{tcolorbox}
\begin{tcolorbox}
\textsubscript{16} Хранящий заповедь хранит душу свою, а нерадящий о путях своих погибнет.
\end{tcolorbox}
\begin{tcolorbox}
\textsubscript{17} Благотворящий бедному дает взаймы Господу, и Он воздаст ему за благодеяние его.
\end{tcolorbox}
\begin{tcolorbox}
\textsubscript{18} Наказывай сына своего, доколе есть надежда, и не возмущайся криком его.
\end{tcolorbox}
\begin{tcolorbox}
\textsubscript{19} Гневливый пусть терпит наказание, потому что, если пощадишь [его], придется тебе еще больше наказывать его.
\end{tcolorbox}
\begin{tcolorbox}
\textsubscript{20} Слушайся совета и принимай обличение, чтобы сделаться тебе впоследствии мудрым.
\end{tcolorbox}
\begin{tcolorbox}
\textsubscript{21} Много замыслов в сердце человека, но состоится только определенное Господом.
\end{tcolorbox}
\begin{tcolorbox}
\textsubscript{22} Радость человеку--благотворительность его, и бедный человек лучше, нежели лживый.
\end{tcolorbox}
\begin{tcolorbox}
\textsubscript{23} Страх Господень [ведет] к жизни, и [кто имеет его], всегда будет доволен, и зло не постигнет его.
\end{tcolorbox}
\begin{tcolorbox}
\textsubscript{24} Ленивый опускает руку свою в чашу, и не хочет донести ее до рта своего.
\end{tcolorbox}
\begin{tcolorbox}
\textsubscript{25} Если ты накажешь кощунника, то и простой сделается благоразумным; и [если] обличишь разумного, то он поймет наставление.
\end{tcolorbox}
\begin{tcolorbox}
\textsubscript{26} Разоряющий отца и выгоняющий мать--сын срамной и бесчестный.
\end{tcolorbox}
\begin{tcolorbox}
\textsubscript{27} Перестань, сын мой, слушать внушения об уклонении от изречений разума.
\end{tcolorbox}
\begin{tcolorbox}
\textsubscript{28} Лукавый свидетель издевается над судом, и уста беззаконных глотают неправду.
\end{tcolorbox}
\begin{tcolorbox}
\textsubscript{29} Готовы для кощунствующих суды, и побои--на тело глупых.
\end{tcolorbox}
\subsection{CHAPTER 20}
\begin{tcolorbox}
\textsubscript{1} Вино--глумливо, сикера--буйна; и всякий, увлекающийся ими, неразумен.
\end{tcolorbox}
\begin{tcolorbox}
\textsubscript{2} Гроза царя--как бы рев льва: кто раздражает его, тот грешит против самого себя.
\end{tcolorbox}
\begin{tcolorbox}
\textsubscript{3} Честь для человека--отстать от ссоры; а всякий глупец задорен.
\end{tcolorbox}
\begin{tcolorbox}
\textsubscript{4} Ленивец зимою не пашет: поищет летом--и нет ничего.
\end{tcolorbox}
\begin{tcolorbox}
\textsubscript{5} Помыслы в сердце человека--глубокие воды, но человек разумный вычерпывает их.
\end{tcolorbox}
\begin{tcolorbox}
\textsubscript{6} Многие хвалят человека за милосердие, но правдивого человека кто находит?
\end{tcolorbox}
\begin{tcolorbox}
\textsubscript{7} Праведник ходит в своей непорочности: блаженны дети его после него!
\end{tcolorbox}
\begin{tcolorbox}
\textsubscript{8} Царь, сидящий на престоле суда, разгоняет очами своими все злое.
\end{tcolorbox}
\begin{tcolorbox}
\textsubscript{9} Кто может сказать: 'я очистил мое сердце, я чист от греха моего?'
\end{tcolorbox}
\begin{tcolorbox}
\textsubscript{10} Неодинаковые весы, неодинаковая мера, то и другое--мерзость пред Господом.
\end{tcolorbox}
\begin{tcolorbox}
\textsubscript{11} Можно узнать даже отрока по занятиям его, чисто ли и правильно ли будет поведение его.
\end{tcolorbox}
\begin{tcolorbox}
\textsubscript{12} Ухо слышащее и глаз видящий--и то и другое создал Господь.
\end{tcolorbox}
\begin{tcolorbox}
\textsubscript{13} Не люби спать, чтобы тебе не обеднеть; держи открытыми глаза твои, и будешь досыта есть хлеб.
\end{tcolorbox}
\begin{tcolorbox}
\textsubscript{14} 'Дурно, дурно', говорит покупатель, а когда отойдет, хвалится.
\end{tcolorbox}
\begin{tcolorbox}
\textsubscript{15} Есть золото и много жемчуга, но драгоценная утварь--уста разумные.
\end{tcolorbox}
\begin{tcolorbox}
\textsubscript{16} Возьми платье его, так как он поручился за чужого; и за стороннего возьми от него залог.
\end{tcolorbox}
\begin{tcolorbox}
\textsubscript{17} Сладок для человека хлеб, [приобретенный] неправдою; но после рот его наполнится дресвою.
\end{tcolorbox}
\begin{tcolorbox}
\textsubscript{18} Предприятия получают твердость чрез совещание, и по совещании веди войну.
\end{tcolorbox}
\begin{tcolorbox}
\textsubscript{19} Кто ходит переносчиком, тот открывает тайну; и кто широко раскрывает рот, с тем не сообщайся.
\end{tcolorbox}
\begin{tcolorbox}
\textsubscript{20} Кто злословит отца своего и свою мать, того светильник погаснет среди глубокой тьмы.
\end{tcolorbox}
\begin{tcolorbox}
\textsubscript{21} Наследство, поспешно захваченное вначале, не благословится впоследствии.
\end{tcolorbox}
\begin{tcolorbox}
\textsubscript{22} Не говори: 'я отплачу за зло'; предоставь Господу, и Он сохранит тебя.
\end{tcolorbox}
\begin{tcolorbox}
\textsubscript{23} Мерзость пред Господом--неодинаковые гири, и неверные весы--не добро.
\end{tcolorbox}
\begin{tcolorbox}
\textsubscript{24} От Господа направляются шаги человека; человеку же как узнать путь свой?
\end{tcolorbox}
\begin{tcolorbox}
\textsubscript{25} Сеть для человека--поспешно давать обет, и после обета обдумывать.
\end{tcolorbox}
\begin{tcolorbox}
\textsubscript{26} Мудрый царь вывеет нечестивых и обратит на них колесо.
\end{tcolorbox}
\begin{tcolorbox}
\textsubscript{27} Светильник Господень--дух человека, испытывающий все глубины сердца.
\end{tcolorbox}
\begin{tcolorbox}
\textsubscript{28} Милость и истина охраняют царя, и милостью он поддерживает престол свой.
\end{tcolorbox}
\begin{tcolorbox}
\textsubscript{29} Слава юношей--сила их, а украшение стариков--седина.
\end{tcolorbox}
\begin{tcolorbox}
\textsubscript{30} Раны от побоев--врачевство против зла, и удары, проникающие во внутренности чрева.
\end{tcolorbox}
\subsection{CHAPTER 21}
\begin{tcolorbox}
\textsubscript{1} Сердце царя--в руке Господа, как потоки вод: куда захочет, Он направляет его.
\end{tcolorbox}
\begin{tcolorbox}
\textsubscript{2} Всякий путь человека прям в глазах его; но Господь взвешивает сердца.
\end{tcolorbox}
\begin{tcolorbox}
\textsubscript{3} Соблюдение правды и правосудия более угодно Господу, нежели жертва.
\end{tcolorbox}
\begin{tcolorbox}
\textsubscript{4} Гордость очей и надменность сердца, отличающие нечестивых, --грех.
\end{tcolorbox}
\begin{tcolorbox}
\textsubscript{5} Помышления прилежного стремятся к изобилию, а всякий торопливый терпит лишение.
\end{tcolorbox}
\begin{tcolorbox}
\textsubscript{6} Приобретение сокровища лживым языком--мимолетное дуновение ищущих смерти.
\end{tcolorbox}
\begin{tcolorbox}
\textsubscript{7} Насилие нечестивых обрушится на них, потому что они отреклись соблюдать правду.
\end{tcolorbox}
\begin{tcolorbox}
\textsubscript{8} Превратен путь человека развращенного; а кто чист, того действие прямо.
\end{tcolorbox}
\begin{tcolorbox}
\textsubscript{9} Лучше жить в углу на кровле, нежели со сварливою женою в пространном доме.
\end{tcolorbox}
\begin{tcolorbox}
\textsubscript{10} Душа нечестивого желает зла: не найдет милости в глазах его и друг его.
\end{tcolorbox}
\begin{tcolorbox}
\textsubscript{11} Когда наказывается кощунник, простой делается мудрым; и когда вразумляется мудрый, то он приобретает знание.
\end{tcolorbox}
\begin{tcolorbox}
\textsubscript{12} Праведник наблюдает за домом нечестивого: как повергаются нечестивые в несчастие.
\end{tcolorbox}
\begin{tcolorbox}
\textsubscript{13} Кто затыкает ухо свое от вопля бедного, тот и сам будет вопить, --и не будет услышан.
\end{tcolorbox}
\begin{tcolorbox}
\textsubscript{14} Подарок тайный тушит гнев, и дар в пазуху--сильную ярость.
\end{tcolorbox}
\begin{tcolorbox}
\textsubscript{15} Соблюдение правосудия--радость для праведника и страх для делающих зло.
\end{tcolorbox}
\begin{tcolorbox}
\textsubscript{16} Человек, сбившийся с пути разума, водворится в собрании мертвецов.
\end{tcolorbox}
\begin{tcolorbox}
\textsubscript{17} Кто любит веселье, обеднеет; а кто любит вино и тук, не разбогатеет.
\end{tcolorbox}
\begin{tcolorbox}
\textsubscript{18} Выкупом будет за праведного нечестивый и за прямодушного--лукавый.
\end{tcolorbox}
\begin{tcolorbox}
\textsubscript{19} Лучше жить в земле пустынной, нежели с женою сварливою и сердитою.
\end{tcolorbox}
\begin{tcolorbox}
\textsubscript{20} Вожделенное сокровище и тук--в доме мудрого; а глупый человек расточает их.
\end{tcolorbox}
\begin{tcolorbox}
\textsubscript{21} Соблюдающий правду и милость найдет жизнь, правду и славу.
\end{tcolorbox}
\begin{tcolorbox}
\textsubscript{22} Мудрый входит в город сильных и ниспровергает крепость, на которую они надеялись.
\end{tcolorbox}
\begin{tcolorbox}
\textsubscript{23} Кто хранит уста свои и язык свой, тот хранит от бед душу свою.
\end{tcolorbox}
\begin{tcolorbox}
\textsubscript{24} Надменный злодей--кощунник имя ему--действует в пылу гордости.
\end{tcolorbox}
\begin{tcolorbox}
\textsubscript{25} Алчба ленивца убьет его, потому что руки его отказываются работать;
\end{tcolorbox}
\begin{tcolorbox}
\textsubscript{26} всякий день он сильно алчет, а праведник дает и не жалеет.
\end{tcolorbox}
\begin{tcolorbox}
\textsubscript{27} Жертва нечестивых--мерзость, особенно когда с лукавством приносят ее.
\end{tcolorbox}
\begin{tcolorbox}
\textsubscript{28} Лжесвидетель погибнет; а человек, который говорит, что знает, будет говорить всегда.
\end{tcolorbox}
\begin{tcolorbox}
\textsubscript{29} Человек нечестивый дерзок лицом своим, а праведный держит прямо путь свой.
\end{tcolorbox}
\begin{tcolorbox}
\textsubscript{30} Нет мудрости, и нет разума, и нет совета вопреки Господу.
\end{tcolorbox}
\begin{tcolorbox}
\textsubscript{31} Коня приготовляют на день битвы, но победа--от Господа.
\end{tcolorbox}
\subsection{CHAPTER 22}
\begin{tcolorbox}
\textsubscript{1} Доброе имя лучше большого богатства, и добрая слава лучше серебра и золота.
\end{tcolorbox}
\begin{tcolorbox}
\textsubscript{2} Богатый и бедный встречаются друг с другом: того и другого создал Господь.
\end{tcolorbox}
\begin{tcolorbox}
\textsubscript{3} Благоразумный видит беду, и укрывается; а неопытные идут вперед, и наказываются.
\end{tcolorbox}
\begin{tcolorbox}
\textsubscript{4} За смирением следует страх Господень, богатство и слава и жизнь.
\end{tcolorbox}
\begin{tcolorbox}
\textsubscript{5} Терны и сети на пути коварного; кто бережет душу свою, удались от них.
\end{tcolorbox}
\begin{tcolorbox}
\textsubscript{6} Наставь юношу при начале пути его: он не уклонится от него, когда и состарится.
\end{tcolorbox}
\begin{tcolorbox}
\textsubscript{7} Богатый господствует над бедным, и должник [делается] рабом заимодавца.
\end{tcolorbox}
\begin{tcolorbox}
\textsubscript{8} Сеющий неправду пожнет беду, и трости гнева его не станет.
\end{tcolorbox}
\begin{tcolorbox}
\textsubscript{9} Милосердый будет благословляем, потому что дает бедному от хлеба своего.
\end{tcolorbox}
\begin{tcolorbox}
\textsubscript{10} Прогони кощунника, и удалится раздор, и прекратятся ссора и брань.
\end{tcolorbox}
\begin{tcolorbox}
\textsubscript{11} Кто любит чистоту сердца, у того приятность на устах, тому царь--друг.
\end{tcolorbox}
\begin{tcolorbox}
\textsubscript{12} Очи Господа охраняют знание, а слова законопреступника Он ниспровергает.
\end{tcolorbox}
\begin{tcolorbox}
\textsubscript{13} Ленивец говорит: 'лев на улице! посреди площади убьют меня!'
\end{tcolorbox}
\begin{tcolorbox}
\textsubscript{14} Глубокая пропасть--уста блудниц: на кого прогневается Господь, тот упадет туда.
\end{tcolorbox}
\begin{tcolorbox}
\textsubscript{15} Глупость привязалась к сердцу юноши, но исправительная розга удалит ее от него.
\end{tcolorbox}
\begin{tcolorbox}
\textsubscript{16} Кто обижает бедного, чтобы умножить свое богатство, и кто дает богатому, тот обеднеет.
\end{tcolorbox}
\begin{tcolorbox}
\textsubscript{17} Приклони ухо твое, и слушай слова мудрых, и сердце твое обрати к моему знанию;
\end{tcolorbox}
\begin{tcolorbox}
\textsubscript{18} потому что утешительно будет, если ты будешь хранить их в сердце твоем, и они будут также в устах твоих.
\end{tcolorbox}
\begin{tcolorbox}
\textsubscript{19} Чтобы упование твое было на Господа, я учу тебя и сегодня, и ты [помни].
\end{tcolorbox}
\begin{tcolorbox}
\textsubscript{20} Не писал ли я тебе трижды в советах и наставлении,
\end{tcolorbox}
\begin{tcolorbox}
\textsubscript{21} чтобы научить тебя точным словам истины, дабы ты мог передавать слова истины посылающим тебя?
\end{tcolorbox}
\begin{tcolorbox}
\textsubscript{22} Не будь грабителем бедного, потому что он беден, и не притесняй несчастного у ворот,
\end{tcolorbox}
\begin{tcolorbox}
\textsubscript{23} потому что Господь вступится в дело их и исхитит душу у грабителей их.
\end{tcolorbox}
\begin{tcolorbox}
\textsubscript{24} Не дружись с гневливым и не сообщайся с человеком вспыльчивым,
\end{tcolorbox}
\begin{tcolorbox}
\textsubscript{25} чтобы не научиться путям его и не навлечь петли на душу твою.
\end{tcolorbox}
\begin{tcolorbox}
\textsubscript{26} Не будь из тех, которые дают руки и поручаются за долги:
\end{tcolorbox}
\begin{tcolorbox}
\textsubscript{27} если тебе нечем заплатить, то для чего доводить себя, чтобы взяли постель твою из-под тебя?
\end{tcolorbox}
\begin{tcolorbox}
\textsubscript{28} Не передвигай межи давней, которую провели отцы твои.
\end{tcolorbox}
\begin{tcolorbox}
\textsubscript{29} Видел ли ты человека проворного в своем деле? Он будет стоять перед царями, он не будет стоять перед простыми.
\end{tcolorbox}
\subsection{CHAPTER 23}
\begin{tcolorbox}
\textsubscript{1} Когда сядешь вкушать пищу с властелином, то тщательно наблюдай, что перед тобою,
\end{tcolorbox}
\begin{tcolorbox}
\textsubscript{2} и поставь преграду в гортани твоей, если ты алчен.
\end{tcolorbox}
\begin{tcolorbox}
\textsubscript{3} Не прельщайся лакомыми яствами его; это--обманчивая пища.
\end{tcolorbox}
\begin{tcolorbox}
\textsubscript{4} Не заботься о том, чтобы нажить богатство; оставь такие мысли твои.
\end{tcolorbox}
\begin{tcolorbox}
\textsubscript{5} Устремишь глаза твои на него, и--его уже нет; потому что оно сделает себе крылья и, как орел, улетит к небу.
\end{tcolorbox}
\begin{tcolorbox}
\textsubscript{6} Не вкушай пищи у человека завистливого и не прельщайся лакомыми яствами его;
\end{tcolorbox}
\begin{tcolorbox}
\textsubscript{7} потому что, каковы мысли в душе его, таков и он; 'ешь и пей', говорит он тебе, а сердце его не с тобою.
\end{tcolorbox}
\begin{tcolorbox}
\textsubscript{8} Кусок, который ты съел, изблюешь, и добрые слова твои ты потратишь напрасно.
\end{tcolorbox}
\begin{tcolorbox}
\textsubscript{9} В уши глупого не говори, потому что он презрит разумные слова твои.
\end{tcolorbox}
\begin{tcolorbox}
\textsubscript{10} Не передвигай межи давней и на поля сирот не заходи,
\end{tcolorbox}
\begin{tcolorbox}
\textsubscript{11} потому что Защитник их силен; Он вступится в дело их с тобою.
\end{tcolorbox}
\begin{tcolorbox}
\textsubscript{12} Приложи сердце твое к учению и уши твои--к умным словам.
\end{tcolorbox}
\begin{tcolorbox}
\textsubscript{13} Не оставляй юноши без наказания: если накажешь его розгою, он не умрет;
\end{tcolorbox}
\begin{tcolorbox}
\textsubscript{14} ты накажешь его розгою и спасешь душу его от преисподней.
\end{tcolorbox}
\begin{tcolorbox}
\textsubscript{15} Сын мой! если сердце твое будет мудро, то порадуется и мое сердце;
\end{tcolorbox}
\begin{tcolorbox}
\textsubscript{16} и внутренности мои будут радоваться, когда уста твои будут говорить правое.
\end{tcolorbox}
\begin{tcolorbox}
\textsubscript{17} Да не завидует сердце твое грешникам, но да пребудет оно во все дни в страхе Господнем;
\end{tcolorbox}
\begin{tcolorbox}
\textsubscript{18} потому что есть будущность, и надежда твоя не потеряна.
\end{tcolorbox}
\begin{tcolorbox}
\textsubscript{19} Слушай, сын мой, и будь мудр, и направляй сердце твое на прямой путь.
\end{tcolorbox}
\begin{tcolorbox}
\textsubscript{20} Не будь между упивающимися вином, между пресыщающимися мясом:
\end{tcolorbox}
\begin{tcolorbox}
\textsubscript{21} потому что пьяница и пресыщающийся обеднеют, и сонливость оденет в рубище.
\end{tcolorbox}
\begin{tcolorbox}
\textsubscript{22} Слушайся отца твоего: он родил тебя; и не пренебрегай матери твоей, когда она и состарится.
\end{tcolorbox}
\begin{tcolorbox}
\textsubscript{23} Купи истину и не продавай мудрости и учения и разума.
\end{tcolorbox}
\begin{tcolorbox}
\textsubscript{24} Торжествует отец праведника, и родивший мудрого радуется о нем.
\end{tcolorbox}
\begin{tcolorbox}
\textsubscript{25} Да веселится отец твой и да торжествует мать твоя, родившая тебя.
\end{tcolorbox}
\begin{tcolorbox}
\textsubscript{26} Сын мой! отдай сердце твое мне, и глаза твои да наблюдают пути мои,
\end{tcolorbox}
\begin{tcolorbox}
\textsubscript{27} потому что блудница--глубокая пропасть, и чужая жена--тесный колодезь;
\end{tcolorbox}
\begin{tcolorbox}
\textsubscript{28} она, как разбойник, сидит в засаде и умножает между людьми законопреступников.
\end{tcolorbox}
\begin{tcolorbox}
\textsubscript{29} У кого вой? у кого стон? у кого ссоры? у кого горе? у кого раны без причины? у кого багровые глаза?
\end{tcolorbox}
\begin{tcolorbox}
\textsubscript{30} У тех, которые долго сидят за вином, которые приходят отыскивать [вина] приправленного.
\end{tcolorbox}
\begin{tcolorbox}
\textsubscript{31} Не смотри на вино, как оно краснеет, как оно искрится в чаше, как оно ухаживается ровно:
\end{tcolorbox}
\begin{tcolorbox}
\textsubscript{32} впоследствии, как змей, оно укусит, и ужалит, как аспид;
\end{tcolorbox}
\begin{tcolorbox}
\textsubscript{33} глаза твои будут смотреть на чужих жен, и сердце твое заговорит развратное,
\end{tcolorbox}
\begin{tcolorbox}
\textsubscript{34} и ты будешь, как спящий среди моря и как спящий на верху мачты.
\end{tcolorbox}
\begin{tcolorbox}
\textsubscript{35} [И скажешь]: 'били меня, мне не было больно; толкали меня, я не чувствовал. Когда проснусь, опять буду искать того же'.
\end{tcolorbox}
\subsection{CHAPTER 24}
\begin{tcolorbox}
\textsubscript{1} Не ревнуй злым людям и не желай быть с ними,
\end{tcolorbox}
\begin{tcolorbox}
\textsubscript{2} потому что о насилии помышляет сердце их, и о злом говорят уста их.
\end{tcolorbox}
\begin{tcolorbox}
\textsubscript{3} Мудростью устрояется дом и разумом утверждается,
\end{tcolorbox}
\begin{tcolorbox}
\textsubscript{4} и с уменьем внутренности его наполняются всяким драгоценным и прекрасным имуществом.
\end{tcolorbox}
\begin{tcolorbox}
\textsubscript{5} Человек мудрый силен, и человек разумный укрепляет силу свою.
\end{tcolorbox}
\begin{tcolorbox}
\textsubscript{6} Поэтому с обдуманностью веди войну твою, и успех [будет] при множестве совещаний.
\end{tcolorbox}
\begin{tcolorbox}
\textsubscript{7} Для глупого слишком высока мудрость; у ворот не откроет он уст своих.
\end{tcolorbox}
\begin{tcolorbox}
\textsubscript{8} Кто замышляет сделать зло, того называют злоумышленником.
\end{tcolorbox}
\begin{tcolorbox}
\textsubscript{9} Помысл глупости--грех, и кощунник--мерзость для людей.
\end{tcolorbox}
\begin{tcolorbox}
\textsubscript{10} Если ты в день бедствия оказался слабым, то бедна сила твоя.
\end{tcolorbox}
\begin{tcolorbox}
\textsubscript{11} Спасай взятых на смерть, и неужели откажешься от обреченных на убиение?
\end{tcolorbox}
\begin{tcolorbox}
\textsubscript{12} Скажешь ли: 'вот, мы не знали этого'? А Испытующий сердца разве не знает? Наблюдающий над душею твоею знает это, и воздаст человеку по делам его.
\end{tcolorbox}
\begin{tcolorbox}
\textsubscript{13} Ешь, сын мой, мед, потому что он приятен, и сот, который сладок для гортани твоей:
\end{tcolorbox}
\begin{tcolorbox}
\textsubscript{14} таково и познание мудрости для души твоей. Если ты нашел [ее], то есть будущность, и надежда твоя не потеряна.
\end{tcolorbox}
\begin{tcolorbox}
\textsubscript{15} Не злоумышляй, нечестивый, против жилища праведника, не опустошай места покоя его,
\end{tcolorbox}
\begin{tcolorbox}
\textsubscript{16} ибо семь раз упадет праведник, и встанет; а нечестивые впадут в погибель.
\end{tcolorbox}
\begin{tcolorbox}
\textsubscript{17} Не радуйся, когда упадет враг твой, и да не веселится сердце твое, когда он споткнется.
\end{tcolorbox}
\begin{tcolorbox}
\textsubscript{18} Иначе, увидит Господь, и неугодно будет это в очах Его, и Он отвратит от него гнев Свой.
\end{tcolorbox}
\begin{tcolorbox}
\textsubscript{19} Не негодуй на злодеев и не завидуй нечестивым,
\end{tcolorbox}
\begin{tcolorbox}
\textsubscript{20} потому что злой не имеет будущности, --светильник нечестивых угаснет.
\end{tcolorbox}
\begin{tcolorbox}
\textsubscript{21} Бойся, сын мой, Господа и царя; с мятежниками не сообщайся,
\end{tcolorbox}
\begin{tcolorbox}
\textsubscript{22} потому что внезапно придет погибель от них, и беду от них обоих кто предузнает?
\end{tcolorbox}
\begin{tcolorbox}
\textsubscript{23} Сказано также мудрыми: иметь лицеприятие на суде--нехорошо.
\end{tcolorbox}
\begin{tcolorbox}
\textsubscript{24} Кто говорит виновному: 'ты прав', того будут проклинать народы, того будут ненавидеть племена;
\end{tcolorbox}
\begin{tcolorbox}
\textsubscript{25} а обличающие будут любимы, и на них придет благословение.
\end{tcolorbox}
\begin{tcolorbox}
\textsubscript{26} В уста целует, кто отвечает словами верными.
\end{tcolorbox}
\begin{tcolorbox}
\textsubscript{27} Соверши дела твои вне дома, окончи их на поле твоем, и потом устрояй и дом твой.
\end{tcolorbox}
\begin{tcolorbox}
\textsubscript{28} Не будь лжесвидетелем на ближнего твоего: к чему тебе обманывать устами твоими?
\end{tcolorbox}
\begin{tcolorbox}
\textsubscript{29} Не говори: 'как он поступил со мною, так и я поступлю с ним, воздам человеку по делам его'.
\end{tcolorbox}
\begin{tcolorbox}
\textsubscript{30} Проходил я мимо поля человека ленивого и мимо виноградника человека скудоумного:
\end{tcolorbox}
\begin{tcolorbox}
\textsubscript{31} и вот, все это заросло терном, поверхность его покрылась крапивою, и каменная ограда его обрушилась.
\end{tcolorbox}
\begin{tcolorbox}
\textsubscript{32} И посмотрел я, и обратил сердце мое, и посмотрел и получил урок:
\end{tcolorbox}
\begin{tcolorbox}
\textsubscript{33} 'немного поспишь, немного подремлешь, немного, сложив руки, полежишь, --
\end{tcolorbox}
\begin{tcolorbox}
\textsubscript{34} и придет, [как] прохожий, бедность твоя, и нужда твоя--как человек вооруженный'.
\end{tcolorbox}
\subsection{CHAPTER 25}
\begin{tcolorbox}
\textsubscript{1} И это притчи Соломона, которые собрали мужи Езекии, царя Иудейского.
\end{tcolorbox}
\begin{tcolorbox}
\textsubscript{2} Слава Божия--облекать тайною дело, а слава царей--исследывать дело.
\end{tcolorbox}
\begin{tcolorbox}
\textsubscript{3} Как небо в высоте и земля в глубине, так сердце царей--неисследимо.
\end{tcolorbox}
\begin{tcolorbox}
\textsubscript{4} Отдели примесь от серебра, и выйдет у серебряника сосуд:
\end{tcolorbox}
\begin{tcolorbox}
\textsubscript{5} удали неправедного от царя, и престол его утвердится правдою.
\end{tcolorbox}
\begin{tcolorbox}
\textsubscript{6} Не величайся пред лицем царя, и на месте великих не становись;
\end{tcolorbox}
\begin{tcolorbox}
\textsubscript{7} потому что лучше, когда скажут тебе: 'пойди сюда повыше', нежели когда понизят тебя пред знатным, которого видели глаза твои.
\end{tcolorbox}
\begin{tcolorbox}
\textsubscript{8} Не вступай поспешно в тяжбу: иначе что будешь делать при окончании, когда соперник твой осрамит тебя?
\end{tcolorbox}
\begin{tcolorbox}
\textsubscript{9} Веди тяжбу с соперником твоим, но тайны другого не открывай,
\end{tcolorbox}
\begin{tcolorbox}
\textsubscript{10} дабы не укорил тебя услышавший это, и тогда бесчестие твое не отойдет от тебя.
\end{tcolorbox}
\begin{tcolorbox}
\textsubscript{11} Золотые яблоки в серебряных прозрачных сосудах--слово, сказанное прилично.
\end{tcolorbox}
\begin{tcolorbox}
\textsubscript{12} Золотая серьга и украшение из чистого золота--мудрый обличитель для внимательного уха.
\end{tcolorbox}
\begin{tcolorbox}
\textsubscript{13} Что прохлада от снега во время жатвы, то верный посол для посылающего его: он доставляет душе господина своего отраду.
\end{tcolorbox}
\begin{tcolorbox}
\textsubscript{14} Что тучи и ветры без дождя, то человек, хвастающий ложными подарками.
\end{tcolorbox}
\begin{tcolorbox}
\textsubscript{15} Кротостью склоняется к милости вельможа, и мягкий язык переламывает кость.
\end{tcolorbox}
\begin{tcolorbox}
\textsubscript{16} Нашел ты мед, --ешь, сколько тебе потребно, чтобы не пресытиться им и не изблевать его.
\end{tcolorbox}
\begin{tcolorbox}
\textsubscript{17} Не учащай входить в дом друга твоего, чтобы он не наскучил тобою и не возненавидел тебя.
\end{tcolorbox}
\begin{tcolorbox}
\textsubscript{18} Что молот и меч и острая стрела, то человек, произносящий ложное свидетельство против ближнего своего.
\end{tcolorbox}
\begin{tcolorbox}
\textsubscript{19} Что сломанный зуб и расслабленная нога, то надежда на ненадежного [человека] в день бедствия.
\end{tcolorbox}
\begin{tcolorbox}
\textsubscript{20} Что снимающий с себя одежду в холодный день, что уксус на рану, то поющий песни печальному сердцу.
\end{tcolorbox}
\begin{tcolorbox}
\textsubscript{21} Если голоден враг твой, накорми его хлебом; и если он жаждет, напой его водою:
\end{tcolorbox}
\begin{tcolorbox}
\textsubscript{22} ибо, [делая сие], ты собираешь горящие угли на голову его, и Господь воздаст тебе.
\end{tcolorbox}
\begin{tcolorbox}
\textsubscript{23} Северный ветер производит дождь, а тайный язык--недовольные лица.
\end{tcolorbox}
\begin{tcolorbox}
\textsubscript{24} Лучше жить в углу на кровле, нежели со сварливою женою в пространном доме.
\end{tcolorbox}
\begin{tcolorbox}
\textsubscript{25} Что холодная вода для истомленной жаждой души, то добрая весть из дальней страны.
\end{tcolorbox}
\begin{tcolorbox}
\textsubscript{26} Что возмущенный источник и поврежденный родник, то праведник, падающий пред нечестивым.
\end{tcolorbox}
\begin{tcolorbox}
\textsubscript{27} Как нехорошо есть много меду, так домогаться славы не есть слава.
\end{tcolorbox}
\begin{tcolorbox}
\textsubscript{28} Что город разрушенный, без стен, то человек, не владеющий духом своим.
\end{tcolorbox}
\subsection{CHAPTER 26}
\begin{tcolorbox}
\textsubscript{1} Как снег летом и дождь во время жатвы, так честь неприлична глупому.
\end{tcolorbox}
\begin{tcolorbox}
\textsubscript{2} Как воробей вспорхнет, как ласточка улетит, так незаслуженное проклятие не сбудется.
\end{tcolorbox}
\begin{tcolorbox}
\textsubscript{3} Бич для коня, узда для осла, а палка для глупых.
\end{tcolorbox}
\begin{tcolorbox}
\textsubscript{4} Не отвечай глупому по глупости его, чтобы и тебе не сделаться подобным ему;
\end{tcolorbox}
\begin{tcolorbox}
\textsubscript{5} но отвечай глупому по глупости его, чтобы он не стал мудрецом в глазах своих.
\end{tcolorbox}
\begin{tcolorbox}
\textsubscript{6} Подрезывает себе ноги, терпит неприятность тот, кто дает словесное поручение глупцу.
\end{tcolorbox}
\begin{tcolorbox}
\textsubscript{7} Неровно поднимаются ноги у хромого, --и притча в устах глупцов.
\end{tcolorbox}
\begin{tcolorbox}
\textsubscript{8} Что влагающий драгоценный камень в пращу, то воздающий глупому честь.
\end{tcolorbox}
\begin{tcolorbox}
\textsubscript{9} Что [колючий] терн в руке пьяного, то притча в устах глупцов.
\end{tcolorbox}
\begin{tcolorbox}
\textsubscript{10} Сильный делает все произвольно: и глупого награждает, и всякого прохожего награждает.
\end{tcolorbox}
\begin{tcolorbox}
\textsubscript{11} Как пес возвращается на блевотину свою, так глупый повторяет глупость свою.
\end{tcolorbox}
\begin{tcolorbox}
\textsubscript{12} Видал ли ты человека, мудрого в глазах его? На глупого больше надежды, нежели на него.
\end{tcolorbox}
\begin{tcolorbox}
\textsubscript{13} Ленивец говорит: 'лев на дороге! лев на площадях!'
\end{tcolorbox}
\begin{tcolorbox}
\textsubscript{14} Дверь ворочается на крючьях своих, а ленивец на постели своей.
\end{tcolorbox}
\begin{tcolorbox}
\textsubscript{15} Ленивец опускает руку свою в чашу, и ему тяжело донести ее до рта своего.
\end{tcolorbox}
\begin{tcolorbox}
\textsubscript{16} Ленивец в глазах своих мудрее семерых, отвечающих обдуманно.
\end{tcolorbox}
\begin{tcolorbox}
\textsubscript{17} Хватает пса за уши, кто, проходя мимо, вмешивается в чужую ссору.
\end{tcolorbox}
\begin{tcolorbox}
\textsubscript{18} Как притворяющийся помешанным бросает огонь, стрелы и смерть,
\end{tcolorbox}
\begin{tcolorbox}
\textsubscript{19} так--человек, который коварно вредит другу своему и потом говорит: 'я только пошутил'.
\end{tcolorbox}
\begin{tcolorbox}
\textsubscript{20} Где нет больше дров, огонь погасает, и где нет наушника, раздор утихает.
\end{tcolorbox}
\begin{tcolorbox}
\textsubscript{21} Уголь--для жара и дрова--для огня, а человек сварливый--для разжжения ссоры.
\end{tcolorbox}
\begin{tcolorbox}
\textsubscript{22} Слова наушника--как лакомства, и они входят во внутренность чрева.
\end{tcolorbox}
\begin{tcolorbox}
\textsubscript{23} Что нечистым серебром обложенный глиняный сосуд, то пламенные уста и сердце злобное.
\end{tcolorbox}
\begin{tcolorbox}
\textsubscript{24} Устами своими притворяется враг, а в сердце своем замышляет коварство.
\end{tcolorbox}
\begin{tcolorbox}
\textsubscript{25} Если он говорит и нежным голосом, не верь ему, потому что семь мерзостей в сердце его.
\end{tcolorbox}
\begin{tcolorbox}
\textsubscript{26} Если ненависть прикрывается наедине, то откроется злоба его в народном собрании.
\end{tcolorbox}
\begin{tcolorbox}
\textsubscript{27} Кто роет яму, тот упадет в нее, и кто покатит вверх камень, к тому он воротится.
\end{tcolorbox}
\begin{tcolorbox}
\textsubscript{28} Лживый язык ненавидит уязвляемых им, и льстивые уста готовят падение.
\end{tcolorbox}
\subsection{CHAPTER 27}
\begin{tcolorbox}
\textsubscript{1} Не хвались завтрашним днем, потому что не знаешь, что родит тот день.
\end{tcolorbox}
\begin{tcolorbox}
\textsubscript{2} Пусть хвалит тебя другой, а не уста твои, --чужой, а не язык твой.
\end{tcolorbox}
\begin{tcolorbox}
\textsubscript{3} Тяжел камень, весок и песок; но гнев глупца тяжелее их обоих.
\end{tcolorbox}
\begin{tcolorbox}
\textsubscript{4} Жесток гнев, неукротима ярость; но кто устоит против ревности?
\end{tcolorbox}
\begin{tcolorbox}
\textsubscript{5} Лучше открытое обличение, нежели скрытая любовь.
\end{tcolorbox}
\begin{tcolorbox}
\textsubscript{6} Искренни укоризны от любящего, и лживы поцелуи ненавидящего.
\end{tcolorbox}
\begin{tcolorbox}
\textsubscript{7} Сытая душа попирает и сот, а голодной душе все горькое сладко.
\end{tcolorbox}
\begin{tcolorbox}
\textsubscript{8} Как птица, покинувшая гнездо свое, так человек, покинувший место свое.
\end{tcolorbox}
\begin{tcolorbox}
\textsubscript{9} Масть и курение радуют сердце; так сладок [всякому] друг сердечным советом своим.
\end{tcolorbox}
\begin{tcolorbox}
\textsubscript{10} Не покидай друга твоего и друга отца твоего, и в дом брата твоего не ходи в день несчастья твоего: лучше сосед вблизи, нежели брат вдали.
\end{tcolorbox}
\begin{tcolorbox}
\textsubscript{11} Будь мудр, сын мой, и радуй сердце мое; и я буду иметь, что отвечать злословящему меня.
\end{tcolorbox}
\begin{tcolorbox}
\textsubscript{12} Благоразумный видит беду и укрывается; а неопытные идут вперед [и] наказываются.
\end{tcolorbox}
\begin{tcolorbox}
\textsubscript{13} Возьми у него платье его, потому что он поручился за чужого, и за стороннего возьми от него залог.
\end{tcolorbox}
\begin{tcolorbox}
\textsubscript{14} Кто громко хвалит друга своего с раннего утра, того сочтут за злословящего.
\end{tcolorbox}
\begin{tcolorbox}
\textsubscript{15} Непрестанная капель в дождливый день и сварливая жена--равны:
\end{tcolorbox}
\begin{tcolorbox}
\textsubscript{16} кто хочет скрыть ее, тот хочет скрыть ветер и масть в правой руке своей, дающую знать о себе.
\end{tcolorbox}
\begin{tcolorbox}
\textsubscript{17} Железо железо острит, и человек изощряет взгляд друга своего.
\end{tcolorbox}
\begin{tcolorbox}
\textsubscript{18} Кто стережет смоковницу, тот будет есть плоды ее; и кто бережет господина своего, тот будет в чести.
\end{tcolorbox}
\begin{tcolorbox}
\textsubscript{19} Как в воде лицо--к лицу, так сердце человека--к человеку.
\end{tcolorbox}
\begin{tcolorbox}
\textsubscript{20} Преисподняя и Аваддон--ненасытимы; так ненасытимы и глаза человеческие.
\end{tcolorbox}
\begin{tcolorbox}
\textsubscript{21} Что плавильня--для серебра, горнило--для золота, то для человека уста, которые хвалят его.
\end{tcolorbox}
\begin{tcolorbox}
\textsubscript{22} Толки глупого в ступе пестом вместе с зерном, не отделится от него глупость его.
\end{tcolorbox}
\begin{tcolorbox}
\textsubscript{23} Хорошо наблюдай за скотом твоим, имей попечение о стадах;
\end{tcolorbox}
\begin{tcolorbox}
\textsubscript{24} потому что [богатство] не навек, да и власть разве из рода в род?
\end{tcolorbox}
\begin{tcolorbox}
\textsubscript{25} Прозябает трава, и является зелень, и собирают горные травы.
\end{tcolorbox}
\begin{tcolorbox}
\textsubscript{26} Овцы--на одежду тебе, и козлы--на покупку поля.
\end{tcolorbox}
\begin{tcolorbox}
\textsubscript{27} И довольно козьего молока в пищу тебе, в пищу домашним твоим и на продовольствие служанкам твоим.
\end{tcolorbox}
\subsection{CHAPTER 28}
\begin{tcolorbox}
\textsubscript{1} Нечестивый бежит, когда никто не гонится [за ним]; а праведник смел, как лев.
\end{tcolorbox}
\begin{tcolorbox}
\textsubscript{2} Когда страна отступит от закона, тогда много в ней начальников; а при разумном и знающем муже она долговечна.
\end{tcolorbox}
\begin{tcolorbox}
\textsubscript{3} Человек бедный и притесняющий слабых [то же, что] проливной дождь, смывающий хлеб.
\end{tcolorbox}
\begin{tcolorbox}
\textsubscript{4} Отступники от закона хвалят нечестивых, а соблюдающие закон негодуют на них.
\end{tcolorbox}
\begin{tcolorbox}
\textsubscript{5} Злые люди не разумеют справедливости, а ищущие Господа разумеют всё.
\end{tcolorbox}
\begin{tcolorbox}
\textsubscript{6} Лучше бедный, ходящий в своей непорочности, нежели тот, кто извращает пути свои, хотя он и богат.
\end{tcolorbox}
\begin{tcolorbox}
\textsubscript{7} Хранящий закон--сын разумный, а знающийся с расточителями срамит отца своего.
\end{tcolorbox}
\begin{tcolorbox}
\textsubscript{8} Умножающий имение свое ростом и лихвою соберет его для благотворителя бедных.
\end{tcolorbox}
\begin{tcolorbox}
\textsubscript{9} Кто отклоняет ухо свое от слушания закона, того и молитва--мерзость.
\end{tcolorbox}
\begin{tcolorbox}
\textsubscript{10} Совращающий праведных на путь зла сам упадет в свою яму, а непорочные наследуют добро.
\end{tcolorbox}
\begin{tcolorbox}
\textsubscript{11} Человек богатый--мудрец в глазах своих, но умный бедняк обличит его.
\end{tcolorbox}
\begin{tcolorbox}
\textsubscript{12} Когда торжествуют праведники, великая слава, но когда возвышаются нечестивые, люди укрываются.
\end{tcolorbox}
\begin{tcolorbox}
\textsubscript{13} Скрывающий свои преступления не будет иметь успеха; а кто сознается и оставляет их, тот будет помилован.
\end{tcolorbox}
\begin{tcolorbox}
\textsubscript{14} Блажен человек, который всегда пребывает в благоговении; а кто ожесточает сердце свое, тот попадет в беду.
\end{tcolorbox}
\begin{tcolorbox}
\textsubscript{15} Как рыкающий лев и голодный медведь, так нечестивый властелин над бедным народом.
\end{tcolorbox}
\begin{tcolorbox}
\textsubscript{16} Неразумный правитель много делает притеснений, а ненавидящий корысть продолжит дни.
\end{tcolorbox}
\begin{tcolorbox}
\textsubscript{17} Человек, виновный в пролитии человеческой крови, будет бегать до могилы, чтобы кто не схватил его.
\end{tcolorbox}
\begin{tcolorbox}
\textsubscript{18} Кто ходит непорочно, то будет невредим; а ходящий кривыми путями упадет на одном из них.
\end{tcolorbox}
\begin{tcolorbox}
\textsubscript{19} Кто возделывает землю свою, тот будет насыщаться хлебом, а кто подражает праздным, тот насытится нищетою.
\end{tcolorbox}
\begin{tcolorbox}
\textsubscript{20} Верный человек богат благословениями, а кто спешит разбогатеть, тот не останется ненаказанным.
\end{tcolorbox}
\begin{tcolorbox}
\textsubscript{21} Быть лицеприятным--нехорошо: такой человек и за кусок хлеба сделает неправду.
\end{tcolorbox}
\begin{tcolorbox}
\textsubscript{22} Спешит к богатству завистливый человек, и не думает, что нищета постигнет его.
\end{tcolorbox}
\begin{tcolorbox}
\textsubscript{23} Обличающий человека найдет после большую приязнь, нежели тот, кто льстит языком.
\end{tcolorbox}
\begin{tcolorbox}
\textsubscript{24} Кто обкрадывает отца своего и мать свою и говорит: 'это не грех', тот--сообщник грабителям.
\end{tcolorbox}
\begin{tcolorbox}
\textsubscript{25} Надменный разжигает ссору, а надеющийся на Господа будет благоденствовать.
\end{tcolorbox}
\begin{tcolorbox}
\textsubscript{26} Кто надеется на себя, тот глуп; а кто ходит в мудрости, тот будет цел.
\end{tcolorbox}
\begin{tcolorbox}
\textsubscript{27} Дающий нищему не обеднеет; а кто закрывает глаза свои от него, на том много проклятий.
\end{tcolorbox}
\begin{tcolorbox}
\textsubscript{28} Когда возвышаются нечестивые, люди укрываются, а когда они падают, умножаются праведники.
\end{tcolorbox}
\subsection{CHAPTER 29}
\begin{tcolorbox}
\textsubscript{1} Человек, который, будучи обличаем, ожесточает выю свою, внезапно сокрушится, и не будет [ему] исцеления.
\end{tcolorbox}
\begin{tcolorbox}
\textsubscript{2} Когда умножаются праведники, веселится народ, а когда господствует нечестивый, народ стенает.
\end{tcolorbox}
\begin{tcolorbox}
\textsubscript{3} Человек, любящий мудрость, радует отца своего; а кто знается с блудницами, тот расточает имение.
\end{tcolorbox}
\begin{tcolorbox}
\textsubscript{4} Царь правосудием утверждает землю, а любящий подарки разоряет ее.
\end{tcolorbox}
\begin{tcolorbox}
\textsubscript{5} Человек, льстящий другу своему, расстилает сеть ногам его.
\end{tcolorbox}
\begin{tcolorbox}
\textsubscript{6} В грехе злого человека--сеть [для него], а праведник веселится и радуется.
\end{tcolorbox}
\begin{tcolorbox}
\textsubscript{7} Праведник тщательно вникает в тяжбу бедных, а нечестивый не разбирает дела.
\end{tcolorbox}
\begin{tcolorbox}
\textsubscript{8} Люди развратные возмущают город, а мудрые утишают мятеж.
\end{tcolorbox}
\begin{tcolorbox}
\textsubscript{9} Умный человек, судясь с человеком глупым, сердится ли, смеется ли, --не имеет покоя.
\end{tcolorbox}
\begin{tcolorbox}
\textsubscript{10} Кровожадные люди ненавидят непорочного, а праведные заботятся о его жизни.
\end{tcolorbox}
\begin{tcolorbox}
\textsubscript{11} Глупый весь гнев свой изливает, а мудрый сдерживает его.
\end{tcolorbox}
\begin{tcolorbox}
\textsubscript{12} Если правитель слушает ложные речи, то и все служащие у него нечестивы.
\end{tcolorbox}
\begin{tcolorbox}
\textsubscript{13} Бедный и лихоимец встречаются друг с другом; но свет глазам того и другого дает Господь.
\end{tcolorbox}
\begin{tcolorbox}
\textsubscript{14} Если царь судит бедных по правде, то престол его навсегда утвердится.
\end{tcolorbox}
\begin{tcolorbox}
\textsubscript{15} Розга и обличение дают мудрость; но отрок, оставленный в небрежении, делает стыд своей матери.
\end{tcolorbox}
\begin{tcolorbox}
\textsubscript{16} При умножении нечестивых умножается беззаконие; но праведники увидят падение их.
\end{tcolorbox}
\begin{tcolorbox}
\textsubscript{17} Наказывай сына твоего, и он даст тебе покой, и доставит радость душе твоей.
\end{tcolorbox}
\begin{tcolorbox}
\textsubscript{18} Без откровения свыше народ необуздан, а соблюдающий закон блажен.
\end{tcolorbox}
\begin{tcolorbox}
\textsubscript{19} Словами не научится раб, потому что, хотя он понимает [их], но не слушается.
\end{tcolorbox}
\begin{tcolorbox}
\textsubscript{20} Видал ли ты человека опрометчивого в словах своих? на глупого больше надежды, нежели на него.
\end{tcolorbox}
\begin{tcolorbox}
\textsubscript{21} Если с детства воспитывать раба в неге, то впоследствии он захочет быть сыном.
\end{tcolorbox}
\begin{tcolorbox}
\textsubscript{22} Человек гневливый заводит ссору, и вспыльчивый много грешит.
\end{tcolorbox}
\begin{tcolorbox}
\textsubscript{23} Гордость человека унижает его, а смиренный духом приобретает честь.
\end{tcolorbox}
\begin{tcolorbox}
\textsubscript{24} Кто делится с вором, тот ненавидит душу свою; слышит он проклятие, но не объявляет о том.
\end{tcolorbox}
\begin{tcolorbox}
\textsubscript{25} Боязнь пред людьми ставит сеть; а надеющийся на Господа будет безопасен.
\end{tcolorbox}
\begin{tcolorbox}
\textsubscript{26} Многие ищут [благосклонного] лица правителя, но судьба человека--от Господа.
\end{tcolorbox}
\begin{tcolorbox}
\textsubscript{27} Мерзость для праведников--человек неправедный, и мерзость для нечестивого--идущий прямым путем.
\end{tcolorbox}
\subsection{CHAPTER 30}
\begin{tcolorbox}
\textsubscript{1} Слова Агура, сына Иакеева. Вдохновенные изречения, [которые] сказал этот человек Ифиилу, Ифиилу и Укалу:
\end{tcolorbox}
\begin{tcolorbox}
\textsubscript{2} подлинно, я более невежда, нежели кто-либо из людей, и разума человеческого нет у меня,
\end{tcolorbox}
\begin{tcolorbox}
\textsubscript{3} и не научился я мудрости, и познания святых не имею.
\end{tcolorbox}
\begin{tcolorbox}
\textsubscript{4} Кто восходил на небо и нисходил? кто собрал ветер в пригоршни свои? кто завязал воду в одежду? кто поставил все пределы земли? какое имя ему? и какое имя сыну его? знаешь ли?
\end{tcolorbox}
\begin{tcolorbox}
\textsubscript{5} Всякое слово Бога чисто; Он--щит уповающим на Него.
\end{tcolorbox}
\begin{tcolorbox}
\textsubscript{6} Не прибавляй к словам Его, чтобы Он не обличил тебя, и ты не оказался лжецом.
\end{tcolorbox}
\begin{tcolorbox}
\textsubscript{7} Двух вещей я прошу у Тебя, не откажи мне, прежде нежели я умру:
\end{tcolorbox}
\begin{tcolorbox}
\textsubscript{8} суету и ложь удали от меня, нищеты и богатства не давай мне, питай меня насущным хлебом,
\end{tcolorbox}
\begin{tcolorbox}
\textsubscript{9} дабы, пресытившись, я не отрекся [Тебя] и не сказал: 'кто Господь?' и чтобы, обеднев, не стал красть и употреблять имя Бога моего всуе.
\end{tcolorbox}
\begin{tcolorbox}
\textsubscript{10} Не злословь раба пред господином его, чтобы он не проклял тебя, и ты не остался виноватым.
\end{tcolorbox}
\begin{tcolorbox}
\textsubscript{11} Есть род, который проклинает отца своего и не благословляет матери своей.
\end{tcolorbox}
\begin{tcolorbox}
\textsubscript{12} Есть род, который чист в глазах своих, тогда как не омыт от нечистот своих.
\end{tcolorbox}
\begin{tcolorbox}
\textsubscript{13} Есть род--о, как высокомерны глаза его, и как подняты ресницы его!
\end{tcolorbox}
\begin{tcolorbox}
\textsubscript{14} Есть род, у которого зубы--мечи, и челюсти--ножи, чтобы пожирать бедных на земле и нищих между людьми.
\end{tcolorbox}
\begin{tcolorbox}
\textsubscript{15} У ненасытимости две дочери: 'давай, давай!' Вот три ненасытимых, и четыре, которые не скажут: 'довольно!'
\end{tcolorbox}
\begin{tcolorbox}
\textsubscript{16} Преисподняя и утроба бесплодная, земля, которая не насыщается водою, и огонь, который не говорит: 'довольно!'
\end{tcolorbox}
\begin{tcolorbox}
\textsubscript{17} Глаз, насмехающийся над отцом и пренебрегающий покорностью к матери, выклюют вороны дольные, и сожрут птенцы орлиные!
\end{tcolorbox}
\begin{tcolorbox}
\textsubscript{18} Три вещи непостижимы для меня, и четырех я не понимаю:
\end{tcolorbox}
\begin{tcolorbox}
\textsubscript{19} пути орла на небе, пути змея на скале, пути корабля среди моря и пути мужчины к девице.
\end{tcolorbox}
\begin{tcolorbox}
\textsubscript{20} Таков путь и жены прелюбодейной; поела и обтерла рот свой, и говорит: 'я ничего худого не сделала'.
\end{tcolorbox}
\begin{tcolorbox}
\textsubscript{21} От трех трясется земля, четырех она не может носить:
\end{tcolorbox}
\begin{tcolorbox}
\textsubscript{22} раба, когда он делается царем; глупого, когда он досыта ест хлеб;
\end{tcolorbox}
\begin{tcolorbox}
\textsubscript{23} позорную женщину, когда она выходит замуж, и служанку, когда она занимает место госпожи своей.
\end{tcolorbox}
\begin{tcolorbox}
\textsubscript{24} Вот четыре малых на земле, но они мудрее мудрых:
\end{tcolorbox}
\begin{tcolorbox}
\textsubscript{25} муравьи--народ не сильный, но летом заготовляют пищу свою;
\end{tcolorbox}
\begin{tcolorbox}
\textsubscript{26} горные мыши--народ слабый, но ставят домы свои на скале;
\end{tcolorbox}
\begin{tcolorbox}
\textsubscript{27} у саранчи нет царя, но выступает вся она стройно;
\end{tcolorbox}
\begin{tcolorbox}
\textsubscript{28} паук лапками цепляется, но бывает в царских чертогах.
\end{tcolorbox}
\begin{tcolorbox}
\textsubscript{29} Вот трое имеют стройную походку, и четверо стройно выступают:
\end{tcolorbox}
\begin{tcolorbox}
\textsubscript{30} лев, силач между зверями, не посторонится ни перед кем;
\end{tcolorbox}
\begin{tcolorbox}
\textsubscript{31} конь и козел, и царь среди народа своего.
\end{tcolorbox}
\begin{tcolorbox}
\textsubscript{32} Если ты в заносчивости своей сделал глупость и помыслил злое, то [положи] руку на уста;
\end{tcolorbox}
\begin{tcolorbox}
\textsubscript{33} потому что, как сбивание молока производит масло, толчок в нос производит кровь, так и возбуждение гнева производит ссору.
\end{tcolorbox}
\subsection{CHAPTER 31}
\begin{tcolorbox}
\textsubscript{1} Слова Лемуила царя. Наставление, которое преподала ему мать его:
\end{tcolorbox}
\begin{tcolorbox}
\textsubscript{2} что, сын мой? что, сын чрева моего? что, сын обетов моих?
\end{tcolorbox}
\begin{tcolorbox}
\textsubscript{3} Не отдавай женщинам сил твоих, ни путей твоих губительницам царей.
\end{tcolorbox}
\begin{tcolorbox}
\textsubscript{4} Не царям, Лемуил, не царям пить вино, и не князьям--сикеру,
\end{tcolorbox}
\begin{tcolorbox}
\textsubscript{5} чтобы, напившись, они не забыли закона и не превратили суда всех угнетаемых.
\end{tcolorbox}
\begin{tcolorbox}
\textsubscript{6} Дайте сикеру погибающему и вино огорченному душею;
\end{tcolorbox}
\begin{tcolorbox}
\textsubscript{7} пусть он выпьет и забудет бедность свою и не вспомнит больше о своем страдании.
\end{tcolorbox}
\begin{tcolorbox}
\textsubscript{8} Открывай уста твои за безгласного и для защиты всех сирот.
\end{tcolorbox}
\begin{tcolorbox}
\textsubscript{9} Открывай уста твои для правосудия и для дела бедного и нищего.
\end{tcolorbox}
\begin{tcolorbox}
\textsubscript{10} Кто найдет добродетельную жену? цена ее выше жемчугов;
\end{tcolorbox}
\begin{tcolorbox}
\textsubscript{11} уверено в ней сердце мужа ее, и он не останется без прибытка;
\end{tcolorbox}
\begin{tcolorbox}
\textsubscript{12} она воздает ему добром, а не злом, во все дни жизни своей.
\end{tcolorbox}
\begin{tcolorbox}
\textsubscript{13} Добывает шерсть и лен, и с охотою работает своими руками.
\end{tcolorbox}
\begin{tcolorbox}
\textsubscript{14} Она, как купеческие корабли, издалека добывает хлеб свой.
\end{tcolorbox}
\begin{tcolorbox}
\textsubscript{15} Она встает еще ночью и раздает пищу в доме своем и урочное служанкам своим.
\end{tcolorbox}
\begin{tcolorbox}
\textsubscript{16} Задумает она о поле, и приобретает его; от плодов рук своих насаждает виноградник.
\end{tcolorbox}
\begin{tcolorbox}
\textsubscript{17} Препоясывает силою чресла свои и укрепляет мышцы свои.
\end{tcolorbox}
\begin{tcolorbox}
\textsubscript{18} Она чувствует, что занятие ее хорошо, и--светильник ее не гаснет и ночью.
\end{tcolorbox}
\begin{tcolorbox}
\textsubscript{19} Протягивает руки свои к прялке, и персты ее берутся за веретено.
\end{tcolorbox}
\begin{tcolorbox}
\textsubscript{20} Длань свою она открывает бедному, и руку свою подает нуждающемуся.
\end{tcolorbox}
\begin{tcolorbox}
\textsubscript{21} Не боится стужи для семьи своей, потому что вся семья ее одета в двойные одежды.
\end{tcolorbox}
\begin{tcolorbox}
\textsubscript{22} Она делает себе ковры; виссон и пурпур--одежда ее.
\end{tcolorbox}
\begin{tcolorbox}
\textsubscript{23} Муж ее известен у ворот, когда сидит со старейшинами земли.
\end{tcolorbox}
\begin{tcolorbox}
\textsubscript{24} Она делает покрывала и продает, и поясы доставляет купцам Финикийским.
\end{tcolorbox}
\begin{tcolorbox}
\textsubscript{25} Крепость и красота--одежда ее, и весело смотрит она на будущее.
\end{tcolorbox}
\begin{tcolorbox}
\textsubscript{26} Уста свои открывает с мудростью, и кроткое наставление на языке ее.
\end{tcolorbox}
\begin{tcolorbox}
\textsubscript{27} Она наблюдает за хозяйством в доме своем и не ест хлеба праздности.
\end{tcolorbox}
\begin{tcolorbox}
\textsubscript{28} Встают дети и ублажают ее, --муж, и хвалит ее:
\end{tcolorbox}
\begin{tcolorbox}
\textsubscript{29} 'много было жен добродетельных, но ты превзошла всех их'.
\end{tcolorbox}
\begin{tcolorbox}
\textsubscript{30} Миловидность обманчива и красота суетна; но жена, боящаяся Господа, достойна хвалы.
\end{tcolorbox}
\begin{tcolorbox}
\textsubscript{31} Дайте ей от плода рук ее, и да прославят ее у ворот дела ее!
\end{tcolorbox}
