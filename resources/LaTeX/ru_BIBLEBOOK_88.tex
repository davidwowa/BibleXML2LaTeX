\section{BOOK 87}
\subsection{CHAPTER 1}
\begin{tcolorbox}
\textsubscript{1} Слова Амоса, одного из пастухов Фекойских, которые он [слышал] в видении об Израиле во дни Озии, царя Иудейского, и во дни Иеровоама, сына Иоасова, царя Израильского, за два года перед землетрясением.
\end{tcolorbox}
\begin{tcolorbox}
\textsubscript{2} И сказал он: Господь возгремит с Сиона и даст глас Свой из Иерусалима, и восплачут хижины пастухов, и иссохнет вершина Кармила.
\end{tcolorbox}
\begin{tcolorbox}
\textsubscript{3} Так говорит Господь: за три преступления Дамаска и за четыре не пощажу его, потому что они молотили Галаад железными молотилами.
\end{tcolorbox}
\begin{tcolorbox}
\textsubscript{4} И пошлю огонь на дом Азаила, и пожрет он чертоги Венадада.
\end{tcolorbox}
\begin{tcolorbox}
\textsubscript{5} И сокрушу затворы Дамаска, и истреблю жителей долины Авен и держащего скипетр--из дома Еденова, и пойдет народ Арамейский в плен в Кир, говорит Господь.
\end{tcolorbox}
\begin{tcolorbox}
\textsubscript{6} Так говорит Господь: за три преступления Газы и за четыре не пощажу ее, потому что они вывели всех в плен, чтобы предать их Едому.
\end{tcolorbox}
\begin{tcolorbox}
\textsubscript{7} И пошлю огонь в стены Газы, --и пожрет чертоги ее.
\end{tcolorbox}
\begin{tcolorbox}
\textsubscript{8} И истреблю жителей Азота и держащего скипетр в Аскалоне, и обращу руку Мою на Екрон, и погибнет остаток Филистимлян, говорит Господь Бог.
\end{tcolorbox}
\begin{tcolorbox}
\textsubscript{9} Так говорит Господь: за три преступления Тира и за четыре не пощажу его, потому что они передали всех пленных Едому и не вспомнили братского союза.
\end{tcolorbox}
\begin{tcolorbox}
\textsubscript{10} Пошлю огонь в стены Тира, и пожрет чертоги его.
\end{tcolorbox}
\begin{tcolorbox}
\textsubscript{11} Так говорит Господь: за три преступления Едома и за четыре не пощажу его, потому что он преследовал брата своего мечом, подавил чувства родства, свирепствовал постоянно во гневе своем и всегда сохранял ярость свою.
\end{tcolorbox}
\begin{tcolorbox}
\textsubscript{12} И пошлю огонь на Феман, и пожрет чертоги Восора.
\end{tcolorbox}
\begin{tcolorbox}
\textsubscript{13} Так говорит Господь: за три преступления сынов Аммоновых и за четыре не пощажу их, потому что они рассекали беременных в Галааде, чтобы расширить пределы свои.
\end{tcolorbox}
\begin{tcolorbox}
\textsubscript{14} И запалю огонь в стенах Раввы, и пожрет чертоги ее, среди крика в день брани, с вихрем в день бури.
\end{tcolorbox}
\begin{tcolorbox}
\textsubscript{15} И пойдет царь их в плен, он и князья его вместе с ним, говорит Господь.
\end{tcolorbox}
\subsection{CHAPTER 2}
\begin{tcolorbox}
\textsubscript{1} Так говорит Господь: за три преступления Моава и за четыре не пощажу его, потому что он пережег кости царя Едомского в известь.
\end{tcolorbox}
\begin{tcolorbox}
\textsubscript{2} И пошлю огонь на Моава, и пожрет чертоги Кериофа, и погибнет Моав среди разгрома с шумом, при звуке трубы.
\end{tcolorbox}
\begin{tcolorbox}
\textsubscript{3} Истреблю судью из среды его и умерщвлю всех князей его вместе с ним, говорит Господь.
\end{tcolorbox}
\begin{tcolorbox}
\textsubscript{4} Так говорит Господь: за три преступления Иуды и за четыре не пощажу его, потому что отвергли закон Господень и постановлений Его не сохранили, и идолы их, вслед которых ходили отцы их, совратили их с пути.
\end{tcolorbox}
\begin{tcolorbox}
\textsubscript{5} И пошлю огонь на Иуду, и пожрет чертоги Иерусалима.
\end{tcolorbox}
\begin{tcolorbox}
\textsubscript{6} Так говорит Господь: за три преступления Израиля и за четыре не пощажу его, потому что продают правого за серебро и бедного--за пару сандалий.
\end{tcolorbox}
\begin{tcolorbox}
\textsubscript{7} Жаждут, чтобы прах земной был на голове бедных, и путь кротких извращают; даже отец и сын ходят к одной женщине, чтобы бесславить святое имя Мое.
\end{tcolorbox}
\begin{tcolorbox}
\textsubscript{8} На одеждах, взятых в залог, возлежат при всяком жертвеннике, и вино, [взыскиваемое] с обвиненных, пьют в доме богов своих.
\end{tcolorbox}
\begin{tcolorbox}
\textsubscript{9} А Я истребил перед лицем их Аморрея, которого высота была как высота кедра и который был крепок как дуб; Я истребил плод его вверху и корни его внизу.
\end{tcolorbox}
\begin{tcolorbox}
\textsubscript{10} Вас же Я вывел из земли Египетской и водил вас в пустыне сорок лет, чтобы вам наследовать землю Аморрейскую.
\end{tcolorbox}
\begin{tcolorbox}
\textsubscript{11} Из сыновей ваших Я избирал в пророки и из юношей ваших--в назореи; не так ли это, сыны Израиля? говорит Господь.
\end{tcolorbox}
\begin{tcolorbox}
\textsubscript{12} А вы назореев поили вином и пророкам приказывали, говоря: 'не пророчествуйте'.
\end{tcolorbox}
\begin{tcolorbox}
\textsubscript{13} Вот, Я придавлю вас, как давит колесница, нагруженная снопами, --
\end{tcolorbox}
\begin{tcolorbox}
\textsubscript{14} и у проворного не станет силы бежать, и крепкий не удержит крепости своей, и храбрый не спасет своей жизни,
\end{tcolorbox}
\begin{tcolorbox}
\textsubscript{15} ни стреляющий из лука не устоит, ни скороход не убежит, ни сидящий на коне не спасет своей жизни.
\end{tcolorbox}
\begin{tcolorbox}
\textsubscript{16} И самый отважный из храбрых убежит нагой в тот день, говорит Господь.
\end{tcolorbox}
\subsection{CHAPTER 3}
\begin{tcolorbox}
\textsubscript{1} Слушайте слово сие, которое Господь изрек на вас, сыны Израилевы, на все племя, которое вывел Я из земли Египетской, говоря:
\end{tcolorbox}
\begin{tcolorbox}
\textsubscript{2} только вас признал Я из всех племен земли, потому и взыщу с вас за все беззакония ваши.
\end{tcolorbox}
\begin{tcolorbox}
\textsubscript{3} Пойдут ли двое вместе, не сговорившись между собою?
\end{tcolorbox}
\begin{tcolorbox}
\textsubscript{4} Ревет ли лев в лесу, когда нет перед ним добычи? подает ли свой голос львенок из логовища своего, когда он ничего не поймал?
\end{tcolorbox}
\begin{tcolorbox}
\textsubscript{5} Попадет ли птица в петлю на земле, когда силка нет для нее? Поднимется ли с земли петля, когда ничего не попало в нее?
\end{tcolorbox}
\begin{tcolorbox}
\textsubscript{6} Трубит ли в городе труба, --и народ не испугался бы? Бывает ли в городе бедствие, которое не Господь попустил бы?
\end{tcolorbox}
\begin{tcolorbox}
\textsubscript{7} Ибо Господь Бог ничего не делает, не открыв Своей тайны рабам Своим, пророкам.
\end{tcolorbox}
\begin{tcolorbox}
\textsubscript{8} Лев начал рыкать, --кто не содрогнется? Господь Бог сказал, --кто не будет пророчествовать?
\end{tcolorbox}
\begin{tcolorbox}
\textsubscript{9} Провозгласите на кровлях в Азоте и на кровлях в земле Египетской и скажите: соберитесь на горы Самарии и посмотрите на великое бесчинство в ней и на притеснения среди нее.
\end{tcolorbox}
\begin{tcolorbox}
\textsubscript{10} Они не умеют поступать справедливо, говорит Господь: насилием и грабежом собирают сокровища в чертоги свои.
\end{tcolorbox}
\begin{tcolorbox}
\textsubscript{11} Посему так говорит Господь Бог: вот неприятель, и притом вокруг всей земли! он низложит могущество твое, и ограблены будут чертоги твои.
\end{tcolorbox}
\begin{tcolorbox}
\textsubscript{12} Так говорит Господь: как [иногда] пастух исторгает из пасти львиной две голени или часть уха, так спасены будут сыны Израилевы, сидящие в Самарии в углу постели и в Дамаске на ложе.
\end{tcolorbox}
\begin{tcolorbox}
\textsubscript{13} Слушайте и засвидетельствуйте дому Иакова, говорит Господь Бог, Бог Саваоф.
\end{tcolorbox}
\begin{tcolorbox}
\textsubscript{14} Ибо в тот день, когда Я взыщу с Израиля за преступления его, взыщу и за жертвенники в Вефиле, и отсечены будут роги алтаря, и падут на землю.
\end{tcolorbox}
\begin{tcolorbox}
\textsubscript{15} И поражу дом зимний вместе с домом летним, и исчезнут домы с украшениями из слоновой кости, и не станет многих домов, говорит Господь.
\end{tcolorbox}
\subsection{CHAPTER 4}
\begin{tcolorbox}
\textsubscript{1} Слушайте слово сие, телицы Васанские, которые на горе Самарийской, вы, притесняющие бедных, угнетающие нищих, говорящие господам своим: 'подавай, и мы будем пить!'
\end{tcolorbox}
\begin{tcolorbox}
\textsubscript{2} Клялся Господь Бог святостью Своею, что вот, придут на вас дни, когда повлекут вас крюками и остальных ваших удами.
\end{tcolorbox}
\begin{tcolorbox}
\textsubscript{3} И сквозь проломы стен выйдете, каждая, как случится, и бросите все убранство чертогов, говорит Господь.
\end{tcolorbox}
\begin{tcolorbox}
\textsubscript{4} Идите в Вефиль--и грешите, в Галгал--и умножайте преступления; приносите жертвы ваши каждое утро, десятины ваши хотя через каждые три дня.
\end{tcolorbox}
\begin{tcolorbox}
\textsubscript{5} Приносите в жертву благодарения квасное, провозглашайте о добровольных приношениях ваших и разглашайте о них, ибо это вы любите, сыны Израилевы, говорит Господь Бог.
\end{tcolorbox}
\begin{tcolorbox}
\textsubscript{6} За то и дал Я вам голые зубы во всех городах ваших и недостаток хлеба во всех селениях ваших; но вы не обратились ко Мне, говорит Господь.
\end{tcolorbox}
\begin{tcolorbox}
\textsubscript{7} И удерживал от вас дождь за три месяца до жатвы; проливал дождь на один город, а на другой город не проливал дождя; один участок напояем был дождем, а другой, не окропленный дождем, засыхал.
\end{tcolorbox}
\begin{tcolorbox}
\textsubscript{8} И сходились два-три города в один город, чтобы напиться воды, и не могли досыта напиться; но и тогда вы не обратились ко Мне, говорит Господь.
\end{tcolorbox}
\begin{tcolorbox}
\textsubscript{9} Я поражал вас ржою и блеклостью хлеба; множество садов ваших и виноградников ваших, и смоковниц ваших, и маслин ваших пожирала гусеница, --и при всем том вы не обратились ко Мне, говорит Господь.
\end{tcolorbox}
\begin{tcolorbox}
\textsubscript{10} Посылал Я на вас моровую язву, подобную Египетской, убивал мечом юношей ваших, отводя коней в плен, так что смрад от станов ваших поднимался в ноздри ваши; и при всем том вы не обратились ко Мне, говорит Господь.
\end{tcolorbox}
\begin{tcolorbox}
\textsubscript{11} Производил Я среди вас разрушения, как разрушил Бог Содом и Гоморру, и вы были выхвачены, как головня из огня, --и при всем том вы не обратились ко Мне, говорит Господь.
\end{tcolorbox}
\begin{tcolorbox}
\textsubscript{12} Посему так поступлю Я с тобою, Израиль; и как Я так поступлю с тобою, то приготовься к сретению Бога твоего, Израиль,
\end{tcolorbox}
\begin{tcolorbox}
\textsubscript{13} ибо вот Он, Который образует горы, и творит ветер, и объявляет человеку намерения его, утренний свет обращает в мрак, и шествует превыше земли; Господь Бог Саваоф--имя Ему.
\end{tcolorbox}
\subsection{CHAPTER 5}
\begin{tcolorbox}
\textsubscript{1} Слушайте это слово, в котором я подниму плач о вас, дом Израилев.
\end{tcolorbox}
\begin{tcolorbox}
\textsubscript{2} Упала, не встает более дева Израилева! повержена на земле своей, и некому поднять ее.
\end{tcolorbox}
\begin{tcolorbox}
\textsubscript{3} Ибо так говорит Господь Бог: город, выступавший тысячею, останется только с сотнею, и выступавший сотнею, останется с десятком у дома Израилева.
\end{tcolorbox}
\begin{tcolorbox}
\textsubscript{4} Ибо так говорит Господь дому Израилеву: взыщите Меня, и будете живы.
\end{tcolorbox}
\begin{tcolorbox}
\textsubscript{5} Не ищите Вефиля и не ходите в Галгал, и в Вирсавию не странствуйте, ибо Галгал весь пойдет в плен и Вефиль обратится в ничто.
\end{tcolorbox}
\begin{tcolorbox}
\textsubscript{6} Взыщите Господа, и будете живы, чтобы Он не устремился на дом Иосифов как огонь, который пожрет его, и некому будет погасить его в Вефиле.
\end{tcolorbox}
\begin{tcolorbox}
\textsubscript{7} О, вы, которые суд превращаете в отраву и правду повергаете на землю!
\end{tcolorbox}
\begin{tcolorbox}
\textsubscript{8} Кто сотворил семизвездие и Орион, и претворяет смертную тень в ясное утро, а день делает темным как ночь, призывает воды морские и разливает их по лицу земли? --Господь имя Ему!
\end{tcolorbox}
\begin{tcolorbox}
\textsubscript{9} Он укрепляет опустошителя против сильного, и опустошитель входит в крепость.
\end{tcolorbox}
\begin{tcolorbox}
\textsubscript{10} А они ненавидят обличающего в воротах и гнушаются тем, кто говорит правду.
\end{tcolorbox}
\begin{tcolorbox}
\textsubscript{11} Итак за то, что вы попираете бедного и берете от него подарки хлебом, вы построите домы из тесаных камней, но жить не будете в них; разведете прекрасные виноградники, а вино из них не будете пить.
\end{tcolorbox}
\begin{tcolorbox}
\textsubscript{12} Ибо Я знаю, как многочисленны преступления ваши и как тяжки грехи ваши: вы враги правого, берете взятки и извращаете в суде дела бедных.
\end{tcolorbox}
\begin{tcolorbox}
\textsubscript{13} Поэтому разумный безмолвствует в это время, ибо злое это время.
\end{tcolorbox}
\begin{tcolorbox}
\textsubscript{14} Ищите добра, а не зла, чтобы вам остаться в живых, --и тогда Господь Бог Саваоф будет с вами, как вы говорите.
\end{tcolorbox}
\begin{tcolorbox}
\textsubscript{15} Возненавидьте зло и возлюбите добро, и восстановите у ворот правосудие; может быть, Господь Бог Саваоф помилует остаток Иосифов.
\end{tcolorbox}
\begin{tcolorbox}
\textsubscript{16} Посему так говорит Господь Бог Саваоф, Вседержитель: на всех улицах будет плач, и на всех дорогах будут восклицать: 'увы, увы!', и призовут земледельца сетовать и искусных в плачевных песнях--плакать,
\end{tcolorbox}
\begin{tcolorbox}
\textsubscript{17} и во всех виноградниках будет плач, ибо Я пройду среди тебя, говорит Господь.
\end{tcolorbox}
\begin{tcolorbox}
\textsubscript{18} Горе желающим дня Господня! для чего вам этот день Господень? он тьма, а не свет,
\end{tcolorbox}
\begin{tcolorbox}
\textsubscript{19} то же, как если бы кто убежал от льва, и попался бы ему навстречу медведь, или если бы пришел домой и оперся рукою о стену, и змея ужалила бы его.
\end{tcolorbox}
\begin{tcolorbox}
\textsubscript{20} Разве день Господень не мрак, а свет? он тьма, и нет в нем сияния.
\end{tcolorbox}
\begin{tcolorbox}
\textsubscript{21} Ненавижу, отвергаю праздники ваши и не обоняю жертв во время торжественных собраний ваших.
\end{tcolorbox}
\begin{tcolorbox}
\textsubscript{22} Если вознесете Мне всесожжение и хлебное приношение, Я не приму их и не призрю на благодарственную жертву из тучных тельцов ваших.
\end{tcolorbox}
\begin{tcolorbox}
\textsubscript{23} Удали от Меня шум песней твоих, ибо звуков гуслей твоих Я не буду слушать.
\end{tcolorbox}
\begin{tcolorbox}
\textsubscript{24} Пусть, как вода, течет суд, и правда--как сильный поток!
\end{tcolorbox}
\begin{tcolorbox}
\textsubscript{25} Приносили ли вы Мне жертвы и хлебные дары в пустыне в течение сорока лет, дом Израилев?
\end{tcolorbox}
\begin{tcolorbox}
\textsubscript{26} Вы носили скинию Молохову и звезду бога вашего Ремфана, изображения, которые вы сделали для себя.
\end{tcolorbox}
\begin{tcolorbox}
\textsubscript{27} За то Я переселю вас за Дамаск, говорит Господь; Бог Саваоф--имя Ему!
\end{tcolorbox}
\subsection{CHAPTER 6}
\begin{tcolorbox}
\textsubscript{1} Горе беспечным на Сионе и надеющимся на гору Самарийскую именитым первенствующего народа, к которым приходит дом Израиля!
\end{tcolorbox}
\begin{tcolorbox}
\textsubscript{2} Пройдите в Калне и посмотрите, оттуда перейдите в Емаф великий и спуститесь в Геф Филистимский: не лучше ли они сих царств? не обширнее ли пределы их пределов ваших?
\end{tcolorbox}
\begin{tcolorbox}
\textsubscript{3} Вы, которые день бедствия считаете далеким и приближаете торжество насилия, --
\end{tcolorbox}
\begin{tcolorbox}
\textsubscript{4} вы, которые лежите на ложах из слоновой кости и нежитесь на постелях ваших, едите лучших овнов из стада и тельцов с тучного пастбища,
\end{tcolorbox}
\begin{tcolorbox}
\textsubscript{5} поете под звуки гуслей, думая, что владеете музыкальным орудием, как Давид,
\end{tcolorbox}
\begin{tcolorbox}
\textsubscript{6} пьете из чаш вино, мажетесь наилучшими мастями, и не болезнуете о бедствии Иосифа!
\end{tcolorbox}
\begin{tcolorbox}
\textsubscript{7} За то ныне пойдут они в плен во главе пленных, и кончится ликование изнеженных.
\end{tcolorbox}
\begin{tcolorbox}
\textsubscript{8} Клянется Господь Бог Самим Собою, и так говорит Господь Бог Саваоф: гнушаюсь высокомерием Иакова и ненавижу чертоги его, и предам город и все, что наполняет его.
\end{tcolorbox}
\begin{tcolorbox}
\textsubscript{9} И будет: если в каком доме останется десять человек, то умрут и они,
\end{tcolorbox}
\begin{tcolorbox}
\textsubscript{10} и возьмет их родственник их или сожигатель, чтобы вынести кости их из дома, и скажет находящемуся при доме: есть ли еще у тебя кто? Тот ответит: нет никого. И скажет сей: молчи! ибо нельзя упоминать имени Господня.
\end{tcolorbox}
\begin{tcolorbox}
\textsubscript{11} Ибо вот, Господь даст повеление и поразит большие дома расселинами, а малые дома--трещинами.
\end{tcolorbox}
\begin{tcolorbox}
\textsubscript{12} Бегают ли кони по скале? можно ли распахивать ее волами? Вы между тем суд превращаете в яд и плод правды в горечь;
\end{tcolorbox}
\begin{tcolorbox}
\textsubscript{13} вы, которые восхищаетесь ничтожными вещами и говорите: 'не своею ли силою мы приобрели себе могущество?'
\end{tcolorbox}
\begin{tcolorbox}
\textsubscript{14} Вот Я, говорит Господь Бог Саваоф, воздвигну народ против вас, дом Израилев, и будут теснить вас от входа в Емаф до потока в пустыне.
\end{tcolorbox}
\subsection{CHAPTER 7}
\begin{tcolorbox}
\textsubscript{1} Такое видение открыл мне Господь Бог: вот, Он создал саранчу в начале произрастания поздней травы, и это была трава после царского покоса.
\end{tcolorbox}
\begin{tcolorbox}
\textsubscript{2} И было, когда она окончила есть траву на земле, я сказал: Господи Боже! пощади; как устоит Иаков? он очень мал.
\end{tcolorbox}
\begin{tcolorbox}
\textsubscript{3} И пожалел Господь о том; 'не будет сего', сказал Господь.
\end{tcolorbox}
\begin{tcolorbox}
\textsubscript{4} Такое видение открыл мне Господь Бог: вот, Господь Бог произвел для суда огонь, --и он пожрал великую пучину, пожрал и часть земли.
\end{tcolorbox}
\begin{tcolorbox}
\textsubscript{5} И сказал я: Господи Боже! останови; как устоит Иаков? он очень мал.
\end{tcolorbox}
\begin{tcolorbox}
\textsubscript{6} И пожалел Господь о том; 'и этого не будет', сказал Господь Бог.
\end{tcolorbox}
\begin{tcolorbox}
\textsubscript{7} Такое видение открыл Он мне: вот, Господь стоял на отвесной стене, и в руке у Него свинцовый отвес.
\end{tcolorbox}
\begin{tcolorbox}
\textsubscript{8} И сказал мне Господь: что ты видишь, Амос? Я ответил: отвес. И Господь сказал: вот, положу отвес среди народа Моего, Израиля; не буду более прощать ему.
\end{tcolorbox}
\begin{tcolorbox}
\textsubscript{9} И опустошены будут [жертвенные] высоты Исааковы, и разрушены будут святилища Израилевы, и восстану с мечом против дома Иеровоамова.
\end{tcolorbox}
\begin{tcolorbox}
\textsubscript{10} И послал Амасия, священник Вефильский, к Иеровоаму, царю Израильскому, сказать: Амос производит возмущение против тебя среди дома Израилева; земля не может терпеть всех слов его.
\end{tcolorbox}
\begin{tcolorbox}
\textsubscript{11} Ибо так говорит Амос: 'от меча умрет Иеровоам, а Израиль непременно отведен будет пленным из земли своей'.
\end{tcolorbox}
\begin{tcolorbox}
\textsubscript{12} И сказал Амасия Амосу: провидец! пойди и удались в землю Иудину; там ешь хлеб, и там пророчествуй,
\end{tcolorbox}
\begin{tcolorbox}
\textsubscript{13} а в Вефиле больше не пророчествуй, ибо он святыня царя и дом царский.
\end{tcolorbox}
\begin{tcolorbox}
\textsubscript{14} И отвечал Амос и сказал Амасии: я не пророк и не сын пророка; я был пастух и собирал сикоморы.
\end{tcolorbox}
\begin{tcolorbox}
\textsubscript{15} Но Господь взял меня от овец и сказал мне Господь: 'иди, пророчествуй к народу Моему, Израилю'.
\end{tcolorbox}
\begin{tcolorbox}
\textsubscript{16} Теперь выслушай слово Господне. Ты говоришь: 'не пророчествуй на Израиля и не произноси слов на дом Исааков'.
\end{tcolorbox}
\begin{tcolorbox}
\textsubscript{17} За это, вот что говорит Господь: жена твоя будет обесчещена в городе, сыновья и дочери твои падут от меча, земля твоя будет разделена межевою вервью, а ты умрешь в земле нечистой, и Израиль непременно выведен будет из земли своей.
\end{tcolorbox}
\subsection{CHAPTER 8}
\begin{tcolorbox}
\textsubscript{1} Такое видение открыл мне Господь Бог: вот корзина со спелыми плодами.
\end{tcolorbox}
\begin{tcolorbox}
\textsubscript{2} И сказал Он: что ты видишь, Амос? Я ответил: корзину со спелыми плодами. Тогда Господь сказал мне: приспел конец народу Моему, Израилю: не буду более прощать ему.
\end{tcolorbox}
\begin{tcolorbox}
\textsubscript{3} Песни чертога в тот день обратятся в рыдание, говорит Господь Бог; много будет трупов, на всяком месте будут бросать их молча.
\end{tcolorbox}
\begin{tcolorbox}
\textsubscript{4} Выслушайте это, алчущие поглотить бедных и погубить нищих, --
\end{tcolorbox}
\begin{tcolorbox}
\textsubscript{5} вы, которые говорите: 'когда-то пройдет новолуние, чтобы нам продавать хлеб, и суббота, чтобы открыть житницы, уменьшить меру, увеличить цену сикля и обманывать неверными весами,
\end{tcolorbox}
\begin{tcolorbox}
\textsubscript{6} чтобы покупать неимущих за серебро и бедных за пару обуви, а высевки из хлеба продавать'.
\end{tcolorbox}
\begin{tcolorbox}
\textsubscript{7} Клялся Господь славою Иакова: поистине во веки не забуду ни одного из дел их!
\end{tcolorbox}
\begin{tcolorbox}
\textsubscript{8} Не поколеблется ли от этого земля, и не восплачет ли каждый, живущий на ней? Взволнуется вся она, как река, и будет подниматься и опускаться, как река Египетская.
\end{tcolorbox}
\begin{tcolorbox}
\textsubscript{9} И будет в тот день, говорит Господь Бог: произведу закат солнца в полдень и омрачу землю среди светлого дня.
\end{tcolorbox}
\begin{tcolorbox}
\textsubscript{10} И обращу праздники ваши в сетование и все песни ваши в плач, и возложу на все чресла вретище и плешь на всякую голову; и произведу [в] [стране] плач, как о единственном сыне, и конец ее будет--как горький день.
\end{tcolorbox}
\begin{tcolorbox}
\textsubscript{11} Вот наступают дни, говорит Господь Бог, когда Я пошлю на землю голод, --не голод хлеба, не жажду воды, но жажду слышания слов Господних.
\end{tcolorbox}
\begin{tcolorbox}
\textsubscript{12} И будут ходить от моря до моря и скитаться от севера к востоку, ища слова Господня, и не найдут его.
\end{tcolorbox}
\begin{tcolorbox}
\textsubscript{13} В тот день истаявать будут от жажды красивые девы и юноши,
\end{tcolorbox}
\begin{tcolorbox}
\textsubscript{14} которые клянутся грехом Самарийским и говорят: 'жив бог твой, Дан! и жив путь в Вирсавию!' --Они падут и уже не встанут.
\end{tcolorbox}
\subsection{CHAPTER 9}
\begin{tcolorbox}
\textsubscript{1} Видел я Господа стоящим над жертвенником, и Он сказал: ударь в притолоку над воротами, чтобы потряслись косяки, и обрушь их на головы всех их, остальных же из них Я поражу мечом: не убежит у них никто бегущий и не спасется из них никто, желающий спастись.
\end{tcolorbox}
\begin{tcolorbox}
\textsubscript{2} Хотя бы они зарылись в преисподнюю, и оттуда рука Моя возьмет их; хотя бы взошли на небо, и оттуда свергну их.
\end{tcolorbox}
\begin{tcolorbox}
\textsubscript{3} И хотя бы они скрылись на вершине Кармила, и там отыщу и возьму их; хотя бы сокрылись от очей Моих на дне моря, и там повелю морскому змею уязвить их.
\end{tcolorbox}
\begin{tcolorbox}
\textsubscript{4} И если пойдут в плен впереди врагов своих, то повелю мечу и там убить их. Обращу на них очи Мои на беду им, а не во благо.
\end{tcolorbox}
\begin{tcolorbox}
\textsubscript{5} Ибо Господь Бог Саваоф коснется земли, --и она растает, и восплачут все живущие на ней; и поднимется вся она как река, и опустится как река Египетская.
\end{tcolorbox}
\begin{tcolorbox}
\textsubscript{6} Он устроил горние чертоги Свои на небесах и свод Свой утвердил на земле, призывает воды морские, и изливает их по лицу земли; Господь имя Ему.
\end{tcolorbox}
\begin{tcolorbox}
\textsubscript{7} Не таковы ли, как сыны Ефиоплян, и вы для Меня, сыны Израилевы? говорит Господь. Не Я ли вывел Израиля из земли Египетской и Филистимлян--из Кафтора, и Арамлян--из Кира?
\end{tcolorbox}
\begin{tcolorbox}
\textsubscript{8} Вот, очи Господа Бога--на грешное царство, и Я истреблю его с лица земли; но дом Иакова не совсем истреблю, говорит Господь.
\end{tcolorbox}
\begin{tcolorbox}
\textsubscript{9} Ибо вот, Я повелю и рассыплю дом Израилев по всем народам, как рассыпают зерна в решете, и ни одно не падает на землю.
\end{tcolorbox}
\begin{tcolorbox}
\textsubscript{10} От меча умрут все грешники из народа Моего, которые говорят: 'не постигнет нас и не придет к нам это бедствие!'
\end{tcolorbox}
\begin{tcolorbox}
\textsubscript{11} В тот день Я восстановлю скинию Давидову падшую, заделаю трещины в ней и разрушенное восстановлю, и устрою ее, как в дни древние,
\end{tcolorbox}
\begin{tcolorbox}
\textsubscript{12} чтобы они овладели остатком Едома и всеми народами, между которыми возвестится имя Мое, говорит Господь, творящий все сие.
\end{tcolorbox}
\begin{tcolorbox}
\textsubscript{13} Вот, наступят дни, говорит Господь, когда пахарь застанет еще жнеца, а топчущий виноград--сеятеля; и горы источать будут виноградный сок, и все холмы потекут.
\end{tcolorbox}
\begin{tcolorbox}
\textsubscript{14} И возвращу из плена народ Мой, Израиля, и застроят опустевшие города и поселятся в них, насадят виноградники и будут пить вино из них, разведут сады и станут есть плоды из них.
\end{tcolorbox}
\begin{tcolorbox}
\textsubscript{15} И водворю их на земле их, и они не будут более исторгаемы из земли своей, которую Я дал им, говорит Господь Бог твой.
\end{tcolorbox}
