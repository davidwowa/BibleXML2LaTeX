Book 52
\subsection{CHAPTER 1}
\begin{tcolorbox}
\textsubscript{1} Павло, і Силуан, і Тимофій до Солунської Церкви в нашім Бозі Отці й Господі Ісусі Христі:
\end{tcolorbox}
\begin{tcolorbox}
\textsubscript{2} благодать вам і мир від Бога Отця й Господа Ісуса Христа!
\end{tcolorbox}
\begin{tcolorbox}
\textsubscript{3} Ми завжди повинні подяку складати за вас Богові, браття, як і годиться, бо сильно росте віра ваша, і примножується любов кожного з усіх вас один до одного.
\end{tcolorbox}
\begin{tcolorbox}
\textsubscript{4} Так що ми самі хвалимось вами по Божих Церквах за ваші страждання та віру в усіх переслідуваннях ваших та в утисках, що їх переносите ви.
\end{tcolorbox}
\begin{tcolorbox}
\textsubscript{5} А це доказ праведного Божого суду, щоб стали ви гідні Божого Царства, що за нього й страждаєте ви!
\end{tcolorbox}
\begin{tcolorbox}
\textsubscript{6} Бо то справедливе в Бога віддати утиском тим, хто вас утискає,
\end{tcolorbox}
\begin{tcolorbox}
\textsubscript{7} а вам, хто утиски терпить, відпочинок із нами, коли з'явиться з неба Господь Ісус з Анголами сили Своєї,
\end{tcolorbox}
\begin{tcolorbox}
\textsubscript{8} в огні полум'яному, що даватиме помсту на тих, хто Бога не знає, і не слухає Євангелії Господа нашого Ісуса.
\end{tcolorbox}
\begin{tcolorbox}
\textsubscript{9} Вони кару приймуть, вічну погибіль від лиця Господнього та від слави потуги Його,
\end{tcolorbox}
\begin{tcolorbox}
\textsubscript{10} як Він прийде того дня прославитися в Своїх святих, і стати дивним у всіх віруючих, бо свідчення наше знайшло віру між вами.
\end{tcolorbox}
\begin{tcolorbox}
\textsubscript{11} За це ми й молимось завжди за вас, щоб наш Бог учинив вас гідними покликання, і міццю наповнив усю добру волю добрости й діло віри,
\end{tcolorbox}
\begin{tcolorbox}
\textsubscript{12} щоб прославилося Ім'я Господа нашого Ісуса в вас, а ви в Ньому, за благодаттю Бога нашого й Господа Ісуса Христа.
\end{tcolorbox}
\subsection{CHAPTER 2}
\begin{tcolorbox}
\textsubscript{1} Благаємо ж, браття, ми вас, щодо приходу Господа нашого Ісуса Христа й нашого згромадження до Нього,
\end{tcolorbox}
\begin{tcolorbox}
\textsubscript{2} щоб ви не хвилювалися зараз умом та не жахались ані через духа, ані через слово, ані через листа, що він ніби від нас, ніби вже настав день Господній.
\end{tcolorbox}
\begin{tcolorbox}
\textsubscript{3} Хай ніхто жадним способом вас не зведе! Бо той день не настане, аж перше прийде відступлення, і виявиться беззаконник, призначений на погибіль,
\end{tcolorbox}
\begin{tcolorbox}
\textsubscript{4} що противиться та несеться над усе, зване Богом чи святощами, так що в Божому храмі він сяде, як Бог, і за Бога себе видаватиме.
\end{tcolorbox}
\begin{tcolorbox}
\textsubscript{5} Чи ви не пам'ятаєте, як, ще в вас живши, я це вам говорив був?
\end{tcolorbox}
\begin{tcolorbox}
\textsubscript{6} І тепер ви знаєте, що саме не допускає з'явитись йому своєчасно.
\end{tcolorbox}
\begin{tcolorbox}
\textsubscript{7} Бо вже діється таємниця беззаконня; тільки той, хто тримає тепер, буде тримати, аж поки не буде усунений він із середини.
\end{tcolorbox}
\begin{tcolorbox}
\textsubscript{8} І тоді то з'явиться той беззаконник, що його Господь Ісус заб'є Духом уст Своїх і знищить з'явленням приходу Свого.
\end{tcolorbox}
\begin{tcolorbox}
\textsubscript{9} Його прихід за чином сатани буде з усякою силою й знаками та з неправдивими чудами,
\end{tcolorbox}
\begin{tcolorbox}
\textsubscript{10} і з усякою обманою неправди між тими, хто гине, бо любови правди вони не прийняли, щоб їм спастися.
\end{tcolorbox}
\begin{tcolorbox}
\textsubscript{11} І за це Бог пошле їм дію обмани, щоб у неправду повірили,
\end{tcolorbox}
\begin{tcolorbox}
\textsubscript{12} щоб стали засуджені всі, хто не вірив у правду, але полюбив неправду.
\end{tcolorbox}
\begin{tcolorbox}
\textsubscript{13} А ми завжди повинні дякувати Богові за вас, улюблені Господом браття, що Бог вибрав вас спочатку на спасіння освяченням Духа та вірою в правду,
\end{tcolorbox}
\begin{tcolorbox}
\textsubscript{14} до чого покликав Він вас через нашу Євангелію, щоб отримати славу Господа нашого Ісуса Христа.
\end{tcolorbox}
\begin{tcolorbox}
\textsubscript{15} Отже, браття, стійте й тримайтеся передань, яких ви навчились чи то словом, чи нашим посланням.
\end{tcolorbox}
\begin{tcolorbox}
\textsubscript{16} Сам же Господь наш Ісус Христос і Бог Отець наш, що нас полюбив і дав у благодаті вічну потіху та добру надію,
\end{tcolorbox}
\begin{tcolorbox}
\textsubscript{17} нехай ваші серця Він потішить, і нехай Він зміцнить вас у всякому доброму ділі та в слові!
\end{tcolorbox}
\subsection{CHAPTER 3}
\begin{tcolorbox}
\textsubscript{1} Наостанку, моліться, браття, за нас, щоб ширилось Слово Господнє та славилось, як і в вас,
\end{tcolorbox}
\begin{tcolorbox}
\textsubscript{2} і щоб ми визволилися від злих та лукавих людей, бо віра не в усіх.
\end{tcolorbox}
\begin{tcolorbox}
\textsubscript{3} І вірний Господь, що зміцнить вас і збереже від лукавого.
\end{tcolorbox}
\begin{tcolorbox}
\textsubscript{4} А про вас покладаємо надію на Господа, що й чините ви, і чинити будете те, що наказуємо вам.
\end{tcolorbox}
\begin{tcolorbox}
\textsubscript{5} Господь же нехай серця ваші спрямує на Божу любов та терпеливість Христову!
\end{tcolorbox}
\begin{tcolorbox}
\textsubscript{6} А ми вам наказуємо, браття, Ім'ям Господа Ісуса Христа, щоб ви цуралися кожного брата, що живе по-ледачому, а не за переданням, яке прийняли ви від нас.
\end{tcolorbox}
\begin{tcolorbox}
\textsubscript{7} Самі бо ви знаєте, як належить наслідувати нас. Бо ми поміж вами не сидні справляли,
\end{tcolorbox}
\begin{tcolorbox}
\textsubscript{8} і хліба не їли ні в кого даремно, але в перевтомі й напруженні день і ніч працювали, щоб не бути нікому із вас тягарем,
\end{tcolorbox}
\begin{tcolorbox}
\textsubscript{9} не тому, щоб ми влади не мали, але щоб себе за взірця дати вам, щоб нас ви наслідували.
\end{tcolorbox}
\begin{tcolorbox}
\textsubscript{10} Бо коли ми в вас перебували, то це вам наказували, що як хто працювати не хоче, нехай той не їсть!
\end{tcolorbox}
\begin{tcolorbox}
\textsubscript{11} Бо ми чуємо, що дехто між вами живуть по-ледачому, нічого не роблять, а тільки вдають, ніби роблять.
\end{tcolorbox}
\begin{tcolorbox}
\textsubscript{12} Таким ми наказуємо та благаємо Господом нашим Ісусом Христом, щоб мовчки вони працювали та власний хліб їли.
\end{tcolorbox}
\begin{tcolorbox}
\textsubscript{13} А ви, браття, не втомлюйтеся, коли чините добре.
\end{tcolorbox}
\begin{tcolorbox}
\textsubscript{14} Коли ж хто не послухає нашого слова через цього листа, зауважте того, і не майте з ним зносин, щоб він був посоромлений.
\end{tcolorbox}
\begin{tcolorbox}
\textsubscript{15} Та не майте його за неприятеля, а навчайте, як брата.
\end{tcolorbox}
\begin{tcolorbox}
\textsubscript{16} А Сам Господь миру нехай завжди дасть вам мир усяким способом. Господь з вами всіма!
\end{tcolorbox}
\begin{tcolorbox}
\textsubscript{17} Привіт вам моєю рукою Павловою, це править за знака в усякім листі. Так пишу я.
\end{tcolorbox}
\begin{tcolorbox}
\textsubscript{18} Благодать Господа нашого Ісуса Христа нехай буде з вами всіма! Амінь.
\end{tcolorbox}
