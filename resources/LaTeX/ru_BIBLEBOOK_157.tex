\section{BOOK 156}
\subsection{CHAPTER 1}
\begin{tcolorbox}
\textsubscript{1} Павел и Силуан и Тимофей--Фессалоникской церкви в Боге Отце нашем и Господе Иисусе Христе:
\end{tcolorbox}
\begin{tcolorbox}
\textsubscript{2} благодать вам и мир от Бога Отца нашего и Господа Иисуса Христа.
\end{tcolorbox}
\begin{tcolorbox}
\textsubscript{3} Всегда по справедливости мы должны благодарить Бога за вас, братия, потому что возрастает вера ваша, и умножается любовь каждого друг ко другу между всеми вами,
\end{tcolorbox}
\begin{tcolorbox}
\textsubscript{4} так что мы сами хвалимся вами в церквах Божиих, терпением вашим и верою во всех гонениях и скорбях, переносимых вами
\end{tcolorbox}
\begin{tcolorbox}
\textsubscript{5} в доказательство того, что будет праведный суд Божий, чтобы вам удостоиться Царствия Божия, для которого и страдаете.
\end{tcolorbox}
\begin{tcolorbox}
\textsubscript{6} Ибо праведно пред Богом--оскорбляющим вас воздать скорбью,
\end{tcolorbox}
\begin{tcolorbox}
\textsubscript{7} а вам, оскорбляемым, отрадою вместе с нами, в явление Господа Иисуса с неба, с Ангелами силы Его,
\end{tcolorbox}
\begin{tcolorbox}
\textsubscript{8} в пламенеющем огне совершающего отмщение не познавшим Бога и не покоряющимся благовествованию Господа нашего Иисуса Христа,
\end{tcolorbox}
\begin{tcolorbox}
\textsubscript{9} которые подвергнутся наказанию, вечной погибели, от лица Господа и от славы могущества Его,
\end{tcolorbox}
\begin{tcolorbox}
\textsubscript{10} когда Он приидет прославиться во святых Своих и явиться дивным в день оный во всех веровавших, так как вы поверили нашему свидетельству.
\end{tcolorbox}
\begin{tcolorbox}
\textsubscript{11} Для сего и молимся всегда за вас, чтобы Бог наш соделал вас достойными звания и совершил всякое благоволение благости и дело веры в силе,
\end{tcolorbox}
\begin{tcolorbox}
\textsubscript{12} да прославится имя Господа нашего Иисуса Христа в вас, и вы в Нем, по благодати Бога нашего и Господа Иисуса Христа.
\end{tcolorbox}
\subsection{CHAPTER 2}
\begin{tcolorbox}
\textsubscript{1} Молим вас, братия, о пришествии Господа нашего Иисуса Христа и нашем собрании к Нему,
\end{tcolorbox}
\begin{tcolorbox}
\textsubscript{2} не спешить колебаться умом и смущаться ни от духа, ни от слова, ни от послания, как бы нами посланного, будто уже наступает день Христов.
\end{tcolorbox}
\begin{tcolorbox}
\textsubscript{3} Да не обольстит вас никто никак: [ибо день тот не] [придет], доколе не придет прежде отступление и не откроется человек греха, сын погибели,
\end{tcolorbox}
\begin{tcolorbox}
\textsubscript{4} противящийся и превозносящийся выше всего, называемого Богом или святынею, так что в храме Божием сядет он, как Бог, выдавая себя за Бога.
\end{tcolorbox}
\begin{tcolorbox}
\textsubscript{5} Не помните ли, что я, еще находясь у вас, говорил вам это?
\end{tcolorbox}
\begin{tcolorbox}
\textsubscript{6} И ныне вы знаете, что не допускает открыться ему в свое время.
\end{tcolorbox}
\begin{tcolorbox}
\textsubscript{7} Ибо тайна беззакония уже в действии, только [не совершится] до тех пор, пока не будет взят от среды удерживающий теперь.
\end{tcolorbox}
\begin{tcolorbox}
\textsubscript{8} И тогда откроется беззаконник, которого Господь Иисус убьет духом уст Своих и истребит явлением пришествия Своего
\end{tcolorbox}
\begin{tcolorbox}
\textsubscript{9} того, которого пришествие, по действию сатаны, будет со всякою силою и знамениями и чудесами ложными,
\end{tcolorbox}
\begin{tcolorbox}
\textsubscript{10} и со всяким неправедным обольщением погибающих за то, что они не приняли любви истины для своего спасения.
\end{tcolorbox}
\begin{tcolorbox}
\textsubscript{11} И за сие пошлет им Бог действие заблуждения, так что они будут верить лжи,
\end{tcolorbox}
\begin{tcolorbox}
\textsubscript{12} да будут осуждены все, не веровавшие истине, но возлюбившие неправду.
\end{tcolorbox}
\begin{tcolorbox}
\textsubscript{13} Мы же всегда должны благодарить Бога за вас, возлюбленные Господом братия, что Бог от начала, через освящение Духа и веру истине, избрал вас ко спасению,
\end{tcolorbox}
\begin{tcolorbox}
\textsubscript{14} к которому и призвал вас благовествованием нашим, для достижения славы Господа нашего Иисуса Христа.
\end{tcolorbox}
\begin{tcolorbox}
\textsubscript{15} Итак, братия, стойте и держите предания, которым вы научены или словом или посланием нашим.
\end{tcolorbox}
\begin{tcolorbox}
\textsubscript{16} Сам же Господь наш Иисус Христос и Бог и Отец наш, возлюбивший нас и давший утешение вечное и надежду благую во благодати,
\end{tcolorbox}
\begin{tcolorbox}
\textsubscript{17} да утешит ваши сердца и да утвердит вас во всяком слове и деле благом.
\end{tcolorbox}
\subsection{CHAPTER 3}
\begin{tcolorbox}
\textsubscript{1} Итак молитесь за нас, братия, чтобы слово Господне распространялось и прославлялось, как и у вас,
\end{tcolorbox}
\begin{tcolorbox}
\textsubscript{2} и чтобы нам избавиться от беспорядочных и лукавых людей, ибо не во всех вера.
\end{tcolorbox}
\begin{tcolorbox}
\textsubscript{3} Но верен Господь, Который утвердит вас и сохранит от лукавого.
\end{tcolorbox}
\begin{tcolorbox}
\textsubscript{4} Мы уверены о вас в Господе, что вы исполняете и будете исполнять то, что мы вам повелеваем.
\end{tcolorbox}
\begin{tcolorbox}
\textsubscript{5} Господь же да управит сердца ваши в любовь Божию и в терпение Христово.
\end{tcolorbox}
\begin{tcolorbox}
\textsubscript{6} Завещеваем же вам, братия, именем Господа нашего Иисуса Христа, удаляться от всякого брата, поступающего бесчинно, а не по преданию, которое приняли от нас,
\end{tcolorbox}
\begin{tcolorbox}
\textsubscript{7} ибо вы сами знаете, как должны вы подражать нам; ибо мы не бесчинствовали у вас,
\end{tcolorbox}
\begin{tcolorbox}
\textsubscript{8} ни у кого не ели хлеба даром, но занимались трудом и работою ночь и день, чтобы не обременить кого из вас, --
\end{tcolorbox}
\begin{tcolorbox}
\textsubscript{9} не потому, чтобы мы не имели власти, но чтобы себя самих дать вам в образец для подражания нам.
\end{tcolorbox}
\begin{tcolorbox}
\textsubscript{10} Ибо когда мы были у вас, то завещевали вам сие: если кто не хочет трудиться, тот и не ешь.
\end{tcolorbox}
\begin{tcolorbox}
\textsubscript{11} Но слышим, что некоторые у вас поступают бесчинно, ничего не делают, а суетятся.
\end{tcolorbox}
\begin{tcolorbox}
\textsubscript{12} Таковых увещеваем и убеждаем Господом нашим Иисусом Христом, чтобы они, работая в безмолвии, ели свой хлеб.
\end{tcolorbox}
\begin{tcolorbox}
\textsubscript{13} Вы же, братия, не унывайте, делая добро.
\end{tcolorbox}
\begin{tcolorbox}
\textsubscript{14} Если же кто не послушает слова нашего в сем послании, того имейте на замечании и не сообщайтесь с ним, чтобы устыдить его.
\end{tcolorbox}
\begin{tcolorbox}
\textsubscript{15} Но не считайте его за врага, а вразумляйте, как брата.
\end{tcolorbox}
\begin{tcolorbox}
\textsubscript{16} Сам же Господь мира да даст вам мир всегда во всем. Господь со всеми вами!
\end{tcolorbox}
\begin{tcolorbox}
\textsubscript{17} Приветствие моею рукою, Павловою, что служит знаком во всяком послании; пишу я так:
\end{tcolorbox}
\begin{tcolorbox}
\textsubscript{18} благодать Господа нашего Иисуса Христа со всеми вами. Аминь.
\end{tcolorbox}
