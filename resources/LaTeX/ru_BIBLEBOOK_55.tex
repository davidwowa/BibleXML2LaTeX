\section{BOOK 54}
\subsection{CHAPTER 1}
\begin{tcolorbox}
\textsubscript{1} Блажен муж, который не ходит на совет нечестивых и не стоит на пути грешных и не сидит в собрании развратителей,
\end{tcolorbox}
\begin{tcolorbox}
\textsubscript{2} но в законе Господа воля его, и о законе Его размышляет он день и ночь!
\end{tcolorbox}
\begin{tcolorbox}
\textsubscript{3} И будет он как дерево, посаженное при потоках вод, которое приносит плод свой во время свое, и лист которого не вянет; и во всем, что он ни делает, успеет.
\end{tcolorbox}
\begin{tcolorbox}
\textsubscript{4} Не так--нечестивые; но они--как прах, возметаемый ветром.
\end{tcolorbox}
\begin{tcolorbox}
\textsubscript{5} Потому не устоят нечестивые на суде, и грешники--в собрании праведных.
\end{tcolorbox}
\begin{tcolorbox}
\textsubscript{6} Ибо знает Господь путь праведных, а путь нечестивых погибнет.
\end{tcolorbox}
\subsection{CHAPTER 2}
\begin{tcolorbox}
\textsubscript{1} ^^Псалом Давида.^^ Зачем мятутся народы, и племена замышляют тщетное?
\end{tcolorbox}
\begin{tcolorbox}
\textsubscript{2} Восстают цари земли, и князья совещаются вместе против Господа и против Помазанника Его.
\end{tcolorbox}
\begin{tcolorbox}
\textsubscript{3} 'Расторгнем узы их, и свергнем с себя оковы их'.
\end{tcolorbox}
\begin{tcolorbox}
\textsubscript{4} Живущий на небесах посмеется, Господь поругается им.
\end{tcolorbox}
\begin{tcolorbox}
\textsubscript{5} Тогда скажет им во гневе Своем и яростью Своею приведет их в смятение:
\end{tcolorbox}
\begin{tcolorbox}
\textsubscript{6} 'Я помазал Царя Моего над Сионом, святою горою Моею;
\end{tcolorbox}
\begin{tcolorbox}
\textsubscript{7} возвещу определение: Господь сказал Мне: Ты Сын Мой; Я ныне родил Тебя;
\end{tcolorbox}
\begin{tcolorbox}
\textsubscript{8} проси у Меня, и дам народы в наследие Тебе и пределы земли во владение Тебе;
\end{tcolorbox}
\begin{tcolorbox}
\textsubscript{9} Ты поразишь их жезлом железным; сокрушишь их, как сосуд горшечника'.
\end{tcolorbox}
\begin{tcolorbox}
\textsubscript{10} Итак вразумитесь, цари; научитесь, судьи земли!
\end{tcolorbox}
\begin{tcolorbox}
\textsubscript{11} Служите Господу со страхом и радуйтесь с трепетом.
\end{tcolorbox}
\begin{tcolorbox}
\textsubscript{12} Почтите Сына, чтобы Он не прогневался, и чтобы вам не погибнуть в пути [вашем], ибо гнев Его возгорится вскоре. Блаженны все, уповающие на Него.
\end{tcolorbox}
\subsection{CHAPTER 3}
\begin{tcolorbox}
\textsubscript{1} ^^Псалом Давида, когда он бежал от Авессалома, сына своего.^^ (3-2) Господи! как умножились враги мои! Многие восстают на меня
\end{tcolorbox}
\begin{tcolorbox}
\textsubscript{2} (3-3) многие говорят душе моей: 'нет ему спасения в Боге'.
\end{tcolorbox}
\begin{tcolorbox}
\textsubscript{3} (3-4) Но Ты, Господи, щит предо мною, слава моя, и Ты возносишь голову мою.
\end{tcolorbox}
\begin{tcolorbox}
\textsubscript{4} (3-5) Гласом моим взываю к Господу, и Он слышит меня со святой горы Своей.
\end{tcolorbox}
\begin{tcolorbox}
\textsubscript{5} (3-6) Ложусь я, сплю и встаю, ибо Господь защищает меня.
\end{tcolorbox}
\begin{tcolorbox}
\textsubscript{6} (3-7) Не убоюсь тем народа, которые со всех сторон ополчились на меня.
\end{tcolorbox}
\begin{tcolorbox}
\textsubscript{7} (3-8) Восстань, Господи! спаси меня, Боже мой! ибо Ты поражаешь в ланиту всех врагов моих; сокрушаешь зубы нечестивых.
\end{tcolorbox}
\begin{tcolorbox}
\textsubscript{8} (3-9) От Господа спасение. Над народом Твоим благословение Твое.
\end{tcolorbox}
\subsection{CHAPTER 4}
\begin{tcolorbox}
\textsubscript{1} ^^Начальнику хора. На струнных [орудиях]. Псалом Давида.^^ (4-2) Когда я взываю, услышь меня, Боже правды моей! В тесноте Ты давал мне простор. Помилуй меня и услышь молитву мою.
\end{tcolorbox}
\begin{tcolorbox}
\textsubscript{2} (4-3) Сыны мужей! доколе слава моя будет в поругании? доколе будете любить суету и искать лжи?
\end{tcolorbox}
\begin{tcolorbox}
\textsubscript{3} (4-4) Знайте, что Господь отделил для Себя святаго Своего; Господь слышит, когда я призываю Его.
\end{tcolorbox}
\begin{tcolorbox}
\textsubscript{4} (4-5) Гневаясь, не согрешайте: размыслите в сердцах ваших на ложах ваших, и утишитесь;
\end{tcolorbox}
\begin{tcolorbox}
\textsubscript{5} (4-6) приносите жертвы правды и уповайте на Господа.
\end{tcolorbox}
\begin{tcolorbox}
\textsubscript{6} (4-7) Многие говорят: 'кто покажет нам благо?' Яви нам свет лица Твоего, Господи!
\end{tcolorbox}
\begin{tcolorbox}
\textsubscript{7} (4-8) Ты исполнил сердце мое веселием с того времени, как у них хлеб и вино умножились.
\end{tcolorbox}
\begin{tcolorbox}
\textsubscript{8} (4-9) Спокойно ложусь я и сплю, ибо Ты, Господи, един даешь мне жить в безопасности.
\end{tcolorbox}
\subsection{CHAPTER 5}
\begin{tcolorbox}
\textsubscript{1} ^^Начальнику хора. На духовых [орудиях]. Псалом Давида.^^ (5-2) Услышь, Господи, слова мои, уразумей помышления мои.
\end{tcolorbox}
\begin{tcolorbox}
\textsubscript{2} (5-3) Внемли гласу вопля моего, Царь мой и Бог мой! ибо я к Тебе молюсь.
\end{tcolorbox}
\begin{tcolorbox}
\textsubscript{3} (5-4) Господи! рано услышь голос мой, --рано предстану пред Тобою, и буду ожидать,
\end{tcolorbox}
\begin{tcolorbox}
\textsubscript{4} (5-5) ибо Ты Бог, не любящий беззакония; у Тебя не водворится злой;
\end{tcolorbox}
\begin{tcolorbox}
\textsubscript{5} (5-6) нечестивые не пребудут пред очами Твоими: Ты ненавидишь всех, делающих беззаконие.
\end{tcolorbox}
\begin{tcolorbox}
\textsubscript{6} (5-7) Ты погубишь говорящих ложь; кровожадного и коварного гнушается Господь.
\end{tcolorbox}
\begin{tcolorbox}
\textsubscript{7} (5-8) А я, по множеству милости Твоей, войду в дом Твой, поклонюсь святому храму Твоему в страхе Твоем.
\end{tcolorbox}
\begin{tcolorbox}
\textsubscript{8} (5-9) Господи! путеводи меня в правде Твоей, ради врагов моих; уровняй предо мною путь Твой.
\end{tcolorbox}
\begin{tcolorbox}
\textsubscript{9} (5-10) Ибо нет в устах их истины: сердце их--пагуба, гортань их--открытый гроб, языком своим льстят.
\end{tcolorbox}
\begin{tcolorbox}
\textsubscript{10} (5-11) Осуди их, Боже, да падут они от замыслов своих; по множеству нечестия их, отвергни их, ибо они возмутились против Тебя.
\end{tcolorbox}
\begin{tcolorbox}
\textsubscript{11} (5-12) И возрадуются все уповающие на Тебя, вечно будут ликовать, и Ты будешь покровительствовать им; и будут хвалиться Тобою любящие имя Твое.
\end{tcolorbox}
\begin{tcolorbox}
\textsubscript{12} (5-13) Ибо Ты благословляешь праведника, Господи; благоволением, как щитом, венчаешь его.
\end{tcolorbox}
\subsection{CHAPTER 6}
\begin{tcolorbox}
\textsubscript{1} ^^Начальнику хора. На восьмиструнном. Псалом Давида.^^ (6-2) Господи! не в ярости Твоей обличай меня и не во гневе Твоем наказывай меня.
\end{tcolorbox}
\begin{tcolorbox}
\textsubscript{2} (6-3) Помилуй меня, Господи, ибо я немощен; исцели меня, Господи, ибо кости мои потрясены;
\end{tcolorbox}
\begin{tcolorbox}
\textsubscript{3} (6-4) и душа моя сильно потрясена; Ты же, Господи, доколе?
\end{tcolorbox}
\begin{tcolorbox}
\textsubscript{4} (6-5) Обратись, Господи, избавь душу мою, спаси меня ради милости Твоей,
\end{tcolorbox}
\begin{tcolorbox}
\textsubscript{5} (6-6) ибо в смерти нет памятования о Тебе: во гробе кто будет славить Тебя?
\end{tcolorbox}
\begin{tcolorbox}
\textsubscript{6} (6-7) Утомлен я воздыханиями моими: каждую ночь омываю ложе мое, слезами моими омочаю постель мою.
\end{tcolorbox}
\begin{tcolorbox}
\textsubscript{7} (6-8) Иссохло от печали око мое, обветшало от всех врагов моих.
\end{tcolorbox}
\begin{tcolorbox}
\textsubscript{8} (6-9) Удалитесь от меня все, делающие беззаконие, ибо услышал Господь голос плача моего,
\end{tcolorbox}
\begin{tcolorbox}
\textsubscript{9} (6-10) услышал Господь моление мое; Господь примет молитву мою.
\end{tcolorbox}
\begin{tcolorbox}
\textsubscript{10} (6-11) Да будут постыжены и жестоко поражены все враги мои; да возвратятся и постыдятся мгновенно.
\end{tcolorbox}
\subsection{CHAPTER 7}
\begin{tcolorbox}
\textsubscript{1} ^^Плачевная песнь, которую Давид воспел Господу по делу Хуса, из племени Вениаминова.^^ (7-2) Господи, Боже мой! на Тебя я уповаю; спаси меня от всех гонителей моих и избавь меня;
\end{tcolorbox}
\begin{tcolorbox}
\textsubscript{2} (7-3) да не исторгнет он, подобно льву, души моей, терзая, когда нет избавляющего.
\end{tcolorbox}
\begin{tcolorbox}
\textsubscript{3} (7-4) Господи, Боже мой! если я что сделал, если есть неправда в руках моих,
\end{tcolorbox}
\begin{tcolorbox}
\textsubscript{4} (7-5) если я платил злом тому, кто был со мною в мире, --я, который спасал даже того, кто без причины стал моим врагом, --
\end{tcolorbox}
\begin{tcolorbox}
\textsubscript{5} (7-6) то пусть враг преследует душу мою и настигнет, пусть втопчет в землю жизнь мою, и славу мою повергнет в прах.
\end{tcolorbox}
\begin{tcolorbox}
\textsubscript{6} (7-7) Восстань, Господи, во гневе Твоем; подвигнись против неистовства врагов моих, пробудись для меня на суд, который Ты заповедал, --
\end{tcolorbox}
\begin{tcolorbox}
\textsubscript{7} (7-8) сонм людей станет вокруг Тебя; над ним поднимись на высоту.
\end{tcolorbox}
\begin{tcolorbox}
\textsubscript{8} (7-9) Господь судит народы. Суди меня, Господи, по правде моей и по непорочности моей во мне.
\end{tcolorbox}
\begin{tcolorbox}
\textsubscript{9} (7-10) Да прекратится злоба нечестивых, а праведника подкрепи, ибо Ты испытуешь сердца и утробы, праведный Боже!
\end{tcolorbox}
\begin{tcolorbox}
\textsubscript{10} (7-11) Щит мой в Боге, спасающем правых сердцем.
\end{tcolorbox}
\begin{tcolorbox}
\textsubscript{11} (7-12) Бог--судия праведный, и Бог, всякий день строго взыскивающий,
\end{tcolorbox}
\begin{tcolorbox}
\textsubscript{12} (7-13) если [кто] не обращается. Он изощряет Свой меч, напрягает лук Свой и направляет его,
\end{tcolorbox}
\begin{tcolorbox}
\textsubscript{13} (7-14) приготовляет для него сосуды смерти, стрелы Свои делает палящими.
\end{tcolorbox}
\begin{tcolorbox}
\textsubscript{14} (7-15) Вот, [нечестивый] зачал неправду, был чреват злобою и родил себе ложь;
\end{tcolorbox}
\begin{tcolorbox}
\textsubscript{15} (7-16) рыл ров, и выкопал его, и упал в яму, которую приготовил:
\end{tcolorbox}
\begin{tcolorbox}
\textsubscript{16} (7-17) злоба его обратится на его голову, и злодейство его упадет на его темя.
\end{tcolorbox}
\begin{tcolorbox}
\textsubscript{17} (7-18) Славлю Господа по правде Его и пою имени Господа Всевышнего.
\end{tcolorbox}
\subsection{CHAPTER 8}
\begin{tcolorbox}
\textsubscript{1} ^^Начальнику хора. На Гефском [орудии]. Псалом Давида.^^ (8-2) Господи, Боже наш! как величественно имя Твое по всей земле! Слава Твоя простирается превыше небес!
\end{tcolorbox}
\begin{tcolorbox}
\textsubscript{2} (8-3) Из уст младенцев и грудных детей Ты устроил хвалу, ради врагов Твоих, дабы сделать безмолвным врага и мстителя.
\end{tcolorbox}
\begin{tcolorbox}
\textsubscript{3} (8-4) Когда взираю я на небеса Твои--дело Твоих перстов, на луну и звезды, которые Ты поставил,
\end{tcolorbox}
\begin{tcolorbox}
\textsubscript{4} (8-5) то что [есть] человек, что Ты помнишь его, и сын человеческий, что Ты посещаешь его?
\end{tcolorbox}
\begin{tcolorbox}
\textsubscript{5} (8-6) Не много Ты умалил его пред Ангелами: славою и честью увенчал его;
\end{tcolorbox}
\begin{tcolorbox}
\textsubscript{6} (8-7) поставил его владыкою над делами рук Твоих; всё положил под ноги его:
\end{tcolorbox}
\begin{tcolorbox}
\textsubscript{7} (8-8) овец и волов всех, и также полевых зверей,
\end{tcolorbox}
\begin{tcolorbox}
\textsubscript{8} (8-9) птиц небесных и рыб морских, все, преходящее морскими стезями.
\end{tcolorbox}
\begin{tcolorbox}
\textsubscript{9} (8-10) Господи, Боже наш! Как величественно имя Твое по всей земле!
\end{tcolorbox}
\subsection{CHAPTER 9}
\begin{tcolorbox}
\textsubscript{1} ^^Начальнику хора. По смерти Лабена. Псалом Давида.^^ (9-2) Буду славить [Тебя], Господи, всем сердцем моим, возвещать все чудеса Твои.
\end{tcolorbox}
\begin{tcolorbox}
\textsubscript{2} (9-3) Буду радоваться и торжествовать о Тебе, петь имени Твоему, Всевышний.
\end{tcolorbox}
\begin{tcolorbox}
\textsubscript{3} (9-4) Когда враги мои обращены назад, то преткнутся и погибнут пред лицем Твоим,
\end{tcolorbox}
\begin{tcolorbox}
\textsubscript{4} (9-5) ибо Ты производил мой суд и мою тяжбу; Ты воссел на престоле, Судия праведный.
\end{tcolorbox}
\begin{tcolorbox}
\textsubscript{5} (9-6) Ты вознегодовал на народы, погубил нечестивого, имя их изгладил на веки и веки.
\end{tcolorbox}
\begin{tcolorbox}
\textsubscript{6} (9-7) У врага совсем не стало оружия, и города Ты разрушил; погибла память их с ними.
\end{tcolorbox}
\begin{tcolorbox}
\textsubscript{7} (9-8) Но Господь пребывает вовек; Он приготовил для суда престол Свой,
\end{tcolorbox}
\begin{tcolorbox}
\textsubscript{8} (9-9) и Он будет судить вселенную по правде, совершит суд над народами по правоте.
\end{tcolorbox}
\begin{tcolorbox}
\textsubscript{9} (9-10) И будет Господь прибежищем угнетенному, прибежищем во времена скорби;
\end{tcolorbox}
\begin{tcolorbox}
\textsubscript{10} (9-11) и будут уповать на Тебя знающие имя Твое, потому что Ты не оставляешь ищущих Тебя, Господи.
\end{tcolorbox}
\begin{tcolorbox}
\textsubscript{11} (9-12) Пойте Господу, живущему на Сионе, возвещайте между народами дела Его,
\end{tcolorbox}
\begin{tcolorbox}
\textsubscript{12} (9-13) ибо Он взыскивает за кровь; помнит их, не забывает вопля угнетенных.
\end{tcolorbox}
\begin{tcolorbox}
\textsubscript{13} (9-14) Помилуй меня, Господи; воззри на страдание мое от ненавидящих меня, --Ты, Который возносишь меня от врат смерти,
\end{tcolorbox}
\begin{tcolorbox}
\textsubscript{14} (9-15) чтобы я возвещал все хвалы Твои во вратах дщери Сионовой: буду радоваться о спасении Твоем.
\end{tcolorbox}
\begin{tcolorbox}
\textsubscript{15} (9-16) Обрушились народы в яму, которую выкопали; в сети, которую скрыли они, запуталась нога их.
\end{tcolorbox}
\begin{tcolorbox}
\textsubscript{16} (9-17) Познан был Господь по суду, который Он совершил; нечестивый уловлен делами рук своих.
\end{tcolorbox}
\begin{tcolorbox}
\textsubscript{17} (9-18) Да обратятся нечестивые в ад, --все народы, забывающие Бога.
\end{tcolorbox}
\begin{tcolorbox}
\textsubscript{18} (9-19) Ибо не навсегда забыт будет нищий, и надежда бедных не до конца погибнет.
\end{tcolorbox}
\begin{tcolorbox}
\textsubscript{19} (9-20) Восстань, Господи, да не преобладает человек, да судятся народы пред лицем Твоим.
\end{tcolorbox}
\begin{tcolorbox}
\textsubscript{20} (9-21) Наведи, Господи, страх на них; да знают народы, что человеки они.
\end{tcolorbox}
\subsection{CHAPTER 10}
\begin{tcolorbox}
\textsubscript{1} (9-22) Для чего, Господи, стоишь вдали, скрываешь Себя во время скорби?
\end{tcolorbox}
\begin{tcolorbox}
\textsubscript{2} (9-23) По гордости своей нечестивый преследует бедного: да уловятся они ухищрениями, которые сами вымышляют.
\end{tcolorbox}
\begin{tcolorbox}
\textsubscript{3} (9-24) Ибо нечестивый хвалится похотью души своей; корыстолюбец ублажает себя.
\end{tcolorbox}
\begin{tcolorbox}
\textsubscript{4} (9-25) В надмении своем нечестивый пренебрегает Господа: 'не взыщет'; во всех помыслах его: 'нет Бога!'
\end{tcolorbox}
\begin{tcolorbox}
\textsubscript{5} (9-26) Во всякое время пути его гибельны; суды Твои далеки для него; на всех врагов своих он смотрит с пренебрежением;
\end{tcolorbox}
\begin{tcolorbox}
\textsubscript{6} (9-27) говорит в сердце своем: 'не поколеблюсь; в род и род не приключится [мне] зла';
\end{tcolorbox}
\begin{tcolorbox}
\textsubscript{7} (9-28) уста его полны проклятия, коварства и лжи; под языком--его мучение и пагуба;
\end{tcolorbox}
\begin{tcolorbox}
\textsubscript{8} (9-29) сидит в засаде за двором, в потаенных местах убивает невинного; глаза его подсматривают за бедным;
\end{tcolorbox}
\begin{tcolorbox}
\textsubscript{9} (9-30) подстерегает в потаенном месте, как лев в логовище; подстерегает в засаде, чтобы схватить бедного; хватает бедного, увлекая в сети свои;
\end{tcolorbox}
\begin{tcolorbox}
\textsubscript{10} (9-31) сгибается, прилегает, --и бедные падают в сильные когти его;
\end{tcolorbox}
\begin{tcolorbox}
\textsubscript{11} (9-32) говорит в сердце своем: 'забыл Бог, закрыл лице Свое, не увидит никогда'.
\end{tcolorbox}
\begin{tcolorbox}
\textsubscript{12} (9-33) Восстань, Господи, Боже [мой], вознеси руку Твою, не забудь угнетенных.
\end{tcolorbox}
\begin{tcolorbox}
\textsubscript{13} (9-34) Зачем нечестивый пренебрегает Бога, говоря в сердце своем: 'Ты не взыщешь'?
\end{tcolorbox}
\begin{tcolorbox}
\textsubscript{14} (9-35) Ты видишь, ибо Ты взираешь на обиды и притеснения, чтобы воздать Твоею рукою. Тебе предает себя бедный; сироте Ты помощник.
\end{tcolorbox}
\begin{tcolorbox}
\textsubscript{15} (9-36) Сокруши мышцу нечестивому и злому, так чтобы искать и не найти его нечестия.
\end{tcolorbox}
\begin{tcolorbox}
\textsubscript{16} (9-37) Господь--царь на веки, навсегда; исчезнут язычники с земли Его.
\end{tcolorbox}
\begin{tcolorbox}
\textsubscript{17} (9-38) Господи! Ты слышишь желания смиренных; укрепи сердце их; открой ухо Твое,
\end{tcolorbox}
\begin{tcolorbox}
\textsubscript{18} (9-39) чтобы дать суд сироте и угнетенному, да не устрашает более человек на земле.
\end{tcolorbox}
\subsection{CHAPTER 11}
\begin{tcolorbox}
\textsubscript{1} (10-1) ^^Начальнику хора. Псалом Давида.^^ На Господа уповаю; как же вы говорите душе моей: 'улетай на гору вашу, [как] птица'?
\end{tcolorbox}
\begin{tcolorbox}
\textsubscript{2} (10-2) Ибо вот, нечестивые натянули лук, стрелу свою приложили к тетиве, чтобы во тьме стрелять в правых сердцем.
\end{tcolorbox}
\begin{tcolorbox}
\textsubscript{3} (10-3) Когда разрушены основания, что сделает праведник?
\end{tcolorbox}
\begin{tcolorbox}
\textsubscript{4} (10-4) Господь во святом храме Своем, Господь, --престол Его на небесах, очи Его зрят; вежды Его испытывают сынов человеческих.
\end{tcolorbox}
\begin{tcolorbox}
\textsubscript{5} (10-5) Господь испытывает праведного, а нечестивого и любящего насилие ненавидит душа Его.
\end{tcolorbox}
\begin{tcolorbox}
\textsubscript{6} (10-6) Дождем прольет Он на нечестивых горящие угли, огонь и серу; и палящий ветер--их доля из чаши;
\end{tcolorbox}
\begin{tcolorbox}
\textsubscript{7} (10-7) ибо Господь праведен, любит правду; лице Его видит праведника.
\end{tcolorbox}
\subsection{CHAPTER 12}
\begin{tcolorbox}
\textsubscript{1} (11-1) ^^Начальнику хора. На восьмиструнном. Псалом Давида.^^ (11-2) Спаси, Господи, ибо не стало праведного, ибо нет верных между сынами человеческими.
\end{tcolorbox}
\begin{tcolorbox}
\textsubscript{2} (11-3) Ложь говорит каждый своему ближнему; уста льстивы, говорят от сердца притворного.
\end{tcolorbox}
\begin{tcolorbox}
\textsubscript{3} (11-4) Истребит Господь все уста льстивые, язык велеречивый,
\end{tcolorbox}
\begin{tcolorbox}
\textsubscript{4} (11-5) [тех], которые говорят: 'языком нашим пересилим, уста наши с нами; кто нам господин'?
\end{tcolorbox}
\begin{tcolorbox}
\textsubscript{5} (11-6) Ради страдания нищих и воздыхания бедных ныне восстану, говорит Господь, поставлю в безопасности того, кого уловить хотят.
\end{tcolorbox}
\begin{tcolorbox}
\textsubscript{6} (11-7) Слова Господни--слова чистые, серебро, очищенное от земли в горниле, семь раз переплавленное.
\end{tcolorbox}
\begin{tcolorbox}
\textsubscript{7} (11-8) Ты, Господи, сохранишь их, соблюдешь от рода сего вовек.
\end{tcolorbox}
\begin{tcolorbox}
\textsubscript{8} (11-9) Повсюду ходят нечестивые, когда ничтожные из сынов человеческих возвысились.
\end{tcolorbox}
\subsection{CHAPTER 13}
\begin{tcolorbox}
\textsubscript{1} (12-1) ^^Начальнику хора. Псалом Давида.^^ (12-2) Доколе, Господи, будешь забывать меня вконец, доколе будешь скрывать лице Твое от меня?
\end{tcolorbox}
\begin{tcolorbox}
\textsubscript{2} (12-3) Доколе мне слагать советы в душе моей, скорбь в сердце моем день [и ночь]? Доколе врагу моему возноситься надо мною?
\end{tcolorbox}
\begin{tcolorbox}
\textsubscript{3} (12-4) Призри, услышь меня, Господи Боже мой! Просвети очи мои, да не усну я [сном] смертным;
\end{tcolorbox}
\begin{tcolorbox}
\textsubscript{4} (12-5) да не скажет враг мой: 'я одолел его'. Да не возрадуются гонители мои, если я поколеблюсь.
\end{tcolorbox}
\begin{tcolorbox}
\textsubscript{5} (12-6) Я же уповаю на милость Твою; сердце мое возрадуется о спасении Твоем;
\end{tcolorbox}
\begin{tcolorbox}
\textsubscript{6} (12-6) воспою Господу, облагодетельствовавшему меня.
\end{tcolorbox}
\subsection{CHAPTER 14}
\begin{tcolorbox}
\textsubscript{1} (13-1) ^^Начальнику хора. Псалом Давида.^^ Сказал безумец в сердце своем: 'нет Бога'. Они развратились, совершили гнусные дела; нет делающего добро.
\end{tcolorbox}
\begin{tcolorbox}
\textsubscript{2} (13-2) Господь с небес призрел на сынов человеческих, чтобы видеть, есть ли разумеющий, ищущий Бога.
\end{tcolorbox}
\begin{tcolorbox}
\textsubscript{3} (13-3) Все уклонились, сделались равно непотребными; нет делающего добро, нет ни одного.
\end{tcolorbox}
\begin{tcolorbox}
\textsubscript{4} (13-4) Неужели не вразумятся все, делающие беззаконие, съедающие народ мой, [как] едят хлеб, и не призывающие Господа?
\end{tcolorbox}
\begin{tcolorbox}
\textsubscript{5} (13-5) Там убоятся они страха, ибо Бог в роде праведных.
\end{tcolorbox}
\begin{tcolorbox}
\textsubscript{6} (13-6) Вы посмеялись над мыслью нищего, что Господь упование его.
\end{tcolorbox}
\begin{tcolorbox}
\textsubscript{7} (13-7) 'Кто даст с Сиона спасение Израилю!' Когда Господь возвратит пленение народа Своего, тогда возрадуется Иаков и возвеселится Израиль.
\end{tcolorbox}
\subsection{CHAPTER 15}
\begin{tcolorbox}
\textsubscript{1} (14-1) ^^Псалом Давида.^^ Господи! кто может пребывать в жилище Твоем? кто может обитать на святой горе Твоей?
\end{tcolorbox}
\begin{tcolorbox}
\textsubscript{2} (14-2) Тот, кто ходит непорочно и делает правду, и говорит истину в сердце своем;
\end{tcolorbox}
\begin{tcolorbox}
\textsubscript{3} (14-3) кто не клевещет языком своим, не делает искреннему своему зла и не принимает поношения на ближнего своего4
\end{tcolorbox}
\begin{tcolorbox}
\textsubscript{4} (14-4) тот, в глазах которого презрен отверженный, но который боящихся Господа славит; кто клянется, [хотя бы] злому, и не изменяет;
\end{tcolorbox}
\begin{tcolorbox}
\textsubscript{5} (14-5) кто серебра своего не отдает в рост и не принимает даров против невинного. Поступающий так не поколеблется вовек.
\end{tcolorbox}
\subsection{CHAPTER 16}
\begin{tcolorbox}
\textsubscript{1} (15-1) ^^Песнь Давида.^^ Храни меня, Боже, ибо я на Тебя уповаю.
\end{tcolorbox}
\begin{tcolorbox}
\textsubscript{2} (15-2) Я сказал Господу: Ты--Господь мой; блага мои Тебе не нужны.
\end{tcolorbox}
\begin{tcolorbox}
\textsubscript{3} (15-3) К святым, которые на земле, и к дивным [Твоим] --к ним все желание мое.
\end{tcolorbox}
\begin{tcolorbox}
\textsubscript{4} (15-4) Пусть умножаются скорби у тех, которые текут к [богу] чужому; я не возлию кровавых возлияний их и не помяну имен их устами моими.
\end{tcolorbox}
\begin{tcolorbox}
\textsubscript{5} (15-5) Господь есть часть наследия моего и чаши моей. Ты держишь жребий мой.
\end{tcolorbox}
\begin{tcolorbox}
\textsubscript{6} (15-6) Межи мои прошли по прекрасным [местам], и наследие мое приятно для меня.
\end{tcolorbox}
\begin{tcolorbox}
\textsubscript{7} (15-7) Благословлю Господа, вразумившего меня; даже и ночью учит меня внутренность моя.
\end{tcolorbox}
\begin{tcolorbox}
\textsubscript{8} (15-8) Всегда видел я пред собою Господа, ибо Он одесную меня; не поколеблюсь.
\end{tcolorbox}
\begin{tcolorbox}
\textsubscript{9} (15-9) Оттого возрадовалось сердце мое и возвеселился язык мой; даже и плоть моя успокоится в уповании,
\end{tcolorbox}
\begin{tcolorbox}
\textsubscript{10} (15-10) ибо Ты не оставишь души моей в аде и не дашь святому Твоему увидеть тление,
\end{tcolorbox}
\begin{tcolorbox}
\textsubscript{11} (15-11) Ты укажешь мне путь жизни: полнота радостей пред лицем Твоим, блаженство в деснице Твоей вовек.
\end{tcolorbox}
\subsection{CHAPTER 17}
\begin{tcolorbox}
\textsubscript{1} (16-1) ^^Молитва Давида.^^ Услышь, Господи, правду, внемли воплю моему, прими мольбу из уст нелживых.
\end{tcolorbox}
\begin{tcolorbox}
\textsubscript{2} (16-2) От Твоего лица суд мне да изыдет; да воззрят очи Твои на правоту.
\end{tcolorbox}
\begin{tcolorbox}
\textsubscript{3} (16-3) Ты испытал сердце мое, посетил меня ночью, искусил меня и ничего не нашел; от мыслей моих не отступают уста мои.
\end{tcolorbox}
\begin{tcolorbox}
\textsubscript{4} (16-4) В делах человеческих, по слову уст Твоих, я охранял себя от путей притеснителя.
\end{tcolorbox}
\begin{tcolorbox}
\textsubscript{5} (16-5) Утверди шаги мои на путях Твоих, да не колеблются стопы мои.
\end{tcolorbox}
\begin{tcolorbox}
\textsubscript{6} (16-6) К Тебе взываю я, ибо Ты услышишь меня, Боже; приклони ухо Твое ко мне, услышь слова мои.
\end{tcolorbox}
\begin{tcolorbox}
\textsubscript{7} (16-7) Яви дивную милость Твою, Спаситель уповающих [на Тебя] от противящихся деснице Твоей.
\end{tcolorbox}
\begin{tcolorbox}
\textsubscript{8} (16-8) Храни меня, как зеницу ока; в тени крыл Твоих укрой меня
\end{tcolorbox}
\begin{tcolorbox}
\textsubscript{9} (16-9) от лица нечестивых, нападающих на меня, --от врагов души моей, окружающих меня:
\end{tcolorbox}
\begin{tcolorbox}
\textsubscript{10} (16-10) они заключились в туке своем, надменно говорят устами своими.
\end{tcolorbox}
\begin{tcolorbox}
\textsubscript{11} (16-11) На всяком шагу нашем ныне окружают нас; они устремили глаза свои, чтобы низложить [меня] на землю;
\end{tcolorbox}
\begin{tcolorbox}
\textsubscript{12} (16-12) они подобны льву, жаждущему добычи, подобны скимну, сидящему в местах скрытных.
\end{tcolorbox}
\begin{tcolorbox}
\textsubscript{13} (16-13) Восстань, Господи, предупреди их, низложи их. Избавь душу мою от нечестивого мечом Твоим,
\end{tcolorbox}
\begin{tcolorbox}
\textsubscript{14} (16-14) от людей--рукою Твоею, Господи, от людей мира, которых удел в [этой] жизни, которых чрево Ты наполняешь из сокровищниц Твоих; сыновья их сыты и оставят остаток детям своим.
\end{tcolorbox}
\begin{tcolorbox}
\textsubscript{15} (16-15) А я в правде буду взирать на лице Твое; пробудившись, буду насыщаться образом Твоим.
\end{tcolorbox}
\subsection{CHAPTER 18}
\begin{tcolorbox}
\textsubscript{1} (17-1) ^^Начальнику хора. Раба Господня Давида, который произнес слова песни сей к Господу, когда Господь избавил его от рук всех врагов его и от руки Саула. И он сказал:^^ (17-2) Возлюблю тебя, Господи, крепость моя!
\end{tcolorbox}
\begin{tcolorbox}
\textsubscript{2} (17-3) Господь--твердыня моя и прибежище мое, Избавитель мой, Бог мой, --скала моя; на Него я уповаю; щит мой, рог спасения моего и убежище мое.
\end{tcolorbox}
\begin{tcolorbox}
\textsubscript{3} (17-4) Призову достопоклоняемого Господа и от врагов моих спасусь.
\end{tcolorbox}
\begin{tcolorbox}
\textsubscript{4} (17-5) Объяли меня муки смертные, и потоки беззакония устрашили меня;
\end{tcolorbox}
\begin{tcolorbox}
\textsubscript{5} (17-6) цепи ада облегли меня, и сети смерти опутали меня.
\end{tcolorbox}
\begin{tcolorbox}
\textsubscript{6} (17-7) В тесноте моей я призвал Господа и к Богу моему воззвал. И Он услышал от чертога Своего голос мой, и вопль мой дошел до слуха Его.
\end{tcolorbox}
\begin{tcolorbox}
\textsubscript{7} (17-8) Потряслась и всколебалась земля, дрогнули и подвиглись основания гор, ибо разгневался [Бог];
\end{tcolorbox}
\begin{tcolorbox}
\textsubscript{8} (17-9) поднялся дым от гнева Его и из уст Его огонь поядающий; горячие угли [сыпались] от Него.
\end{tcolorbox}
\begin{tcolorbox}
\textsubscript{9} (17-10) Наклонил Он небеса и сошел, --и мрак под ногами Его.
\end{tcolorbox}
\begin{tcolorbox}
\textsubscript{10} (17-11) И воссел на Херувимов и полетел, и понесся на крыльях ветра.
\end{tcolorbox}
\begin{tcolorbox}
\textsubscript{11} (17-12) И мрак сделал покровом Своим, сению вокруг Себя мрак вод, облаков воздушных.
\end{tcolorbox}
\begin{tcolorbox}
\textsubscript{12} (17-13) От блистания пред Ним бежали облака Его, град и угли огненные.
\end{tcolorbox}
\begin{tcolorbox}
\textsubscript{13} (17-14) Возгремел на небесах Господь, и Всевышний дал глас Свой, град и угли огненные.
\end{tcolorbox}
\begin{tcolorbox}
\textsubscript{14} (17-15) Пустил стрелы Свои и рассеял их, множество молний, и рассыпал их.
\end{tcolorbox}
\begin{tcolorbox}
\textsubscript{15} (17-16) И явились источники вод, и открылись основания вселенной от грозного [гласа] Твоего, Господи, от дуновения духа гнева Твоего.
\end{tcolorbox}
\begin{tcolorbox}
\textsubscript{16} (17-17) Он простер [руку] с высоты и взял меня, и извлек меня из вод многих;
\end{tcolorbox}
\begin{tcolorbox}
\textsubscript{17} (17-18) избавил меня от врага моего сильного и от ненавидящих меня, которые были сильнее меня.
\end{tcolorbox}
\begin{tcolorbox}
\textsubscript{18} (17-19) Они восстали на меня в день бедствия моего, но Господь был мне опорою.
\end{tcolorbox}
\begin{tcolorbox}
\textsubscript{19} (17-20) Он вывел меня на пространное место и избавил меня, ибо Он благоволит ко мне.
\end{tcolorbox}
\begin{tcolorbox}
\textsubscript{20} (17-21) Воздал мне Господь по правде моей, по чистоте рук моих вознаградил меня,
\end{tcolorbox}
\begin{tcolorbox}
\textsubscript{21} (17-22) ибо я хранил пути Господни и не был нечестивым пред Богом моим;
\end{tcolorbox}
\begin{tcolorbox}
\textsubscript{22} (17-23) ибо все заповеди Его предо мною, и от уставов Его я не отступал.
\end{tcolorbox}
\begin{tcolorbox}
\textsubscript{23} (17-24) Я был непорочен пред Ним и остерегался, чтобы не согрешить мне;
\end{tcolorbox}
\begin{tcolorbox}
\textsubscript{24} (17-25) и воздал мне Господь по правде моей, по чистоте рук моих пред очами Его.
\end{tcolorbox}
\begin{tcolorbox}
\textsubscript{25} (17-26) С милостивым Ты поступаешь милостиво, с мужем искренним--искренно,
\end{tcolorbox}
\begin{tcolorbox}
\textsubscript{26} (17-27) с чистым--чисто, а с лукавым--по лукавству его,
\end{tcolorbox}
\begin{tcolorbox}
\textsubscript{27} (17-28) ибо Ты людей угнетенных спасаешь, а очи надменные унижаешь.
\end{tcolorbox}
\begin{tcolorbox}
\textsubscript{28} (17-29) Ты возжигаешь светильник мой, Господи; Бог мой просвещает тьму мою.
\end{tcolorbox}
\begin{tcolorbox}
\textsubscript{29} (17-30) С Тобою я поражаю войско, с Богом моим восхожу на стену.
\end{tcolorbox}
\begin{tcolorbox}
\textsubscript{30} (17-31) Бог! --Непорочен путь Его, чисто слово Господа; щит Он для всех, уповающих на Него.
\end{tcolorbox}
\begin{tcolorbox}
\textsubscript{31} (17-32) Ибо кто Бог, кроме Господа, и кто защита, кроме Бога нашего?
\end{tcolorbox}
\begin{tcolorbox}
\textsubscript{32} (17-33) Бог препоясывает меня силою и устрояет мне верный путь;
\end{tcolorbox}
\begin{tcolorbox}
\textsubscript{33} (17-34) делает ноги мои, как оленьи, и на высотах моих поставляет меня;
\end{tcolorbox}
\begin{tcolorbox}
\textsubscript{34} (17-35) научает руки мои брани, и мышцы мои сокрушают медный лук.
\end{tcolorbox}
\begin{tcolorbox}
\textsubscript{35} (17-36) Ты дал мне щит спасения Твоего, и десница Твоя поддерживает меня, и милость Твоя возвеличивает меня.
\end{tcolorbox}
\begin{tcolorbox}
\textsubscript{36} (17-37) Ты расширяешь шаг мой подо мною, и не колеблются ноги мои.
\end{tcolorbox}
\begin{tcolorbox}
\textsubscript{37} (17-38) Я преследую врагов моих и настигаю их, и не возвращаюсь, доколе не истреблю их;
\end{tcolorbox}
\begin{tcolorbox}
\textsubscript{38} (17-39) поражаю их, и они не могут встать, падают под ноги мои,
\end{tcolorbox}
\begin{tcolorbox}
\textsubscript{39} (17-40) ибо Ты препоясал меня силою для войны и низложил под ноги мои восставших на меня;
\end{tcolorbox}
\begin{tcolorbox}
\textsubscript{40} (17-41) Ты обратил ко мне тыл врагов моих, и я истребляю ненавидящих меня:
\end{tcolorbox}
\begin{tcolorbox}
\textsubscript{41} (17-42) они вопиют, но нет спасающего; ко Господу, --но Он не внемлет им;
\end{tcolorbox}
\begin{tcolorbox}
\textsubscript{42} (17-43) я рассеваю их, как прах пред лицем ветра, как уличную грязь попираю их.
\end{tcolorbox}
\begin{tcolorbox}
\textsubscript{43} (17-44) Ты избавил меня от мятежа народа, поставил меня главою иноплеменников; народ, которого я не знал, служит мне;
\end{tcolorbox}
\begin{tcolorbox}
\textsubscript{44} (17-45) по одному слуху о мне повинуются мне; иноплеменники ласкательствуют предо мною;
\end{tcolorbox}
\begin{tcolorbox}
\textsubscript{45} (17-46) иноплеменники бледнеют и трепещут в укреплениях своих.
\end{tcolorbox}
\begin{tcolorbox}
\textsubscript{46} (17-47) Жив Господь и благословен защитник мой! Да будет превознесен Бог спасения моего,
\end{tcolorbox}
\begin{tcolorbox}
\textsubscript{47} (17-48) Бог, мстящий за меня и покоряющий мне народы,
\end{tcolorbox}
\begin{tcolorbox}
\textsubscript{48} (17-49) и избавляющий меня от врагов моих! Ты вознес меня над восстающими против меня и от человека жестокого избавил меня.
\end{tcolorbox}
\begin{tcolorbox}
\textsubscript{49} (17-50) За то буду славить Тебя, Господи, между иноплеменниками и буду петь имени Твоему,
\end{tcolorbox}
\begin{tcolorbox}
\textsubscript{50} (17-51) величественно спасающий царя и творящий милость помазаннику Твоему Давиду и потомству его во веки.
\end{tcolorbox}
\subsection{CHAPTER 19}
\begin{tcolorbox}
\textsubscript{1} (18-1) ^^Начальнику хора. Псалом Давида.^^ (18-2) Небеса проповедуют славу Божию, и о делах рук Его вещает твердь.
\end{tcolorbox}
\begin{tcolorbox}
\textsubscript{2} (18-3) День дню передает речь, и ночь ночи открывает знание.
\end{tcolorbox}
\begin{tcolorbox}
\textsubscript{3} (18-4) Нет языка, и нет наречия, где не слышался бы голос их.
\end{tcolorbox}
\begin{tcolorbox}
\textsubscript{4} (18-5) По всей земле проходит звук их, и до пределов вселенной слова их. Он поставил в них жилище солнцу,
\end{tcolorbox}
\begin{tcolorbox}
\textsubscript{5} (18-6) и оно выходит, как жених из брачного чертога своего, радуется, как исполин, пробежать поприще:
\end{tcolorbox}
\begin{tcolorbox}
\textsubscript{6} (18-7) от края небес исход его, и шествие его до края их, и ничто не укрыто от теплоты его.
\end{tcolorbox}
\begin{tcolorbox}
\textsubscript{7} (18-8) Закон Господа совершен, укрепляет душу; откровение Господа верно, умудряет простых.
\end{tcolorbox}
\begin{tcolorbox}
\textsubscript{8} (18-9) Повеления Господа праведны, веселят сердце; заповедь Господа светла, просвещает очи.
\end{tcolorbox}
\begin{tcolorbox}
\textsubscript{9} (18-10) Страх Господень чист, пребывает вовек. Суды Господни истина, все праведны;
\end{tcolorbox}
\begin{tcolorbox}
\textsubscript{10} (18-11) они вожделеннее золота и даже множества золота чистого, слаще меда и капель сота;
\end{tcolorbox}
\begin{tcolorbox}
\textsubscript{11} (18-12) и раб Твой охраняется ими, в соблюдении их великая награда.
\end{tcolorbox}
\begin{tcolorbox}
\textsubscript{12} (18-13) Кто усмотрит погрешности свои? От тайных [моих] очисти меня
\end{tcolorbox}
\begin{tcolorbox}
\textsubscript{13} (18-14) и от умышленных удержи раба Твоего, чтобы не возобладали мною. Тогда я буду непорочен и чист от великого развращения.
\end{tcolorbox}
\begin{tcolorbox}
\textsubscript{14} (18-15) Да будут слова уст моих и помышление сердца моего благоугодны пред Тобою, Господи, твердыня моя и Избавитель мой!
\end{tcolorbox}
\subsection{CHAPTER 20}
\begin{tcolorbox}
\textsubscript{1} (19-1) ^^Начальнику хора. Псалом Давида.^^ (19-2) Да услышит тебя Господь в день печали, да защитит тебя имя Бога Иаковлева.
\end{tcolorbox}
\begin{tcolorbox}
\textsubscript{2} (19-3) Да пошлет тебе помощь из Святилища и с Сиона да подкрепит тебя.
\end{tcolorbox}
\begin{tcolorbox}
\textsubscript{3} (19-4) Да воспомянет все жертвоприношения твои и всесожжение твое да соделает тучным.
\end{tcolorbox}
\begin{tcolorbox}
\textsubscript{4} (19-5) Да даст тебе по сердцу твоему и все намерения твои да исполнит.
\end{tcolorbox}
\begin{tcolorbox}
\textsubscript{5} (19-6) Мы возрадуемся о спасении твоем и во имя Бога нашего поднимем знамя. Да исполнит Господь все прошения твои.
\end{tcolorbox}
\begin{tcolorbox}
\textsubscript{6} (19-7) Ныне познал я, что Господь спасает помазанника Своего, отвечает ему со святых небес Своих могуществом спасающей десницы Своей.
\end{tcolorbox}
\begin{tcolorbox}
\textsubscript{7} (19-8) Иные колесницами, иные конями, а мы именем Господа Бога нашего хвалимся:
\end{tcolorbox}
\begin{tcolorbox}
\textsubscript{8} (19-9) они поколебались и пали, а мы встали и стоим прямо.
\end{tcolorbox}
\begin{tcolorbox}
\textsubscript{9} (19-10) Господи! спаси царя и услышь нас, когда будем взывать [к Тебе].
\end{tcolorbox}
\subsection{CHAPTER 21}
\begin{tcolorbox}
\textsubscript{1} (20-1) ^^Начальнику хора. Псалом Давида.^^ (20-2) Господи! силою Твоею веселится царь и о спасении Твоем безмерно радуется.
\end{tcolorbox}
\begin{tcolorbox}
\textsubscript{2} (20-3) Ты дал ему, чего желало сердце его, и прошения уст его не отринул,
\end{tcolorbox}
\begin{tcolorbox}
\textsubscript{3} (20-4) ибо Ты встретил его благословениями благости, возложил на голову его венец из чистого золота.
\end{tcolorbox}
\begin{tcolorbox}
\textsubscript{4} (20-5) Он просил у Тебя жизни; Ты дал ему долгоденствие на век и век.
\end{tcolorbox}
\begin{tcolorbox}
\textsubscript{5} (20-6) Велика слава его в спасении Твоем; Ты возложил на него честь и величие.
\end{tcolorbox}
\begin{tcolorbox}
\textsubscript{6} (20-7) Ты положил на него благословения на веки, возвеселил его радостью лица Твоего,
\end{tcolorbox}
\begin{tcolorbox}
\textsubscript{7} (20-8) ибо царь уповает на Господа, и во благости Всевышнего не поколеблется.
\end{tcolorbox}
\begin{tcolorbox}
\textsubscript{8} (20-9) Рука Твоя найдет всех врагов Твоих, десница Твоя найдет ненавидящих Тебя.
\end{tcolorbox}
\begin{tcolorbox}
\textsubscript{9} (20-10) Во время гнева Твоего Ты сделаешь их, как печь огненную; во гневе Своем Господь погубит их, и пожрет их огонь.
\end{tcolorbox}
\begin{tcolorbox}
\textsubscript{10} (20-11) Ты истребишь плод их с земли и семя их--из среды сынов человеческих,
\end{tcolorbox}
\begin{tcolorbox}
\textsubscript{11} (20-12) ибо они предприняли против Тебя злое, составили замыслы, но не могли [выполнить их].
\end{tcolorbox}
\begin{tcolorbox}
\textsubscript{12} (20-13) Ты поставишь их целью, из луков Твоих пустишь стрелы в лице их.
\end{tcolorbox}
\begin{tcolorbox}
\textsubscript{13} (20-14) Вознесись, Господи, силою Твоею: мы будем воспевать и прославлять Твое могущество.
\end{tcolorbox}
\subsection{CHAPTER 22}
\begin{tcolorbox}
\textsubscript{1} (21-1) ^^Начальнику хора. При появлении зари. Псалом Давида.^^ (21-2) Боже мой! Боже мой! для чего Ты оставил меня? Далеки от спасения моего слова вопля моего.
\end{tcolorbox}
\begin{tcolorbox}
\textsubscript{2} (21-3) Боже мой! я вопию днем, --и Ты не внемлешь мне, ночью, --и нет мне успокоения.
\end{tcolorbox}
\begin{tcolorbox}
\textsubscript{3} (21-4) Но Ты, Святый, живешь среди славословий Израиля.
\end{tcolorbox}
\begin{tcolorbox}
\textsubscript{4} (21-5) На Тебя уповали отцы наши; уповали, и Ты избавлял их;
\end{tcolorbox}
\begin{tcolorbox}
\textsubscript{5} (21-6) к Тебе взывали они, и были спасаемы; на Тебя уповали, и не оставались в стыде.
\end{tcolorbox}
\begin{tcolorbox}
\textsubscript{6} (21-7) Я же червь, а не человек, поношение у людей и презрение в народе.
\end{tcolorbox}
\begin{tcolorbox}
\textsubscript{7} (21-8) Все, видящие меня, ругаются надо мною, говорят устами, кивая головою:
\end{tcolorbox}
\begin{tcolorbox}
\textsubscript{8} (21-9) 'он уповал на Господа; пусть избавит его, пусть спасет, если он угоден Ему'.
\end{tcolorbox}
\begin{tcolorbox}
\textsubscript{9} (21-10) Но Ты извел меня из чрева, вложил в меня упование у грудей матери моей.
\end{tcolorbox}
\begin{tcolorbox}
\textsubscript{10} (21-11) На Тебя оставлен я от утробы; от чрева матери моей Ты--Бог мой.
\end{tcolorbox}
\begin{tcolorbox}
\textsubscript{11} (21-12) Не удаляйся от меня, ибо скорбь близка, а помощника нет.
\end{tcolorbox}
\begin{tcolorbox}
\textsubscript{12} (21-13) Множество тельцов обступили меня; тучные Васанские окружили меня,
\end{tcolorbox}
\begin{tcolorbox}
\textsubscript{13} (21-14) раскрыли на меня пасть свою, как лев, алчущий добычи и рыкающий.
\end{tcolorbox}
\begin{tcolorbox}
\textsubscript{14} (21-15) Я пролился, как вода; все кости мои рассыпались; сердце мое сделалось, как воск, растаяло посреди внутренности моей.
\end{tcolorbox}
\begin{tcolorbox}
\textsubscript{15} (21-16) Сила моя иссохла, как черепок; язык мой прильпнул к гортани моей, и Ты свел меня к персти смертной.
\end{tcolorbox}
\begin{tcolorbox}
\textsubscript{16} (21-17) Ибо псы окружили меня, скопище злых обступило меня, пронзили руки мои и ноги мои.
\end{tcolorbox}
\begin{tcolorbox}
\textsubscript{17} (21-18) Можно было бы перечесть все кости мои; а они смотрят и делают из меня зрелище;
\end{tcolorbox}
\begin{tcolorbox}
\textsubscript{18} (21-19) делят ризы мои между собою и об одежде моей бросают жребий.
\end{tcolorbox}
\begin{tcolorbox}
\textsubscript{19} (21-20) Но Ты, Господи, не удаляйся от меня; сила моя! поспеши на помощь мне;
\end{tcolorbox}
\begin{tcolorbox}
\textsubscript{20} (21-21) избавь от меча душу мою и от псов одинокую мою;
\end{tcolorbox}
\begin{tcolorbox}
\textsubscript{21} (21-22) спаси меня от пасти льва и от рогов единорогов, услышав, [избавь] меня.
\end{tcolorbox}
\begin{tcolorbox}
\textsubscript{22} (21-23) Буду возвещать имя Твое братьям моим, посреди собрания восхвалять Тебя.
\end{tcolorbox}
\begin{tcolorbox}
\textsubscript{23} (21-24) Боящиеся Господа! восхвалите Его. Все семя Иакова! прославь Его. Да благоговеет пред Ним все семя Израиля,
\end{tcolorbox}
\begin{tcolorbox}
\textsubscript{24} (21-25) ибо Он не презрел и не пренебрег скорби страждущего, не скрыл от него лица Своего, но услышал его, когда сей воззвал к Нему.
\end{tcolorbox}
\begin{tcolorbox}
\textsubscript{25} (21-26) О Тебе хвала моя в собрании великом; воздам обеты мои пред боящимися Его.
\end{tcolorbox}
\begin{tcolorbox}
\textsubscript{26} (21-27) Да едят бедные и насыщаются; да восхвалят Господа ищущие Его; да живут сердца ваши во веки!
\end{tcolorbox}
\begin{tcolorbox}
\textsubscript{27} (21-28) Вспомнят, и обратятся к Господу все концы земли, и поклонятся пред Тобою все племена язычников,
\end{tcolorbox}
\begin{tcolorbox}
\textsubscript{28} (21-29) ибо Господне есть царство, и Он--Владыка над народами.
\end{tcolorbox}
\begin{tcolorbox}
\textsubscript{29} (21-30) Будут есть и поклоняться все тучные земли; преклонятся пред Ним все нисходящие в персть и не могущие сохранить жизни своей.
\end{tcolorbox}
\begin{tcolorbox}
\textsubscript{30} (21-31) Потомство [мое] будет служить Ему, и будет называться Господним вовек:
\end{tcolorbox}
\begin{tcolorbox}
\textsubscript{31} (21-32) придут и будут возвещать правду Его людям, которые родятся, что сотворил Господь.
\end{tcolorbox}
\subsection{CHAPTER 23}
\begin{tcolorbox}
\textsubscript{1} (22-1) ^^Псалом Давида.^^ Господь--Пастырь мой; я ни в чем не буду нуждаться:
\end{tcolorbox}
\begin{tcolorbox}
\textsubscript{2} (22-2) Он покоит меня на злачных пажитях и водит меня к водам тихим,
\end{tcolorbox}
\begin{tcolorbox}
\textsubscript{3} (22-3) подкрепляет душу мою, направляет меня на стези правды ради имени Своего.
\end{tcolorbox}
\begin{tcolorbox}
\textsubscript{4} (22-4) Если я пойду и долиною смертной тени, не убоюсь зла, потому что Ты со мной; Твой жезл и Твой посох--они успокаивают меня.
\end{tcolorbox}
\begin{tcolorbox}
\textsubscript{5} (22-5) Ты приготовил предо мною трапезу в виду врагов моих; умастил елеем голову мою; чаша моя преисполнена.
\end{tcolorbox}
\begin{tcolorbox}
\textsubscript{6} (22-6) Так, благость и милость да сопровождают меня во все дни жизни моей, и я пребуду в доме Господнем многие дни.
\end{tcolorbox}
\subsection{CHAPTER 24}
\begin{tcolorbox}
\textsubscript{1} (23-1) ^^Псалом Давида.^^ Господня--земля и что наполняет ее, вселенная и все живущее в ней,
\end{tcolorbox}
\begin{tcolorbox}
\textsubscript{2} (23-2) ибо Он основал ее на морях и на реках утвердил ее.
\end{tcolorbox}
\begin{tcolorbox}
\textsubscript{3} (23-3) Кто взойдет на гору Господню, или кто станет на святом месте Его?
\end{tcolorbox}
\begin{tcolorbox}
\textsubscript{4} (23-4) Тот, у которого руки неповинны и сердце чисто, кто не клялся душею своею напрасно и не божился ложно, --
\end{tcolorbox}
\begin{tcolorbox}
\textsubscript{5} (23-5) [тот] получит благословение от Господа и милость от Бога, Спасителя своего.
\end{tcolorbox}
\begin{tcolorbox}
\textsubscript{6} (23-6) Таков род ищущих Его, ищущих лица Твоего, Боже Иакова!
\end{tcolorbox}
\begin{tcolorbox}
\textsubscript{7} (23-7) Поднимите, врата, верхи ваши, и поднимитесь, двери вечные, и войдет Царь славы!
\end{tcolorbox}
\begin{tcolorbox}
\textsubscript{8} (23-8) Кто сей Царь славы? --Господь крепкий и сильный, Господь, сильный в брани.
\end{tcolorbox}
\begin{tcolorbox}
\textsubscript{9} (23-9) Поднимите, врата, верхи ваши, и поднимитесь, двери вечные, и войдет Царь славы!
\end{tcolorbox}
\begin{tcolorbox}
\textsubscript{10} (23-10) Кто сей Царь славы? --Господь сил, Он--царь славы.
\end{tcolorbox}
\subsection{CHAPTER 25}
\begin{tcolorbox}
\textsubscript{1} (24-1) ^^Псалом Давида.^^ К Тебе, Господи, возношу душу мою.
\end{tcolorbox}
\begin{tcolorbox}
\textsubscript{2} (24-2) Боже мой! на Тебя уповаю, да не постыжусь, да не восторжествуют надо мною враги мои,
\end{tcolorbox}
\begin{tcolorbox}
\textsubscript{3} (24-3) да не постыдятся и все надеющиеся на Тебя: да постыдятся беззаконнующие втуне.
\end{tcolorbox}
\begin{tcolorbox}
\textsubscript{4} (24-4) Укажи мне, Господи, пути Твои и научи меня стезям Твоим.
\end{tcolorbox}
\begin{tcolorbox}
\textsubscript{5} (24-5) Направь меня на истину Твою и научи меня, ибо Ты Бог спасения моего; на Тебя надеюсь всякий день.
\end{tcolorbox}
\begin{tcolorbox}
\textsubscript{6} (24-6) Вспомни щедроты Твои, Господи, и милости Твои, ибо они от века.
\end{tcolorbox}
\begin{tcolorbox}
\textsubscript{7} (24-7) Грехов юности моей и преступлений моих не вспоминай; по милости Твоей вспомни меня Ты, ради благости Твоей, Господи!
\end{tcolorbox}
\begin{tcolorbox}
\textsubscript{8} (24-8) Благ и праведен Господь, посему наставляет грешников на путь,
\end{tcolorbox}
\begin{tcolorbox}
\textsubscript{9} (24-9) направляет кротких к правде, и научает кротких путям Своим.
\end{tcolorbox}
\begin{tcolorbox}
\textsubscript{10} (24-10) Все пути Господни--милость и истина к хранящим завет Его и откровения Его.
\end{tcolorbox}
\begin{tcolorbox}
\textsubscript{11} (24-11) Ради имени Твоего, Господи, прости согрешение мое, ибо велико оно.
\end{tcolorbox}
\begin{tcolorbox}
\textsubscript{12} (24-12) Кто есть человек, боящийся Господа? Ему укажет Он путь, который избрать.
\end{tcolorbox}
\begin{tcolorbox}
\textsubscript{13} (24-13) Душа его пребудет во благе, и семя его наследует землю.
\end{tcolorbox}
\begin{tcolorbox}
\textsubscript{14} (24-14) Тайна Господня--боящимся Его, и завет Свой Он открывает им.
\end{tcolorbox}
\begin{tcolorbox}
\textsubscript{15} (24-15) Очи мои всегда к Господу, ибо Он извлекает из сети ноги мои.
\end{tcolorbox}
\begin{tcolorbox}
\textsubscript{16} (24-16) Призри на меня и помилуй меня, ибо я одинок и угнетен.
\end{tcolorbox}
\begin{tcolorbox}
\textsubscript{17} (24-17) Скорби сердца моего умножились; выведи меня из бед моих,
\end{tcolorbox}
\begin{tcolorbox}
\textsubscript{18} (24-18) призри на страдание мое и на изнеможение мое и прости все грехи мои.
\end{tcolorbox}
\begin{tcolorbox}
\textsubscript{19} (24-19) Посмотри на врагов моих, как много их, и [какою] лютою ненавистью они ненавидят меня.
\end{tcolorbox}
\begin{tcolorbox}
\textsubscript{20} (24-20) Сохрани душу мою и избавь меня, да не постыжусь, что я на Тебя уповаю.
\end{tcolorbox}
\begin{tcolorbox}
\textsubscript{21} (24-21) Непорочность и правота да охраняют меня, ибо я на Тебя надеюсь.
\end{tcolorbox}
\begin{tcolorbox}
\textsubscript{22} (24-22) Избавь, Боже, Израиля от всех скорбей его.
\end{tcolorbox}
\subsection{CHAPTER 26}
\begin{tcolorbox}
\textsubscript{1} (25-1) ^^Псалом Давида.^^ Рассуди меня, Господи, ибо я ходил в непорочности моей, и, уповая на Господа, не поколеблюсь.
\end{tcolorbox}
\begin{tcolorbox}
\textsubscript{2} (25-2) Искуси меня, Господи, и испытай меня; расплавь внутренности мои и сердце мое,
\end{tcolorbox}
\begin{tcolorbox}
\textsubscript{3} (25-3) ибо милость Твоя пред моими очами, и я ходил в истине Твоей,
\end{tcolorbox}
\begin{tcolorbox}
\textsubscript{4} (25-4) не сидел я с людьми лживыми, и с коварными не пойду;
\end{tcolorbox}
\begin{tcolorbox}
\textsubscript{5} (25-5) возненавидел я сборище злонамеренных, и с нечестивыми не сяду;
\end{tcolorbox}
\begin{tcolorbox}
\textsubscript{6} (25-6) буду омывать в невинности руки мои и обходить жертвенник Твой, Господи,
\end{tcolorbox}
\begin{tcolorbox}
\textsubscript{7} (25-7) чтобы возвещать гласом хвалы и поведать все чудеса Твои.
\end{tcolorbox}
\begin{tcolorbox}
\textsubscript{8} (25-8) Господи! возлюбил я обитель дома Твоего и место жилища славы Твоей.
\end{tcolorbox}
\begin{tcolorbox}
\textsubscript{9} (25-9) Не погуби души моей с грешниками и жизни моей с кровожадными,
\end{tcolorbox}
\begin{tcolorbox}
\textsubscript{10} (25-10) у которых в руках злодейство, и которых правая рука полна мздоимства.
\end{tcolorbox}
\begin{tcolorbox}
\textsubscript{11} (25-11) А я хожу в моей непорочности; избавь меня, и помилуй меня.
\end{tcolorbox}
\begin{tcolorbox}
\textsubscript{12} (25-12) Моя нога стоит на прямом [пути]; в собраниях благословлю Господа.
\end{tcolorbox}
\subsection{CHAPTER 27}
\begin{tcolorbox}
\textsubscript{1} (26-1) ^^Псалом Давида.^^ Господь--свет мой и спасение мое: кого мне бояться? Господь крепость жизни моей: кого мне страшиться?
\end{tcolorbox}
\begin{tcolorbox}
\textsubscript{2} (26-2) Если будут наступать на меня злодеи, противники и враги мои, чтобы пожрать плоть мою, то они сами преткнутся и падут.
\end{tcolorbox}
\begin{tcolorbox}
\textsubscript{3} (26-3) Если ополчится против меня полк, не убоится сердце мое; если восстанет на меня война, и тогда буду надеяться.
\end{tcolorbox}
\begin{tcolorbox}
\textsubscript{4} (26-4) Одного просил я у Господа, того только ищу, чтобы пребывать мне в доме Господнем во все дни жизни моей, созерцать красоту Господню и посещать храм Его,
\end{tcolorbox}
\begin{tcolorbox}
\textsubscript{5} (26-5) ибо Он укрыл бы меня в скинии Своей в день бедствия, скрыл бы меня в потаенном месте селения Своего, вознес бы меня на скалу.
\end{tcolorbox}
\begin{tcolorbox}
\textsubscript{6} (26-6) Тогда вознеслась бы голова моя над врагами, окружающими меня; и я принес бы в Его скинии жертвы славословия, стал бы петь и воспевать пред Господом.
\end{tcolorbox}
\begin{tcolorbox}
\textsubscript{7} (26-7) Услышь, Господи, голос мой, которым я взываю, помилуй меня и внемли мне.
\end{tcolorbox}
\begin{tcolorbox}
\textsubscript{8} (26-8) Сердце мое говорит от Тебя: 'ищите лица Моего'; и я буду искать лица Твоего, Господи.
\end{tcolorbox}
\begin{tcolorbox}
\textsubscript{9} (26-9) Не скрой от меня лица Твоего; не отринь во гневе раба Твоего. Ты был помощником моим; не отвергни меня и не оставь меня, Боже, Спаситель мой!
\end{tcolorbox}
\begin{tcolorbox}
\textsubscript{10} (26-10) ибо отец мой и мать моя оставили меня, но Господь примет меня.
\end{tcolorbox}
\begin{tcolorbox}
\textsubscript{11} (26-11) Научи меня, Господи, пути Твоему и наставь меня на стезю правды, ради врагов моих;
\end{tcolorbox}
\begin{tcolorbox}
\textsubscript{12} (26-12) не предавай меня на произвол врагам моим, ибо восстали на меня свидетели лживые и дышат злобою.
\end{tcolorbox}
\begin{tcolorbox}
\textsubscript{13} (26-13) Но я верую, что увижу благость Господа на земле живых.
\end{tcolorbox}
\begin{tcolorbox}
\textsubscript{14} (26-14) Надейся на Господа, мужайся, и да укрепляется сердце твое, и надейся на Господа.
\end{tcolorbox}
\subsection{CHAPTER 28}
\begin{tcolorbox}
\textsubscript{1} (27-1) ^^Псалом Давида.^^ К тебе, Господи, взываю: твердыня моя! не будь безмолвен для меня, чтобы при безмолвии Твоем я не уподобился нисходящим в могилу.
\end{tcolorbox}
\begin{tcolorbox}
\textsubscript{2} (27-2) Услышь голос молений моих, когда я взываю к Тебе, когда поднимаю руки мои к святому храму Твоему.
\end{tcolorbox}
\begin{tcolorbox}
\textsubscript{3} (27-3) Не погуби меня с нечестивыми и с делающими неправду, которые с ближними своими говорят о мире, а в сердце у них зло.
\end{tcolorbox}
\begin{tcolorbox}
\textsubscript{4} (27-4) Воздай им по делам их, по злым поступкам их; по делам рук их воздай им, отдай им заслуженное ими.
\end{tcolorbox}
\begin{tcolorbox}
\textsubscript{5} (27-5) За то, что они невнимательны к действиям Господа и к делу рук Его, Он разрушит их и не созиждет их.
\end{tcolorbox}
\begin{tcolorbox}
\textsubscript{6} (27-6) Благословен Господь, ибо Он услышал голос молений моих.
\end{tcolorbox}
\begin{tcolorbox}
\textsubscript{7} (27-7) Господь--крепость моя и щит мой; на Него уповало сердце мое, и Он помог мне, и возрадовалось сердце мое; и я прославлю Его песнью моею.
\end{tcolorbox}
\begin{tcolorbox}
\textsubscript{8} (27-8) Господь--крепость народа Своего и спасительная защита помазанника Своего.
\end{tcolorbox}
\begin{tcolorbox}
\textsubscript{9} (27-9) Спаси народ Твой и благослови наследие Твое; паси их и возвышай их во веки!
\end{tcolorbox}
\subsection{CHAPTER 29}
\begin{tcolorbox}
\textsubscript{1} (28-1) ^^Псалом Давида.^^ Воздайте Господу, сыны Божии, воздайте Господу славу и честь,
\end{tcolorbox}
\begin{tcolorbox}
\textsubscript{2} (28-2) воздайте Господу славу имени Его; поклонитесь Господу в благолепном святилище [Его].
\end{tcolorbox}
\begin{tcolorbox}
\textsubscript{3} (28-3) Глас Господень над водами; Бог славы возгремел, Господь над водами многими.
\end{tcolorbox}
\begin{tcolorbox}
\textsubscript{4} (28-4) Глас Господа силен, глас Господа величествен.
\end{tcolorbox}
\begin{tcolorbox}
\textsubscript{5} (28-5) Глас Господа сокрушает кедры; Господь сокрушает кедры Ливанские
\end{tcolorbox}
\begin{tcolorbox}
\textsubscript{6} (28-6) и заставляет их скакать подобно тельцу, Ливан и Сирион, подобно молодому единорогу.
\end{tcolorbox}
\begin{tcolorbox}
\textsubscript{7} (28-7) Глас Господа высекает пламень огня.
\end{tcolorbox}
\begin{tcolorbox}
\textsubscript{8} (28-8) Глас Господа потрясает пустыню; потрясает Господь пустыню Кадес.
\end{tcolorbox}
\begin{tcolorbox}
\textsubscript{9} (28-9) Глас Господа разрешает от бремени ланей и обнажает леса; и во храме Его все возвещает о [Его] славе.
\end{tcolorbox}
\begin{tcolorbox}
\textsubscript{10} (28-10) Господь восседал над потопом, и будет восседать Господь царем вовек.
\end{tcolorbox}
\begin{tcolorbox}
\textsubscript{11} (28-11) Господь даст силу народу Своему, Господь благословит народ Свой миром.
\end{tcolorbox}
\subsection{CHAPTER 30}
\begin{tcolorbox}
\textsubscript{1} (29-1) ^^Псалом Давида; песнь при обновлении дома.^^ (29-2) Превознесу Тебя, Господи, что Ты поднял меня и не дал моим врагам восторжествовать надо мною.
\end{tcolorbox}
\begin{tcolorbox}
\textsubscript{2} (29-3) Господи, Боже мой! я воззвал к Тебе, и Ты исцелил меня.
\end{tcolorbox}
\begin{tcolorbox}
\textsubscript{3} (29-4) Господи! Ты вывел из ада душу мою и оживил меня, чтобы я не сошел в могилу.
\end{tcolorbox}
\begin{tcolorbox}
\textsubscript{4} (29-5) Пойте Господу, святые Его, славьте память святыни Его,
\end{tcolorbox}
\begin{tcolorbox}
\textsubscript{5} (29-6) ибо на мгновение гнев Его, на [всю] жизнь благоволение Его: вечером водворяется плач, а на утро радость.
\end{tcolorbox}
\begin{tcolorbox}
\textsubscript{6} (29-7) И я говорил в благоденствии моем: 'не поколеблюсь вовек'.
\end{tcolorbox}
\begin{tcolorbox}
\textsubscript{7} (29-8) По благоволению Твоему, Господи, Ты укрепил гору мою; но Ты сокрыл лице Твое, [и] я смутился.
\end{tcolorbox}
\begin{tcolorbox}
\textsubscript{8} (29-9) [Тогда] к Тебе, Господи, взывал я, и Господа умолял:
\end{tcolorbox}
\begin{tcolorbox}
\textsubscript{9} (29-10) 'что пользы в крови моей, когда я сойду в могилу? будет ли прах славить Тебя? будет ли возвещать истину Твою?
\end{tcolorbox}
\begin{tcolorbox}
\textsubscript{10} (29-11) услышь, Господи, и помилуй меня; Господи! будь мне помощником'.
\end{tcolorbox}
\begin{tcolorbox}
\textsubscript{11} (29-12) И Ты обратил сетование мое в ликование, снял с меня вретище и препоясал меня веселием,
\end{tcolorbox}
\begin{tcolorbox}
\textsubscript{12} (29-13) да славит Тебя душа моя и да не умолкает. Господи, Боже мой! буду славить Тебя вечно.
\end{tcolorbox}
\subsection{CHAPTER 31}
\begin{tcolorbox}
\textsubscript{1} (30-1) ^^Начальнику хора. Псалом Давида.^^ (30-2) На Тебя, Господи, уповаю, да не постыжусь вовек; по правде Твоей избавь меня;
\end{tcolorbox}
\begin{tcolorbox}
\textsubscript{2} (30-3) приклони ко мне ухо Твое, поспеши избавить меня. Будь мне каменною твердынею, домом прибежища, чтобы спасти меня,
\end{tcolorbox}
\begin{tcolorbox}
\textsubscript{3} (30-4) ибо Ты каменная гора моя и ограда моя; ради имени Твоего води меня и управляй мною.
\end{tcolorbox}
\begin{tcolorbox}
\textsubscript{4} (30-5) Выведи меня из сети, которую тайно поставили мне, ибо Ты крепость моя.
\end{tcolorbox}
\begin{tcolorbox}
\textsubscript{5} (30-6) В Твою руку предаю дух мой; Ты избавлял меня, Господи, Боже истины.
\end{tcolorbox}
\begin{tcolorbox}
\textsubscript{6} (30-7) Ненавижу почитателей суетных идолов, но на Господа уповаю.
\end{tcolorbox}
\begin{tcolorbox}
\textsubscript{7} (30-8) Буду радоваться и веселиться о милости Твоей, потому что Ты призрел на бедствие мое, узнал горесть души моей
\end{tcolorbox}
\begin{tcolorbox}
\textsubscript{8} (30-9) и не предал меня в руки врага; поставил ноги мои на пространном месте.
\end{tcolorbox}
\begin{tcolorbox}
\textsubscript{9} (30-10) Помилуй меня, Господи, ибо тесно мне; иссохло от горести око мое, душа моя и утроба моя.
\end{tcolorbox}
\begin{tcolorbox}
\textsubscript{10} (30-11) Истощилась в печали жизнь моя и лета мои в стенаниях; изнемогла от грехов моих сила моя, и кости мои иссохли.
\end{tcolorbox}
\begin{tcolorbox}
\textsubscript{11} (30-12) От всех врагов моих я сделался поношением даже у соседей моих и страшилищем для знакомых моих; видящие меня на улице бегут от меня.
\end{tcolorbox}
\begin{tcolorbox}
\textsubscript{12} (30-13) Я забыт в сердцах, как мертвый; я--как сосуд разбитый,
\end{tcolorbox}
\begin{tcolorbox}
\textsubscript{13} (30-14) ибо слышу злоречие многих; отвсюду ужас, когда они сговариваются против меня, умышляют исторгнуть душу мою.
\end{tcolorbox}
\begin{tcolorbox}
\textsubscript{14} (30-15) А я на Тебя, Господи, уповаю; я говорю: Ты--мой Бог.
\end{tcolorbox}
\begin{tcolorbox}
\textsubscript{15} (30-16) В Твоей руке дни мои; избавь меня от руки врагов моих и от гонителей моих.
\end{tcolorbox}
\begin{tcolorbox}
\textsubscript{16} (30-17) Яви светлое лице Твое рабу Твоему; спаси меня милостью Твоею.
\end{tcolorbox}
\begin{tcolorbox}
\textsubscript{17} (30-18) Господи! да не постыжусь, что я к Тебе взываю; нечестивые же да посрамятся, да умолкнут в аде.
\end{tcolorbox}
\begin{tcolorbox}
\textsubscript{18} (30-19) Да онемеют уста лживые, которые против праведника говорят злое с гордостью и презреньем.
\end{tcolorbox}
\begin{tcolorbox}
\textsubscript{19} (30-20) Как много у Тебя благ, которые Ты хранишь для боящихся Тебя и которые приготовил уповающим на Тебя пред сынами человеческими!
\end{tcolorbox}
\begin{tcolorbox}
\textsubscript{20} (30-21) Ты укрываешь их под покровом лица Твоего от мятежей людских, скрываешь их под сенью от пререкания языков.
\end{tcolorbox}
\begin{tcolorbox}
\textsubscript{21} (30-22) Благословен Господь, что явил мне дивную милость Свою в укрепленном городе!
\end{tcolorbox}
\begin{tcolorbox}
\textsubscript{22} (30-23) В смятении моем я думал: 'отвержен я от очей Твоих'; но Ты услышал голос молитвы моей, когда я воззвал к Тебе.
\end{tcolorbox}
\begin{tcolorbox}
\textsubscript{23} (30-24) Любите Господа, все праведные Его; Господь хранит верных и поступающим надменно воздает с избытком.
\end{tcolorbox}
\begin{tcolorbox}
\textsubscript{24} (30-25) Мужайтесь, и да укрепляется сердце ваше, все надеющиеся на Господа!
\end{tcolorbox}
\subsection{CHAPTER 32}
\begin{tcolorbox}
\textsubscript{1} (31-1) ^^Псалом Давида. Учение.^^ Блажен, кому отпущены беззакония, и чьи грехи покрыты!
\end{tcolorbox}
\begin{tcolorbox}
\textsubscript{2} (31-2) Блажен человек, которому Господь не вменит греха, и в чьем духе нет лукавства!
\end{tcolorbox}
\begin{tcolorbox}
\textsubscript{3} (31-3) Когда я молчал, обветшали кости мои от вседневного стенания моего,
\end{tcolorbox}
\begin{tcolorbox}
\textsubscript{4} (31-4) ибо день и ночь тяготела надо мною рука Твоя; свежесть моя исчезла, как в летнюю засуху.
\end{tcolorbox}
\begin{tcolorbox}
\textsubscript{5} (31-5) Но я открыл Тебе грех мой и не скрыл беззакония моего; я сказал: 'исповедаю Господу преступления мои', и Ты снял с меня вину греха моего.
\end{tcolorbox}
\begin{tcolorbox}
\textsubscript{6} (31-6) За то помолится Тебе каждый праведник во время благопотребное, и тогда разлитие многих вод не достигнет его.
\end{tcolorbox}
\begin{tcolorbox}
\textsubscript{7} (31-7) Ты покров мой: Ты охраняешь меня от скорби, окружаешь меня радостями избавления.
\end{tcolorbox}
\begin{tcolorbox}
\textsubscript{8} (31-8) 'Вразумлю тебя, наставлю тебя на путь, по которому тебе идти; буду руководить тебя, око Мое над тобою'.
\end{tcolorbox}
\begin{tcolorbox}
\textsubscript{9} (31-9) 'Не будьте как конь, как лошак несмысленный, которых челюсти нужно обуздывать уздою и удилами, чтобы они покорялись тебе'.
\end{tcolorbox}
\begin{tcolorbox}
\textsubscript{10} (31-10) Много скорбей нечестивому, а уповающего на Господа окружает милость.
\end{tcolorbox}
\begin{tcolorbox}
\textsubscript{11} (31-11) Веселитесь о Господе и радуйтесь, праведные; торжествуйте, все правые сердцем.
\end{tcolorbox}
\subsection{CHAPTER 33}
\begin{tcolorbox}
\textsubscript{1} (32-1) Радуйтесь, праведные, о Господе: правым прилично славословить.
\end{tcolorbox}
\begin{tcolorbox}
\textsubscript{2} (32-2) Славьте Господа на гуслях, пойте Ему на десятиструнной псалтири;
\end{tcolorbox}
\begin{tcolorbox}
\textsubscript{3} (32-3) пойте Ему новую песнь; пойте Ему стройно, с восклицанием,
\end{tcolorbox}
\begin{tcolorbox}
\textsubscript{4} (32-4) ибо слово Господне право и все дела Его верны.
\end{tcolorbox}
\begin{tcolorbox}
\textsubscript{5} (32-5) Он любит правду и суд; милости Господней полна земля.
\end{tcolorbox}
\begin{tcolorbox}
\textsubscript{6} (32-6) Словом Господа сотворены небеса, и духом уст Его--все воинство их:
\end{tcolorbox}
\begin{tcolorbox}
\textsubscript{7} (32-7) Он собрал, будто груды, морские воды, положил бездны в хранилищах.
\end{tcolorbox}
\begin{tcolorbox}
\textsubscript{8} (32-8) Да боится Господа вся земля; да трепещут пред Ним все живущие во вселенной,
\end{tcolorbox}
\begin{tcolorbox}
\textsubscript{9} (32-9) ибо Он сказал, --и сделалось; Он повелел, --и явилось.
\end{tcolorbox}
\begin{tcolorbox}
\textsubscript{10} (32-10) Господь разрушает советы язычников, уничтожает замыслы народов.
\end{tcolorbox}
\begin{tcolorbox}
\textsubscript{11} (32-11) Совет же Господень стоит вовек; помышления сердца Его--в род и род.
\end{tcolorbox}
\begin{tcolorbox}
\textsubscript{12} (32-12) Блажен народ, у которого Господь есть Бог, --племя, которое Он избрал в наследие Себе.
\end{tcolorbox}
\begin{tcolorbox}
\textsubscript{13} (32-13) С небес призирает Господь, видит всех сынов человеческих;
\end{tcolorbox}
\begin{tcolorbox}
\textsubscript{14} (32-14) с престола, на котором восседает, Он призирает на всех, живущих на земле:
\end{tcolorbox}
\begin{tcolorbox}
\textsubscript{15} (32-15) Он создал сердца всех их и вникает во все дела их.
\end{tcolorbox}
\begin{tcolorbox}
\textsubscript{16} (32-16) Не спасется царь множеством воинства; исполина не защитит великая сила.
\end{tcolorbox}
\begin{tcolorbox}
\textsubscript{17} (32-17) Ненадежен конь для спасения, не избавит великою силою своею.
\end{tcolorbox}
\begin{tcolorbox}
\textsubscript{18} (32-18) Вот, око Господне над боящимися Его и уповающими на милость Его,
\end{tcolorbox}
\begin{tcolorbox}
\textsubscript{19} (32-19) что Он душу их спасет от смерти и во время голода пропитает их.
\end{tcolorbox}
\begin{tcolorbox}
\textsubscript{20} (32-20) Душа наша уповает на Господа: Он--помощь наша и защита наша;
\end{tcolorbox}
\begin{tcolorbox}
\textsubscript{21} (32-21) о Нем веселится сердце наше, ибо на святое имя Его мы уповали.
\end{tcolorbox}
\begin{tcolorbox}
\textsubscript{22} (32-22) Да будет милость Твоя, Господи, над нами, как мы уповаем на Тебя.
\end{tcolorbox}
\subsection{CHAPTER 34}
\begin{tcolorbox}
\textsubscript{1} (33-1) ^^Псалом Давида, когда он притворился безумным пред Авимелехом и был изгнан от него и удалился.^^ (33-2) Благословлю Господа во всякое время; хвала Ему непрестанно в устах моих.
\end{tcolorbox}
\begin{tcolorbox}
\textsubscript{2} (33-3) Господом будет хвалиться душа моя; услышат кроткие и возвеселятся.
\end{tcolorbox}
\begin{tcolorbox}
\textsubscript{3} (33-4) Величайте Господа со мною, и превознесем имя Его вместе.
\end{tcolorbox}
\begin{tcolorbox}
\textsubscript{4} (33-5) Я взыскал Господа, и Он услышал меня, и от всех опасностей моих избавил меня.
\end{tcolorbox}
\begin{tcolorbox}
\textsubscript{5} (33-6) Кто обращал взор к Нему, те просвещались, и лица их не постыдятся.
\end{tcolorbox}
\begin{tcolorbox}
\textsubscript{6} (33-7) Сей нищий воззвал, --и Господь услышал и спас его от всех бед его.
\end{tcolorbox}
\begin{tcolorbox}
\textsubscript{7} (33-8) Ангел Господень ополчается вокруг боящихся Его и избавляет их.
\end{tcolorbox}
\begin{tcolorbox}
\textsubscript{8} (33-9) Вкусите, и увидите, как благ Господь! Блажен человек, который уповает на Него!
\end{tcolorbox}
\begin{tcolorbox}
\textsubscript{9} (33-10) Бойтесь Господа, святые Его, ибо нет скудости у боящихся Его.
\end{tcolorbox}
\begin{tcolorbox}
\textsubscript{10} (33-11) Скимны бедствуют и терпят голод, а ищущие Господа не терпят нужды ни в каком благе.
\end{tcolorbox}
\begin{tcolorbox}
\textsubscript{11} (33-12) Придите, дети, послушайте меня: страху Господню научу вас.
\end{tcolorbox}
\begin{tcolorbox}
\textsubscript{12} (33-13) Хочет ли человек жить и любит ли долгоденствие, чтобы видеть благо?
\end{tcolorbox}
\begin{tcolorbox}
\textsubscript{13} (33-14) Удерживай язык свой от зла и уста свои от коварных слов.
\end{tcolorbox}
\begin{tcolorbox}
\textsubscript{14} (33-15) Уклоняйся от зла и делай добро; ищи мира и следуй за ним.
\end{tcolorbox}
\begin{tcolorbox}
\textsubscript{15} (33-16) Очи Господни [обращены] на праведников, и уши Его--к воплю их.
\end{tcolorbox}
\begin{tcolorbox}
\textsubscript{16} (33-17) Но лице Господне против делающих зло, чтобы истребить с земли память о них.
\end{tcolorbox}
\begin{tcolorbox}
\textsubscript{17} (33-18) Взывают [праведные], и Господь слышит, и от всех скорбей их избавляет их.
\end{tcolorbox}
\begin{tcolorbox}
\textsubscript{18} (33-19) Близок Господь к сокрушенным сердцем и смиренных духом спасет.
\end{tcolorbox}
\begin{tcolorbox}
\textsubscript{19} (33-20) Много скорбей у праведного, и от всех их избавит его Господь.
\end{tcolorbox}
\begin{tcolorbox}
\textsubscript{20} (33-21) Он хранит все кости его; ни одна из них не сокрушится.
\end{tcolorbox}
\begin{tcolorbox}
\textsubscript{21} (33-22) Убьет грешника зло, и ненавидящие праведного погибнут.
\end{tcolorbox}
\begin{tcolorbox}
\textsubscript{22} (33-23) Избавит Господь душу рабов Своих, и никто из уповающих на Него не погибнет.
\end{tcolorbox}
\subsection{CHAPTER 35}
\begin{tcolorbox}
\textsubscript{1} (34-1) ^^Псалом Давида.^^ Вступись, Господи, в тяжбу с тяжущимися со мною, побори борющихся со мною;
\end{tcolorbox}
\begin{tcolorbox}
\textsubscript{2} (34-2) возьми щит и латы и восстань на помощь мне;
\end{tcolorbox}
\begin{tcolorbox}
\textsubscript{3} (34-3) обнажи мечь и прегради [путь] преследующим меня; скажи душе моей: 'Я--спасение твое!'
\end{tcolorbox}
\begin{tcolorbox}
\textsubscript{4} (34-4) Да постыдятся и посрамятся ищущие души моей; да обратятся назад и покроются бесчестием умышляющие мне зло;
\end{tcolorbox}
\begin{tcolorbox}
\textsubscript{5} (34-5) да будут они, как прах пред лицем ветра, и Ангел Господень да прогоняет [их];
\end{tcolorbox}
\begin{tcolorbox}
\textsubscript{6} (34-6) да будет путь их темен и скользок, и Ангел Господень да преследует их,
\end{tcolorbox}
\begin{tcolorbox}
\textsubscript{7} (34-7) ибо они без вины скрыли для меня яму--сеть свою, без вины выкопали [ее] для души моей.
\end{tcolorbox}
\begin{tcolorbox}
\textsubscript{8} (34-8) Да придет на него гибель неожиданная, и сеть его, которую он скрыл [для меня], да уловит его самого; да впадет в нее на погибель.
\end{tcolorbox}
\begin{tcolorbox}
\textsubscript{9} (34-9) А моя душа будет радоваться о Господе, будет веселиться о спасении от Него.
\end{tcolorbox}
\begin{tcolorbox}
\textsubscript{10} (34-10) Все кости мои скажут: 'Господи! кто подобен Тебе, избавляющему слабого от сильного, бедного и нищего от грабителя его?'
\end{tcolorbox}
\begin{tcolorbox}
\textsubscript{11} (34-11) Восстали на меня свидетели неправедные: чего я не знаю, о том допрашивают меня;
\end{tcolorbox}
\begin{tcolorbox}
\textsubscript{12} (34-12) воздают мне злом за добро, сиротством душе моей.
\end{tcolorbox}
\begin{tcolorbox}
\textsubscript{13} (34-13) Я во время болезни их одевался во вретище, изнурял постом душу мою, и молитва моя возвращалась в недро мое.
\end{tcolorbox}
\begin{tcolorbox}
\textsubscript{14} (34-14) Я поступал, как бы это был друг мой, брат мой; я ходил скорбный, с поникшею головою, как бы оплакивающий мать.
\end{tcolorbox}
\begin{tcolorbox}
\textsubscript{15} (34-15) А когда я претыкался, они радовались и собирались; собирались ругатели против меня, не знаю за что, поносили и не переставали;
\end{tcolorbox}
\begin{tcolorbox}
\textsubscript{16} (34-16) с лицемерными насмешниками скрежетали на меня зубами своими.
\end{tcolorbox}
\begin{tcolorbox}
\textsubscript{17} (34-17) Господи! долго ли будешь смотреть [на это]? Отведи душу мою от злодейств их, от львов--одинокую мою.
\end{tcolorbox}
\begin{tcolorbox}
\textsubscript{18} (34-18) Я прославлю Тебя в собрании великом, среди народа многочисленного восхвалю Тебя,
\end{tcolorbox}
\begin{tcolorbox}
\textsubscript{19} (34-19) чтобы не торжествовали надо мною враждующие против меня неправедно, и не перемигивались глазами ненавидящие меня безвинно;
\end{tcolorbox}
\begin{tcolorbox}
\textsubscript{20} (34-20) ибо не о мире говорят они, но против мирных земли составляют лукавые замыслы;
\end{tcolorbox}
\begin{tcolorbox}
\textsubscript{21} (34-21) расширяют на меня уста свои; говорят: 'хорошо! хорошо! видел глаз наш'.
\end{tcolorbox}
\begin{tcolorbox}
\textsubscript{22} (34-22) Ты видел, Господи, не умолчи; Господи! не удаляйся от меня.
\end{tcolorbox}
\begin{tcolorbox}
\textsubscript{23} (34-23) Подвигнись, пробудись для суда моего, для тяжбы моей, Боже мой и Господи мой!
\end{tcolorbox}
\begin{tcolorbox}
\textsubscript{24} (34-24) Суди меня по правде Твоей, Господи, Боже мой, и да не торжествуют они надо мною;
\end{tcolorbox}
\begin{tcolorbox}
\textsubscript{25} (34-25) да не говорят в сердце своем: 'хорошо! по душе нашей!' Да не говорят: 'мы поглотили его'.
\end{tcolorbox}
\begin{tcolorbox}
\textsubscript{26} (34-26) Да постыдятся и посрамятся все, радующиеся моему несчастью; да облекутся в стыд и позор величающиеся надо мною.
\end{tcolorbox}
\begin{tcolorbox}
\textsubscript{27} (34-27) Да радуются и веселятся желающие правоты моей и говорят непрестанно: 'да возвеличится Господь, желающий мира рабу Своему!'
\end{tcolorbox}
\begin{tcolorbox}
\textsubscript{28} (34-28) И язык мой будет проповедывать правду Твою и хвалу Твою всякий день.
\end{tcolorbox}
\subsection{CHAPTER 36}
\begin{tcolorbox}
\textsubscript{1} (35-1) ^^Начальнику хора. Раба Господня Давида.^^ (35-2) Нечестие беззаконного говорит в сердце моем: нет страха Божия пред глазами его,
\end{tcolorbox}
\begin{tcolorbox}
\textsubscript{2} (35-3) ибо он льстит себе в глазах своих, будто отыскивает беззаконие свое, чтобы возненавидеть его;
\end{tcolorbox}
\begin{tcolorbox}
\textsubscript{3} (35-4) слова уст его--неправда и лукавство; не хочет он вразумиться, чтобы делать добро;
\end{tcolorbox}
\begin{tcolorbox}
\textsubscript{4} (35-5) на ложе своем замышляет беззаконие, становится на путь недобрый, не гнушается злом.
\end{tcolorbox}
\begin{tcolorbox}
\textsubscript{5} (35-6) Господи! милость Твоя до небес, истина Твоя до облаков!
\end{tcolorbox}
\begin{tcolorbox}
\textsubscript{6} (35-7) Правда Твоя, как горы Божии, и судьбы Твои--бездна великая! Человеков и скотов хранишь Ты, Господи!
\end{tcolorbox}
\begin{tcolorbox}
\textsubscript{7} (35-8) Как драгоценна милость Твоя, Боже! Сыны человеческие в тени крыл Твоих покойны:
\end{tcolorbox}
\begin{tcolorbox}
\textsubscript{8} (35-9) насыщаются от тука дома Твоего, и из потока сладостей Твоих Ты напояешь их,
\end{tcolorbox}
\begin{tcolorbox}
\textsubscript{9} (35-10) ибо у Тебя источник жизни; во свете Твоем мы видим свет.
\end{tcolorbox}
\begin{tcolorbox}
\textsubscript{10} (35-11) Продли милость Твою к знающим Тебя и правду Твою к правым сердцем,
\end{tcolorbox}
\begin{tcolorbox}
\textsubscript{11} (35-12) да не наступит на меня нога гордыни, и рука грешника да не изгонит меня:
\end{tcolorbox}
\begin{tcolorbox}
\textsubscript{12} (35-13) там пали делающие беззаконие, низринуты и не могут встать.
\end{tcolorbox}
\subsection{CHAPTER 37}
\begin{tcolorbox}
\textsubscript{1} (36-1) ^^Псалом Давида.^^ Не ревнуй злодеям, не завидуй делающим беззаконие,
\end{tcolorbox}
\begin{tcolorbox}
\textsubscript{2} (36-2) ибо они, как трава, скоро будут подкошены и, как зеленеющий злак, увянут.
\end{tcolorbox}
\begin{tcolorbox}
\textsubscript{3} (36-3) Уповай на Господа и делай добро; живи на земле и храни истину.
\end{tcolorbox}
\begin{tcolorbox}
\textsubscript{4} (36-4) Утешайся Господом, и Он исполнит желания сердца твоего.
\end{tcolorbox}
\begin{tcolorbox}
\textsubscript{5} (36-5) Предай Господу путь твой и уповай на Него, и Он совершит,
\end{tcolorbox}
\begin{tcolorbox}
\textsubscript{6} (36-6) и выведет, как свет, правду твою и справедливость твою, как полдень.
\end{tcolorbox}
\begin{tcolorbox}
\textsubscript{7} (36-7) Покорись Господу и надейся на Него. Не ревнуй успевающему в пути своем, человеку лукавствующему.
\end{tcolorbox}
\begin{tcolorbox}
\textsubscript{8} (36-8) Перестань гневаться и оставь ярость; не ревнуй до того, чтобы делать зло,
\end{tcolorbox}
\begin{tcolorbox}
\textsubscript{9} (36-9) ибо делающие зло истребятся, уповающие же на Господа наследуют землю.
\end{tcolorbox}
\begin{tcolorbox}
\textsubscript{10} (36-10) Еще немного, и не станет нечестивого; посмотришь на его место, и нет его.
\end{tcolorbox}
\begin{tcolorbox}
\textsubscript{11} (36-11) А кроткие наследуют землю и насладятся множеством мира.
\end{tcolorbox}
\begin{tcolorbox}
\textsubscript{12} (36-12) Нечестивый злоумышляет против праведника и скрежещет на него зубами своими:
\end{tcolorbox}
\begin{tcolorbox}
\textsubscript{13} (36-13) Господь же посмевается над ним, ибо видит, что приходит день его.
\end{tcolorbox}
\begin{tcolorbox}
\textsubscript{14} (36-14) Нечестивые обнажают меч и натягивают лук свой, чтобы низложить бедного и нищего, чтобы пронзить [идущих] прямым путем:
\end{tcolorbox}
\begin{tcolorbox}
\textsubscript{15} (36-15) меч их войдет в их же сердце, и луки их сокрушатся.
\end{tcolorbox}
\begin{tcolorbox}
\textsubscript{16} (36-16) Малое у праведника--лучше богатства многих нечестивых,
\end{tcolorbox}
\begin{tcolorbox}
\textsubscript{17} (36-17) ибо мышцы нечестивых сокрушатся, а праведников подкрепляет Господь.
\end{tcolorbox}
\begin{tcolorbox}
\textsubscript{18} (36-18) Господь знает дни непорочных, и достояние их пребудет вовек:
\end{tcolorbox}
\begin{tcolorbox}
\textsubscript{19} (36-19) не будут они постыжены во время лютое и во дни голода будут сыты;
\end{tcolorbox}
\begin{tcolorbox}
\textsubscript{20} (36-20) а нечестивые погибнут, и враги Господни, как тук агнцев, исчезнут, в дыме исчезнут.
\end{tcolorbox}
\begin{tcolorbox}
\textsubscript{21} (36-21) Нечестивый берет взаймы и не отдает, а праведник милует и дает,
\end{tcolorbox}
\begin{tcolorbox}
\textsubscript{22} (36-22) ибо благословенные Им наследуют землю, а проклятые Им истребятся.
\end{tcolorbox}
\begin{tcolorbox}
\textsubscript{23} (36-23) Господом утверждаются стопы [такого] человека, и Он благоволит к пути его:
\end{tcolorbox}
\begin{tcolorbox}
\textsubscript{24} (36-24) когда он будет падать, не упадет, ибо Господь поддерживает его за руку.
\end{tcolorbox}
\begin{tcolorbox}
\textsubscript{25} (36-25) Я был молод и состарился, и не видал праведника оставленным и потомков его просящими хлеба:
\end{tcolorbox}
\begin{tcolorbox}
\textsubscript{26} (36-26) он всякий день милует и взаймы дает, и потомство его в благословение будет.
\end{tcolorbox}
\begin{tcolorbox}
\textsubscript{27} (36-27) Уклоняйся от зла, и делай добро, и будешь жить вовек:
\end{tcolorbox}
\begin{tcolorbox}
\textsubscript{28} (36-28) ибо Господь любит правду и не оставляет святых Своих; вовек сохранятся они; и потомство нечестивых истребится.
\end{tcolorbox}
\begin{tcolorbox}
\textsubscript{29} (36-29) Праведники наследуют землю и будут жить на ней вовек.
\end{tcolorbox}
\begin{tcolorbox}
\textsubscript{30} (36-30) Уста праведника изрекают премудрость, и язык его произносит правду.
\end{tcolorbox}
\begin{tcolorbox}
\textsubscript{31} (36-31) Закон Бога его в сердце у него; не поколеблются стопы его.
\end{tcolorbox}
\begin{tcolorbox}
\textsubscript{32} (36-32) Нечестивый подсматривает за праведником и ищет умертвить его;
\end{tcolorbox}
\begin{tcolorbox}
\textsubscript{33} (36-33) но Господь не отдаст его в руки его и не допустит обвинить его, когда он будет судим.
\end{tcolorbox}
\begin{tcolorbox}
\textsubscript{34} (36-34) Уповай на Господа и держись пути Его: и Он вознесет тебя, чтобы ты наследовал землю; и когда будут истребляемы нечестивые, ты увидишь.
\end{tcolorbox}
\begin{tcolorbox}
\textsubscript{35} (36-35) Видел я нечестивца грозного, расширявшегося, подобно укоренившемуся многоветвистому дереву;
\end{tcolorbox}
\begin{tcolorbox}
\textsubscript{36} (36-36) но он прошел, и вот нет его; ищу его и не нахожу.
\end{tcolorbox}
\begin{tcolorbox}
\textsubscript{37} (36-37) Наблюдай за непорочным и смотри на праведного, ибо будущность [такого] человека есть мир;
\end{tcolorbox}
\begin{tcolorbox}
\textsubscript{38} (36-38) а беззаконники все истребятся; будущность нечестивых погибнет.
\end{tcolorbox}
\begin{tcolorbox}
\textsubscript{39} (36-39) От Господа спасение праведникам, Он--защита их во время скорби;
\end{tcolorbox}
\begin{tcolorbox}
\textsubscript{40} (36-40) и поможет им Господь и избавит их; избавит их от нечестивых и спасет их, ибо они на Него уповают.
\end{tcolorbox}
\subsection{CHAPTER 38}
\begin{tcolorbox}
\textsubscript{1} (37-1) ^^Псалом Давида. В воспоминание^^. (37-2) Господи! не в ярости Твоей обличай меня и не во гневе Твоем наказывай меня,
\end{tcolorbox}
\begin{tcolorbox}
\textsubscript{2} (37-3) ибо стрелы Твои вонзились в меня, и рука Твоя тяготеет на мне.
\end{tcolorbox}
\begin{tcolorbox}
\textsubscript{3} (37-4) Нет целого места в плоти моей от гнева Твоего; нет мира в костях моих от грехов моих,
\end{tcolorbox}
\begin{tcolorbox}
\textsubscript{4} (37-5) ибо беззакония мои превысили голову мою, как тяжелое бремя отяготели на мне,
\end{tcolorbox}
\begin{tcolorbox}
\textsubscript{5} (37-6) смердят, гноятся раны мои от безумия моего.
\end{tcolorbox}
\begin{tcolorbox}
\textsubscript{6} (37-7) Я согбен и совсем поник, весь день сетуя хожу,
\end{tcolorbox}
\begin{tcolorbox}
\textsubscript{7} (37-8) ибо чресла мои полны воспалениями, и нет целого места в плоти моей.
\end{tcolorbox}
\begin{tcolorbox}
\textsubscript{8} (37-9) Я изнемог и сокрушен чрезмерно; кричу от терзания сердца моего.
\end{tcolorbox}
\begin{tcolorbox}
\textsubscript{9} (37-10) Господи! пред Тобою все желания мои, и воздыхание мое не сокрыто от Тебя.
\end{tcolorbox}
\begin{tcolorbox}
\textsubscript{10} (37-11) Сердце мое трепещет; оставила меня сила моя, и свет очей моих, --и того нет у меня.
\end{tcolorbox}
\begin{tcolorbox}
\textsubscript{11} (37-12) Друзья мои и искренние отступили от язвы моей, и ближние мои стоят вдали.
\end{tcolorbox}
\begin{tcolorbox}
\textsubscript{12} (37-13) Ищущие же души моей ставят сети, и желающие мне зла говорят о погибели [моей] и замышляют всякий день козни;
\end{tcolorbox}
\begin{tcolorbox}
\textsubscript{13} (37-14) а я, как глухой, не слышу, и как немой, который не открывает уст своих;
\end{tcolorbox}
\begin{tcolorbox}
\textsubscript{14} (37-15) и стал я, как человек, который не слышит и не имеет в устах своих ответа,
\end{tcolorbox}
\begin{tcolorbox}
\textsubscript{15} (37-16) ибо на Тебя, Господи, уповаю я; Ты услышишь, Господи, Боже мой.
\end{tcolorbox}
\begin{tcolorbox}
\textsubscript{16} (37-17) И я сказал: да не восторжествуют надо мною [враги мои]; когда колеблется нога моя, они величаются надо мною.
\end{tcolorbox}
\begin{tcolorbox}
\textsubscript{17} (37-18) Я близок к падению, и скорбь моя всегда предо мною.
\end{tcolorbox}
\begin{tcolorbox}
\textsubscript{18} (37-19) Беззаконие мое я сознаю, сокрушаюсь о грехе моем.
\end{tcolorbox}
\begin{tcolorbox}
\textsubscript{19} (37-20) А враги мои живут и укрепляются, и умножаются ненавидящие меня безвинно;
\end{tcolorbox}
\begin{tcolorbox}
\textsubscript{20} (37-21) и воздающие мне злом за добро враждуют против меня за то, что я следую добру.
\end{tcolorbox}
\begin{tcolorbox}
\textsubscript{21} (37-22) Не оставь меня, Господи, Боже мой! Не удаляйся от меня;
\end{tcolorbox}
\begin{tcolorbox}
\textsubscript{22} (37-23) поспеши на помощь мне, Господи, Спаситель мой!
\end{tcolorbox}
\subsection{CHAPTER 39}
\begin{tcolorbox}
\textsubscript{1} (38-1) ^^Начальнику хора, Идифуму. Псалом Давида.^^ (38-2) Я сказал: буду я наблюдать за путями моими, чтобы не согрешать мне языком моим; буду обуздывать уста мои, доколе нечестивый предо мною.
\end{tcolorbox}
\begin{tcolorbox}
\textsubscript{2} (38-3) Я был нем и безгласен, и молчал [даже] о добром; и скорбь моя подвиглась.
\end{tcolorbox}
\begin{tcolorbox}
\textsubscript{3} (38-4) Воспламенилось сердце мое во мне; в мыслях моих возгорелся огонь; я стал говорить языком моим:
\end{tcolorbox}
\begin{tcolorbox}
\textsubscript{4} (38-5) скажи мне, Господи, кончину мою и число дней моих, какое оно, дабы я знал, какой век мой.
\end{tcolorbox}
\begin{tcolorbox}
\textsubscript{5} (38-6) Вот, Ты дал мне дни, [как] пяди, и век мой как ничто пред Тобою. Подлинно, совершенная суета--всякий человек живущий.
\end{tcolorbox}
\begin{tcolorbox}
\textsubscript{6} (38-7) Подлинно, человек ходит подобно призраку; напрасно он суетится, собирает и не знает, кому достанется то.
\end{tcolorbox}
\begin{tcolorbox}
\textsubscript{7} (38-8) И ныне чего ожидать мне, Господи? надежда моя--на Тебя.
\end{tcolorbox}
\begin{tcolorbox}
\textsubscript{8} (38-9) От всех беззаконий моих избавь меня, не предавай меня на поругание безумному.
\end{tcolorbox}
\begin{tcolorbox}
\textsubscript{9} (38-10) Я стал нем, не открываю уст моих; потому что Ты соделал это.
\end{tcolorbox}
\begin{tcolorbox}
\textsubscript{10} (38-11) Отклони от меня удары Твои; я исчезаю от поражающей руки Твоей.
\end{tcolorbox}
\begin{tcolorbox}
\textsubscript{11} (38-12) Если Ты обличениями будешь наказывать человека за преступления, то рассыплется, как от моли, краса его. Так, суетен всякий человек!
\end{tcolorbox}
\begin{tcolorbox}
\textsubscript{12} (38-13) Услышь, Господи, молитву мою и внемли воплю моему; не будь безмолвен к слезам моим, ибо странник я у Тебя [и] пришлец, как и все отцы мои.
\end{tcolorbox}
\begin{tcolorbox}
\textsubscript{13} (38-14) Отступи от меня, чтобы я мог подкрепиться, прежде нежели отойду и не будет меня.
\end{tcolorbox}
\subsection{CHAPTER 40}
\begin{tcolorbox}
\textsubscript{1} (39-1) ^^Начальнику хора. Псалом Давида.^^ (39-2) Твердо уповал я на Господа, и Он приклонился ко мне и услышал вопль мой;
\end{tcolorbox}
\begin{tcolorbox}
\textsubscript{2} (39-3) извлек меня из страшного рва, из тинистого болота, и поставил на камне ноги мои и утвердил стопы мои;
\end{tcolorbox}
\begin{tcolorbox}
\textsubscript{3} (39-4) и вложил в уста мои новую песнь--хвалу Богу нашему. Увидят многие и убоятся и будут уповать на Господа.
\end{tcolorbox}
\begin{tcolorbox}
\textsubscript{4} (39-5) Блажен человек, который на Господа возлагает надежду свою и не обращается к гордым и к уклоняющимся ко лжи.
\end{tcolorbox}
\begin{tcolorbox}
\textsubscript{5} (39-6) Много соделал Ты, Господи, Боже мой: о чудесах и помышлениях Твоих о нас--кто уподобится Тебе! --хотел бы я проповедывать и говорить, но они превышают число.
\end{tcolorbox}
\begin{tcolorbox}
\textsubscript{6} (39-7) Жертвы и приношения Ты не восхотел; Ты открыл мне уши; всесожжения и жертвы за грех Ты не потребовал.
\end{tcolorbox}
\begin{tcolorbox}
\textsubscript{7} (39-8) Тогда я сказал: вот, иду; в свитке книжном написано о мне:
\end{tcolorbox}
\begin{tcolorbox}
\textsubscript{8} (39-9) я желаю исполнить волю Твою, Боже мой, и закон Твой у меня в сердце.
\end{tcolorbox}
\begin{tcolorbox}
\textsubscript{9} (39-10) Я возвещал правду Твою в собрании великом; я не возбранял устам моим: Ты, Господи, знаешь.
\end{tcolorbox}
\begin{tcolorbox}
\textsubscript{10} (39-11) Правды Твоей не скрывал в сердце моем, возвещал верность Твою и спасение Твое, не утаивал милости Твоей и истины Твоей пред собранием великим.
\end{tcolorbox}
\begin{tcolorbox}
\textsubscript{11} (39-12) Не удерживай, Господи, щедрот Твоих от меня; милость Твоя и истина Твоя да охраняют меня непрестанно,
\end{tcolorbox}
\begin{tcolorbox}
\textsubscript{12} (39-13) ибо окружили меня беды неисчислимые; постигли меня беззакония мои, так что видеть не могу: их более, нежели волос на голове моей; сердце мое оставило меня.
\end{tcolorbox}
\begin{tcolorbox}
\textsubscript{13} (39-14) Благоволи, Господи, избавить меня; Господи! поспеши на помощь мне.
\end{tcolorbox}
\begin{tcolorbox}
\textsubscript{14} (39-15) Да постыдятся и посрамятся все, ищущие погибели душе моей! Да будут обращены назад и преданы посмеянию желающие мне зла!
\end{tcolorbox}
\begin{tcolorbox}
\textsubscript{15} (39-16) Да смятутся от посрамления своего говорящие мне: 'хорошо! хорошо!'
\end{tcolorbox}
\begin{tcolorbox}
\textsubscript{16} (39-17) Да радуются и веселятся Тобою все ищущие Тебя, и любящие спасение Твое да говорят непрестанно: 'велик Господь!'
\end{tcolorbox}
\begin{tcolorbox}
\textsubscript{17} (39-18) Я же беден и нищ, но Господь печется о мне. Ты--помощь моя и избавитель мой, Боже мой! не замедли.
\end{tcolorbox}
\subsection{CHAPTER 41}
\begin{tcolorbox}
\textsubscript{1} (40-1) ^^Начальнику хора. Псалом Давида.^^ (40-2) Блажен, кто помышляет о бедном! В день бедствия избавит его Господь.
\end{tcolorbox}
\begin{tcolorbox}
\textsubscript{2} (40-3) Господь сохранит его и сбережет ему жизнь; блажен будет он на земле. И Ты не отдашь его на волю врагов его.
\end{tcolorbox}
\begin{tcolorbox}
\textsubscript{3} (40-4) Господь укрепит его на одре болезни его. Ты изменишь все ложе его в болезни его.
\end{tcolorbox}
\begin{tcolorbox}
\textsubscript{4} (40-5) Я сказал: Господи! помилуй меня, исцели душу мою, ибо согрешил я пред Тобою.
\end{tcolorbox}
\begin{tcolorbox}
\textsubscript{5} (40-6) Враги мои говорят обо мне злое: 'когда он умрет и погибнет имя его?'
\end{tcolorbox}
\begin{tcolorbox}
\textsubscript{6} (40-7) И если приходит кто видеть меня, говорит ложь; сердце его слагает в себе неправду, и он, выйдя вон, толкует.
\end{tcolorbox}
\begin{tcolorbox}
\textsubscript{7} (40-8) Все ненавидящие меня шепчут между собою против меня, замышляют на меня зло:
\end{tcolorbox}
\begin{tcolorbox}
\textsubscript{8} (40-9) 'слово велиала пришло на него; он слег; не встать ему более'.
\end{tcolorbox}
\begin{tcolorbox}
\textsubscript{9} (40-10) Даже человек мирный со мною, на которого я полагался, который ел хлеб мой, поднял на меня пяту.
\end{tcolorbox}
\begin{tcolorbox}
\textsubscript{10} (40-11) Ты же, Господи, помилуй меня и восставь меня, и я воздам им.
\end{tcolorbox}
\begin{tcolorbox}
\textsubscript{11} (40-12) Из того узнаю, что Ты благоволишь ко мне, если враг мой не восторжествует надо мною,
\end{tcolorbox}
\begin{tcolorbox}
\textsubscript{12} (40-13) а меня сохранишь в целости моей и поставишь пред лицем Твоим на веки.
\end{tcolorbox}
\begin{tcolorbox}
\textsubscript{13} (40-14) Благословен Господь Бог Израилев от века и до века! Аминь, аминь!
\end{tcolorbox}
\subsection{CHAPTER 42}
\begin{tcolorbox}
\textsubscript{1} (41-1) ^^Начальнику хора. Учение. Сынов Кореевых.^^ (41-2) Как лань желает к потокам воды, так желает душа моя к Тебе, Боже!
\end{tcolorbox}
\begin{tcolorbox}
\textsubscript{2} (41-3) Жаждет душа моя к Богу крепкому, живому: когда приду и явлюсь пред лице Божие!
\end{tcolorbox}
\begin{tcolorbox}
\textsubscript{3} (41-4) Слезы мои были для меня хлебом день и ночь, когда говорили мне всякий день: 'где Бог твой?'
\end{tcolorbox}
\begin{tcolorbox}
\textsubscript{4} (41-5) Вспоминая об этом, изливаю душу мою, потому что я ходил в многолюдстве, вступал с ними в дом Божий со гласом радости и славословия празднующего сонма.
\end{tcolorbox}
\begin{tcolorbox}
\textsubscript{5} (41-6) Что унываешь ты, душа моя, и что смущаешься? Уповай на Бога, ибо я буду еще славить Его, Спасителя моего и Бога моего.
\end{tcolorbox}
\begin{tcolorbox}
\textsubscript{6} (41-7) Унывает во мне душа моя; посему я воспоминаю о Тебе с земли Иорданской, с Ермона, с горы Цоар.
\end{tcolorbox}
\begin{tcolorbox}
\textsubscript{7} (41-8) Бездна бездну призывает голосом водопадов Твоих; все воды Твои и волны Твои прошли надо мною.
\end{tcolorbox}
\begin{tcolorbox}
\textsubscript{8} (41-9) Днем явит Господь милость Свою, и ночью песнь Ему у меня, молитва к Богу жизни моей.
\end{tcolorbox}
\begin{tcolorbox}
\textsubscript{9} (41-10) Скажу Богу, заступнику моему: для чего Ты забыл меня? Для чего я сетуя хожу от оскорблений врага?
\end{tcolorbox}
\begin{tcolorbox}
\textsubscript{10} (41-11) Как бы поражая кости мои, ругаются надо мною враги мои, когда говорят мне всякий день: 'где Бог твой?'
\end{tcolorbox}
\begin{tcolorbox}
\textsubscript{11} (41-12) Что унываешь ты, душа моя, и что смущаешься? Уповай на Бога, ибо я буду еще славить Его, Спасителя моего и Бога моего.
\end{tcolorbox}
\subsection{CHAPTER 43}
\begin{tcolorbox}
\textsubscript{1} (42-1) Суди меня, Боже, и вступись в тяжбу мою с народом недобрым. От человека лукавого и несправедливого избавь меня,
\end{tcolorbox}
\begin{tcolorbox}
\textsubscript{2} (42-2) ибо Ты Бог крепости моей. Для чего Ты отринул меня? для чего я сетуя хожу от оскорблений врага?
\end{tcolorbox}
\begin{tcolorbox}
\textsubscript{3} (42-3) Пошли свет Твой и истину Твою; да ведут они меня и приведут на святую гору Твою и в обители Твои.
\end{tcolorbox}
\begin{tcolorbox}
\textsubscript{4} (42-4) И подойду я к жертвеннику Божию, к Богу радости и веселия моего, и на гуслях буду славить Тебя, Боже, Боже мой!
\end{tcolorbox}
\begin{tcolorbox}
\textsubscript{5} (42-5) Что унываешь ты, душа моя, и что смущаешься? Уповай на Бога; ибо я буду еще славить Его, Спасителя моего и Бога моего.
\end{tcolorbox}
\subsection{CHAPTER 44}
\begin{tcolorbox}
\textsubscript{1} (43-1) ^^Начальнику хора. Учение. Сынов Кореевых.^^ (43-2) Боже, мы слышали ушами своими, отцы наши рассказывали нам о деле, какое Ты соделал во дни их, во дни древние:
\end{tcolorbox}
\begin{tcolorbox}
\textsubscript{2} (43-3) Ты рукою Твоею истребил народы, а их насадил; поразил племена и изгнал их;
\end{tcolorbox}
\begin{tcolorbox}
\textsubscript{3} (43-4) ибо они не мечом своим приобрели землю, и не их мышца спасла их, но Твоя десница и Твоя мышца и свет лица Твоего, ибо Ты благоволил к ним.
\end{tcolorbox}
\begin{tcolorbox}
\textsubscript{4} (43-5) Боже, Царь мой! Ты--тот же; даруй спасение Иакову.
\end{tcolorbox}
\begin{tcolorbox}
\textsubscript{5} (43-6) С Тобою избодаем рогами врагов наших; во имя Твое попрем ногами восстающих на нас:
\end{tcolorbox}
\begin{tcolorbox}
\textsubscript{6} (43-7) ибо не на лук мой уповаю, и не меч мой спасет меня;
\end{tcolorbox}
\begin{tcolorbox}
\textsubscript{7} (43-8) но Ты спасешь нас от врагов наших, и посрамишь ненавидящих нас.
\end{tcolorbox}
\begin{tcolorbox}
\textsubscript{8} (43-9) О Боге похвалимся всякий день, и имя Твое будем прославлять вовек.
\end{tcolorbox}
\begin{tcolorbox}
\textsubscript{9} (43-10) Но ныне Ты отринул и посрамил нас, и не выходишь с войсками нашими;
\end{tcolorbox}
\begin{tcolorbox}
\textsubscript{10} (43-11) обратил нас в бегство от врага, и ненавидящие нас грабят нас;
\end{tcolorbox}
\begin{tcolorbox}
\textsubscript{11} (43-12) Ты отдал нас, как овец, на съедение и рассеял нас между народами;
\end{tcolorbox}
\begin{tcolorbox}
\textsubscript{12} (43-13) без выгоды Ты продал народ Твой и не возвысил цены его;
\end{tcolorbox}
\begin{tcolorbox}
\textsubscript{13} (43-14) отдал нас на поношение соседям нашим, на посмеяние и поругание живущим вокруг нас;
\end{tcolorbox}
\begin{tcolorbox}
\textsubscript{14} (43-15) Ты сделал нас притчею между народами, покиванием головы между иноплеменниками.
\end{tcolorbox}
\begin{tcolorbox}
\textsubscript{15} (43-16) Всякий день посрамление мое предо мною, и стыд покрывает лице мое
\end{tcolorbox}
\begin{tcolorbox}
\textsubscript{16} (43-17) от голоса поносителя и клеветника, от взоров врага и мстителя:
\end{tcolorbox}
\begin{tcolorbox}
\textsubscript{17} (43-18) все это пришло на нас, но мы не забыли Тебя и не нарушили завета Твоего.
\end{tcolorbox}
\begin{tcolorbox}
\textsubscript{18} (43-19) Не отступило назад сердце наше, и стопы наши не уклонились от пути Твоего,
\end{tcolorbox}
\begin{tcolorbox}
\textsubscript{19} (43-20) когда Ты сокрушил нас в земле драконов и покрыл нас тенью смертною.
\end{tcolorbox}
\begin{tcolorbox}
\textsubscript{20} (43-21) Если бы мы забыли имя Бога нашего и простерли руки наши к богу чужому,
\end{tcolorbox}
\begin{tcolorbox}
\textsubscript{21} (43-22) то не взыскал ли бы сего Бог? Ибо Он знает тайны сердца.
\end{tcolorbox}
\begin{tcolorbox}
\textsubscript{22} (43-23) Но за Тебя умерщвляют нас всякий день, считают нас за овец, [обреченных] на заклание.
\end{tcolorbox}
\begin{tcolorbox}
\textsubscript{23} (43-24) Восстань, что спишь, Господи! пробудись, не отринь навсегда.
\end{tcolorbox}
\begin{tcolorbox}
\textsubscript{24} (43-25) Для чего скрываешь лице Твое, забываешь скорбь нашу и угнетение наше?
\end{tcolorbox}
\begin{tcolorbox}
\textsubscript{25} (43-26) ибо душа наша унижена до праха, утроба наша прильнула к земле.
\end{tcolorbox}
\begin{tcolorbox}
\textsubscript{26} (43-27) Восстань на помощь нам и избавь нас ради милости Твоей.
\end{tcolorbox}
\subsection{CHAPTER 45}
\begin{tcolorbox}
\textsubscript{1} (44-1) ^^Начальнику хора. На [музыкальном орудии] Шошан. Учение. Сынов Кореевых. Песнь любви.^^ (44-2) Излилось из сердца моего слово благое; я говорю: песнь моя о Царе; язык мой--трость скорописца.
\end{tcolorbox}
\begin{tcolorbox}
\textsubscript{2} (44-3) Ты прекраснее сынов человеческих; благодать излилась из уст Твоих; посему благословил Тебя Бог на веки.
\end{tcolorbox}
\begin{tcolorbox}
\textsubscript{3} (44-4) Препояшь Себя по бедру мечом Твоим, Сильный, славою Твоею и красотою Твоею,
\end{tcolorbox}
\begin{tcolorbox}
\textsubscript{4} (44-5) и в сем украшении Твоем поспеши, воссядь на колесницу ради истины и кротости и правды, и десница Твоя покажет Тебе дивные дела.
\end{tcolorbox}
\begin{tcolorbox}
\textsubscript{5} (44-6) Остры стрелы Твои; --народы падут пред Тобою, --они--в сердце врагов Царя.
\end{tcolorbox}
\begin{tcolorbox}
\textsubscript{6} (44-7) Престол Твой, Боже, вовек; жезл правоты--жезл царства Твоего.
\end{tcolorbox}
\begin{tcolorbox}
\textsubscript{7} (44-8) Ты возлюбил правду и возненавидел беззаконие, посему помазал Тебя, Боже, Бог Твой елеем радости более соучастников Твоих.
\end{tcolorbox}
\begin{tcolorbox}
\textsubscript{8} (44-9) Все одежды Твои, как смирна и алой и касия; из чертогов слоновой кости увеселяют Тебя.
\end{tcolorbox}
\begin{tcolorbox}
\textsubscript{9} (44-10) Дочери царей между почетными у Тебя; стала царица одесную Тебя в Офирском золоте.
\end{tcolorbox}
\begin{tcolorbox}
\textsubscript{10} (44-11) Слыши, дщерь, и смотри, и приклони ухо твое, и забудь народ твой и дом отца твоего.
\end{tcolorbox}
\begin{tcolorbox}
\textsubscript{11} (44-12) И возжелает Царь красоты твоей; ибо Он Господь твой, и ты поклонись Ему.
\end{tcolorbox}
\begin{tcolorbox}
\textsubscript{12} (44-13) И дочь Тира с дарами, и богатейшие из народа будут умолять лице Твое.
\end{tcolorbox}
\begin{tcolorbox}
\textsubscript{13} (44-14) Вся слава дщери Царя внутри; одежда ее шита золотом;
\end{tcolorbox}
\begin{tcolorbox}
\textsubscript{14} (44-15) в испещренной одежде ведется она к Царю; за нею ведутся к Тебе девы, подруги ее,
\end{tcolorbox}
\begin{tcolorbox}
\textsubscript{15} (44-16) приводятся с весельем и ликованьем, входят в чертог Царя.
\end{tcolorbox}
\begin{tcolorbox}
\textsubscript{16} (44-17) Вместо отцов Твоих, будут сыновья Твои; Ты поставишь их князьями по всей земле.
\end{tcolorbox}
\begin{tcolorbox}
\textsubscript{17} (44-18) Сделаю имя Твое памятным в род и род; посему народы будут славить Тебя во веки и веки.
\end{tcolorbox}
\subsection{CHAPTER 46}
\begin{tcolorbox}
\textsubscript{1} (45-1) ^^Начальнику хора. Сынов Кореевых. На [музыкальном] [орудии] Аламоф. Песнь.^^ (45-2) Бог нам прибежище и сила, скорый помощник в бедах,
\end{tcolorbox}
\begin{tcolorbox}
\textsubscript{2} (45-3) посему не убоимся, хотя бы поколебалась земля, и горы двинулись в сердце морей.
\end{tcolorbox}
\begin{tcolorbox}
\textsubscript{3} (45-4) Пусть шумят, вздымаются воды их, трясутся горы от волнения их.
\end{tcolorbox}
\begin{tcolorbox}
\textsubscript{4} (45-5) Речные потоки веселят град Божий, святое жилище Всевышнего.
\end{tcolorbox}
\begin{tcolorbox}
\textsubscript{5} (45-6) Бог посреди его; он не поколеблется: Бог поможет ему с раннего утра.
\end{tcolorbox}
\begin{tcolorbox}
\textsubscript{6} (45-7) Восшумели народы; двинулись царства: [Всевышний] дал глас Свой, и растаяла земля.
\end{tcolorbox}
\begin{tcolorbox}
\textsubscript{7} (45-8) Господь сил с нами, Бог Иакова заступник наш.
\end{tcolorbox}
\begin{tcolorbox}
\textsubscript{8} (45-9) Придите и видите дела Господа, --какие произвел Он опустошения на земле:
\end{tcolorbox}
\begin{tcolorbox}
\textsubscript{9} (45-10) прекращая брани до края земли, сокрушил лук и переломил копье, колесницы сжег огнем.
\end{tcolorbox}
\begin{tcolorbox}
\textsubscript{10} (45-11) Остановитесь и познайте, что Я--Бог: буду превознесен в народах, превознесен на земле.
\end{tcolorbox}
\begin{tcolorbox}
\textsubscript{11} (45-12) Господь сил с нами, заступник наш Бог Иакова.
\end{tcolorbox}
\subsection{CHAPTER 47}
\begin{tcolorbox}
\textsubscript{1} (46-1) ^^Начальнику хора. Сынов Кореевых. Псалом.^^ (46-2) Восплещите руками все народы, воскликните Богу гласом радости;
\end{tcolorbox}
\begin{tcolorbox}
\textsubscript{2} (46-3) ибо Господь Всевышний страшен, --великий Царь над всею землею;
\end{tcolorbox}
\begin{tcolorbox}
\textsubscript{3} (46-4) покорил нам народы и племена под ноги наши;
\end{tcolorbox}
\begin{tcolorbox}
\textsubscript{4} (46-5) избрал нам наследие наше, красу Иакова, которого возлюбил.
\end{tcolorbox}
\begin{tcolorbox}
\textsubscript{5} (46-6) Восшел Бог при восклицаниях, Господь при звуке трубном.
\end{tcolorbox}
\begin{tcolorbox}
\textsubscript{6} (46-7) Пойте Богу нашему, пойте; пойте Царю нашему, пойте,
\end{tcolorbox}
\begin{tcolorbox}
\textsubscript{7} (46-8) ибо Бог--Царь всей земли; пойте все разумно.
\end{tcolorbox}
\begin{tcolorbox}
\textsubscript{8} (46-9) Бог воцарился над народами, Бог воссел на святом престоле Своем;
\end{tcolorbox}
\begin{tcolorbox}
\textsubscript{9} (46-10) Князья народов собрались к народу Бога Авраамова, ибо щиты земли--Божии; Он превознесен [над ними].
\end{tcolorbox}
\subsection{CHAPTER 48}
\begin{tcolorbox}
\textsubscript{1} (47-1) ^^Песнь. Псалом. Сынов Кореевых.^^ (47-2) Велик Господь и всехвален во граде Бога нашего, на святой горе Его.
\end{tcolorbox}
\begin{tcolorbox}
\textsubscript{2} (47-3) Прекрасная возвышенность, радость всей земли гора Сион; на северной стороне [ее] город великого Царя.
\end{tcolorbox}
\begin{tcolorbox}
\textsubscript{3} (47-4) Бог в жилищах его ведом, как заступник:
\end{tcolorbox}
\begin{tcolorbox}
\textsubscript{4} (47-5) ибо вот, сошлись цари и прошли все мимо;
\end{tcolorbox}
\begin{tcolorbox}
\textsubscript{5} (47-6) увидели и изумились, смутились и обратились в бегство;
\end{tcolorbox}
\begin{tcolorbox}
\textsubscript{6} (47-7) страх объял их там и мука, как у женщин в родах;
\end{tcolorbox}
\begin{tcolorbox}
\textsubscript{7} (47-8) восточным ветром Ты сокрушил Фарсийские корабли.
\end{tcolorbox}
\begin{tcolorbox}
\textsubscript{8} (47-9) Как слышали мы, так и увидели во граде Господа сил, во граде Бога нашего: Бог утвердит его на веки.
\end{tcolorbox}
\begin{tcolorbox}
\textsubscript{9} (47-10) Мы размышляли, Боже, о благости Твоей посреди храма Твоего.
\end{tcolorbox}
\begin{tcolorbox}
\textsubscript{10} (47-11) Как имя Твое, Боже, так и хвала Твоя до концов земли; десница Твоя полна правды.
\end{tcolorbox}
\begin{tcolorbox}
\textsubscript{11} (47-12) Да веселится гора Сион, да радуются дщери Иудейские ради судов Твоих, [Господи].
\end{tcolorbox}
\begin{tcolorbox}
\textsubscript{12} (47-13) Пойдите вокруг Сиона и обойдите его, пересчитайте башни его;
\end{tcolorbox}
\begin{tcolorbox}
\textsubscript{13} (47-14) обратите сердце ваше к укреплениям его, рассмотрите домы его, чтобы пересказать грядущему роду,
\end{tcolorbox}
\begin{tcolorbox}
\textsubscript{14} (47-15) ибо сей Бог есть Бог наш на веки и веки: Он будет вождем нашим до самой смерти.
\end{tcolorbox}
\subsection{CHAPTER 49}
\begin{tcolorbox}
\textsubscript{1} (48-1) ^^Начальнику хора. Сынов Кореевых. Псалом.^^ (48-2) Слушайте сие, все народы; внимайте сему, все живущие во вселенной, --
\end{tcolorbox}
\begin{tcolorbox}
\textsubscript{2} (48-3) и простые и знатные, богатый, равно как бедный.
\end{tcolorbox}
\begin{tcolorbox}
\textsubscript{3} (48-4) Уста мои изрекут премудрость, и размышления сердца моего--знание.
\end{tcolorbox}
\begin{tcolorbox}
\textsubscript{4} (48-5) Приклоню ухо мое к притче, на гуслях открою загадку мою:
\end{tcolorbox}
\begin{tcolorbox}
\textsubscript{5} (48-6) 'для чего бояться мне во дни бедствия, [когда] беззаконие путей моих окружит меня?'
\end{tcolorbox}
\begin{tcolorbox}
\textsubscript{6} (48-7) Надеющиеся на силы свои и хвалящиеся множеством богатства своего!
\end{tcolorbox}
\begin{tcolorbox}
\textsubscript{7} (48-8) человек никак не искупит брата своего и не даст Богу выкупа за него:
\end{tcolorbox}
\begin{tcolorbox}
\textsubscript{8} (48-9) дорога цена искупления души их, и не будет того вовек,
\end{tcolorbox}
\begin{tcolorbox}
\textsubscript{9} (48-10) чтобы остался [кто] жить навсегда и не увидел могилы.
\end{tcolorbox}
\begin{tcolorbox}
\textsubscript{10} (48-11) Каждый видит, что и мудрые умирают, равно как и невежды и бессмысленные погибают и оставляют имущество свое другим.
\end{tcolorbox}
\begin{tcolorbox}
\textsubscript{11} (48-12) В мыслях у них, что домы их вечны, и что жилища их в род и род, и земли свои они называют своими именами.
\end{tcolorbox}
\begin{tcolorbox}
\textsubscript{12} (48-13) Но человек в чести не пребудет; он уподобится животным, которые погибают.
\end{tcolorbox}
\begin{tcolorbox}
\textsubscript{13} (48-14) Этот путь их есть безумие их, хотя последующие за ними одобряют мнение их.
\end{tcolorbox}
\begin{tcolorbox}
\textsubscript{14} (48-15) Как овец, заключат их в преисподнюю; смерть будет пасти их, и наутро праведники будут владычествовать над ними; сила их истощится; могила--жилище их.
\end{tcolorbox}
\begin{tcolorbox}
\textsubscript{15} (48-16) Но Бог избавит душу мою от власти преисподней, когда примет меня.
\end{tcolorbox}
\begin{tcolorbox}
\textsubscript{16} (48-17) Не бойся, когда богатеет человек, когда слава дома его умножается:
\end{tcolorbox}
\begin{tcolorbox}
\textsubscript{17} (48-18) ибо умирая не возьмет ничего; не пойдет за ним слава его;
\end{tcolorbox}
\begin{tcolorbox}
\textsubscript{18} (48-19) хотя при жизни он ублажает душу свою, и прославляют тебя, что ты удовлетворяешь себе,
\end{tcolorbox}
\begin{tcolorbox}
\textsubscript{19} (48-20) но он пойдет к роду отцов своих, которые никогда не увидят света.
\end{tcolorbox}
\begin{tcolorbox}
\textsubscript{20} (48-21) Человек, который в чести и неразумен, подобен животным, которые погибают.
\end{tcolorbox}
\subsection{CHAPTER 50}
\begin{tcolorbox}
\textsubscript{1} (49-1) ^^Псалом Асафа.^^ Бог Богов, Господь возглаголал и призывает землю, от восхода солнца до запада.
\end{tcolorbox}
\begin{tcolorbox}
\textsubscript{2} (49-2) С Сиона, который есть верх красоты, является Бог,
\end{tcolorbox}
\begin{tcolorbox}
\textsubscript{3} (49-3) грядет Бог наш, и не в безмолвии: пред Ним огонь поядающий, и вокруг Его сильная буря.
\end{tcolorbox}
\begin{tcolorbox}
\textsubscript{4} (49-4) Он призывает свыше небо и землю, судить народ Свой:
\end{tcolorbox}
\begin{tcolorbox}
\textsubscript{5} (49-5) 'соберите ко Мне святых Моих, вступивших в завет со Мною при жертве'.
\end{tcolorbox}
\begin{tcolorbox}
\textsubscript{6} (49-6) И небеса провозгласят правду Его, ибо судия сей есть Бог.
\end{tcolorbox}
\begin{tcolorbox}
\textsubscript{7} (49-7) 'Слушай, народ Мой, Я буду говорить; Израиль! Я буду свидетельствовать против тебя: Я Бог, твой Бог.
\end{tcolorbox}
\begin{tcolorbox}
\textsubscript{8} (49-8) Не за жертвы твои Я буду укорять тебя; всесожжения твои всегда предо Мною;
\end{tcolorbox}
\begin{tcolorbox}
\textsubscript{9} (49-9) не приму тельца из дома твоего, ни козлов из дворов твоих,
\end{tcolorbox}
\begin{tcolorbox}
\textsubscript{10} (49-10) ибо Мои все звери в лесу, и скот на тысяче гор,
\end{tcolorbox}
\begin{tcolorbox}
\textsubscript{11} (49-11) знаю всех птиц на горах, и животные на полях предо Мною.
\end{tcolorbox}
\begin{tcolorbox}
\textsubscript{12} (49-12) Если бы Я взалкал, то не сказал бы тебе, ибо Моя вселенная и все, что наполняет ее.
\end{tcolorbox}
\begin{tcolorbox}
\textsubscript{13} (49-13) Ем ли Я мясо волов и пью ли кровь козлов?
\end{tcolorbox}
\begin{tcolorbox}
\textsubscript{14} (49-14) Принеси в жертву Богу хвалу и воздай Всевышнему обеты твои,
\end{tcolorbox}
\begin{tcolorbox}
\textsubscript{15} (49-15) и призови Меня в день скорби; Я избавлю тебя, и ты прославишь Меня'.
\end{tcolorbox}
\begin{tcolorbox}
\textsubscript{16} (49-16) Грешнику же говорит Бог: 'что ты проповедуешь уставы Мои и берешь завет Мой в уста твои,
\end{tcolorbox}
\begin{tcolorbox}
\textsubscript{17} (49-17) а сам ненавидишь наставление Мое и слова Мои бросаешь за себя?
\end{tcolorbox}
\begin{tcolorbox}
\textsubscript{18} (49-18) когда видишь вора, сходишься с ним, и с прелюбодеями сообщаешься;
\end{tcolorbox}
\begin{tcolorbox}
\textsubscript{19} (49-19) уста твои открываешь на злословие, и язык твой сплетает коварство;
\end{tcolorbox}
\begin{tcolorbox}
\textsubscript{20} (49-20) сидишь и говоришь на брата твоего, на сына матери твоей клевещешь;
\end{tcolorbox}
\begin{tcolorbox}
\textsubscript{21} (49-21) ты это делал, и Я молчал; ты подумал, что Я такой же, как ты. Изобличу тебя и представлю пред глаза твои [грехи твои].
\end{tcolorbox}
\begin{tcolorbox}
\textsubscript{22} (49-22) Уразумейте это, забывающие Бога, дабы Я не восхитил, --и не будет избавляющего.
\end{tcolorbox}
\begin{tcolorbox}
\textsubscript{23} (49-23) Кто приносит в жертву хвалу, тот чтит Меня, и кто наблюдает за путем своим, тому явлю Я спасение Божие'.
\end{tcolorbox}
\subsection{CHAPTER 51}
\begin{tcolorbox}
\textsubscript{1} (50-1) ^^Начальнику хора. Псалом Давида, (50-2) Когда приходил к нему пророк Нафан, после того, как Давид вошел к Вирсавии. (50-3) Помилуй меня, Боже, по великой милости Твоей, и по множеству щедрот Твоих изгладь беззакония мои.^^
\end{tcolorbox}
\begin{tcolorbox}
\textsubscript{2} (50-4) Многократно омой меня от беззакония моего, и от греха моего очисти меня,
\end{tcolorbox}
\begin{tcolorbox}
\textsubscript{3} (50-5) ибо беззакония мои я сознаю, и грех мой всегда предо мною.
\end{tcolorbox}
\begin{tcolorbox}
\textsubscript{4} (50-6) Тебе, Тебе единому согрешил я и лукавое пред очами Твоими сделал, так что Ты праведен в приговоре Твоем и чист в суде Твоем.
\end{tcolorbox}
\begin{tcolorbox}
\textsubscript{5} (50-7) Вот, я в беззаконии зачат, и во грехе родила меня мать моя.
\end{tcolorbox}
\begin{tcolorbox}
\textsubscript{6} (50-8) Вот, Ты возлюбил истину в сердце и внутрь меня явил мне мудрость.
\end{tcolorbox}
\begin{tcolorbox}
\textsubscript{7} (50-9) Окропи меня иссопом, и буду чист; омой меня, и буду белее снега.
\end{tcolorbox}
\begin{tcolorbox}
\textsubscript{8} (50-10) Дай мне услышать радость и веселие, и возрадуются кости, Тобою сокрушенные.
\end{tcolorbox}
\begin{tcolorbox}
\textsubscript{9} (50-11) Отврати лице Твое от грехов моих и изгладь все беззакония мои.
\end{tcolorbox}
\begin{tcolorbox}
\textsubscript{10} (50-12) Сердце чистое сотвори во мне, Боже, и дух правый обнови внутри меня.
\end{tcolorbox}
\begin{tcolorbox}
\textsubscript{11} (50-13) Не отвергни меня от лица Твоего и Духа Твоего Святаго не отними от меня.
\end{tcolorbox}
\begin{tcolorbox}
\textsubscript{12} (50-14) Возврати мне радость спасения Твоего и Духом владычественным утверди меня.
\end{tcolorbox}
\begin{tcolorbox}
\textsubscript{13} (50-15) Научу беззаконных путям Твоим, и нечестивые к Тебе обратятся.
\end{tcolorbox}
\begin{tcolorbox}
\textsubscript{14} (50-16) Избавь меня от кровей, Боже, Боже спасения моего, и язык мой восхвалит правду Твою.
\end{tcolorbox}
\begin{tcolorbox}
\textsubscript{15} (50-17) Господи! отверзи уста мои, и уста мои возвестят хвалу Твою:
\end{tcolorbox}
\begin{tcolorbox}
\textsubscript{16} (50-18) ибо жертвы Ты не желаешь, --я дал бы ее; к всесожжению не благоволишь.
\end{tcolorbox}
\begin{tcolorbox}
\textsubscript{17} (50-19) Жертва Богу--дух сокрушенный; сердца сокрушенного и смиренного Ты не презришь, Боже.
\end{tcolorbox}
\begin{tcolorbox}
\textsubscript{18} (50-20) Облагодетельствуй по благоволению Твоему Сион; воздвигни стены Иерусалима:
\end{tcolorbox}
\begin{tcolorbox}
\textsubscript{19} (50-21) тогда благоугодны будут Тебе жертвы правды, возношение и всесожжение; тогда возложат на алтарь Твой тельцов.
\end{tcolorbox}
\subsection{CHAPTER 52}
\begin{tcolorbox}
\textsubscript{1} (51-1) ^^Начальнику хора. Учение Давида, (51-2) после того, как приходил Доик Идумеянин и донес Саулу и сказал ему, что Давид пришел в дом Ахимелеха. (51-3) Что хвалишься злодейством, сильный? милость Божия всегда [со мною;]^^
\end{tcolorbox}
\begin{tcolorbox}
\textsubscript{2} (51-4) гибель вымышляет язык твой; как изощренная бритва, он [у] [тебя], коварный!
\end{tcolorbox}
\begin{tcolorbox}
\textsubscript{3} (51-5) ты любишь больше зло, нежели добро, больше ложь, нежели говорить правду;
\end{tcolorbox}
\begin{tcolorbox}
\textsubscript{4} (51-6) ты любишь всякие гибельные речи, язык коварный:
\end{tcolorbox}
\begin{tcolorbox}
\textsubscript{5} (51-7) за то Бог сокрушит тебя вконец, изринет тебя и исторгнет тебя из жилища [твоего] и корень твой из земли живых.
\end{tcolorbox}
\begin{tcolorbox}
\textsubscript{6} (51-8) Увидят праведники и убоятся, посмеются над ним [и скажут]:
\end{tcolorbox}
\begin{tcolorbox}
\textsubscript{7} (51-9) 'вот человек, который не в Боге полагал крепость свою, а надеялся на множество богатства своего, укреплялся в злодействе своем'.
\end{tcolorbox}
\begin{tcolorbox}
\textsubscript{8} (51-10) А я, как зеленеющая маслина, в доме Божием, и уповаю на милость Божию во веки веков,
\end{tcolorbox}
\begin{tcolorbox}
\textsubscript{9} (51-11) вечно буду славить Тебя за то, что Ты соделал, и уповать на имя Твое, ибо оно благо пред святыми Твоими.
\end{tcolorbox}
\subsection{CHAPTER 53}
\begin{tcolorbox}
\textsubscript{1} (52-1) ^^Начальнику хора. На духовом [орудии]. Учение Давида.^^ (52-2) Сказал безумец в сердце своем: 'нет Бога'. Развратились они и совершили гнусные преступления; нет делающего добро.
\end{tcolorbox}
\begin{tcolorbox}
\textsubscript{2} (52-3) Бог с небес призрел на сынов человеческих, чтобы видеть, есть ли разумеющий, ищущий Бога.
\end{tcolorbox}
\begin{tcolorbox}
\textsubscript{3} (52-4) Все уклонились, сделались равно непотребными; нет делающего добро, нет ни одного.
\end{tcolorbox}
\begin{tcolorbox}
\textsubscript{4} (52-5) Неужели не вразумятся делающие беззаконие, съедающие народ мой, [как] едят хлеб, и не призывающие Бога?
\end{tcolorbox}
\begin{tcolorbox}
\textsubscript{5} (52-6) Там убоятся они страха, где нет страха, ибо рассыплет Бог кости ополчающихся против тебя. Ты постыдишь их, потому что Бог отверг их.
\end{tcolorbox}
\begin{tcolorbox}
\textsubscript{6} (52-7) Кто даст с Сиона спасение Израилю! Когда Бог возвратит пленение народа Своего, тогда возрадуется Иаков и возвеселится Израиль.
\end{tcolorbox}
\subsection{CHAPTER 54}
\begin{tcolorbox}
\textsubscript{1} (53-1) ^^Начальнику хора. На струнных [орудиях]. Учение Давида, (53-2) когда пришли Зифеи и сказали Саулу: 'не у нас ли скрывается Давид?' (53-3) Боже! именем Твоим спаси меня, и силою Твоею суди меня.^^
\end{tcolorbox}
\begin{tcolorbox}
\textsubscript{2} (53-4) Боже! услышь молитву мою, внемли словам уст моих,
\end{tcolorbox}
\begin{tcolorbox}
\textsubscript{3} (53-5) ибо чужие восстали на меня, и сильные ищут души моей; они не имеют Бога пред собою.
\end{tcolorbox}
\begin{tcolorbox}
\textsubscript{4} (53-6) Вот, Бог помощник мой; Господь подкрепляет душу мою.
\end{tcolorbox}
\begin{tcolorbox}
\textsubscript{5} (53-7) Он воздаст за зло врагам моим; истиною Твоею истреби их.
\end{tcolorbox}
\begin{tcolorbox}
\textsubscript{6} (53-8) Я усердно принесу Тебе жертву, прославлю имя Твое, Господи, ибо оно благо,
\end{tcolorbox}
\begin{tcolorbox}
\textsubscript{7} (53-9) ибо Ты избавил меня от всех бед, и на врагов моих смотрело око мое.
\end{tcolorbox}
\subsection{CHAPTER 55}
\begin{tcolorbox}
\textsubscript{1} (54-1) ^^Начальнику хора. На струнных [орудиях]. Учение Давида.^^ (54-2) Услышь, Боже, молитву мою и не скрывайся от моления моего;
\end{tcolorbox}
\begin{tcolorbox}
\textsubscript{2} (54-3) внемли мне и услышь меня; я стенаю в горести моей, и смущаюсь
\end{tcolorbox}
\begin{tcolorbox}
\textsubscript{3} (54-4) от голоса врага, от притеснения нечестивого, ибо они возводят на меня беззаконие и в гневе враждуют против меня.
\end{tcolorbox}
\begin{tcolorbox}
\textsubscript{4} (54-5) Сердце мое трепещет во мне, и смертные ужасы напали на меня;
\end{tcolorbox}
\begin{tcolorbox}
\textsubscript{5} (54-6) страх и трепет нашел на меня, и ужас объял меня.
\end{tcolorbox}
\begin{tcolorbox}
\textsubscript{6} (54-7) И я сказал: 'кто дал бы мне крылья, как у голубя? я улетел бы и успокоился бы;
\end{tcolorbox}
\begin{tcolorbox}
\textsubscript{7} (54-8) далеко удалился бы я, и оставался бы в пустыне;
\end{tcolorbox}
\begin{tcolorbox}
\textsubscript{8} (54-9) поспешил бы укрыться от вихря, от бури'.
\end{tcolorbox}
\begin{tcolorbox}
\textsubscript{9} (54-10) Расстрой, Господи, и раздели языки их, ибо я вижу насилие и распри в городе;
\end{tcolorbox}
\begin{tcolorbox}
\textsubscript{10} (54-11) днем и ночью ходят они кругом по стенам его; злодеяния и бедствие посреди его;
\end{tcolorbox}
\begin{tcolorbox}
\textsubscript{11} (54-12) посреди его пагуба; обман и коварство не сходят с улиц его:
\end{tcolorbox}
\begin{tcolorbox}
\textsubscript{12} (54-13) ибо не враг поносит меня, --это я перенес бы; не ненавистник мой величается надо мною, --от него я укрылся бы;
\end{tcolorbox}
\begin{tcolorbox}
\textsubscript{13} (54-14) но ты, который был для меня то же, что я, друг мой и близкий мой,
\end{tcolorbox}
\begin{tcolorbox}
\textsubscript{14} (54-15) с которым мы разделяли искренние беседы и ходили вместе в дом Божий.
\end{tcolorbox}
\begin{tcolorbox}
\textsubscript{15} (54-16) Да найдет на них смерть; да сойдут они живыми в ад, ибо злодейство в жилищах их, посреди их.
\end{tcolorbox}
\begin{tcolorbox}
\textsubscript{16} (54-17) Я же воззову к Богу, и Господь спасет меня.
\end{tcolorbox}
\begin{tcolorbox}
\textsubscript{17} (54-18) Вечером и утром и в полдень буду умолять и вопиять, и Он услышит голос мой,
\end{tcolorbox}
\begin{tcolorbox}
\textsubscript{18} (54-19) избавит в мире душу мою от восстающих на меня, ибо их много у меня;
\end{tcolorbox}
\begin{tcolorbox}
\textsubscript{19} (54-20) услышит Бог, и смирит их от века Живущий, потому что нет в них перемены; они не боятся Бога,
\end{tcolorbox}
\begin{tcolorbox}
\textsubscript{20} (54-21) простерли руки свои на тех, которые с ними в мире, нарушили союз свой;
\end{tcolorbox}
\begin{tcolorbox}
\textsubscript{21} (54-22) уста их мягче масла, а в сердце их вражда; слова их нежнее елея, но они суть обнаженные мечи.
\end{tcolorbox}
\begin{tcolorbox}
\textsubscript{22} (54-23) Возложи на Господа заботы твои, и Он поддержит тебя. Никогда не даст Он поколебаться праведнику.
\end{tcolorbox}
\begin{tcolorbox}
\textsubscript{23} (54-24) Ты, Боже, низведешь их в ров погибели; кровожадные и коварные не доживут и до половины дней своих. А я на Тебя, [Господи], уповаю.
\end{tcolorbox}
\subsection{CHAPTER 56}
\begin{tcolorbox}
\textsubscript{1} (55-1) ^^Начальнику хора. О голубице, безмолвствующей в удалении. Писание Давида, когда Филистимляне захватили его в Гефе.^^ (55-2) Помилуй меня, Боже! ибо человек хочет поглотить меня; нападая всякий день, теснит меня.
\end{tcolorbox}
\begin{tcolorbox}
\textsubscript{2} (55-3) Враги мои всякий день ищут поглотить меня, ибо много восстающих на меня, о, Всевышний!
\end{tcolorbox}
\begin{tcolorbox}
\textsubscript{3} (55-4) Когда я в страхе, на Тебя я уповаю.
\end{tcolorbox}
\begin{tcolorbox}
\textsubscript{4} (55-5) В Боге восхвалю я слово Его; на Бога уповаю, не боюсь; что сделает мне плоть?
\end{tcolorbox}
\begin{tcolorbox}
\textsubscript{5} (55-6) Всякий день извращают слова мои; все помышления их обо мне--на зло:
\end{tcolorbox}
\begin{tcolorbox}
\textsubscript{6} (55-7) собираются, притаиваются, наблюдают за моими пятами, чтобы уловить душу мою.
\end{tcolorbox}
\begin{tcolorbox}
\textsubscript{7} (55-8) Неужели они избегнут воздаяния за неправду [свою]? Во гневе низложи, Боже, народы.
\end{tcolorbox}
\begin{tcolorbox}
\textsubscript{8} (55-9) У Тебя исчислены мои скитания; положи слезы мои в сосуд у Тебя, --не в книге ли они Твоей?
\end{tcolorbox}
\begin{tcolorbox}
\textsubscript{9} (55-10) Враги мои обращаются назад, когда я взываю к Тебе, из этого я узнаю, что Бог за меня.
\end{tcolorbox}
\begin{tcolorbox}
\textsubscript{10} (55-11) В Боге восхвалю я слово [Его], в Господе восхвалю слово [Его].
\end{tcolorbox}
\begin{tcolorbox}
\textsubscript{11} (55-12) На Бога уповаю, не боюсь; что сделает мне человек?
\end{tcolorbox}
\begin{tcolorbox}
\textsubscript{12} (55-13) На мне, Боже, обеты Тебе; Тебе воздам хвалы,
\end{tcolorbox}
\begin{tcolorbox}
\textsubscript{13} (55-14) ибо Ты избавил душу мою от смерти, да и ноги мои от преткновения, чтобы я ходил пред лицем Божиим во свете живых.
\end{tcolorbox}
\subsection{CHAPTER 57}
\begin{tcolorbox}
\textsubscript{1} (56-1) ^^Начальнику хора. Не погуби. Писание Давида, когда он убежал от Саула в пещеру.^^ (56-2) Помилуй меня, Боже, помилуй меня, ибо на Тебя уповает душа моя, и в тени крыл Твоих я укроюсь, доколе не пройдут беды.
\end{tcolorbox}
\begin{tcolorbox}
\textsubscript{2} (56-3) Воззову к Богу Всевышнему, Богу, благодетельствующему мне;
\end{tcolorbox}
\begin{tcolorbox}
\textsubscript{3} (56-4) Он пошлет с небес и спасет меня; посрамит ищущего поглотить меня; пошлет Бог милость Свою и истину Свою.
\end{tcolorbox}
\begin{tcolorbox}
\textsubscript{4} (56-5) Душа моя среди львов; я лежу среди дышущих пламенем, среди сынов человеческих, у которых зубы--копья и стрелы, и у которых язык--острый меч.
\end{tcolorbox}
\begin{tcolorbox}
\textsubscript{5} (56-6) Будь превознесен выше небес, Боже, и над всею землею да будет слава Твоя!
\end{tcolorbox}
\begin{tcolorbox}
\textsubscript{6} (56-7) Приготовили сеть ногам моим; душа моя поникла; выкопали предо мною яму, и [сами] упали в нее.
\end{tcolorbox}
\begin{tcolorbox}
\textsubscript{7} (56-8) Готово сердце мое, Боже, готово сердце мое: буду петь и славить.
\end{tcolorbox}
\begin{tcolorbox}
\textsubscript{8} (56-9) Воспрянь, слава моя, воспрянь, псалтирь и гусли! Я встану рано.
\end{tcolorbox}
\begin{tcolorbox}
\textsubscript{9} (56-10) Буду славить Тебя, Господи, между народами; буду воспевать Тебя среди племен,
\end{tcolorbox}
\begin{tcolorbox}
\textsubscript{10} (56-11) ибо до небес велика милость Твоя и до облаков истина Твоя.
\end{tcolorbox}
\begin{tcolorbox}
\textsubscript{11} (56-12) Будь превознесен выше небес, Боже, и над всею землею да будет слава Твоя!
\end{tcolorbox}
\subsection{CHAPTER 58}
\begin{tcolorbox}
\textsubscript{1} (57-1) ^^Начальнику хора. Не погуби. Писание Давида.^^ (57-2) Подлинно ли правду говорите вы, судьи, и справедливо судите, сыны человеческие?
\end{tcolorbox}
\begin{tcolorbox}
\textsubscript{2} (57-3) Беззаконие составляете в сердце, кладете на весы злодеяния рук ваших на земле.
\end{tcolorbox}
\begin{tcolorbox}
\textsubscript{3} (57-4) С самого рождения отступили нечестивые, от утробы [матери] заблуждаются, говоря ложь.
\end{tcolorbox}
\begin{tcolorbox}
\textsubscript{4} (57-5) Яд у них--как яд змеи, как глухого аспида, который затыкает уши свои
\end{tcolorbox}
\begin{tcolorbox}
\textsubscript{5} (57-6) и не слышит голоса заклинателя, самого искусного в заклинаниях.
\end{tcolorbox}
\begin{tcolorbox}
\textsubscript{6} (57-7) Боже! сокруши зубы их в устах их; разбей, Господи, челюсти львов!
\end{tcolorbox}
\begin{tcolorbox}
\textsubscript{7} (57-8) Да исчезнут, как вода протекающая; когда напрягут стрелы, пусть они будут как переломленные.
\end{tcolorbox}
\begin{tcolorbox}
\textsubscript{8} (57-9) Да исчезнут, как распускающаяся улитка; да не видят солнца, как выкидыш женщины.
\end{tcolorbox}
\begin{tcolorbox}
\textsubscript{9} (57-10) Прежде нежели котлы ваши ощутят горящий терн, и свежее и обгоревшее да разнесет вихрь.
\end{tcolorbox}
\begin{tcolorbox}
\textsubscript{10} (57-11) Возрадуется праведник, когда увидит отмщение; омоет стопы свои в крови нечестивого.
\end{tcolorbox}
\begin{tcolorbox}
\textsubscript{11} (57-12) И скажет человек: 'подлинно есть плод праведнику! итак есть Бог, судящий на земле!'
\end{tcolorbox}
\subsection{CHAPTER 59}
\begin{tcolorbox}
\textsubscript{1} (58-1) ^^Начальнику хора. Не погуби. Писание Давида, когда Саул послал стеречь дом его, чтобы умертвить его.^^ (58-2) Избавь меня от врагов моих, Боже мой! защити меня от восстающих на меня;
\end{tcolorbox}
\begin{tcolorbox}
\textsubscript{2} (58-3) избавь меня от делающих беззаконие; спаси от кровожадных,
\end{tcolorbox}
\begin{tcolorbox}
\textsubscript{3} (58-4) ибо вот, они подстерегают душу мою; собираются на меня сильные не за преступление мое и не за грех мой, Господи;
\end{tcolorbox}
\begin{tcolorbox}
\textsubscript{4} (58-5) без вины [моей] сбегаются и вооружаются; подвигнись на помощь мне и воззри.
\end{tcolorbox}
\begin{tcolorbox}
\textsubscript{5} (58-6) Ты, Господи, Боже сил, Боже Израилев, восстань посетить все народы, не пощади ни одного из нечестивых беззаконников:
\end{tcolorbox}
\begin{tcolorbox}
\textsubscript{6} (58-7) вечером возвращаются они, воют, как псы, и ходят вокруг города;
\end{tcolorbox}
\begin{tcolorbox}
\textsubscript{7} (58-8) вот они изрыгают хулу языком своим; в устах их мечи: 'ибо', [думают они], 'кто слышит?'
\end{tcolorbox}
\begin{tcolorbox}
\textsubscript{8} (58-9) Но Ты, Господи, посмеешься над ними; Ты посрамишь все народы.
\end{tcolorbox}
\begin{tcolorbox}
\textsubscript{9} (58-10) Сила--у них, но я к Тебе прибегаю, ибо Бог--заступник мой.
\end{tcolorbox}
\begin{tcolorbox}
\textsubscript{10} (58-11) Бой мой, милующий меня, предварит меня; Бог даст мне смотреть на врагов моих.
\end{tcolorbox}
\begin{tcolorbox}
\textsubscript{11} (58-12) Не умерщвляй их, чтобы не забыл народ мой; расточи их силою Твоею и низложи их, Господи, защитник наш.
\end{tcolorbox}
\begin{tcolorbox}
\textsubscript{12} (58-13) Слово языка их есть грех уст их, да уловятся они в гордости своей за клятву и ложь, которую произносят.
\end{tcolorbox}
\begin{tcolorbox}
\textsubscript{13} (58-14) Расточи их во гневе, расточи, чтобы их не было; и да познают, что Бог владычествует над Иаковом до пределов земли.
\end{tcolorbox}
\begin{tcolorbox}
\textsubscript{14} (58-15) Пусть возвращаются вечером, воют, как псы, и ходят вокруг города;
\end{tcolorbox}
\begin{tcolorbox}
\textsubscript{15} (58-16) пусть бродят, чтобы найти пищу, и несытые проводят ночи.
\end{tcolorbox}
\begin{tcolorbox}
\textsubscript{16} (58-17) А я буду воспевать силу Твою и с раннего утра провозглашать милость Твою, ибо Ты был мне защитою и убежищем в день бедствия моего.
\end{tcolorbox}
\begin{tcolorbox}
\textsubscript{17} (58-18) Сила моя! Тебя буду воспевать я, ибо Бог--заступник мой, Бог мой, милующий меня.
\end{tcolorbox}
\subsection{CHAPTER 60}
\begin{tcolorbox}
\textsubscript{1} (59-1) ^^Начальнику хора. На [музыкальном орудии] Шушан-Эдуф. Писание Давида для изучения, (59-2) когда он воевал с Сириею Месопотамскою и с Сириею Цованскою, и когда Иоав, возвращаясь, поразил двенадцать тысяч Идумеев в долине Соляной. (59-3) Боже! Ты отринул нас, Ты сокрушил нас, Ты прогневался: обратись к нам.^^
\end{tcolorbox}
\begin{tcolorbox}
\textsubscript{2} (59-4) Ты потряс землю, разбил ее: исцели повреждения ее, ибо она колеблется.
\end{tcolorbox}
\begin{tcolorbox}
\textsubscript{3} (59-5) Ты дал испытать народу твоему жестокое, напоил нас вином изумления.
\end{tcolorbox}
\begin{tcolorbox}
\textsubscript{4} (59-6) Даруй боящимся Тебя знамя, чтобы они подняли его ради истины,
\end{tcolorbox}
\begin{tcolorbox}
\textsubscript{5} (59-7) чтобы избавились возлюбленные Твои; спаси десницею Твоею и услышь меня.
\end{tcolorbox}
\begin{tcolorbox}
\textsubscript{6} (59-8) Бог сказал во святилище Своем: 'восторжествую, разделю Сихем и долину Сокхоф размерю:
\end{tcolorbox}
\begin{tcolorbox}
\textsubscript{7} (59-9) Мой Галаад, Мой Манассия, Ефрем крепость главы Моей, Иуда скипетр Мой,
\end{tcolorbox}
\begin{tcolorbox}
\textsubscript{8} (59-10) Моав умывальная чаша Моя; на Едома простру сапог Мой. Восклицай Мне, земля Филистимская!'
\end{tcolorbox}
\begin{tcolorbox}
\textsubscript{9} (59-11) Кто введет меня в укрепленный город? Кто доведет меня до Едома?
\end{tcolorbox}
\begin{tcolorbox}
\textsubscript{10} (59-12) Не Ты ли, Боже, [Который] отринул нас, и не выходишь, Боже, с войсками нашими?
\end{tcolorbox}
\begin{tcolorbox}
\textsubscript{11} (59-13) Подай нам помощь в тесноте, ибо защита человеческая суетна.
\end{tcolorbox}
\begin{tcolorbox}
\textsubscript{12} (59-14) С Богом мы окажем силу, Он низложит врагов наших.
\end{tcolorbox}
\subsection{CHAPTER 61}
\begin{tcolorbox}
\textsubscript{1} (60-1) ^^Начальнику хора. На струнном [орудии]. Псалом Давида.^^ (60-2) Услышь, Боже, вопль мой, внемли молитве моей!
\end{tcolorbox}
\begin{tcolorbox}
\textsubscript{2} (60-3) От конца земли взываю к Тебе в унынии сердца моего; возведи меня на скалу, для меня недосягаемую,
\end{tcolorbox}
\begin{tcolorbox}
\textsubscript{3} (60-4) ибо Ты прибежище мое, Ты крепкая защита от врага.
\end{tcolorbox}
\begin{tcolorbox}
\textsubscript{4} (60-5) Да живу я вечно в жилище Твоем и покоюсь под кровом крыл Твоих,
\end{tcolorbox}
\begin{tcolorbox}
\textsubscript{5} (60-6) ибо Ты, Боже, услышал обеты мои и дал [мне] наследие боящихся имени Твоего.
\end{tcolorbox}
\begin{tcolorbox}
\textsubscript{6} (60-7) Приложи дни ко дням царя, лета его [продли] в род и род,
\end{tcolorbox}
\begin{tcolorbox}
\textsubscript{7} (60-8) да пребудет он вечно пред Богом; заповедуй милости и истине охранять его.
\end{tcolorbox}
\begin{tcolorbox}
\textsubscript{8} (60-9) И я буду петь имени Твоему вовек, исполняя обеты мои всякий день.
\end{tcolorbox}
\subsection{CHAPTER 62}
\begin{tcolorbox}
\textsubscript{1} (61-1) ^^Начальнику хора Идифумова. Псалом Давида.^^ (61-2) Только в Боге успокаивается душа моя: от Него спасение мое.
\end{tcolorbox}
\begin{tcolorbox}
\textsubscript{2} (61-3) Только Он--твердыня моя, спасение мое, убежище мое: не поколеблюсь более.
\end{tcolorbox}
\begin{tcolorbox}
\textsubscript{3} (61-4) Доколе вы будете налегать на человека? Вы будете низринуты, все вы, как наклонившаяся стена, как ограда пошатнувшаяся.
\end{tcolorbox}
\begin{tcolorbox}
\textsubscript{4} (61-5) Они задумали свергнуть его с высоты, прибегли ко лжи; устами благословляют, а в сердце своем клянут.
\end{tcolorbox}
\begin{tcolorbox}
\textsubscript{5} (61-6) Только в Боге успокаивайся, душа моя! ибо на Него надежда моя.
\end{tcolorbox}
\begin{tcolorbox}
\textsubscript{6} (61-7) Только Он--твердыня моя и спасение мое, убежище мое: не поколеблюсь.
\end{tcolorbox}
\begin{tcolorbox}
\textsubscript{7} (61-8) В Боге спасение мое и слава моя; крепость силы моей и упование мое в Боге.
\end{tcolorbox}
\begin{tcolorbox}
\textsubscript{8} (61-9) Народ! надейтесь на Него во всякое время; изливайте пред Ним сердце ваше: Бог нам прибежище.
\end{tcolorbox}
\begin{tcolorbox}
\textsubscript{9} (61-10) Сыны человеческие--только суета; сыны мужей--ложь; если положить их на весы, все они вместе легче пустоты.
\end{tcolorbox}
\begin{tcolorbox}
\textsubscript{10} (61-11) Не надейтесь на грабительство и не тщеславьтесь хищением; когда богатство умножается, не прилагайте [к нему] сердца.
\end{tcolorbox}
\begin{tcolorbox}
\textsubscript{11} (61-12) Однажды сказал Бог, и дважды слышал я это, что сила у Бога,
\end{tcolorbox}
\begin{tcolorbox}
\textsubscript{12} (61-13) и у Тебя, Господи, милость, ибо Ты воздаешь каждому по делам его.
\end{tcolorbox}
\subsection{CHAPTER 63}
\begin{tcolorbox}
\textsubscript{1} (62-1) ^^Псалом Давида, когда он был в пустыне Иудейской.^^ (62-2) Боже! Ты Бог мой, Тебя от ранней зари ищу я; Тебя жаждет душа моя, по Тебе томится плоть моя в земле пустой, иссохшей и безводной,
\end{tcolorbox}
\begin{tcolorbox}
\textsubscript{2} (62-3) чтобы видеть силу Твою и славу Твою, как я видел Тебя во святилище:
\end{tcolorbox}
\begin{tcolorbox}
\textsubscript{3} (62-4) ибо милость Твоя лучше, нежели жизнь. Уста мои восхвалят Тебя.
\end{tcolorbox}
\begin{tcolorbox}
\textsubscript{4} (62-5) Так благословлю Тебя в жизни моей; во имя Твое вознесу руки мои.
\end{tcolorbox}
\begin{tcolorbox}
\textsubscript{5} (62-6) Как туком и елеем насыщается душа моя, и радостным гласом восхваляют Тебя уста мои,
\end{tcolorbox}
\begin{tcolorbox}
\textsubscript{6} (62-7) когда я вспоминаю о Тебе на постели моей, размышляю о Тебе в [ночные] стражи,
\end{tcolorbox}
\begin{tcolorbox}
\textsubscript{7} (62-8) ибо Ты помощь моя, и в тени крыл Твоих я возрадуюсь;
\end{tcolorbox}
\begin{tcolorbox}
\textsubscript{8} (62-9) к Тебе прилепилась душа моя; десница Твоя поддерживает меня.
\end{tcolorbox}
\begin{tcolorbox}
\textsubscript{9} (62-10) А те, которые ищут погибели душе моей, сойдут в преисподнюю земли;
\end{tcolorbox}
\begin{tcolorbox}
\textsubscript{10} (62-11) Сразят их силою меча; достанутся они в добычу лисицам.
\end{tcolorbox}
\begin{tcolorbox}
\textsubscript{11} (62-12) Царь же возвеселится о Боге, восхвален будет всякий, клянущийся Им, ибо заградятся уста говорящих неправду.
\end{tcolorbox}
\subsection{CHAPTER 64}
\begin{tcolorbox}
\textsubscript{1} (63-1) ^^Начальнику хора. Псалом Давида.^^ (63-2) Услышь, Боже, голос мой в молитве моей, сохрани жизнь мою от страха врага;
\end{tcolorbox}
\begin{tcolorbox}
\textsubscript{2} (63-3) укрой меня от замысла коварных, от мятежа злодеев,
\end{tcolorbox}
\begin{tcolorbox}
\textsubscript{3} (63-4) которые изострили язык свой, как меч; напрягли лук свой--язвительное слово,
\end{tcolorbox}
\begin{tcolorbox}
\textsubscript{4} (63-5) чтобы втайне стрелять в непорочного; они внезапно стреляют в него и не боятся.
\end{tcolorbox}
\begin{tcolorbox}
\textsubscript{5} (63-6) Они утвердились в злом намерении, совещались скрыть сеть, говорили: кто их увидит?
\end{tcolorbox}
\begin{tcolorbox}
\textsubscript{6} (63-7) Изыскивают неправду, делают расследование за расследованием даже до внутренней жизни человека и до глубины сердца.
\end{tcolorbox}
\begin{tcolorbox}
\textsubscript{7} (63-8) Но поразит их Бог стрелою: внезапно будут они уязвлены;
\end{tcolorbox}
\begin{tcolorbox}
\textsubscript{8} (63-9) языком своим они поразят самих себя; все, видящие их, удалятся [от них].
\end{tcolorbox}
\begin{tcolorbox}
\textsubscript{9} (63-10) И убоятся все человеки, и возвестят дело Божие, и уразумеют, что это Его дело.
\end{tcolorbox}
\begin{tcolorbox}
\textsubscript{10} (63-11) А праведник возвеселится о Господе и будет уповать на Него; и похвалятся все правые сердцем.
\end{tcolorbox}
\subsection{CHAPTER 65}
\begin{tcolorbox}
\textsubscript{1} (64-1) ^^Начальнику хора. Псалом Давида для пения.^^ (64-2) Тебе, Боже, принадлежит хвала на Сионе, и Тебе воздастся обет [в Иерусалиме].
\end{tcolorbox}
\begin{tcolorbox}
\textsubscript{2} (64-3) Ты слышишь молитву; к Тебе прибегает всякая плоть.
\end{tcolorbox}
\begin{tcolorbox}
\textsubscript{3} (64-4) Дела беззаконий превозмогают меня; Ты очистишь преступления наши.
\end{tcolorbox}
\begin{tcolorbox}
\textsubscript{4} (64-5) Блажен, кого Ты избрал и приблизил, чтобы он жил во дворах Твоих. Насытимся благами дома Твоего, святаго храма Твоего.
\end{tcolorbox}
\begin{tcolorbox}
\textsubscript{5} (64-6) Страшный в правосудии, услышь нас, Боже, Спаситель наш, упование всех концов земли и находящихся в море далеко,
\end{tcolorbox}
\begin{tcolorbox}
\textsubscript{6} (64-7) поставивший горы силою Своею, препоясанный могуществом,
\end{tcolorbox}
\begin{tcolorbox}
\textsubscript{7} (64-8) укрощающий шум морей, шум волн их и мятеж народов!
\end{tcolorbox}
\begin{tcolorbox}
\textsubscript{8} (64-9) И убоятся знамений Твоих живущие на пределах [земли]. Утро и вечер возбудишь к славе [Твоей].
\end{tcolorbox}
\begin{tcolorbox}
\textsubscript{9} (64-10) Ты посещаешь землю и утоляешь жажду ее, обильно обогащаешь ее: поток Божий полон воды; Ты приготовляешь хлеб, ибо так устроил ее;
\end{tcolorbox}
\begin{tcolorbox}
\textsubscript{10} (64-11) напояешь борозды ее, уравниваешь глыбы ее, размягчаешь ее каплями дождя, благословляешь произрастания ее;
\end{tcolorbox}
\begin{tcolorbox}
\textsubscript{11} (64-12) венчаешь лето благости Твоей, и стези Твои источают тук,
\end{tcolorbox}
\begin{tcolorbox}
\textsubscript{12} (64-13) источают на пустынные пажити, и холмы препоясываются радостью;
\end{tcolorbox}
\begin{tcolorbox}
\textsubscript{13} (64-14) луга одеваются стадами, и долины покрываются хлебом, восклицают и поют.
\end{tcolorbox}
\subsection{CHAPTER 66}
\begin{tcolorbox}
\textsubscript{1} (65-1) ^^Начальнику хора. Песнь.^^ Воскликните Богу, вся земля.
\end{tcolorbox}
\begin{tcolorbox}
\textsubscript{2} (65-2) Пойте славу имени Его, воздайте славу, хвалу Ему.
\end{tcolorbox}
\begin{tcolorbox}
\textsubscript{3} (65-3) Скажите Богу: как страшен Ты в делах Твоих! По множеству силы Твоей, покорятся Тебе враги Твои.
\end{tcolorbox}
\begin{tcolorbox}
\textsubscript{4} (65-4) Вся земля да поклонится Тебе и поет Тебе, да поет имени Твоему.
\end{tcolorbox}
\begin{tcolorbox}
\textsubscript{5} (65-5) Придите и воззрите на дела Бога, страшного в делах над сынами человеческими.
\end{tcolorbox}
\begin{tcolorbox}
\textsubscript{6} (65-6) Он превратил море в сушу; через реку перешли стопами, там веселились мы о Нем.
\end{tcolorbox}
\begin{tcolorbox}
\textsubscript{7} (65-7) Могуществом Своим владычествует Он вечно; очи Его зрят на народы, да не возносятся мятежники.
\end{tcolorbox}
\begin{tcolorbox}
\textsubscript{8} (65-8) Благословите, народы, Бога нашего и провозгласите хвалу Ему.
\end{tcolorbox}
\begin{tcolorbox}
\textsubscript{9} (65-9) Он сохранил душе нашей жизнь и ноге нашей не дал поколебаться.
\end{tcolorbox}
\begin{tcolorbox}
\textsubscript{10} (65-10) Ты испытал нас, Боже, переплавил нас, как переплавляют серебро.
\end{tcolorbox}
\begin{tcolorbox}
\textsubscript{11} (65-11) Ты ввел нас в сеть, положил оковы на чресла наши,
\end{tcolorbox}
\begin{tcolorbox}
\textsubscript{12} (65-12) посадил человека на главу нашу. Мы вошли в огонь и в воду, и Ты вывел нас на свободу.
\end{tcolorbox}
\begin{tcolorbox}
\textsubscript{13} (65-13) Войду в дом Твой со всесожжениями, воздам Тебе обеты мои,
\end{tcolorbox}
\begin{tcolorbox}
\textsubscript{14} (65-14) которые произнесли уста мои и изрек язык мой в скорби моей.
\end{tcolorbox}
\begin{tcolorbox}
\textsubscript{15} (65-15) Всесожжения тучные вознесу Тебе с воскурением тука овнов, принесу в жертву волов и козлов.
\end{tcolorbox}
\begin{tcolorbox}
\textsubscript{16} (65-16) Придите, послушайте, все боящиеся Бога, и я возвещу [вам], что сотворил Он для души моей.
\end{tcolorbox}
\begin{tcolorbox}
\textsubscript{17} (65-17) Я воззвал к Нему устами моими и превознес Его языком моим.
\end{tcolorbox}
\begin{tcolorbox}
\textsubscript{18} (65-18) Если бы я видел беззаконие в сердце моем, то не услышал бы меня Господь.
\end{tcolorbox}
\begin{tcolorbox}
\textsubscript{19} (65-19) Но Бог услышал, внял гласу моления моего.
\end{tcolorbox}
\begin{tcolorbox}
\textsubscript{20} (65-20) Благословен Бог, Который не отверг молитвы моей и не отвратил от меня милости Своей.
\end{tcolorbox}
\subsection{CHAPTER 67}
\begin{tcolorbox}
\textsubscript{1} (66-1) ^^Начальнику хора. На струнных [орудиях]. Псалом. Песнь.^^ (66-2) Боже! будь милостив к нам и благослови нас, освети нас лицем Твоим,
\end{tcolorbox}
\begin{tcolorbox}
\textsubscript{2} (66-3) дабы познали на земле путь Твой, во всех народах спасение Твое.
\end{tcolorbox}
\begin{tcolorbox}
\textsubscript{3} (66-4) Да восхвалят Тебя народы, Боже; да восхвалят Тебя народы все.
\end{tcolorbox}
\begin{tcolorbox}
\textsubscript{4} (66-5) Да веселятся и радуются племена, ибо Ты судишь народы праведно и управляешь на земле племенами.
\end{tcolorbox}
\begin{tcolorbox}
\textsubscript{5} (66-6) Да восхвалят Тебя народы, Боже, да восхвалят Тебя народы все.
\end{tcolorbox}
\begin{tcolorbox}
\textsubscript{6} (66-7) Земля дала плод свой; да благословит нас Бог, Бог наш.
\end{tcolorbox}
\begin{tcolorbox}
\textsubscript{7} (66-8) Да благословит нас Бог, и да убоятся Его все пределы земли.
\end{tcolorbox}
\subsection{CHAPTER 68}
\begin{tcolorbox}
\textsubscript{1} (67-1) ^^Начальнику хора. Псалом Давида. Песнь.^^ (67-2) Да восстанет Бог, и расточатся враги Его, и да бегут от лица Его ненавидящие Его.
\end{tcolorbox}
\begin{tcolorbox}
\textsubscript{2} (67-3) Как рассеивается дым, Ты рассей их; как тает воск от огня, так нечестивые да погибнут от лица Божия.
\end{tcolorbox}
\begin{tcolorbox}
\textsubscript{3} (67-4) А праведники да возвеселятся, да возрадуются пред Богом и восторжествуют в радости.
\end{tcolorbox}
\begin{tcolorbox}
\textsubscript{4} (67-5) Пойте Богу нашему, пойте имени Его, превозносите Шествующего на небесах; имя Ему: Господь, и радуйтесь пред лицем Его.
\end{tcolorbox}
\begin{tcolorbox}
\textsubscript{5} (67-6) Отец сирот и судья вдов Бог во святом Своем жилище.
\end{tcolorbox}
\begin{tcolorbox}
\textsubscript{6} (67-7) Бог одиноких вводит в дом, освобождает узников от оков, а непокорные остаются в знойной пустыне.
\end{tcolorbox}
\begin{tcolorbox}
\textsubscript{7} (67-8) Боже! когда Ты выходил пред народом Твоим, когда Ты шествовал пустынею,
\end{tcolorbox}
\begin{tcolorbox}
\textsubscript{8} (67-9) земля тряслась, даже небеса таяли от лица Божия, и этот Синай--от лица Бога, Бога Израилева.
\end{tcolorbox}
\begin{tcolorbox}
\textsubscript{9} (67-10) Обильный дождь проливал Ты, Боже, на наследие Твое, и когда оно изнемогало от труда, Ты подкреплял его.
\end{tcolorbox}
\begin{tcolorbox}
\textsubscript{10} (67-11) Народ Твой обитал там; по благости Твоей, Боже, Ты готовил [необходимое] для бедного.
\end{tcolorbox}
\begin{tcolorbox}
\textsubscript{11} (67-12) Господь даст слово: провозвестниц великое множество.
\end{tcolorbox}
\begin{tcolorbox}
\textsubscript{12} (67-13) Цари воинств бегут, бегут, а сидящая дома делит добычу.
\end{tcolorbox}
\begin{tcolorbox}
\textsubscript{13} (67-14) Расположившись в уделах [своих], вы стали, как голубица, которой крылья покрыты серебром, а перья чистым золотом:
\end{tcolorbox}
\begin{tcolorbox}
\textsubscript{14} (67-15) когда Всемогущий рассеял царей на сей [земле], она забелела, как снег на Селмоне.
\end{tcolorbox}
\begin{tcolorbox}
\textsubscript{15} (67-16) Гора Божия--гора Васанская! гора высокая--гора Васанская!
\end{tcolorbox}
\begin{tcolorbox}
\textsubscript{16} (67-17) что вы завистливо смотрите, горы высокие, на гору, на которой Бог благоволит обитать и будет Господь обитать вечно?
\end{tcolorbox}
\begin{tcolorbox}
\textsubscript{17} (67-18) Колесниц Божиих тьмы, тысячи тысяч; среди их Господь на Синае, во святилище.
\end{tcolorbox}
\begin{tcolorbox}
\textsubscript{18} (67-19) Ты восшел на высоту, пленил плен, принял дары для человеков, так чтоб и из противящихся могли обитать у Господа Бога.
\end{tcolorbox}
\begin{tcolorbox}
\textsubscript{19} (67-20) Благословен Господь всякий день. Бог возлагает на нас бремя, но Он же и спасает нас.
\end{tcolorbox}
\begin{tcolorbox}
\textsubscript{20} (67-21) Бог для нас--Бог во спасение; во власти Господа Вседержителя врата смерти.
\end{tcolorbox}
\begin{tcolorbox}
\textsubscript{21} (67-22) Но Бог сокрушит голову врагов Своих, волосатое темя закоснелого в своих беззакониях.
\end{tcolorbox}
\begin{tcolorbox}
\textsubscript{22} (67-23) Господь сказал: 'от Васана возвращу, выведу из глубины морской,
\end{tcolorbox}
\begin{tcolorbox}
\textsubscript{23} (67-24) чтобы ты погрузил ногу твою, как и псы твои язык свой, в крови врагов'.
\end{tcolorbox}
\begin{tcolorbox}
\textsubscript{24} (67-25) Видели шествие Твое, Боже, шествие Бога моего, Царя моего во святыне:
\end{tcolorbox}
\begin{tcolorbox}
\textsubscript{25} (67-26) впереди шли поющие, позади играющие на орудиях, в средине девы с тимпанами:
\end{tcolorbox}
\begin{tcolorbox}
\textsubscript{26} (67-27) 'в собраниях благословите [Бога Господа], вы--от семени Израилева!'
\end{tcolorbox}
\begin{tcolorbox}
\textsubscript{27} (67-28) Там Вениамин младший--князь их; князья Иудины--владыки их, князья Завулоновы, князья Неффалимовы.
\end{tcolorbox}
\begin{tcolorbox}
\textsubscript{28} (67-29) Бог твой предназначил тебе силу. Утверди, Боже, то, что Ты соделал для нас!
\end{tcolorbox}
\begin{tcolorbox}
\textsubscript{29} (67-30) Ради храма Твоего в Иерусалиме цари принесут Тебе дары.
\end{tcolorbox}
\begin{tcolorbox}
\textsubscript{30} (67-31) Укроти зверя в тростнике, стадо волов среди тельцов народов, хвалящихся слитками серебра; рассыпь народы, желающие браней.
\end{tcolorbox}
\begin{tcolorbox}
\textsubscript{31} (67-32) Придут вельможи из Египта; Ефиопия прострет руки свои к Богу.
\end{tcolorbox}
\begin{tcolorbox}
\textsubscript{32} (67-33) Царства земные! пойте Богу, воспевайте Господа,
\end{tcolorbox}
\begin{tcolorbox}
\textsubscript{33} (67-34) шествующего на небесах небес от века. Вот, Он дает гласу Своему глас силы.
\end{tcolorbox}
\begin{tcolorbox}
\textsubscript{34} (67-35) Воздайте славу Богу! величие Его--над Израилем, и могущество Его--на облаках.
\end{tcolorbox}
\begin{tcolorbox}
\textsubscript{35} (67-36) Страшен Ты, Боже, во святилище Твоем. Бог Израилев--Он дает силу и крепость народу [Своему]. Благословен Бог!
\end{tcolorbox}
\subsection{CHAPTER 69}
\begin{tcolorbox}
\textsubscript{1} (68-1) ^^Начальнику хора. На Шошанниме. Псалом Давида.^^ (68-2) Спаси меня, Боже, ибо воды дошли до души [моей].
\end{tcolorbox}
\begin{tcolorbox}
\textsubscript{2} (68-3) Я погряз в глубоком болоте, и не на чем стать; вошел во глубину вод, и быстрое течение их увлекает меня.
\end{tcolorbox}
\begin{tcolorbox}
\textsubscript{3} (68-4) Я изнемог от вопля, засохла гортань моя, истомились глаза мои от ожидания Бога [моего].
\end{tcolorbox}
\begin{tcolorbox}
\textsubscript{4} (68-5) Ненавидящих меня без вины больше, нежели волос на голове моей; враги мои, преследующие меня несправедливо, усилились; чего я не отнимал, то должен отдать.
\end{tcolorbox}
\begin{tcolorbox}
\textsubscript{5} (68-6) Боже! Ты знаешь безумие мое, и грехи мои не сокрыты от Тебя.
\end{tcolorbox}
\begin{tcolorbox}
\textsubscript{6} (68-7) Да не постыдятся во мне все, надеющиеся на Тебя, Господи, Боже сил. Да не посрамятся во мне ищущие Тебя, Боже Израилев,
\end{tcolorbox}
\begin{tcolorbox}
\textsubscript{7} (68-8) ибо ради Тебя несу я поношение, и бесчестием покрывают лице мое.
\end{tcolorbox}
\begin{tcolorbox}
\textsubscript{8} (68-9) Чужим стал я для братьев моих и посторонним для сынов матери моей,
\end{tcolorbox}
\begin{tcolorbox}
\textsubscript{9} (68-10) ибо ревность по доме Твоем снедает меня, и злословия злословящих Тебя падают на меня;
\end{tcolorbox}
\begin{tcolorbox}
\textsubscript{10} (68-11) и плачу, постясь душею моею, и это ставят в поношение мне;
\end{tcolorbox}
\begin{tcolorbox}
\textsubscript{11} (68-12) и возлагаю на себя вместо одежды вретище, --и делаюсь для них притчею;
\end{tcolorbox}
\begin{tcolorbox}
\textsubscript{12} (68-13) о мне толкуют сидящие у ворот, и поют в песнях пьющие вино.
\end{tcolorbox}
\begin{tcolorbox}
\textsubscript{13} (68-14) А я с молитвою моею к Тебе, Господи; во время благоугодное, Боже, по великой благости Твоей услышь меня в истине спасения Твоего;
\end{tcolorbox}
\begin{tcolorbox}
\textsubscript{14} (68-15) извлеки меня из тины, чтобы не погрязнуть мне; да избавлюсь от ненавидящих меня и от глубоких вод;
\end{tcolorbox}
\begin{tcolorbox}
\textsubscript{15} (68-16) да не увлечет меня стремление вод, да не поглотит меня пучина, да не затворит надо мною пропасть зева своего.
\end{tcolorbox}
\begin{tcolorbox}
\textsubscript{16} (68-17) Услышь меня, Господи, ибо блага милость Твоя; по множеству щедрот Твоих призри на меня;
\end{tcolorbox}
\begin{tcolorbox}
\textsubscript{17} (68-18) не скрывай лица Твоего от раба Твоего, ибо я скорблю; скоро услышь меня;
\end{tcolorbox}
\begin{tcolorbox}
\textsubscript{18} (68-19) приблизься к душе моей, избавь ее; ради врагов моих спаси меня.
\end{tcolorbox}
\begin{tcolorbox}
\textsubscript{19} (68-20) Ты знаешь поношение мое, стыд мой и посрамление мое: враги мои все пред Тобою.
\end{tcolorbox}
\begin{tcolorbox}
\textsubscript{20} (68-21) Поношение сокрушило сердце мое, и я изнемог, ждал сострадания, но нет его, --утешителей, но не нахожу.
\end{tcolorbox}
\begin{tcolorbox}
\textsubscript{21} (68-22) И дали мне в пищу желчь, и в жажде моей напоили меня уксусом.
\end{tcolorbox}
\begin{tcolorbox}
\textsubscript{22} (68-23) Да будет трапеза их сетью им, и мирное пиршество их--западнею;
\end{tcolorbox}
\begin{tcolorbox}
\textsubscript{23} (68-24) да помрачатся глаза их, чтоб им не видеть, и чресла их расслабь навсегда;
\end{tcolorbox}
\begin{tcolorbox}
\textsubscript{24} (68-25) излей на них ярость Твою, и пламень гнева Твоего да обымет их;
\end{tcolorbox}
\begin{tcolorbox}
\textsubscript{25} (68-26) жилище их да будет пусто, и в шатрах их да не будет живущих,
\end{tcolorbox}
\begin{tcolorbox}
\textsubscript{26} (68-27) ибо, кого Ты поразил, они [еще] преследуют, и страдания уязвленных Тобою умножают.
\end{tcolorbox}
\begin{tcolorbox}
\textsubscript{27} (68-28) Приложи беззаконие к беззаконию их, и да не войдут они в правду Твою;
\end{tcolorbox}
\begin{tcolorbox}
\textsubscript{28} (68-29) да изгладятся они из книги живых и с праведниками да не напишутся.
\end{tcolorbox}
\begin{tcolorbox}
\textsubscript{29} (68-30) А я беден и страдаю; помощь Твоя, Боже, да восставит меня.
\end{tcolorbox}
\begin{tcolorbox}
\textsubscript{30} (68-31) Я буду славить имя Бога [моего] в песни, буду превозносить Его в славословии,
\end{tcolorbox}
\begin{tcolorbox}
\textsubscript{31} (68-32) и будет это благоугоднее Господу, нежели вол, нежели телец с рогами и с копытами.
\end{tcolorbox}
\begin{tcolorbox}
\textsubscript{32} (68-33) Увидят [это] страждущие и возрадуются. И оживет сердце ваше, ищущие Бога,
\end{tcolorbox}
\begin{tcolorbox}
\textsubscript{33} (68-34) ибо Господь внемлет нищим и не пренебрегает узников Своих.
\end{tcolorbox}
\begin{tcolorbox}
\textsubscript{34} (68-35) Да восхвалят Его небеса и земля, моря и все движущееся в них;
\end{tcolorbox}
\begin{tcolorbox}
\textsubscript{35} (68-36) ибо спасет Бог Сион, создаст города Иудины, и поселятся там и наследуют его,
\end{tcolorbox}
\begin{tcolorbox}
\textsubscript{36} (68-37) и потомство рабов Его утвердится в нем, и любящие имя Его будут поселяться на нем.
\end{tcolorbox}
\subsection{CHAPTER 70}
\begin{tcolorbox}
\textsubscript{1} (69-1) ^^Начальнику хора. Псалом Давида. В воспоминание.^^ (69-2) Поспеши, Боже, избавить меня, [поспеши], Господи, на помощь мне.
\end{tcolorbox}
\begin{tcolorbox}
\textsubscript{2} (69-3) Да постыдятся и посрамятся ищущие души моей! Да будут обращены назад и преданы посмеянию желающие мне зла!
\end{tcolorbox}
\begin{tcolorbox}
\textsubscript{3} (69-4) Да будут обращены назад за поношение меня говорящие [мне]: 'хорошо! хорошо!'
\end{tcolorbox}
\begin{tcolorbox}
\textsubscript{4} (69-5) Да возрадуются и возвеселятся о Тебе все, ищущие Тебя, и любящие спасение Твое да говорят непрестанно: 'велик Бог!'
\end{tcolorbox}
\begin{tcolorbox}
\textsubscript{5} (69-6) Я же беден и нищ; Боже, поспеши ко мне! Ты помощь моя и Избавитель мой; Господи! не замедли.
\end{tcolorbox}
\subsection{CHAPTER 71}
\begin{tcolorbox}
\textsubscript{1} (70-1) На Тебя, Господи, уповаю, да не постыжусь вовек.
\end{tcolorbox}
\begin{tcolorbox}
\textsubscript{2} (70-2) По правде Твоей избавь меня и освободи меня; приклони ухо Твое ко мне и спаси меня.
\end{tcolorbox}
\begin{tcolorbox}
\textsubscript{3} (70-3) Будь мне твердым прибежищем, куда я всегда мог бы укрываться; Ты заповедал спасти меня, ибо твердыня моя и крепость моя--Ты.
\end{tcolorbox}
\begin{tcolorbox}
\textsubscript{4} (70-4) Боже мой! избавь меня из руки нечестивого, из руки беззаконника и притеснителя,
\end{tcolorbox}
\begin{tcolorbox}
\textsubscript{5} (70-5) ибо Ты--надежда моя, Господи Боже, упование мое от юности моей.
\end{tcolorbox}
\begin{tcolorbox}
\textsubscript{6} (70-6) На Тебе утверждался я от утробы; Ты извел меня из чрева матери моей; Тебе хвала моя не престанет.
\end{tcolorbox}
\begin{tcolorbox}
\textsubscript{7} (70-7) Для многих я был как бы дивом, но Ты твердая моя надежда.
\end{tcolorbox}
\begin{tcolorbox}
\textsubscript{8} (70-8) Да наполнятся уста мои хвалою, [чтобы воспевать] всякий день великолепие Твое.
\end{tcolorbox}
\begin{tcolorbox}
\textsubscript{9} (70-9) Не отвергни меня во время старости; когда будет оскудевать сила моя, не оставь меня,
\end{tcolorbox}
\begin{tcolorbox}
\textsubscript{10} (70-10) ибо враги мои говорят против меня, и подстерегающие душу мою советуются между собою,
\end{tcolorbox}
\begin{tcolorbox}
\textsubscript{11} (70-11) говоря: 'Бог оставил его; преследуйте и схватите его, ибо нет избавляющего'.
\end{tcolorbox}
\begin{tcolorbox}
\textsubscript{12} (70-12) Боже! не удаляйся от меня; Боже мой! поспеши на помощь мне.
\end{tcolorbox}
\begin{tcolorbox}
\textsubscript{13} (70-13) Да постыдятся и исчезнут враждующие против души моей, да покроются стыдом и бесчестием ищущие мне зла!
\end{tcolorbox}
\begin{tcolorbox}
\textsubscript{14} (70-14) А я всегда буду уповать [на Тебя] и умножать всякую хвалу Тебе.
\end{tcolorbox}
\begin{tcolorbox}
\textsubscript{15} (70-15) Уста мои будут возвещать правду Твою, всякий день благодеяния Твои; ибо я не знаю им числа.
\end{tcolorbox}
\begin{tcolorbox}
\textsubscript{16} (70-16) Войду в [размышление] о силах Господа Бога; воспомяну правду Твою--единственно Твою.
\end{tcolorbox}
\begin{tcolorbox}
\textsubscript{17} (70-17) Боже! Ты наставлял меня от юности моей, и доныне я возвещаю чудеса Твои.
\end{tcolorbox}
\begin{tcolorbox}
\textsubscript{18} (70-18) И до старости, и до седины не оставь меня, Боже, доколе не возвещу силы Твоей роду сему и всем грядущим могущества Твоего.
\end{tcolorbox}
\begin{tcolorbox}
\textsubscript{19} (70-19) Правда Твоя, Боже, до превыспренних; великие дела соделал Ты; Боже, кто подобен Тебе?
\end{tcolorbox}
\begin{tcolorbox}
\textsubscript{20} (70-20) Ты посылал на меня многие и лютые беды, но и опять оживлял меня и из бездн земли опять выводил меня.
\end{tcolorbox}
\begin{tcolorbox}
\textsubscript{21} (70-21) Ты возвышал меня и утешал меня.
\end{tcolorbox}
\begin{tcolorbox}
\textsubscript{22} (70-22) И я буду славить Тебя на псалтири, Твою истину, Боже мой; буду воспевать Тебя на гуслях, Святый Израилев!
\end{tcolorbox}
\begin{tcolorbox}
\textsubscript{23} (70-23) Радуются уста мои, когда я пою Тебе, и душа моя, которую Ты избавил;
\end{tcolorbox}
\begin{tcolorbox}
\textsubscript{24} (70-24) и язык мой всякий день будет возвещать правду Твою, ибо постыжены и посрамлены ищущие мне зла. О Соломоне.
\end{tcolorbox}
\subsection{CHAPTER 72}
\begin{tcolorbox}
\textsubscript{1} (71-1) ^^Псалом Давида.^^ Боже! даруй царю Твой суд и сыну царя Твою правду,
\end{tcolorbox}
\begin{tcolorbox}
\textsubscript{2} (71-2) да судит праведно людей Твоих и нищих Твоих на суде;
\end{tcolorbox}
\begin{tcolorbox}
\textsubscript{3} (71-3) да принесут горы мир людям и холмы правду;
\end{tcolorbox}
\begin{tcolorbox}
\textsubscript{4} (71-4) да судит нищих народа, да спасет сынов убогого и смирит притеснителя, --
\end{tcolorbox}
\begin{tcolorbox}
\textsubscript{5} (71-5) и будут бояться Тебя, доколе пребудут солнце и луна, в роды родов.
\end{tcolorbox}
\begin{tcolorbox}
\textsubscript{6} (71-6) Он сойдет, как дождь на скошенный луг, как капли, орошающие землю;
\end{tcolorbox}
\begin{tcolorbox}
\textsubscript{7} (71-7) во дни его процветет праведник, и будет обилие мира, доколе не престанет луна;
\end{tcolorbox}
\begin{tcolorbox}
\textsubscript{8} (71-8) он будет обладать от моря до моря и от реки до концов земли;
\end{tcolorbox}
\begin{tcolorbox}
\textsubscript{9} (71-9) падут пред ним жители пустынь, и враги его будут лизать прах;
\end{tcolorbox}
\begin{tcolorbox}
\textsubscript{10} (71-10) цари Фарсиса и островов поднесут ему дань; цари Аравии и Савы принесут дары;
\end{tcolorbox}
\begin{tcolorbox}
\textsubscript{11} (71-11) и поклонятся ему все цари; все народы будут служить ему;
\end{tcolorbox}
\begin{tcolorbox}
\textsubscript{12} (71-12) ибо он избавит нищего, вопиющего и угнетенного, у которого нет помощника.
\end{tcolorbox}
\begin{tcolorbox}
\textsubscript{13} (71-13) Будет милосерд к нищему и убогому, и души убогих спасет;
\end{tcolorbox}
\begin{tcolorbox}
\textsubscript{14} (71-14) от коварства и насилия избавит души их, и драгоценна будет кровь их пред очами его;
\end{tcolorbox}
\begin{tcolorbox}
\textsubscript{15} (71-15) и будет жить, и будут давать ему от золота Аравии, и будут молиться о нем непрестанно, всякий день благословлять его;
\end{tcolorbox}
\begin{tcolorbox}
\textsubscript{16} (71-16) будет обилие хлеба на земле, наверху гор; плоды его будут волноваться, как [лес] на Ливане, и в городах размножатся люди, как трава на земле;
\end{tcolorbox}
\begin{tcolorbox}
\textsubscript{17} (71-17) будет имя его вовек; доколе пребывает солнце, будет передаваться имя его; и благословятся в нем [племена], все народы ублажат его.
\end{tcolorbox}
\begin{tcolorbox}
\textsubscript{18} (71-18) Благословен Господь Бог, Бог Израилев, един творящий чудеса,
\end{tcolorbox}
\begin{tcolorbox}
\textsubscript{19} (71-19) и благословенно имя славы Его вовек, и наполнится славою Его вся земля. Аминь и аминь.
\end{tcolorbox}
\begin{tcolorbox}
\textsubscript{20} (71-20) Кончились молитвы Давида, сына Иесеева.
\end{tcolorbox}
\subsection{CHAPTER 73}
\begin{tcolorbox}
\textsubscript{1} (72-1) ^^Псалом Асафа.^^ Как благ Бог к Израилю, к чистым сердцем!
\end{tcolorbox}
\begin{tcolorbox}
\textsubscript{2} (72-2) А я--едва не пошатнулись ноги мои, едва не поскользнулись стопы мои, --
\end{tcolorbox}
\begin{tcolorbox}
\textsubscript{3} (72-3) я позавидовал безумным, видя благоденствие нечестивых,
\end{tcolorbox}
\begin{tcolorbox}
\textsubscript{4} (72-4) ибо им нет страданий до смерти их, и крепки силы их;
\end{tcolorbox}
\begin{tcolorbox}
\textsubscript{5} (72-5) на работе человеческой нет их, и с [прочими] людьми не подвергаются ударам.
\end{tcolorbox}
\begin{tcolorbox}
\textsubscript{6} (72-6) Оттого гордость, как ожерелье, обложила их, и дерзость, [как] наряд, одевает их;
\end{tcolorbox}
\begin{tcolorbox}
\textsubscript{7} (72-7) выкатились от жира глаза их, бродят помыслы в сердце;
\end{tcolorbox}
\begin{tcolorbox}
\textsubscript{8} (72-8) над всем издеваются, злобно разглашают клевету, говорят свысока;
\end{tcolorbox}
\begin{tcolorbox}
\textsubscript{9} (72-9) поднимают к небесам уста свои, и язык их расхаживает по земле.
\end{tcolorbox}
\begin{tcolorbox}
\textsubscript{10} (72-10) Потому туда же обращается народ Его, и пьют воду полною чашею,
\end{tcolorbox}
\begin{tcolorbox}
\textsubscript{11} (72-11) и говорят: 'как узнает Бог? и есть ли ведение у Вышнего?'
\end{tcolorbox}
\begin{tcolorbox}
\textsubscript{12} (72-12) И вот, эти нечестивые благоденствуют в веке сем, умножают богатство.
\end{tcolorbox}
\begin{tcolorbox}
\textsubscript{13} (72-13) так не напрасно ли я очищал сердце мое и омывал в невинности руки мои,
\end{tcolorbox}
\begin{tcolorbox}
\textsubscript{14} (72-14) и подвергал себя ранам всякий день и обличениям всякое утро?
\end{tcolorbox}
\begin{tcolorbox}
\textsubscript{15} (72-15) [Но] если бы я сказал: 'буду рассуждать так', --то я виновен был бы пред родом сынов Твоих.
\end{tcolorbox}
\begin{tcolorbox}
\textsubscript{16} (72-16) И думал я, как бы уразуметь это, но это трудно было в глазах моих,
\end{tcolorbox}
\begin{tcolorbox}
\textsubscript{17} (72-17) доколе не вошел я во святилище Божие и не уразумел конца их.
\end{tcolorbox}
\begin{tcolorbox}
\textsubscript{18} (72-18) Так! на скользких путях поставил Ты их и низвергаешь их в пропасти.
\end{tcolorbox}
\begin{tcolorbox}
\textsubscript{19} (72-19) Как нечаянно пришли они в разорение, исчезли, погибли от ужасов!
\end{tcolorbox}
\begin{tcolorbox}
\textsubscript{20} (72-20) Как сновидение по пробуждении, так Ты, Господи, пробудив [их], уничтожишь мечты их.
\end{tcolorbox}
\begin{tcolorbox}
\textsubscript{21} (72-21) Когда кипело сердце мое, и терзалась внутренность моя,
\end{tcolorbox}
\begin{tcolorbox}
\textsubscript{22} (72-22) тогда я был невежда и не разумел; как скот был я пред Тобою.
\end{tcolorbox}
\begin{tcolorbox}
\textsubscript{23} (72-23) Но я всегда с Тобою: Ты держишь меня за правую руку;
\end{tcolorbox}
\begin{tcolorbox}
\textsubscript{24} (72-24) Ты руководишь меня советом Твоим и потом примешь меня в славу.
\end{tcolorbox}
\begin{tcolorbox}
\textsubscript{25} (72-25) Кто мне на небе? и с Тобою ничего не хочу на земле.
\end{tcolorbox}
\begin{tcolorbox}
\textsubscript{26} (72-26) Изнемогает плоть моя и сердце мое: Бог твердыня сердца моего и часть моя вовек.
\end{tcolorbox}
\begin{tcolorbox}
\textsubscript{27} (72-27) Ибо вот, удаляющие себя от Тебя гибнут; Ты истребляешь всякого отступающего от Тебя.
\end{tcolorbox}
\begin{tcolorbox}
\textsubscript{28} (72-28) А мне благо приближаться к Богу! На Господа Бога я возложил упование мое, чтобы возвещать все дела Твои.
\end{tcolorbox}
\subsection{CHAPTER 74}
\begin{tcolorbox}
\textsubscript{1} (73-1) ^^Учение Асафа.^^ Для чего, Боже, отринул нас навсегда? возгорелся гнев Твой на овец пажити Твоей?
\end{tcolorbox}
\begin{tcolorbox}
\textsubscript{2} (73-2) Вспомни сонм Твой, [который] Ты стяжал издревле, искупил в жезл достояния Твоего, --эту гору Сион, на которой Ты веселился.
\end{tcolorbox}
\begin{tcolorbox}
\textsubscript{3} (73-3) Подвигни стопы Твои к вековым развалинам: все разрушил враг во святилище.
\end{tcolorbox}
\begin{tcolorbox}
\textsubscript{4} (73-4) Рыкают враги Твои среди собраний Твоих; поставили знаки свои вместо знамений [наших];
\end{tcolorbox}
\begin{tcolorbox}
\textsubscript{5} (73-5) показывали себя подобными поднимающему вверх секиру на сплетшиеся ветви дерева;
\end{tcolorbox}
\begin{tcolorbox}
\textsubscript{6} (73-6) и ныне все резьбы в нем в один раз разрушили секирами и бердышами;
\end{tcolorbox}
\begin{tcolorbox}
\textsubscript{7} (73-7) предали огню святилище Твое; совсем осквернили жилище имени Твоего;
\end{tcolorbox}
\begin{tcolorbox}
\textsubscript{8} (73-8) сказали в сердце своем: 'разорим их совсем', --и сожгли все места собраний Божиих на земле.
\end{tcolorbox}
\begin{tcolorbox}
\textsubscript{9} (73-9) Знамений наших мы не видим, нет уже пророка, и нет с нами, кто знал бы, доколе [это будет].
\end{tcolorbox}
\begin{tcolorbox}
\textsubscript{10} (73-10) Доколе, Боже, будет поносить враг? вечно ли будет хулить противник имя Твое?
\end{tcolorbox}
\begin{tcolorbox}
\textsubscript{11} (73-11) Для чего отклоняешь руку Твою и десницу Твою? Из среды недра Твоего порази [их].
\end{tcolorbox}
\begin{tcolorbox}
\textsubscript{12} (73-12) Боже, Царь мой от века, устрояющий спасение посреди земли!
\end{tcolorbox}
\begin{tcolorbox}
\textsubscript{13} (73-13) Ты расторг силою Твоею море, Ты сокрушил головы змиев в воде;
\end{tcolorbox}
\begin{tcolorbox}
\textsubscript{14} (73-14) Ты сокрушил голову левиафана, отдал его в пищу людям пустыни.
\end{tcolorbox}
\begin{tcolorbox}
\textsubscript{15} (73-15) Ты иссек источник и поток, Ты иссушил сильные реки.
\end{tcolorbox}
\begin{tcolorbox}
\textsubscript{16} (73-16) Твой день и Твоя ночь: Ты уготовал светила и солнце;
\end{tcolorbox}
\begin{tcolorbox}
\textsubscript{17} (73-17) Ты установил все пределы земли, лето и зиму Ты учредил.
\end{tcolorbox}
\begin{tcolorbox}
\textsubscript{18} (73-18) Вспомни же: враг поносит Господа, и люди безумные хулят имя Твое.
\end{tcolorbox}
\begin{tcolorbox}
\textsubscript{19} (73-19) Не предай зверям душу горлицы Твоей; собрания убогих Твоих не забудь навсегда.
\end{tcolorbox}
\begin{tcolorbox}
\textsubscript{20} (73-20) Призри на завет Твой; ибо наполнились все мрачные места земли жилищами насилия.
\end{tcolorbox}
\begin{tcolorbox}
\textsubscript{21} (73-21) Да не возвратится угнетенный посрамленным; нищий и убогий да восхвалят имя Твое.
\end{tcolorbox}
\begin{tcolorbox}
\textsubscript{22} (73-22) Восстань, Боже, защити дело Твое, вспомни вседневное поношение Твое от безумного;
\end{tcolorbox}
\begin{tcolorbox}
\textsubscript{23} (73-23) не забудь крика врагов Твоих; шум восстающих против Тебя непрестанно поднимается.
\end{tcolorbox}
\subsection{CHAPTER 75}
\begin{tcolorbox}
\textsubscript{1} (74-1) ^^Начальнику хора. Не погуби. Псалом Асафа. Песнь.^^ (74-2) Славим Тебя, Боже, славим, ибо близко имя Твое; возвещают чудеса Твои.
\end{tcolorbox}
\begin{tcolorbox}
\textsubscript{2} (74-3) 'Когда изберу время, Я произведу суд по правде.
\end{tcolorbox}
\begin{tcolorbox}
\textsubscript{3} (74-4) Колеблется земля и все живущие на ней: Я утвержу столпы ее'.
\end{tcolorbox}
\begin{tcolorbox}
\textsubscript{4} (74-5) Говорю безумствующим: 'не безумствуйте', и нечестивым: 'не поднимайте рога,
\end{tcolorbox}
\begin{tcolorbox}
\textsubscript{5} (74-6) не поднимайте высоко рога вашего, [не] говорите жестоковыйно',
\end{tcolorbox}
\begin{tcolorbox}
\textsubscript{6} (74-7) ибо не от востока и не от запада и не от пустыни возвышение,
\end{tcolorbox}
\begin{tcolorbox}
\textsubscript{7} (74-8) но Бог есть судия: одного унижает, а другого возносит;
\end{tcolorbox}
\begin{tcolorbox}
\textsubscript{8} (74-9) ибо чаша в руке Господа, вино кипит в ней, полное смешения, и Он наливает из нее. Даже дрожжи ее будут выжимать и пить все нечестивые земли.
\end{tcolorbox}
\begin{tcolorbox}
\textsubscript{9} (74-10) А я буду возвещать вечно, буду воспевать Бога Иаковлева,
\end{tcolorbox}
\begin{tcolorbox}
\textsubscript{10} (74-11) все роги нечестивых сломлю, и вознесутся роги праведника.
\end{tcolorbox}
\subsection{CHAPTER 76}
\begin{tcolorbox}
\textsubscript{1} (75-1) ^^Начальнику хора. На струнных [орудиях]. Псалом Асафа. Песнь.^^ (75-2) Ведом в Иудее Бог; у Израиля велико имя Его.
\end{tcolorbox}
\begin{tcolorbox}
\textsubscript{2} (75-3) И было в Салиме жилище Его и пребывание Его на Сионе.
\end{tcolorbox}
\begin{tcolorbox}
\textsubscript{3} (75-4) Там сокрушил Он стрелы лука, щит и меч и брань.
\end{tcolorbox}
\begin{tcolorbox}
\textsubscript{4} (75-5) Ты славен, могущественнее гор хищнических.
\end{tcolorbox}
\begin{tcolorbox}
\textsubscript{5} (75-6) Крепкие сердцем стали добычею, уснули сном своим, и не нашли все мужи силы рук своих.
\end{tcolorbox}
\begin{tcolorbox}
\textsubscript{6} (75-7) От прещения Твоего, Боже Иакова, вздремали и колесница и конь.
\end{tcolorbox}
\begin{tcolorbox}
\textsubscript{7} (75-8) Ты страшен, и кто устоит пред лицем Твоим во время гнева Твоего?
\end{tcolorbox}
\begin{tcolorbox}
\textsubscript{8} (75-9) С небес Ты возвестил суд; земля убоялась и утихла,
\end{tcolorbox}
\begin{tcolorbox}
\textsubscript{9} (75-10) когда восстал Бог на суд, чтобы спасти всех угнетенных земли.
\end{tcolorbox}
\begin{tcolorbox}
\textsubscript{10} (75-11) И гнев человеческий обратится во славу Тебе: остаток гнева Ты укротишь.
\end{tcolorbox}
\begin{tcolorbox}
\textsubscript{11} (75-12) Делайте и воздавайте обеты Господу, Богу вашему; все, которые вокруг Него, да принесут дары Страшному:
\end{tcolorbox}
\begin{tcolorbox}
\textsubscript{12} (75-13) Он укрощает дух князей, Он страшен для царей земных.
\end{tcolorbox}
\subsection{CHAPTER 77}
\begin{tcolorbox}
\textsubscript{1} (76-1) ^^Начальнику хора Идифумова. Псалом Асафа.^^ (76-2) Глас мой к Богу, и я буду взывать; глас мой к Богу, и Он услышит меня.
\end{tcolorbox}
\begin{tcolorbox}
\textsubscript{2} (76-3) В день скорби моей ищу Господа; рука моя простерта ночью и не опускается; душа моя отказывается от утешения.
\end{tcolorbox}
\begin{tcolorbox}
\textsubscript{3} (76-4) Вспоминаю о Боге и трепещу; помышляю, и изнемогает дух мой.
\end{tcolorbox}
\begin{tcolorbox}
\textsubscript{4} (76-5) Ты не даешь мне сомкнуть очей моих; я потрясен и не могу говорить.
\end{tcolorbox}
\begin{tcolorbox}
\textsubscript{5} (76-6) Размышляю о днях древних, о летах веков [минувших];
\end{tcolorbox}
\begin{tcolorbox}
\textsubscript{6} (76-7) припоминаю песни мои в ночи, беседую с сердцем моим, и дух мой испытывает:
\end{tcolorbox}
\begin{tcolorbox}
\textsubscript{7} (76-8) неужели навсегда отринул Господь, и не будет более благоволить?
\end{tcolorbox}
\begin{tcolorbox}
\textsubscript{8} (76-9) неужели навсегда престала милость Его, и пресеклось слово Его в род и род?
\end{tcolorbox}
\begin{tcolorbox}
\textsubscript{9} (76-10) неужели Бог забыл миловать? Неужели во гневе затворил щедроты Свои?
\end{tcolorbox}
\begin{tcolorbox}
\textsubscript{10} (76-11) И сказал я: 'вот мое горе--изменение десницы Всевышнего'.
\end{tcolorbox}
\begin{tcolorbox}
\textsubscript{11} (76-12) Буду вспоминать о делах Господа; буду вспоминать о чудесах Твоих древних;
\end{tcolorbox}
\begin{tcolorbox}
\textsubscript{12} (76-13) буду вникать во все дела Твои, размышлять о великих Твоих деяниях.
\end{tcolorbox}
\begin{tcolorbox}
\textsubscript{13} (76-14) Боже! свят путь Твой. Кто Бог так великий, как Бог [наш]!
\end{tcolorbox}
\begin{tcolorbox}
\textsubscript{14} (76-15) Ты--Бог, творящий чудеса; Ты явил могущество Свое среди народов;
\end{tcolorbox}
\begin{tcolorbox}
\textsubscript{15} (76-16) Ты избавил мышцею народ Твой, сынов Иакова и Иосифа.
\end{tcolorbox}
\begin{tcolorbox}
\textsubscript{16} (76-17) Видели Тебя, Боже, воды, видели Тебя воды и убоялись, и вострепетали бездны.
\end{tcolorbox}
\begin{tcolorbox}
\textsubscript{17} (76-18) Облака изливали воды, тучи издавали гром, и стрелы Твои летали.
\end{tcolorbox}
\begin{tcolorbox}
\textsubscript{18} (76-19) Глас грома Твоего в круге небесном; молнии освещали вселенную; земля содрогалась и тряслась.
\end{tcolorbox}
\begin{tcolorbox}
\textsubscript{19} (76-20) Путь Твой в море, и стезя Твоя в водах великих, и следы Твои неведомы.
\end{tcolorbox}
\begin{tcolorbox}
\textsubscript{20} (76-21) Как стадо, вел Ты народ Твой рукою Моисея и Аарона.
\end{tcolorbox}
\subsection{CHAPTER 78}
\begin{tcolorbox}
\textsubscript{1} (77-1) ^^Учение Асафа.^^ Внимай, народ мой, закону моему, приклоните ухо ваше к словам уст моих.
\end{tcolorbox}
\begin{tcolorbox}
\textsubscript{2} (77-2) Открою уста мои в притче и произнесу гадания из древности.
\end{tcolorbox}
\begin{tcolorbox}
\textsubscript{3} (77-3) Что слышали мы и узнали, и отцы наши рассказали нам,
\end{tcolorbox}
\begin{tcolorbox}
\textsubscript{4} (77-4) не скроем от детей их, возвещая роду грядущему славу Господа, и силу Его, и чудеса Его, которые Он сотворил.
\end{tcolorbox}
\begin{tcolorbox}
\textsubscript{5} (77-5) Он постановил устав в Иакове и положил закон в Израиле, который заповедал отцам нашим возвещать детям их,
\end{tcolorbox}
\begin{tcolorbox}
\textsubscript{6} (77-6) чтобы знал грядущий род, дети, которые родятся, и чтобы они в свое время возвещали своим детям, --
\end{tcolorbox}
\begin{tcolorbox}
\textsubscript{7} (77-7) возлагать надежду свою на Бога и не забывать дел Божиих, и хранить заповеди Его,
\end{tcolorbox}
\begin{tcolorbox}
\textsubscript{8} (77-8) и не быть подобными отцам их, роду упорному и мятежному, неустроенному сердцем и неверному Богу духом своим.
\end{tcolorbox}
\begin{tcolorbox}
\textsubscript{9} (77-9) Сыны Ефремовы, вооруженные, стреляющие из луков, обратились назад в день брани:
\end{tcolorbox}
\begin{tcolorbox}
\textsubscript{10} (77-10) они не сохранили завета Божия и отреклись ходить в законе Его;
\end{tcolorbox}
\begin{tcolorbox}
\textsubscript{11} (77-11) забыли дела Его и чудеса, которые Он явил им.
\end{tcolorbox}
\begin{tcolorbox}
\textsubscript{12} (77-12) Он пред глазами отцов их сотворил чудеса в земле Египетской, на поле Цоан:
\end{tcolorbox}
\begin{tcolorbox}
\textsubscript{13} (77-13) разделил море, и провел их чрез него, и поставил воды стеною;
\end{tcolorbox}
\begin{tcolorbox}
\textsubscript{14} (77-14) и днем вел их облаком, а во всю ночь светом огня;
\end{tcolorbox}
\begin{tcolorbox}
\textsubscript{15} (77-15) рассек камень в пустыне и напоил их, как из великой бездны;
\end{tcolorbox}
\begin{tcolorbox}
\textsubscript{16} (77-16) из скалы извел потоки, и воды потекли, как реки.
\end{tcolorbox}
\begin{tcolorbox}
\textsubscript{17} (77-17) Но они продолжали грешить пред Ним и раздражать Всевышнего в пустыне:
\end{tcolorbox}
\begin{tcolorbox}
\textsubscript{18} (77-18) искушали Бога в сердце своем, требуя пищи по душе своей,
\end{tcolorbox}
\begin{tcolorbox}
\textsubscript{19} (77-19) и говорили против Бога и сказали: 'может ли Бог приготовить трапезу в пустыне?'
\end{tcolorbox}
\begin{tcolorbox}
\textsubscript{20} (77-20) Вот, Он ударил в камень, и потекли воды, и полились ручьи. 'Может ли Он дать и хлеб, может ли приготовлять мясо народу Своему?'
\end{tcolorbox}
\begin{tcolorbox}
\textsubscript{21} (77-21) Господь услышал и воспламенился гневом, и огонь возгорелся на Иакова, и гнев подвигнулся на Израиля
\end{tcolorbox}
\begin{tcolorbox}
\textsubscript{22} (77-22) за то, что не веровали в Бога и не уповали на спасение Его.
\end{tcolorbox}
\begin{tcolorbox}
\textsubscript{23} (77-23) Он повелел облакам свыше и отверз двери неба,
\end{tcolorbox}
\begin{tcolorbox}
\textsubscript{24} (77-24) и одождил на них манну в пищу, и хлеб небесный дал им.
\end{tcolorbox}
\begin{tcolorbox}
\textsubscript{25} (77-25) Хлеб ангельский ел человек; послал Он им пищу до сытости.
\end{tcolorbox}
\begin{tcolorbox}
\textsubscript{26} (77-26) Он возбудил на небе восточный ветер и навел южный силою Своею
\end{tcolorbox}
\begin{tcolorbox}
\textsubscript{27} (77-27) и, как пыль, одождил на них мясо и, как песок морской, птиц пернатых:
\end{tcolorbox}
\begin{tcolorbox}
\textsubscript{28} (77-28) поверг их среди стана их, около жилищ их, --
\end{tcolorbox}
\begin{tcolorbox}
\textsubscript{29} (77-29) и они ели и пресытились; и желаемое ими дал им.
\end{tcolorbox}
\begin{tcolorbox}
\textsubscript{30} (77-30) Но еще не прошла прихоть их, еще пища была в устах их,
\end{tcolorbox}
\begin{tcolorbox}
\textsubscript{31} (77-31) гнев Божий пришел на них, убил тучных их и юношей Израилевых низложил.
\end{tcolorbox}
\begin{tcolorbox}
\textsubscript{32} (77-32) При всем этом они продолжали грешить и не верили чудесам Его.
\end{tcolorbox}
\begin{tcolorbox}
\textsubscript{33} (77-33) И погубил дни их в суете и лета их в смятении.
\end{tcolorbox}
\begin{tcolorbox}
\textsubscript{34} (77-34) Когда Он убивал их, они искали Его и обращались, и с раннего утра прибегали к Богу,
\end{tcolorbox}
\begin{tcolorbox}
\textsubscript{35} (77-35) и вспоминали, что Бог--их прибежище, и Бог Всевышний--Избавитель их,
\end{tcolorbox}
\begin{tcolorbox}
\textsubscript{36} (77-36) и льстили Ему устами своими и языком своим лгали пред Ним;
\end{tcolorbox}
\begin{tcolorbox}
\textsubscript{37} (77-37) сердце же их было неправо пред Ним, и они не были верны завету Его.
\end{tcolorbox}
\begin{tcolorbox}
\textsubscript{38} (77-38) Но Он, Милостивый, прощал грех и не истреблял их, многократно отвращал гнев Свой и не возбуждал всей ярости Своей:
\end{tcolorbox}
\begin{tcolorbox}
\textsubscript{39} (77-39) Он помнил, что они плоть, дыхание, которое уходит и не возвращается.
\end{tcolorbox}
\begin{tcolorbox}
\textsubscript{40} (77-40) Сколько раз они раздражали Его в пустыне и прогневляли Его в [стране] необитаемой!
\end{tcolorbox}
\begin{tcolorbox}
\textsubscript{41} (77-41) и снова искушали Бога и оскорбляли Святаго Израилева,
\end{tcolorbox}
\begin{tcolorbox}
\textsubscript{42} (77-42) не помнили руки Его, дня, когда Он избавил их от угнетения,
\end{tcolorbox}
\begin{tcolorbox}
\textsubscript{43} (77-43) когда сотворил в Египте знамения Свои и чудеса Свои на поле Цоан;
\end{tcolorbox}
\begin{tcolorbox}
\textsubscript{44} (77-44) и превратил реки их и потоки их в кровь, чтобы они не могли пить;
\end{tcolorbox}
\begin{tcolorbox}
\textsubscript{45} (77-45) послал на них насекомых, чтобы жалили их, и жаб, чтобы губили их;
\end{tcolorbox}
\begin{tcolorbox}
\textsubscript{46} (77-46) земные произрастения их отдал гусенице и труд их--саранче;
\end{tcolorbox}
\begin{tcolorbox}
\textsubscript{47} (77-47) виноград их побил градом и сикоморы их--льдом;
\end{tcolorbox}
\begin{tcolorbox}
\textsubscript{48} (77-48) скот их предал граду и стада их--молниям;
\end{tcolorbox}
\begin{tcolorbox}
\textsubscript{49} (77-49) послал на них пламень гнева Своего, и негодование, и ярость и бедствие, посольство злых ангелов;
\end{tcolorbox}
\begin{tcolorbox}
\textsubscript{50} (77-50) уравнял стезю гневу Своему, не охранял души их от смерти, и скот их предал моровой язве;
\end{tcolorbox}
\begin{tcolorbox}
\textsubscript{51} (77-51) поразил всякого первенца в Египте, начатки сил в шатрах Хамовых;
\end{tcolorbox}
\begin{tcolorbox}
\textsubscript{52} (77-52) и повел народ Свой, как овец, и вел их, как стадо, пустынею;
\end{tcolorbox}
\begin{tcolorbox}
\textsubscript{53} (77-53) вел их безопасно, и они не страшились, а врагов их покрыло море;
\end{tcolorbox}
\begin{tcolorbox}
\textsubscript{54} (77-54) и привел их в область святую Свою, на гору сию, которую стяжала десница Его;
\end{tcolorbox}
\begin{tcolorbox}
\textsubscript{55} (77-55) прогнал от лица их народы и землю их разделил в наследие им, и колена Израилевы поселил в шатрах их.
\end{tcolorbox}
\begin{tcolorbox}
\textsubscript{56} (77-56) Но они еще искушали и огорчали Бога Всевышнего, и уставов Его не сохраняли;
\end{tcolorbox}
\begin{tcolorbox}
\textsubscript{57} (77-57) отступали и изменяли, как отцы их, обращались назад, как неверный лук;
\end{tcolorbox}
\begin{tcolorbox}
\textsubscript{58} (77-58) огорчали Его высотами своими и истуканами своими возбуждали ревность Его.
\end{tcolorbox}
\begin{tcolorbox}
\textsubscript{59} (77-59) Услышал Бог и воспламенился гневом и сильно вознегодовал на Израиля;
\end{tcolorbox}
\begin{tcolorbox}
\textsubscript{60} (77-60) отринул жилище в Силоме, скинию, в которой обитал Он между человеками;
\end{tcolorbox}
\begin{tcolorbox}
\textsubscript{61} (77-61) и отдал в плен крепость Свою и славу Свою в руки врага,
\end{tcolorbox}
\begin{tcolorbox}
\textsubscript{62} (77-62) и предал мечу народ Свой и прогневался на наследие Свое.
\end{tcolorbox}
\begin{tcolorbox}
\textsubscript{63} (77-63) Юношей его поедал огонь, и девицам его не пели брачных песен;
\end{tcolorbox}
\begin{tcolorbox}
\textsubscript{64} (77-64) священники его падали от меча, и вдовы его не плакали.
\end{tcolorbox}
\begin{tcolorbox}
\textsubscript{65} (77-65) Но, как бы от сна, воспрянул Господь, как бы исполин, побежденный вином,
\end{tcolorbox}
\begin{tcolorbox}
\textsubscript{66} (77-66) и поразил врагов его в тыл, вечному сраму предал их;
\end{tcolorbox}
\begin{tcolorbox}
\textsubscript{67} (77-67) и отверг шатер Иосифов и колена Ефремова не избрал,
\end{tcolorbox}
\begin{tcolorbox}
\textsubscript{68} (77-68) а избрал колено Иудино, гору Сион, которую возлюбил.
\end{tcolorbox}
\begin{tcolorbox}
\textsubscript{69} (77-69) И устроил, как небо, святилище Свое и, как землю, утвердил его навек,
\end{tcolorbox}
\begin{tcolorbox}
\textsubscript{70} (77-70) и избрал Давида, раба Своего, и взял его от дворов овчих
\end{tcolorbox}
\begin{tcolorbox}
\textsubscript{71} (77-71) и от доящих привел его пасти народ Свой, Иакова, и наследие Свое, Израиля.
\end{tcolorbox}
\begin{tcolorbox}
\textsubscript{72} (77-72) И он пас их в чистоте сердца своего и руками мудрыми водил их.
\end{tcolorbox}
\subsection{CHAPTER 79}
\begin{tcolorbox}
\textsubscript{1} (78-1) ^^Псалом Асафа.^^ Боже! язычники пришли в наследие Твое, осквернили святый храм Твой, Иерусалим превратили в развалины;
\end{tcolorbox}
\begin{tcolorbox}
\textsubscript{2} (78-2) трупы рабов Твоих отдали на съедение птицам небесным, тела святых Твоих--зверям земным;
\end{tcolorbox}
\begin{tcolorbox}
\textsubscript{3} (78-3) пролили кровь их, как воду, вокруг Иерусалима, и некому было похоронить их.
\end{tcolorbox}
\begin{tcolorbox}
\textsubscript{4} (78-4) Мы сделались посмешищем у соседей наших, поруганием и посрамлением у окружающих нас.
\end{tcolorbox}
\begin{tcolorbox}
\textsubscript{5} (78-5) Доколе, Господи, будешь гневаться непрестанно, будет пылать ревность Твоя, как огонь?
\end{tcolorbox}
\begin{tcolorbox}
\textsubscript{6} (78-6) Пролей гнев Твой на народы, которые не знают Тебя, и на царства, которые имени Твоего не призывают,
\end{tcolorbox}
\begin{tcolorbox}
\textsubscript{7} (78-7) ибо они пожрали Иакова и жилище его опустошили.
\end{tcolorbox}
\begin{tcolorbox}
\textsubscript{8} (78-8) Не помяни нам грехов [наших] предков; скоро да предварят нас щедроты Твои, ибо мы весьма истощены.
\end{tcolorbox}
\begin{tcolorbox}
\textsubscript{9} (78-9) Помоги нам, Боже, Спаситель наш, ради славы имени Твоего; избавь нас и прости нам грехи наши ради имени Твоего.
\end{tcolorbox}
\begin{tcolorbox}
\textsubscript{10} (78-10) Для чего язычникам говорить: 'где Бог их?' Да сделается известным между язычниками пред глазами нашими отмщение за пролитую кровь рабов Твоих.
\end{tcolorbox}
\begin{tcolorbox}
\textsubscript{11} (78-11) Да придет пред лице Твое стенание узника; могуществом мышцы Твоей сохрани обреченных на смерть.
\end{tcolorbox}
\begin{tcolorbox}
\textsubscript{12} (78-12) Семикратно возврати соседям нашим в недро их поношение, которым они Тебя, Господи, поносили.
\end{tcolorbox}
\begin{tcolorbox}
\textsubscript{13} (78-13) А мы, народ Твой и Твоей пажити овцы, вечно будем славить Тебя и в род и род возвещать хвалу Тебе.
\end{tcolorbox}
\subsection{CHAPTER 80}
\begin{tcolorbox}
\textsubscript{1} (79-1) ^^Начальнику хора. На музыкальном [орудии] Шошанним-Эдуф. Псалом Асафа.^^ (79-2) Пастырь Израиля! внемли; водящий, как овец, Иосифа, восседающий на Херувимах, яви Себя.
\end{tcolorbox}
\begin{tcolorbox}
\textsubscript{2} (79-3) Пред Ефремом и Вениамином и Манассиею воздвигни силу Твою, и приди спасти нас.
\end{tcolorbox}
\begin{tcolorbox}
\textsubscript{3} (79-4) Боже! восстанови нас; да воссияет лице Твое, и спасемся!
\end{tcolorbox}
\begin{tcolorbox}
\textsubscript{4} (79-5) Господи, Боже сил! доколе будешь гневен к молитвам народа Твоего?
\end{tcolorbox}
\begin{tcolorbox}
\textsubscript{5} (79-6) Ты напитал их хлебом слезным, и напоил их слезами в большой мере,
\end{tcolorbox}
\begin{tcolorbox}
\textsubscript{6} (79-7) положил нас в пререкание соседям нашим, и враги наши издеваются [над нами].
\end{tcolorbox}
\begin{tcolorbox}
\textsubscript{7} (79-8) Боже сил! восстанови нас; да воссияет лице Твое, и спасемся!
\end{tcolorbox}
\begin{tcolorbox}
\textsubscript{8} (79-9) Из Египта перенес Ты виноградную лозу, выгнал народы и посадил ее;
\end{tcolorbox}
\begin{tcolorbox}
\textsubscript{9} (79-10) очистил для нее место, и утвердил корни ее, и она наполнила землю.
\end{tcolorbox}
\begin{tcolorbox}
\textsubscript{10} (79-11) Горы покрылись тенью ее, и ветви ее как кедры Божии;
\end{tcolorbox}
\begin{tcolorbox}
\textsubscript{11} (79-12) она пустила ветви свои до моря и отрасли свои до реки.
\end{tcolorbox}
\begin{tcolorbox}
\textsubscript{12} (79-13) Для чего разрушил Ты ограды ее, так что обрывают ее все, проходящие по пути?
\end{tcolorbox}
\begin{tcolorbox}
\textsubscript{13} (79-14) Лесной вепрь подрывает ее, и полевой зверь объедает ее.
\end{tcolorbox}
\begin{tcolorbox}
\textsubscript{14} (79-15) Боже сил! обратись же, призри с неба, и воззри, и посети виноград сей;
\end{tcolorbox}
\begin{tcolorbox}
\textsubscript{15} (79-16) охрани то, что насадила десница Твоя, и отрасли, которые Ты укрепил Себе.
\end{tcolorbox}
\begin{tcolorbox}
\textsubscript{16} (79-17) Он пожжен огнем, обсечен; от прещения лица Твоего погибнут.
\end{tcolorbox}
\begin{tcolorbox}
\textsubscript{17} (79-18) Да будет рука Твоя над мужем десницы Твоей, над сыном человеческим, которого Ты укрепил Себе,
\end{tcolorbox}
\begin{tcolorbox}
\textsubscript{18} (79-19) и мы не отступим от Тебя; оживи нас, и мы будем призывать имя Твое.
\end{tcolorbox}
\begin{tcolorbox}
\textsubscript{19} (79-20) Господи, Боже сил! восстанови нас; да воссияет лице Твое, и спасемся!
\end{tcolorbox}
\subsection{CHAPTER 81}
\begin{tcolorbox}
\textsubscript{1} (80-1) ^^Начальнику хора. На Гефском орудии. Псалом Асафа.^^ (80-2) Радостно пойте Богу, твердыне нашей; восклицайте Богу Иакова;
\end{tcolorbox}
\begin{tcolorbox}
\textsubscript{2} (80-3) возьмите псалом, дайте тимпан, сладкозвучные гусли с псалтирью;
\end{tcolorbox}
\begin{tcolorbox}
\textsubscript{3} (80-4) трубите в новомесячие трубою, в определенное время, в день праздника нашего;
\end{tcolorbox}
\begin{tcolorbox}
\textsubscript{4} (80-5) ибо это закон для Израиля, устав от Бога Иаковлева.
\end{tcolorbox}
\begin{tcolorbox}
\textsubscript{5} (80-6) Он установил это во свидетельство для Иосифа, когда он вышел из земли Египетской, где услышал звуки языка, которого не знал:
\end{tcolorbox}
\begin{tcolorbox}
\textsubscript{6} (80-7) 'Я снял с рамен его тяжести, и руки его освободились от корзин.
\end{tcolorbox}
\begin{tcolorbox}
\textsubscript{7} (80-8) В бедствии ты призвал Меня, и Я избавил тебя; из среды грома Я услышал тебя, при водах Меривы испытал тебя.
\end{tcolorbox}
\begin{tcolorbox}
\textsubscript{8} (80-9) Слушай, народ Мой, и Я буду свидетельствовать тебе: Израиль! о, если бы ты послушал Меня!
\end{tcolorbox}
\begin{tcolorbox}
\textsubscript{9} (80-10) Да не будет у тебя иного бога, и не поклоняйся богу чужеземному.
\end{tcolorbox}
\begin{tcolorbox}
\textsubscript{10} (80-11) Я Господь, Бог твой, изведший тебя из земли Египетской; открой уста твои, и Я наполню их'.
\end{tcolorbox}
\begin{tcolorbox}
\textsubscript{11} (80-12) Но народ Мой не слушал гласа Моего, и Израиль не покорялся Мне;
\end{tcolorbox}
\begin{tcolorbox}
\textsubscript{12} (80-13) потому Я оставил их упорству сердца их, пусть ходят по своим помыслам.
\end{tcolorbox}
\begin{tcolorbox}
\textsubscript{13} (80-14) О, если бы народ Мой слушал Меня и Израиль ходил Моими путями!
\end{tcolorbox}
\begin{tcolorbox}
\textsubscript{14} (80-15) Я скоро смирил бы врагов их и обратил бы руку Мою на притеснителей их:
\end{tcolorbox}
\begin{tcolorbox}
\textsubscript{15} (80-16) ненавидящие Господа раболепствовали бы им, а их благоденствие продолжалось бы навсегда;
\end{tcolorbox}
\begin{tcolorbox}
\textsubscript{16} (80-17) Я питал бы их туком пшеницы и насыщал бы их медом из скалы.
\end{tcolorbox}
\subsection{CHAPTER 82}
\begin{tcolorbox}
\textsubscript{1} (81-1) ^^Псалом Асафа.^^ Бог стал в сонме богов; среди богов произнес суд:
\end{tcolorbox}
\begin{tcolorbox}
\textsubscript{2} (81-2) доколе будете вы судить неправедно и оказывать лицеприятие нечестивым?
\end{tcolorbox}
\begin{tcolorbox}
\textsubscript{3} (81-3) Давайте суд бедному и сироте; угнетенному и нищему оказывайте справедливость;
\end{tcolorbox}
\begin{tcolorbox}
\textsubscript{4} (81-4) избавляйте бедного и нищего; исторгайте [его] из руки нечестивых.
\end{tcolorbox}
\begin{tcolorbox}
\textsubscript{5} (81-5) Не знают, не разумеют, во тьме ходят; все основания земли колеблются.
\end{tcolorbox}
\begin{tcolorbox}
\textsubscript{6} (81-6) Я сказал: вы--боги, и сыны Всевышнего--все вы;
\end{tcolorbox}
\begin{tcolorbox}
\textsubscript{7} (81-7) но вы умрете, как человеки, и падете, как всякий из князей.
\end{tcolorbox}
\begin{tcolorbox}
\textsubscript{8} (81-8) Восстань, Боже, суди землю, ибо Ты наследуешь все народы.
\end{tcolorbox}
\subsection{CHAPTER 83}
\begin{tcolorbox}
\textsubscript{1} (82-1) ^^Песнь. Псалом Асафа.^^ (82-2) Боже! Не премолчи, не безмолвствуй и не оставайся в покое, Боже,
\end{tcolorbox}
\begin{tcolorbox}
\textsubscript{2} (82-3) ибо вот, враги Твои шумят, и ненавидящие Тебя подняли голову;
\end{tcolorbox}
\begin{tcolorbox}
\textsubscript{3} (82-4) против народа Твоего составили коварный умысел и совещаются против хранимых Тобою;
\end{tcolorbox}
\begin{tcolorbox}
\textsubscript{4} (82-5) сказали: 'пойдем и истребим их из народов, чтобы не вспоминалось более имя Израиля.'
\end{tcolorbox}
\begin{tcolorbox}
\textsubscript{5} (82-6) Сговорились единодушно, заключили против Тебя союз:
\end{tcolorbox}
\begin{tcolorbox}
\textsubscript{6} (82-7) селения Едомовы и Измаильтяне, Моав и Агаряне,
\end{tcolorbox}
\begin{tcolorbox}
\textsubscript{7} (82-8) Гевал и Аммон и Амалик, Филистимляне с жителями Тира.
\end{tcolorbox}
\begin{tcolorbox}
\textsubscript{8} (82-9) И Ассур пристал к ним: они стали мышцею для сынов Лотовых.
\end{tcolorbox}
\begin{tcolorbox}
\textsubscript{9} (82-10) Сделай им то же, что Мадиаму, что Сисаре, что Иавину у потока Киссона,
\end{tcolorbox}
\begin{tcolorbox}
\textsubscript{10} (82-11) которые истреблены в Аендоре, сделались навозом для земли.
\end{tcolorbox}
\begin{tcolorbox}
\textsubscript{11} (82-12) Поступи с ними, с князьями их, как с Оривом и Зивом и со всеми вождями их, как с Зевеем и Салманом,
\end{tcolorbox}
\begin{tcolorbox}
\textsubscript{12} (82-13) которые говорили: 'возьмем себе во владение селения Божии'.
\end{tcolorbox}
\begin{tcolorbox}
\textsubscript{13} (82-14) Боже мой! Да будут они, как пыль в вихре, как солома перед ветром.
\end{tcolorbox}
\begin{tcolorbox}
\textsubscript{14} (82-15) Как огонь сжигает лес, и как пламя опаляет горы,
\end{tcolorbox}
\begin{tcolorbox}
\textsubscript{15} (82-16) так погони их бурею Твоею и вихрем Твоим приведи их в смятение;
\end{tcolorbox}
\begin{tcolorbox}
\textsubscript{16} (82-17) исполни лица их бесчестием, чтобы они взыскали имя Твое, Господи!
\end{tcolorbox}
\begin{tcolorbox}
\textsubscript{17} (82-18) Да постыдятся и смятутся на веки, да посрамятся и погибнут,
\end{tcolorbox}
\begin{tcolorbox}
\textsubscript{18} (82-19) и да познают, что Ты, Которого одного имя Господь, Всевышний над всею землею.
\end{tcolorbox}
\subsection{CHAPTER 84}
\begin{tcolorbox}
\textsubscript{1} (83-1) ^^Начальнику хора. На Гефском [орудии]. Кореевых сынов. Псалом.^^ (83-2) Как вожделенны жилища Твои, Господи сил!
\end{tcolorbox}
\begin{tcolorbox}
\textsubscript{2} (83-3) Истомилась душа моя, желая во дворы Господни; сердце мое и плоть моя восторгаются к Богу живому.
\end{tcolorbox}
\begin{tcolorbox}
\textsubscript{3} (83-4) И птичка находит себе жилье, и ласточка гнездо себе, где положить птенцов своих, у алтарей Твоих, Господи сил, Царь мой и Бог мой!
\end{tcolorbox}
\begin{tcolorbox}
\textsubscript{4} (83-5) Блаженны живущие в доме Твоем: они непрестанно будут восхвалять Тебя.
\end{tcolorbox}
\begin{tcolorbox}
\textsubscript{5} (83-6) Блажен человек, которого сила в Тебе и у которого в сердце стези направлены [к Тебе].
\end{tcolorbox}
\begin{tcolorbox}
\textsubscript{6} (83-7) Проходя долиною плача, они открывают в ней источники, и дождь покрывает ее благословением;
\end{tcolorbox}
\begin{tcolorbox}
\textsubscript{7} (83-8) приходят от силы в силу, являются пред Богом на Сионе.
\end{tcolorbox}
\begin{tcolorbox}
\textsubscript{8} (83-9) Господи, Боже сил! Услышь молитву мою, внемли, Боже Иаковлев!
\end{tcolorbox}
\begin{tcolorbox}
\textsubscript{9} (83-10) Боже, защитник наш! Приникни и призри на лице помазанника Твоего.
\end{tcolorbox}
\begin{tcolorbox}
\textsubscript{10} (83-11) Ибо один день во дворах Твоих лучше тысячи. Желаю лучше быть у порога в доме Божием, нежели жить в шатрах нечестия.
\end{tcolorbox}
\begin{tcolorbox}
\textsubscript{11} (83-12) Ибо Господь Бог есть солнце и щит, Господь дает благодать и славу; ходящих в непорочности Он не лишает благ.
\end{tcolorbox}
\begin{tcolorbox}
\textsubscript{12} (83-13) Господи сил! Блажен человек, уповающий на Тебя!
\end{tcolorbox}
\subsection{CHAPTER 85}
\begin{tcolorbox}
\textsubscript{1} (84-1) ^^Начальнику хора. Кореевых сынов. Псалом.^^ (84-2) Господи! Ты умилосердился к земле Твоей, возвратил плен Иакова;
\end{tcolorbox}
\begin{tcolorbox}
\textsubscript{2} (84-3) простил беззаконие народа Твоего, покрыл все грехи его,
\end{tcolorbox}
\begin{tcolorbox}
\textsubscript{3} (84-4) отъял всю ярость Твою, отвратил лютость гнева Твоего.
\end{tcolorbox}
\begin{tcolorbox}
\textsubscript{4} (84-5) Восстанови нас, Боже спасения нашего, и прекрати негодование Твое на нас.
\end{tcolorbox}
\begin{tcolorbox}
\textsubscript{5} (84-6) Неужели вечно будешь гневаться на нас, прострешь гнев Твой от рода в род?
\end{tcolorbox}
\begin{tcolorbox}
\textsubscript{6} (84-7) Неужели снова не оживишь нас, чтобы народ Твой возрадовался о Тебе?
\end{tcolorbox}
\begin{tcolorbox}
\textsubscript{7} (84-8) Яви нам, Господи, милость Твою, и спасение Твое даруй нам.
\end{tcolorbox}
\begin{tcolorbox}
\textsubscript{8} (84-9) Послушаю, что скажет Господь Бог. Он скажет мир народу Своему и избранным Своим, но да не впадут они снова в безрассудство.
\end{tcolorbox}
\begin{tcolorbox}
\textsubscript{9} (84-10) Так, близко к боящимся Его спасение Его, чтобы обитала слава в земле нашей!
\end{tcolorbox}
\begin{tcolorbox}
\textsubscript{10} (84-11) Милость и истина сретятся, правда и мир облобызаются;
\end{tcolorbox}
\begin{tcolorbox}
\textsubscript{11} (84-12) истина возникнет из земли, и правда приникнет с небес;
\end{tcolorbox}
\begin{tcolorbox}
\textsubscript{12} (84-13) и Господь даст благо, и земля наша даст плод свой;
\end{tcolorbox}
\begin{tcolorbox}
\textsubscript{13} (84-14) правда пойдет пред Ним и поставит на путь стопы свои.
\end{tcolorbox}
\subsection{CHAPTER 86}
\begin{tcolorbox}
\textsubscript{1} (85-1) ^^Молитва Давида.^^ Приклони, Господи, ухо Твое и услышь меня, ибо я беден и нищ.
\end{tcolorbox}
\begin{tcolorbox}
\textsubscript{2} (85-2) Сохрани душу мою, ибо я благоговею пред Тобою; спаси, Боже мой, раба Твоего, уповающего на Тебя.
\end{tcolorbox}
\begin{tcolorbox}
\textsubscript{3} (85-3) Помилуй меня, Господи, ибо к Тебе взываю каждый день.
\end{tcolorbox}
\begin{tcolorbox}
\textsubscript{4} (85-4) Возвесели душу раба Твоего, ибо к Тебе, Господи, возношу душу мою,
\end{tcolorbox}
\begin{tcolorbox}
\textsubscript{5} (85-5) ибо Ты, Господи, благ и милосерд и многомилостив ко всем, призывающим Тебя.
\end{tcolorbox}
\begin{tcolorbox}
\textsubscript{6} (85-6) Услышь, Господи, молитву мою и внемли гласу моления моего.
\end{tcolorbox}
\begin{tcolorbox}
\textsubscript{7} (85-7) В день скорби моей взываю к Тебе, потому что Ты услышишь меня.
\end{tcolorbox}
\begin{tcolorbox}
\textsubscript{8} (85-8) Нет между богами, как Ты, Господи, и нет дел, как Твои.
\end{tcolorbox}
\begin{tcolorbox}
\textsubscript{9} (85-9) Все народы, Тобою сотворенные, приидут и поклонятся пред Тобою, Господи, и прославят имя Твое,
\end{tcolorbox}
\begin{tcolorbox}
\textsubscript{10} (85-10) ибо Ты велик и творишь чудеса, --Ты, Боже, един Ты.
\end{tcolorbox}
\begin{tcolorbox}
\textsubscript{11} (85-11) Наставь меня, Господи, на путь Твой, и буду ходить в истине Твоей; утверди сердце мое в страхе имени Твоего.
\end{tcolorbox}
\begin{tcolorbox}
\textsubscript{12} (85-12) Буду восхвалять Тебя, Господи, Боже мой, всем сердцем моим и славить имя Твое вечно,
\end{tcolorbox}
\begin{tcolorbox}
\textsubscript{13} (85-13) ибо велика милость Твоя ко мне: Ты избавил душу мою от ада преисподнего.
\end{tcolorbox}
\begin{tcolorbox}
\textsubscript{14} (85-14) Боже! гордые восстали на меня, и скопище мятежников ищет души моей: не представляют они Тебя пред собою.
\end{tcolorbox}
\begin{tcolorbox}
\textsubscript{15} (85-15) Но Ты, Господи, Боже щедрый и благосердный, долготерпеливый и многомилостивый и истинный,
\end{tcolorbox}
\begin{tcolorbox}
\textsubscript{16} (85-16) призри на меня и помилуй меня; даруй крепость Твою рабу Твоему, и спаси сына рабы Твоей;
\end{tcolorbox}
\begin{tcolorbox}
\textsubscript{17} (85-17) покажи на мне знамение во благо, да видят ненавидящие меня и устыдятся, потому что Ты, Господи, помог мне и утешил меня.
\end{tcolorbox}
\subsection{CHAPTER 87}
\begin{tcolorbox}
\textsubscript{1} (86-1) Сынов Кореевых. Псалом. Песнь.
\end{tcolorbox}
\begin{tcolorbox}
\textsubscript{2} (86-2) Основание его на горах святых. Господь любит врата Сиона более всех селений Иакова.
\end{tcolorbox}
\begin{tcolorbox}
\textsubscript{3} (86-3) Славное возвещается о тебе, град Божий!
\end{tcolorbox}
\begin{tcolorbox}
\textsubscript{4} (86-4) Упомяну знающим меня о Рааве и Вавилоне; вот Филистимляне и Тир с Ефиопиею, --[скажут]: 'такой-то родился там'.
\end{tcolorbox}
\begin{tcolorbox}
\textsubscript{5} (86-5) О Сионе же будут говорить: 'такой-то и такой-то муж родился в нем, и Сам Всевышний укрепил его'.
\end{tcolorbox}
\begin{tcolorbox}
\textsubscript{6} (86-6) Господь в переписи народов напишет: 'такой-то родился там'.
\end{tcolorbox}
\begin{tcolorbox}
\textsubscript{7} (86-7) И поющие и играющие, --все источники мои в тебе.
\end{tcolorbox}
\subsection{CHAPTER 88}
\begin{tcolorbox}
\textsubscript{1} (87-1) ^^Песнь. Псалом, Сынов Кореевых. Начальнику хора на Махалаф, для пения. Учение Емана Езрахита.^^ (87-2) Господи, Боже спасения моего! днем вопию и ночью пред Тобою:
\end{tcolorbox}
\begin{tcolorbox}
\textsubscript{2} (87-3) да внидет пред лице Твое молитва моя; приклони ухо Твое к молению моему,
\end{tcolorbox}
\begin{tcolorbox}
\textsubscript{3} (87-4) ибо душа моя насытилась бедствиями, и жизнь моя приблизилась к преисподней.
\end{tcolorbox}
\begin{tcolorbox}
\textsubscript{4} (87-5) Я сравнялся с нисходящими в могилу; я стал, как человек без силы,
\end{tcolorbox}
\begin{tcolorbox}
\textsubscript{5} (87-6) между мертвыми брошенный, --как убитые, лежащие во гробе, о которых Ты уже не вспоминаешь и которые от руки Твоей отринуты.
\end{tcolorbox}
\begin{tcolorbox}
\textsubscript{6} (87-7) Ты положил меня в ров преисподний, во мрак, в бездну.
\end{tcolorbox}
\begin{tcolorbox}
\textsubscript{7} (87-8) Отяготела на мне ярость Твоя, и всеми волнами Твоими Ты поразил [меня].
\end{tcolorbox}
\begin{tcolorbox}
\textsubscript{8} (87-9) Ты удалил от меня знакомых моих, сделал меня отвратительным для них; я заключен, и не могу выйти.
\end{tcolorbox}
\begin{tcolorbox}
\textsubscript{9} (87-10) Око мое истомилось от горести: весь день я взывал к Тебе, Господи, простирал к Тебе руки мои.
\end{tcolorbox}
\begin{tcolorbox}
\textsubscript{10} (87-11) Разве над мертвыми Ты сотворишь чудо? Разве мертвые встанут и будут славить Тебя?
\end{tcolorbox}
\begin{tcolorbox}
\textsubscript{11} (87-12) или во гробе будет возвещаема милость Твоя, и истина Твоя--в месте тления?
\end{tcolorbox}
\begin{tcolorbox}
\textsubscript{12} (87-13) разве во мраке познают чудеса Твои, и в земле забвения--правду Твою?
\end{tcolorbox}
\begin{tcolorbox}
\textsubscript{13} (87-14) Но я к Тебе, Господи, взываю, и рано утром молитва моя предваряет Тебя.
\end{tcolorbox}
\begin{tcolorbox}
\textsubscript{14} (87-15) Для чего, Господи, отреваешь душу мою, скрываешь лице Твое от меня?
\end{tcolorbox}
\begin{tcolorbox}
\textsubscript{15} (87-16) Я несчастен и истаеваю с юности; несу ужасы Твои и изнемогаю.
\end{tcolorbox}
\begin{tcolorbox}
\textsubscript{16} (87-17) Надо мною прошла ярость Твоя, устрашения Твои сокрушили меня,
\end{tcolorbox}
\begin{tcolorbox}
\textsubscript{17} (87-18) всякий день окружают меня, как вода: облегают меня все вместе.
\end{tcolorbox}
\begin{tcolorbox}
\textsubscript{18} (87-19) Ты удалил от меня друга и искреннего; знакомых моих не видно.
\end{tcolorbox}
\subsection{CHAPTER 89}
\begin{tcolorbox}
\textsubscript{1} (88-1) ^^Учение Ефама Езрахита.^^ (88-2) Милости [Твои], Господи, буду петь вечно, в род и род возвещать истину Твою устами моими.
\end{tcolorbox}
\begin{tcolorbox}
\textsubscript{2} (88-3) Ибо говорю: навек основана милость, на небесах утвердил Ты истину Твою, [когда сказал]:
\end{tcolorbox}
\begin{tcolorbox}
\textsubscript{3} (88-4) 'Я поставил завет с избранным Моим, клялся Давиду, рабу Моему:
\end{tcolorbox}
\begin{tcolorbox}
\textsubscript{4} (88-5) навек утвержу семя твое, в род и род устрою престол твой'.
\end{tcolorbox}
\begin{tcolorbox}
\textsubscript{5} (88-6) И небеса прославят чудные дела Твои, Господи, и истину Твою в собрании святых.
\end{tcolorbox}
\begin{tcolorbox}
\textsubscript{6} (88-7) Ибо кто на небесах сравнится с Господом? кто между сынами Божиими уподобится Господу?
\end{tcolorbox}
\begin{tcolorbox}
\textsubscript{7} (88-8) Страшен Бог в великом сонме святых, страшен Он для всех окружающих Его.
\end{tcolorbox}
\begin{tcolorbox}
\textsubscript{8} (88-9) Господи, Боже сил! кто силен, как Ты, Господи? И истина Твоя окрест Тебя.
\end{tcolorbox}
\begin{tcolorbox}
\textsubscript{9} (88-10) Ты владычествуешь над яростью моря: когда воздымаются волны его, Ты укрощаешь их.
\end{tcolorbox}
\begin{tcolorbox}
\textsubscript{10} (88-11) Ты низложил Раава, как пораженного; крепкою мышцею Твоею рассеял врагов Твоих.
\end{tcolorbox}
\begin{tcolorbox}
\textsubscript{11} (88-12) Твои небеса и Твоя земля; вселенную и что наполняет ее, Ты основал.
\end{tcolorbox}
\begin{tcolorbox}
\textsubscript{12} (88-13) Север и юг Ты сотворил; Фавор и Ермон о имени Твоем радуются.
\end{tcolorbox}
\begin{tcolorbox}
\textsubscript{13} (88-14) Крепка мышца Твоя, сильна рука Твоя, высока десница Твоя!
\end{tcolorbox}
\begin{tcolorbox}
\textsubscript{14} (88-15) Правосудие и правота--основание престола Твоего; милость и истина предходят пред лицем Твоим.
\end{tcolorbox}
\begin{tcolorbox}
\textsubscript{15} (88-16) Блажен народ, знающий трубный зов! Они ходят во свете лица Твоего, Господи,
\end{tcolorbox}
\begin{tcolorbox}
\textsubscript{16} (88-17) о имени Твоем радуются весь день и правдою Твоею возносятся,
\end{tcolorbox}
\begin{tcolorbox}
\textsubscript{17} (88-18) ибо Ты украшение силы их, и благоволением Твоим возвышается рог наш.
\end{tcolorbox}
\begin{tcolorbox}
\textsubscript{18} (88-19) От Господа--щит наш, и от Святаго Израилева--царь наш.
\end{tcolorbox}
\begin{tcolorbox}
\textsubscript{19} (88-20) Некогда говорил Ты в видении святому Твоему, и сказал: 'Я оказал помощь мужественному, вознес избранного из народа.
\end{tcolorbox}
\begin{tcolorbox}
\textsubscript{20} (88-21) Я обрел Давида, раба Моего, святым елеем Моим помазал его.
\end{tcolorbox}
\begin{tcolorbox}
\textsubscript{21} (88-22) Рука Моя пребудет с ним, и мышца Моя укрепит его.
\end{tcolorbox}
\begin{tcolorbox}
\textsubscript{22} (88-23) Враг не превозможет его, и сын беззакония не притеснит его.
\end{tcolorbox}
\begin{tcolorbox}
\textsubscript{23} (88-24) Сокрушу пред ним врагов его и поражу ненавидящих его.
\end{tcolorbox}
\begin{tcolorbox}
\textsubscript{24} (88-25) И истина Моя и милость Моя с ним, и Моим именем возвысится рог его.
\end{tcolorbox}
\begin{tcolorbox}
\textsubscript{25} (88-26) И положу на море руку его, и на реки--десницу его.
\end{tcolorbox}
\begin{tcolorbox}
\textsubscript{26} (88-27) Он будет звать Меня: Ты отец мой, Бог мой и твердыня спасения моего.
\end{tcolorbox}
\begin{tcolorbox}
\textsubscript{27} (88-28) И Я сделаю его первенцем, превыше царей земли,
\end{tcolorbox}
\begin{tcolorbox}
\textsubscript{28} (88-29) вовек сохраню ему милость Мою, и завет Мой с ним будет верен.
\end{tcolorbox}
\begin{tcolorbox}
\textsubscript{29} (88-30) И продолжу вовек семя его, и престол его--как дни неба.
\end{tcolorbox}
\begin{tcolorbox}
\textsubscript{30} (88-31) Если сыновья его оставят закон Мой и не будут ходить по заповедям Моим;
\end{tcolorbox}
\begin{tcolorbox}
\textsubscript{31} (88-32) если нарушат уставы Мои и повелений Моих не сохранят:
\end{tcolorbox}
\begin{tcolorbox}
\textsubscript{32} (88-33) посещу жезлом беззаконие их, и ударами--неправду их;
\end{tcolorbox}
\begin{tcolorbox}
\textsubscript{33} (88-34) милости же Моей не отниму от него, и не изменю истины Моей.
\end{tcolorbox}
\begin{tcolorbox}
\textsubscript{34} (88-35) Не нарушу завета Моего, и не переменю того, что вышло из уст Моих.
\end{tcolorbox}
\begin{tcolorbox}
\textsubscript{35} (88-36) Однажды Я поклялся святостью Моею: солгу ли Давиду?
\end{tcolorbox}
\begin{tcolorbox}
\textsubscript{36} (88-37) Семя его пребудет вечно, и престол его, как солнце, предо Мною,
\end{tcolorbox}
\begin{tcolorbox}
\textsubscript{37} (88-38) вовек будет тверд, как луна, и верный свидетель на небесах'.
\end{tcolorbox}
\begin{tcolorbox}
\textsubscript{38} (88-39) Но [ныне] Ты отринул и презрел, прогневался на помазанника Твоего;
\end{tcolorbox}
\begin{tcolorbox}
\textsubscript{39} (88-40) пренебрег завет с рабом Твоим, поверг на землю венец его;
\end{tcolorbox}
\begin{tcolorbox}
\textsubscript{40} (88-41) разрушил все ограды его, превратил в развалины крепости его.
\end{tcolorbox}
\begin{tcolorbox}
\textsubscript{41} (88-42) Расхищают его все проходящие путем; он сделался посмешищем у соседей своих.
\end{tcolorbox}
\begin{tcolorbox}
\textsubscript{42} (88-43) Ты возвысил десницу противников его, обрадовал всех врагов его;
\end{tcolorbox}
\begin{tcolorbox}
\textsubscript{43} (88-44) Ты обратил назад острие меча его и не укрепил его на брани;
\end{tcolorbox}
\begin{tcolorbox}
\textsubscript{44} (88-45) отнял у него блеск и престол его поверг на землю;
\end{tcolorbox}
\begin{tcolorbox}
\textsubscript{45} (88-46) сократил дни юности его и покрыл его стыдом.
\end{tcolorbox}
\begin{tcolorbox}
\textsubscript{46} (88-47) Доколе, Господи, будешь скрываться непрестанно, будет пылать ярость Твоя, как огонь?
\end{tcolorbox}
\begin{tcolorbox}
\textsubscript{47} (88-48) Вспомни, какой мой век: на какую суету сотворил Ты всех сынов человеческих?
\end{tcolorbox}
\begin{tcolorbox}
\textsubscript{48} (88-49) Кто из людей жил--и не видел смерти, избавил душу свою от руки преисподней?
\end{tcolorbox}
\begin{tcolorbox}
\textsubscript{49} (88-50) Где прежние милости Твои, Господи? Ты клялся Давиду истиною Твоею.
\end{tcolorbox}
\begin{tcolorbox}
\textsubscript{50} (88-51) Вспомни, Господи, поругание рабов Твоих, которое я ношу в недре моем от всех сильных народов;
\end{tcolorbox}
\begin{tcolorbox}
\textsubscript{51} (88-52) как поносят враги Твои, Господи, как бесславят следы помазанника Твоего.
\end{tcolorbox}
\begin{tcolorbox}
\textsubscript{52} (88-53) Благословен Господь вовек! Аминь, аминь.
\end{tcolorbox}
\subsection{CHAPTER 90}
\begin{tcolorbox}
\textsubscript{1} (89-1) Молитва Моисея, человека Божия.
\end{tcolorbox}
\begin{tcolorbox}
\textsubscript{2} (89-2) Господи! Ты нам прибежище в род и род.
\end{tcolorbox}
\begin{tcolorbox}
\textsubscript{3} (89-3) Прежде нежели родились горы, и Ты образовал землю и вселенную, и от века и до века Ты--Бог.
\end{tcolorbox}
\begin{tcolorbox}
\textsubscript{4} (89-4) Ты возвращаешь человека в тление и говоришь: 'возвратитесь, сыны человеческие!'
\end{tcolorbox}
\begin{tcolorbox}
\textsubscript{5} (89-5) Ибо пред очами Твоими тысяча лет, как день вчерашний, когда он прошел, и [как] стража в ночи.
\end{tcolorbox}
\begin{tcolorbox}
\textsubscript{6} (89-6) Ты [как] наводнением уносишь их; они--[как] сон, как трава, которая утром вырастает, утром цветет и зеленеет, вечером подсекается и засыхает;
\end{tcolorbox}
\begin{tcolorbox}
\textsubscript{7} (89-7) ибо мы исчезаем от гнева Твоего и от ярости Твоей мы в смятении.
\end{tcolorbox}
\begin{tcolorbox}
\textsubscript{8} (89-8) Ты положил беззакония наши пред Тобою и тайное наше пред светом лица Твоего.
\end{tcolorbox}
\begin{tcolorbox}
\textsubscript{9} (89-9) Все дни наши прошли во гневе Твоем; мы теряем лета наши, как звук.
\end{tcolorbox}
\begin{tcolorbox}
\textsubscript{10} (89-10) Дней лет наших--семьдесят лет, а при большей крепости--восемьдесят лет; и самая лучшая пора их--труд и болезнь, ибо проходят быстро, и мы летим.
\end{tcolorbox}
\begin{tcolorbox}
\textsubscript{11} (89-11) Кто знает силу гнева Твоего, и ярость Твою по мере страха Твоего?
\end{tcolorbox}
\begin{tcolorbox}
\textsubscript{12} (89-12) Научи нас так счислять дни наши, чтобы нам приобрести сердце мудрое.
\end{tcolorbox}
\begin{tcolorbox}
\textsubscript{13} (89-13) Обратись, Господи! Доколе? Умилосердись над рабами Твоими.
\end{tcolorbox}
\begin{tcolorbox}
\textsubscript{14} (89-14) Рано насыти нас милостью Твоею, и мы будем радоваться и веселиться во все дни наши.
\end{tcolorbox}
\begin{tcolorbox}
\textsubscript{15} (89-15) Возвесели нас за дни, [в которые] Ты поражал нас, за лета, [в которые] мы видели бедствие.
\end{tcolorbox}
\begin{tcolorbox}
\textsubscript{16} (89-16) Да явится на рабах Твоих дело Твое и на сынах их слава Твоя;
\end{tcolorbox}
\begin{tcolorbox}
\textsubscript{17} (89-17) и да будет благоволение Господа Бога нашего на нас, и в деле рук наших споспешествуй нам, в деле рук наших споспешествуй.
\end{tcolorbox}
\subsection{CHAPTER 91}
\begin{tcolorbox}
\textsubscript{1} (90-1) Живущий под кровом Всевышнего под сенью Всемогущего покоится,
\end{tcolorbox}
\begin{tcolorbox}
\textsubscript{2} (90-2) говорит Господу: 'прибежище мое и защита моя, Бог мой, на Которого я уповаю!'
\end{tcolorbox}
\begin{tcolorbox}
\textsubscript{3} (90-3) Он избавит тебя от сети ловца, от гибельной язвы,
\end{tcolorbox}
\begin{tcolorbox}
\textsubscript{4} (90-4) перьями Своими осенит тебя, и под крыльями Его будешь безопасен; щит и ограждение--истина Его.
\end{tcolorbox}
\begin{tcolorbox}
\textsubscript{5} (90-5) Не убоишься ужасов в ночи, стрелы, летящей днем,
\end{tcolorbox}
\begin{tcolorbox}
\textsubscript{6} (90-6) язвы, ходящей во мраке, заразы, опустошающей в полдень.
\end{tcolorbox}
\begin{tcolorbox}
\textsubscript{7} (90-7) Падут подле тебя тысяча и десять тысяч одесную тебя; но к тебе не приблизится:
\end{tcolorbox}
\begin{tcolorbox}
\textsubscript{8} (90-8) только смотреть будешь очами твоими и видеть возмездие нечестивым.
\end{tcolorbox}
\begin{tcolorbox}
\textsubscript{9} (90-9) Ибо ты [сказал]: 'Господь--упование мое'; Всевышнего избрал ты прибежищем твоим;
\end{tcolorbox}
\begin{tcolorbox}
\textsubscript{10} (90-10) не приключится тебе зло, и язва не приблизится к жилищу твоему;
\end{tcolorbox}
\begin{tcolorbox}
\textsubscript{11} (90-11) ибо Ангелам Своим заповедает о тебе--охранять тебя на всех путях твоих:
\end{tcolorbox}
\begin{tcolorbox}
\textsubscript{12} (90-12) на руках понесут тебя, да не преткнешься о камень ногою твоею;
\end{tcolorbox}
\begin{tcolorbox}
\textsubscript{13} (90-13) на аспида и василиска наступишь; попирать будешь льва и дракона.
\end{tcolorbox}
\begin{tcolorbox}
\textsubscript{14} (90-14) 'За то, что он возлюбил Меня, избавлю его; защищу его, потому что он познал имя Мое.
\end{tcolorbox}
\begin{tcolorbox}
\textsubscript{15} (90-15) Воззовет ко Мне, и услышу его; с ним Я в скорби; избавлю его и прославлю его,
\end{tcolorbox}
\begin{tcolorbox}
\textsubscript{16} (90-16) долготою дней насыщу его, и явлю ему спасение Мое'.
\end{tcolorbox}
\subsection{CHAPTER 92}
\begin{tcolorbox}
\textsubscript{1} (91-1) ^^Псалом. Песнь на день субботний.^^ (91-2) Благо есть славить Господа и петь имени Твоему, Всевышний,
\end{tcolorbox}
\begin{tcolorbox}
\textsubscript{2} (91-3) возвещать утром милость Твою и истину Твою в ночи,
\end{tcolorbox}
\begin{tcolorbox}
\textsubscript{3} (91-4) на десятиструнном и псалтири, с песнью на гуслях.
\end{tcolorbox}
\begin{tcolorbox}
\textsubscript{4} (91-5) Ибо Ты возвеселил меня, Господи, творением Твоим: я восхищаюсь делами рук Твоих.
\end{tcolorbox}
\begin{tcolorbox}
\textsubscript{5} (91-6) Как велики дела Твои, Господи! дивно глубоки помышления Твои!
\end{tcolorbox}
\begin{tcolorbox}
\textsubscript{6} (91-7) Человек несмысленный не знает, и невежда не разумеет того.
\end{tcolorbox}
\begin{tcolorbox}
\textsubscript{7} (91-8) Тогда как нечестивые возникают, как трава, и делающие беззаконие цветут, чтобы исчезнуть на веки, --
\end{tcolorbox}
\begin{tcolorbox}
\textsubscript{8} (91-9) Ты, Господи, высок во веки!
\end{tcolorbox}
\begin{tcolorbox}
\textsubscript{9} (91-10) Ибо вот, враги Твои, Господи, --вот, враги Твои гибнут, и рассыпаются все делающие беззаконие;
\end{tcolorbox}
\begin{tcolorbox}
\textsubscript{10} (91-11) а мой рог Ты возносишь, как рог единорога, и я умащен свежим елеем;
\end{tcolorbox}
\begin{tcolorbox}
\textsubscript{11} (91-12) и око мое смотрит на врагов моих, и уши мои слышат о восстающих на меня злодеях.
\end{tcolorbox}
\begin{tcolorbox}
\textsubscript{12} (91-13) Праведник цветет, как пальма, возвышается подобно кедру на Ливане.
\end{tcolorbox}
\begin{tcolorbox}
\textsubscript{13} (91-14) Насажденные в доме Господнем, они цветут во дворах Бога нашего;
\end{tcolorbox}
\begin{tcolorbox}
\textsubscript{14} (91-15) они и в старости плодовиты, сочны и свежи,
\end{tcolorbox}
\begin{tcolorbox}
\textsubscript{15} (91-16) чтобы возвещать, что праведен Господь, твердыня моя, и нет неправды в Нем.
\end{tcolorbox}
\subsection{CHAPTER 93}
\begin{tcolorbox}
\textsubscript{1} (92-1) Господь царствует; Он облечен величием, облечен Господь могуществом [и] препоясан: потому вселенная тверда, не подвигнется.
\end{tcolorbox}
\begin{tcolorbox}
\textsubscript{2} (92-2) Престол Твой утвержден искони: Ты--от века.
\end{tcolorbox}
\begin{tcolorbox}
\textsubscript{3} (92-3) Возвышают реки, Господи, возвышают реки голос свой, возвышают реки волны свои.
\end{tcolorbox}
\begin{tcolorbox}
\textsubscript{4} (92-4) Но паче шума вод многих, сильных волн морских, силен в вышних Господь.
\end{tcolorbox}
\begin{tcolorbox}
\textsubscript{5} (92-5) Откровения Твои несомненно верны. Дому Твоему, Господи, принадлежит святость на долгие дни.
\end{tcolorbox}
\subsection{CHAPTER 94}
\begin{tcolorbox}
\textsubscript{1} (93-1) Боже отмщений, Господи, Боже отмщений, яви Себя!
\end{tcolorbox}
\begin{tcolorbox}
\textsubscript{2} (93-2) Восстань, Судия земли, воздай возмездие гордым.
\end{tcolorbox}
\begin{tcolorbox}
\textsubscript{3} (93-3) Доколе, Господи, нечестивые, доколе нечестивые торжествовать будут?
\end{tcolorbox}
\begin{tcolorbox}
\textsubscript{4} (93-4) Они изрыгают дерзкие речи; величаются все делающие беззаконие;
\end{tcolorbox}
\begin{tcolorbox}
\textsubscript{5} (93-5) попирают народ Твой, Господи, угнетают наследие Твое;
\end{tcolorbox}
\begin{tcolorbox}
\textsubscript{6} (93-6) вдову и пришельца убивают, и сирот умерщвляют
\end{tcolorbox}
\begin{tcolorbox}
\textsubscript{7} (93-7) и говорят: 'не увидит Господь, и не узнает Бог Иаковлев'.
\end{tcolorbox}
\begin{tcolorbox}
\textsubscript{8} (93-8) Образумьтесь, бессмысленные люди! когда вы будете умны, невежды?
\end{tcolorbox}
\begin{tcolorbox}
\textsubscript{9} (93-9) Насадивший ухо не услышит ли? и образовавший глаз не увидит ли?
\end{tcolorbox}
\begin{tcolorbox}
\textsubscript{10} (93-10) Вразумляющий народы неужели не обличит, --Тот, Кто учит человека разумению?
\end{tcolorbox}
\begin{tcolorbox}
\textsubscript{11} (93-11) Господь знает мысли человеческие, что они суетны.
\end{tcolorbox}
\begin{tcolorbox}
\textsubscript{12} (93-12) Блажен человек, которого вразумляешь Ты, Господи, и наставляешь законом Твоим,
\end{tcolorbox}
\begin{tcolorbox}
\textsubscript{13} (93-13) чтобы дать ему покой в бедственные дни, доколе нечестивому выроется яма!
\end{tcolorbox}
\begin{tcolorbox}
\textsubscript{14} (93-14) Ибо не отринет Господь народа Своего и не оставит наследия Своего.
\end{tcolorbox}
\begin{tcolorbox}
\textsubscript{15} (93-15) Ибо суд возвратится к правде, и за ним [последуют] все правые сердцем.
\end{tcolorbox}
\begin{tcolorbox}
\textsubscript{16} (93-16) Кто восстанет за меня против злодеев? кто станет за меня против делающих беззаконие?
\end{tcolorbox}
\begin{tcolorbox}
\textsubscript{17} (93-17) Если бы не Господь был мне помощником, вскоре вселилась бы душа моя в [страну] молчания.
\end{tcolorbox}
\begin{tcolorbox}
\textsubscript{18} (93-18) Когда я говорил: 'колеблется нога моя', --милость Твоя, Господи, поддерживала меня.
\end{tcolorbox}
\begin{tcolorbox}
\textsubscript{19} (93-19) При умножении скорбей моих в сердце моем, утешения Твои услаждают душу мою.
\end{tcolorbox}
\begin{tcolorbox}
\textsubscript{20} (93-20) Станет ли близ Тебя седалище губителей, умышляющих насилие вопреки закону?
\end{tcolorbox}
\begin{tcolorbox}
\textsubscript{21} (93-21) Толпою устремляются они на душу праведника и осуждают кровь неповинную.
\end{tcolorbox}
\begin{tcolorbox}
\textsubscript{22} (93-22) Но Господь--защита моя, и Бог мой--твердыня убежища моего,
\end{tcolorbox}
\begin{tcolorbox}
\textsubscript{23} (93-23) и обратит на них беззаконие их, и злодейством их истребит их, истребит их Господь Бог наш.
\end{tcolorbox}
\subsection{CHAPTER 95}
\begin{tcolorbox}
\textsubscript{1} (94-1) Приидите, воспоем Господу, воскликнем твердыне спасения нашего;
\end{tcolorbox}
\begin{tcolorbox}
\textsubscript{2} (94-2) предстанем лицу Его со славословием, в песнях воскликнем Ему,
\end{tcolorbox}
\begin{tcolorbox}
\textsubscript{3} (94-3) ибо Господь есть Бог великий и Царь великий над всеми богами.
\end{tcolorbox}
\begin{tcolorbox}
\textsubscript{4} (94-4) В Его руке глубины земли, и вершины гор--Его же;
\end{tcolorbox}
\begin{tcolorbox}
\textsubscript{5} (94-5) Его--море, и Он создал его, и сушу образовали руки Его.
\end{tcolorbox}
\begin{tcolorbox}
\textsubscript{6} (94-6) Приидите, поклонимся и припадем, преклоним колени пред лицем Господа, Творца нашего;
\end{tcolorbox}
\begin{tcolorbox}
\textsubscript{7} (94-7) ибо Он есть Бог наш, и мы--народ паствы Его и овцы руки Его. О, если бы вы ныне послушали гласа Его:
\end{tcolorbox}
\begin{tcolorbox}
\textsubscript{8} (94-8) 'не ожесточите сердца вашего, как в Мериве, как в день искушения в пустыне,
\end{tcolorbox}
\begin{tcolorbox}
\textsubscript{9} (94-9) где искушали Меня отцы ваши, испытывали Меня, и видели дело Мое.
\end{tcolorbox}
\begin{tcolorbox}
\textsubscript{10} (94-10) Сорок лет Я был раздражаем родом сим, и сказал: это народ, заблуждающийся сердцем; они не познали путей Моих,
\end{tcolorbox}
\begin{tcolorbox}
\textsubscript{11} (94-11) и потому Я поклялся во гневе Моем, что они не войдут в покой Мой'.
\end{tcolorbox}
\subsection{CHAPTER 96}
\begin{tcolorbox}
\textsubscript{1} (95-1) Воспойте Господу песнь новую; воспойте Господу, вся земля;
\end{tcolorbox}
\begin{tcolorbox}
\textsubscript{2} (95-2) пойте Господу, благословляйте имя Его, благовествуйте со дня на день спасение Его;
\end{tcolorbox}
\begin{tcolorbox}
\textsubscript{3} (95-3) возвещайте в народах славу Его, во всех племенах чудеса Его;
\end{tcolorbox}
\begin{tcolorbox}
\textsubscript{4} (95-4) ибо велик Господь и достохвален, страшен Он паче всех богов.
\end{tcolorbox}
\begin{tcolorbox}
\textsubscript{5} (95-5) Ибо все боги народов--идолы, а Господь небеса сотворил.
\end{tcolorbox}
\begin{tcolorbox}
\textsubscript{6} (95-6) Слава и величие пред лицем Его, сила и великолепие во святилище Его.
\end{tcolorbox}
\begin{tcolorbox}
\textsubscript{7} (95-7) Воздайте Господу, племена народов, воздайте Господу славу и честь;
\end{tcolorbox}
\begin{tcolorbox}
\textsubscript{8} (95-8) воздайте Господу славу имени Его, несите дары и идите во дворы Его;
\end{tcolorbox}
\begin{tcolorbox}
\textsubscript{9} (95-9) поклонитесь Господу во благолепии святыни. Трепещи пред лицем Его, вся земля!
\end{tcolorbox}
\begin{tcolorbox}
\textsubscript{10} (95-10) Скажите народам: Господь царствует! потому тверда вселенная, не поколеблется. Он будет судить народы по правде.
\end{tcolorbox}
\begin{tcolorbox}
\textsubscript{11} (95-11) Да веселятся небеса и да торжествует земля; да шумит море и что наполняет его;
\end{tcolorbox}
\begin{tcolorbox}
\textsubscript{12} (95-12) да радуется поле и все, что на нем, и да ликуют все дерева дубравные
\end{tcolorbox}
\begin{tcolorbox}
\textsubscript{13} (95-13) пред лицем Господа; ибо идет, ибо идет судить землю. Он будет судить вселенную по правде, и народы--по истине Своей.
\end{tcolorbox}
\subsection{CHAPTER 97}
\begin{tcolorbox}
\textsubscript{1} (96-1) Господь царствует: да радуется земля; да веселятся многочисленные острова.
\end{tcolorbox}
\begin{tcolorbox}
\textsubscript{2} (96-2) Облако и мрак окрест Его; правда и суд--основание престола Его.
\end{tcolorbox}
\begin{tcolorbox}
\textsubscript{3} (96-3) Пред Ним идет огонь и вокруг попаляет врагов Его.
\end{tcolorbox}
\begin{tcolorbox}
\textsubscript{4} (96-4) Молнии Его освещают вселенную; земля видит и трепещет.
\end{tcolorbox}
\begin{tcolorbox}
\textsubscript{5} (96-5) Горы, как воск, тают от лица Господа, от лица Господа всей земли.
\end{tcolorbox}
\begin{tcolorbox}
\textsubscript{6} (96-6) Небеса возвещают правду Его, и все народы видят славу Его.
\end{tcolorbox}
\begin{tcolorbox}
\textsubscript{7} (96-7) Да постыдятся все служащие истуканам, хвалящиеся идолами. Поклонитесь пред Ним, все боги.
\end{tcolorbox}
\begin{tcolorbox}
\textsubscript{8} (96-8) Слышит Сион и радуется, и веселятся дщери Иудины ради судов Твоих, Господи,
\end{tcolorbox}
\begin{tcolorbox}
\textsubscript{9} (96-9) ибо Ты, Господи, высок над всею землею, превознесен над всеми богами.
\end{tcolorbox}
\begin{tcolorbox}
\textsubscript{10} (96-10) Любящие Господа, ненавидьте зло! Он хранит души святых Своих; из руки нечестивых избавляет их.
\end{tcolorbox}
\begin{tcolorbox}
\textsubscript{11} (96-11) Свет сияет на праведника, и на правых сердцем--веселие.
\end{tcolorbox}
\begin{tcolorbox}
\textsubscript{12} (96-12) Радуйтесь, праведные, о Господе и славьте память святыни Его.
\end{tcolorbox}
\subsection{CHAPTER 98}
\begin{tcolorbox}
\textsubscript{1} (97-1) ^^Псалом^^ Воспойте Господу новую песнь, ибо Он сотворил чудеса. Его десница и святая мышца Его доставили Ему победу.
\end{tcolorbox}
\begin{tcolorbox}
\textsubscript{2} (97-2) Явил Господь спасение Свое, открыл пред очами народов правду Свою.
\end{tcolorbox}
\begin{tcolorbox}
\textsubscript{3} (97-3) Вспомнил Он милость Свою и верность Свою к дому Израилеву. Все концы земли увидели спасение Бога нашего.
\end{tcolorbox}
\begin{tcolorbox}
\textsubscript{4} (97-4) Восклицайте Господу, вся земля; торжествуйте, веселитесь и пойте;
\end{tcolorbox}
\begin{tcolorbox}
\textsubscript{5} (97-5) пойте Господу с гуслями, с гуслями и с гласом псалмопения;
\end{tcolorbox}
\begin{tcolorbox}
\textsubscript{6} (97-6) при звуке труб и рога торжествуйте пред Царем Господом.
\end{tcolorbox}
\begin{tcolorbox}
\textsubscript{7} (97-7) Да шумит море и что наполняет его, вселенная и живущие в ней;
\end{tcolorbox}
\begin{tcolorbox}
\textsubscript{8} (97-8) да рукоплещут реки, да ликуют вместе горы
\end{tcolorbox}
\begin{tcolorbox}
\textsubscript{9} (97-9) пред лицем Господа, ибо Он идет судить землю. Он будет судить вселенную праведно и народы--верно.
\end{tcolorbox}
\subsection{CHAPTER 99}
\begin{tcolorbox}
\textsubscript{1} (98-1) Господь царствует: да трепещут народы! Он восседает на Херувимах: да трясется земля!
\end{tcolorbox}
\begin{tcolorbox}
\textsubscript{2} (98-2) Господь на Сионе велик, и высок Он над всеми народами.
\end{tcolorbox}
\begin{tcolorbox}
\textsubscript{3} (98-3) Да славят великое и страшное имя Твое: свято оно!
\end{tcolorbox}
\begin{tcolorbox}
\textsubscript{4} (98-4) И могущество царя любит суд. Ты утвердил справедливость; суд и правду Ты совершил в Иакове.
\end{tcolorbox}
\begin{tcolorbox}
\textsubscript{5} (98-5) Превозносите Господа, Бога нашего, и поклоняйтесь подножию Его: свято оно!
\end{tcolorbox}
\begin{tcolorbox}
\textsubscript{6} (98-6) Моисей и Аарон между священниками и Самуил между призывающими имя Его взывали к Господу, и Он внимал им.
\end{tcolorbox}
\begin{tcolorbox}
\textsubscript{7} (98-7) В столпе облачном говорил Он к ним; они хранили Его заповеди и устав, который Он дал им.
\end{tcolorbox}
\begin{tcolorbox}
\textsubscript{8} (98-8) Господи, Боже наш! Ты внимал им; Ты был для них Богом прощающим и наказывающим за дела их.
\end{tcolorbox}
\begin{tcolorbox}
\textsubscript{9} (98-9) Превозносите Господа, Бога нашего, и поклоняйтесь на святой горе Его, ибо свят Господь, Бог наш.
\end{tcolorbox}
\subsection{CHAPTER 100}
\begin{tcolorbox}
\textsubscript{1} (99-1) ^^Псалом хвалебный.^^ Воскликните Господу, вся земля!
\end{tcolorbox}
\begin{tcolorbox}
\textsubscript{2} (99-2) Служите Господу с веселием; идите пред лице Его с восклицанием!
\end{tcolorbox}
\begin{tcolorbox}
\textsubscript{3} (99-3) Познайте, что Господь есть Бог, что Он сотворил нас, и мы--Его, Его народ и овцы паствы Его.
\end{tcolorbox}
\begin{tcolorbox}
\textsubscript{4} (99-4) Входите во врата Его со славословием, во дворы Его--с хвалою. Славьте Его, благословляйте имя Его,
\end{tcolorbox}
\begin{tcolorbox}
\textsubscript{5} (99-5) ибо благ Господь: милость Его вовек, и истина Его в род и род.
\end{tcolorbox}
\subsection{CHAPTER 101}
\begin{tcolorbox}
\textsubscript{1} (100-1) ^^Псалом Давида.^^ Милость и суд буду петь; Тебе, Господи, буду петь.
\end{tcolorbox}
\begin{tcolorbox}
\textsubscript{2} (100-2) Буду размышлять о пути непорочном: 'когда ты придешь ко мне?' Буду ходить в непорочности моего сердца посреди дома моего.
\end{tcolorbox}
\begin{tcolorbox}
\textsubscript{3} (100-3) Не положу пред очами моими вещи непотребной; дело преступное я ненавижу: не прилепится оно ко мне.
\end{tcolorbox}
\begin{tcolorbox}
\textsubscript{4} (100-4) Сердце развращенное будет удалено от меня; злого я не буду знать.
\end{tcolorbox}
\begin{tcolorbox}
\textsubscript{5} (100-5) Тайно клевещущего на ближнего своего изгоню; гордого очами и надменного сердцем не потерплю.
\end{tcolorbox}
\begin{tcolorbox}
\textsubscript{6} (100-6) Глаза мои на верных земли, чтобы они пребывали при мне; кто ходит путем непорочности, тот будет служить мне.
\end{tcolorbox}
\begin{tcolorbox}
\textsubscript{7} (100-7) Не будет жить в доме моем поступающий коварно; говорящий ложь не останется пред глазами моими.
\end{tcolorbox}
\begin{tcolorbox}
\textsubscript{8} (100-8) С раннего утра буду истреблять всех нечестивцев земли, дабы искоренить из града Господня всех делающих беззаконие.
\end{tcolorbox}
\subsection{CHAPTER 102}
\begin{tcolorbox}
\textsubscript{1} (101-1) ^^Молитва страждущего, когда он унывает и изливает пред Господом печаль свою.^^ (101-2) Господи! услышь молитву мою, и вопль мой да придет к Тебе.
\end{tcolorbox}
\begin{tcolorbox}
\textsubscript{2} (101-3) Не скрывай лица Твоего от меня; в день скорби моей приклони ко мне ухо Твое; в день, [когда воззову к Тебе], скоро услышь меня;
\end{tcolorbox}
\begin{tcolorbox}
\textsubscript{3} (101-4) ибо исчезли, как дым, дни мои, и кости мои обожжены, как головня;
\end{tcolorbox}
\begin{tcolorbox}
\textsubscript{4} (101-5) сердце мое поражено, и иссохло, как трава, так что я забываю есть хлеб мой;
\end{tcolorbox}
\begin{tcolorbox}
\textsubscript{5} (101-6) от голоса стенания моего кости мои прильпнули к плоти моей.
\end{tcolorbox}
\begin{tcolorbox}
\textsubscript{6} (101-7) Я уподобился пеликану в пустыне; я стал как филин на развалинах;
\end{tcolorbox}
\begin{tcolorbox}
\textsubscript{7} (101-8) не сплю и сижу, как одинокая птица на кровле.
\end{tcolorbox}
\begin{tcolorbox}
\textsubscript{8} (101-9) Всякий день поносят меня враги мои, и злобствующие на меня клянут мною.
\end{tcolorbox}
\begin{tcolorbox}
\textsubscript{9} (101-10) Я ем пепел, как хлеб, и питье мое растворяю слезами,
\end{tcolorbox}
\begin{tcolorbox}
\textsubscript{10} (101-11) от гнева Твоего и негодования Твоего, ибо Ты вознес меня и низверг меня.
\end{tcolorbox}
\begin{tcolorbox}
\textsubscript{11} (101-12) Дни мои--как уклоняющаяся тень, и я иссох, как трава.
\end{tcolorbox}
\begin{tcolorbox}
\textsubscript{12} (101-13) Ты же, Господи, вовек пребываешь, и память о Тебе в род и род.
\end{tcolorbox}
\begin{tcolorbox}
\textsubscript{13} (101-14) Ты восстанешь, умилосердишься над Сионом, ибо время помиловать его, --ибо пришло время;
\end{tcolorbox}
\begin{tcolorbox}
\textsubscript{14} (101-15) ибо рабы Твои возлюбили и камни его, и о прахе его жалеют.
\end{tcolorbox}
\begin{tcolorbox}
\textsubscript{15} (101-16) И убоятся народы имени Господня, и все цари земные--славы Твоей.
\end{tcolorbox}
\begin{tcolorbox}
\textsubscript{16} (101-17) Ибо созиждет Господь Сион и явится во славе Своей;
\end{tcolorbox}
\begin{tcolorbox}
\textsubscript{17} (101-18) призрит на молитву беспомощных и не презрит моления их.
\end{tcolorbox}
\begin{tcolorbox}
\textsubscript{18} (101-19) Напишется о сем для рода последующего, и поколение грядущее восхвалит Господа,
\end{tcolorbox}
\begin{tcolorbox}
\textsubscript{19} (101-20) ибо Он приникнул со святой высоты Своей, с небес призрел Господь на землю,
\end{tcolorbox}
\begin{tcolorbox}
\textsubscript{20} (101-21) чтобы услышать стон узников, разрешить сынов смерти,
\end{tcolorbox}
\begin{tcolorbox}
\textsubscript{21} (101-22) дабы возвещали на Сионе имя Господне и хвалу Его--в Иерусалиме,
\end{tcolorbox}
\begin{tcolorbox}
\textsubscript{22} (101-23) когда соберутся народы вместе и царства для служения Господу.
\end{tcolorbox}
\begin{tcolorbox}
\textsubscript{23} (101-24) Изнурил Он на пути силы мои, сократил дни мои.
\end{tcolorbox}
\begin{tcolorbox}
\textsubscript{24} (101-25) Я сказал: Боже мой! не восхити меня в половине дней моих. Твои лета в роды родов.
\end{tcolorbox}
\begin{tcolorbox}
\textsubscript{25} (101-26) В начале Ты, основал землю, и небеса--дело Твоих рук;
\end{tcolorbox}
\begin{tcolorbox}
\textsubscript{26} (101-27) они погибнут, а Ты пребудешь; и все они, как риза, обветшают, и, как одежду, Ты переменишь их, и изменятся;
\end{tcolorbox}
\begin{tcolorbox}
\textsubscript{27} (101-28) но Ты--тот же, и лета Твои не кончатся.
\end{tcolorbox}
\begin{tcolorbox}
\textsubscript{28} (101-29) Сыны рабов Твоих будут жить, и семя их утвердится пред лицем Твоим.
\end{tcolorbox}
\subsection{CHAPTER 103}
\begin{tcolorbox}
\textsubscript{1} (102-1) ^^Псалом Давида.^^ Благослови, душа моя, Господа, и вся внутренность моя--святое имя Его.
\end{tcolorbox}
\begin{tcolorbox}
\textsubscript{2} (102-2) Благослови, душа моя, Господа и не забывай всех благодеяний Его.
\end{tcolorbox}
\begin{tcolorbox}
\textsubscript{3} (102-3) Он прощает все беззакония твои, исцеляет все недуги твои;
\end{tcolorbox}
\begin{tcolorbox}
\textsubscript{4} (102-4) избавляет от могилы жизнь твою, венчает тебя милостью и щедротами;
\end{tcolorbox}
\begin{tcolorbox}
\textsubscript{5} (102-5) насыщает благами желание твое: обновляется, подобно орлу, юность твоя.
\end{tcolorbox}
\begin{tcolorbox}
\textsubscript{6} (102-6) Господь творит правду и суд всем обиженным.
\end{tcolorbox}
\begin{tcolorbox}
\textsubscript{7} (102-7) Он показал пути Свои Моисею, сынам Израилевым--дела Свои.
\end{tcolorbox}
\begin{tcolorbox}
\textsubscript{8} (102-8) Щедр и милостив Господь, долготерпелив и многомилостив:
\end{tcolorbox}
\begin{tcolorbox}
\textsubscript{9} (102-9) не до конца гневается, и не вовек негодует.
\end{tcolorbox}
\begin{tcolorbox}
\textsubscript{10} (102-10) Не по беззакониям нашим сотворил нам, и не по грехам нашим воздал нам:
\end{tcolorbox}
\begin{tcolorbox}
\textsubscript{11} (102-11) ибо как высоко небо над землею, так велика милость [Господа] к боящимся Его;
\end{tcolorbox}
\begin{tcolorbox}
\textsubscript{12} (102-12) как далеко восток от запада, так удалил Он от нас беззакония наши;
\end{tcolorbox}
\begin{tcolorbox}
\textsubscript{13} (102-13) как отец милует сынов, так милует Господь боящихся Его.
\end{tcolorbox}
\begin{tcolorbox}
\textsubscript{14} (102-14) Ибо Он знает состав наш, помнит, что мы--персть.
\end{tcolorbox}
\begin{tcolorbox}
\textsubscript{15} (102-15) Дни человека--как трава; как цвет полевой, так он цветет.
\end{tcolorbox}
\begin{tcolorbox}
\textsubscript{16} (102-16) Пройдет над ним ветер, и нет его, и место его уже не узнает его.
\end{tcolorbox}
\begin{tcolorbox}
\textsubscript{17} (102-17) Милость же Господня от века и до века к боящимся Его,
\end{tcolorbox}
\begin{tcolorbox}
\textsubscript{18} (102-18) и правда Его на сынах сынов, хранящих завет Его и помнящих заповеди Его, чтобы исполнять их.
\end{tcolorbox}
\begin{tcolorbox}
\textsubscript{19} (102-19) Господь на небесах поставил престол Свой, и царство Его всем обладает.
\end{tcolorbox}
\begin{tcolorbox}
\textsubscript{20} (102-20) Благословите Господа, [все] Ангелы Его, крепкие силою, исполняющие слово Его, повинуясь гласу слова Его;
\end{tcolorbox}
\begin{tcolorbox}
\textsubscript{21} (102-21) благословите Господа, все воинства Его, служители Его, исполняющие волю Его;
\end{tcolorbox}
\begin{tcolorbox}
\textsubscript{22} (102-22) благословите Господа, все дела Его, во всех местах владычества Его. Благослови, душа моя, Господа!
\end{tcolorbox}
\subsection{CHAPTER 104}
\begin{tcolorbox}
\textsubscript{1} (103-1) Благослови, душа моя, Господа! Господи, Боже мой! Ты дивно велик, Ты облечен славою и величием;
\end{tcolorbox}
\begin{tcolorbox}
\textsubscript{2} (103-2) Ты одеваешься светом, как ризою, простираешь небеса, как шатер;
\end{tcolorbox}
\begin{tcolorbox}
\textsubscript{3} (103-3) устрояешь над водами горние чертоги Твои, делаешь облака Твоею колесницею, шествуешь на крыльях ветра.
\end{tcolorbox}
\begin{tcolorbox}
\textsubscript{4} (103-4) Ты творишь ангелами Твоими духов, служителями Твоими--огонь пылающий.
\end{tcolorbox}
\begin{tcolorbox}
\textsubscript{5} (103-5) Ты поставил землю на твердых основах: не поколеблется она во веки и веки.
\end{tcolorbox}
\begin{tcolorbox}
\textsubscript{6} (103-6) Бездною, как одеянием, покрыл Ты ее, на горах стоят воды.
\end{tcolorbox}
\begin{tcolorbox}
\textsubscript{7} (103-7) От прещения Твоего бегут они, от гласа грома Твоего быстро уходят;
\end{tcolorbox}
\begin{tcolorbox}
\textsubscript{8} (103-8) восходят на горы, нисходят в долины, на место, которое Ты назначил для них.
\end{tcolorbox}
\begin{tcolorbox}
\textsubscript{9} (103-9) Ты положил предел, которого не перейдут, и не возвратятся покрыть землю.
\end{tcolorbox}
\begin{tcolorbox}
\textsubscript{10} (103-10) Ты послал источники в долины: между горами текут,
\end{tcolorbox}
\begin{tcolorbox}
\textsubscript{11} (103-11) поят всех полевых зверей; дикие ослы утоляют жажду свою.
\end{tcolorbox}
\begin{tcolorbox}
\textsubscript{12} (103-12) При них обитают птицы небесные, из среды ветвей издают голос.
\end{tcolorbox}
\begin{tcolorbox}
\textsubscript{13} (103-13) Ты напояешь горы с высот Твоих, плодами дел Твоих насыщается земля.
\end{tcolorbox}
\begin{tcolorbox}
\textsubscript{14} (103-14) Ты произращаешь траву для скота, и зелень на пользу человека, чтобы произвести из земли пищу,
\end{tcolorbox}
\begin{tcolorbox}
\textsubscript{15} (103-15) и вино, которое веселит сердце человека, и елей, от которого блистает лице его, и хлеб, который укрепляет сердце человека.
\end{tcolorbox}
\begin{tcolorbox}
\textsubscript{16} (103-16) Насыщаются древа Господа, кедры Ливанские, которые Он насадил;
\end{tcolorbox}
\begin{tcolorbox}
\textsubscript{17} (103-17) на них гнездятся птицы: ели--жилище аисту,
\end{tcolorbox}
\begin{tcolorbox}
\textsubscript{18} (103-18) высокие горы--сернам; каменные утесы--убежище зайцам.
\end{tcolorbox}
\begin{tcolorbox}
\textsubscript{19} (103-19) Он сотворил луну для [указания] времен, солнце знает свой запад.
\end{tcolorbox}
\begin{tcolorbox}
\textsubscript{20} (103-20) Ты простираешь тьму и бывает ночь: во время нее бродят все лесные звери;
\end{tcolorbox}
\begin{tcolorbox}
\textsubscript{21} (103-21) львы рыкают о добыче и просят у Бога пищу себе.
\end{tcolorbox}
\begin{tcolorbox}
\textsubscript{22} (103-22) Восходит солнце, [и] они собираются и ложатся в свои логовища;
\end{tcolorbox}
\begin{tcolorbox}
\textsubscript{23} (103-23) выходит человек на дело свое и на работу свою до вечера.
\end{tcolorbox}
\begin{tcolorbox}
\textsubscript{24} (103-24) Как многочисленны дела Твои, Господи! Все соделал Ты премудро; земля полна произведений Твоих.
\end{tcolorbox}
\begin{tcolorbox}
\textsubscript{25} (103-25) Это--море великое и пространное: там пресмыкающиеся, которым нет числа, животные малые с большими;
\end{tcolorbox}
\begin{tcolorbox}
\textsubscript{26} (103-26) там плавают корабли, там этот левиафан, которого Ты сотворил играть в нем.
\end{tcolorbox}
\begin{tcolorbox}
\textsubscript{27} (103-27) Все они от Тебя ожидают, чтобы Ты дал им пищу их в свое время.
\end{tcolorbox}
\begin{tcolorbox}
\textsubscript{28} (103-28) Даешь им--принимают, отверзаешь руку Твою--насыщаются благом;
\end{tcolorbox}
\begin{tcolorbox}
\textsubscript{29} (103-29) скроешь лице Твое--мятутся, отнимешь дух их--умирают и в персть свою возвращаются;
\end{tcolorbox}
\begin{tcolorbox}
\textsubscript{30} (103-30) пошлешь дух Твой--созидаются, и Ты обновляешь лице земли.
\end{tcolorbox}
\begin{tcolorbox}
\textsubscript{31} (103-31) Да будет Господу слава во веки; да веселится Господь о делах Своих!
\end{tcolorbox}
\begin{tcolorbox}
\textsubscript{32} (103-32) Призирает на землю, и она трясется; прикасается к горам, и дымятся.
\end{tcolorbox}
\begin{tcolorbox}
\textsubscript{33} (103-33) Буду петь Господу во [всю] жизнь мою, буду петь Богу моему, доколе есмь.
\end{tcolorbox}
\begin{tcolorbox}
\textsubscript{34} (103-34) Да будет благоприятна Ему песнь моя; буду веселиться о Господе.
\end{tcolorbox}
\begin{tcolorbox}
\textsubscript{35} (103-35) Да исчезнут грешники с земли, и беззаконных да не будет более. Благослови, душа моя, Господа! Аллилуия!
\end{tcolorbox}
\subsection{CHAPTER 105}
\begin{tcolorbox}
\textsubscript{1} (104-1) Славьте Господа; призывайте имя Его; возвещайте в народах дела Его;
\end{tcolorbox}
\begin{tcolorbox}
\textsubscript{2} (104-2) воспойте Ему и пойте Ему; поведайте о всех чудесах Его.
\end{tcolorbox}
\begin{tcolorbox}
\textsubscript{3} (104-3) Хвалитесь именем Его святым; да веселится сердце ищущих Господа.
\end{tcolorbox}
\begin{tcolorbox}
\textsubscript{4} (104-4) Ищите Господа и силы Его, ищите лица Его всегда.
\end{tcolorbox}
\begin{tcolorbox}
\textsubscript{5} (104-5) Воспоминайте чудеса Его, которые сотворил, знамения Его и суды уст Его,
\end{tcolorbox}
\begin{tcolorbox}
\textsubscript{6} (104-6) вы, семя Авраамово, рабы Его, сыны Иакова, избранные Его.
\end{tcolorbox}
\begin{tcolorbox}
\textsubscript{7} (104-7) Он Господь Бог наш: по всей земле суды Его.
\end{tcolorbox}
\begin{tcolorbox}
\textsubscript{8} (104-8) Вечно помнит завет Свой, слово, [которое] заповедал в тысячу родов,
\end{tcolorbox}
\begin{tcolorbox}
\textsubscript{9} (104-9) которое завещал Аврааму, и клятву Свою Исааку,
\end{tcolorbox}
\begin{tcolorbox}
\textsubscript{10} (104-10) и поставил то Иакову в закон и Израилю в завет вечный,
\end{tcolorbox}
\begin{tcolorbox}
\textsubscript{11} (104-11) говоря: 'тебе дам землю Ханаанскую в удел наследия вашего'.
\end{tcolorbox}
\begin{tcolorbox}
\textsubscript{12} (104-12) Когда их было еще мало числом, очень мало, и они были пришельцами в ней
\end{tcolorbox}
\begin{tcolorbox}
\textsubscript{13} (104-13) и переходили от народа к народу, из царства к иному племени,
\end{tcolorbox}
\begin{tcolorbox}
\textsubscript{14} (104-14) никому не позволял обижать их и возбранял о них царям:
\end{tcolorbox}
\begin{tcolorbox}
\textsubscript{15} (104-15) 'не прикасайтесь к помазанным Моим, и пророкам Моим не делайте зла'.
\end{tcolorbox}
\begin{tcolorbox}
\textsubscript{16} (104-16) И призвал голод на землю; всякий стебель хлебный истребил.
\end{tcolorbox}
\begin{tcolorbox}
\textsubscript{17} (104-17) Послал пред ними человека: в рабы продан был Иосиф.
\end{tcolorbox}
\begin{tcolorbox}
\textsubscript{18} (104-18) Стеснили оковами ноги его; в железо вошла душа его,
\end{tcolorbox}
\begin{tcolorbox}
\textsubscript{19} (104-19) доколе исполнилось слово Его: слово Господне испытало его.
\end{tcolorbox}
\begin{tcolorbox}
\textsubscript{20} (104-20) Послал царь, и разрешил его владетель народов и освободил его;
\end{tcolorbox}
\begin{tcolorbox}
\textsubscript{21} (104-21) поставил его господином над домом своим и правителем над всем владением своим,
\end{tcolorbox}
\begin{tcolorbox}
\textsubscript{22} (104-22) чтобы он наставлял вельмож его по своей душе и старейшин его учил мудрости.
\end{tcolorbox}
\begin{tcolorbox}
\textsubscript{23} (104-23) Тогда пришел Израиль в Египет, и переселился Иаков в землю Хамову.
\end{tcolorbox}
\begin{tcolorbox}
\textsubscript{24} (104-24) И весьма размножил [Бог] народ Свой и сделал его сильнее врагов его.
\end{tcolorbox}
\begin{tcolorbox}
\textsubscript{25} (104-25) Возбудил в сердце их ненависть против народа Его и ухищрение против рабов Его.
\end{tcolorbox}
\begin{tcolorbox}
\textsubscript{26} (104-26) Послал Моисея, раба Своего, Аарона, которого избрал.
\end{tcolorbox}
\begin{tcolorbox}
\textsubscript{27} (104-27) Они показали между ними слова знамений Его и чудеса [Его] в земле Хамовой.
\end{tcolorbox}
\begin{tcolorbox}
\textsubscript{28} (104-28) Послал тьму и сделал мрак, и не воспротивились слову Его.
\end{tcolorbox}
\begin{tcolorbox}
\textsubscript{29} (104-29) Преложил воду их в кровь, и уморил рыбу их.
\end{tcolorbox}
\begin{tcolorbox}
\textsubscript{30} (104-30) Земля их произвела множество жаб [даже] в спальне царей их.
\end{tcolorbox}
\begin{tcolorbox}
\textsubscript{31} (104-31) Он сказал, и пришли разные насекомые, скнипы во все пределы их.
\end{tcolorbox}
\begin{tcolorbox}
\textsubscript{32} (104-32) Вместо дождя послал на них град, палящий огонь на землю их,
\end{tcolorbox}
\begin{tcolorbox}
\textsubscript{33} (104-33) и побил виноград их и смоковницы их, и сокрушил дерева в пределах их.
\end{tcolorbox}
\begin{tcolorbox}
\textsubscript{34} (104-34) Сказал, и пришла саранча и гусеницы без числа;
\end{tcolorbox}
\begin{tcolorbox}
\textsubscript{35} (104-35) и съели всю траву на земле их, и съели плоды на полях их.
\end{tcolorbox}
\begin{tcolorbox}
\textsubscript{36} (104-36) И поразил всякого первенца в земле их, начатки всей силы их.
\end{tcolorbox}
\begin{tcolorbox}
\textsubscript{37} (104-37) И вывел [Израильтян] с серебром и золотом, и не было в коленах их болящего.
\end{tcolorbox}
\begin{tcolorbox}
\textsubscript{38} (104-38) Обрадовался Египет исшествию их; ибо страх от них напал на него.
\end{tcolorbox}
\begin{tcolorbox}
\textsubscript{39} (104-39) Простер облако в покров [им] и огонь, чтобы светить [им] ночью.
\end{tcolorbox}
\begin{tcolorbox}
\textsubscript{40} (104-40) Просили, и Он послал перепелов, и хлебом небесным насыщал их.
\end{tcolorbox}
\begin{tcolorbox}
\textsubscript{41} (104-41) Разверз камень, и потекли воды, потекли рекою по местам сухим,
\end{tcolorbox}
\begin{tcolorbox}
\textsubscript{42} (104-42) ибо вспомнил Он святое слово Свое к Аврааму, рабу Своему,
\end{tcolorbox}
\begin{tcolorbox}
\textsubscript{43} (104-43) и вывел народ Свой в радости, избранных Своих в веселии,
\end{tcolorbox}
\begin{tcolorbox}
\textsubscript{44} (104-44) и дал им земли народов, и они наследовали труд иноплеменных,
\end{tcolorbox}
\begin{tcolorbox}
\textsubscript{45} (104-45) чтобы соблюдали уставы Его и хранили законы Его. Аллилуия! Аллилуия.
\end{tcolorbox}
\subsection{CHAPTER 106}
\begin{tcolorbox}
\textsubscript{1} (105-1) Славьте Господа, ибо Он благ, ибо вовек милость Его.
\end{tcolorbox}
\begin{tcolorbox}
\textsubscript{2} (105-2) Кто изречет могущество Господа, возвестит все хвалы Его?
\end{tcolorbox}
\begin{tcolorbox}
\textsubscript{3} (105-3) Блаженны хранящие суд и творящие правду во всякое время!
\end{tcolorbox}
\begin{tcolorbox}
\textsubscript{4} (105-4) Вспомни о мне, Господи, в благоволении к народу Твоему; посети меня спасением Твоим,
\end{tcolorbox}
\begin{tcolorbox}
\textsubscript{5} (105-5) дабы мне видеть благоденствие избранных Твоих, веселиться веселием народа Твоего, хвалиться с наследием Твоим.
\end{tcolorbox}
\begin{tcolorbox}
\textsubscript{6} (105-6) Согрешили мы с отцами нашими, совершили беззаконие, соделали неправду.
\end{tcolorbox}
\begin{tcolorbox}
\textsubscript{7} (105-7) Отцы наши в Египте не уразумели чудес Твоих, не помнили множества милостей Твоих, и возмутились у моря, у Чермного моря.
\end{tcolorbox}
\begin{tcolorbox}
\textsubscript{8} (105-8) Но Он спас их ради имени Своего, дабы показать могущество Свое.
\end{tcolorbox}
\begin{tcolorbox}
\textsubscript{9} (105-9) Грозно рек морю Чермному, и оно иссохло; и провел их по безднам, как по суше;
\end{tcolorbox}
\begin{tcolorbox}
\textsubscript{10} (105-10) и спас их от руки ненавидящего и избавил их от руки врага.
\end{tcolorbox}
\begin{tcolorbox}
\textsubscript{11} (105-11) Воды покрыли врагов их, ни одного из них не осталось.
\end{tcolorbox}
\begin{tcolorbox}
\textsubscript{12} (105-12) И поверили они словам Его, [и] воспели хвалу Ему.
\end{tcolorbox}
\begin{tcolorbox}
\textsubscript{13} (105-13) [Но] скоро забыли дела Его, не дождались Его изволения;
\end{tcolorbox}
\begin{tcolorbox}
\textsubscript{14} (105-14) увлеклись похотением в пустыне, и искусили Бога в необитаемой.
\end{tcolorbox}
\begin{tcolorbox}
\textsubscript{15} (105-15) И Он исполнил прошение их, [но] послал язву на души их.
\end{tcolorbox}
\begin{tcolorbox}
\textsubscript{16} (105-16) И позавидовали в стане Моисею [и] Аарону, святому Господню.
\end{tcolorbox}
\begin{tcolorbox}
\textsubscript{17} (105-17) Разверзлась земля, и поглотила Дафана и покрыла скопище Авирона.
\end{tcolorbox}
\begin{tcolorbox}
\textsubscript{18} (105-18) И возгорелся огонь в скопище их, пламень попалил нечестивых.
\end{tcolorbox}
\begin{tcolorbox}
\textsubscript{19} (105-19) Сделали тельца у Хорива и поклонились истукану;
\end{tcolorbox}
\begin{tcolorbox}
\textsubscript{20} (105-20) и променяли славу свою на изображение вола, ядущего траву.
\end{tcolorbox}
\begin{tcolorbox}
\textsubscript{21} (105-21) Забыли Бога, Спасителя своего, совершившего великое в Египте,
\end{tcolorbox}
\begin{tcolorbox}
\textsubscript{22} (105-22) дивное в земле Хамовой, страшное у Чермного моря.
\end{tcolorbox}
\begin{tcolorbox}
\textsubscript{23} (105-23) И хотел истребить их, если бы Моисей, избранный Его, не стал пред Ним в расселине, чтобы отвратить ярость Его, да не погубит [их].
\end{tcolorbox}
\begin{tcolorbox}
\textsubscript{24} (105-24) И презрели они землю желанную, не верили слову Его;
\end{tcolorbox}
\begin{tcolorbox}
\textsubscript{25} (105-25) и роптали в шатрах своих, не слушались гласа Господня.
\end{tcolorbox}
\begin{tcolorbox}
\textsubscript{26} (105-26) И поднял Он руку Свою на них, чтобы низложить их в пустыне,
\end{tcolorbox}
\begin{tcolorbox}
\textsubscript{27} (105-27) низложить племя их в народах и рассеять их по землям.
\end{tcolorbox}
\begin{tcolorbox}
\textsubscript{28} (105-28) Они прилепились к Ваалфегору и ели жертвы бездушным,
\end{tcolorbox}
\begin{tcolorbox}
\textsubscript{29} (105-29) и раздражали [Бога] делами своими, и вторглась к ним язва.
\end{tcolorbox}
\begin{tcolorbox}
\textsubscript{30} (105-30) И восстал Финеес и произвел суд, --и остановилась язва.
\end{tcolorbox}
\begin{tcolorbox}
\textsubscript{31} (105-31) И [это] вменено ему в праведность в роды и роды во веки.
\end{tcolorbox}
\begin{tcolorbox}
\textsubscript{32} (105-32) И прогневали [Бога] у вод Меривы, и Моисей потерпел за них,
\end{tcolorbox}
\begin{tcolorbox}
\textsubscript{33} (105-33) ибо они огорчили дух его, и он погрешил устами своими.
\end{tcolorbox}
\begin{tcolorbox}
\textsubscript{34} (105-34) Не истребили народов, о которых сказал им Господь,
\end{tcolorbox}
\begin{tcolorbox}
\textsubscript{35} (105-35) но смешались с язычниками и научились делам их;
\end{tcolorbox}
\begin{tcolorbox}
\textsubscript{36} (105-36) служили истуканам их, [которые] были для них сетью,
\end{tcolorbox}
\begin{tcolorbox}
\textsubscript{37} (105-37) и приносили сыновей своих и дочерей своих в жертву бесам;
\end{tcolorbox}
\begin{tcolorbox}
\textsubscript{38} (105-38) проливали кровь невинную, кровь сыновей своих и дочерей своих, которых приносили в жертву идолам Ханаанским, --и осквернилась земля кровью;
\end{tcolorbox}
\begin{tcolorbox}
\textsubscript{39} (105-39) оскверняли себя делами своими, блудодействовали поступками своими.
\end{tcolorbox}
\begin{tcolorbox}
\textsubscript{40} (105-40) И воспылал гнев Господа на народ Его, и возгнушался Он наследием Своим
\end{tcolorbox}
\begin{tcolorbox}
\textsubscript{41} (105-41) и предал их в руки язычников, и ненавидящие их стали обладать ими.
\end{tcolorbox}
\begin{tcolorbox}
\textsubscript{42} (105-42) Враги их утесняли их, и они смирялись под рукою их.
\end{tcolorbox}
\begin{tcolorbox}
\textsubscript{43} (105-43) Много раз Он избавлял их; они же раздражали [Его] упорством своим, и были уничижаемы за беззаконие свое.
\end{tcolorbox}
\begin{tcolorbox}
\textsubscript{44} (105-44) Но Он призирал на скорбь их, когда слышал вопль их,
\end{tcolorbox}
\begin{tcolorbox}
\textsubscript{45} (105-45) и вспоминал завет Свой с ними и раскаивался по множеству милости Своей;
\end{tcolorbox}
\begin{tcolorbox}
\textsubscript{46} (105-46) и возбуждал к ним сострадание во всех, пленявших их.
\end{tcolorbox}
\begin{tcolorbox}
\textsubscript{47} (105-47) Спаси нас, Господи, Боже наш, и собери нас от народов, дабы славить святое имя Твое, хвалиться Твоею славою.
\end{tcolorbox}
\begin{tcolorbox}
\textsubscript{48} (105-48) Благословен Господь, Бог Израилев, от века и до века! И да скажет весь народ: аминь! Аллилуия!
\end{tcolorbox}
\subsection{CHAPTER 107}
\begin{tcolorbox}
\textsubscript{1} (106-1) Славьте Господа, ибо Он благ, ибо вовек милость Его!
\end{tcolorbox}
\begin{tcolorbox}
\textsubscript{2} (106-2) Так да скажут избавленные Господом, которых избавил Он от руки врага,
\end{tcolorbox}
\begin{tcolorbox}
\textsubscript{3} (106-3) и собрал от стран, от востока и запада, от севера и моря.
\end{tcolorbox}
\begin{tcolorbox}
\textsubscript{4} (106-4) Они блуждали в пустыне по безлюдному пути и не находили населенного города;
\end{tcolorbox}
\begin{tcolorbox}
\textsubscript{5} (106-5) терпели голод и жажду, душа их истаевала в них.
\end{tcolorbox}
\begin{tcolorbox}
\textsubscript{6} (106-6) Но воззвали к Господу в скорби своей, и Он избавил их от бедствий их,
\end{tcolorbox}
\begin{tcolorbox}
\textsubscript{7} (106-7) и повел их прямым путем, чтобы они шли к населенному городу.
\end{tcolorbox}
\begin{tcolorbox}
\textsubscript{8} (106-8) Да славят Господа за милость Его и за чудные дела Его для сынов человеческих:
\end{tcolorbox}
\begin{tcolorbox}
\textsubscript{9} (106-9) ибо Он насытил душу жаждущую и душу алчущую исполнил благами.
\end{tcolorbox}
\begin{tcolorbox}
\textsubscript{10} (106-10) Они сидели во тьме и тени смертной, окованные скорбью и железом;
\end{tcolorbox}
\begin{tcolorbox}
\textsubscript{11} (106-11) ибо не покорялись словам Божиим и небрегли о воле Всевышнего.
\end{tcolorbox}
\begin{tcolorbox}
\textsubscript{12} (106-12) Он смирил сердце их работами; они преткнулись, и не было помогающего.
\end{tcolorbox}
\begin{tcolorbox}
\textsubscript{13} (106-13) Но воззвали к Господу в скорби своей, и Он спас их от бедствий их;
\end{tcolorbox}
\begin{tcolorbox}
\textsubscript{14} (106-14) вывел их из тьмы и тени смертной, и расторгнул узы их.
\end{tcolorbox}
\begin{tcolorbox}
\textsubscript{15} (106-15) Да славят Господа за милость Его и за чудные дела Его для сынов человеческих:
\end{tcolorbox}
\begin{tcolorbox}
\textsubscript{16} (106-16) ибо Он сокрушил врата медные и вереи железные сломил.
\end{tcolorbox}
\begin{tcolorbox}
\textsubscript{17} (106-17) Безрассудные страдали за беззаконные пути свои и за неправды свои;
\end{tcolorbox}
\begin{tcolorbox}
\textsubscript{18} (106-18) от всякой пищи отвращалась душа их, и они приближались ко вратам смерти.
\end{tcolorbox}
\begin{tcolorbox}
\textsubscript{19} (106-19) Но воззвали к Господу в скорби своей, и Он спас их от бедствий их;
\end{tcolorbox}
\begin{tcolorbox}
\textsubscript{20} (106-20) послал слово Свое и исцелил их, и избавил их от могил их.
\end{tcolorbox}
\begin{tcolorbox}
\textsubscript{21} (106-21) Да славят Господа за милость Его и за чудные дела Его для сынов человеческих!
\end{tcolorbox}
\begin{tcolorbox}
\textsubscript{22} (106-22) Да приносят Ему жертву хвалы и да возвещают о делах Его с пением!
\end{tcolorbox}
\begin{tcolorbox}
\textsubscript{23} (106-23) Отправляющиеся на кораблях в море, производящие дела на больших водах,
\end{tcolorbox}
\begin{tcolorbox}
\textsubscript{24} (106-24) видят дела Господа и чудеса Его в пучине:
\end{tcolorbox}
\begin{tcolorbox}
\textsubscript{25} (106-25) Он речет, --и восстанет бурный ветер и высоко поднимает волны его:
\end{tcolorbox}
\begin{tcolorbox}
\textsubscript{26} (106-26) восходят до небес, нисходят до бездны; душа их истаевает в бедствии;
\end{tcolorbox}
\begin{tcolorbox}
\textsubscript{27} (106-27) они кружатся и шатаются, как пьяные, и вся мудрость их исчезает.
\end{tcolorbox}
\begin{tcolorbox}
\textsubscript{28} (106-28) Но воззвали к Господу в скорби своей, и Он вывел их из бедствия их.
\end{tcolorbox}
\begin{tcolorbox}
\textsubscript{29} (106-29) Он превращает бурю в тишину, и волны умолкают.
\end{tcolorbox}
\begin{tcolorbox}
\textsubscript{30} (106-30) И веселятся, что они утихли, и Он приводит их к желаемой пристани.
\end{tcolorbox}
\begin{tcolorbox}
\textsubscript{31} (106-31) Да славят Господа за милость Его и за чудные дела Его для сынов человеческих!
\end{tcolorbox}
\begin{tcolorbox}
\textsubscript{32} (106-32) Да превозносят Его в собрании народном и да славят Его в сонме старейшин!
\end{tcolorbox}
\begin{tcolorbox}
\textsubscript{33} (106-33) Он превращает реки в пустыню и источники вод--в сушу,
\end{tcolorbox}
\begin{tcolorbox}
\textsubscript{34} (106-34) землю плодородную--в солончатую, за нечестие живущих на ней.
\end{tcolorbox}
\begin{tcolorbox}
\textsubscript{35} (106-35) Он превращает пустыню в озеро, и землю иссохшую--в источники вод;
\end{tcolorbox}
\begin{tcolorbox}
\textsubscript{36} (106-36) и поселяет там алчущих, и они строят город для обитания;
\end{tcolorbox}
\begin{tcolorbox}
\textsubscript{37} (106-37) засевают поля, насаждают виноградники, которые приносят им обильные плоды.
\end{tcolorbox}
\begin{tcolorbox}
\textsubscript{38} (106-38) Он благословляет их, и они весьма размножаются, и скота их не умаляет.
\end{tcolorbox}
\begin{tcolorbox}
\textsubscript{39} (106-39) Уменьшились они и упали от угнетения, бедствия и скорби, --
\end{tcolorbox}
\begin{tcolorbox}
\textsubscript{40} (106-40) он изливает бесчестие на князей и оставляет их блуждать в пустыне, где нет путей.
\end{tcolorbox}
\begin{tcolorbox}
\textsubscript{41} (106-41) Бедного же извлекает из бедствия и умножает род его, как стада овец.
\end{tcolorbox}
\begin{tcolorbox}
\textsubscript{42} (106-42) Праведники видят сие и радуются, а всякое нечестие заграждает уста свои.
\end{tcolorbox}
\begin{tcolorbox}
\textsubscript{43} (106-43) Кто мудр, тот заметит сие и уразумеет милость Господа.
\end{tcolorbox}
\subsection{CHAPTER 108}
\begin{tcolorbox}
\textsubscript{1} (107-1) ^^Песнь. Псалом Давида.^^ (107-2) Готово сердце мое, Боже; буду петь и воспевать во славе моей.
\end{tcolorbox}
\begin{tcolorbox}
\textsubscript{2} (107-3) Воспрянь, псалтирь и гусли! Я встану рано.
\end{tcolorbox}
\begin{tcolorbox}
\textsubscript{3} (107-4) Буду славить Тебя, Господи, между народами; буду воспевать Тебя среди племен,
\end{tcolorbox}
\begin{tcolorbox}
\textsubscript{4} (107-5) ибо превыше небес милость Твоя и до облаков истина Твоя.
\end{tcolorbox}
\begin{tcolorbox}
\textsubscript{5} (107-6) Будь превознесен выше небес, Боже; над всею землею [да] [будет] слава Твоя,
\end{tcolorbox}
\begin{tcolorbox}
\textsubscript{6} (107-7) дабы избавились возлюбленные Твои: спаси десницею Твоею и услышь меня.
\end{tcolorbox}
\begin{tcolorbox}
\textsubscript{7} (107-8) Бог сказал во святилище Своем: 'восторжествую, разделю Сихем и долину Сокхоф размерю;
\end{tcolorbox}
\begin{tcolorbox}
\textsubscript{8} (107-9) Мой Галаад, Мой Манассия, Ефрем--крепость главы Моей, Иуда--скипетр Мой,
\end{tcolorbox}
\begin{tcolorbox}
\textsubscript{9} (107-10) Моав--умывальная чаша Моя, на Едома простру сапог Мой, над землею Филистимскою восклицать буду'.
\end{tcolorbox}
\begin{tcolorbox}
\textsubscript{10} (107-11) Кто введет меня в укрепленный город? Кто доведет меня до Едома?
\end{tcolorbox}
\begin{tcolorbox}
\textsubscript{11} (107-12) Не Ты ли, Боже, [Который] отринул нас и не выходишь, Боже, с войсками нашими?
\end{tcolorbox}
\begin{tcolorbox}
\textsubscript{12} (107-13) Подай нам помощь в тесноте, ибо защита человеческая суетна.
\end{tcolorbox}
\begin{tcolorbox}
\textsubscript{13} (107-14) С Богом мы окажем силу: Он низложит врагов наших.
\end{tcolorbox}
\subsection{CHAPTER 109}
\begin{tcolorbox}
\textsubscript{1} (108-1) ^^Начальнику хора. Псалом Давида.^^ Боже хвалы моей! не премолчи,
\end{tcolorbox}
\begin{tcolorbox}
\textsubscript{2} (108-2) ибо отверзлись на меня уста нечестивые и уста коварные; говорят со мною языком лживым;
\end{tcolorbox}
\begin{tcolorbox}
\textsubscript{3} (108-3) отвсюду окружают меня словами ненависти, вооружаются против меня без причины;
\end{tcolorbox}
\begin{tcolorbox}
\textsubscript{4} (108-4) за любовь мою они враждуют на меня, а я молюсь;
\end{tcolorbox}
\begin{tcolorbox}
\textsubscript{5} (108-5) воздают мне за добро злом, за любовь мою--ненавистью.
\end{tcolorbox}
\begin{tcolorbox}
\textsubscript{6} (108-6) Поставь над ним нечестивого, и диавол да станет одесную его.
\end{tcolorbox}
\begin{tcolorbox}
\textsubscript{7} (108-7) Когда будет судиться, да выйдет виновным, и молитва его да будет в грех;
\end{tcolorbox}
\begin{tcolorbox}
\textsubscript{8} (108-8) да будут дни его кратки, и достоинство его да возьмет другой;
\end{tcolorbox}
\begin{tcolorbox}
\textsubscript{9} (108-9) дети его да будут сиротами, и жена его--вдовою;
\end{tcolorbox}
\begin{tcolorbox}
\textsubscript{10} (108-10) да скитаются дети его и нищенствуют, и просят [хлеба] из развалин своих;
\end{tcolorbox}
\begin{tcolorbox}
\textsubscript{11} (108-11) да захватит заимодавец все, что есть у него, и чужие да расхитят труд его;
\end{tcolorbox}
\begin{tcolorbox}
\textsubscript{12} (108-12) да не будет сострадающего ему, да не будет милующего сирот его;
\end{tcolorbox}
\begin{tcolorbox}
\textsubscript{13} (108-13) да будет потомство его на погибель, и да изгладится имя их в следующем роде;
\end{tcolorbox}
\begin{tcolorbox}
\textsubscript{14} (108-14) да будет воспомянуто пред Господом беззаконие отцов его, и грех матери его да не изгладится;
\end{tcolorbox}
\begin{tcolorbox}
\textsubscript{15} (108-15) да будут они всегда в очах Господа, и да истребит Он память их на земле,
\end{tcolorbox}
\begin{tcolorbox}
\textsubscript{16} (108-16) за то, что он не думал оказывать милость, но преследовал человека бедного и нищего и сокрушенного сердцем, чтобы умертвить его;
\end{tcolorbox}
\begin{tcolorbox}
\textsubscript{17} (108-17) возлюбил проклятие, --оно и придет на него; не восхотел благословения, --оно и удалится от него;
\end{tcolorbox}
\begin{tcolorbox}
\textsubscript{18} (108-18) да облечется проклятием, как ризою, и да войдет оно, как вода, во внутренность его и, как елей, в кости его;
\end{tcolorbox}
\begin{tcolorbox}
\textsubscript{19} (108-19) да будет оно ему, как одежда, в которую он одевается, и как пояс, которым всегда опоясывается.
\end{tcolorbox}
\begin{tcolorbox}
\textsubscript{20} (108-20) Таково воздаяние от Господа врагам моим и говорящим злое на душу мою!
\end{tcolorbox}
\begin{tcolorbox}
\textsubscript{21} (108-21) Со мною же, Господи, Господи, твори ради имени Твоего, ибо блага милость Твоя; спаси меня,
\end{tcolorbox}
\begin{tcolorbox}
\textsubscript{22} (108-22) ибо я беден и нищ, и сердце мое уязвлено во мне.
\end{tcolorbox}
\begin{tcolorbox}
\textsubscript{23} (108-23) Я исчезаю, как уклоняющаяся тень; гонят меня, как саранчу.
\end{tcolorbox}
\begin{tcolorbox}
\textsubscript{24} (108-24) Колени мои изнемогли от поста, и тело мое лишилось тука.
\end{tcolorbox}
\begin{tcolorbox}
\textsubscript{25} (108-25) Я стал для них посмешищем: увидев меня, кивают головами.
\end{tcolorbox}
\begin{tcolorbox}
\textsubscript{26} (108-26) Помоги мне, Господи, Боже мой, спаси меня по милости Твоей,
\end{tcolorbox}
\begin{tcolorbox}
\textsubscript{27} (108-27) да познают, что это--Твоя рука, и что Ты, Господи, соделал это.
\end{tcolorbox}
\begin{tcolorbox}
\textsubscript{28} (108-28) Они проклинают, а Ты благослови; они восстают, но да будут постыжены; раб же Твой да возрадуется.
\end{tcolorbox}
\begin{tcolorbox}
\textsubscript{29} (108-29) Да облекутся противники мои бесчестьем и, как одеждою, покроются стыдом своим.
\end{tcolorbox}
\begin{tcolorbox}
\textsubscript{30} (108-30) И я громко буду устами моими славить Господа и среди множества прославлять Его,
\end{tcolorbox}
\begin{tcolorbox}
\textsubscript{31} (108-31) ибо Он стоит одесную бедного, чтобы спасти его от судящих душу его.
\end{tcolorbox}
\subsection{CHAPTER 110}
\begin{tcolorbox}
\textsubscript{1} (109-1) ^^Псалом Давида.^^ Сказал Господь Господу моему: седи одесную Меня, доколе положу врагов Твоих в подножие ног Твоих.
\end{tcolorbox}
\begin{tcolorbox}
\textsubscript{2} (109-2) Жезл силы Твоей пошлет Господь с Сиона: господствуй среди врагов Твоих.
\end{tcolorbox}
\begin{tcolorbox}
\textsubscript{3} (109-3) В день силы Твоей народ Твой готов во благолепии святыни; из чрева прежде денницы подобно росе рождение Твое.
\end{tcolorbox}
\begin{tcolorbox}
\textsubscript{4} (109-4) Клялся Господь и не раскается: Ты священник вовек по чину Мелхиседека.
\end{tcolorbox}
\begin{tcolorbox}
\textsubscript{5} (109-5) Господь одесную Тебя. Он в день гнева Своего поразит царей;
\end{tcolorbox}
\begin{tcolorbox}
\textsubscript{6} (109-6) совершит суд над народами, наполнит [землю] трупами, сокрушит голову в земле обширной.
\end{tcolorbox}
\begin{tcolorbox}
\textsubscript{7} (109-7) Из потока на пути будет пить, и потому вознесет главу.
\end{tcolorbox}
\subsection{CHAPTER 111}
\begin{tcolorbox}
\textsubscript{1} (110-1) ^^Аллилуия.^^ Славлю [Тебя], Господи, всем сердцем [моим] в совете праведных и в собрании.
\end{tcolorbox}
\begin{tcolorbox}
\textsubscript{2} (110-2) Велики дела Господни, вожделенны для всех, любящих оные.
\end{tcolorbox}
\begin{tcolorbox}
\textsubscript{3} (110-3) Дело Его--слава и красота, и правда Его пребывает вовек.
\end{tcolorbox}
\begin{tcolorbox}
\textsubscript{4} (110-4) Памятными соделал Он чудеса Свои; милостив и щедр Господь.
\end{tcolorbox}
\begin{tcolorbox}
\textsubscript{5} (110-5) Пищу дает боящимся Его; вечно помнит завет Свой.
\end{tcolorbox}
\begin{tcolorbox}
\textsubscript{6} (110-6) Силу дел Своих явил Он народу Своему, чтобы дать ему наследие язычников.
\end{tcolorbox}
\begin{tcolorbox}
\textsubscript{7} (110-7) Дела рук Его--истина и суд; все заповеди Его верны,
\end{tcolorbox}
\begin{tcolorbox}
\textsubscript{8} (110-8) тверды на веки и веки, основаны на истине и правоте.
\end{tcolorbox}
\begin{tcolorbox}
\textsubscript{9} (110-9) Избавление послал Он народу Своему; заповедал на веки завет Свой. Свято и страшно имя Его!
\end{tcolorbox}
\begin{tcolorbox}
\textsubscript{10} (110-10) Начало мудрости--страх Господень; разум верный у всех, исполняющих [заповеди Его]. Хвала Ему пребудет вовек.
\end{tcolorbox}
\subsection{CHAPTER 112}
\begin{tcolorbox}
\textsubscript{1} (111-1) ^^Аллилуия.^^ Блажен муж, боящийся Господа и крепко любящий заповеди Его.
\end{tcolorbox}
\begin{tcolorbox}
\textsubscript{2} (111-2) Сильно будет на земле семя его; род правых благословится.
\end{tcolorbox}
\begin{tcolorbox}
\textsubscript{3} (111-3) Обилие и богатство в доме его, и правда его пребывает вовек.
\end{tcolorbox}
\begin{tcolorbox}
\textsubscript{4} (111-4) Во тьме восходит свет правым; благ он и милосерд и праведен.
\end{tcolorbox}
\begin{tcolorbox}
\textsubscript{5} (111-5) Добрый человек милует и взаймы дает; он даст твердость словам своим на суде.
\end{tcolorbox}
\begin{tcolorbox}
\textsubscript{6} (111-6) Он вовек не поколеблется; в вечной памяти будет праведник.
\end{tcolorbox}
\begin{tcolorbox}
\textsubscript{7} (111-7) Не убоится худой молвы: сердце его твердо, уповая на Господа.
\end{tcolorbox}
\begin{tcolorbox}
\textsubscript{8} (111-8) Утверждено сердце его: он не убоится, когда посмотрит на врагов своих.
\end{tcolorbox}
\begin{tcolorbox}
\textsubscript{9} (111-9) Он расточил, роздал нищим; правда его пребывает во веки; рог его вознесется во славе.
\end{tcolorbox}
\begin{tcolorbox}
\textsubscript{10} (111-10) Нечестивый увидит [это] и будет досадовать, заскрежещет зубами своими и истает. Желание нечестивых погибнет.
\end{tcolorbox}
\subsection{CHAPTER 113}
\begin{tcolorbox}
\textsubscript{1} (112-1) ^^Аллилуия.^^ Хвалите, рабы Господни, хвалите имя Господне.
\end{tcolorbox}
\begin{tcolorbox}
\textsubscript{2} (112-2) Да будет имя Господне благословенно отныне и вовек.
\end{tcolorbox}
\begin{tcolorbox}
\textsubscript{3} (112-3) От восхода солнца до запада [да будет] прославляемо имя Господне.
\end{tcolorbox}
\begin{tcolorbox}
\textsubscript{4} (112-4) Высок над всеми народами Господь; над небесами слава Его.
\end{tcolorbox}
\begin{tcolorbox}
\textsubscript{5} (112-5) Кто, как Господь, Бог наш, Который, обитая на высоте,
\end{tcolorbox}
\begin{tcolorbox}
\textsubscript{6} (112-6) приклоняется, чтобы призирать на небо и на землю;
\end{tcolorbox}
\begin{tcolorbox}
\textsubscript{7} (112-7) из праха поднимает бедного, из брения возвышает нищего,
\end{tcolorbox}
\begin{tcolorbox}
\textsubscript{8} (112-8) чтобы посадить его с князьями, с князьями народа его;
\end{tcolorbox}
\begin{tcolorbox}
\textsubscript{9} (112-9) неплодную вселяет в дом матерью, радующеюся о детях? Аллилуия!
\end{tcolorbox}
\subsection{CHAPTER 114}
\begin{tcolorbox}
\textsubscript{1} (113-1) Когда вышел Израиль из Египта, дом Иакова--из народа иноплеменного,
\end{tcolorbox}
\begin{tcolorbox}
\textsubscript{2} (113-2) Иуда сделался святынею Его, Израиль--владением Его.
\end{tcolorbox}
\begin{tcolorbox}
\textsubscript{3} (113-3) Море увидело и побежало; Иордан обратился назад.
\end{tcolorbox}
\begin{tcolorbox}
\textsubscript{4} (113-4) Горы прыгали, как овны, и холмы, как агнцы.
\end{tcolorbox}
\begin{tcolorbox}
\textsubscript{5} (113-5) Что с тобою, море, что ты побежало, и [с тобою], Иордан, что ты обратился назад?
\end{tcolorbox}
\begin{tcolorbox}
\textsubscript{6} (113-6) Что вы прыгаете, горы, как овны, и вы, холмы, как агнцы?
\end{tcolorbox}
\begin{tcolorbox}
\textsubscript{7} (113-7) Пред лицем Господа трепещи, земля, пред лицем Бога Иаковлева,
\end{tcolorbox}
\begin{tcolorbox}
\textsubscript{8} (113-8) превращающего скалу в озеро воды и камень в источник вод.
\end{tcolorbox}
\subsection{CHAPTER 115}
\begin{tcolorbox}
\textsubscript{1} (113-9) Не нам, Господи, не нам, но имени Твоему дай славу, ради милости Твоей, ради истины Твоей.
\end{tcolorbox}
\begin{tcolorbox}
\textsubscript{2} (113-10) Для чего язычникам говорить: 'где же Бог их'?
\end{tcolorbox}
\begin{tcolorbox}
\textsubscript{3} (113-11) Бог наш на небесах; творит все, что хочет.
\end{tcolorbox}
\begin{tcolorbox}
\textsubscript{4} (113-12) А их идолы--серебро и золото, дело рук человеческих.
\end{tcolorbox}
\begin{tcolorbox}
\textsubscript{5} (113-13) Есть у них уста, но не говорят; есть у них глаза, но не видят;
\end{tcolorbox}
\begin{tcolorbox}
\textsubscript{6} (113-14) есть у них уши, но не слышат; есть у них ноздри, но не обоняют;
\end{tcolorbox}
\begin{tcolorbox}
\textsubscript{7} (113-15) есть у них руки, но не осязают; есть у них ноги, но не ходят; и они не издают голоса гортанью своею.
\end{tcolorbox}
\begin{tcolorbox}
\textsubscript{8} (113-16) Подобны им да будут делающие их и все, надеющиеся на них.
\end{tcolorbox}
\begin{tcolorbox}
\textsubscript{9} (113-17) [Дом] Израилев! уповай на Господа: Он наша помощь и щит.
\end{tcolorbox}
\begin{tcolorbox}
\textsubscript{10} (113-18) Дом Ааронов! уповай на Господа: Он наша помощь и щит.
\end{tcolorbox}
\begin{tcolorbox}
\textsubscript{11} (113-19) Боящиеся Господа! уповайте на Господа: Он наша помощь и щит.
\end{tcolorbox}
\begin{tcolorbox}
\textsubscript{12} (113-20) Господь помнит нас, благословляет [нас], благословляет дом Израилев, благословляет дом Ааронов;
\end{tcolorbox}
\begin{tcolorbox}
\textsubscript{13} (113-21) благословляет боящихся Господа, малых с великими.
\end{tcolorbox}
\begin{tcolorbox}
\textsubscript{14} (113-22) Да приложит вам Господь более и более, вам и детям вашим.
\end{tcolorbox}
\begin{tcolorbox}
\textsubscript{15} (113-23) Благословенны вы Господом, сотворившим небо и землю.
\end{tcolorbox}
\begin{tcolorbox}
\textsubscript{16} (113-24) Небо--небо Господу, а землю Он дал сынам человеческим.
\end{tcolorbox}
\begin{tcolorbox}
\textsubscript{17} (113-25) Ни мертвые восхвалят Господа, ни все нисходящие в могилу;
\end{tcolorbox}
\begin{tcolorbox}
\textsubscript{18} (113-26) но мы будем благословлять Господа отныне и вовек. Аллилуия.
\end{tcolorbox}
\subsection{CHAPTER 116}
\begin{tcolorbox}
\textsubscript{1} (114-1) Я радуюсь, что Господь услышал голос мой, моление мое;
\end{tcolorbox}
\begin{tcolorbox}
\textsubscript{2} (114-2) приклонил ко мне ухо Свое, и потому буду призывать Его во [все] дни мои.
\end{tcolorbox}
\begin{tcolorbox}
\textsubscript{3} (114-3) Объяли меня болезни смертные, муки адские постигли меня; я встретил тесноту и скорбь.
\end{tcolorbox}
\begin{tcolorbox}
\textsubscript{4} (114-4) Тогда призвал я имя Господне: Господи! избавь душу мою.
\end{tcolorbox}
\begin{tcolorbox}
\textsubscript{5} (114-5) Милостив Господь и праведен, и милосерд Бог наш.
\end{tcolorbox}
\begin{tcolorbox}
\textsubscript{6} (114-6) Хранит Господь простодушных: я изнемог, и Он помог мне.
\end{tcolorbox}
\begin{tcolorbox}
\textsubscript{7} (114-7) Возвратись, душа моя, в покой твой, ибо Господь облагодетельствовал тебя.
\end{tcolorbox}
\begin{tcolorbox}
\textsubscript{8} (114-8) Ты избавил душу мою от смерти, очи мои от слез и ноги мои от преткновения.
\end{tcolorbox}
\begin{tcolorbox}
\textsubscript{9} (114-9) Буду ходить пред лицем Господним на земле живых.
\end{tcolorbox}
\begin{tcolorbox}
\textsubscript{10} (115-1) Я веровал, и потому говорил: я сильно сокрушен.
\end{tcolorbox}
\begin{tcolorbox}
\textsubscript{11} (115-2) Я сказал в опрометчивости моей: всякий человек ложь.
\end{tcolorbox}
\begin{tcolorbox}
\textsubscript{12} (115-3) Что воздам Господу за все благодеяния Его ко мне?
\end{tcolorbox}
\begin{tcolorbox}
\textsubscript{13} (115-4) Чашу спасения прииму и имя Господне призову.
\end{tcolorbox}
\begin{tcolorbox}
\textsubscript{14} (115-5) Обеты мои воздам Господу пред всем народом Его.
\end{tcolorbox}
\begin{tcolorbox}
\textsubscript{15} (115-6) Дорога в очах Господних смерть святых Его!
\end{tcolorbox}
\begin{tcolorbox}
\textsubscript{16} (115-7) О, Господи! я раб Твой, я раб Твой и сын рабы Твоей; Ты разрешил узы мои.
\end{tcolorbox}
\begin{tcolorbox}
\textsubscript{17} (115-8) Тебе принесу жертву хвалы, и имя Господне призову.
\end{tcolorbox}
\begin{tcolorbox}
\textsubscript{18} (115-9) Обеты мои воздам Господу пред всем народом Его,
\end{tcolorbox}
\begin{tcolorbox}
\textsubscript{19} (115-10) во дворах дома Господня, посреди тебя, Иерусалим! Аллилуия.
\end{tcolorbox}
\subsection{CHAPTER 117}
\begin{tcolorbox}
\textsubscript{1} (116-1) Хвалите Господа, все народы, прославляйте Его, все племена;
\end{tcolorbox}
\begin{tcolorbox}
\textsubscript{2} (116-2) ибо велика милость Его к нам, и истина Господня вовек. Аллилуия.
\end{tcolorbox}
\subsection{CHAPTER 118}
\begin{tcolorbox}
\textsubscript{1} (117-1) Славьте Господа, ибо Он благ, ибо вовек милость Его.
\end{tcolorbox}
\begin{tcolorbox}
\textsubscript{2} (117-2) Да скажет ныне [дом] Израилев: ибо вовек милость Его.
\end{tcolorbox}
\begin{tcolorbox}
\textsubscript{3} (117-3) Да скажет ныне дом Ааронов: ибо вовек милость Его.
\end{tcolorbox}
\begin{tcolorbox}
\textsubscript{4} (117-4) Да скажут ныне боящиеся Господа: ибо вовек милость Его.
\end{tcolorbox}
\begin{tcolorbox}
\textsubscript{5} (117-5) Из тесноты воззвал я к Господу, --и услышал меня, и на пространное место [вывел меня] Господь.
\end{tcolorbox}
\begin{tcolorbox}
\textsubscript{6} (117-6) Господь за меня--не устрашусь: что сделает мне человек?
\end{tcolorbox}
\begin{tcolorbox}
\textsubscript{7} (117-7) Господь мне помощник: буду смотреть на врагов моих.
\end{tcolorbox}
\begin{tcolorbox}
\textsubscript{8} (117-8) Лучше уповать на Господа, нежели надеяться на человека.
\end{tcolorbox}
\begin{tcolorbox}
\textsubscript{9} (117-9) Лучше уповать на Господа, нежели надеяться на князей.
\end{tcolorbox}
\begin{tcolorbox}
\textsubscript{10} (117-10) Все народы окружили меня, но именем Господним я низложил их;
\end{tcolorbox}
\begin{tcolorbox}
\textsubscript{11} (117-11) обступили меня, окружили меня, но именем Господним я низложил их;
\end{tcolorbox}
\begin{tcolorbox}
\textsubscript{12} (117-12) окружили меня, как пчелы, и угасли, как огонь в терне: именем Господним я низложил их.
\end{tcolorbox}
\begin{tcolorbox}
\textsubscript{13} (117-13) Сильно толкнули меня, чтобы я упал, но Господь поддержал меня.
\end{tcolorbox}
\begin{tcolorbox}
\textsubscript{14} (117-14) Господь--сила моя и песнь; Он соделался моим спасением.
\end{tcolorbox}
\begin{tcolorbox}
\textsubscript{15} (117-15) Глас радости и спасения в жилищах праведников: десница Господня творит силу!
\end{tcolorbox}
\begin{tcolorbox}
\textsubscript{16} (117-16) Десница Господня высока, десница Господня творит силу!
\end{tcolorbox}
\begin{tcolorbox}
\textsubscript{17} (117-17) Не умру, но буду жить и возвещать дела Господни.
\end{tcolorbox}
\begin{tcolorbox}
\textsubscript{18} (117-18) Строго наказал меня Господь, но смерти не предал меня.
\end{tcolorbox}
\begin{tcolorbox}
\textsubscript{19} (117-19) Отворите мне врата правды; войду в них, прославлю Господа.
\end{tcolorbox}
\begin{tcolorbox}
\textsubscript{20} (117-20) Вот врата Господа; праведные войдут в них.
\end{tcolorbox}
\begin{tcolorbox}
\textsubscript{21} (117-21) Славлю Тебя, что Ты услышал меня и соделался моим спасением.
\end{tcolorbox}
\begin{tcolorbox}
\textsubscript{22} (117-22) Камень, который отвергли строители, соделался главою угла:
\end{tcolorbox}
\begin{tcolorbox}
\textsubscript{23} (117-23) это--от Господа, и есть дивно в очах наших.
\end{tcolorbox}
\begin{tcolorbox}
\textsubscript{24} (117-24) Сей день сотворил Господь: возрадуемся и возвеселимся в оный!
\end{tcolorbox}
\begin{tcolorbox}
\textsubscript{25} (117-25) О, Господи, спаси же! О, Господи, споспешествуй же!
\end{tcolorbox}
\begin{tcolorbox}
\textsubscript{26} (117-26) Благословен грядущий во имя Господне! Благословляем вас из дома Господня.
\end{tcolorbox}
\begin{tcolorbox}
\textsubscript{27} (117-27) Бог--Господь, и осиял нас; вяжите вервями жертву, [ведите] к рогам жертвенника.
\end{tcolorbox}
\begin{tcolorbox}
\textsubscript{28} (117-28) Ты Бог мой: буду славить Тебя; Ты Бог мой: буду превозносить Тебя.
\end{tcolorbox}
\begin{tcolorbox}
\textsubscript{29} (117-29) Славьте Господа, ибо Он благ, ибо вовек милость Его.
\end{tcolorbox}
\subsection{CHAPTER 119}
\begin{tcolorbox}
\textsubscript{1} (118-1) Блаженны непорочные в пути, ходящие в законе Господнем.
\end{tcolorbox}
\begin{tcolorbox}
\textsubscript{2} (118-2) Блаженны хранящие откровения Его, всем сердцем ищущие Его.
\end{tcolorbox}
\begin{tcolorbox}
\textsubscript{3} (118-3) Они не делают беззакония, ходят путями Его.
\end{tcolorbox}
\begin{tcolorbox}
\textsubscript{4} (118-4) Ты заповедал повеления Твои хранить твердо.
\end{tcolorbox}
\begin{tcolorbox}
\textsubscript{5} (118-5) О, если бы направлялись пути мои к соблюдению уставов Твоих!
\end{tcolorbox}
\begin{tcolorbox}
\textsubscript{6} (118-6) Тогда я не постыдился бы, взирая на все заповеди Твои:
\end{tcolorbox}
\begin{tcolorbox}
\textsubscript{7} (118-7) я славил бы Тебя в правоте сердца, поучаясь судам правды Твоей.
\end{tcolorbox}
\begin{tcolorbox}
\textsubscript{8} (118-8) Буду хранить уставы Твои; не оставляй меня совсем.
\end{tcolorbox}
\begin{tcolorbox}
\textsubscript{9} (118-9) Как юноше содержать в чистоте путь свой? --Хранением себя по слову Твоему.
\end{tcolorbox}
\begin{tcolorbox}
\textsubscript{10} (118-10) Всем сердцем моим ищу Тебя; не дай мне уклониться от заповедей Твоих.
\end{tcolorbox}
\begin{tcolorbox}
\textsubscript{11} (118-11) В сердце моем сокрыл я слово Твое, чтобы не грешить пред Тобою.
\end{tcolorbox}
\begin{tcolorbox}
\textsubscript{12} (118-12) Благословен Ты, Господи! научи меня уставам Твоим.
\end{tcolorbox}
\begin{tcolorbox}
\textsubscript{13} (118-13) Устами моими возвещал я все суды уст Твоих.
\end{tcolorbox}
\begin{tcolorbox}
\textsubscript{14} (118-14) На пути откровений Твоих я радуюсь, как во всяком богатстве.
\end{tcolorbox}
\begin{tcolorbox}
\textsubscript{15} (118-15) О заповедях Твоих размышляю, и взираю на пути Твои.
\end{tcolorbox}
\begin{tcolorbox}
\textsubscript{16} (118-16) Уставами Твоими утешаюсь, не забываю слова Твоего.
\end{tcolorbox}
\begin{tcolorbox}
\textsubscript{17} (118-17) Яви милость рабу Твоему, и буду жить и хранить слово Твое.
\end{tcolorbox}
\begin{tcolorbox}
\textsubscript{18} (118-18) Открой очи мои, и увижу чудеса закона Твоего.
\end{tcolorbox}
\begin{tcolorbox}
\textsubscript{19} (118-19) Странник я на земле; не скрывай от меня заповедей Твоих.
\end{tcolorbox}
\begin{tcolorbox}
\textsubscript{20} (118-20) Истомилась душа моя желанием судов Твоих во всякое время.
\end{tcolorbox}
\begin{tcolorbox}
\textsubscript{21} (118-21) Ты укротил гордых, проклятых, уклоняющихся от заповедей Твоих.
\end{tcolorbox}
\begin{tcolorbox}
\textsubscript{22} (118-22) Сними с меня поношение и посрамление, ибо я храню откровения Твои.
\end{tcolorbox}
\begin{tcolorbox}
\textsubscript{23} (118-23) Князья сидят и сговариваются против меня, а раб Твой размышляет об уставах Твоих.
\end{tcolorbox}
\begin{tcolorbox}
\textsubscript{24} (118-24) Откровения Твои--утешение мое, --советники мои.
\end{tcolorbox}
\begin{tcolorbox}
\textsubscript{25} (118-25) Душа моя повержена в прах; оживи меня по слову Твоему.
\end{tcolorbox}
\begin{tcolorbox}
\textsubscript{26} (118-26) Объявил я пути мои, и Ты услышал меня; научи меня уставам Твоим.
\end{tcolorbox}
\begin{tcolorbox}
\textsubscript{27} (118-27) Дай мне уразуметь путь повелений Твоих, и буду размышлять о чудесах Твоих.
\end{tcolorbox}
\begin{tcolorbox}
\textsubscript{28} (118-28) Душа моя истаевает от скорби: укрепи меня по слову Твоему.
\end{tcolorbox}
\begin{tcolorbox}
\textsubscript{29} (118-29) Удали от меня путь лжи, и закон Твой даруй мне.
\end{tcolorbox}
\begin{tcolorbox}
\textsubscript{30} (118-30) Я избрал путь истины, поставил пред собою суды Твои.
\end{tcolorbox}
\begin{tcolorbox}
\textsubscript{31} (118-31) Я прилепился к откровениям Твоим, Господи; не постыди меня.
\end{tcolorbox}
\begin{tcolorbox}
\textsubscript{32} (118-32) Потеку путем заповедей Твоих, когда Ты расширишь сердце мое.
\end{tcolorbox}
\begin{tcolorbox}
\textsubscript{33} (118-33) Укажи мне, Господи, путь уставов Твоих, и я буду держаться его до конца.
\end{tcolorbox}
\begin{tcolorbox}
\textsubscript{34} (118-34) Вразуми меня, и буду соблюдать закон Твой и хранить его всем сердцем.
\end{tcolorbox}
\begin{tcolorbox}
\textsubscript{35} (118-35) Поставь меня на стезю заповедей Твоих, ибо я возжелал ее.
\end{tcolorbox}
\begin{tcolorbox}
\textsubscript{36} (118-36) Приклони сердце мое к откровениям Твоим, а не к корысти.
\end{tcolorbox}
\begin{tcolorbox}
\textsubscript{37} (118-37) Отврати очи мои, чтобы не видеть суеты; животвори меня на пути Твоем.
\end{tcolorbox}
\begin{tcolorbox}
\textsubscript{38} (118-38) Утверди слово Твое рабу Твоему, ради благоговения пред Тобою.
\end{tcolorbox}
\begin{tcolorbox}
\textsubscript{39} (118-39) Отврати поношение мое, которого я страшусь, ибо суды Твои благи.
\end{tcolorbox}
\begin{tcolorbox}
\textsubscript{40} (118-40) Вот, я возжелал повелений Твоих; животвори меня правдою Твоею.
\end{tcolorbox}
\begin{tcolorbox}
\textsubscript{41} (118-41) Да придут ко мне милости Твои, Господи, спасение Твое по слову Твоему, --
\end{tcolorbox}
\begin{tcolorbox}
\textsubscript{42} (118-42) и я дам ответ поносящему меня, ибо уповаю на слово Твое.
\end{tcolorbox}
\begin{tcolorbox}
\textsubscript{43} (118-43) Не отнимай совсем от уст моих слова истины, ибо я уповаю на суды Твои
\end{tcolorbox}
\begin{tcolorbox}
\textsubscript{44} (118-44) и буду хранить закон Твой всегда, во веки и веки;
\end{tcolorbox}
\begin{tcolorbox}
\textsubscript{45} (118-45) буду ходить свободно, ибо я взыскал повелений Твоих;
\end{tcolorbox}
\begin{tcolorbox}
\textsubscript{46} (118-46) буду говорить об откровениях Твоих пред царями и не постыжусь;
\end{tcolorbox}
\begin{tcolorbox}
\textsubscript{47} (118-47) буду утешаться заповедями Твоими, которые возлюбил;
\end{tcolorbox}
\begin{tcolorbox}
\textsubscript{48} (118-48) руки мои буду простирать к заповедям Твоим, которые возлюбил, и размышлять об уставах Твоих.
\end{tcolorbox}
\begin{tcolorbox}
\textsubscript{49} (118-49) Вспомни слово Твое к рабу Твоему, на которое Ты повелел мне уповать:
\end{tcolorbox}
\begin{tcolorbox}
\textsubscript{50} (118-50) это--утешение в бедствии моем, что слово Твое оживляет меня.
\end{tcolorbox}
\begin{tcolorbox}
\textsubscript{51} (118-51) Гордые крайне ругались надо мною, но я не уклонился от закона Твоего.
\end{tcolorbox}
\begin{tcolorbox}
\textsubscript{52} (118-52) Вспоминал суды Твои, Господи, от века, и утешался.
\end{tcolorbox}
\begin{tcolorbox}
\textsubscript{53} (118-53) Ужас овладевает мною при виде нечестивых, оставляющих закон Твой.
\end{tcolorbox}
\begin{tcolorbox}
\textsubscript{54} (118-54) Уставы Твои были песнями моими на месте странствований моих.
\end{tcolorbox}
\begin{tcolorbox}
\textsubscript{55} (118-55) Ночью вспоминал я имя Твое, Господи, и хранил закон Твой.
\end{tcolorbox}
\begin{tcolorbox}
\textsubscript{56} (118-56) Он стал моим, ибо повеления Твои храню.
\end{tcolorbox}
\begin{tcolorbox}
\textsubscript{57} (118-57) Удел мой, Господи, сказал я, соблюдать слова Твои.
\end{tcolorbox}
\begin{tcolorbox}
\textsubscript{58} (118-58) Молился я Тебе всем сердцем: помилуй меня по слову Твоему.
\end{tcolorbox}
\begin{tcolorbox}
\textsubscript{59} (118-59) Размышлял о путях моих и обращал стопы мои к откровениям Твоим.
\end{tcolorbox}
\begin{tcolorbox}
\textsubscript{60} (118-60) Спешил и не медлил соблюдать заповеди Твои.
\end{tcolorbox}
\begin{tcolorbox}
\textsubscript{61} (118-61) Сети нечестивых окружили меня, но я не забывал закона Твоего.
\end{tcolorbox}
\begin{tcolorbox}
\textsubscript{62} (118-62) В полночь вставал славословить Тебя за праведные суды Твои.
\end{tcolorbox}
\begin{tcolorbox}
\textsubscript{63} (118-63) Общник я всем боящимся Тебя и хранящим повеления Твои.
\end{tcolorbox}
\begin{tcolorbox}
\textsubscript{64} (118-64) Милости Твоей, Господи, полна земля; научи меня уставам Твоим.
\end{tcolorbox}
\begin{tcolorbox}
\textsubscript{65} (118-65) Благо сотворил Ты рабу Твоему, Господи, по слову Твоему.
\end{tcolorbox}
\begin{tcolorbox}
\textsubscript{66} (118-66) Доброму разумению и ведению научи меня, ибо заповедям Твоим я верую.
\end{tcolorbox}
\begin{tcolorbox}
\textsubscript{67} (118-67) Прежде страдания моего я заблуждался; а ныне слово Твое храню.
\end{tcolorbox}
\begin{tcolorbox}
\textsubscript{68} (118-68) Благ и благодетелен Ты, --научи меня уставам Твоим.
\end{tcolorbox}
\begin{tcolorbox}
\textsubscript{69} (118-69) Гордые сплетают на меня ложь; я же всем сердцем буду хранить повеления Твои.
\end{tcolorbox}
\begin{tcolorbox}
\textsubscript{70} (118-70) Ожирело сердце их, как тук; я же законом Твоим утешаюсь.
\end{tcolorbox}
\begin{tcolorbox}
\textsubscript{71} (118-71) Благо мне, что я пострадал, дабы научиться уставам Твоим.
\end{tcolorbox}
\begin{tcolorbox}
\textsubscript{72} (118-72) Закон уст Твоих для меня лучше тысяч золота и серебра.
\end{tcolorbox}
\begin{tcolorbox}
\textsubscript{73} (118-73) Руки Твои сотворили меня и устроили меня; вразуми меня, и научусь заповедям Твоим.
\end{tcolorbox}
\begin{tcolorbox}
\textsubscript{74} (118-74) Боящиеся Тебя увидят меня--и возрадуются, что я уповаю на слово Твое.
\end{tcolorbox}
\begin{tcolorbox}
\textsubscript{75} (118-75) Знаю, Господи, что суды Твои праведны и по справедливости Ты наказал меня.
\end{tcolorbox}
\begin{tcolorbox}
\textsubscript{76} (118-76) Да будет же милость Твоя утешением моим, по слову Твоему к рабу Твоему.
\end{tcolorbox}
\begin{tcolorbox}
\textsubscript{77} (118-77) Да придет ко мне милосердие Твое, и я буду жить; ибо закон Твой--утешение мое.
\end{tcolorbox}
\begin{tcolorbox}
\textsubscript{78} (118-78) Да будут постыжены гордые, ибо безвинно угнетают меня; я размышляю о повелениях Твоих.
\end{tcolorbox}
\begin{tcolorbox}
\textsubscript{79} (118-79) Да обратятся ко мне боящиеся Тебя и знающие откровения Твои.
\end{tcolorbox}
\begin{tcolorbox}
\textsubscript{80} (118-80) Да будет сердце мое непорочно в уставах Твоих, чтобы я не посрамился.
\end{tcolorbox}
\begin{tcolorbox}
\textsubscript{81} (118-81) Истаевает душа моя о спасении Твоем; уповаю на слово Твое.
\end{tcolorbox}
\begin{tcolorbox}
\textsubscript{82} (118-82) Истаевают очи мои о слове Твоем; я говорю: когда Ты утешишь меня?
\end{tcolorbox}
\begin{tcolorbox}
\textsubscript{83} (118-83) Я стал, как мех в дыму, [но] уставов Твоих не забыл.
\end{tcolorbox}
\begin{tcolorbox}
\textsubscript{84} (118-84) Сколько дней раба Твоего? Когда произведешь суд над гонителями моими?
\end{tcolorbox}
\begin{tcolorbox}
\textsubscript{85} (118-85) Яму вырыли мне гордые, вопреки закону Твоему.
\end{tcolorbox}
\begin{tcolorbox}
\textsubscript{86} (118-86) Все заповеди Твои--истина; несправедливо преследуют меня: помоги мне;
\end{tcolorbox}
\begin{tcolorbox}
\textsubscript{87} (118-87) едва не погубили меня на земле, но я не оставил повелений Твоих.
\end{tcolorbox}
\begin{tcolorbox}
\textsubscript{88} (118-88) По милости Твоей оживляй меня, и буду хранить откровения уст Твоих.
\end{tcolorbox}
\begin{tcolorbox}
\textsubscript{89} (118-89) На веки, Господи, слово Твое утверждено на небесах;
\end{tcolorbox}
\begin{tcolorbox}
\textsubscript{90} (118-90) истина Твоя в род и род. Ты поставил землю, и она стоит.
\end{tcolorbox}
\begin{tcolorbox}
\textsubscript{91} (118-91) По определениям Твоим все стоит доныне, ибо все служит Тебе.
\end{tcolorbox}
\begin{tcolorbox}
\textsubscript{92} (118-92) Если бы не закон Твой был утешением моим, погиб бы я в бедствии моем.
\end{tcolorbox}
\begin{tcolorbox}
\textsubscript{93} (118-93) Вовек не забуду повелений Твоих, ибо ими Ты оживляешь меня.
\end{tcolorbox}
\begin{tcolorbox}
\textsubscript{94} (118-94) Твой я, спаси меня; ибо я взыскал повелений Твоих.
\end{tcolorbox}
\begin{tcolorbox}
\textsubscript{95} (118-95) Нечестивые подстерегают меня, чтобы погубить; [а] я углубляюсь в откровения Твои.
\end{tcolorbox}
\begin{tcolorbox}
\textsubscript{96} (118-96) Я видел предел всякого совершенства, [но] Твоя заповедь безмерно обширна.
\end{tcolorbox}
\begin{tcolorbox}
\textsubscript{97} (118-97) Как люблю я закон Твой! весь день размышляю о нем.
\end{tcolorbox}
\begin{tcolorbox}
\textsubscript{98} (118-98) Заповедью Твоею Ты соделал меня мудрее врагов моих, ибо она всегда со мною.
\end{tcolorbox}
\begin{tcolorbox}
\textsubscript{99} (118-99) Я стал разумнее всех учителей моих, ибо размышляю об откровениях Твоих.
\end{tcolorbox}
\begin{tcolorbox}
\textsubscript{100} (118-100) Я сведущ более старцев, ибо повеления Твои храню.
\end{tcolorbox}
\begin{tcolorbox}
\textsubscript{101} (118-101) От всякого злого пути удерживаю ноги мои, чтобы хранить слово Твое;
\end{tcolorbox}
\begin{tcolorbox}
\textsubscript{102} (118-102) от судов Твоих не уклоняюсь, ибо Ты научаешь меня.
\end{tcolorbox}
\begin{tcolorbox}
\textsubscript{103} (118-103) Как сладки гортани моей слова Твои! лучше меда устам моим.
\end{tcolorbox}
\begin{tcolorbox}
\textsubscript{104} (118-104) Повелениями Твоими я вразумлен; потому ненавижу всякий путь лжи.
\end{tcolorbox}
\begin{tcolorbox}
\textsubscript{105} (118-105) Слово Твое--светильник ноге моей и свет стезе моей.
\end{tcolorbox}
\begin{tcolorbox}
\textsubscript{106} (118-106) Я клялся хранить праведные суды Твои, и исполню.
\end{tcolorbox}
\begin{tcolorbox}
\textsubscript{107} (118-107) Сильно угнетен я, Господи; оживи меня по слову Твоему.
\end{tcolorbox}
\begin{tcolorbox}
\textsubscript{108} (118-108) Благоволи же, Господи, принять добровольную жертву уст моих, и судам Твоим научи меня.
\end{tcolorbox}
\begin{tcolorbox}
\textsubscript{109} (118-109) Душа моя непрестанно в руке моей, но закона Твоего не забываю.
\end{tcolorbox}
\begin{tcolorbox}
\textsubscript{110} (118-110) Нечестивые поставили для меня сеть, но я не уклонился от повелений Твоих.
\end{tcolorbox}
\begin{tcolorbox}
\textsubscript{111} (118-111) Откровения Твои я принял, как наследие на веки, ибо они веселие сердца моего.
\end{tcolorbox}
\begin{tcolorbox}
\textsubscript{112} (118-112) Я приклонил сердце мое к исполнению уставов Твоих навек, до конца.
\end{tcolorbox}
\begin{tcolorbox}
\textsubscript{113} (118-113) Вымыслы [человеческие] ненавижу, а закон Твой люблю.
\end{tcolorbox}
\begin{tcolorbox}
\textsubscript{114} (118-114) Ты покров мой и щит мой; на слово Твое уповаю.
\end{tcolorbox}
\begin{tcolorbox}
\textsubscript{115} (118-115) Удалитесь от меня, беззаконные, и буду хранить заповеди Бога моего.
\end{tcolorbox}
\begin{tcolorbox}
\textsubscript{116} (118-116) Укрепи меня по слову Твоему, и буду жить; не посрами меня в надежде моей;
\end{tcolorbox}
\begin{tcolorbox}
\textsubscript{117} (118-117) поддержи меня, и спасусь; и в уставы Твои буду вникать непрестанно.
\end{tcolorbox}
\begin{tcolorbox}
\textsubscript{118} (118-118) Всех, отступающих от уставов Твоих, Ты низлагаешь, ибо ухищрения их--ложь.
\end{tcolorbox}
\begin{tcolorbox}
\textsubscript{119} (118-119) [Как] изгарь, отметаешь Ты всех нечестивых земли; потому я возлюбил откровения Твои.
\end{tcolorbox}
\begin{tcolorbox}
\textsubscript{120} (118-120) Трепещет от страха Твоего плоть моя, и судов Твоих я боюсь.
\end{tcolorbox}
\begin{tcolorbox}
\textsubscript{121} (118-121) Я совершал суд и правду; не предай меня гонителям моим.
\end{tcolorbox}
\begin{tcolorbox}
\textsubscript{122} (118-122) Заступи раба Твоего ко благу [его], чтобы не угнетали меня гордые.
\end{tcolorbox}
\begin{tcolorbox}
\textsubscript{123} (118-123) Истаевают очи мои, ожидая спасения Твоего и слова правды Твоей.
\end{tcolorbox}
\begin{tcolorbox}
\textsubscript{124} (118-124) Сотвори с рабом Твоим по милости Твоей, и уставам Твоим научи меня.
\end{tcolorbox}
\begin{tcolorbox}
\textsubscript{125} (118-125) Я раб Твой: вразуми меня, и познаю откровения Твои.
\end{tcolorbox}
\begin{tcolorbox}
\textsubscript{126} (118-126) Время Господу действовать: закон Твой разорили.
\end{tcolorbox}
\begin{tcolorbox}
\textsubscript{127} (118-127) А я люблю заповеди Твои более золота, и золота чистого.
\end{tcolorbox}
\begin{tcolorbox}
\textsubscript{128} (118-128) Все повеления Твои--все признаю справедливыми; всякий путь лжи ненавижу.
\end{tcolorbox}
\begin{tcolorbox}
\textsubscript{129} (118-129) Дивны откровения Твои; потому хранит их душа моя.
\end{tcolorbox}
\begin{tcolorbox}
\textsubscript{130} (118-130) Откровение слов Твоих просвещает, вразумляет простых.
\end{tcolorbox}
\begin{tcolorbox}
\textsubscript{131} (118-131) Открываю уста мои и вздыхаю, ибо заповедей Твоих жажду.
\end{tcolorbox}
\begin{tcolorbox}
\textsubscript{132} (118-132) Призри на меня и помилуй меня, как поступаешь с любящими имя Твое.
\end{tcolorbox}
\begin{tcolorbox}
\textsubscript{133} (118-133) Утверди стопы мои в слове Твоем и не дай овладеть мною никакому беззаконию;
\end{tcolorbox}
\begin{tcolorbox}
\textsubscript{134} (118-134) избавь меня от угнетения человеческого, и буду хранить повеления Твои;
\end{tcolorbox}
\begin{tcolorbox}
\textsubscript{135} (118-135) осияй раба Твоего светом лица Твоего и научи меня уставам Твоим.
\end{tcolorbox}
\begin{tcolorbox}
\textsubscript{136} (118-136) Из глаз моих текут потоки вод оттого, что не хранят закона Твоего.
\end{tcolorbox}
\begin{tcolorbox}
\textsubscript{137} (118-137) Праведен Ты, Господи, и справедливы суды Твои.
\end{tcolorbox}
\begin{tcolorbox}
\textsubscript{138} (118-138) Откровения Твои, которые Ты заповедал, --правда и совершенная истина.
\end{tcolorbox}
\begin{tcolorbox}
\textsubscript{139} (118-139) Ревность моя снедает меня, потому что мои враги забыли слова Твои.
\end{tcolorbox}
\begin{tcolorbox}
\textsubscript{140} (118-140) Слово Твое весьма чисто, и раб Твой возлюбил его.
\end{tcolorbox}
\begin{tcolorbox}
\textsubscript{141} (118-141) Мал я и презрен, [но] повелений Твоих не забываю.
\end{tcolorbox}
\begin{tcolorbox}
\textsubscript{142} (118-142) Правда Твоя--правда вечная, и закон Твой--истина.
\end{tcolorbox}
\begin{tcolorbox}
\textsubscript{143} (118-143) Скорбь и горесть постигли меня; заповеди Твои--утешение мое.
\end{tcolorbox}
\begin{tcolorbox}
\textsubscript{144} (118-144) Правда откровений Твоих вечна: вразуми меня, и буду жить.
\end{tcolorbox}
\begin{tcolorbox}
\textsubscript{145} (118-145) Взываю всем сердцем [моим]: услышь меня, Господи, --и сохраню уставы Твои.
\end{tcolorbox}
\begin{tcolorbox}
\textsubscript{146} (118-146) Призываю Тебя: спаси меня, и буду хранить откровения Твои.
\end{tcolorbox}
\begin{tcolorbox}
\textsubscript{147} (118-147) Предваряю рассвет и взываю; на слово Твое уповаю.
\end{tcolorbox}
\begin{tcolorbox}
\textsubscript{148} (118-148) Очи мои предваряют [утреннюю] стражу, чтобы мне углубляться в слово Твое.
\end{tcolorbox}
\begin{tcolorbox}
\textsubscript{149} (118-149) Услышь голос мой по милости Твоей, Господи; по суду Твоему оживи меня.
\end{tcolorbox}
\begin{tcolorbox}
\textsubscript{150} (118-150) Приблизились замышляющие лукавство; далеки они от закона Твоего.
\end{tcolorbox}
\begin{tcolorbox}
\textsubscript{151} (118-151) Близок Ты, Господи, и все заповеди Твои--истина.
\end{tcolorbox}
\begin{tcolorbox}
\textsubscript{152} (118-152) Издавна узнал я об откровениях Твоих, что Ты утвердил их на веки.
\end{tcolorbox}
\begin{tcolorbox}
\textsubscript{153} (118-153) Воззри на бедствие мое и избавь меня, ибо я не забываю закона Твоего.
\end{tcolorbox}
\begin{tcolorbox}
\textsubscript{154} (118-154) Вступись в дело мое и защити меня; по слову Твоему оживи меня.
\end{tcolorbox}
\begin{tcolorbox}
\textsubscript{155} (118-155) Далеко от нечестивых спасение, ибо они уставов Твоих не ищут.
\end{tcolorbox}
\begin{tcolorbox}
\textsubscript{156} (118-156) Много щедрот Твоих, Господи; по суду Твоему оживи меня.
\end{tcolorbox}
\begin{tcolorbox}
\textsubscript{157} (118-157) Много у меня гонителей и врагов, [но] от откровений Твоих я не удаляюсь.
\end{tcolorbox}
\begin{tcolorbox}
\textsubscript{158} (118-158) Вижу отступников, и сокрушаюсь, ибо они не хранят слова Твоего.
\end{tcolorbox}
\begin{tcolorbox}
\textsubscript{159} (118-159) Зри, как я люблю повеления Твои; по милости Твоей, Господи, оживи меня.
\end{tcolorbox}
\begin{tcolorbox}
\textsubscript{160} (118-160) Основание слова Твоего истинно, и вечен всякий суд правды Твоей.
\end{tcolorbox}
\begin{tcolorbox}
\textsubscript{161} (118-161) Князья гонят меня безвинно, но сердце мое боится слова Твоего.
\end{tcolorbox}
\begin{tcolorbox}
\textsubscript{162} (118-162) Радуюсь я слову Твоему, как получивший великую прибыль.
\end{tcolorbox}
\begin{tcolorbox}
\textsubscript{163} (118-163) Ненавижу ложь и гнушаюсь ею; закон же Твой люблю.
\end{tcolorbox}
\begin{tcolorbox}
\textsubscript{164} (118-164) Семикратно в день прославляю Тебя за суды правды Твоей.
\end{tcolorbox}
\begin{tcolorbox}
\textsubscript{165} (118-165) Велик мир у любящих закон Твой, и нет им преткновения.
\end{tcolorbox}
\begin{tcolorbox}
\textsubscript{166} (118-166) Уповаю на спасение Твое, Господи, и заповеди Твои исполняю.
\end{tcolorbox}
\begin{tcolorbox}
\textsubscript{167} (118-167) Душа моя хранит откровения Твои, и я люблю их крепко.
\end{tcolorbox}
\begin{tcolorbox}
\textsubscript{168} (118-168) Храню повеления Твои и откровения Твои, ибо все пути мои пред Тобою.
\end{tcolorbox}
\begin{tcolorbox}
\textsubscript{169} (118-169) Да приблизится вопль мой пред лице Твое, Господи; по слову Твоему вразуми меня.
\end{tcolorbox}
\begin{tcolorbox}
\textsubscript{170} (118-170) Да придет моление мое пред лице Твое; по слову Твоему избавь меня.
\end{tcolorbox}
\begin{tcolorbox}
\textsubscript{171} (118-171) Уста мои произнесут хвалу, когда Ты научишь меня уставам Твоим.
\end{tcolorbox}
\begin{tcolorbox}
\textsubscript{172} (118-172) Язык мой возгласит слово Твое, ибо все заповеди Твои праведны.
\end{tcolorbox}
\begin{tcolorbox}
\textsubscript{173} (118-173) Да будет рука Твоя в помощь мне, ибо я повеления Твои избрал.
\end{tcolorbox}
\begin{tcolorbox}
\textsubscript{174} (118-174) Жажду спасения Твоего, Господи, и закон Твой--утешение мое.
\end{tcolorbox}
\begin{tcolorbox}
\textsubscript{175} (118-175) Да живет душа моя и славит Тебя, и суды Твои да помогут мне.
\end{tcolorbox}
\begin{tcolorbox}
\textsubscript{176} (118-176) Я заблудился, как овца потерянная: взыщи раба Твоего, ибо я заповедей Твоих не забыл.
\end{tcolorbox}
\subsection{CHAPTER 120}
\begin{tcolorbox}
\textsubscript{1} (119-1) ^^Песнь восхождения.^^ К Господу воззвал я в скорби моей, и Он услышал меня.
\end{tcolorbox}
\begin{tcolorbox}
\textsubscript{2} (119-2) Господи! избавь душу мою от уст лживых, от языка лукавого.
\end{tcolorbox}
\begin{tcolorbox}
\textsubscript{3} (119-3) Что даст тебе и что прибавит тебе язык лукавый?
\end{tcolorbox}
\begin{tcolorbox}
\textsubscript{4} (119-4) Изощренные стрелы сильного, с горящими углями дроковыми.
\end{tcolorbox}
\begin{tcolorbox}
\textsubscript{5} (119-5) Горе мне, что я пребываю у Мосоха, живу у шатров Кидарских.
\end{tcolorbox}
\begin{tcolorbox}
\textsubscript{6} (119-6) Долго жила душа моя с ненавидящими мир.
\end{tcolorbox}
\begin{tcolorbox}
\textsubscript{7} (119-7) Я мирен: но только заговорю, они--к войне.
\end{tcolorbox}
\subsection{CHAPTER 121}
\begin{tcolorbox}
\textsubscript{1} (120-1) ^^Песнь восхождения.^^ Возвожу очи мои к горам, откуда придет помощь моя.
\end{tcolorbox}
\begin{tcolorbox}
\textsubscript{2} (120-2) Помощь моя от Господа, сотворившего небо и землю.
\end{tcolorbox}
\begin{tcolorbox}
\textsubscript{3} (120-3) Не даст Он поколебаться ноге твоей, не воздремлет хранящий тебя;
\end{tcolorbox}
\begin{tcolorbox}
\textsubscript{4} (120-4) не дремлет и не спит хранящий Израиля.
\end{tcolorbox}
\begin{tcolorbox}
\textsubscript{5} (120-5) Господь--хранитель твой; Господь--сень твоя с правой руки твоей.
\end{tcolorbox}
\begin{tcolorbox}
\textsubscript{6} (120-6) Днем солнце не поразит тебя, ни луна ночью.
\end{tcolorbox}
\begin{tcolorbox}
\textsubscript{7} (120-7) Господь сохранит тебя от всякого зла; сохранит душу твою [Господь].
\end{tcolorbox}
\begin{tcolorbox}
\textsubscript{8} (120-8) Господь будет охранять выхождение твое и вхождение твое отныне и вовек.
\end{tcolorbox}
\subsection{CHAPTER 122}
\begin{tcolorbox}
\textsubscript{1} (121-1) ^^Песнь восхождения. Давида.^^ Возрадовался я, когда сказали мне: 'пойдем в дом Господень'.
\end{tcolorbox}
\begin{tcolorbox}
\textsubscript{2} (121-2) Вот, стоят ноги наши во вратах твоих, Иерусалим, --
\end{tcolorbox}
\begin{tcolorbox}
\textsubscript{3} (121-3) Иерусалим, устроенный как город, слитый в одно,
\end{tcolorbox}
\begin{tcolorbox}
\textsubscript{4} (121-4) куда восходят колена, колена Господни, по закону Израилеву, славить имя Господне.
\end{tcolorbox}
\begin{tcolorbox}
\textsubscript{5} (121-5) Там стоят престолы суда, престолы дома Давидова.
\end{tcolorbox}
\begin{tcolorbox}
\textsubscript{6} (121-6) Просите мира Иерусалиму: да благоденствуют любящие тебя!
\end{tcolorbox}
\begin{tcolorbox}
\textsubscript{7} (121-7) Да будет мир в стенах твоих, благоденствие--в чертогах твоих!
\end{tcolorbox}
\begin{tcolorbox}
\textsubscript{8} (121-8) Ради братьев моих и ближних моих говорю я: 'мир тебе!'
\end{tcolorbox}
\begin{tcolorbox}
\textsubscript{9} (121-9) Ради дома Господа, Бога нашего, желаю блага тебе.
\end{tcolorbox}
\subsection{CHAPTER 123}
\begin{tcolorbox}
\textsubscript{1} (122-1) ^^Песнь восхождения.^^ К Тебе возвожу очи мои, Живущий на небесах!
\end{tcolorbox}
\begin{tcolorbox}
\textsubscript{2} (122-2) Вот, как очи рабов [обращены] на руку господ их, как очи рабы--на руку госпожи ее, так очи наши--к Господу, Богу нашему, доколе Он помилует нас.
\end{tcolorbox}
\begin{tcolorbox}
\textsubscript{3} (122-3) Помилуй нас, Господи, помилуй нас, ибо довольно мы насыщены презрением;
\end{tcolorbox}
\begin{tcolorbox}
\textsubscript{4} (122-4) довольно насыщена душа наша поношением от надменных и уничижением от гордых.
\end{tcolorbox}
\subsection{CHAPTER 124}
\begin{tcolorbox}
\textsubscript{1} (123-1) ^^Песнь восхождения. Давида.^^ Если бы не Господь был с нами, --да скажет Израиль, --
\end{tcolorbox}
\begin{tcolorbox}
\textsubscript{2} (123-2) если бы не Господь был с нами, когда восстали на нас люди,
\end{tcolorbox}
\begin{tcolorbox}
\textsubscript{3} (123-3) то живых они поглотили бы нас, когда возгорелась ярость их на нас;
\end{tcolorbox}
\begin{tcolorbox}
\textsubscript{4} (123-4) воды потопили бы нас, поток прошел бы над душею нашею;
\end{tcolorbox}
\begin{tcolorbox}
\textsubscript{5} (123-5) прошли бы над душею нашею воды бурные.
\end{tcolorbox}
\begin{tcolorbox}
\textsubscript{6} (123-6) Благословен Господь, Который не дал нас в добычу зубам их!
\end{tcolorbox}
\begin{tcolorbox}
\textsubscript{7} (123-7) Душа наша избавилась, как птица, из сети ловящих: сеть расторгнута, и мы избавились.
\end{tcolorbox}
\begin{tcolorbox}
\textsubscript{8} (123-8) Помощь наша--в имени Господа, сотворившего небо и землю.
\end{tcolorbox}
\subsection{CHAPTER 125}
\begin{tcolorbox}
\textsubscript{1} (124-1) ^^Песнь восхождения.^^ Надеющийся на Господа, как гора Сион, не подвигнется: пребывает вовек.
\end{tcolorbox}
\begin{tcolorbox}
\textsubscript{2} (124-2) Горы окрест Иерусалима, а Господь окрест народа Своего отныне и вовек.
\end{tcolorbox}
\begin{tcolorbox}
\textsubscript{3} (124-3) Ибо не оставит [Господь] жезла нечестивых над жребием праведных, дабы праведные не простерли рук своих к беззаконию.
\end{tcolorbox}
\begin{tcolorbox}
\textsubscript{4} (124-4) Благотвори, Господи, добрым и правым в сердцах своих;
\end{tcolorbox}
\begin{tcolorbox}
\textsubscript{5} (124-5) а совращающихся на кривые пути свои да оставит Господь ходить с делающими беззаконие. Мир на Израиля!
\end{tcolorbox}
\subsection{CHAPTER 126}
\begin{tcolorbox}
\textsubscript{1} (125-1) ^^Песнь восхождения.^^ Когда возвращал Господь плен Сиона, мы были как бы видящие во сне:
\end{tcolorbox}
\begin{tcolorbox}
\textsubscript{2} (125-2) тогда уста наши были полны веселья, и язык наш--пения; тогда между народами говорили: 'великое сотворил Господь над ними!'
\end{tcolorbox}
\begin{tcolorbox}
\textsubscript{3} (125-3) Великое сотворил Господь над нами: мы радовались.
\end{tcolorbox}
\begin{tcolorbox}
\textsubscript{4} (125-4) Возврати, Господи, пленников наших, как потоки на полдень.
\end{tcolorbox}
\begin{tcolorbox}
\textsubscript{5} (125-5) Сеявшие со слезами будут пожинать с радостью.
\end{tcolorbox}
\begin{tcolorbox}
\textsubscript{6} (125-6) С плачем несущий семена возвратится с радостью, неся снопы свои.
\end{tcolorbox}
\subsection{CHAPTER 127}
\begin{tcolorbox}
\textsubscript{1} (126-1) ^^Песнь восхождения. Соломона.^^ Если Господь не созиждет дома, напрасно трудятся строящие его; если Господь не охранит города, напрасно бодрствует страж.
\end{tcolorbox}
\begin{tcolorbox}
\textsubscript{2} (126-2) Напрасно вы рано встаете, поздно просиживаете, едите хлеб печали, тогда как возлюбленному Своему Он дает сон.
\end{tcolorbox}
\begin{tcolorbox}
\textsubscript{3} (126-3) Вот наследие от Господа: дети; награда от Него--плод чрева.
\end{tcolorbox}
\begin{tcolorbox}
\textsubscript{4} (126-4) Что стрелы в руке сильного, то сыновья молодые.
\end{tcolorbox}
\begin{tcolorbox}
\textsubscript{5} (126-5) Блажен человек, который наполнил ими колчан свой! Не останутся они в стыде, когда будут говорить с врагами в воротах.
\end{tcolorbox}
\subsection{CHAPTER 128}
\begin{tcolorbox}
\textsubscript{1} (127-1) ^^Песнь восхождения.^^ Блажен всякий боящийся Господа, ходящий путями Его!
\end{tcolorbox}
\begin{tcolorbox}
\textsubscript{2} (127-2) Ты будешь есть от трудов рук твоих: блажен ты, и благо тебе!
\end{tcolorbox}
\begin{tcolorbox}
\textsubscript{3} (127-3) Жена твоя, как плодовитая лоза, в доме твоем; сыновья твои, как масличные ветви, вокруг трапезы твоей:
\end{tcolorbox}
\begin{tcolorbox}
\textsubscript{4} (127-4) так благословится человек, боящийся Господа!
\end{tcolorbox}
\begin{tcolorbox}
\textsubscript{5} (127-5) Благословит тебя Господь с Сиона, и увидишь благоденствие Иерусалима во все дни жизни твоей;
\end{tcolorbox}
\begin{tcolorbox}
\textsubscript{6} (127-6) увидишь сыновей у сыновей твоих. Мир на Израиля!
\end{tcolorbox}
\subsection{CHAPTER 129}
\begin{tcolorbox}
\textsubscript{1} (128-1) ^^Песнь восхождения.^^ Много теснили меня от юности моей, да скажет Израиль:
\end{tcolorbox}
\begin{tcolorbox}
\textsubscript{2} (128-2) много теснили меня от юности моей, но не одолели меня.
\end{tcolorbox}
\begin{tcolorbox}
\textsubscript{3} (128-3) На хребте моем орали оратаи, проводили длинные борозды свои.
\end{tcolorbox}
\begin{tcolorbox}
\textsubscript{4} (128-4) Но Господь праведен: Он рассек узы нечестивых.
\end{tcolorbox}
\begin{tcolorbox}
\textsubscript{5} (128-5) Да постыдятся и обратятся назад все ненавидящие Сион!
\end{tcolorbox}
\begin{tcolorbox}
\textsubscript{6} (128-6) Да будут, как трава на кровлях, которая прежде, нежели будет исторгнута, засыхает,
\end{tcolorbox}
\begin{tcolorbox}
\textsubscript{7} (128-7) которою жнец не наполнит руки своей, и вяжущий снопы--горсти своей;
\end{tcolorbox}
\begin{tcolorbox}
\textsubscript{8} (128-8) и проходящие мимо не скажут: 'благословение Господне на вас; благословляем вас именем Господним!'
\end{tcolorbox}
\subsection{CHAPTER 130}
\begin{tcolorbox}
\textsubscript{1} (129-1) ^^Песнь восхождения.^^ Из глубины взываю к Тебе, Господи.
\end{tcolorbox}
\begin{tcolorbox}
\textsubscript{2} (129-2) Господи! услышь голос мой. Да будут уши Твои внимательны к голосу молений моих.
\end{tcolorbox}
\begin{tcolorbox}
\textsubscript{3} (129-3) Если Ты, Господи, будешь замечать беззакония, --Господи! кто устоит?
\end{tcolorbox}
\begin{tcolorbox}
\textsubscript{4} (129-4) Но у Тебя прощение, да благоговеют пред Тобою.
\end{tcolorbox}
\begin{tcolorbox}
\textsubscript{5} (129-5) Надеюсь на Господа, надеется душа моя; на слово Его уповаю.
\end{tcolorbox}
\begin{tcolorbox}
\textsubscript{6} (129-6) Душа моя ожидает Господа более, нежели стражи--утра, более, нежели стражи--утра.
\end{tcolorbox}
\begin{tcolorbox}
\textsubscript{7} (129-7) Да уповает Израиль на Господа, ибо у Господа милость и многое у Него избавление,
\end{tcolorbox}
\begin{tcolorbox}
\textsubscript{8} (129-8) и Он избавит Израиля от всех беззаконий его.
\end{tcolorbox}
\subsection{CHAPTER 131}
\begin{tcolorbox}
\textsubscript{1} (130-1) ^^Песнь восхождения. Давида.^^ Господи! не надмевалось сердце мое и не возносились очи мои, и я не входил в великое и для меня недосягаемое.
\end{tcolorbox}
\begin{tcolorbox}
\textsubscript{2} (130-2) Не смирял ли я и не успокаивал ли души моей, как дитяти, отнятого от груди матери? душа моя была во мне, как дитя, отнятое от груди.
\end{tcolorbox}
\begin{tcolorbox}
\textsubscript{3} (130-3) Да уповает Израиль на Господа отныне и вовек.
\end{tcolorbox}
\subsection{CHAPTER 132}
\begin{tcolorbox}
\textsubscript{1} (131-1) ^^Песнь восхождения.^^ Вспомни, Господи, Давида и все сокрушение его:
\end{tcolorbox}
\begin{tcolorbox}
\textsubscript{2} (131-2) как он клялся Господу, давал обет Сильному Иакова:
\end{tcolorbox}
\begin{tcolorbox}
\textsubscript{3} (131-3) 'не войду в шатер дома моего, не взойду на ложе мое;
\end{tcolorbox}
\begin{tcolorbox}
\textsubscript{4} (131-4) не дам сна очам моим и веждам моим--дремания,
\end{tcolorbox}
\begin{tcolorbox}
\textsubscript{5} (131-5) доколе не найду места Господу, жилища--Сильному Иакова'.
\end{tcolorbox}
\begin{tcolorbox}
\textsubscript{6} (131-6) Вот, мы слышали о нем в Ефрафе, нашли его на полях Иарима.
\end{tcolorbox}
\begin{tcolorbox}
\textsubscript{7} (131-7) Пойдем к жилищу Его, поклонимся подножию ног Его.
\end{tcolorbox}
\begin{tcolorbox}
\textsubscript{8} (131-8) Стань, Господи, на [место] покоя Твоего, --Ты и ковчег могущества Твоего.
\end{tcolorbox}
\begin{tcolorbox}
\textsubscript{9} (131-9) Священники Твои облекутся правдою, и святые Твои возрадуются.
\end{tcolorbox}
\begin{tcolorbox}
\textsubscript{10} (131-10) Ради Давида, раба Твоего, не отврати лица помазанника Твоего.
\end{tcolorbox}
\begin{tcolorbox}
\textsubscript{11} (131-11) Клялся Господь Давиду в истине, и не отречется ее: 'от плода чрева твоего посажу на престоле твоем.
\end{tcolorbox}
\begin{tcolorbox}
\textsubscript{12} (131-12) Если сыновья твои будут сохранять завет Мой и откровения Мои, которым Я научу их, то и их сыновья во веки будут сидеть на престоле твоем'.
\end{tcolorbox}
\begin{tcolorbox}
\textsubscript{13} (131-13) Ибо избрал Господь Сион, возжелал [его] в жилище Себе.
\end{tcolorbox}
\begin{tcolorbox}
\textsubscript{14} (131-14) 'Это покой Мой на веки: здесь вселюсь, ибо Я возжелал его.
\end{tcolorbox}
\begin{tcolorbox}
\textsubscript{15} (131-15) Пищу его благословляя благословлю, нищих его насыщу хлебом;
\end{tcolorbox}
\begin{tcolorbox}
\textsubscript{16} (131-16) священников его облеку во спасение, и святые его радостью возрадуются.
\end{tcolorbox}
\begin{tcolorbox}
\textsubscript{17} (131-17) Там возращу рог Давиду, поставлю светильник помазаннику Моему.
\end{tcolorbox}
\begin{tcolorbox}
\textsubscript{18} (131-18) Врагов его облеку стыдом, а на нем будет сиять венец его'.
\end{tcolorbox}
\subsection{CHAPTER 133}
\begin{tcolorbox}
\textsubscript{1} (132-1) ^^Песнь восхождения. Давида.^^ Как хорошо и как приятно жить братьям вместе!
\end{tcolorbox}
\begin{tcolorbox}
\textsubscript{2} (132-2) [Это] --как драгоценный елей на голове, стекающий на бороду, бороду Ааронову, стекающий на края одежды его;
\end{tcolorbox}
\begin{tcolorbox}
\textsubscript{3} (132-3) как роса Ермонская, сходящая на горы Сионские, ибо там заповедал Господь благословение и жизнь на веки.
\end{tcolorbox}
\subsection{CHAPTER 134}
\begin{tcolorbox}
\textsubscript{1} (133-1) ^^Песнь восхождения.^^ Благословите ныне Господа, все рабы Господни, стоящие в доме Господнем, во время ночи.
\end{tcolorbox}
\begin{tcolorbox}
\textsubscript{2} (133-2) Воздвигните руки ваши к святилищу, и благословите Господа.
\end{tcolorbox}
\begin{tcolorbox}
\textsubscript{3} (133-3) Благословит тебя Господь с Сиона, сотворивший небо и землю.
\end{tcolorbox}
\subsection{CHAPTER 135}
\begin{tcolorbox}
\textsubscript{1} (134-1) ^^Аллилуия.^^ Хвалите имя Господне, хвалите, рабы Господни,
\end{tcolorbox}
\begin{tcolorbox}
\textsubscript{2} (134-2) стоящие в доме Господнем, во дворах дома Бога нашего.
\end{tcolorbox}
\begin{tcolorbox}
\textsubscript{3} (134-3) Хвалите Господа, ибо Господь благ; пойте имени Его, ибо это сладостно,
\end{tcolorbox}
\begin{tcolorbox}
\textsubscript{4} (134-4) ибо Господь избрал Себе Иакова, Израиля в собственность Свою.
\end{tcolorbox}
\begin{tcolorbox}
\textsubscript{5} (134-5) Я познал, что велик Господь, и Господь наш превыше всех богов.
\end{tcolorbox}
\begin{tcolorbox}
\textsubscript{6} (134-6) Господь творит все, что хочет, на небесах и на земле, на морях и во всех безднах;
\end{tcolorbox}
\begin{tcolorbox}
\textsubscript{7} (134-7) возводит облака от края земли, творит молнии при дожде, изводит ветер из хранилищ Своих.
\end{tcolorbox}
\begin{tcolorbox}
\textsubscript{8} (134-8) Он поразил первенцев Египта, от человека до скота,
\end{tcolorbox}
\begin{tcolorbox}
\textsubscript{9} (134-9) послал знамения и чудеса среди тебя, Египет, на фараона и на всех рабов его,
\end{tcolorbox}
\begin{tcolorbox}
\textsubscript{10} (134-10) поразил народы многие и истребил царей сильных:
\end{tcolorbox}
\begin{tcolorbox}
\textsubscript{11} (134-11) Сигона, царя Аморрейского, и Ога, царя Васанского, и все царства Ханаанские;
\end{tcolorbox}
\begin{tcolorbox}
\textsubscript{12} (134-12) и отдал землю их в наследие, в наследие Израилю, народу Своему.
\end{tcolorbox}
\begin{tcolorbox}
\textsubscript{13} (134-13) Господи! имя Твое вовек; Господи! память о Тебе в род и род.
\end{tcolorbox}
\begin{tcolorbox}
\textsubscript{14} (134-14) Ибо Господь будет судить народ Свой и над рабами Своими умилосердится.
\end{tcolorbox}
\begin{tcolorbox}
\textsubscript{15} (134-15) Идолы язычников--серебро и золото, дело рук человеческих:
\end{tcolorbox}
\begin{tcolorbox}
\textsubscript{16} (134-16) есть у них уста, но не говорят; есть у них глаза, но не видят;
\end{tcolorbox}
\begin{tcolorbox}
\textsubscript{17} (134-17) есть у них уши, но не слышат, и нет дыхания в устах их.
\end{tcolorbox}
\begin{tcolorbox}
\textsubscript{18} (134-18) Подобны им будут делающие их и всякий, кто надеется на них.
\end{tcolorbox}
\begin{tcolorbox}
\textsubscript{19} (134-19) Дом Израилев! благословите Господа. Дом Ааронов! благословите Господа.
\end{tcolorbox}
\begin{tcolorbox}
\textsubscript{20} (134-20) Дом Левиин! благословите Господа. Боящиеся Господа! благословите Господа.
\end{tcolorbox}
\begin{tcolorbox}
\textsubscript{21} (134-21) Благословен Господь от Сиона, живущий в Иерусалиме! Аллилуия!
\end{tcolorbox}
\subsection{CHAPTER 136}
\begin{tcolorbox}
\textsubscript{1} (135-1) Славьте Господа, ибо Он благ, ибо вовек милость Его.
\end{tcolorbox}
\begin{tcolorbox}
\textsubscript{2} (135-2) Славьте Бога богов, ибо вовек милость Его.
\end{tcolorbox}
\begin{tcolorbox}
\textsubscript{3} (135-3) Славьте Господа господствующих, ибо вовек милость Его;
\end{tcolorbox}
\begin{tcolorbox}
\textsubscript{4} (135-4) Того, Который один творит чудеса великие, ибо вовек милость Его;
\end{tcolorbox}
\begin{tcolorbox}
\textsubscript{5} (135-5) Который сотворил небеса премудро, ибо вовек милость Его;
\end{tcolorbox}
\begin{tcolorbox}
\textsubscript{6} (135-6) утвердил землю на водах, ибо вовек милость Его;
\end{tcolorbox}
\begin{tcolorbox}
\textsubscript{7} (135-7) сотворил светила великие, ибо вовек милость Его;
\end{tcolorbox}
\begin{tcolorbox}
\textsubscript{8} (135-8) солнце--для управления днем, ибо вовек милость Его;
\end{tcolorbox}
\begin{tcolorbox}
\textsubscript{9} (135-9) луну и звезды--для управления ночью, ибо вовек милость Его;
\end{tcolorbox}
\begin{tcolorbox}
\textsubscript{10} (135-10) поразил Египет в первенцах его, ибо вовек милость Его;
\end{tcolorbox}
\begin{tcolorbox}
\textsubscript{11} (135-11) и вывел Израиля из среды его, ибо вовек милость Его;
\end{tcolorbox}
\begin{tcolorbox}
\textsubscript{12} (135-12) рукою крепкою и мышцею простертою, ибо вовек милость Его;
\end{tcolorbox}
\begin{tcolorbox}
\textsubscript{13} (135-13) разделил Чермное море, ибо вовек милость Его;
\end{tcolorbox}
\begin{tcolorbox}
\textsubscript{14} (135-14) и провел Израиля посреди его, ибо вовек милость Его;
\end{tcolorbox}
\begin{tcolorbox}
\textsubscript{15} (135-15) и низверг фараона и войско его в море Чермное, ибо вовек милость Его;
\end{tcolorbox}
\begin{tcolorbox}
\textsubscript{16} (135-16) провел народ Свой чрез пустыню, ибо вовек милость Его;
\end{tcolorbox}
\begin{tcolorbox}
\textsubscript{17} (135-17) поразил царей великих, ибо вовек милость Его;
\end{tcolorbox}
\begin{tcolorbox}
\textsubscript{18} (135-18) и убил царей сильных, ибо вовек милость Его;
\end{tcolorbox}
\begin{tcolorbox}
\textsubscript{19} (135-19) Сигона, царя Аморрейского, ибо вовек милость Его;
\end{tcolorbox}
\begin{tcolorbox}
\textsubscript{20} (135-20) и Ога, царя Васанского, ибо вовек милость Его;
\end{tcolorbox}
\begin{tcolorbox}
\textsubscript{21} (135-21) и отдал землю их в наследие, ибо вовек милость Его;
\end{tcolorbox}
\begin{tcolorbox}
\textsubscript{22} (135-22) в наследие Израилю, рабу Своему, ибо вовек милость Его;
\end{tcolorbox}
\begin{tcolorbox}
\textsubscript{23} (135-23) вспомнил нас в унижении нашем, ибо вовек милость Его;
\end{tcolorbox}
\begin{tcolorbox}
\textsubscript{24} (135-24) и избавил нас от врагов наших, ибо вовек милость Его;
\end{tcolorbox}
\begin{tcolorbox}
\textsubscript{25} (135-25) дает пищу всякой плоти, ибо вовек милость Его.
\end{tcolorbox}
\begin{tcolorbox}
\textsubscript{26} (135-26) Славьте Бога небес, ибо вовек милость Его.
\end{tcolorbox}
\subsection{CHAPTER 137}
\begin{tcolorbox}
\textsubscript{1} (136-1) При реках Вавилона, там сидели мы и плакали, когда вспоминали о Сионе;
\end{tcolorbox}
\begin{tcolorbox}
\textsubscript{2} (136-2) на вербах, посреди его, повесили мы наши арфы.
\end{tcolorbox}
\begin{tcolorbox}
\textsubscript{3} (136-3) Там пленившие нас требовали от нас слов песней, и притеснители наши--веселья: 'пропойте нам из песней Сионских'.
\end{tcolorbox}
\begin{tcolorbox}
\textsubscript{4} (136-4) Как нам петь песнь Господню на земле чужой?
\end{tcolorbox}
\begin{tcolorbox}
\textsubscript{5} (136-5) Если я забуду тебя, Иерусалим, --забудь меня десница моя;
\end{tcolorbox}
\begin{tcolorbox}
\textsubscript{6} (136-6) прилипни язык мой к гортани моей, если не буду помнить тебя, если не поставлю Иерусалима во главе веселия моего.
\end{tcolorbox}
\begin{tcolorbox}
\textsubscript{7} (136-7) Припомни, Господи, сынам Едомовым день Иерусалима, когда они говорили: 'разрушайте, разрушайте до основания его'.
\end{tcolorbox}
\begin{tcolorbox}
\textsubscript{8} (136-8) Дочь Вавилона, опустошительница! блажен, кто воздаст тебе за то, что ты сделала нам!
\end{tcolorbox}
\begin{tcolorbox}
\textsubscript{9} (136-9) Блажен, кто возьмет и разобьет младенцев твоих о камень!
\end{tcolorbox}
\subsection{CHAPTER 138}
\begin{tcolorbox}
\textsubscript{1} (137-1) ^^Давида.^^ Славлю Тебя всем сердцем моим, пред богами пою Тебе.
\end{tcolorbox}
\begin{tcolorbox}
\textsubscript{2} (137-2) Поклоняюсь пред святым храмом Твоим и славлю имя Твое за милость Твою и за истину Твою, ибо Ты возвеличил слово Твое превыше всякого имени Твоего.
\end{tcolorbox}
\begin{tcolorbox}
\textsubscript{3} (137-3) В день, когда я воззвал, Ты услышал меня, вселил в душу мою бодрость.
\end{tcolorbox}
\begin{tcolorbox}
\textsubscript{4} (137-4) Прославят Тебя, Господи, все цари земные, когда услышат слова уст Твоих
\end{tcolorbox}
\begin{tcolorbox}
\textsubscript{5} (137-5) и воспоют пути Господни, ибо велика слава Господня.
\end{tcolorbox}
\begin{tcolorbox}
\textsubscript{6} (137-6) Высок Господь: и смиренного видит, и гордого узнает издали.
\end{tcolorbox}
\begin{tcolorbox}
\textsubscript{7} (137-7) Если я пойду посреди напастей, Ты оживишь меня, прострешь на ярость врагов моих руку Твою, и спасет меня десница Твоя.
\end{tcolorbox}
\begin{tcolorbox}
\textsubscript{8} (137-8) Господь совершит за меня! Милость Твоя, Господи, вовек: дело рук Твоих не оставляй.
\end{tcolorbox}
\subsection{CHAPTER 139}
\begin{tcolorbox}
\textsubscript{1} (138-1) ^^Начальнику хора. Псалом Давида.^^ Господи! Ты испытал меня и знаешь.
\end{tcolorbox}
\begin{tcolorbox}
\textsubscript{2} (138-2) Ты знаешь, когда я сажусь и когда встаю; Ты разумеешь помышления мои издали.
\end{tcolorbox}
\begin{tcolorbox}
\textsubscript{3} (138-3) Иду ли я, отдыхаю ли--Ты окружаешь меня, и все пути мои известны Тебе.
\end{tcolorbox}
\begin{tcolorbox}
\textsubscript{4} (138-4) Еще нет слова на языке моем, --Ты, Господи, уже знаешь его совершенно.
\end{tcolorbox}
\begin{tcolorbox}
\textsubscript{5} (138-5) Сзади и спереди Ты объемлешь меня, и полагаешь на мне руку Твою.
\end{tcolorbox}
\begin{tcolorbox}
\textsubscript{6} (138-6) Дивно для меня ведение [Твое], --высоко, не могу постигнуть его!
\end{tcolorbox}
\begin{tcolorbox}
\textsubscript{7} (138-7) Куда пойду от Духа Твоего, и от лица Твоего куда убегу?
\end{tcolorbox}
\begin{tcolorbox}
\textsubscript{8} (138-8) Взойду ли на небо--Ты там; сойду ли в преисподнюю--и там Ты.
\end{tcolorbox}
\begin{tcolorbox}
\textsubscript{9} (138-9) Возьму ли крылья зари и переселюсь на край моря, --
\end{tcolorbox}
\begin{tcolorbox}
\textsubscript{10} (138-10) и там рука Твоя поведет меня, и удержит меня десница Твоя.
\end{tcolorbox}
\begin{tcolorbox}
\textsubscript{11} (138-11) Скажу ли: 'может быть, тьма скроет меня, и свет вокруг меня [сделается] ночью';
\end{tcolorbox}
\begin{tcolorbox}
\textsubscript{12} (138-12) но и тьма не затмит от Тебя, и ночь светла, как день: как тьма, так и свет.
\end{tcolorbox}
\begin{tcolorbox}
\textsubscript{13} (138-13) Ибо Ты устроил внутренности мои и соткал меня во чреве матери моей.
\end{tcolorbox}
\begin{tcolorbox}
\textsubscript{14} (138-14) Славлю Тебя, потому что я дивно устроен. Дивны дела Твои, и душа моя вполне сознает это.
\end{tcolorbox}
\begin{tcolorbox}
\textsubscript{15} (138-15) Не сокрыты были от Тебя кости мои, когда я созидаем был в тайне, образуем был во глубине утробы.
\end{tcolorbox}
\begin{tcolorbox}
\textsubscript{16} (138-16) Зародыш мой видели очи Твои; в Твоей книге записаны все дни, для меня назначенные, когда ни одного из них еще не было.
\end{tcolorbox}
\begin{tcolorbox}
\textsubscript{17} (138-17) Как возвышенны для меня помышления Твои, Боже, и как велико число их!
\end{tcolorbox}
\begin{tcolorbox}
\textsubscript{18} (138-18) Стану ли исчислять их, но они многочисленнее песка; когда я пробуждаюсь, я все еще с Тобою.
\end{tcolorbox}
\begin{tcolorbox}
\textsubscript{19} (138-19) О, если бы Ты, Боже, поразил нечестивого! Удалитесь от меня, кровожадные!
\end{tcolorbox}
\begin{tcolorbox}
\textsubscript{20} (138-20) Они говорят против Тебя нечестиво; суетное замышляют враги Твои.
\end{tcolorbox}
\begin{tcolorbox}
\textsubscript{21} (138-21) Мне ли не возненавидеть ненавидящих Тебя, Господи, и не возгнушаться восстающими на Тебя?
\end{tcolorbox}
\begin{tcolorbox}
\textsubscript{22} (138-22) Полною ненавистью ненавижу их: враги они мне.
\end{tcolorbox}
\begin{tcolorbox}
\textsubscript{23} (138-23) Испытай меня, Боже, и узнай сердце мое; испытай меня и узнай помышления мои;
\end{tcolorbox}
\begin{tcolorbox}
\textsubscript{24} (138-24) и зри, не на опасном ли я пути, и направь меня на путь вечный.
\end{tcolorbox}
\subsection{CHAPTER 140}
\begin{tcolorbox}
\textsubscript{1} (139-1) ^^Псалом. Начальнику хора. Псалом Давида.^^ (139-2) Избавь меня, Господи, от человека злого; сохрани меня от притеснителя:
\end{tcolorbox}
\begin{tcolorbox}
\textsubscript{2} (139-3) они злое мыслят в сердце, всякий день ополчаются на брань,
\end{tcolorbox}
\begin{tcolorbox}
\textsubscript{3} (139-4) изощряют язык свой, как змея; яд аспида под устами их.
\end{tcolorbox}
\begin{tcolorbox}
\textsubscript{4} (139-5) Соблюди меня, Господи, от рук нечестивого, сохрани меня от притеснителей, которые замыслили поколебать стопы мои.
\end{tcolorbox}
\begin{tcolorbox}
\textsubscript{5} (139-6) Гордые скрыли силки для меня и петли, раскинули сеть по дороге, тенета разложили для меня.
\end{tcolorbox}
\begin{tcolorbox}
\textsubscript{6} (139-7) Я сказал Господу: Ты Бог мой; услышь, Господи, голос молений моих!
\end{tcolorbox}
\begin{tcolorbox}
\textsubscript{7} (139-8) Господи, Господи, сила спасения моего! Ты покрыл голову мою в день брани.
\end{tcolorbox}
\begin{tcolorbox}
\textsubscript{8} (139-9) Не дай, Господи, желаемого нечестивому; не дай успеха злому замыслу его: они возгордятся.
\end{tcolorbox}
\begin{tcolorbox}
\textsubscript{9} (139-10) Да покроет головы окружающих меня зло собственных уст их.
\end{tcolorbox}
\begin{tcolorbox}
\textsubscript{10} (139-11) Да падут на них горящие угли; да будут они повержены в огонь, в пропасти, так, чтобы не встали.
\end{tcolorbox}
\begin{tcolorbox}
\textsubscript{11} (139-12) Человек злоязычный не утвердится на земле; зло увлечет притеснителя в погибель.
\end{tcolorbox}
\begin{tcolorbox}
\textsubscript{12} (139-13) Знаю, что Господь сотворит суд угнетенным и справедливость бедным.
\end{tcolorbox}
\begin{tcolorbox}
\textsubscript{13} (139-14) Так! праведные будут славить имя Твое; непорочные будут обитать пред лицем Твоим.
\end{tcolorbox}
\subsection{CHAPTER 141}
\begin{tcolorbox}
\textsubscript{1} (140-1) ^^Псалом Давида.^^ Господи! к тебе взываю: поспеши ко мне, внемли голосу моления моего, когда взываю к Тебе.
\end{tcolorbox}
\begin{tcolorbox}
\textsubscript{2} (140-2) Да направится молитва моя, как фимиам, пред лице Твое, воздеяние рук моих--как жертва вечерняя.
\end{tcolorbox}
\begin{tcolorbox}
\textsubscript{3} (140-3) Положи, Господи, охрану устам моим, и огради двери уст моих;
\end{tcolorbox}
\begin{tcolorbox}
\textsubscript{4} (140-4) не дай уклониться сердцу моему к словам лукавым для извинения дел греховных вместе с людьми, делающими беззаконие, и да не вкушу я от сластей их.
\end{tcolorbox}
\begin{tcolorbox}
\textsubscript{5} (140-5) Пусть наказывает меня праведник: это милость; пусть обличает меня: это лучший елей, который не повредит голове моей; но мольбы мои--против злодейств их.
\end{tcolorbox}
\begin{tcolorbox}
\textsubscript{6} (140-6) Вожди их рассыпались по утесам и слышат слова мои, что они кротки.
\end{tcolorbox}
\begin{tcolorbox}
\textsubscript{7} (140-7) Как будто землю рассекают и дробят нас; сыплются кости наши в челюсти преисподней.
\end{tcolorbox}
\begin{tcolorbox}
\textsubscript{8} (140-8) Но к Тебе, Господи, Господи, очи мои; на Тебя уповаю, не отринь души моей!
\end{tcolorbox}
\begin{tcolorbox}
\textsubscript{9} (140-9) Сохрани меня от силков, поставленных для меня, от тенет беззаконников.
\end{tcolorbox}
\begin{tcolorbox}
\textsubscript{10} (140-10) Падут нечестивые в сети свои, а я перейду.
\end{tcolorbox}
\subsection{CHAPTER 142}
\begin{tcolorbox}
\textsubscript{1} (141-1) ^^Учение Давида. Молитва его, когда он был в пещере.^^ Голосом моим к Господу воззвал я, голосом моим к Господу помолился;
\end{tcolorbox}
\begin{tcolorbox}
\textsubscript{2} (141-2) излил пред Ним моление мое; печаль мою открыл Ему.
\end{tcolorbox}
\begin{tcolorbox}
\textsubscript{3} (141-3) Когда изнемогал во мне дух мой, Ты знал стезю мою. На пути, которым я ходил, они скрытно поставили сети для меня.
\end{tcolorbox}
\begin{tcolorbox}
\textsubscript{4} (141-4) Смотрю на правую сторону, и вижу, что никто не признаёт меня: не стало для меня убежища, никто не заботится о душе моей.
\end{tcolorbox}
\begin{tcolorbox}
\textsubscript{5} (141-5) Я воззвал к Тебе, Господи, я сказал: Ты прибежище мое и часть моя на земле живых.
\end{tcolorbox}
\begin{tcolorbox}
\textsubscript{6} (141-6) Внемли воплю моему, ибо я очень изнемог; избавь меня от гонителей моих, ибо они сильнее меня.
\end{tcolorbox}
\begin{tcolorbox}
\textsubscript{7} (141-7) Выведи из темницы душу мою, чтобы мне славить имя Твое. Вокруг меня соберутся праведные, когда Ты явишь мне благодеяние.
\end{tcolorbox}
\subsection{CHAPTER 143}
\begin{tcolorbox}
\textsubscript{1} (142-1) ^^Псалом Давида.^^ Господи! услышь молитву мою, внемли молению моему по истине Твоей; услышь меня по правде Твоей
\end{tcolorbox}
\begin{tcolorbox}
\textsubscript{2} (142-2) и не входи в суд с рабом Твоим, потому что не оправдается пред Тобой ни один из живущих.
\end{tcolorbox}
\begin{tcolorbox}
\textsubscript{3} (142-3) Враг преследует душу мою, втоптал в землю жизнь мою, принудил меня жить во тьме, как давно умерших, --
\end{tcolorbox}
\begin{tcolorbox}
\textsubscript{4} (142-4) и уныл во мне дух мой, онемело во мне сердце мое.
\end{tcolorbox}
\begin{tcolorbox}
\textsubscript{5} (142-5) Вспоминаю дни древние, размышляю о всех делах Твоих, рассуждаю о делах рук Твоих.
\end{tcolorbox}
\begin{tcolorbox}
\textsubscript{6} (142-6) Простираю к Тебе руки мои; душа моя--к Тебе, как жаждущая земля.
\end{tcolorbox}
\begin{tcolorbox}
\textsubscript{7} (142-7) Скоро услышь меня, Господи: дух мой изнемогает; не скрывай лица Твоего от меня, чтобы я не уподобился нисходящим в могилу.
\end{tcolorbox}
\begin{tcolorbox}
\textsubscript{8} (142-8) Даруй мне рано услышать милость Твою, ибо я на Тебя уповаю. Укажи мне путь, по которому мне идти, ибо к Тебе возношу я душу мою.
\end{tcolorbox}
\begin{tcolorbox}
\textsubscript{9} (142-9) Избавь меня, Господи, от врагов моих; к Тебе прибегаю.
\end{tcolorbox}
\begin{tcolorbox}
\textsubscript{10} (142-10) Научи меня исполнять волю Твою, потому что Ты Бог мой; Дух Твой благий да ведет меня в землю правды.
\end{tcolorbox}
\begin{tcolorbox}
\textsubscript{11} (142-11) Ради имени Твоего, Господи, оживи меня; ради правды Твоей выведи из напасти душу мою.
\end{tcolorbox}
\begin{tcolorbox}
\textsubscript{12} (142-12) И по милости Твоей истреби врагов моих и погуби всех, угнетающих душу мою, ибо я Твой раб.
\end{tcolorbox}
\subsection{CHAPTER 144}
\begin{tcolorbox}
\textsubscript{1} (143-1) ^^Давида.^^ Благословен Господь, твердыня моя, научающий руки мои битве и персты мои брани,
\end{tcolorbox}
\begin{tcolorbox}
\textsubscript{2} (143-2) милость моя и ограждение мое, прибежище мое и Избавитель мой, щит мой, --и я на Него уповаю; Он подчиняет мне народ мой.
\end{tcolorbox}
\begin{tcolorbox}
\textsubscript{3} (143-3) Господи! что есть человек, что Ты знаешь о нем, и сын человеческий, что обращаешь на него внимание?
\end{tcolorbox}
\begin{tcolorbox}
\textsubscript{4} (143-4) Человек подобен дуновению; дни его--как уклоняющаяся тень.
\end{tcolorbox}
\begin{tcolorbox}
\textsubscript{5} (143-5) Господи! Приклони небеса Твои и сойди; коснись гор, и воздымятся;
\end{tcolorbox}
\begin{tcolorbox}
\textsubscript{6} (143-6) блесни молниею и рассей их; пусти стрелы Твои и расстрой их;
\end{tcolorbox}
\begin{tcolorbox}
\textsubscript{7} (143-7) простри с высоты руку Твою, избавь меня и спаси меня от вод многих, от руки сынов иноплеменных,
\end{tcolorbox}
\begin{tcolorbox}
\textsubscript{8} (143-8) которых уста говорят суетное и которых десница--десница лжи.
\end{tcolorbox}
\begin{tcolorbox}
\textsubscript{9} (143-9) Боже! новую песнь воспою Тебе, на десятиструнной псалтири воспою Тебе,
\end{tcolorbox}
\begin{tcolorbox}
\textsubscript{10} (143-10) дарующему спасение царям и избавляющему Давида, раба Твоего, от лютого меча.
\end{tcolorbox}
\begin{tcolorbox}
\textsubscript{11} (143-11) Избавь меня и спаси меня от руки сынов иноплеменных, которых уста говорят суетное и которых десница--десница лжи.
\end{tcolorbox}
\begin{tcolorbox}
\textsubscript{12} (143-12) Да будут сыновья наши, как разросшиеся растения в их молодости; дочери наши--как искусно изваянные столпы в чертогах.
\end{tcolorbox}
\begin{tcolorbox}
\textsubscript{13} (143-13) Да будут житницы наши полны, обильны всяким хлебом; да плодятся овцы наши тысячами и тьмами на пажитях наших;
\end{tcolorbox}
\begin{tcolorbox}
\textsubscript{14} (143-14) [да будут] волы наши тучны; да не будет ни расхищения, ни пропажи, ни воплей на улицах наших.
\end{tcolorbox}
\begin{tcolorbox}
\textsubscript{15} (143-15) Блажен народ, у которого это есть. Блажен народ, у которого Господь есть Бог.
\end{tcolorbox}
\subsection{CHAPTER 145}
\begin{tcolorbox}
\textsubscript{1} (144-1) ^^Хвала Давида.^^ Буду превозносить Тебя, Боже мой, Царь [мой], и благословлять имя Твое во веки и веки.
\end{tcolorbox}
\begin{tcolorbox}
\textsubscript{2} (144-2) Всякий день буду благословлять Тебя и восхвалять имя Твое во веки и веки.
\end{tcolorbox}
\begin{tcolorbox}
\textsubscript{3} (144-3) Велик Господь и достохвален, и величие Его неисследимо.
\end{tcolorbox}
\begin{tcolorbox}
\textsubscript{4} (144-4) Род роду будет восхвалять дела Твои и возвещать о могуществе Твоем.
\end{tcolorbox}
\begin{tcolorbox}
\textsubscript{5} (144-5) А я буду размышлять о высокой славе величия Твоего и о дивных делах Твоих.
\end{tcolorbox}
\begin{tcolorbox}
\textsubscript{6} (144-6) Будут говорить о могуществе страшных дел Твоих, и я буду возвещать о величии Твоем.
\end{tcolorbox}
\begin{tcolorbox}
\textsubscript{7} (144-7) Будут провозглашать память великой благости Твоей и воспевать правду Твою.
\end{tcolorbox}
\begin{tcolorbox}
\textsubscript{8} (144-8) Щедр и милостив Господь, долготерпелив и многомилостив.
\end{tcolorbox}
\begin{tcolorbox}
\textsubscript{9} (144-9) Благ Господь ко всем, и щедроты Его на всех делах Его.
\end{tcolorbox}
\begin{tcolorbox}
\textsubscript{10} (144-10) Да славят Тебя, Господи, все дела Твои, и да благословляют Тебя святые Твои;
\end{tcolorbox}
\begin{tcolorbox}
\textsubscript{11} (144-11) да проповедуют славу царства Твоего, и да повествуют о могуществе Твоем,
\end{tcolorbox}
\begin{tcolorbox}
\textsubscript{12} (144-12) чтобы дать знать сынам человеческим о могуществе Твоем и о славном величии царства Твоего.
\end{tcolorbox}
\begin{tcolorbox}
\textsubscript{13} (144-13) Царство Твое--царство всех веков, и владычество Твое во все роды.
\end{tcolorbox}
\begin{tcolorbox}
\textsubscript{14} (144-14) Господь поддерживает всех падающих и восставляет всех низверженных.
\end{tcolorbox}
\begin{tcolorbox}
\textsubscript{15} (144-15) Очи всех уповают на Тебя, и Ты даешь им пищу их в свое время;
\end{tcolorbox}
\begin{tcolorbox}
\textsubscript{16} (144-16) открываешь руку Твою и насыщаешь все живущее по благоволению.
\end{tcolorbox}
\begin{tcolorbox}
\textsubscript{17} (144-17) Праведен Господь во всех путях Своих и благ во всех делах Своих.
\end{tcolorbox}
\begin{tcolorbox}
\textsubscript{18} (144-18) Близок Господь ко всем призывающим Его, ко всем призывающим Его в истине.
\end{tcolorbox}
\begin{tcolorbox}
\textsubscript{19} (144-19) Желание боящихся Его Он исполняет, вопль их слышит и спасает их.
\end{tcolorbox}
\begin{tcolorbox}
\textsubscript{20} (144-20) Хранит Господь всех любящих Его, а всех нечестивых истребит.
\end{tcolorbox}
\begin{tcolorbox}
\textsubscript{21} (144-21) Уста мои изрекут хвалу Господню, и да благословляет всякая плоть святое имя Его во веки и веки.
\end{tcolorbox}
\subsection{CHAPTER 146}
\begin{tcolorbox}
\textsubscript{1} (145-1) Хвали, душа моя, Господа.
\end{tcolorbox}
\begin{tcolorbox}
\textsubscript{2} (145-2) Буду восхвалять Господа, доколе жив; буду петь Богу моему, доколе есмь.
\end{tcolorbox}
\begin{tcolorbox}
\textsubscript{3} (145-3) Не надейтесь на князей, на сына человеческого, в котором нет спасения.
\end{tcolorbox}
\begin{tcolorbox}
\textsubscript{4} (145-4) Выходит дух его, и он возвращается в землю свою: в тот день исчезают [все] помышления его.
\end{tcolorbox}
\begin{tcolorbox}
\textsubscript{5} (145-5) Блажен, кому помощник Бог Иаковлев, у кого надежда на Господа Бога его,
\end{tcolorbox}
\begin{tcolorbox}
\textsubscript{6} (145-6) сотворившего небо и землю, море и все, что в них, вечно хранящего верность,
\end{tcolorbox}
\begin{tcolorbox}
\textsubscript{7} (145-7) творящего суд обиженным, дающего хлеб алчущим. Господь разрешает узников,
\end{tcolorbox}
\begin{tcolorbox}
\textsubscript{8} (145-8) Господь отверзает очи слепым, Господь восставляет согбенных, Господь любит праведных.
\end{tcolorbox}
\begin{tcolorbox}
\textsubscript{9} (145-9) Господь хранит пришельцев, поддерживает сироту и вдову, а путь нечестивых извращает.
\end{tcolorbox}
\begin{tcolorbox}
\textsubscript{10} (145-10) Господь будет царствовать во веки, Бог твой, Сион, в род и род. Аллилуия.
\end{tcolorbox}
\subsection{CHAPTER 147}
\begin{tcolorbox}
\textsubscript{1} (146-1) ^^Аллилуия.^^ Хвалите Господа, ибо благо петь Богу нашему, ибо это сладостно, --хвала подобающая.
\end{tcolorbox}
\begin{tcolorbox}
\textsubscript{2} (146-2) Господь созидает Иерусалим, собирает изгнанников Израиля.
\end{tcolorbox}
\begin{tcolorbox}
\textsubscript{3} (146-3) Он исцеляет сокрушенных сердцем и врачует скорби их;
\end{tcolorbox}
\begin{tcolorbox}
\textsubscript{4} (146-4) исчисляет количество звезд; всех их называет именами их.
\end{tcolorbox}
\begin{tcolorbox}
\textsubscript{5} (146-5) Велик Господь наш и велика крепость [Его], и разум Его неизмерим.
\end{tcolorbox}
\begin{tcolorbox}
\textsubscript{6} (146-6) Смиренных возвышает Господь, а нечестивых унижает до земли.
\end{tcolorbox}
\begin{tcolorbox}
\textsubscript{7} (146-7) Пойте поочередно славословие Господу; пойте Богу нашему на гуслях.
\end{tcolorbox}
\begin{tcolorbox}
\textsubscript{8} (146-8) Он покрывает небо облаками, приготовляет для земли дождь, произращает на горах траву;
\end{tcolorbox}
\begin{tcolorbox}
\textsubscript{9} (146-9) дает скоту пищу его и птенцам ворона, взывающим [к] [Нему].
\end{tcolorbox}
\begin{tcolorbox}
\textsubscript{10} (146-10) Не на силу коня смотрит Он, не к [быстроте] ног человеческих благоволит, --
\end{tcolorbox}
\begin{tcolorbox}
\textsubscript{11} (146-11) благоволит Господь к боящимся Его, к уповающим на милость Его.
\end{tcolorbox}
\begin{tcolorbox}
\textsubscript{12} (147-1) Хвали, Иерусалим, Господа; хвали, Сион, Бога твоего,
\end{tcolorbox}
\begin{tcolorbox}
\textsubscript{13} (147-2) ибо Он укрепляет вереи ворот твоих, благословляет сынов твоих среди тебя;
\end{tcolorbox}
\begin{tcolorbox}
\textsubscript{14} (147-3) утверждает в пределах твоих мир; туком пшеницы насыщает тебя;
\end{tcolorbox}
\begin{tcolorbox}
\textsubscript{15} (147-4) посылает слово Свое на землю; быстро течет слово Его;
\end{tcolorbox}
\begin{tcolorbox}
\textsubscript{16} (147-5) дает снег, как волну; сыплет иней, как пепел;
\end{tcolorbox}
\begin{tcolorbox}
\textsubscript{17} (147-6) бросает град Свой кусками; перед морозом Его кто устоит?
\end{tcolorbox}
\begin{tcolorbox}
\textsubscript{18} (147-7) Пошлет слово Свое, и все растает; подует ветром Своим, и потекут воды.
\end{tcolorbox}
\begin{tcolorbox}
\textsubscript{19} (147-8) Он возвестил слово Свое Иакову, уставы Свои и суды Свои Израилю.
\end{tcolorbox}
\begin{tcolorbox}
\textsubscript{20} (147-9) Не сделал Он того никакому [другому] народу, и судов Его они не знают. Аллилуия.
\end{tcolorbox}
\subsection{CHAPTER 148}
\begin{tcolorbox}
\textsubscript{1} Хвалите Господа с небес, хвалите Его в вышних.
\end{tcolorbox}
\begin{tcolorbox}
\textsubscript{2} Хвалите Его, все Ангелы Его, хвалите Его, все воинства Его.
\end{tcolorbox}
\begin{tcolorbox}
\textsubscript{3} Хвалите Его, солнце и луна, хвалите Его, все звезды света.
\end{tcolorbox}
\begin{tcolorbox}
\textsubscript{4} Хвалите Его, небеса небес и воды, которые превыше небес.
\end{tcolorbox}
\begin{tcolorbox}
\textsubscript{5} Да хвалят имя Господа, ибо Он повелел, и сотворились;
\end{tcolorbox}
\begin{tcolorbox}
\textsubscript{6} поставил их на веки и веки; дал устав, который не прейдет.
\end{tcolorbox}
\begin{tcolorbox}
\textsubscript{7} Хвалите Господа от земли, великие рыбы и все бездны,
\end{tcolorbox}
\begin{tcolorbox}
\textsubscript{8} огонь и град, снег и туман, бурный ветер, исполняющий слово Его,
\end{tcolorbox}
\begin{tcolorbox}
\textsubscript{9} горы и все холмы, дерева плодоносные и все кедры,
\end{tcolorbox}
\begin{tcolorbox}
\textsubscript{10} звери и всякий скот, пресмыкающиеся и птицы крылатые,
\end{tcolorbox}
\begin{tcolorbox}
\textsubscript{11} цари земные и все народы, князья и все судьи земные,
\end{tcolorbox}
\begin{tcolorbox}
\textsubscript{12} юноши и девицы, старцы и отроки
\end{tcolorbox}
\begin{tcolorbox}
\textsubscript{13} да хвалят имя Господа, ибо имя Его единого превознесенно, слава Его на земле и на небесах.
\end{tcolorbox}
\begin{tcolorbox}
\textsubscript{14} Он возвысил рог народа Своего, славу всех святых Своих, сынов Израилевых, народа, близкого к Нему. Аллилуия.
\end{tcolorbox}
\subsection{CHAPTER 149}
\begin{tcolorbox}
\textsubscript{1} Пойте Господу песнь новую; хвала Ему в собрании святых.
\end{tcolorbox}
\begin{tcolorbox}
\textsubscript{2} Да веселится Израиль о Создателе своем; сыны Сиона да радуются о Царе своем.
\end{tcolorbox}
\begin{tcolorbox}
\textsubscript{3} да хвалят имя Его с ликами, на тимпане и гуслях да поют Ему,
\end{tcolorbox}
\begin{tcolorbox}
\textsubscript{4} ибо благоволит Господь к народу Своему, прославляет смиренных спасением.
\end{tcolorbox}
\begin{tcolorbox}
\textsubscript{5} Да торжествуют святые во славе, да радуются на ложах своих.
\end{tcolorbox}
\begin{tcolorbox}
\textsubscript{6} Да будут славословия Богу в устах их, и меч обоюдоострый в руке их,
\end{tcolorbox}
\begin{tcolorbox}
\textsubscript{7} для того, чтобы совершать мщение над народами, наказание над племенами,
\end{tcolorbox}
\begin{tcolorbox}
\textsubscript{8} заключать царей их в узы и вельмож их в оковы железные,
\end{tcolorbox}
\begin{tcolorbox}
\textsubscript{9} производить над ними суд писанный. Честь сия--всем святым Его. Аллилуия.
\end{tcolorbox}
\subsection{CHAPTER 150}
\begin{tcolorbox}
\textsubscript{1} Хвалите Бога во святыне Его, хвалите Его на тверди силы Его.
\end{tcolorbox}
\begin{tcolorbox}
\textsubscript{2} Хвалите Его по могуществу Его, хвалите Его по множеству величия Его.
\end{tcolorbox}
\begin{tcolorbox}
\textsubscript{3} Хвалите Его со звуком трубным, хвалите Его на псалтири и гуслях.
\end{tcolorbox}
\begin{tcolorbox}
\textsubscript{4} Хвалите Его с тимпаном и ликами, хвалите Его на струнах и органе.
\end{tcolorbox}
\begin{tcolorbox}
\textsubscript{5} Хвалите Его на звучных кимвалах, хвалите Его на кимвалах громогласных.
\end{tcolorbox}
\begin{tcolorbox}
\textsubscript{6} Все дышащее да хвалит Господа! Аллилуия.
\end{tcolorbox}
