\section{BOOK 99}
\subsection{CHAPTER 1}
\begin{tcolorbox}
\textsubscript{1} Пророчество о Ниневии; книга видений Наума Елкосеянина.
\end{tcolorbox}
\begin{tcolorbox}
\textsubscript{2} Господь есть Бог ревнитель и мститель; мститель Господь и страшен в гневе: мстит Господь врагам Своим и не пощадит противников Своих.
\end{tcolorbox}
\begin{tcolorbox}
\textsubscript{3} Господь долготерпелив и велик могуществом, и не оставляет без наказания; в вихре и в буре шествие Господа, облако--пыль от ног Его.
\end{tcolorbox}
\begin{tcolorbox}
\textsubscript{4} Запретит Он морю, и оно высыхает, и все реки иссякают; вянет Васан и Кармил, и блекнет цвет на Ливане.
\end{tcolorbox}
\begin{tcolorbox}
\textsubscript{5} Горы трясутся пред Ним, и холмы тают, и земля колеблется пред лицем Его, и вселенная и все живущие в ней.
\end{tcolorbox}
\begin{tcolorbox}
\textsubscript{6} Пред негодованием Его кто устоит? И кто стерпит пламя гнева Его? Гнев Его разливается как огонь; скалы распадаются пред Ним.
\end{tcolorbox}
\begin{tcolorbox}
\textsubscript{7} Благ Господь, убежище в день скорби, и знает надеющихся на Него.
\end{tcolorbox}
\begin{tcolorbox}
\textsubscript{8} Но всепотопляющим наводнением разрушит до основания [Ниневию], и врагов Его постигнет мрак.
\end{tcolorbox}
\begin{tcolorbox}
\textsubscript{9} Что умышляете вы против Господа? Он совершит истребление, и бедствие уже не повторится,
\end{tcolorbox}
\begin{tcolorbox}
\textsubscript{10} ибо сплетшиеся между собою как терновник и упившиеся как пьяницы, они пожраны будут совершенно, как сухая солома.
\end{tcolorbox}
\begin{tcolorbox}
\textsubscript{11} Из тебя произошел умысливший злое против Господа, составивший совет нечестивый.
\end{tcolorbox}
\begin{tcolorbox}
\textsubscript{12} Так говорит Господь: хотя они безопасны и многочисленны, но они будут посечены и исчезнут; а тебя, хотя Я отягощал, более не буду отягощать.
\end{tcolorbox}
\begin{tcolorbox}
\textsubscript{13} И ныне Я сокрушу ярмо его, лежащее на тебе, и узы твои разорву.
\end{tcolorbox}
\begin{tcolorbox}
\textsubscript{14} А о тебе, [Ассур], Господь определил: не будет более семени с твоим именем; из дома бога твоего истреблю истуканов и кумиров; приготовлю тебе в нем могилу, потому что ты будешь в презрении.
\end{tcolorbox}
\begin{tcolorbox}
\textsubscript{15} Вот, на горах--стопы благовестника, возвещающего мир: празднуй, Иудея, праздники твои, исполняй обеты твои, ибо не будет более проходить по тебе нечестивый: он совсем уничтожен.
\end{tcolorbox}
\subsection{CHAPTER 2}
\begin{tcolorbox}
\textsubscript{1} Поднимается на тебя разрушитель: охраняй твердыни, стереги дорогу, укрепи чресла, собирайся с силами.
\end{tcolorbox}
\begin{tcolorbox}
\textsubscript{2} Ибо восстановит Господь величие Иакова, как величие Израиля, потому что опустошили их опустошители и виноградные ветви их истребили.
\end{tcolorbox}
\begin{tcolorbox}
\textsubscript{3} Щит героев его красен; воины его в одеждах багряных; огнем сверкают колесницы в день приготовления к бою, и лес копьев волнуется.
\end{tcolorbox}
\begin{tcolorbox}
\textsubscript{4} По улицам несутся колесницы, гремят на площадях; блеск от них, как от огня; сверкают, как молния.
\end{tcolorbox}
\begin{tcolorbox}
\textsubscript{5} Он вызывает храбрых своих, но они спотыкаются на ходу своем; поспешают на стены города, но осада уже устроена.
\end{tcolorbox}
\begin{tcolorbox}
\textsubscript{6} Речные ворота отворяются, и дворец разрушается.
\end{tcolorbox}
\begin{tcolorbox}
\textsubscript{7} Решено: она будет обнажена и отведена в плен, и рабыни ее будут стонать как голуби, ударяя себя в грудь.
\end{tcolorbox}
\begin{tcolorbox}
\textsubscript{8} Ниневия со времени существования своего была как пруд, полный водою, а они бегут. 'Стойте, стойте!' Но никто не оглядывается.
\end{tcolorbox}
\begin{tcolorbox}
\textsubscript{9} Расхищайте серебро, расхищайте золото! нет конца запасам всякой драгоценной утвари.
\end{tcolorbox}
\begin{tcolorbox}
\textsubscript{10} Разграблена, опустошена и разорена она, --и тает сердце, колени трясутся; у всех в чреслах сильная боль, и лица у всех потемнели.
\end{tcolorbox}
\begin{tcolorbox}
\textsubscript{11} Где теперь логовище львов и то пастбище для львенков, по которому ходил лев, львица и львенок, и никто не пугал их, --
\end{tcolorbox}
\begin{tcolorbox}
\textsubscript{12} лев, похищающий для насыщения щенков своих, и задушающий для львиц своих, и наполняющий добычею пещеры свои и логовища свои похищенным?
\end{tcolorbox}
\begin{tcolorbox}
\textsubscript{13} Вот, Я--на тебя! говорит Господь Саваоф. И сожгу в дыму колесницы твои, и меч пожрет львенков твоих, и истреблю с земли добычу твою, и не будет более слышим голос послов твоих.
\end{tcolorbox}
\subsection{CHAPTER 3}
\begin{tcolorbox}
\textsubscript{1} Горе городу кровей! весь он полон обмана и убийства; не прекращается в нем грабительство.
\end{tcolorbox}
\begin{tcolorbox}
\textsubscript{2} Слышны хлопанье бича и стук крутящихся колес, ржание коня и грохот скачущей колесницы.
\end{tcolorbox}
\begin{tcolorbox}
\textsubscript{3} Несется конница, сверкает меч и блестят копья; убитых множество и груды трупов: нет конца трупам, спотыкаются о трупы их.
\end{tcolorbox}
\begin{tcolorbox}
\textsubscript{4} Это--за многие блудодеяния развратницы приятной наружности, искусной в чародеянии, которая блудодеяниями своими продает народы и чарованиями своими--племена.
\end{tcolorbox}
\begin{tcolorbox}
\textsubscript{5} Вот, Я--на тебя! говорит Господь Саваоф. И подниму на лице твое края одежды твоей и покажу народам наготу твою и царствам срамоту твою.
\end{tcolorbox}
\begin{tcolorbox}
\textsubscript{6} И забросаю тебя мерзостями, сделаю тебя презренною и выставлю тебя на позор.
\end{tcolorbox}
\begin{tcolorbox}
\textsubscript{7} И будет то, что всякий, увидев тебя, побежит от тебя и скажет: 'разорена Ниневия! Кто пожалеет о ней? где найду я утешителей для тебя?'
\end{tcolorbox}
\begin{tcolorbox}
\textsubscript{8} Разве ты лучше Но-Аммона, находящегося между реками, окруженного водою, которого вал было море, и море служило стеною его?
\end{tcolorbox}
\begin{tcolorbox}
\textsubscript{9} Ефиопия и Египет с бесчисленным множеством других служили ему подкреплением; Копты и Ливийцы приходили на помощь тебе.
\end{tcolorbox}
\begin{tcolorbox}
\textsubscript{10} Но и он переселен, пошел в плен; даже и младенцы его разбиты на перекрестках всех улиц, а о знатных его бросали жребий, и все вельможи его окованы цепями.
\end{tcolorbox}
\begin{tcolorbox}
\textsubscript{11} Так и ты--опьянеешь и скроешься; так и ты будешь искать защиты от неприятеля.
\end{tcolorbox}
\begin{tcolorbox}
\textsubscript{12} Все укрепления твои подобны смоковнице со спелыми плодами: если тряхнуть их, то они упадут прямо в рот желающего есть.
\end{tcolorbox}
\begin{tcolorbox}
\textsubscript{13} Вот, и народ твой, как женщины у тебя: врагам твоим настежь отворятся ворота земли твоей, огонь пожрет запоры твои.
\end{tcolorbox}
\begin{tcolorbox}
\textsubscript{14} Начерпай воды на время осады; укрепляй крепости твои; пойди в грязь, топчи глину, исправь печь для обжигания кирпичей.
\end{tcolorbox}
\begin{tcolorbox}
\textsubscript{15} Там пожрет тебя огонь, посечет тебя меч, поест тебя как гусеница, хотя бы ты умножился как гусеница, умножился как саранча.
\end{tcolorbox}
\begin{tcolorbox}
\textsubscript{16} Купцов у тебя стало более, нежели звезд на небе; но эта саранча рассеется и улетит.
\end{tcolorbox}
\begin{tcolorbox}
\textsubscript{17} Князья твои--как саранча, и военачальники твои--как рои мошек, которые во время холода гнездятся в щелях [стен], и когда взойдет солнце, то разлетаются, и не узнаешь места, где они были.
\end{tcolorbox}
\begin{tcolorbox}
\textsubscript{18} Спят пастыри твои, царь Ассирийский, покоятся вельможи твои; народ твой рассеялся по горам, и некому собрать его.
\end{tcolorbox}
\begin{tcolorbox}
\textsubscript{19} Нет врачевства для раны твоей, болезненна язва твоя. Все, услышавшие весть о тебе, будут рукоплескать о тебе, ибо на кого не простиралась беспрестанно злоба твоя?
\end{tcolorbox}
