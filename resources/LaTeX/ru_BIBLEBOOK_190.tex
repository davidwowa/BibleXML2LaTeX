\section{BOOK 189}
\subsection{CHAPTER 1}
\begin{tcolorbox}
\textsubscript{1} Старец--возлюбленному Гаию, которого я люблю по истине.
\end{tcolorbox}
\begin{tcolorbox}
\textsubscript{2} Возлюбленный! молюсь, чтобы ты здравствовал и преуспевал во всем, как преуспевает душа твоя.
\end{tcolorbox}
\begin{tcolorbox}
\textsubscript{3} Ибо я весьма обрадовался, когда пришли братия и засвидетельствовали о твоей верности, как ты ходишь в истине.
\end{tcolorbox}
\begin{tcolorbox}
\textsubscript{4} Для меня нет большей радости, как слышать, что дети мои ходят в истине.
\end{tcolorbox}
\begin{tcolorbox}
\textsubscript{5} Возлюбленный! ты как верный поступаешь в том, что делаешь для братьев и для странников.
\end{tcolorbox}
\begin{tcolorbox}
\textsubscript{6} Они засвидетельствовали перед церковью о твоей любви. Ты хорошо поступишь, если отпустишь их, как должно ради Бога,
\end{tcolorbox}
\begin{tcolorbox}
\textsubscript{7} ибо они ради имени Его пошли, не взяв ничего от язычников.
\end{tcolorbox}
\begin{tcolorbox}
\textsubscript{8} Итак мы должны принимать таковых, чтобы сделаться споспешниками истине.
\end{tcolorbox}
\begin{tcolorbox}
\textsubscript{9} Я писал церкви; но любящий первенствовать у них Диотреф не принимает нас.
\end{tcolorbox}
\begin{tcolorbox}
\textsubscript{10} Посему, если я приду, то напомню о делах, которые он делает, понося нас злыми словами, и не довольствуясь тем, и сам не принимает братьев, и запрещает желающим, и изгоняет из церкви.
\end{tcolorbox}
\begin{tcolorbox}
\textsubscript{11} Возлюбленный! не подражай злу, но добру. Кто делает добро, тот от Бога; а делающий зло не видел Бога.
\end{tcolorbox}
\begin{tcolorbox}
\textsubscript{12} О Димитрии засвидетельствовано всеми и самою истиною; свидетельствуем также и мы, и вы знаете, что свидетельство наше истинно.
\end{tcolorbox}
\begin{tcolorbox}
\textsubscript{13} Многое имел я писать; но не хочу писать к тебе чернилами и тростью,
\end{tcolorbox}
\begin{tcolorbox}
\textsubscript{14} а надеюсь скоро увидеть тебя и поговорить устами к устам. (1-15) Мир тебе. Приветствуют тебя друзья; приветствуй друзей поименно. Аминь.
\end{tcolorbox}
