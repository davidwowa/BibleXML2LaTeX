\section{BOOK 65}
\subsection{CHAPTER 1}
\begin{tcolorbox}
\textsubscript{1} Об'явлення Ісуса Христа, яке дав Йому Бог, щоб показати Своїм рабам, що незабаром статися має. І Він показав, і послав Своїм Анголом рабові Своєму Іванові,
\end{tcolorbox}
\begin{tcolorbox}
\textsubscript{2} який свідчив про Слово Боже, і про свідчення Ісуса Христа, і про все, що він бачив.
\end{tcolorbox}
\begin{tcolorbox}
\textsubscript{3} Блаженний, хто читає, і ті, хто слухає слова пророцтва та додержує написане в ньому, час бо близький!
\end{tcolorbox}
\begin{tcolorbox}
\textsubscript{4} Іван до семи Церков, що в Азії: благодать вам і мир від Того, Хто є, Хто був і Хто має прийти; і від семи духів, що перед престолом Його,
\end{tcolorbox}
\begin{tcolorbox}
\textsubscript{5} та від Ісуса Христа, а Він Свідок вірний, Первенець з мертвих і Владика земних царів. Йому, що нас полюбив і кров'ю Своєю обмив нас від наших гріхів,
\end{tcolorbox}
\begin{tcolorbox}
\textsubscript{6} що вчинив нас царями, священиками Богові й Отцеві Своєму, Тому слава та сила на вічні віки! Амінь.
\end{tcolorbox}
\begin{tcolorbox}
\textsubscript{7} Ото Він із хмарами йде, і побачить Його кожне око, і ті, що Його прокололи були, і всі племена землі будуть плакати за Ним. Так, амінь!
\end{tcolorbox}
\begin{tcolorbox}
\textsubscript{8} Я Альфа й Омега, говорить Господь, Бог, Той, Хто є, і Хто був, і Хто має прийти, Вседержитель!
\end{tcolorbox}
\begin{tcolorbox}
\textsubscript{9} Я, Іван, ваш брат і спільник у біді, і в царстві, і в терпінні в Ісусі, був на острові, що зветься Патмос, за Слово Боже і за свідчення Ісуса Христа.
\end{tcolorbox}
\begin{tcolorbox}
\textsubscript{10} Я був у дусі Господнього дня, і почув за собою голос гучний, немов сурми,
\end{tcolorbox}
\begin{tcolorbox}
\textsubscript{11} який говорив: Що бачиш, напиши те до книги, і пошли до сімох Церков: до Ефесу, і до Смірни, і до Пергаму, і до Тіятирів, і до Сард, і до Філядельфії, і до Лаодикії.
\end{tcolorbox}
\begin{tcolorbox}
\textsubscript{12} І я оглянувся, щоб побачити голос, що говорив зо мною. І, оглянувшись, я побачив сім свічників золотих;
\end{tcolorbox}
\begin{tcolorbox}
\textsubscript{13} а посеред семи свічників Подібного до Людського Сина, одягненого в довгу одежу і підперезаного по грудях золотим поясом.
\end{tcolorbox}
\begin{tcolorbox}
\textsubscript{14} А Його голова та волосся білі, немов біла вовна, як сніг; а очі Його немов полум'я огняне.
\end{tcolorbox}
\begin{tcolorbox}
\textsubscript{15} А ноги Його подібні до міді, розпалені, наче в печі; а голос Його немов шум великої води.
\end{tcolorbox}
\begin{tcolorbox}
\textsubscript{16} І сім зір Він держав у правиці Своїй, а з уст Його меч обосічний виходив, а обличчя Його, немов сонце, що світить у силі своїй.
\end{tcolorbox}
\begin{tcolorbox}
\textsubscript{17} І коли я побачив Його, то до ніг Йому впав, немов мертвий. І поклав Він на мене правицю Свою та й промовив мені: Не лякайся! Я Перший і Останній,
\end{tcolorbox}
\begin{tcolorbox}
\textsubscript{18} і Живий. І був Я мертвий, а ось Я Живий на вічні віки. І маю ключі Я від смерти й від аду.
\end{tcolorbox}
\begin{tcolorbox}
\textsubscript{19} Отже, напиши, що ти бачив, і що є, і що має бути по цьому!
\end{tcolorbox}
\begin{tcolorbox}
\textsubscript{20} Таємниця семи зір, що бачив ти їх на правиці Моїй, і семи свічників золотих: сім зір, то Анголи семи Церков, а сім свічників, що ти бачив, то сім Церков.
\end{tcolorbox}
\subsection{CHAPTER 2}
\begin{tcolorbox}
\textsubscript{1} До Ангола Церкви в Ефесі напиши: Оце каже Той, Хто тримає сім зір у правиці Своїй, Хто ходить серед семи свічників золотих:
\end{tcolorbox}
\begin{tcolorbox}
\textsubscript{2} Я знаю діла твої, і працю твою, і твою терпеливість, і що не можеш терпіти лихих, і випробував тих, хто себе називає апостолами, але ними не є, і знайшов, що фальшиві вони.
\end{tcolorbox}
\begin{tcolorbox}
\textsubscript{3} І ти маєш терпіння, і працював для Ймення Мого, але не знемігся.
\end{tcolorbox}
\begin{tcolorbox}
\textsubscript{4} Але маю на тебе, що ти покинув свою першу любов.
\end{tcolorbox}
\begin{tcolorbox}
\textsubscript{5} Отож, пам'ятай, звідки ти впав, і покайся, і вчинки давніші роби. Коли ж ні, то до тебе прийду незабаром, і зрушу твого свічника з його місця, якщо не покаєшся.
\end{tcolorbox}
\begin{tcolorbox}
\textsubscript{6} Але маєш оце, що ненавидиш учинки Николаїтів, яких і Я ненавиджу.
\end{tcolorbox}
\begin{tcolorbox}
\textsubscript{7} Хто має вухо, хай чує, що Дух промовляє Церквам: переможцеві дам їсти від дерева життя, яке в раю Божім.
\end{tcolorbox}
\begin{tcolorbox}
\textsubscript{8} А до Ангола Церкви в Смірні напиши: Оце каже Перший й Останній, що був мертвий й ожив:
\end{tcolorbox}
\begin{tcolorbox}
\textsubscript{9} Я знаю діла твої, і біду, і убозтво, але ти багатий, і зневагу тих, що говорять про себе, ніби юдеї вони, та ними не є, але вони зборище сатани.
\end{tcolorbox}
\begin{tcolorbox}
\textsubscript{10} Не бійся того, що маєш страждати! Ось диявол вкидатиме декого з вас до в'язниць, щоб вас випробувати. І будете мати біду десять день. Будь вірний до смерти, і Я тобі дам вінця життя!
\end{tcolorbox}
\begin{tcolorbox}
\textsubscript{11} Хто має вухо, хай чує, що Дух промовляє Церквам: переможець не буде пошкоджений від другої смерти.
\end{tcolorbox}
\begin{tcolorbox}
\textsubscript{12} А до Ангола Церкви в Пергамі напиши: Оце каже Той, що має меча обосічного:
\end{tcolorbox}
\begin{tcolorbox}
\textsubscript{13} Я знаю діла твої, і що де ти живеш, там престол сатани. І тримаєш ти Ймення Моє, і ти не відкинувся від віри Моєї навіть за днів, коли в вас, де живе сатана, був убитий Антипа, свідок Мій вірний.
\end{tcolorbox}
\begin{tcolorbox}
\textsubscript{14} Але трохи Я маю на тебе, бо маєш там тих, хто тримається науки Валаама, що навчав був Балака покласти спотикання перед синами Ізраїля, щоб їли ідольські жертви й розпусту чинили.
\end{tcolorbox}
\begin{tcolorbox}
\textsubscript{15} Так маєш і ти таких, що тримаються науки Николаїтської так само.
\end{tcolorbox}
\begin{tcolorbox}
\textsubscript{16} Тож покайся! Коли ж ні, то до тебе прийду незабаром, і воюватиму з ними мечем Своїх уст.
\end{tcolorbox}
\begin{tcolorbox}
\textsubscript{17} Хто має вухо, хай чує, що Дух промовляє Церквам: переможцеві дам їсти приховану манну, і дам йому білого каменя, а на камені написане ймення нове, якого не знає ніхто, тільки той, хто приймає його.
\end{tcolorbox}
\begin{tcolorbox}
\textsubscript{18} І до Ангола Церкви в Тіятирах напиши: Оце каже Син Божий, що має очі Свої, як полум'я огняне, а ноги Його подібні до міді:
\end{tcolorbox}
\begin{tcolorbox}
\textsubscript{19} Я знаю діла твої, і любов, і віру, і службу, і твою терпеливість, і останні вчинки твої, що більші за перші.
\end{tcolorbox}
\begin{tcolorbox}
\textsubscript{20} Але маю на тебе, що жінці Єзавелі, яка каже, ніби вона пророкиня, ти попускаєш навчати та зводити рабів Моїх, чинити розпусту та їсти ідольські жертви.
\end{tcolorbox}
\begin{tcolorbox}
\textsubscript{21} І Я дав був їй часу, щоб покаялася, та вона не схотіла покаятися в розпусті своїй.
\end{tcolorbox}
\begin{tcolorbox}
\textsubscript{22} Ось Я кину її на ложе, а тих, що чинять із нею розпусту, у велику біду, коли тільки в учинках своїх не покаються,
\end{tcolorbox}
\begin{tcolorbox}
\textsubscript{23} а діти її поб'ю смертю. І пізнають усі Церкви, що Я Той, Хто нирки й серця вивіряє, і Я кожному з вас дам за вчинками вашими.
\end{tcolorbox}
\begin{tcolorbox}
\textsubscript{24} А вам, та іншим, що в Тіятирах, що не мають науки цієї, і як кажуть не розуміють так званих глибин сатани, кажу: не накладу на вас іншого тягару,
\end{tcolorbox}
\begin{tcolorbox}
\textsubscript{25} тільки те, що ви маєте, тримайте, аж поки прийду.
\end{tcolorbox}
\begin{tcolorbox}
\textsubscript{26} А переможцеві, і тому, хто аж до кінця додержує Мої вчинки, Я дам йому владу над поганами,
\end{tcolorbox}
\begin{tcolorbox}
\textsubscript{27} і буде пасти їх залізним жезлом; вони, немов глиняний посуд, покрушаться, як і Я одержав владу від Свого Отця,
\end{tcolorbox}
\begin{tcolorbox}
\textsubscript{28} і дам Я йому зорю досвітню.
\end{tcolorbox}
\begin{tcolorbox}
\textsubscript{29} Хто має вухо, хай чує, що Дух промовляє Церквам!
\end{tcolorbox}
\subsection{CHAPTER 3}
\begin{tcolorbox}
\textsubscript{1} А до Ангола Церкви в Сардах напиши: Оце каже Той, Хто має сім Божих духів і сім зір: Я знаю діла твої, що маєш ім'я, ніби живий, а ти мертвий.
\end{tcolorbox}
\begin{tcolorbox}
\textsubscript{2} Будь чуйний та решту зміцняй, що мають померти. Бо Я не знайшов твоїх діл закінченими перед Богом Моїм.
\end{tcolorbox}
\begin{tcolorbox}
\textsubscript{3} Отож, пам'ятай, як ти взяв і почув, і бережи, і покайся. А коли ти не чуйний, то на тебе прийду, немов злодій, і ти знати не будеш, якої години на тебе прийду.
\end{tcolorbox}
\begin{tcolorbox}
\textsubscript{4} Та ти маєш і в Сардах кілька імен, що одежі своєї вони не споганили, і в білій зо Мною ходитимуть, бо гідні вони.
\end{tcolorbox}
\begin{tcolorbox}
\textsubscript{5} Переможець зодягнеться в білу одежу, а ймення його Я не змию із книги життя, і ймення його визнаю перед Отцем Своїм і перед Його Анголами.
\end{tcolorbox}
\begin{tcolorbox}
\textsubscript{6} Хто має вухо, хай чує, що Дух промовляє Церквам!
\end{tcolorbox}
\begin{tcolorbox}
\textsubscript{7} І до Ангола Церкви в Філядельфії напиши: Оце каже Святий, Правдивий, що має ключа Давидового, що Він відчиняє, і ніхто не зачинить, що Він зачиняє, і ніхто не відчинить.
\end{tcolorbox}
\begin{tcolorbox}
\textsubscript{8} Я знаю діла твої. Ось Я перед тобою дверей не зачинив, і їх зачинити не може ніхто. Хоч малу маєш силу, але слово Моє ти зберіг, і від Ймення Мого не відкинувся.
\end{tcolorbox}
\begin{tcolorbox}
\textsubscript{9} Ось Я зроблю, що декого з зборища сатани, із тих, що себе називають юдеями, та ними не є, але кажуть неправду, ось Я зроблю, що вони прийдуть та вклоняться перед ногами твоїми, і пізнають, що Я полюбив тебе.
\end{tcolorbox}
\begin{tcolorbox}
\textsubscript{10} А що ти зберіг слово терпіння Мого, то й Я тебе збережу від години випробовування, що має прийти на ввесь всесвіт, щоб випробувати мешканців землі.
\end{tcolorbox}
\begin{tcolorbox}
\textsubscript{11} Я прийду незабаром. Тримай, що ти маєш, щоб твого вінця ніхто не забрав.
\end{tcolorbox}
\begin{tcolorbox}
\textsubscript{12} Переможця зроблю Я стовпом у храмі Бога Мого, і він вже не вийде назовні, і на нім напишу Ім'я Бога Мого й ім'я міста Бога Мого, Єрусалиму Нового, що з неба сходить від Бога Мого, та нове Ім'я Своє.
\end{tcolorbox}
\begin{tcolorbox}
\textsubscript{13} Хто має вухо, хай чує, що Дух промовляє Церквам!
\end{tcolorbox}
\begin{tcolorbox}
\textsubscript{14} І до Ангола Церкви в Лаодикії напиши: Оце каже Амінь, Свідок вірний і правдивий, початок Божого творива:
\end{tcolorbox}
\begin{tcolorbox}
\textsubscript{15} Я знаю діла твої, що ти не холодний, ані гарячий. Якби то холодний чи гарячий ти був!
\end{tcolorbox}
\begin{tcolorbox}
\textsubscript{16} А що ти літеплий, і ні гарячий, ані холодний, то виплюну тебе з Своїх уст...
\end{tcolorbox}
\begin{tcolorbox}
\textsubscript{17} Бо ти кажеш: Я багатий, і збагатів, і не потребую нічого. А не знаєш, що ти нужденний, і мізерний, і вбогий, і сліпий, і голий!
\end{tcolorbox}
\begin{tcolorbox}
\textsubscript{18} Раджу тобі купити в Мене золота, в огні перечищеного, щоб збагатитись, і білу одежу, щоб зодягтися, і щоб ганьба наготи твоєї не видна була, а мастю на очі намасти свої очі, щоб бачити.
\end{tcolorbox}
\begin{tcolorbox}
\textsubscript{19} Кого Я люблю, тому докоряю й караю того. Будь же ревний і покайся!
\end{tcolorbox}
\begin{tcolorbox}
\textsubscript{20} Ось Я стою під дверима та стукаю: коли хто почує Мій голос і двері відчинить, Я до нього ввійду, і буду вечеряти з ним, а він зо Мною.
\end{tcolorbox}
\begin{tcolorbox}
\textsubscript{21} Переможцеві сісти Я дам на Моєму престолі зо Мною, як і Я переміг був, і з Отцем Своїм сів на престолі Його.
\end{tcolorbox}
\begin{tcolorbox}
\textsubscript{22} Хто має вухо, хай чує, що Дух промовляє Церквам!
\end{tcolorbox}
\subsection{CHAPTER 4}
\begin{tcolorbox}
\textsubscript{1} По цьому я поглянув, і ось двері на небі відчинені, і перший голос, що я чув його, як сурму, що зо мною говорив, сказав: Іди сюди, і Я тобі покажу, що статися має по цьому!
\end{tcolorbox}
\begin{tcolorbox}
\textsubscript{2} І зараз у дусі я був. І ось престол стояв на небі, а на престолі Сидячий.
\end{tcolorbox}
\begin{tcolorbox}
\textsubscript{3} А Сидячий подібний був з вигляду до каменя яспіса й сардиса, а веселка навколо престолу видом подібна була до смарагду.
\end{tcolorbox}
\begin{tcolorbox}
\textsubscript{4} А навколо престолу двадцять чотири престоли, а на престолах я бачив двадцятьох чотирьох старців, що сиділи, у шати білі одягнені, а на головах своїх мали вінці золоті.
\end{tcolorbox}
\begin{tcolorbox}
\textsubscript{5} А від престолу виходили блискавки, і голоси, і громи. А перед престолом горіли сім свічників огняних, а вони сім духів Божих.
\end{tcolorbox}
\begin{tcolorbox}
\textsubscript{6} І перед престолом як море скляне, до кришталю подібне. А серед престолу й навколо престолу четверо тварин, повні очей спереду й ззаду.
\end{tcolorbox}
\begin{tcolorbox}
\textsubscript{7} І перша тварина подібна до лева, а друга тварина подібна до теляти, а третя тварина мала лице, як людина, а четверта тварина подібна до орла, що летить.
\end{tcolorbox}
\begin{tcolorbox}
\textsubscript{8} І ті чотири тварині, кожна з них мала навколо по шість крил, а всередині повна очей. І спокою не мають вони день і ніч, промовляючи: Свят, свят, свят Господь, Бог Вседержитель, що Він був, і що є, і що має прийти!
\end{tcolorbox}
\begin{tcolorbox}
\textsubscript{9} І коли ті тварини складають славу, і честь, і подяку Тому, Хто сидить на престолі й живе віки вічні,
\end{tcolorbox}
\begin{tcolorbox}
\textsubscript{10} тоді падають двадцять чотири старці перед Тим, Хто сидить на престолі, і вклоняються Тому, Хто живе віки вічні, і складають вінці свої перед престолом та кажуть:
\end{tcolorbox}
\begin{tcolorbox}
\textsubscript{11} Достойний Ти, Господи й Боже наш, прийняти славу, і честь, і силу, бо все Ти створив, і з волі Твоєї існує та створене все!
\end{tcolorbox}
\subsection{CHAPTER 5}
\begin{tcolorbox}
\textsubscript{1} І я бачив в правиці Того, Хто сидить на престолі, книгу, написану всередині й назовні, і запечатану сімома печатками.
\end{tcolorbox}
\begin{tcolorbox}
\textsubscript{2} І бачив я потужного Ангола, який гучним голосом кликав: Хто гідний розгорнути книгу, і зламати печатки її?
\end{tcolorbox}
\begin{tcolorbox}
\textsubscript{3} І не міг ніхто ні на небі, ні на землі, ані під землею розгорнути книги, ані навіть зазирнути в неї.
\end{tcolorbox}
\begin{tcolorbox}
\textsubscript{4} І плакав я гірко, що не знайшовся ані один гідний розгорнути й прочитати книгу, ані навіть зазирнути в неї.
\end{tcolorbox}
\begin{tcolorbox}
\textsubscript{5} А один із старців промовив до мене: Не плач! Ось Лев, що з племени Юдиного, корень Давидів, переміг так, що може розгорнути книгу, і зламати сім печаток її.
\end{tcolorbox}
\begin{tcolorbox}
\textsubscript{6} І я глянув, і ось серед престолу й чотирьох тварин і серед старців стоїть Агнець, як заколений, що має сім рогів і сім очей, а це сім Божих духів, посланих на всю землю.
\end{tcolorbox}
\begin{tcolorbox}
\textsubscript{7} І Він підійшов, і взяв книгу з правиці Того, Хто сидить на престолі.
\end{tcolorbox}
\begin{tcolorbox}
\textsubscript{8} А коли Він узяв книгу, то чотири тварині й двадцять чотири старці попадали перед Агнцем, а кожен мав гусла й золоті чаші, повні пахощів, а вони молитви святих.
\end{tcolorbox}
\begin{tcolorbox}
\textsubscript{9} І нову пісню співають вони, промовляючи: Ти достойний узяти цю книгу, і розкрити печатки її, бо Ти був заколений, і кров'ю Своєю Ти викупив людей Богові з усякого племени, і язика, і народу, і люду.
\end{tcolorbox}
\begin{tcolorbox}
\textsubscript{10} І Ти їх зробив для нашого Бога царями, і священиками, і вони на землі царюватимуть!
\end{tcolorbox}
\begin{tcolorbox}
\textsubscript{11} І я бачив, і чув голос багатьох Анголів навколо престолу, і тварин, і старців, і число їх було десятки тисяч раз по десять тисяч і тисячі тисяч.
\end{tcolorbox}
\begin{tcolorbox}
\textsubscript{12} І казали вони гучним голосом: Достойний Агнець, що заколений, прийняти силу, і багатство, і мудрість, і міць, і честь, і славу, і благословення!
\end{tcolorbox}
\begin{tcolorbox}
\textsubscript{13} І кожне створіння, що воно на небі, і на землі, і під землею, і на морі, і все, що в них, чув я, говорило: Тому, Хто сидить на престолі, і Агнцеві благословення, і честь, і слава, і сила на вічні віки!
\end{tcolorbox}
\begin{tcolorbox}
\textsubscript{14} А чотири тварині казали: Амінь! І двадцять чотири старці попадали та поклонились Тому, Хто живе повік віку!
\end{tcolorbox}
\subsection{CHAPTER 6}
\begin{tcolorbox}
\textsubscript{1} І я бачив, що Агнець розкрив одну з семи печаток, і почув я одну з чотирьох тих тварин, яка говорила, як голосом грому: Підійди!
\end{tcolorbox}
\begin{tcolorbox}
\textsubscript{2} І я глянув, і ось кінь білий, а той, хто на ньому сидів, мав лука. І вінця йому дано, і він вийшов, немов переможець, і щоб перемогти.
\end{tcolorbox}
\begin{tcolorbox}
\textsubscript{3} І коли другу печатку розкрив, я другу тварину почув, що казала: Підійди!
\end{tcolorbox}
\begin{tcolorbox}
\textsubscript{4} І вийшов кінь другий, червоний. А тому, хто на ньому сидів, було дано взяти мир із землі та щоб убивали один одного. І меч великий був даний йому.
\end{tcolorbox}
\begin{tcolorbox}
\textsubscript{5} І коли третю печатку розкрив, я третю тварину почув, що казала: Підійди! І я глянув, і ось кінь вороний. А той, хто на ньому сидів, мав вагу в своїй руці.
\end{tcolorbox}
\begin{tcolorbox}
\textsubscript{6} І я ніби голос почув посеред чотирьох тих тварин, що казав: Ківш пшениці за динарія, і три ковші ячменю за динарія, а оливи й вина не марнуй!
\end{tcolorbox}
\begin{tcolorbox}
\textsubscript{7} А коли Він четверту печатку розкрив, я четверту тварину почув, що казала: Підійди!
\end{tcolorbox}
\begin{tcolorbox}
\textsubscript{8} І я глянув, і ось кінь чалий. А той, хто на ньому сидів, на ім'я йому Смерть, за ним же слідом ішов Ад. І дана їм влада була на четвертій частині землі забивати мечем, і голодом, і мором, і земними звірми.
\end{tcolorbox}
\begin{tcolorbox}
\textsubscript{9} І коли п'яту печатку розкрив, я побачив під жертівником душі побитих за Боже Слово, і за свідчення, яке вони мали.
\end{tcolorbox}
\begin{tcolorbox}
\textsubscript{10} І кликнули вони гучним голосом, кажучи: Аж доки, Владико святий та правдивий, не будеш судити, і не мститимеш тим, хто живе на землі, за кров нашу?
\end{tcolorbox}
\begin{tcolorbox}
\textsubscript{11} І кожному з них дано білу одежу, і сказано їм іще трохи спочити, аж поки доповнять число їхні співслуги, і брати їхні, що будуть побиті, як і вони.
\end{tcolorbox}
\begin{tcolorbox}
\textsubscript{12} І коли шосту печатку розкрив, я поглянув, і ось сталось велике трясіння землі, і сонце зчорніло, як міх волосяний, і ввесь місяць зробився, як кров...
\end{tcolorbox}
\begin{tcolorbox}
\textsubscript{13} І на землю попадали зорі небесні, як фіґове дерево ронить свої недозрілі плоди, коли потрясе сильний вітер...
\end{tcolorbox}
\begin{tcolorbox}
\textsubscript{14} І небо сховалось, згорнувшись, немов той сувій пергамену, і кожна гора, і кожен острів порушилися з своїх місць...
\end{tcolorbox}
\begin{tcolorbox}
\textsubscript{15} І земні царі, і вельможі та тисячники, і багаті та сильні, і кожен раб та кожен вільний, поховались у печери та в скелі гірські,
\end{tcolorbox}
\begin{tcolorbox}
\textsubscript{16} та й кажуть до гір та до скель: Поспадайте на нас, і позакривайте ви нас від лиця Того, Хто сидить на престолі, і від гніву Агнця!...
\end{tcolorbox}
\begin{tcolorbox}
\textsubscript{17} Бо прийшов це великий день гніву Його, і хто встояти може?
\end{tcolorbox}
\subsection{CHAPTER 7}
\begin{tcolorbox}
\textsubscript{1} А по цьому я бачив чотирьох Анголів, що стояли на чотирьох кутах землі та тримали чотири земні вітри, щоб вітер не віяв на землю, ані на море, ані на жодне дерево.
\end{tcolorbox}
\begin{tcolorbox}
\textsubscript{2} І бачив я іншого Ангола, що від схід сонця виходив, і мав печатку Бога Живого. І він гучним голосом крикнув до чотирьох Анголів, що їм дано пошкодити землі та морю,
\end{tcolorbox}
\begin{tcolorbox}
\textsubscript{3} говорячи: Не шкодьте ані землі, ані морю, ані дереву, аж поки ми покладемо печатки рабам Бога нашого на їхніх чолах!
\end{tcolorbox}
\begin{tcolorbox}
\textsubscript{4} І почув я число попечатаних: сто сорок чотири тисячі попечатаних від усіх племен Ізраїлевих синів:
\end{tcolorbox}
\begin{tcolorbox}
\textsubscript{5} з племени Юдиного дванадцять тисяч попечатаних, з племени Рувимового дванадцять тисяч, з племени Ґадового дванадцять тисяч,
\end{tcolorbox}
\begin{tcolorbox}
\textsubscript{6} з племени Асирового дванадцять тисяч, з племени Нефталимового дванадцять тисяч, з племени Манасіїного дванадцять тисяч,
\end{tcolorbox}
\begin{tcolorbox}
\textsubscript{7} з племени Симеонового дванадцять тисяч, з племени Левіїного дванадцять тисяч, з племени Іссахарового дванадцять тисяч,
\end{tcolorbox}
\begin{tcolorbox}
\textsubscript{8} з племени Завулонового дванадцять тисяч, з племени Йосипового дванадцять тисяч, з племени Веніяминового дванадцять тисяч попечатаних.
\end{tcolorbox}
\begin{tcolorbox}
\textsubscript{9} Потому я глянув, і ось натовп великий, що його зрахувати не може ніхто, з усякого люду, і племен, і народів, і язиків, стояв перед престолом і перед Агнцем, зодягнені в білу одежу, а в їхніх руках було пальмове віття.
\end{tcolorbox}
\begin{tcolorbox}
\textsubscript{10} І взивали вони гучним голосом, кажучи: Спасіння нашому Богові, що сидить на престолі, і Агнцеві!
\end{tcolorbox}
\begin{tcolorbox}
\textsubscript{11} А всі Анголи стояли навколо престолу та старців і чотирьох тих тварин. І вони на обличчя попадали перед престолом, і вклонилися Богові,
\end{tcolorbox}
\begin{tcolorbox}
\textsubscript{12} кажучи: Амінь! Благословення, і слава, і мудрість, і хвала, і честь, і сила, і міць нашому Богу на вічні віки! Амінь!
\end{tcolorbox}
\begin{tcolorbox}
\textsubscript{13} І відповів один із старців, і до мене сказав: Оці, що зодягнені в білу одежу, хто вони й звідкіля поприходили?
\end{tcolorbox}
\begin{tcolorbox}
\textsubscript{14} І сказав я йому: Мій пане, ти знаєш! Він же мені відказав: Це ті, що прийшли від великого горя, і випрали одіж свою, та вибілили її в крові Агнця...
\end{tcolorbox}
\begin{tcolorbox}
\textsubscript{15} Тому то вони перед Божим престолом, і в храмі Його день і ніч Йому служать. А Той, Хто сидить на престолі, розтягне намета над ними.
\end{tcolorbox}
\begin{tcolorbox}
\textsubscript{16} Вони голоду й спраги терпіти не будуть уже, і не буде палити їх сонце, ані спека яка.
\end{tcolorbox}
\begin{tcolorbox}
\textsubscript{17} Бо Агнець, що серед престолу, буде їх пасти, і водитиме їх до джерел вод життя. І Бог кожну сльозу з очей їхніх зітре!
\end{tcolorbox}
\subsection{CHAPTER 8}
\begin{tcolorbox}
\textsubscript{1} І коли сьому печатку розкрив, німа тиша настала на небі десь на півгодини.
\end{tcolorbox}
\begin{tcolorbox}
\textsubscript{2} І я бачив сімох Анголів, що стояли перед Богом. І дано було їм сім сурем.
\end{tcolorbox}
\begin{tcolorbox}
\textsubscript{3} І прийшов другий Ангол, та й став перед жертівником із золотою кадильницею. І було йому дано багато кадила, щоб до молитов усіх святих додав на золотого жертівника, що перед престолом.
\end{tcolorbox}
\begin{tcolorbox}
\textsubscript{4} І знявся дим кадильний з молитвами святих від руки Ангола перед Бога.
\end{tcolorbox}
\begin{tcolorbox}
\textsubscript{5} А Ангол кадильницю взяв, і наповнив її огнем із жертівника, та й кинув на землю. І зчинилися громи, і гуркотнява, і блискавиці та трясіння землі...
\end{tcolorbox}
\begin{tcolorbox}
\textsubscript{6} І сім Анголів, що мали сім сурем, приготувалися, щоб сурмити.
\end{tcolorbox}
\begin{tcolorbox}
\textsubscript{7} І засурмив перший Ангол, і вчинилися град та огонь, перемішані з кров'ю, і впали на землю. І спалилась третина землі, і згоріла третина дерев, і всіляка зелена трава погоріла...
\end{tcolorbox}
\begin{tcolorbox}
\textsubscript{8} І засурмив другий Ангол, і немов би велика гора, розпалена огнем, була вкинена в море. І третина моря зробилася кров'ю,
\end{tcolorbox}
\begin{tcolorbox}
\textsubscript{9} і померла третина морського створіння, що мають життя, і загинула третина кораблів...
\end{tcolorbox}
\begin{tcolorbox}
\textsubscript{10} І засурмив третій Ангол, і велика зоря спала з неба, палаючи, як смолоскип. І спала вона на третину річок та на водні джерела.
\end{tcolorbox}
\begin{tcolorbox}
\textsubscript{11} А ймення зорі тій Полин. І стала третина води, як полин, і багато з людей повмирали з води, бо згіркла вона...
\end{tcolorbox}
\begin{tcolorbox}
\textsubscript{12} І засурмив Ангол четвертий, і вдарено третину сонця, і третину місяця, і третину зір, щоб затьмилася їхня третина, щоб третина дня не світила, так само ж і ніч...
\end{tcolorbox}
\begin{tcolorbox}
\textsubscript{13} І бачив, і чув я одного орла, що летів серед неба і кликав гучним голосом: Горе, горе, горе тим, хто живе на землі, від голосів сурмових позосталих трьох Анголів, що мають сурмити!...
\end{tcolorbox}
\subsection{CHAPTER 9}
\begin{tcolorbox}
\textsubscript{1} І засурмив п'ятий Ангол, і я бачив зорю, що спала із неба додолу. І їй даний був ключ від криниці безодньої.
\end{tcolorbox}
\begin{tcolorbox}
\textsubscript{2} І вона відімкнула криницю безодню, і дим повалив із криниці, мов дим із великої печі. І затьмилося сонце й повітря від криничного диму...
\end{tcolorbox}
\begin{tcolorbox}
\textsubscript{3} А з диму на землю вийшла сарана, і дано їй міць, як мають міць скорпіони земні.
\end{tcolorbox}
\begin{tcolorbox}
\textsubscript{4} І наказано їй, щоб вона не шкодила земній траві, ані жадному зіллю, ані жадному дереву, але тільки тим людям, які на чолах не мають печатки Божої.
\end{tcolorbox}
\begin{tcolorbox}
\textsubscript{5} І було дано їй, щоб їх не вбивати, але мучити п'ять місяців; а мука від неї, як мука від скорпіона, коли вкусить людину.
\end{tcolorbox}
\begin{tcolorbox}
\textsubscript{6} І в ті дні люди смерти шукатимуть, та не знайдуть її! Померти вони захотять, та втече від них смерть!...
\end{tcolorbox}
\begin{tcolorbox}
\textsubscript{7} А вигляд сарани був подібний до коней, на війну приготованих; а на головах у неї немов би вінки, подібні на золото, а обличчя її немов людські обличчя.
\end{tcolorbox}
\begin{tcolorbox}
\textsubscript{8} І мала волосся як волосся жіноче, а її зуби були немов лев'ячі.
\end{tcolorbox}
\begin{tcolorbox}
\textsubscript{9} І мала вона панцери, немов панцери залізні; а шум її крил немов шум колесниць, коли коней багато біжить на війну.
\end{tcolorbox}
\begin{tcolorbox}
\textsubscript{10} І мала хвости, подібні до скорпіонових, та жала, а в неї в хвостах її влада п'ять місяців шкодити людям.
\end{tcolorbox}
\begin{tcolorbox}
\textsubscript{11} І мала вона над собою царя, ангола безодні; йому по-єврейському ім'я Аваддон, а по-грецькому звався він Аполліон!
\end{tcolorbox}
\begin{tcolorbox}
\textsubscript{12} Одне горе минуло! Ось за ним ще два горя надходять!
\end{tcolorbox}
\begin{tcolorbox}
\textsubscript{13} І засурмив шостий Ангол, і я почув один голос із чотирьох рогів золотого жертівника, який перед Богом,
\end{tcolorbox}
\begin{tcolorbox}
\textsubscript{14} що казав шостому Анголові, який мав сурму: Розв'яжи чотирьох Анголів, що пов'язані при великій річці Ефраті.
\end{tcolorbox}
\begin{tcolorbox}
\textsubscript{15} І були порозв'язувані чотири Анголи, приготовані на годину, і на день, і на місяць, і на рік, щоб убили третину людей.
\end{tcolorbox}
\begin{tcolorbox}
\textsubscript{16} А число кінного війська двадцять тисяч раз по десять тисяч; і я чув їхнє число.
\end{tcolorbox}
\begin{tcolorbox}
\textsubscript{17} І так бачив я коней в видінні, а на них верхівців, що панцери мали огняні, і гіяцинтові, і сірчані. А голови в коней немов голови лев'ячі, а з їхнього рота виходив огонь, і дим, і сірка.
\end{tcolorbox}
\begin{tcolorbox}
\textsubscript{18} І побита була третина людей від цих трьох поразок, від огню, і від диму, і від сірки, що виходили з їхніх ротів.
\end{tcolorbox}
\begin{tcolorbox}
\textsubscript{19} Сила бо коней була в їхнім роті та в їхніх хвостах. А хвости їхні подібні до вужів, що мають голови, і ними вони шкоду чинять.
\end{tcolorbox}
\begin{tcolorbox}
\textsubscript{20} А решта людей, що не вбита була цими поразками, не покаялася за діла своїх рук, щоб не кланятись демонам, ані ідолам золотим, і срібним, і мідяним, і кам'яним, і дерев'яним, що не можуть вони ані бачити, ані чути, ані ходити.
\end{tcolorbox}
\begin{tcolorbox}
\textsubscript{21} І вони не покаялися в своїх убивствах, ані в чарах своїх, ні в розпусті своїй, ні в крадіжках своїх...
\end{tcolorbox}
\subsection{CHAPTER 10}
\begin{tcolorbox}
\textsubscript{1} І бачив я іншого потужного Ангола, що сходив із неба. Був одягнений в хмару, і над його головою веселка була, а обличчя його як стовпи огняні,
\end{tcolorbox}
\begin{tcolorbox}
\textsubscript{2} і мав у руці своїй книжку розгорнену. І він поставив свою праву ногу на море, а ліву на землю,
\end{tcolorbox}
\begin{tcolorbox}
\textsubscript{3} і закричав гучним голосом, як лев той ричить. І як він закричав, то заговорили сім громів голосами своїми.
\end{tcolorbox}
\begin{tcolorbox}
\textsubscript{4} А як заговорили сім громів голосами своїми, я хотів був писати. Та я почув голос із неба, що до мене казав: Запечатай оте, що сім громів казали, і того не пиши!
\end{tcolorbox}
\begin{tcolorbox}
\textsubscript{5} А Ангол, що я бачив його, як стояв він на морі й землі, зняв до неба правицю свою
\end{tcolorbox}
\begin{tcolorbox}
\textsubscript{6} та й поклявся Живучим по вічні віки, Який створив небо та те, що на ньому, і землю та те, що на ній, і море й що в нім, що вже часу не буде,
\end{tcolorbox}
\begin{tcolorbox}
\textsubscript{7} а дня голосу сьомого Ангола, коли він засурмить, довершиться Божа таємниця, як Він благовістив був Своїм рабам пророкам.
\end{tcolorbox}
\begin{tcolorbox}
\textsubscript{8} І голос, що я чув його з неба, став знов говорити зо мною й казати: Піди, та візьми розгорнену книжку з руки Ангола, що стоїть на морі й землі.
\end{tcolorbox}
\begin{tcolorbox}
\textsubscript{9} І пішов я до Ангола та й промовив йому, щоб дав мені книжку. А він мені каже: Візьми, і з'їж її! І гіркість учинить вона для твого живота, та в устах твоїх буде солодка, як мед.
\end{tcolorbox}
\begin{tcolorbox}
\textsubscript{10} І я взяв з руки Ангола книжку та й з'їв її. І була вона в устах моїх, немов мед той, солодка. Та коли її з'їв, вона гіркість зробила в моїм животі...
\end{tcolorbox}
\begin{tcolorbox}
\textsubscript{11} І сказали мені: Ти мусиш знову пророкувати про народи, і поган, і язики, і про багато царів.
\end{tcolorbox}
\subsection{CHAPTER 11}
\begin{tcolorbox}
\textsubscript{1} І дано тростину мені, подібну до палиці, і сказано: Устань, і зміряй храма Божого й жертівника, і тих, хто вклоняється в ньому.
\end{tcolorbox}
\begin{tcolorbox}
\textsubscript{2} А двір, що за храмом, лиши та не міряй його, бо він даний поганам, і сорок два місяці будуть топтати вони святе місто.
\end{tcolorbox}
\begin{tcolorbox}
\textsubscript{3} І звелю Я двом свідкам Своїм, і будуть вони пророкувати тисячу двісті й шістдесят день, зодягнені в волосяницю.
\end{tcolorbox}
\begin{tcolorbox}
\textsubscript{4} Вони дві оливі та два свічники, що стоять перед Богом землі.
\end{tcolorbox}
\begin{tcolorbox}
\textsubscript{5} І коли б хто схотів учинити їм кривду, то вийде огонь з їхніх уст, і поїсть ворогів їхніх. А коли хто захоче вчинити їм кривду, той отак мусить бути забитий.
\end{tcolorbox}
\begin{tcolorbox}
\textsubscript{6} Вони мають владу небо замкнути, щоб за днів їхніх пророцтва не йшов дощ. І мають владу вони над водою, у кров обертати її, і вдарити землю всілякою карою, скільки разів вони схочуть.
\end{tcolorbox}
\begin{tcolorbox}
\textsubscript{7} А коли вони скінчать свідоцтво своє, то звірина, що з безодні виходить, із ними війну поведе, і вона їх переможе та їх повбиває.
\end{tcolorbox}
\begin{tcolorbox}
\textsubscript{8} І їхні трупи полишить на майдані великого міста, що зветься духовно Содом і Єгипет, де й Господь наш був розп'ятий.
\end{tcolorbox}
\begin{tcolorbox}
\textsubscript{9} І багато з народів, і з племен, і з язиків, і з поган будуть дивитися півчверта дні на їхні трупи, не дозволять покласти в гроби їхніх трупів.
\end{tcolorbox}
\begin{tcolorbox}
\textsubscript{10} А мешканці землі будуть тішитися та радіти над ними, і дарунки пошлють один одному, бо мучили ці два пророки мешканців землі.
\end{tcolorbox}
\begin{tcolorbox}
\textsubscript{11} А по півчверта днях дух життя ввійшов у них від Бога, і вони повставали на ноги свої. І напав жах великий на тих, хто дивився на них!
\end{tcolorbox}
\begin{tcolorbox}
\textsubscript{12} І почули вони гучний голос із неба, що їм говорив: Зійдіть сюди! І на небо зійшли вони в хмарі, і вороги їхні дивились на них.
\end{tcolorbox}
\begin{tcolorbox}
\textsubscript{13} І тієї години зчинився страшний землетрус, і десята частина міста того завалилась... І в цім трусі загинуло сім тисяч людських імен, а решта обгорнена жахом була, і вони віддали славу Богу Небесному!...
\end{tcolorbox}
\begin{tcolorbox}
\textsubscript{14} Друге горе минуло! Ото незабаром настане за ним третє горе!
\end{tcolorbox}
\begin{tcolorbox}
\textsubscript{15} І засурмив сьомий Ангол, і на небі зчинились гучні голоси, що казали: Перейшло панування над світом до Господа нашого та до Христа Його, і Він зацарює на вічні віки!
\end{tcolorbox}
\begin{tcolorbox}
\textsubscript{16} І двадцять чотири старці, що на престолах своїх перед Богом сидять, попадали на обличчя свої, та й уклонилися Богові,
\end{tcolorbox}
\begin{tcolorbox}
\textsubscript{17} кажучи: Дяку складаємо Тобі, Господи, Боже Вседержителю, що Ти є й що Ти був, що прийняв Свою силу велику та й зацарював!
\end{tcolorbox}
\begin{tcolorbox}
\textsubscript{18} А погани розлютилися, та гнів Твій прийшов, і час настав мертвих судити, і дати заплату рабам Твоїм, пророкам і святим, і тим, хто Ймення Твого боїться малим і великим, і знищити тих, хто нищить землю.
\end{tcolorbox}
\begin{tcolorbox}
\textsubscript{19} І розкрився храм Божий на небі, і ковчег заповіту Його в Його храмі з'явився. І зчинилися блискавки, і гуркіт, і громи, і землетрус, і великий град...
\end{tcolorbox}
\subsection{CHAPTER 12}
\begin{tcolorbox}
\textsubscript{1} І з'явилась на небі велика ознака: Жінка, зодягнена в сонце, а під ногами її місяць, а на її голові вінок із дванадцяти зір.
\end{tcolorbox}
\begin{tcolorbox}
\textsubscript{2} І вона мала в утробі, і кричала від болю, та муки терпіла від породу.
\end{tcolorbox}
\begin{tcolorbox}
\textsubscript{3} І з'явилася інша ознака на небі, ось змій червоноогняний, великий, що мав сім голів та десять рогів, а на його головах сім вінців.
\end{tcolorbox}
\begin{tcolorbox}
\textsubscript{4} Його хвіст змів третину зір із Неба та й кинув додолу. І змій стояв перед жінкою, що мала вродити, щоб з'їсти дитину її, коли вродить...
\end{tcolorbox}
\begin{tcolorbox}
\textsubscript{5} І дитину вродила вона чоловічої статі, що всі народи має пасти залізним жезлом. І дитина її була взята до Бога, і до престолу Його.
\end{tcolorbox}
\begin{tcolorbox}
\textsubscript{6} А жінка втекла на пустиню, де вона мала місце, від Бога для неї вготоване, щоб там годували її тисячу двісті шістдесят день.
\end{tcolorbox}
\begin{tcolorbox}
\textsubscript{7} І сталась на небі війна: Михаїл та його Анголи вчинили зо змієм війну. І змій воював та його анголи,
\end{tcolorbox}
\begin{tcolorbox}
\textsubscript{8} та не втрималися, і вже не знайшлося їм місця на небі.
\end{tcolorbox}
\begin{tcolorbox}
\textsubscript{9} І скинений був змій великий, вуж стародавній, що зветься диявол і сатана, що зводить усесвіт, і скинений був він додолу, а з ним і його анголи були скинені.
\end{tcolorbox}
\begin{tcolorbox}
\textsubscript{10} І я почув гучний голос на небі, який говорив: Тепер настало спасіння, і сила, і царство нашого Бога, і влада Христа Його, бо скинений той, хто братів наших скаржив, хто перед нашим Богом оскаржував їх день і ніч!
\end{tcolorbox}
\begin{tcolorbox}
\textsubscript{11} І вони його перемогли кров'ю Агнця та словом свого засвідчення, і не полюбили життя свого навіть до смерти!
\end{tcolorbox}
\begin{tcolorbox}
\textsubscript{12} Через це звеселися ти, небо, та ті, хто на нім пробуває! Горе землі та морю, до вас бо диявол зійшов, маючи лютість велику, знаючи, що короткий час має!
\end{tcolorbox}
\begin{tcolorbox}
\textsubscript{13} А коли змій побачив, що додолу він скинений, то став переслідувати жінку, що вродила хлоп'я.
\end{tcolorbox}
\begin{tcolorbox}
\textsubscript{14} І жінці дані були дві крилі великого орла, щоб від змія летіла в пустиню до місця свого, де будуть її годувати час, і часи, і півчасу.
\end{tcolorbox}
\begin{tcolorbox}
\textsubscript{15} І пустив змій за жінкою з уст своїх воду, як річку, щоб річка схопила її.
\end{tcolorbox}
\begin{tcolorbox}
\textsubscript{16} Та жінці земля помогла, і розкрила земля свої уста, та й випила річку, яку змій був пустив із своїх уст...
\end{tcolorbox}
\begin{tcolorbox}
\textsubscript{17} І змій розлютувався на жінку, і пішов воювати з останком насіння її, що вони бережуть Божі заповіді та мають свідоцтво Ісусове.
\end{tcolorbox}
\subsection{CHAPTER 13}
\begin{tcolorbox}
\textsubscript{1} (12-18) І я став на морському піску. (13-1) І я бачив звірину, що виходила з моря, яка мала десять рогів та сім голів, а на рогах її було десять вінців, а на її головах богозневажні імена.
\end{tcolorbox}
\begin{tcolorbox}
\textsubscript{2} А звірина, що я її бачив, подібна до рися була, а ноги її як ведмежі, а паща її немов лев'яча паща. І змій дав їй свою силу, і престола свого, і владу велику.
\end{tcolorbox}
\begin{tcolorbox}
\textsubscript{3} А одна з її голів була ніби забита на смерть, але рана смертельна її вздоровилась. І вся земля дивувалась, слідкуючи за звіриною!
\end{tcolorbox}
\begin{tcolorbox}
\textsubscript{4} І вклонилися змієві, що дав владу звірині. І вклонились звірині, говорячи: Хто до звірини подібний, і хто воювати з нею може?
\end{tcolorbox}
\begin{tcolorbox}
\textsubscript{5} І їй дано уста, що говорили зухвале та богозневажне. І їй дано владу діяти сорок два місяці.
\end{tcolorbox}
\begin{tcolorbox}
\textsubscript{6} І відкрила вона свої уста на зневагу проти Бога, щоб богозневажати Ім'я Його й оселю Його, та тих, хто на небі живе.
\end{tcolorbox}
\begin{tcolorbox}
\textsubscript{7} І їй дано провадити війну зо святими, та їх перемогти. І їй дана влада над кожним племенем, і народом, і язиком, і людом.
\end{tcolorbox}
\begin{tcolorbox}
\textsubscript{8} І їй вклоняться всі, хто живе на землі, що їхні імена не написані в книгах життя Агнця, заколеного від закладин світу.
\end{tcolorbox}
\begin{tcolorbox}
\textsubscript{9} Коли має хто вухо, нехай слухає:
\end{tcolorbox}
\begin{tcolorbox}
\textsubscript{10} Коли хто до полону веде, сам піде в полон. Коли хто мечем убиває, такий мусить сам бути вбитий мечем! Отут терпеливість та віра святих!
\end{tcolorbox}
\begin{tcolorbox}
\textsubscript{11} І бачив я іншу звірину, що виходила з землі. І вона мала два роги, подібні ягнячим, та говорила, як змій.
\end{tcolorbox}
\begin{tcolorbox}
\textsubscript{12} І вона виконувала всю владу першої звірини перед нею, і робила, щоб земля та ті, хто живе на ній, вклонилися першій звірині, що в неї вздоровлена була її рана смертельна.
\end{tcolorbox}
\begin{tcolorbox}
\textsubscript{13} І чинить вона великі ознаки, так що й огонь зводить з неба додолу перед людьми.
\end{tcolorbox}
\begin{tcolorbox}
\textsubscript{14} І зводить вона мешканців землі через ознаки, що їх дано їй чинити перед звіриною, намовляючи мешканців землі зробити образа звірини, що має рану від меча, та живе.
\end{tcolorbox}
\begin{tcolorbox}
\textsubscript{15} І дано їй вкласти духа образові звірини, щоб заговорив образ звірини, і зробити, щоб усі, хто не поклониться образові звірини, побиті були.
\end{tcolorbox}
\begin{tcolorbox}
\textsubscript{16} І зробить вона, щоб усім малим і великим, багатим і вбогим, вільним і рабам було дано знамено на їхню правицю або на їхні чола,
\end{tcolorbox}
\begin{tcolorbox}
\textsubscript{17} щоб ніхто не міг ані купити, ані продати, якщо він не має знамена ймення звірини, або числа ймення його...
\end{tcolorbox}
\begin{tcolorbox}
\textsubscript{18} Тут мудрість! Хто має розум, нехай порахує число звірини, бо воно число людське. А число її шістсот шістдесят шість.
\end{tcolorbox}
\subsection{CHAPTER 14}
\begin{tcolorbox}
\textsubscript{1} І я глянув, і ось Агнець стоїть на Сіонській горі, а з Ним сто сорок чотири тисячі, що мають Ім'я Його й Ім'я Отця Його, написане на своїх чолах.
\end{tcolorbox}
\begin{tcolorbox}
\textsubscript{2} І почув я голос із неба, немов шум великої води, і немов гук міцного грому. І почув я голос гуслярів, що грали на гуслах своїх,
\end{tcolorbox}
\begin{tcolorbox}
\textsubscript{3} і співали, як пісню нову перед престолом і перед чотирьома тваринами й старцями. І ніхто не міг навчитися пісні, окрім цих ста сорока чотирьох тисяч, викуплених від землі.
\end{tcolorbox}
\begin{tcolorbox}
\textsubscript{4} Це ті, хто не осквернився з жінками, бо чисті вони. Вони йдуть за Агнцем, куди Він іде. Вони викуплені від людей, первістки Богові й Агнцеві,
\end{tcolorbox}
\begin{tcolorbox}
\textsubscript{5} не знайшлося бо підступу в їхніх устах, бо вони непорочні!
\end{tcolorbox}
\begin{tcolorbox}
\textsubscript{6} І побачив я іншого Ангола, що летів серед неба, і мав благовістити вічну Євангелію мешканцям землі, і кожному людові, і племені, і язику, і народові.
\end{tcolorbox}
\begin{tcolorbox}
\textsubscript{7} І він говорив гучним голосом: Побійтеся Бога та славу віддайте Йому, бо настала година суду Його, і вклоніться Тому, Хто створив небо, і землю, і море, і водні джерела!
\end{tcolorbox}
\begin{tcolorbox}
\textsubscript{8} А інший, другий Ангол летів слідом і казав: Упав, упав Вавилон, город великий, бо лютим вином розпусти своєї він напоїв усі народи!
\end{tcolorbox}
\begin{tcolorbox}
\textsubscript{9} А інший, третій Ангол летів услід за ним, гучним голосом кажучи: Коли хто вклоняється звірині та образу її, і приймає знамено на чолі своїм чи на руці своїй,
\end{tcolorbox}
\begin{tcolorbox}
\textsubscript{10} то той питиме з вина Божого гніву, вина незмішаного в чаші гніву Його, і буде мучений в огні й сірці перед Анголами святими та перед Агнцем.
\end{tcolorbox}
\begin{tcolorbox}
\textsubscript{11} А дим їхніх мук підійматиметься вічні віки. І не мають спокою день і ніч усі ті, хто вклоняється звірині та образу її, і приймає знамено ймення його.
\end{tcolorbox}
\begin{tcolorbox}
\textsubscript{12} Тут терпеливість святих, що додержують заповіді Божі та Ісусову віру!
\end{tcolorbox}
\begin{tcolorbox}
\textsubscript{13} І почув я голос із неба, що до мене казав: Напиши: Блаженні ті мертві, хто з цього часу вмирає в Господі! Так, каже Дух, вони від праць своїх заспокояться, бо їхні діла йдуть за ними слідом.
\end{tcolorbox}
\begin{tcolorbox}
\textsubscript{14} І я глянув, і ото біла хмара, а на хмарі сидить подібний до Людського Сина. Він мав на своїй голові золотого вінця, а в руці його гострий серп.
\end{tcolorbox}
\begin{tcolorbox}
\textsubscript{15} І інший Ангол вийшов із храму, і гучним голосом кликнув до того, хто на хмарі сидів: Пошли серпа свого й жни, бо настала година пожати, дозріло бо жниво землі!
\end{tcolorbox}
\begin{tcolorbox}
\textsubscript{16} І той, хто на хмарі сидів, скинув додолу серпа свого, і земля була вижата.
\end{tcolorbox}
\begin{tcolorbox}
\textsubscript{17} І інший Ангол вийшов із храму, що на небі, і він мав гострого серпа.
\end{tcolorbox}
\begin{tcolorbox}
\textsubscript{18} І інший Ангол, що мав владу над огнем, вийшов від жертівника. І він гучним голосом кликнув до того, що мав гострого серпа, говорячи: Пошли свого гострого серпа, і позбирай грона земної виноградини, бо грона її вже доспіли.
\end{tcolorbox}
\begin{tcolorbox}
\textsubscript{19} І Ангол кинув додолу серпа свого, і зібрав виноград на землі, і вкинув в велике чавило Божого гніву.
\end{tcolorbox}
\begin{tcolorbox}
\textsubscript{20} І потовчене було чавило за містом, і потекла кров із чавила аж до кінських вуздечок, на тисячу шістсот стадій...
\end{tcolorbox}
\subsection{CHAPTER 15}
\begin{tcolorbox}
\textsubscript{1} І бачив я інше знамено на небі, велике та дивне, сім Анголів, що сім кар вони мали, бо ними кінчався гнів Божий.
\end{tcolorbox}
\begin{tcolorbox}
\textsubscript{2} І я бачив щось, ніби як море скляне, з огнем перемішане. А ті, що перемогли звірину та образа його, і знамено його, і число його ймення, стояли на морі склянім, та мали гусла Божі.
\end{tcolorbox}
\begin{tcolorbox}
\textsubscript{3} І співали вони пісню Мойсея, раба Божого, і пісню Агнця, говорячи: Великі та дивні діла Твої, о Господи, Боже Вседержителю! Справедливі й правдиві дороги Твої, о Царю святих!
\end{tcolorbox}
\begin{tcolorbox}
\textsubscript{4} Хто Тебе, Господи, не побоїться, та Ймення Твого не прославить? Бо один Ти святий, бо народи всі прийдуть та вклоняться перед Тобою, бо з'явилися суди Твої!
\end{tcolorbox}
\begin{tcolorbox}
\textsubscript{5} А по цьому я глянув, і ось відчинився храм скинії свідчення в небі,
\end{tcolorbox}
\begin{tcolorbox}
\textsubscript{6} і сім Анголів вийшли з храму, і сім кар вони мали. Вони були вдягнені в шати льняні, чисті й ясні, і підперезані довкола грудей золотими поясами.
\end{tcolorbox}
\begin{tcolorbox}
\textsubscript{7} І одна з чотирьох тих тварин дала сімом Анголам сім чаш золотих, наповнених гніву Бога, що живе повік віку.
\end{tcolorbox}
\begin{tcolorbox}
\textsubscript{8} І храм переповнився димом від Божої слави, і від сили Його. Та до храму ніхто не спромігся ввійти, аж поки не скінчилися ті сім кар сімох Анголів.
\end{tcolorbox}
\subsection{CHAPTER 16}
\begin{tcolorbox}
\textsubscript{1} І я почув гучний голос із храму, що казав до семи Анголів: Ідіть, і вилийте на землю сім чаш гніву Божого!
\end{tcolorbox}
\begin{tcolorbox}
\textsubscript{2} І пішов перший Ангол, і вилив на землю чашу свою. І шкідливі та люті болячки обсіли людей, хто мав знамено звірини й вклонявсь її образу.
\end{tcolorbox}
\begin{tcolorbox}
\textsubscript{3} А другий Ангол вилив свою чашу до моря. І сталася кров, немов у мерця, і кожна істота жива вмерла в морі.
\end{tcolorbox}
\begin{tcolorbox}
\textsubscript{4} Третій же Ангол вилив чашу свою на річки та на водні джерела, і сталася кров.
\end{tcolorbox}
\begin{tcolorbox}
\textsubscript{5} І почув я Ангола вод, який говорив: Ти праведний, що Ти є й що Ти був, і святий, що Ти це присудив!
\end{tcolorbox}
\begin{tcolorbox}
\textsubscript{6} Бо вони пролили кров святих та пророків, і Ти дав їм напитися крови. Вони варті того!
\end{tcolorbox}
\begin{tcolorbox}
\textsubscript{7} І я чув, як жертівник говорив: Так, Господи, Боже Вседержителю! Правдиві й справедливі суди Твої!
\end{tcolorbox}
\begin{tcolorbox}
\textsubscript{8} А Ангол четвертий вилив свою чашу на сонце. І дано йому палити людей огнем.
\end{tcolorbox}
\begin{tcolorbox}
\textsubscript{9} І спека велика палила людей, і зневажали вони Ім'я Бога, що має владу над карами тими, і вони не покаялися, щоб славу віддати Йому.
\end{tcolorbox}
\begin{tcolorbox}
\textsubscript{10} А п'ятий Ангол вилив чашу свою на престола звірини. І затьмилося царство її, і люди від болю кусали свої язики,
\end{tcolorbox}
\begin{tcolorbox}
\textsubscript{11} і Бога Небесного вони зневажали від болю свого й від своїх болячок, та в учинках своїх не покаялись!
\end{tcolorbox}
\begin{tcolorbox}
\textsubscript{12} Шостий же Ангол вилив чашу свою на річку велику Ефрат, і вода її висохла, щоб приготовити дорогу царям, які від схід сонця.
\end{tcolorbox}
\begin{tcolorbox}
\textsubscript{13} І я бачив, що виходили з уст змія, і з уст звірини, і з уст неправдивого пророка три духи нечисті, як жаби,
\end{tcolorbox}
\begin{tcolorbox}
\textsubscript{14} це духи демонські, що чинять ознаки. Вони виходять до царів усього всесвіту, щоб зібрати їх на війну того великого дня Вседержителя Бога.
\end{tcolorbox}
\begin{tcolorbox}
\textsubscript{15} Ось іду, немов злодій! Блаженний, хто чуйний, і одежу свою береже, щоб нагим не ходити, і щоб не бачили ганьби його!
\end{tcolorbox}
\begin{tcolorbox}
\textsubscript{16} І зібрав їх на місце, яке по-єврейському зветься Армагеддон.
\end{tcolorbox}
\begin{tcolorbox}
\textsubscript{17} Сьомий же Ангол вилив чашу свою на повітря. І голос гучний залунав від небесного храму з престолу, говорячи: Сталося!
\end{tcolorbox}
\begin{tcolorbox}
\textsubscript{18} І сталися блискавки й гуркіт та громи, і сталось велике трясіння землі, якого не було, відколи людина живе на землі... Великий такий землетрус, такий міцний!
\end{tcolorbox}
\begin{tcolorbox}
\textsubscript{19} І місто велике розпалося на три частині, і попадали людські міста... І великий Вавилон був згаданий перед Богом, щоб дати йому чашу вина Його лютого гніву...
\end{tcolorbox}
\begin{tcolorbox}
\textsubscript{20} І зник кожен острів, і не знайдено гір!...
\end{tcolorbox}
\begin{tcolorbox}
\textsubscript{21} І великий град, як важкі тягарі, падав із неба на людей. І люди зневажали Бога за покарання градом, бо кара Його була дуже велика!...
\end{tcolorbox}
\subsection{CHAPTER 17}
\begin{tcolorbox}
\textsubscript{1} І прийшов один із семи Анголів, що мають сім чаш, і говорив зо мною, кажучи: Підійди, я покажу тобі засудження великої розпусниці, що сидить над багатьма водами.
\end{tcolorbox}
\begin{tcolorbox}
\textsubscript{2} З нею розпусту чинили земні царі, і вином розпусти її впивались мешканці землі.
\end{tcolorbox}
\begin{tcolorbox}
\textsubscript{3} І в дусі повів він мене на пустиню. І побачив я жінку, що сиділа на червоній звірині, переповненій іменами богозневажними, яка мала сім голів і десять рогів.
\end{tcolorbox}
\begin{tcolorbox}
\textsubscript{4} А жінка була одягнена в порфіру й кармазин, і приоздоблена золотом і дорогоцінним камінням та перлами. У руці своїй мала вона золоту чашу, повну гидоти та нечести розпусти її.
\end{tcolorbox}
\begin{tcolorbox}
\textsubscript{5} А на чолі її було написане ім'я, таємниця: Великий Вавилон, мати розпусти й гидоти землі.
\end{tcolorbox}
\begin{tcolorbox}
\textsubscript{6} І бачив я жінку, п'яну від крови святих і від крови мучеників Ісусових, і, бачивши її, дивувався я дивом великим.
\end{tcolorbox}
\begin{tcolorbox}
\textsubscript{7} А Ангол промовив до мене: Чого ти дивуєшся? Я скажу тобі таємницю жінки й звірини, яка носить її, яка має сім голів і десять рогів.
\end{tcolorbox}
\begin{tcolorbox}
\textsubscript{8} Звірина, яку бачив я, була і нема, і має вийти з безодні і піде вона на погибіль. А мешканці землі, що їхні імена не записані в книгу життя від закладин світу, дивуватися будуть, як побачать, що звірина була і нема, і з'явиться.
\end{tcolorbox}
\begin{tcolorbox}
\textsubscript{9} Тут розум, що має він мудрість. Сім голів це сім гір, що на них сидить жінка. І сім царів,
\end{tcolorbox}
\begin{tcolorbox}
\textsubscript{10} п'ять їх упало, один є, другий іще не прийшов, а як прийде, то мусить він трохи пробути.
\end{tcolorbox}
\begin{tcolorbox}
\textsubscript{11} І звірина, що була і нема, і вона сама восьма й з сімох, і йде на погибіль.
\end{tcolorbox}
\begin{tcolorbox}
\textsubscript{12} А десять тих рогів, що бачив ти їх, то десять царів, що ще не прийняли царства, але приймуть владу царську із звіриною на одну годину.
\end{tcolorbox}
\begin{tcolorbox}
\textsubscript{13} Вони мають одну думку, а силу та владу свою віддадуть звірині.
\end{tcolorbox}
\begin{tcolorbox}
\textsubscript{14} Вони воюватимуть проти Агнця та Агнець переможе їх, бо Він Господь над панами та Цар над царями. А ті, хто з Ним, покликані, і вибрані, і вірні.
\end{tcolorbox}
\begin{tcolorbox}
\textsubscript{15} І говорить до мене: Води, що бачив ти їх, де сидить та розпусниця, то народи та люди, і племена та язики.
\end{tcolorbox}
\begin{tcolorbox}
\textsubscript{16} А десять рогів, що ти бачив їх, та звірина, вони зненавидять розпусницю, спустошать її й обнажать, і з'їдять її тіло, і огнем її спалять.
\end{tcolorbox}
\begin{tcolorbox}
\textsubscript{17} Бо Бог дав їм до серця, щоб волю чинили Його, маючи одну думку, і щоб царство своє віддали звірині, аж поки не виповняться слова Божі.
\end{tcolorbox}
\begin{tcolorbox}
\textsubscript{18} А жінка, яку ти бачив, то місто велике, що панує над царями земними.
\end{tcolorbox}
\subsection{CHAPTER 18}
\begin{tcolorbox}
\textsubscript{1} Після цього побачив я іншого Ангола, що сходив із неба, і що владу велику він мав. І земля освітилась від слави його.
\end{tcolorbox}
\begin{tcolorbox}
\textsubscript{2} І він гучним голосом кликнув, говорячи: Упав, упав великий Вавилон! Став він оселею демонів, і сховищем усякому духові нечистому, і сховищем усіх птахів нечистих та ненавидних,
\end{tcolorbox}
\begin{tcolorbox}
\textsubscript{3} бо лютим вином розпусти своєї він напоїв всі народи! І земні царі з ним розпусту чинили, а земні купці збагатіли від сили розкоші його!
\end{tcolorbox}
\begin{tcolorbox}
\textsubscript{4} І почув я інший голос із неба, який говорив: Вийдіть із нього, люди мої, щоб не сталися ви спільниками гріхів його, і щоб не потрапили в карання його.
\end{tcolorbox}
\begin{tcolorbox}
\textsubscript{5} Гріхи бо його досягли аж до неба, і Бог ізгадав про неправди його.
\end{tcolorbox}
\begin{tcolorbox}
\textsubscript{6} Відплатіть ви йому, як і він вам платив, і вдвоє подвойте йому за вчинки його! Удвоє налийте до чаші, що нею він вам наливав!
\end{tcolorbox}
\begin{tcolorbox}
\textsubscript{7} Скільки він славив себе та розкошував, стільки муки та смутку завдайте йому! Бо в серці своєму говорить: Сиджу, як цариця, і я не вдова, і бачити смутку не буду!
\end{tcolorbox}
\begin{tcolorbox}
\textsubscript{8} Через це одного дня прийдуть кари його, смерть, і плач, і голод, і спалений буде огнем, бо міцний Господь, Бог, що судить його!
\end{tcolorbox}
\begin{tcolorbox}
\textsubscript{9} І будуть плакати та голосити за ним царі земні, що з ним розпусту чинили та розкошували, коли побачать дим пожежі його.
\end{tcolorbox}
\begin{tcolorbox}
\textsubscript{10} Вони через страх його мук стоятимуть здалека та говоритимуть: Горе, горе, о місто велике, Вавилоне, місто могутнє, бо суд твій прийшов однієї години!
\end{tcolorbox}
\begin{tcolorbox}
\textsubscript{11} І земні купці будуть плакати та голосити за ним, бо ніхто не купує вже їхнього вантажу,
\end{tcolorbox}
\begin{tcolorbox}
\textsubscript{12} вантажу золота, і срібла, і каміння дорогоцінного, і перел, і віссону, і порфіри, і шовку, і кармазину, і всякого дерева запашного, і всякого посуду з слонової кости, і всякого посуду з дорогоцінного дерева, і мідяного, і залізного, і мармурового,
\end{tcolorbox}
\begin{tcolorbox}
\textsubscript{13} і кориці, і шафрану, і пахощів, і мирри, і ливану, і вина, і оливи, і тонкої муки, і пшениці, і товару, і вівців, і коней, і возів, і рабів, і душ людських.
\end{tcolorbox}
\begin{tcolorbox}
\textsubscript{14} І плоди пожадливости душі твоєї відійшли від тебе, і все сите та світле пропало для тебе, і вже їх ти не знайдеш!
\end{tcolorbox}
\begin{tcolorbox}
\textsubscript{15} Купці цими речами, що вони збагатилися з нього, від страху мук його стануть здалека, і будуть плакати та голосити,
\end{tcolorbox}
\begin{tcolorbox}
\textsubscript{16} і казати: Горе, горе, місто велике, зодягнене в віссон і порфіру та в кармазин, і прикрашене золотом і дорогоцінним камінням та перлами,
\end{tcolorbox}
\begin{tcolorbox}
\textsubscript{17} бо за одну годину згинуло таке велике багатство... І кожен стерник, і кожен, хто пливає на кораблях, і моряки, і всі, хто працює на морі, стали здалека,
\end{tcolorbox}
\begin{tcolorbox}
\textsubscript{18} і, бачивши дим від пожежі його, кричали й казали: Котре до великого міста подібне?
\end{tcolorbox}
\begin{tcolorbox}
\textsubscript{19} І вони посипали порохом голови свої, і закричали, плачучи та голосячи, і кажучи: Горе, горе, місто велике, що в ньому з його дорогоцінностей збагатилися всі, хто має кораблі на морі, бо за одну годину воно спорожніло!
\end{tcolorbox}
\begin{tcolorbox}
\textsubscript{20} Радій з цього, небо, і святі апостоли та пророки, бо Бог виконав суд ваш над ним!
\end{tcolorbox}
\begin{tcolorbox}
\textsubscript{21} І один сильний Ангол узяв великого каменя, як жорно, і кинув до моря, говорячи: З таким розгоном буде кинений Вавилон, місто велике, і вже він не знайдеться!
\end{tcolorbox}
\begin{tcolorbox}
\textsubscript{22} І голос гуслярів, і співаків, і сопільників, і сурмачів уже не буде чутий в тобі! І вже не знайдеться в тобі жадного мистця й ніякого мистецтва, і шум жорен уже не буде чутий в тобі!
\end{tcolorbox}
\begin{tcolorbox}
\textsubscript{23} І світло свічника вже не буде світити в тобі, і голос молодого й молодої вже не буде чутий в тобі. Бо купці твої були земні вельможі, бо твоїм ворожбитством були зведені всі народи!
\end{tcolorbox}
\begin{tcolorbox}
\textsubscript{24} Бо в нім знайдена кров пророків, і святих, і побитих усіх на землі...
\end{tcolorbox}
\subsection{CHAPTER 19}
\begin{tcolorbox}
\textsubscript{1} По цьому почув я наче гучний голос великого натовпу в небі, який говорив: Алілуя! Спасіння, і слава, і сила Господеві нашому,
\end{tcolorbox}
\begin{tcolorbox}
\textsubscript{2} правдиві бо та справедливі суди Його, бо Він засудив ту велику розпусницю, що землю зіпсула своєю розпустою, і помстив за кров Своїх рабів з її рук!
\end{tcolorbox}
\begin{tcolorbox}
\textsubscript{3} І вдруге сказали вони: Алілуя! І з неї дим виступає на вічні віки!
\end{tcolorbox}
\begin{tcolorbox}
\textsubscript{4} І попадали двадцять чотири старці й чотири тварині, і поклонилися Богові, що сидить на престолі, говорячи: Амінь, алілуя!
\end{tcolorbox}
\begin{tcolorbox}
\textsubscript{5} А від престолу вийшов голос, що кликав: Хваліть Бога нашого, усі раби Його, і всі, хто боїться Його, і малі, і великі!
\end{tcolorbox}
\begin{tcolorbox}
\textsubscript{6} І почув я ніби голос великого натовпу, і наче шум великої води, і мов голос громів гучних, що вигукували: Алілуя, бо запанував Господь, наш Бог Вседержитель!
\end{tcolorbox}
\begin{tcolorbox}
\textsubscript{7} Радіймо та тішмося, і даймо славу Йому, бо весілля Агнця настало, і жона Його себе приготувала!
\end{tcolorbox}
\begin{tcolorbox}
\textsubscript{8} І їй дано було зодягнутися в чистий та світлий вісон, бо віссон то праведність святих.
\end{tcolorbox}
\begin{tcolorbox}
\textsubscript{9} І сказав він мені: Напиши: Блаженні покликані на весільну вечерю Агнця! І сказав він мені: Це правдиві Божі слова!
\end{tcolorbox}
\begin{tcolorbox}
\textsubscript{10} І я впав до його ніг, щоб вклонитись йому. І він каже мені: Таж ні! Я співслуга твій та братів твоїх, хто має засвідчення Ісусове, Богові вклонися! Бо засвідчення Ісусове, то дух пророцтва.
\end{tcolorbox}
\begin{tcolorbox}
\textsubscript{11} І побачив я небо відкрите. І ось білий кінь, а Той, Хто на ньому сидів, зветься Вірний і Правдивий, і Він справедливо судить і воює.
\end{tcolorbox}
\begin{tcolorbox}
\textsubscript{12} Очі Його немов полум'я огняне, а на голові Його багато вінців. Він ім'я мав написане, якого не знає ніхто, тільки Він Сам.
\end{tcolorbox}
\begin{tcolorbox}
\textsubscript{13} І зодягнений був Він у шату, покрашену кров'ю. А Йому на ім'я: Слово Боже.
\end{tcolorbox}
\begin{tcolorbox}
\textsubscript{14} А війська небесні, зодягнені в білий та чистий віссон, їхали вслід за Ним на білих конях.
\end{tcolorbox}
\begin{tcolorbox}
\textsubscript{15} А з Його уст виходив гострий меч, щоб ним бити народи. І Він пастиме їх залізним жезлом, і Він буде топтати чавило вина лютого гніву Бога Вседержителя!
\end{tcolorbox}
\begin{tcolorbox}
\textsubscript{16} І Він має на шаті й на стегнах Своїх написане ймення: Цар над царями, і Господь над панами.
\end{tcolorbox}
\begin{tcolorbox}
\textsubscript{17} І бачив я одного Ангола, що на сонці стояв. І він гучним голосом кликнув, кажучи до всіх птахів, що серед неба літали: Ходіть, і зберіться на велику Божу вечерю,
\end{tcolorbox}
\begin{tcolorbox}
\textsubscript{18} щоб ви їли тіла царів, і тіла тисячників, і тіла сильних, і тіла коней і тих, хто сидить на них, і тіла всіх вільних і рабів, і малих, і великих...
\end{tcolorbox}
\begin{tcolorbox}
\textsubscript{19} І я побачив звірину, і земних царів, і війська їхні, зібрані, щоб учинити війну з Тим, Хто сидить на коні, та з військом Його.
\end{tcolorbox}
\begin{tcolorbox}
\textsubscript{20} І схоплена була звірина, а з нею неправдивий пророк, що ознаки чинив перед нею, що ними звів тих, хто знамено звірини прийняв і поклонився був образові її. Обоє вони були вкинені живими до огняного озера, що сіркою горіло...
\end{tcolorbox}
\begin{tcolorbox}
\textsubscript{21} А решта побита була мечем Того, Хто сидів на коні, що виходив із уст Його. І все птаство наїлося їхніми трупами...
\end{tcolorbox}
\subsection{CHAPTER 20}
\begin{tcolorbox}
\textsubscript{1} І бачив я Ангола, що сходив із неба, що мав ключа від безодні, і кайдани великі в руці своїй.
\end{tcolorbox}
\begin{tcolorbox}
\textsubscript{2} І схопив він змія, вужа стародавнього, що диявол він і сатана, і зв'язав його на тисячу років,
\end{tcolorbox}
\begin{tcolorbox}
\textsubscript{3} та й кинув його до безодні, і замкнув його, і печатку над ним поклав, щоб народи не зводив уже, аж поки не скінчиться тисяча років. А по цьому він розв'язаний буде на короткий час.
\end{tcolorbox}
\begin{tcolorbox}
\textsubscript{4} І бачив я престоли та тих, хто сидів на них, і суд їм був даний, і душі стятих за свідчення про Ісуса й за Слово Боже, які не вклонились звірині, ані образові її, і не прийняли знамена на чола свої та на руку свою. І вони ожили, і царювали з Христом тисячу років.
\end{tcolorbox}
\begin{tcolorbox}
\textsubscript{5} А інші померлі не ожили, аж поки не скінчиться тисяча років. Це перше воскресіння.
\end{tcolorbox}
\begin{tcolorbox}
\textsubscript{6} Блаженний і святий, хто має частку в першому воскресінні! Над ними друга смерть не матиме влади, але вони будуть священиками Бога й Христа, і царюватимуть з Ним тисячу років.
\end{tcolorbox}
\begin{tcolorbox}
\textsubscript{7} Коли ж скінчиться тисяча років, сатана буде випущений із в'язниці своєї.
\end{tcolorbox}
\begin{tcolorbox}
\textsubscript{8} І вийде він зводити народи, що вони на чотирьох краях землі, Ґоґа й Маґоґа, щоб зібрати їх до бою, а число їхнє як морський пісок.
\end{tcolorbox}
\begin{tcolorbox}
\textsubscript{9} І вийшли вони на ширину землі, і оточили табір святих та улюблене місто. І зійшов огонь з неба, і пожер їх.
\end{tcolorbox}
\begin{tcolorbox}
\textsubscript{10} А диявол, що зводив їх, був укинений в озеро огняне та сірчане, де звірина й пророк неправдивий. І мучені будуть вони день і ніч на вічні віки.
\end{tcolorbox}
\begin{tcolorbox}
\textsubscript{11} І я бачив престола великого білого, і Того, Хто на ньому сидів, що від лиця Його втекла земля й небо, і місця для них не знайшлося.
\end{tcolorbox}
\begin{tcolorbox}
\textsubscript{12} І бачив я мертвих малих і великих, що стояли перед Богом. І розгорнулися книги, і розгорнулась інша книга, то книга життя. І суджено мертвих, як написано в книгах, за вчинками їхніми.
\end{tcolorbox}
\begin{tcolorbox}
\textsubscript{13} І дало море мертвих, що в ньому, і смерть і ад дали мертвих, що в них, і суджено їх згідно з їхніми вчинками.
\end{tcolorbox}
\begin{tcolorbox}
\textsubscript{14} Смерть же та ад були вкинені в озеро огняне. Це друга смерть, озеро огняне.
\end{tcolorbox}
\begin{tcolorbox}
\textsubscript{15} А хто не знайшовся написаний в книзі життя, той укинений буде в озеро огняне...
\end{tcolorbox}
\subsection{CHAPTER 21}
\begin{tcolorbox}
\textsubscript{1} І бачив я небо нове й нову землю, перше бо небо та перша земля проминули, і моря вже не було.
\end{tcolorbox}
\begin{tcolorbox}
\textsubscript{2} І я, Іван, бачив місто святе, Новий Єрусалим, що сходив із неба від Бога, що був приготований, як невіста, прикрашена для чоловіка свого.
\end{tcolorbox}
\begin{tcolorbox}
\textsubscript{3} І почув я гучний голос із престолу, який кликав: Оце оселя Бога з людьми, і Він житиме з ними! Вони будуть народом Його, і Сам Бог буде з ними,
\end{tcolorbox}
\begin{tcolorbox}
\textsubscript{4} і Бог кожну сльозу з очей їхніх зітре, і не буде вже смерти. Ані смутку, ані крику, ані болю вже не буде, бо перше минулося!
\end{tcolorbox}
\begin{tcolorbox}
\textsubscript{5} І сказав Той, Хто сидить на престолі: Ось нове все творю! І говорить: Напиши, що слова ці правдиві та вірні!
\end{tcolorbox}
\begin{tcolorbox}
\textsubscript{6} І сказав Він мені: Сталося! Я Альфа й Омега, Початок і Кінець. Хто прагне, тому дармо Я дам від джерела живої води.
\end{tcolorbox}
\begin{tcolorbox}
\textsubscript{7} Переможець наслідить усе, і Я буду Богом для нього, а він Мені буде за сина!
\end{tcolorbox}
\begin{tcolorbox}
\textsubscript{8} А лякливим, і невірним, і мерзким, і душогубам, і розпусникам, і чарівникам, і ідолянам, і всім неправдомовцям, їхня частина в озері, що горить огнем та сіркою, а це друга смерть!
\end{tcolorbox}
\begin{tcolorbox}
\textsubscript{9} І прийшов до мене один із семи Анголів, що мають сім чаш, наповнених сімома останніми карами, та й промовив до мене, говорячи: Ходи, покажу я тобі невісту, жону Агнця.
\end{tcolorbox}
\begin{tcolorbox}
\textsubscript{10} І заніс мене духом на гору велику й високу, і місто велике мені показав, святий Єрусалим, що сходив із неба від Бога.
\end{tcolorbox}
\begin{tcolorbox}
\textsubscript{11} Славу Божу він має. А світлість його подібна до каменя дорогоцінного, як каменя ясписа, що блищить, як кришталь.
\end{tcolorbox}
\begin{tcolorbox}
\textsubscript{12} Мур воно мало великий і високий, мало дванадцять брам, а на брамах дванадцять Анголів та ймення написані, а вони імення дванадцятьох племен синів Ізраїля.
\end{tcolorbox}
\begin{tcolorbox}
\textsubscript{13} Від сходу три брамі, і від півночі три брамі, і від півдня три брамі, і від заходу три брамі.
\end{tcolorbox}
\begin{tcolorbox}
\textsubscript{14} І міський мур мав дванадцять підвалин, а на них дванадцять імен дванадцяти апостолів Агнця.
\end{tcolorbox}
\begin{tcolorbox}
\textsubscript{15} А той, хто зо мною говорив, мав міру, золоту тростину, щоб зміряти місто, і брами його і його мур.
\end{tcolorbox}
\begin{tcolorbox}
\textsubscript{16} А місто чотирикутнє, а довжина його така, як і ширина. І він зміряв місто тростиною на дванадцять тисяч стадій; довжина, і ширина, і вишина його рівні.
\end{tcolorbox}
\begin{tcolorbox}
\textsubscript{17} І зміряв він мура його на сто сорок чотири лікті міри людської, яка й міра Ангола.
\end{tcolorbox}
\begin{tcolorbox}
\textsubscript{18} Його мур був збудований з яспису, а місто було щире золото, подібне до чистого скла.
\end{tcolorbox}
\begin{tcolorbox}
\textsubscript{19} Підвалини муру міського прикрашені були всяким дорогоцінним камінням. Перша підвалина яспис, друга сапфір, третя халкидон, четверта смарагд,
\end{tcolorbox}
\begin{tcolorbox}
\textsubscript{20} п'ята сардонікс, шоста сардій, сьома хризоліт, восьма берил, дев'ята топаз, десята хрисопрас, одинадцята якинт, дванадцята аметист.
\end{tcolorbox}
\begin{tcolorbox}
\textsubscript{21} А дванадцять брам то дванадцять перлин, і кожна брама зокрема була з однієї перлини. А вулиці міста щире золото, прозорі, як скло.
\end{tcolorbox}
\begin{tcolorbox}
\textsubscript{22} А храму не бачив я в ньому, бо Господь, Бог Вседержитель то йому храм і Агнець.
\end{tcolorbox}
\begin{tcolorbox}
\textsubscript{23} І місто не має потреби ні в сонці, ні в місяці, щоб у ньому світили, слава бо Божа його освітила, а світильник для нього Агнець.
\end{tcolorbox}
\begin{tcolorbox}
\textsubscript{24} І народи ходитимуть у світлі його, а земські царі принесуть свою славу до нього.
\end{tcolorbox}
\begin{tcolorbox}
\textsubscript{25} А брами його зачинятись не будуть удень, бо там ночі не буде.
\end{tcolorbox}
\begin{tcolorbox}
\textsubscript{26} І принесуть до нього славу й честь народів.
\end{tcolorbox}
\begin{tcolorbox}
\textsubscript{27} І не ввійде до нього ніщо нечисте, ані той, хто чинить гидоту й неправду, але тільки ті, хто записаний у книзі життя Агнця.
\end{tcolorbox}
\subsection{CHAPTER 22}
\begin{tcolorbox}
\textsubscript{1} І показав він мені чисту ріку живої води, ясну, мов кришталь, що випливала з престолу Бога й Агнця.
\end{tcolorbox}
\begin{tcolorbox}
\textsubscript{2} Посеред його вулиці, і по цей бік і по той бік ріки дерево життя, що родить дванадцять раз плоди, кожного місяця приносячи плід свій. А листя дерев на вздоровлення народів.
\end{tcolorbox}
\begin{tcolorbox}
\textsubscript{3} І жадного прокляття більше не буде. І буде в ньому Престол Бога та Агнця, а раби Його будуть служити Йому,
\end{tcolorbox}
\begin{tcolorbox}
\textsubscript{4} і побачать лице Його, а Ймення Його на їхніх чолах.
\end{tcolorbox}
\begin{tcolorbox}
\textsubscript{5} А ночі вже більше не буде, і не буде потреби в світлі світильника, ані в світлі сонця, бо освітлює їх Господь, Бог, а вони царюватимуть вічні віки.
\end{tcolorbox}
\begin{tcolorbox}
\textsubscript{6} І сказав він до мене: Це вірні й правдиві слова, а Господь, Бог духів пророчих, послав Свого Ангола, щоб він показав своїм рабам, що незабаром статися мусить.
\end{tcolorbox}
\begin{tcolorbox}
\textsubscript{7} Ото, незабаром приходжу. Блаженний, хто зберігає пророчі слова цієї книги!
\end{tcolorbox}
\begin{tcolorbox}
\textsubscript{8} І я, Іван, чув і бачив оце. А коли я почув та побачив, я впав до ніг Ангола, що мені це показував, щоб вклонитись йому.
\end{tcolorbox}
\begin{tcolorbox}
\textsubscript{9} І сказав він до мене: Таж ні! Бо я співслуга твій і братів твоїх пророків, і тих, хто зберігає слова цієї книги. Богові вклонися!
\end{tcolorbox}
\begin{tcolorbox}
\textsubscript{10} І сказав він до мене: Не запечатуй слів пророцтва цієї книги. Час бо близький!
\end{tcolorbox}
\begin{tcolorbox}
\textsubscript{11} Неправедний нехай чинить неправду ще, і поганий нехай ще опоганюється. А праведний нехай ще чинить правду, а святий нехай ще освячується!
\end{tcolorbox}
\begin{tcolorbox}
\textsubscript{12} Ото, незабаром приходжу, і зо Мною заплата Моя, щоб кожному віддати згідно з ділами його.
\end{tcolorbox}
\begin{tcolorbox}
\textsubscript{13} Я Альфа й Омега, Перший і Останній, Початок і Кінець.
\end{tcolorbox}
\begin{tcolorbox}
\textsubscript{14} Блаженні, хто випере шати свої, щоб мати право на дерево життя, і ввійти брамами в місто!
\end{tcolorbox}
\begin{tcolorbox}
\textsubscript{15} А поза ним будуть пси, і чарівники, і розпусники, і душогуби, і ідоляни, і кожен, хто любить та чинить неправду.
\end{tcolorbox}
\begin{tcolorbox}
\textsubscript{16} Я, Ісус, послав Свого Ангола, щоб засвідчити вам це у Церквах. Я корінь і рід Давидів, зоря ясна і досвітня!
\end{tcolorbox}
\begin{tcolorbox}
\textsubscript{17} А Дух і невіста говорять: Прийди! А хто чує, хай каже: Прийди! І хто прагне, хай прийде, і хто хоче, хай воду життя бере дармо!
\end{tcolorbox}
\begin{tcolorbox}
\textsubscript{18} Свідкую я кожному, хто чує слова пророцтва цієї книги: Коли хто до цього додасть що, то накладе на нього Бог кари, що написані в книзі оцій.
\end{tcolorbox}
\begin{tcolorbox}
\textsubscript{19} А коли хто що відійме від слів книги пророцтва цього, то відійме Бог частку його від дерева життя, і від міста святого, що написане в книзі оцій.
\end{tcolorbox}
\begin{tcolorbox}
\textsubscript{20} Той, Хто свідкує, говорить оце: Так, незабаром прийду! Амінь. Прийди, Господи Ісусе!
\end{tcolorbox}
\begin{tcolorbox}
\textsubscript{21} Благодать Господа нашого Ісуса Христа зо всіма вами! Амінь.
\end{tcolorbox}
\section{BOOK 1}
\section{BOOK 1}
\section{BOOK 1}
\subsection{CHAPTER 1}
\begin{tcolorbox}
\textsubscript{1} В начале сотворил Бог небо и землю.
\end{tcolorbox}
\begin{tcolorbox}
\textsubscript{2} Земля же была безвидна и пуста, и тьма над бездною, и Дух Божий носился над водою.
\end{tcolorbox}
\begin{tcolorbox}
\textsubscript{3} И сказал Бог: да будет свет. И стал свет.
\end{tcolorbox}
\begin{tcolorbox}
\textsubscript{4} И увидел Бог свет, что он хорош, и отделил Бог свет от тьмы.
\end{tcolorbox}
\begin{tcolorbox}
\textsubscript{5} И назвал Бог свет днем, а тьму ночью. И был вечер, и было утро: день один.
\end{tcolorbox}
\begin{tcolorbox}
\textsubscript{6} И сказал Бог: да будет твердь посреди воды, и да отделяет она воду от воды.
\end{tcolorbox}
\begin{tcolorbox}
\textsubscript{7} И создал Бог твердь, и отделил воду, которая под твердью, от воды, которая над твердью. И стало так.
\end{tcolorbox}
\begin{tcolorbox}
\textsubscript{8} И назвал Бог твердь небом. И был вечер, и было утро: день второй.
\end{tcolorbox}
\begin{tcolorbox}
\textsubscript{9} И сказал Бог: да соберется вода, которая под небом, в одно место, и да явится суша. И стало так.
\end{tcolorbox}
\begin{tcolorbox}
\textsubscript{10} И назвал Бог сушу землею, а собрание вод назвал морями. И увидел Бог, что [это] хорошо.
\end{tcolorbox}
\begin{tcolorbox}
\textsubscript{11} И сказал Бог: да произрастит земля зелень, траву, сеющую семя дерево плодовитое, приносящее по роду своему плод, в котором семя его на земле. И стало так.
\end{tcolorbox}
\begin{tcolorbox}
\textsubscript{12} И произвела земля зелень, траву, сеющую семя по роду ее, и дерево, приносящее плод, в котором семя его по роду его. И увидел Бог, что [это] хорошо.
\end{tcolorbox}
\begin{tcolorbox}
\textsubscript{13} И был вечер, и было утро: день третий.
\end{tcolorbox}
\begin{tcolorbox}
\textsubscript{14} И сказал Бог: да будут светила на тверди небесной для отделения дня от ночи, и для знамений, и времен, и дней, и годов;
\end{tcolorbox}
\begin{tcolorbox}
\textsubscript{15} и да будут они светильниками на тверди небесной, чтобы светить на землю. И стало так.
\end{tcolorbox}
\begin{tcolorbox}
\textsubscript{16} И создал Бог два светила великие: светило большее, для управления днем, и светило меньшее, для управления ночью, и звезды;
\end{tcolorbox}
\begin{tcolorbox}
\textsubscript{17} и поставил их Бог на тверди небесной, чтобы светить на землю,
\end{tcolorbox}
\begin{tcolorbox}
\textsubscript{18} и управлять днем и ночью, и отделять свет от тьмы. И увидел Бог, что [это] хорошо.
\end{tcolorbox}
\begin{tcolorbox}
\textsubscript{19} И был вечер, и было утро: день четвёртый.
\end{tcolorbox}
\begin{tcolorbox}
\textsubscript{20} И сказал Бог: да произведет вода пресмыкающихся, душу живую; и птицы да полетят над землею, по тверди небесной.
\end{tcolorbox}
\begin{tcolorbox}
\textsubscript{21} И сотворил Бог рыб больших и всякую душу животных пресмыкающихся, которых произвела вода, по роду их, и всякую птицу пернатую по роду ее. И увидел Бог, что [это] хорошо.
\end{tcolorbox}
\begin{tcolorbox}
\textsubscript{22} И благословил их Бог, говоря: плодитесь и размножайтесь, и наполняйте воды в морях, и птицы да размножаются на земле.
\end{tcolorbox}
\begin{tcolorbox}
\textsubscript{23} И был вечер, и было утро: день пятый.
\end{tcolorbox}
\begin{tcolorbox}
\textsubscript{24} И сказал Бог: да произведет земля душу живую по роду ее, скотов, и гадов, и зверей земных по роду их. И стало так.
\end{tcolorbox}
\begin{tcolorbox}
\textsubscript{25} И создал Бог зверей земных по роду их, и скот по роду его, и всех гадов земных по роду их. И увидел Бог, что [это] хорошо.
\end{tcolorbox}
\begin{tcolorbox}
\textsubscript{26} И сказал Бог: сотворим человека по образу Нашему по подобию Нашему, и да владычествуют они над рыбами морскими, и над птицами небесными, и над скотом, и над всею землею, и над всеми гадами, пресмыкающимися по земле.
\end{tcolorbox}
\begin{tcolorbox}
\textsubscript{27} И сотворил Бог человека по образу Своему, по образу Божию сотворил его; мужчину и женщину сотворил их.
\end{tcolorbox}
\begin{tcolorbox}
\textsubscript{28} И благословил их Бог, и сказал им Бог: плодитесь и размножайтесь, и наполняйте землю, и обладайте ею, и владычествуйте над рыбами морскими и над птицами небесными, и над всяким животным, пресмыкающимся по земле.
\end{tcolorbox}
\begin{tcolorbox}
\textsubscript{29} И сказал Бог: вот, Я дал вам всякую траву, сеющую семя, какая есть на всей земле, и всякое дерево, у которого плод древесный, сеющий семя; --вам [сие] будет в пищу;
\end{tcolorbox}
\begin{tcolorbox}
\textsubscript{30} а всем зверям земным, и всем птицам небесным, и всякому пресмыкающемуся по земле, в котором душа живая, [дал] Я всю зелень травную в пищу. И стало так.
\end{tcolorbox}
\begin{tcolorbox}
\textsubscript{31} И увидел Бог все, что Он создал, и вот, хорошо весьма. И был вечер, и было утро: день шестой.
\end{tcolorbox}
\subsection{CHAPTER 2}
\begin{tcolorbox}
\textsubscript{1} Так совершены небо и земля и все воинство их.
\end{tcolorbox}
\begin{tcolorbox}
\textsubscript{2} И совершил Бог к седьмому дню дела Свои, которые Он делал, и почил в день седьмый от всех дел Своих, которые делал.
\end{tcolorbox}
\begin{tcolorbox}
\textsubscript{3} И благословил Бог седьмой день, и освятил его, ибо в оный почил от всех дел Своих, которые Бог творил и созидал.
\end{tcolorbox}
\begin{tcolorbox}
\textsubscript{4} Вот происхождение неба и земли, при сотворении их, в то время, когда Господь Бог создал землю и небо,
\end{tcolorbox}
\begin{tcolorbox}
\textsubscript{5} и всякий полевой кустарник, которого еще не было на земле, и всякую полевую траву, которая еще не росла, ибо Господь Бог не посылал дождя на землю, и не было человека для возделывания земли,
\end{tcolorbox}
\begin{tcolorbox}
\textsubscript{6} но пар поднимался с земли и орошал все лице земли.
\end{tcolorbox}
\begin{tcolorbox}
\textsubscript{7} И создал Господь Бог человека из праха земного, и вдунул в лице его дыхание жизни, и стал человек душею живою.
\end{tcolorbox}
\begin{tcolorbox}
\textsubscript{8} И насадил Господь Бог рай в Едеме на востоке, и поместил там человека, которого создал.
\end{tcolorbox}
\begin{tcolorbox}
\textsubscript{9} И произрастил Господь Бог из земли всякое дерево, приятное на вид и хорошее для пищи, и дерево жизни посреди рая, и дерево познания добра и зла.
\end{tcolorbox}
\begin{tcolorbox}
\textsubscript{10} Из Едема выходила река для орошения рая; и потом разделялась на четыре реки.
\end{tcolorbox}
\begin{tcolorbox}
\textsubscript{11} Имя одной Фисон: она обтекает всю землю Хавила, ту, где золото;
\end{tcolorbox}
\begin{tcolorbox}
\textsubscript{12} и золото той земли хорошее; там бдолах и камень оникс.
\end{tcolorbox}
\begin{tcolorbox}
\textsubscript{13} Имя второй реки Гихон: она обтекает всю землю Куш.
\end{tcolorbox}
\begin{tcolorbox}
\textsubscript{14} Имя третьей реки Хиддекель: она протекает пред Ассириею. Четвертая река Евфрат.
\end{tcolorbox}
\begin{tcolorbox}
\textsubscript{15} И взял Господь Бог человека, и поселил его в саду Едемском, чтобы возделывать его и хранить его.
\end{tcolorbox}
\begin{tcolorbox}
\textsubscript{16} И заповедал Господь Бог человеку, говоря: от всякого дерева в саду ты будешь есть,
\end{tcolorbox}
\begin{tcolorbox}
\textsubscript{17} а от дерева познания добра и зла не ешь от него, ибо в день, в который ты вкусишь от него, смертью умрешь.
\end{tcolorbox}
\begin{tcolorbox}
\textsubscript{18} И сказал Господь Бог: не хорошо быть человеку одному; сотворим ему помощника, соответственного ему.
\end{tcolorbox}
\begin{tcolorbox}
\textsubscript{19} Господь Бог образовал из земли всех животных полевых и всех птиц небесных, и привел к человеку, чтобы видеть, как он назовет их, и чтобы, как наречет человек всякую душу живую, так и было имя ей.
\end{tcolorbox}
\begin{tcolorbox}
\textsubscript{20} И нарек человек имена всем скотам и птицам небесным и всем зверям полевым; но для человека не нашлось помощника, подобного ему.
\end{tcolorbox}
\begin{tcolorbox}
\textsubscript{21} И навел Господь Бог на человека крепкий сон; и, когда он уснул, взял одно из ребр его, и закрыл то место плотию.
\end{tcolorbox}
\begin{tcolorbox}
\textsubscript{22} И создал Господь Бог из ребра, взятого у человека, жену, и привел ее к человеку.
\end{tcolorbox}
\begin{tcolorbox}
\textsubscript{23} И сказал человек: вот, это кость от костей моих и плоть от плоти моей; она будет называться женою, ибо взята от мужа.
\end{tcolorbox}
\begin{tcolorbox}
\textsubscript{24} Потому оставит человек отца своего и мать свою и прилепится к жене своей; и будут одна плоть.
\end{tcolorbox}
\begin{tcolorbox}
\textsubscript{25} И были оба наги, Адам и жена его, и не стыдились.
\end{tcolorbox}
\subsection{CHAPTER 3}
\begin{tcolorbox}
\textsubscript{1} Змей был хитрее всех зверей полевых, которых создал Господь Бог. И сказал змей жене: подлинно ли сказал Бог: не ешьте ни от какого дерева в раю?
\end{tcolorbox}
\begin{tcolorbox}
\textsubscript{2} И сказала жена змею: плоды с дерев мы можем есть,
\end{tcolorbox}
\begin{tcolorbox}
\textsubscript{3} только плодов дерева, которое среди рая, сказал Бог, не ешьте их и не прикасайтесь к ним, чтобы вам не умереть.
\end{tcolorbox}
\begin{tcolorbox}
\textsubscript{4} И сказал змей жене: нет, не умрете,
\end{tcolorbox}
\begin{tcolorbox}
\textsubscript{5} но знает Бог, что в день, в который вы вкусите их, откроются глаза ваши, и вы будете, как боги, знающие добро и зло.
\end{tcolorbox}
\begin{tcolorbox}
\textsubscript{6} И увидела жена, что дерево хорошо для пищи, и что оно приятно для глаз и вожделенно, потому что дает знание; и взяла плодов его и ела; и дала также мужу своему, и он ел.
\end{tcolorbox}
\begin{tcolorbox}
\textsubscript{7} И открылись глаза у них обоих, и узнали они, что наги, и сшили смоковные листья, и сделали себе опоясания.
\end{tcolorbox}
\begin{tcolorbox}
\textsubscript{8} И услышали голос Господа Бога, ходящего в раю во время прохлады дня; и скрылся Адам и жена его от лица Господа Бога между деревьями рая.
\end{tcolorbox}
\begin{tcolorbox}
\textsubscript{9} И воззвал Господь Бог к Адаму и сказал ему: где ты?
\end{tcolorbox}
\begin{tcolorbox}
\textsubscript{10} Он сказал: голос Твой я услышал в раю, и убоялся, потому что я наг, и скрылся.
\end{tcolorbox}
\begin{tcolorbox}
\textsubscript{11} И сказал: кто сказал тебе, что ты наг? не ел ли ты от дерева, с которого Я запретил тебе есть?
\end{tcolorbox}
\begin{tcolorbox}
\textsubscript{12} Адам сказал: жена, которую Ты мне дал, она дала мне от дерева, и я ел.
\end{tcolorbox}
\begin{tcolorbox}
\textsubscript{13} И сказал Господь Бог жене: что ты это сделала? Жена сказала: змей обольстил меня, и я ела.
\end{tcolorbox}
\begin{tcolorbox}
\textsubscript{14} И сказал Господь Бог змею: за то, что ты сделал это, проклят ты пред всеми скотами и пред всеми зверями полевыми; ты будешь ходить на чреве твоем, и будешь есть прах во все дни жизни твоей;
\end{tcolorbox}
\begin{tcolorbox}
\textsubscript{15} и вражду положу между тобою и между женою, и между семенем твоим и между семенем ее; оно будет поражать тебя в голову, а ты будешь жалить его в пяту.
\end{tcolorbox}
\begin{tcolorbox}
\textsubscript{16} Жене сказал: умножая умножу скорбь твою в беременности твоей; в болезни будешь рождать детей; и к мужу твоему влечение твое, и он будет господствовать над тобою.
\end{tcolorbox}
\begin{tcolorbox}
\textsubscript{17} Адаму же сказал: за то, что ты послушал голоса жены твоей и ел от дерева, о котором Я заповедал тебе, сказав: не ешь от него, проклята земля за тебя; со скорбью будешь питаться от нее во все дни жизни твоей;
\end{tcolorbox}
\begin{tcolorbox}
\textsubscript{18} терния и волчцы произрастит она тебе; и будешь питаться полевою травою;
\end{tcolorbox}
\begin{tcolorbox}
\textsubscript{19} в поте лица твоего будешь есть хлеб, доколе не возвратишься в землю, из которой ты взят, ибо прах ты и в прах возвратишься.
\end{tcolorbox}
\begin{tcolorbox}
\textsubscript{20} И нарек Адам имя жене своей: Ева, ибо она стала матерью всех живущих.
\end{tcolorbox}
\begin{tcolorbox}
\textsubscript{21} И сделал Господь Бог Адаму и жене его одежды кожаные и одел их.
\end{tcolorbox}
\begin{tcolorbox}
\textsubscript{22} И сказал Господь Бог: вот, Адам стал как один из Нас, зная добро и зло; и теперь как бы не простер он руки своей, и не взял также от дерева жизни, и не вкусил, и не стал жить вечно.
\end{tcolorbox}
\begin{tcolorbox}
\textsubscript{23} И выслал его Господь Бог из сада Едемского, чтобы возделывать землю, из которой он взят.
\end{tcolorbox}
\begin{tcolorbox}
\textsubscript{24} И изгнал Адама, и поставил на востоке у сада Едемского Херувима и пламенный меч обращающийся, чтобы охранять путь к дереву жизни.
\end{tcolorbox}
\subsection{CHAPTER 4}
\begin{tcolorbox}
\textsubscript{1} Адам познал Еву, жену свою; и она зачала, и родила Каина, и сказала: приобрела я человека от Господа.
\end{tcolorbox}
\begin{tcolorbox}
\textsubscript{2} И еще родила брата его, Авеля. И был Авель пастырь овец, а Каин был земледелец.
\end{tcolorbox}
\begin{tcolorbox}
\textsubscript{3} Спустя несколько времени, Каин принес от плодов земли дар Господу,
\end{tcolorbox}
\begin{tcolorbox}
\textsubscript{4} и Авель также принес от первородных стада своего и от тука их. И призрел Господь на Авеля и на дар его,
\end{tcolorbox}
\begin{tcolorbox}
\textsubscript{5} а на Каина и на дар его не призрел. Каин сильно огорчился, и поникло лице его.
\end{tcolorbox}
\begin{tcolorbox}
\textsubscript{6} И сказал Господь Каину: почему ты огорчился? и отчего поникло лице твое?
\end{tcolorbox}
\begin{tcolorbox}
\textsubscript{7} если делаешь доброе, то не поднимаешь ли лица? а если не делаешь доброго, то у дверей грех лежит; он влечет тебя к себе, но ты господствуй над ним.
\end{tcolorbox}
\begin{tcolorbox}
\textsubscript{8} И сказал Каин Авелю, брату своему. И когда они были в поле, восстал Каин на Авеля, брата своего, и убил его.
\end{tcolorbox}
\begin{tcolorbox}
\textsubscript{9} И сказал Господь Каину: где Авель, брат твой? Он сказал: не знаю; разве я сторож брату моему?
\end{tcolorbox}
\begin{tcolorbox}
\textsubscript{10} И сказал: что ты сделал? голос крови брата твоего вопиет ко Мне от земли;
\end{tcolorbox}
\begin{tcolorbox}
\textsubscript{11} и ныне проклят ты от земли, которая отверзла уста свои принять кровь брата твоего от руки твоей;
\end{tcolorbox}
\begin{tcolorbox}
\textsubscript{12} когда ты будешь возделывать землю, она не станет более давать силы своей для тебя; ты будешь изгнанником и скитальцем на земле.
\end{tcolorbox}
\begin{tcolorbox}
\textsubscript{13} И сказал Каин Господу: наказание мое больше, нежели снести можно;
\end{tcolorbox}
\begin{tcolorbox}
\textsubscript{14} вот, Ты теперь сгоняешь меня с лица земли, и от лица Твоего я скроюсь, и буду изгнанником и скитальцем на земле; и всякий, кто встретится со мною, убьет меня.
\end{tcolorbox}
\begin{tcolorbox}
\textsubscript{15} И сказал ему Господь: за то всякому, кто убьет Каина, отмстится всемеро. И сделал Господь Каину знамение, чтобы никто, встретившись с ним, не убил его.
\end{tcolorbox}
\begin{tcolorbox}
\textsubscript{16} И пошел Каин от лица Господня и поселился в земле Нод, на восток от Едема.
\end{tcolorbox}
\begin{tcolorbox}
\textsubscript{17} И познал Каин жену свою; и она зачала и родила Еноха. И построил он город; и назвал город по имени сына своего: Енох.
\end{tcolorbox}
\begin{tcolorbox}
\textsubscript{18} У Еноха родился Ирад; Ирад родил Мехиаеля; Мехиаель родил Мафусала; Мафусал родил Ламеха.
\end{tcolorbox}
\begin{tcolorbox}
\textsubscript{19} И взял себе Ламех две жены: имя одной: Ада, и имя второй: Цилла.
\end{tcolorbox}
\begin{tcolorbox}
\textsubscript{20} Ада родила Иавала: он был отец живущих в шатрах со стадами.
\end{tcolorbox}
\begin{tcolorbox}
\textsubscript{21} Имя брату его Иувал: он был отец всех играющих на гуслях и свирели.
\end{tcolorbox}
\begin{tcolorbox}
\textsubscript{22} Цилла также родила Тувалкаина, который был ковачом всех орудий из меди и железа. И сестра Тувалкаина Ноема.
\end{tcolorbox}
\begin{tcolorbox}
\textsubscript{23} И сказал Ламех женам своим: Ада и Цилла! послушайте голоса моего; жены Ламеховы! внимайте словам моим: я убил мужа в язву мне и отрока в рану мне;
\end{tcolorbox}
\begin{tcolorbox}
\textsubscript{24} если за Каина отмстится всемеро, то за Ламеха в семьдесят раз всемеро.
\end{tcolorbox}
\begin{tcolorbox}
\textsubscript{25} И познал Адам еще жену свою, и она родила сына, и нарекла ему имя: Сиф, потому что, [говорила она], Бог положил мне другое семя, вместо Авеля, которого убил Каин.
\end{tcolorbox}
\begin{tcolorbox}
\textsubscript{26} У Сифа также родился сын, и он нарек ему имя: Енос; тогда начали призывать имя Господа.
\end{tcolorbox}
\subsection{CHAPTER 5}
\begin{tcolorbox}
\textsubscript{1} Вот родословие Адама: когда Бог сотворил человека, по подобию Божию создал его,
\end{tcolorbox}
\begin{tcolorbox}
\textsubscript{2} мужчину и женщину сотворил их, и благословил их, и нарек им имя: человек, в день сотворения их.
\end{tcolorbox}
\begin{tcolorbox}
\textsubscript{3} Адам жил сто тридцать лет и родил [сына] по подобию своему по образу своему, и нарек ему имя: Сиф.
\end{tcolorbox}
\begin{tcolorbox}
\textsubscript{4} Дней Адама по рождении им Сифа было восемьсот лет, и родил он сынов и дочерей.
\end{tcolorbox}
\begin{tcolorbox}
\textsubscript{5} Всех же дней жизни Адамовой было девятьсот тридцать лет; и он умер.
\end{tcolorbox}
\begin{tcolorbox}
\textsubscript{6} Сиф жил сто пять лет и родил Еноса.
\end{tcolorbox}
\begin{tcolorbox}
\textsubscript{7} По рождении Еноса Сиф жил восемьсот семь лет и родил сынов и дочерей.
\end{tcolorbox}
\begin{tcolorbox}
\textsubscript{8} Всех же дней Сифовых было девятьсот двенадцать лет; и он умер.
\end{tcolorbox}
\begin{tcolorbox}
\textsubscript{9} Енос жил девяносто лет и родил Каинана.
\end{tcolorbox}
\begin{tcolorbox}
\textsubscript{10} По рождении Каинана Енос жил восемьсот пятнадцать лет и родил сынов и дочерей.
\end{tcolorbox}
\begin{tcolorbox}
\textsubscript{11} Всех же дней Еноса было девятьсот пять лет; и он умер.
\end{tcolorbox}
\begin{tcolorbox}
\textsubscript{12} Каинан жил семьдесят лет и родил Малелеила.
\end{tcolorbox}
\begin{tcolorbox}
\textsubscript{13} По рождении Малелеила Каинан жил восемьсот сорок лет и родил сынов и дочерей.
\end{tcolorbox}
\begin{tcolorbox}
\textsubscript{14} Всех же дней Каинана было девятьсот десять лет; и он умер.
\end{tcolorbox}
\begin{tcolorbox}
\textsubscript{15} Малелеил жил шестьдесят пять лет и родил Иареда.
\end{tcolorbox}
\begin{tcolorbox}
\textsubscript{16} По рождении Иареда Малелеил жил восемьсот тридцать лет и родил сынов и дочерей.
\end{tcolorbox}
\begin{tcolorbox}
\textsubscript{17} Всех же дней Малелеила было восемьсот девяносто пять лет; и он умер.
\end{tcolorbox}
\begin{tcolorbox}
\textsubscript{18} Иаред жил сто шестьдесят два года и родил Еноха.
\end{tcolorbox}
\begin{tcolorbox}
\textsubscript{19} По рождении Еноха Иаред жил восемьсот лет и родил сынов и дочерей.
\end{tcolorbox}
\begin{tcolorbox}
\textsubscript{20} Всех же дней Иареда было девятьсот шестьдесят два года; и он умер.
\end{tcolorbox}
\begin{tcolorbox}
\textsubscript{21} Енох жил шестьдесят пять лет и родил Мафусала.
\end{tcolorbox}
\begin{tcolorbox}
\textsubscript{22} И ходил Енох пред Богом, по рождении Мафусала, триста лет и родил сынов и дочерей.
\end{tcolorbox}
\begin{tcolorbox}
\textsubscript{23} Всех же дней Еноха было триста шестьдесят пять лет.
\end{tcolorbox}
\begin{tcolorbox}
\textsubscript{24} И ходил Енох пред Богом; и не стало его, потому что Бог взял его.
\end{tcolorbox}
\begin{tcolorbox}
\textsubscript{25} Мафусал жил сто восемьдесят семь лет и родил Ламеха.
\end{tcolorbox}
\begin{tcolorbox}
\textsubscript{26} По рождении Ламеха Мафусал жил семьсот восемьдесят два года и родил сынов и дочерей.
\end{tcolorbox}
\begin{tcolorbox}
\textsubscript{27} Всех же дней Мафусала было девятьсот шестьдесят девять лет; и он умер.
\end{tcolorbox}
\begin{tcolorbox}
\textsubscript{28} Ламех жил сто восемьдесят два года и родил сына,
\end{tcolorbox}
\begin{tcolorbox}
\textsubscript{29} и нарек ему имя: Ной, сказав: он утешит нас в работе нашей и в трудах рук наших при [возделывании] земли, которую проклял Господь.
\end{tcolorbox}
\begin{tcolorbox}
\textsubscript{30} И жил Ламех по рождении Ноя пятьсот девяносто пять лет и родил сынов и дочерей.
\end{tcolorbox}
\begin{tcolorbox}
\textsubscript{31} Всех же дней Ламеха было семьсот семьдесят семь лет; и он умер.
\end{tcolorbox}
\begin{tcolorbox}
\textsubscript{32} Ною было пятьсот лет и родил Ной Сима, Хама и Иафета.
\end{tcolorbox}
\subsection{CHAPTER 6}
\begin{tcolorbox}
\textsubscript{1} Когда люди начали умножаться на земле и родились у них дочери,
\end{tcolorbox}
\begin{tcolorbox}
\textsubscript{2} тогда сыны Божии увидели дочерей человеческих, что они красивы, и брали [их] себе в жены, какую кто избрал.
\end{tcolorbox}
\begin{tcolorbox}
\textsubscript{3} И сказал Господь: не вечно Духу Моему быть пренебрегаемым человеками; потому что они плоть; пусть будут дни их сто двадцать лет.
\end{tcolorbox}
\begin{tcolorbox}
\textsubscript{4} В то время были на земле исполины, особенно же с того времени, как сыны Божии стали входить к дочерям человеческим, и они стали рождать им: это сильные, издревле славные люди.
\end{tcolorbox}
\begin{tcolorbox}
\textsubscript{5} И увидел Господь, что велико развращение человеков на земле, и что все мысли и помышления сердца их были зло во всякое время;
\end{tcolorbox}
\begin{tcolorbox}
\textsubscript{6} и раскаялся Господь, что создал человека на земле, и восскорбел в сердце Своем.
\end{tcolorbox}
\begin{tcolorbox}
\textsubscript{7} И сказал Господь: истреблю с лица земли человеков, которых Я сотворил, от человека до скотов, и гадов и птиц небесных истреблю, ибо Я раскаялся, что создал их.
\end{tcolorbox}
\begin{tcolorbox}
\textsubscript{8} Ной же обрел благодать пред очами Господа.
\end{tcolorbox}
\begin{tcolorbox}
\textsubscript{9} Вот житие Ноя: Ной был человек праведный и непорочный в роде своем; Ной ходил пред Богом.
\end{tcolorbox}
\begin{tcolorbox}
\textsubscript{10} Ной родил трех сынов: Сима, Хама и Иафета.
\end{tcolorbox}
\begin{tcolorbox}
\textsubscript{11} Но земля растлилась пред лицем Божиим, и наполнилась земля злодеяниями.
\end{tcolorbox}
\begin{tcolorbox}
\textsubscript{12} И воззрел Бог на землю, и вот, она растленна, ибо всякая плоть извратила путь свой на земле.
\end{tcolorbox}
\begin{tcolorbox}
\textsubscript{13} И сказал Бог Ною: конец всякой плоти пришел пред лице Мое, ибо земля наполнилась от них злодеяниями; и вот, Я истреблю их с земли.
\end{tcolorbox}
\begin{tcolorbox}
\textsubscript{14} Сделай себе ковчег из дерева гофер; отделения сделай в ковчеге и осмоли его смолою внутри и снаружи.
\end{tcolorbox}
\begin{tcolorbox}
\textsubscript{15} И сделай его так: длина ковчега триста локтей; ширина его пятьдесят локтей, а высота его тридцать локтей.
\end{tcolorbox}
\begin{tcolorbox}
\textsubscript{16} И сделай отверстие в ковчеге, и в локоть сведи его вверху, и дверь в ковчег сделай с боку его; устрой в нем нижнее, второе и третье [жилье].
\end{tcolorbox}
\begin{tcolorbox}
\textsubscript{17} И вот, Я наведу на землю потоп водный, чтоб истребить всякую плоть, в которой есть дух жизни, под небесами; все, что есть на земле, лишится жизни.
\end{tcolorbox}
\begin{tcolorbox}
\textsubscript{18} Но с тобою Я поставлю завет Мой, и войдешь в ковчег ты, и сыновья твои, и жена твоя, и жены сынов твоих с тобою.
\end{tcolorbox}
\begin{tcolorbox}
\textsubscript{19} Введи также в ковчег из всех животных, и от всякой плоти по паре, чтоб они остались с тобою в живых; мужеского пола и женского пусть они будут.
\end{tcolorbox}
\begin{tcolorbox}
\textsubscript{20} Из птиц по роду их, и из скотов по роду их, и из всех пресмыкающихся по земле по роду их, из всех по паре войдут к тебе, чтобы остались в живых.
\end{tcolorbox}
\begin{tcolorbox}
\textsubscript{21} Ты же возьми себе всякой пищи, какою питаются, и собери к себе; и будет она для тебя и для них пищею.
\end{tcolorbox}
\begin{tcolorbox}
\textsubscript{22} И сделал Ной всё: как повелел ему Бог, так он и сделал.
\end{tcolorbox}
\subsection{CHAPTER 7}
\begin{tcolorbox}
\textsubscript{1} И сказал Господь Ною: войди ты и все семейство твое в ковчег, ибо тебя увидел Я праведным предо Мною в роде сем;
\end{tcolorbox}
\begin{tcolorbox}
\textsubscript{2} и всякого скота чистого возьми по семи, мужеского пола и женского, а из скота нечистого по два, мужеского пола и женского;
\end{tcolorbox}
\begin{tcolorbox}
\textsubscript{3} также и из птиц небесных по семи, мужеского пола и женского, чтобы сохранить племя для всей земли,
\end{tcolorbox}
\begin{tcolorbox}
\textsubscript{4} ибо чрез семь дней Я буду изливать дождь на землю сорок дней и сорок ночей; и истреблю все существующее, что Я создал, с лица земли.
\end{tcolorbox}
\begin{tcolorbox}
\textsubscript{5} Ной сделал все, что Господь повелел ему.
\end{tcolorbox}
\begin{tcolorbox}
\textsubscript{6} Ной же был шестисот лет, как потоп водный пришел на землю.
\end{tcolorbox}
\begin{tcolorbox}
\textsubscript{7} И вошел Ной и сыновья его, и жена его, и жены сынов его с ним в ковчег от вод потопа.
\end{tcolorbox}
\begin{tcolorbox}
\textsubscript{8} И из скотов чистых и из скотов нечистых, и из всех пресмыкающихся по земле
\end{tcolorbox}
\begin{tcolorbox}
\textsubscript{9} по паре, мужеского пола и женского, вошли к Ною в ковчег, как Бог повелел Ною.
\end{tcolorbox}
\begin{tcolorbox}
\textsubscript{10} Чрез семь дней воды потопа пришли на землю.
\end{tcolorbox}
\begin{tcolorbox}
\textsubscript{11} В шестисотый год жизни Ноевой, во второй месяц, в семнадцатый день месяца, в сей день разверзлись все источники великой бездны, и окна небесные отворились;
\end{tcolorbox}
\begin{tcolorbox}
\textsubscript{12} и лился на землю дождь сорок дней и сорок ночей.
\end{tcolorbox}
\begin{tcolorbox}
\textsubscript{13} В сей самый день вошел в ковчег Ной, и Сим, Хам и Иафет, сыновья Ноевы, и жена Ноева, и три жены сынов его с ними.
\end{tcolorbox}
\begin{tcolorbox}
\textsubscript{14} Они, и все звери по роду их, и всякий скот по роду его, и все гады, пресмыкающиеся по земле, по роду их, и все летающие по роду их, все птицы, все крылатые,
\end{tcolorbox}
\begin{tcolorbox}
\textsubscript{15} и вошли к Ною в ковчег по паре от всякой плоти, в которой есть дух жизни;
\end{tcolorbox}
\begin{tcolorbox}
\textsubscript{16} и вошедшие мужеский и женский пол всякой плоти вошли, как повелел ему Бог. И затворил Господь за ним.
\end{tcolorbox}
\begin{tcolorbox}
\textsubscript{17} И продолжалось на земле наводнение сорок дней, и умножилась вода, и подняла ковчег, и он возвысился над землею;
\end{tcolorbox}
\begin{tcolorbox}
\textsubscript{18} вода же усиливалась и весьма умножалась на земле, и ковчег плавал по поверхности вод.
\end{tcolorbox}
\begin{tcolorbox}
\textsubscript{19} И усилилась вода на земле чрезвычайно, так что покрылись все высокие горы, какие есть под всем небом;
\end{tcolorbox}
\begin{tcolorbox}
\textsubscript{20} на пятнадцать локтей поднялась над ними вода, и покрылись горы.
\end{tcolorbox}
\begin{tcolorbox}
\textsubscript{21} И лишилась жизни всякая плоть, движущаяся по земле, и птицы, и скоты, и звери, и все гады, ползающие по земле, и все люди;
\end{tcolorbox}
\begin{tcolorbox}
\textsubscript{22} все, что имело дыхание духа жизни в ноздрях своих на суше, умерло.
\end{tcolorbox}
\begin{tcolorbox}
\textsubscript{23} Истребилось всякое существо, которое было на поверхности земли; от человека до скота, и гадов, и птиц небесных, --все истребилось с земли, остался только Ной и что [было] с ним в ковчеге.
\end{tcolorbox}
\begin{tcolorbox}
\textsubscript{24} Вода же усиливалась на земле сто пятьдесят дней.
\end{tcolorbox}
\subsection{CHAPTER 8}
\begin{tcolorbox}
\textsubscript{1} И вспомнил Бог о Ное, и о всех зверях, и о всех скотах, (и о всех птицах, и о всех гадах пресмыкающихся,) бывших с ним в ковчеге; и навел Бог ветер на землю, и воды остановились.
\end{tcolorbox}
\begin{tcolorbox}
\textsubscript{2} И закрылись источники бездны и окна небесные, и перестал дождь с неба.
\end{tcolorbox}
\begin{tcolorbox}
\textsubscript{3} Вода же постепенно возвращалась с земли, и стала убывать вода по окончании ста пятидесяти дней.
\end{tcolorbox}
\begin{tcolorbox}
\textsubscript{4} И остановился ковчег в седьмом месяце, в семнадцатый день месяца, на горах Араратских.
\end{tcolorbox}
\begin{tcolorbox}
\textsubscript{5} Вода постоянно убывала до десятого месяца; в первый день десятого месяца показались верхи гор.
\end{tcolorbox}
\begin{tcolorbox}
\textsubscript{6} По прошествии сорока дней Ной открыл сделанное им окно ковчега
\end{tcolorbox}
\begin{tcolorbox}
\textsubscript{7} и выпустил ворона, который, вылетев, отлетал и прилетал, пока осушилась земля от воды.
\end{tcolorbox}
\begin{tcolorbox}
\textsubscript{8} Потом выпустил от себя голубя, чтобы видеть, сошла ли вода с лица земли,
\end{tcolorbox}
\begin{tcolorbox}
\textsubscript{9} но голубь не нашел места покоя для ног своих и возвратился к нему в ковчег, ибо вода была еще на поверхности всей земли; и он простер руку свою, и взял его, и принял к себе в ковчег.
\end{tcolorbox}
\begin{tcolorbox}
\textsubscript{10} И помедлил еще семь дней других и опять выпустил голубя из ковчега.
\end{tcolorbox}
\begin{tcolorbox}
\textsubscript{11} Голубь возвратился к нему в вечернее время, и вот, свежий масличный лист во рту у него, и Ной узнал, что вода сошла с земли.
\end{tcolorbox}
\begin{tcolorbox}
\textsubscript{12} Он помедлил еще семь дней других и выпустил голубя; и он уже не возвратился к нему.
\end{tcolorbox}
\begin{tcolorbox}
\textsubscript{13} Шестьсот первого года к первому [дню] первого месяца иссякла вода на земле; и открыл Ной кровлю ковчега и посмотрел, и вот, обсохла поверхность земли.
\end{tcolorbox}
\begin{tcolorbox}
\textsubscript{14} И во втором месяце, к двадцать седьмому дню месяца, земля высохла.
\end{tcolorbox}
\begin{tcolorbox}
\textsubscript{15} И сказал Бог Ною:
\end{tcolorbox}
\begin{tcolorbox}
\textsubscript{16} выйди из ковчега ты и жена твоя, и сыновья твои, и жены сынов твоих с тобою;
\end{tcolorbox}
\begin{tcolorbox}
\textsubscript{17} выведи с собою всех животных, которые с тобою, от всякой плоти, из птиц, и скотов, и всех гадов, пресмыкающихся по земле: пусть разойдутся они по земле, и пусть плодятся и размножаются на земле.
\end{tcolorbox}
\begin{tcolorbox}
\textsubscript{18} И вышел Ной и сыновья его, и жена его, и жены сынов его с ним;
\end{tcolorbox}
\begin{tcolorbox}
\textsubscript{19} все звери, и все гады, и все птицы, все движущееся по земле, по родам своим, вышли из ковчега.
\end{tcolorbox}
\begin{tcolorbox}
\textsubscript{20} И устроил Ной жертвенник Господу; и взял из всякого скота чистого и из всех птиц чистых и принес во всесожжение на жертвеннике.
\end{tcolorbox}
\begin{tcolorbox}
\textsubscript{21} И обонял Господь приятное благоухание, и сказал Господь в сердце Своем: не буду больше проклинать землю за человека, потому что помышление сердца человеческого--зло от юности его; и не буду больше поражать всего живущего, как Я сделал:
\end{tcolorbox}
\begin{tcolorbox}
\textsubscript{22} впредь во все дни земли сеяние и жатва, холод и зной, лето и зима, день и ночь не прекратятся.
\end{tcolorbox}
\subsection{CHAPTER 9}
\begin{tcolorbox}
\textsubscript{1} И благословил Бог Ноя и сынов его и сказал им: плодитесь и размножайтесь, и наполняйте землю.
\end{tcolorbox}
\begin{tcolorbox}
\textsubscript{2} да страшатся и да трепещут вас все звери земные, и все птицы небесные, все, что движется на земле, и все рыбы морские: в ваши руки отданы они;
\end{tcolorbox}
\begin{tcolorbox}
\textsubscript{3} все движущееся, что живет, будет вам в пищу; как зелень травную даю вам все;
\end{tcolorbox}
\begin{tcolorbox}
\textsubscript{4} только плоти с душею ее, с кровью ее, не ешьте;
\end{tcolorbox}
\begin{tcolorbox}
\textsubscript{5} Я взыщу и вашу кровь, [в которой] жизнь ваша, взыщу ее от всякого зверя, взыщу также душу человека от руки человека, от руки брата его;
\end{tcolorbox}
\begin{tcolorbox}
\textsubscript{6} кто прольет кровь человеческую, того кровь прольется рукою человека: ибо человек создан по образу Божию;
\end{tcolorbox}
\begin{tcolorbox}
\textsubscript{7} вы же плодитесь и размножайтесь, и распространяйтесь по земле, и умножайтесь на ней.
\end{tcolorbox}
\begin{tcolorbox}
\textsubscript{8} И сказал Бог Ною и сынам его с ним:
\end{tcolorbox}
\begin{tcolorbox}
\textsubscript{9} вот, Я поставляю завет Мой с вами и с потомством вашим после вас,
\end{tcolorbox}
\begin{tcolorbox}
\textsubscript{10} и со всякою душею живою, которая с вами, с птицами и со скотами, и со всеми зверями земными, которые у вас, со всеми вышедшими из ковчега, со всеми животными земными;
\end{tcolorbox}
\begin{tcolorbox}
\textsubscript{11} поставляю завет Мой с вами, что не будет более истреблена всякая плоть водами потопа, и не будет уже потопа на опустошение земли.
\end{tcolorbox}
\begin{tcolorbox}
\textsubscript{12} И сказал Бог: вот знамение завета, который Я поставляю между Мною и между вами и между всякою душею живою, которая с вами, в роды навсегда:
\end{tcolorbox}
\begin{tcolorbox}
\textsubscript{13} Я полагаю радугу Мою в облаке, чтоб она была знамением завета между Мною и между землею.
\end{tcolorbox}
\begin{tcolorbox}
\textsubscript{14} И будет, когда Я наведу облако на землю, то явится радуга в облаке;
\end{tcolorbox}
\begin{tcolorbox}
\textsubscript{15} и Я вспомню завет Мой, который между Мною и между вами и между всякою душею живою во всякой плоти; и не будет более вода потопом на истребление всякой плоти.
\end{tcolorbox}
\begin{tcolorbox}
\textsubscript{16} И будет радуга в облаке, и Я увижу ее, и вспомню завет вечный между Богом и между всякою душею живою во всякой плоти, которая на земле.
\end{tcolorbox}
\begin{tcolorbox}
\textsubscript{17} И сказал Бог Ною: вот знамение завета, который Я поставил между Мною и между всякою плотью, которая на земле.
\end{tcolorbox}
\begin{tcolorbox}
\textsubscript{18} Сыновья Ноя, вышедшие из ковчега, были: Сим, Хам и Иафет. Хам же был отец Ханаана.
\end{tcolorbox}
\begin{tcolorbox}
\textsubscript{19} Сии трое были сыновья Ноевы, и от них населилась вся земля.
\end{tcolorbox}
\begin{tcolorbox}
\textsubscript{20} Ной начал возделывать землю и насадил виноградник;
\end{tcolorbox}
\begin{tcolorbox}
\textsubscript{21} и выпил он вина, и опьянел, и [лежал] обнаженным в шатре своем.
\end{tcolorbox}
\begin{tcolorbox}
\textsubscript{22} И увидел Хам, отец Ханаана, наготу отца своего, и выйдя рассказал двум братьям своим.
\end{tcolorbox}
\begin{tcolorbox}
\textsubscript{23} Сим же и Иафет взяли одежду и, положив ее на плечи свои, пошли задом и покрыли наготу отца своего; лица их были обращены назад, и они не видали наготы отца своего.
\end{tcolorbox}
\begin{tcolorbox}
\textsubscript{24} Ной проспался от вина своего и узнал, что сделал над ним меньший сын его,
\end{tcolorbox}
\begin{tcolorbox}
\textsubscript{25} и сказал: проклят Ханаан; раб рабов будет он у братьев своих.
\end{tcolorbox}
\begin{tcolorbox}
\textsubscript{26} Потом сказал: благословен Господь Бог Симов; Ханаан же будет рабом ему;
\end{tcolorbox}
\begin{tcolorbox}
\textsubscript{27} да распространит Бог Иафета, и да вселится он в шатрах Симовых; Ханаан же будет рабом ему.
\end{tcolorbox}
\begin{tcolorbox}
\textsubscript{28} И жил Ной после потопа триста пятьдесят лет.
\end{tcolorbox}
\begin{tcolorbox}
\textsubscript{29} Всех же дней Ноевых было девятьсот пятьдесят лет, и он умер.
\end{tcolorbox}
\subsection{CHAPTER 10}
\begin{tcolorbox}
\textsubscript{1} Вот родословие сынов Ноевых: Сима, Хама и Иафета. После потопа родились у них дети.
\end{tcolorbox}
\begin{tcolorbox}
\textsubscript{2} Сыны Иафета: Гомер, Магог, Мадай, Иаван, Фувал, Мешех и Фирас.
\end{tcolorbox}
\begin{tcolorbox}
\textsubscript{3} Сыны Гомера: Аскеназ, Рифат и Фогарма.
\end{tcolorbox}
\begin{tcolorbox}
\textsubscript{4} Сыны Иавана: Елиса, Фарсис, Киттим и Доданим.
\end{tcolorbox}
\begin{tcolorbox}
\textsubscript{5} От сих населились острова народов в землях их, каждый по языку своему, по племенам своим, в народах своих.
\end{tcolorbox}
\begin{tcolorbox}
\textsubscript{6} Сыны Хама: Хуш, Мицраим, Фут и Ханаан.
\end{tcolorbox}
\begin{tcolorbox}
\textsubscript{7} Сыны Хуша: Сева, Хавила, Савта, Раама и Савтеха. Сыны Раамы: Шева и дедан.
\end{tcolorbox}
\begin{tcolorbox}
\textsubscript{8} Хуш родил также Нимрода: сей начал быть силен на земле.
\end{tcolorbox}
\begin{tcolorbox}
\textsubscript{9} Он был сильный зверолов пред Господом; потому и говориться: сильный зверолов, как Нимрод, пред Господом.
\end{tcolorbox}
\begin{tcolorbox}
\textsubscript{10} Царство его вначале сщставляли: Вавилон, Эрех, аккад и Халне, в земле Сеннаар.
\end{tcolorbox}
\begin{tcolorbox}
\textsubscript{11} Из сей земли вышел Ассур, и построил Ниневию, Реховофир, Калах.
\end{tcolorbox}
\begin{tcolorbox}
\textsubscript{12} И ресен между Ниневию и между Калахом; это город великий.
\end{tcolorbox}
\begin{tcolorbox}
\textsubscript{13} От Мицраима произщшли Лудим, Анамим, Легавим, Нафтухим,
\end{tcolorbox}
\begin{tcolorbox}
\textsubscript{14} Патрусим, Каслухим, откуда вышли Филистимляне, и Кафторим.
\end{tcolorbox}
\begin{tcolorbox}
\textsubscript{15} От Ханаана родились: Сидон, первенец его, Хет,
\end{tcolorbox}
\begin{tcolorbox}
\textsubscript{16} Иевусей, Аморей, Гергесей,
\end{tcolorbox}
\begin{tcolorbox}
\textsubscript{17} Евей, Аркей, Синей,
\end{tcolorbox}
\begin{tcolorbox}
\textsubscript{18} Арвадей, Цемарей и Химарей. В последствии племена Ханаанские рассеялись.
\end{tcolorbox}
\begin{tcolorbox}
\textsubscript{19} И были пределы Хананеев от Сидона к Герару до Газы, Отсюда к Садому, Гаморре, Адме и Цевоиму до Лаши.
\end{tcolorbox}
\begin{tcolorbox}
\textsubscript{20} Это сыны Хамовы, по племенам их, по языкам их, в землях их, в народах их.
\end{tcolorbox}
\begin{tcolorbox}
\textsubscript{21} Были дети и у Сима, отца всех сынов Еверовых, старшего брата Иафетова.
\end{tcolorbox}
\begin{tcolorbox}
\textsubscript{22} Сыны Сима: Елам, Асур, Арфаксад, Луд, Арам.
\end{tcolorbox}
\begin{tcolorbox}
\textsubscript{23} Сыны Арама: Уц, Хул, Гефер и Маш.
\end{tcolorbox}
\begin{tcolorbox}
\textsubscript{24} Арфаксад родил Салу, Сала родил Евера.
\end{tcolorbox}
\begin{tcolorbox}
\textsubscript{25} У Евера родились два сына; имя одному: Фалек, потому что во дни его земля разделена; имя брата его: Иоктан.
\end{tcolorbox}
\begin{tcolorbox}
\textsubscript{26} Иоктан родил Алмодада, Шалефа, Хацармавефа, Иераха,
\end{tcolorbox}
\begin{tcolorbox}
\textsubscript{27} Гадорама, Узала, Диклу,
\end{tcolorbox}
\begin{tcolorbox}
\textsubscript{28} Овала, Авимаила, Шеву,
\end{tcolorbox}
\begin{tcolorbox}
\textsubscript{29} Офира, Хавилу и Иовава. Все эти сыновья Иоктана.
\end{tcolorbox}
\begin{tcolorbox}
\textsubscript{30} Поселения их были от Меши до Сефара, горы восточной.
\end{tcolorbox}
\begin{tcolorbox}
\textsubscript{31} Это сыновья Симовы по племенам их, по языкам их, в землях их, по народам их.
\end{tcolorbox}
\begin{tcolorbox}
\textsubscript{32} Вот племена сынов Ноевых, по Родословию их, в народах их. От них распространились народы по земле после потопа.
\end{tcolorbox}
\subsection{CHAPTER 11}
\begin{tcolorbox}
\textsubscript{1} На всей земле был один язык и одно наречие.
\end{tcolorbox}
\begin{tcolorbox}
\textsubscript{2} Двинувшись с востока, они нашли в земле Сеннаар равнину и поселились там.
\end{tcolorbox}
\begin{tcolorbox}
\textsubscript{3} И сказали друг другу: наделаем кирпичей и обожжем огнем. И стали у них кирпичи вместо камней, а земляная смола вместо извести.
\end{tcolorbox}
\begin{tcolorbox}
\textsubscript{4} И сказали они: построим себе город и башню, высотою до небес, и сделаем себе имя, прежде нежели рассеемся по лицу всей земли.
\end{tcolorbox}
\begin{tcolorbox}
\textsubscript{5} И сошел Господь посмотреть город и башню, которые строили сыны человеческие.
\end{tcolorbox}
\begin{tcolorbox}
\textsubscript{6} И сказал Господь: вот, один народ, и один у всех язык; и вот что начали они делать, и не отстанут они от того, что задумали делать;
\end{tcolorbox}
\begin{tcolorbox}
\textsubscript{7} сойдем же и смешаем там язык их, так чтобы один не понимал речи другого.
\end{tcolorbox}
\begin{tcolorbox}
\textsubscript{8} И рассеял их Господь оттуда по всей земле; и они перестали строить город.
\end{tcolorbox}
\begin{tcolorbox}
\textsubscript{9} Посему дано ему имя: Вавилон, ибо там смешал Господь язык всей земли, и оттуда рассеял их Господь по всей земле.
\end{tcolorbox}
\begin{tcolorbox}
\textsubscript{10} Вот родословие Сима: Сим был ста лет и родил Арфаксада, чрез два года после потопа;
\end{tcolorbox}
\begin{tcolorbox}
\textsubscript{11} по рождении Арфаксада Сим жил пятьсот лет и родил сынов и дочерей.
\end{tcolorbox}
\begin{tcolorbox}
\textsubscript{12} Арфаксад жил тридцать пять лет и родил Салу.
\end{tcolorbox}
\begin{tcolorbox}
\textsubscript{13} По рождении Салы Арфаксад жил четыреста три года и родил сынов и дочерей.
\end{tcolorbox}
\begin{tcolorbox}
\textsubscript{14} Сала жил тридцать лет и родил Евера.
\end{tcolorbox}
\begin{tcolorbox}
\textsubscript{15} По рождении Евера Сала жил четыреста три года и родил сынов и дочерей.
\end{tcolorbox}
\begin{tcolorbox}
\textsubscript{16} Евер жил тридцать четыре года и родил Фалека.
\end{tcolorbox}
\begin{tcolorbox}
\textsubscript{17} По рождении Фалека Евер жил четыреста тридцать лет и родил сынов и дочерей.
\end{tcolorbox}
\begin{tcolorbox}
\textsubscript{18} Фалек жил тридцать лет и родил Рагава.
\end{tcolorbox}
\begin{tcolorbox}
\textsubscript{19} По рождении Рагава Фалек жил двести девять лет и родил сынов и дочерей.
\end{tcolorbox}
\begin{tcolorbox}
\textsubscript{20} Рагав жил тридцать два года и родил Серуха.
\end{tcolorbox}
\begin{tcolorbox}
\textsubscript{21} По рождении Серуха Рагав жил двести семь лет и родил сынов и дочерей.
\end{tcolorbox}
\begin{tcolorbox}
\textsubscript{22} Серух жил тридцать лет и родил Нахора.
\end{tcolorbox}
\begin{tcolorbox}
\textsubscript{23} По рождении Нахора Серух жил двести лет и родил сынов и дочерей.
\end{tcolorbox}
\begin{tcolorbox}
\textsubscript{24} Нахор жил двадцать девять лет и родил Фарру.
\end{tcolorbox}
\begin{tcolorbox}
\textsubscript{25} По рождении Фарры Нахор жил сто девятнадцать лет и родил сынов и дочерей.
\end{tcolorbox}
\begin{tcolorbox}
\textsubscript{26} Фарра жил семьдесят лет и родил Аврама, Нахора и Арана.
\end{tcolorbox}
\begin{tcolorbox}
\textsubscript{27} Вот родословие Фарры: Фарра родил Аврама, Нахора и Арана. Аран родил Лота.
\end{tcolorbox}
\begin{tcolorbox}
\textsubscript{28} И умер Аран при Фарре, отце своем, в земле рождения своего, в Уре Халдейском.
\end{tcolorbox}
\begin{tcolorbox}
\textsubscript{29} Аврам и Нахор взяли себе жен; имя жены Аврамовой: Сара; имя жены Нахоровой: Милка, дочь Арана, отца Милки и отца Иски.
\end{tcolorbox}
\begin{tcolorbox}
\textsubscript{30} И Сара была неплодна и бездетна.
\end{tcolorbox}
\begin{tcolorbox}
\textsubscript{31} И взял Фарра Аврама, сына своего, и Лота, сына Аранова, внука своего, и Сару, невестку свою, жену Аврама, сына своего, и вышел с ними из Ура Халдейского, чтобы идти в землю Ханаанскую; но, дойдя до Харрана, они остановились там.
\end{tcolorbox}
\begin{tcolorbox}
\textsubscript{32} И было дней [жизни] Фарры двести пять лет, и умер Фарра в Харране.
\end{tcolorbox}
\subsection{CHAPTER 12}
\begin{tcolorbox}
\textsubscript{1} И сказал Господь Авраму: пойди из земли твоей, от родства твоего и из дома отца твоего, в землю, которую Я укажу тебе;
\end{tcolorbox}
\begin{tcolorbox}
\textsubscript{2} и Я произведу от тебя великий народ, и благословлю тебя, и возвеличу имя твое, и будешь ты в благословение;
\end{tcolorbox}
\begin{tcolorbox}
\textsubscript{3} Я благословлю благословляющих тебя, и злословящих тебя прокляну; и благословятся в тебе все племена земные.
\end{tcolorbox}
\begin{tcolorbox}
\textsubscript{4} И пошел Аврам, как сказал ему Господь; и с ним пошел Лот. Аврам был семидесяти пяти лет, когда вышел из Харрана.
\end{tcolorbox}
\begin{tcolorbox}
\textsubscript{5} И взял Аврам с собою Сару, жену свою, Лота, сына брата своего, и все имение, которое они приобрели, и всех людей, которых они имели в Харране; и вышли, чтобы идти в землю Ханаанскую; и пришли в землю Ханаанскую.
\end{tcolorbox}
\begin{tcolorbox}
\textsubscript{6} И прошел Аврам по земле сей до места Сихема, до дубравы Море. В этой земле тогда [жили] Хананеи.
\end{tcolorbox}
\begin{tcolorbox}
\textsubscript{7} И явился Господь Авраму и сказал: потомству твоему отдам Я землю сию. И создал [он] там жертвенник Господу, Который явился ему.
\end{tcolorbox}
\begin{tcolorbox}
\textsubscript{8} Оттуда двинулся он к горе, на восток от Вефиля; и поставил шатер свой [так, что от него] Вефиль [был] на запад, а Гай на восток; и создал там жертвенник Господу и призвал имя Господа.
\end{tcolorbox}
\begin{tcolorbox}
\textsubscript{9} И поднялся Аврам и продолжал идти к югу.
\end{tcolorbox}
\begin{tcolorbox}
\textsubscript{10} И был голод в той земле. И сошел Аврам в Египет, пожить там, потому что усилился голод в земле той.
\end{tcolorbox}
\begin{tcolorbox}
\textsubscript{11} Когда же он приближался к Египту, то сказал Саре, жене своей: вот, я знаю, что ты женщина, прекрасная видом;
\end{tcolorbox}
\begin{tcolorbox}
\textsubscript{12} и когда Египтяне увидят тебя, то скажут: это жена его; и убьют меня, а тебя оставят в живых;
\end{tcolorbox}
\begin{tcolorbox}
\textsubscript{13} скажи же, что ты мне сестра, дабы мне хорошо было ради тебя, и дабы жива была душа моя чрез тебя.
\end{tcolorbox}
\begin{tcolorbox}
\textsubscript{14} И было, когда пришел Аврам в Египет, Египтяне увидели, что она женщина весьма красивая;
\end{tcolorbox}
\begin{tcolorbox}
\textsubscript{15} увидели ее и вельможи фараоновы и похвалили ее фараону; и взята была она в дом фараонов.
\end{tcolorbox}
\begin{tcolorbox}
\textsubscript{16} И Авраму хорошо было ради ее; и был у него мелкий и крупный скот и ослы, и рабы и рабыни, и лошаки и верблюды.
\end{tcolorbox}
\begin{tcolorbox}
\textsubscript{17} Но Господь поразил тяжкими ударами фараона и дом его за Сару, жену Аврамову.
\end{tcolorbox}
\begin{tcolorbox}
\textsubscript{18} И призвал фараон Аврама и сказал: что ты это сделал со мною? для чего не сказал мне, что она жена твоя?
\end{tcolorbox}
\begin{tcolorbox}
\textsubscript{19} для чего ты сказал: она сестра моя? и я взял было ее себе в жену. И теперь вот жена твоя; возьми и пойди.
\end{tcolorbox}
\begin{tcolorbox}
\textsubscript{20} И дал о нем фараон повеление людям, и проводили его, и жену его, и все, что у него было.
\end{tcolorbox}
\subsection{CHAPTER 13}
\begin{tcolorbox}
\textsubscript{1} И поднялся Аврам из Египта, сам и жена его, и всё, что у него было, и Лот с ним, на юг.
\end{tcolorbox}
\begin{tcolorbox}
\textsubscript{2} И был Аврам очень богат скотом, и серебром, и золотом.
\end{tcolorbox}
\begin{tcolorbox}
\textsubscript{3} И продолжал он переходы свои от юга до Вефиля, до места, где прежде был шатер его между Вефилем и между Гаем,
\end{tcolorbox}
\begin{tcolorbox}
\textsubscript{4} до места жертвенника, который он сделал там вначале; и там призвал Аврам имя Господа.
\end{tcolorbox}
\begin{tcolorbox}
\textsubscript{5} И у Лота, который ходил с Аврамом, также был мелкий и крупный скот и шатры.
\end{tcolorbox}
\begin{tcolorbox}
\textsubscript{6} И непоместительна была земля для них, чтобы жить вместе, ибо имущество их было так велико, что они не могли жить вместе.
\end{tcolorbox}
\begin{tcolorbox}
\textsubscript{7} И был спор между пастухами скота Аврамова и между пастухами скота Лотова; и Хананеи и Ферезеи жили тогда в той земле.
\end{tcolorbox}
\begin{tcolorbox}
\textsubscript{8} И сказал Аврам Лоту: да не будет раздора между мною и тобою, и между пастухами моими и пастухами твоими, ибо мы родственники;
\end{tcolorbox}
\begin{tcolorbox}
\textsubscript{9} не вся ли земля пред тобою? отделись же от меня: если ты налево, то я направо; а если ты направо, то я налево.
\end{tcolorbox}
\begin{tcolorbox}
\textsubscript{10} Лот возвел очи свои и увидел всю окрестность Иорданскую, что она, прежде нежели истребил Господь Содом и Гоморру, вся до Сигора орошалась водою, как сад Господень, как земля Египетская;
\end{tcolorbox}
\begin{tcolorbox}
\textsubscript{11} и избрал себе Лот всю окрестность Иорданскую; и двинулся Лот к востоку. И отделились они друг от друга.
\end{tcolorbox}
\begin{tcolorbox}
\textsubscript{12} Аврам стал жить на земле Ханаанской; а Лот стал жить в городах окрестности и раскинул шатры до Содома.
\end{tcolorbox}
\begin{tcolorbox}
\textsubscript{13} Жители же Содомские были злы и весьма грешны пред Господом.
\end{tcolorbox}
\begin{tcolorbox}
\textsubscript{14} И сказал Господь Авраму, после того как Лот отделился от него: возведи очи твои и с места, на котором ты теперь, посмотри к северу и к югу, и к востоку и к западу;
\end{tcolorbox}
\begin{tcolorbox}
\textsubscript{15} ибо всю землю, которую ты видишь, тебе дам Я и потомству твоему навеки,
\end{tcolorbox}
\begin{tcolorbox}
\textsubscript{16} и сделаю потомство твое, как песок земной; если кто может сосчитать песок земной, то и потомство твое сочтено будет;
\end{tcolorbox}
\begin{tcolorbox}
\textsubscript{17} встань, пройди по земле сей в долготу и в широту ее, ибо Я тебе дам ее.
\end{tcolorbox}
\begin{tcolorbox}
\textsubscript{18} И двинул Аврам шатер, и пошел, и поселился у дубравы Мамре, что в Хевроне; и создал там жертвенник Господу.
\end{tcolorbox}
\subsection{CHAPTER 14}
\begin{tcolorbox}
\textsubscript{1} И было во дни Амрафела, царя Сеннаарского, Ариоха, царя Елласарского, Кедорлаомера, царя Еламского, и Фидала, царя Гоимского,
\end{tcolorbox}
\begin{tcolorbox}
\textsubscript{2} пошли они войною против Беры, царя Содомского, против Бирши, царя Гоморрского, Шинава, царя Адмы, Шемевера, царя Севоимского, и против царя Белы, которая есть Сигор.
\end{tcolorbox}
\begin{tcolorbox}
\textsubscript{3} Все сии соединились в долине Сиддим, где [ныне] море Соленое.
\end{tcolorbox}
\begin{tcolorbox}
\textsubscript{4} Двенадцать лет были они в порабощении у Кедорлаомера, а в тринадцатом году возмутились.
\end{tcolorbox}
\begin{tcolorbox}
\textsubscript{5} В четырнадцатом году пришел Кедорлаомер и цари, которые с ним, и поразили Рефаимов в Аштероф-Карнаиме, Зузимов в Гаме, Эмимов в Шаве-Кириафаиме,
\end{tcolorbox}
\begin{tcolorbox}
\textsubscript{6} и Хорреев в горе их Сеире, до Эл-Фарана, что при пустыне.
\end{tcolorbox}
\begin{tcolorbox}
\textsubscript{7} И возвратившись оттуда, они пришли к источнику Мишпат, который есть Кадес, и поразили всю страну Амаликитян, и также Аморреев, живущих в Хацацон-Фамаре.
\end{tcolorbox}
\begin{tcolorbox}
\textsubscript{8} И вышли царь Содомский, царь Гоморрский, царь Адмы, царь Севоимский и царь Белы, которая есть Сигор; и вступили в сражение с ними в долине Сиддим,
\end{tcolorbox}
\begin{tcolorbox}
\textsubscript{9} с Кедорлаомером, царем Еламским, Фидалом, царем Гоимским, Амрафелом, царем Сеннаарским, Ариохом, царем Елласарским, --четыре царя против пяти.
\end{tcolorbox}
\begin{tcolorbox}
\textsubscript{10} В долине же Сиддим было много смоляных ям. И цари Содомский и Гоморрский, обратившись в бегство, упали в них, а остальные убежали в горы.
\end{tcolorbox}
\begin{tcolorbox}
\textsubscript{11} [Победители] взяли все имущество Содома и Гоморры и весь запас их и ушли.
\end{tcolorbox}
\begin{tcolorbox}
\textsubscript{12} И взяли Лота, племянника Аврамова, жившего в Содоме, и имущество его и ушли.
\end{tcolorbox}
\begin{tcolorbox}
\textsubscript{13} И пришел один из уцелевших и известил Аврама Еврея, жившего тогда у дубравы Мамре, Аморреянина, брата Эшколу и брата Анеру, которые были союзники Аврамовы.
\end{tcolorbox}
\begin{tcolorbox}
\textsubscript{14} Аврам, услышав, что сродник его взят в плен, вооружил рабов своих, рожденных в доме его, триста восемнадцать, и преследовал [неприятелей] до Дана;
\end{tcolorbox}
\begin{tcolorbox}
\textsubscript{15} и, разделившись, [напал] на них ночью, сам и рабы его, и поразил их, и преследовал их до Ховы, что по левую сторону Дамаска;
\end{tcolorbox}
\begin{tcolorbox}
\textsubscript{16} и возвратил все имущество и Лота, сродника своего, и имущество его возвратил, также и женщин и народ.
\end{tcolorbox}
\begin{tcolorbox}
\textsubscript{17} Когда он возвращался после поражения Кедорлаомера и царей, бывших с ним, царь Содомский вышел ему навстречу в долину Шаве, что [ныне] долина царская;
\end{tcolorbox}
\begin{tcolorbox}
\textsubscript{18} и Мелхиседек, царь Салимский, вынес хлеб и вино, --он был священник Бога Всевышнего, --
\end{tcolorbox}
\begin{tcolorbox}
\textsubscript{19} и благословил его, и сказал: благословен Аврам от Бога Всевышнего, Владыки неба и земли;
\end{tcolorbox}
\begin{tcolorbox}
\textsubscript{20} и благословен Бог Всевышний, Который предал врагов твоих в руки твои. [Аврам] дал ему десятую часть из всего.
\end{tcolorbox}
\begin{tcolorbox}
\textsubscript{21} И сказал царь Содомский Авраму: отдай мне людей, а имение возьми себе.
\end{tcolorbox}
\begin{tcolorbox}
\textsubscript{22} Но Аврам сказал царю Содомскому: поднимаю руку мою к Господу Богу Всевышнему, Владыке неба и земли,
\end{tcolorbox}
\begin{tcolorbox}
\textsubscript{23} что даже нитки и ремня от обуви не возьму из всего твоего, чтобы ты не сказал: я обогатил Аврама;
\end{tcolorbox}
\begin{tcolorbox}
\textsubscript{24} кроме того, что съели отроки, и кроме доли, принадлежащей людям, которые ходили со мною; Анер, Эшкол и Мамрий пусть возьмут свою долю.
\end{tcolorbox}
\subsection{CHAPTER 15}
\begin{tcolorbox}
\textsubscript{1} После сих происшествий было слово Господа к Авраму в видении, и сказано: не бойся, Аврам; Я твой щит; награда твоя весьма велика.
\end{tcolorbox}
\begin{tcolorbox}
\textsubscript{2} Аврам сказал: Владыка Господи! что Ты дашь мне? я остаюсь бездетным; распорядитель в доме моем этот Елиезер из Дамаска.
\end{tcolorbox}
\begin{tcolorbox}
\textsubscript{3} И сказал Аврам: вот, Ты не дал мне потомства, и вот, домочадец мой наследник мой.
\end{tcolorbox}
\begin{tcolorbox}
\textsubscript{4} И было слово Господа к нему, и сказано: не будет он твоим наследником, но тот, кто произойдет из чресл твоих, будет твоим наследником.
\end{tcolorbox}
\begin{tcolorbox}
\textsubscript{5} И вывел его вон и сказал: посмотри на небо и сосчитай звезды, если ты можешь счесть их. И сказал ему: столько будет у тебя потомков.
\end{tcolorbox}
\begin{tcolorbox}
\textsubscript{6} Аврам поверил Господу, и Он вменил ему это в праведность.
\end{tcolorbox}
\begin{tcolorbox}
\textsubscript{7} И сказал ему: Я Господь, Который вывел тебя из Ура Халдейского, чтобы дать тебе землю сию во владение.
\end{tcolorbox}
\begin{tcolorbox}
\textsubscript{8} Он сказал: Владыка Господи! по чему мне узнать, что я буду владеть ею?
\end{tcolorbox}
\begin{tcolorbox}
\textsubscript{9} [Господь] сказал ему: возьми Мне трехлетнюю телицу, трехлетнюю козу, трехлетнего овна, горлицу и молодого голубя.
\end{tcolorbox}
\begin{tcolorbox}
\textsubscript{10} Он взял всех их, рассек их пополам и положил одну часть против другой; только птиц не рассек.
\end{tcolorbox}
\begin{tcolorbox}
\textsubscript{11} И налетели на трупы хищные птицы; но Аврам отгонял их.
\end{tcolorbox}
\begin{tcolorbox}
\textsubscript{12} При захождении солнца крепкий сон напал на Аврама, и вот, напал на него ужас и мрак великий.
\end{tcolorbox}
\begin{tcolorbox}
\textsubscript{13} И сказал [Господь] Авраму: знай, что потомки твои будут пришельцами в земле не своей, и поработят их, и будут угнетать их четыреста лет,
\end{tcolorbox}
\begin{tcolorbox}
\textsubscript{14} но Я произведу суд над народом, у которого они будут в порабощении; после сего они выйдут с большим имуществом,
\end{tcolorbox}
\begin{tcolorbox}
\textsubscript{15} а ты отойдешь к отцам твоим в мире [и] будешь погребен в старости доброй;
\end{tcolorbox}
\begin{tcolorbox}
\textsubscript{16} в четвертом роде возвратятся они сюда: ибо [мера] беззаконий Аморреев доселе еще не наполнилась.
\end{tcolorbox}
\begin{tcolorbox}
\textsubscript{17} Когда зашло солнце и наступила тьма, вот, дым [как бы из] печи и пламя огня прошли между рассеченными [животными].
\end{tcolorbox}
\begin{tcolorbox}
\textsubscript{18} В этот день заключил Господь завет с Аврамом, сказав: потомству твоему даю Я землю сию, от реки Египетской до великой реки, реки Евфрата:
\end{tcolorbox}
\begin{tcolorbox}
\textsubscript{19} Кенеев, Кенезеев, Кедмонеев,
\end{tcolorbox}
\begin{tcolorbox}
\textsubscript{20} Хеттеев, Ферезеев, Рефаимов,
\end{tcolorbox}
\begin{tcolorbox}
\textsubscript{21} Аморреев, Хананеев, Гергесеев и Иевусеев.
\end{tcolorbox}
\subsection{CHAPTER 16}
\begin{tcolorbox}
\textsubscript{1} Но Сара, жена Аврамова, не рождала ему. У ней была служанка Египтянка, именем Агарь.
\end{tcolorbox}
\begin{tcolorbox}
\textsubscript{2} И сказала Сара Авраму: вот, Господь заключил чрево мое, чтобы мне не рождать; войди же к служанке моей: может быть, я буду иметь детей от нее. Аврам послушался слов Сары.
\end{tcolorbox}
\begin{tcolorbox}
\textsubscript{3} И взяла Сара, жена Аврамова, служанку свою, Египтянку Агарь, по истечении десяти лет пребывания Аврамова в земле Ханаанской, и дала ее Авраму, мужу своему, в жену.
\end{tcolorbox}
\begin{tcolorbox}
\textsubscript{4} Он вошел к Агари, и она зачала. Увидев же, что зачала, она стала презирать госпожу свою.
\end{tcolorbox}
\begin{tcolorbox}
\textsubscript{5} И сказала Сара Авраму: в обиде моей ты виновен; я отдала служанку мою в недро твое; а она, увидев, что зачала, стала презирать меня; Господь пусть будет судьею между мною и между тобою.
\end{tcolorbox}
\begin{tcolorbox}
\textsubscript{6} Аврам сказал Саре: вот, служанка твоя в твоих руках; делай с нею, что тебе угодно. И Сара стала притеснять ее, и она убежала от нее.
\end{tcolorbox}
\begin{tcolorbox}
\textsubscript{7} И нашел ее Ангел Господень у источника воды в пустыне, у источника на дороге к Суру.
\end{tcolorbox}
\begin{tcolorbox}
\textsubscript{8} И сказал ей: Агарь, служанка Сарина! откуда ты пришла и куда идешь? Она сказала: я бегу от лица Сары, госпожи моей.
\end{tcolorbox}
\begin{tcolorbox}
\textsubscript{9} Ангел Господень сказал ей: возвратись к госпоже своей и покорись ей.
\end{tcolorbox}
\begin{tcolorbox}
\textsubscript{10} И сказал ей Ангел Господень: умножая умножу потомство твое, так что нельзя будет и счесть его от множества.
\end{tcolorbox}
\begin{tcolorbox}
\textsubscript{11} И еще сказал ей Ангел Господень: вот, ты беременна, и родишь сына, и наречешь ему имя Измаил, ибо услышал Господь страдание твое;
\end{tcolorbox}
\begin{tcolorbox}
\textsubscript{12} он будет [между] людьми, [как] дикий осел; руки его на всех, и руки всех на него; жить будет он пред лицем всех братьев своих.
\end{tcolorbox}
\begin{tcolorbox}
\textsubscript{13} И нарекла [Агарь] Господа, Который говорил к ней, [сим] именем: Ты Бог видящий меня. Ибо сказала она: точно я видела здесь в след видящего меня.
\end{tcolorbox}
\begin{tcolorbox}
\textsubscript{14} Посему источник [тот] называется: Беэр-лахай-рои. Он находится между Кадесом и между Баредом.
\end{tcolorbox}
\begin{tcolorbox}
\textsubscript{15} Агарь родила Авраму сына; и нарек [Аврам] имя сыну своему, рожденному от Агари: Измаил.
\end{tcolorbox}
\begin{tcolorbox}
\textsubscript{16} Аврам был восьмидесяти шести лет, когда Агарь родила Авраму Измаила.
\end{tcolorbox}
\subsection{CHAPTER 17}
\begin{tcolorbox}
\textsubscript{1} Аврам был девяноста девяти лет, и Господь явился Авраму и сказал ему: Я Бог Всемогущий; ходи предо Мною и будь непорочен;
\end{tcolorbox}
\begin{tcolorbox}
\textsubscript{2} и поставлю завет Мой между Мною и тобою, и весьма, весьма размножу тебя.
\end{tcolorbox}
\begin{tcolorbox}
\textsubscript{3} И пал Аврам на лице свое. Бог продолжал говорить с ним и сказал:
\end{tcolorbox}
\begin{tcolorbox}
\textsubscript{4} Я--вот завет Мой с тобою: ты будешь отцом множества народов,
\end{tcolorbox}
\begin{tcolorbox}
\textsubscript{5} и не будешь ты больше называться Аврамом, но будет тебе имя: Авраам, ибо Я сделаю тебя отцом множества народов;
\end{tcolorbox}
\begin{tcolorbox}
\textsubscript{6} и весьма, весьма распложу тебя, и произведу от тебя народы, и цари произойдут от тебя;
\end{tcolorbox}
\begin{tcolorbox}
\textsubscript{7} и поставлю завет Мой между Мною и тобою и между потомками твоими после тебя в роды их, завет вечный в том, что Я буду Богом твоим и потомков твоих после тебя;
\end{tcolorbox}
\begin{tcolorbox}
\textsubscript{8} и дам тебе и потомкам твоим после тебя землю, по которой ты странствуешь, всю землю Ханаанскую, во владение вечное; и буду им Богом.
\end{tcolorbox}
\begin{tcolorbox}
\textsubscript{9} И сказал Бог Аврааму: ты же соблюди завет Мой, ты и потомки твои после тебя в роды их.
\end{tcolorbox}
\begin{tcolorbox}
\textsubscript{10} Сей есть завет Мой, который вы [должны] соблюдать между Мною и между вами и между потомками твоими после тебя: да будет у вас обрезан весь мужеский пол;
\end{tcolorbox}
\begin{tcolorbox}
\textsubscript{11} обрезывайте крайнюю плоть вашу: и сие будет знамением завета между Мною и вами.
\end{tcolorbox}
\begin{tcolorbox}
\textsubscript{12} Восьми дней от рождения да будет обрезан у вас в роды ваши всякий [младенец] мужеского пола, рожденный в доме и купленный за серебро у какого-нибудь иноплеменника, который не от твоего семени.
\end{tcolorbox}
\begin{tcolorbox}
\textsubscript{13} Непременно да будет обрезан рожденный в доме твоем и купленный за серебро твое, и будет завет Мой на теле вашем заветом вечным.
\end{tcolorbox}
\begin{tcolorbox}
\textsubscript{14} Необрезанный же мужеского пола, который не обрежет крайней плоти своей, истребится душа та из народа своего, [ибо] он нарушил завет Мой.
\end{tcolorbox}
\begin{tcolorbox}
\textsubscript{15} И сказал Бог Аврааму: Сару, жену твою, не называй Сарою, но да будет имя ей: Сарра;
\end{tcolorbox}
\begin{tcolorbox}
\textsubscript{16} Я благословлю ее и дам тебе от нее сына; благословлю ее, и произойдут от нее народы, и цари народов произойдут от нее.
\end{tcolorbox}
\begin{tcolorbox}
\textsubscript{17} И пал Авраам на лице свое, и рассмеялся, и сказал сам в себе: неужели от столетнего будет сын? и Сарра, девяностолетняя, неужели родит?
\end{tcolorbox}
\begin{tcolorbox}
\textsubscript{18} И сказал Авраам Богу: о, хотя бы Измаил был жив пред лицем Твоим!
\end{tcolorbox}
\begin{tcolorbox}
\textsubscript{19} Бог же сказал: именно Сарра, жена твоя, родит тебе сына, и ты наречешь ему имя: Исаак; и поставлю завет Мой с ним заветом вечным [и] потомству его после него.
\end{tcolorbox}
\begin{tcolorbox}
\textsubscript{20} И о Измаиле Я услышал тебя: вот, Я благословлю его, и возращу его, и весьма, весьма размножу; двенадцать князей родятся от него; и Я произведу от него великий народ.
\end{tcolorbox}
\begin{tcolorbox}
\textsubscript{21} Но завет Мой поставлю с Исааком, которого родит тебе Сарра в сие самое время на другой год.
\end{tcolorbox}
\begin{tcolorbox}
\textsubscript{22} И Бог перестал говорить с Авраамом и восшел от него.
\end{tcolorbox}
\begin{tcolorbox}
\textsubscript{23} И взял Авраам Измаила, сына своего, и всех рожденных в доме своем и всех купленных за серебро свое, весь мужеский пол людей дома Авраамова; и обрезал крайнюю плоть их в тот самый день, как сказал ему Бог.
\end{tcolorbox}
\begin{tcolorbox}
\textsubscript{24} Авраам был девяноста девяти лет, когда была обрезана крайняя плоть его.
\end{tcolorbox}
\begin{tcolorbox}
\textsubscript{25} А Измаил, сын его, был тринадцати лет, когда была обрезана крайняя плоть его.
\end{tcolorbox}
\begin{tcolorbox}
\textsubscript{26} В тот же самый день обрезаны были Авраам и Измаил, сын его,
\end{tcolorbox}
\begin{tcolorbox}
\textsubscript{27} и с ним обрезан был весь мужеский пол дома его, рожденные в доме и купленные за серебро у иноплеменников.
\end{tcolorbox}
\subsection{CHAPTER 18}
\begin{tcolorbox}
\textsubscript{1} И явился ему Господь у дубравы Мамре, когда он сидел при входе в шатер, во время зноя дневного.
\end{tcolorbox}
\begin{tcolorbox}
\textsubscript{2} Он возвел очи свои и взглянул, и вот, три мужа стоят против него. Увидев, он побежал навстречу им от входа в шатер и поклонился до земли,
\end{tcolorbox}
\begin{tcolorbox}
\textsubscript{3} и сказал: Владыка! если я обрел благоволение пред очами Твоими, не пройди мимо раба Твоего;
\end{tcolorbox}
\begin{tcolorbox}
\textsubscript{4} и принесут немного воды, и омоют ноги ваши; и отдохните под сим деревом,
\end{tcolorbox}
\begin{tcolorbox}
\textsubscript{5} а я принесу хлеба, и вы подкрепите сердца ваши; потом пойдите; так как вы идете мимо раба вашего. Они сказали: сделай так, как говоришь.
\end{tcolorbox}
\begin{tcolorbox}
\textsubscript{6} И поспешил Авраам в шатер к Сарре и сказал: поскорее замеси три саты лучшей муки и сделай пресные хлебы.
\end{tcolorbox}
\begin{tcolorbox}
\textsubscript{7} И побежал Авраам к стаду, и взял теленка нежного и хорошего, и дал отроку, и тот поспешил приготовить его.
\end{tcolorbox}
\begin{tcolorbox}
\textsubscript{8} И взял масла и молока и теленка приготовленного, и поставил перед ними, а сам стоял подле них под деревом. И они ели.
\end{tcolorbox}
\begin{tcolorbox}
\textsubscript{9} И сказали ему: где Сарра, жена твоя? Он отвечал: здесь, в шатре.
\end{tcolorbox}
\begin{tcolorbox}
\textsubscript{10} И сказал [один из них]: Я опять буду у тебя в это же время, и будет сын у Сарры, жены твоей. А Сарра слушала у входа в шатер, сзади его.
\end{tcolorbox}
\begin{tcolorbox}
\textsubscript{11} Авраам же и Сарра были стары и в летах преклонных, и обыкновенное у женщин у Сарры прекратилось.
\end{tcolorbox}
\begin{tcolorbox}
\textsubscript{12} Сарра внутренно рассмеялась, сказав: мне ли, когда я состарилась, иметь сие утешение? и господин мой стар.
\end{tcolorbox}
\begin{tcolorbox}
\textsubscript{13} И сказал Господь Аврааму: отчего это рассмеялась Сарра, сказав: 'неужели я действительно могу родить, когда я состарилась'?
\end{tcolorbox}
\begin{tcolorbox}
\textsubscript{14} Есть ли что трудное для Господа? В назначенный срок буду Я у тебя в следующем году, и у Сарры [будет] сын.
\end{tcolorbox}
\begin{tcolorbox}
\textsubscript{15} Сарра же не призналась, а сказала: я не смеялась. Ибо она испугалась. Но Он сказал: нет, ты рассмеялась.
\end{tcolorbox}
\begin{tcolorbox}
\textsubscript{16} И встали те мужи и оттуда отправились к Содому; Авраам же пошел с ними, проводить их.
\end{tcolorbox}
\begin{tcolorbox}
\textsubscript{17} И сказал Господь: утаю ли Я от Авраама, что хочу делать!
\end{tcolorbox}
\begin{tcolorbox}
\textsubscript{18} От Авраама точно произойдет народ великий и сильный, и благословятся в нем все народы земли,
\end{tcolorbox}
\begin{tcolorbox}
\textsubscript{19} ибо Я избрал его для того, чтобы он заповедал сынам своим и дому своему после себя, ходить путем Господним, творя правду и суд; и исполнит Господь над Авраамом, что сказал о нем.
\end{tcolorbox}
\begin{tcolorbox}
\textsubscript{20} И сказал Господь: вопль Содомский и Гоморрский, велик он, и грех их, тяжел он весьма;
\end{tcolorbox}
\begin{tcolorbox}
\textsubscript{21} сойду и посмотрю, точно ли они поступают так, каков вопль на них, восходящий ко Мне, или нет; узнаю.
\end{tcolorbox}
\begin{tcolorbox}
\textsubscript{22} И обратились мужи оттуда и пошли в Содом; Авраам же еще стоял пред лицем Господа.
\end{tcolorbox}
\begin{tcolorbox}
\textsubscript{23} И подошел Авраам и сказал: неужели Ты погубишь праведного с нечестивым?
\end{tcolorbox}
\begin{tcolorbox}
\textsubscript{24} может быть, есть в этом городе пятьдесят праведников? неужели Ты погубишь, и не пощадишь места сего ради пятидесяти праведников, в нем?
\end{tcolorbox}
\begin{tcolorbox}
\textsubscript{25} не может быть, чтобы Ты поступил так, чтобы Ты погубил праведного с нечестивым, чтобы то же было с праведником, что с нечестивым; не может быть от Тебя! Судия всей земли поступит ли неправосудно?
\end{tcolorbox}
\begin{tcolorbox}
\textsubscript{26} Господь сказал: если Я найду в городе Содоме пятьдесят праведников, то Я ради них пощажу все место сие.
\end{tcolorbox}
\begin{tcolorbox}
\textsubscript{27} Авраам сказал в ответ: вот, я решился говорить Владыке, я, прах и пепел:
\end{tcolorbox}
\begin{tcolorbox}
\textsubscript{28} может быть, до пятидесяти праведников недостанет пяти, неужели за [недостатком] пяти Ты истребишь весь город? Он сказал: не истреблю, если найду там сорок пять.
\end{tcolorbox}
\begin{tcolorbox}
\textsubscript{29} [Авраам] продолжал говорить с Ним и сказал: может быть, найдется там сорок? Он сказал: не сделаю [того] и ради сорока.
\end{tcolorbox}
\begin{tcolorbox}
\textsubscript{30} И сказал [Авраам]: да не прогневается Владыка, что я буду говорить: может быть, найдется там тридцать? Он сказал: не сделаю, если найдется там тридцать.
\end{tcolorbox}
\begin{tcolorbox}
\textsubscript{31} [Авраам] сказал: вот, я решился говорить Владыке: может быть, найдется там двадцать? Он сказал: не истреблю ради двадцати.
\end{tcolorbox}
\begin{tcolorbox}
\textsubscript{32} [Авраам] сказал: да не прогневается Владыка, что я скажу еще однажды: может быть, найдется там десять? Он сказал: не истреблю ради десяти.
\end{tcolorbox}
\begin{tcolorbox}
\textsubscript{33} И пошел Господь, перестав говорить с Авраамом; Авраам же возвратился в свое место.
\end{tcolorbox}
\subsection{CHAPTER 19}
\begin{tcolorbox}
\textsubscript{1} И пришли те два Ангела в Содом вечером, когда Лот сидел у ворот Содома. Лот увидел, и встал, чтобы встретить их, и поклонился лицем до земли
\end{tcolorbox}
\begin{tcolorbox}
\textsubscript{2} и сказал: государи мои! зайдите в дом раба вашего и ночуйте, и умойте ноги ваши, и встаньте поутру и пойдете в путь свой. Но они сказали: нет, мы ночуем на улице.
\end{tcolorbox}
\begin{tcolorbox}
\textsubscript{3} Он же сильно упрашивал их; и они пошли к нему и пришли в дом его. Он сделал им угощение и испек пресные хлебы, и они ели.
\end{tcolorbox}
\begin{tcolorbox}
\textsubscript{4} Еще не легли они спать, как городские жители, Содомляне, от молодого до старого, весь народ со [всех] концов [города], окружили дом
\end{tcolorbox}
\begin{tcolorbox}
\textsubscript{5} и вызвали Лота и говорили ему: где люди, пришедшие к тебе на ночь? выведи их к нам; мы познаем их.
\end{tcolorbox}
\begin{tcolorbox}
\textsubscript{6} Лот вышел к ним ко входу, и запер за собою дверь,
\end{tcolorbox}
\begin{tcolorbox}
\textsubscript{7} и сказал: братья мои, не делайте зла;
\end{tcolorbox}
\begin{tcolorbox}
\textsubscript{8} вот у меня две дочери, которые не познали мужа; лучше я выведу их к вам, делайте с ними, что вам угодно, только людям сим не делайте ничего, так как они пришли под кров дома моего.
\end{tcolorbox}
\begin{tcolorbox}
\textsubscript{9} Но они сказали: пойди сюда. И сказали: вот пришлец, и хочет судить? теперь мы хуже поступим с тобою, нежели с ними. И очень приступали к человеку сему, к Лоту, и подошли, чтобы выломать дверь.
\end{tcolorbox}
\begin{tcolorbox}
\textsubscript{10} Тогда мужи те простерли руки свои и ввели Лота к себе в дом, и дверь заперли;
\end{tcolorbox}
\begin{tcolorbox}
\textsubscript{11} а людей, бывших при входе в дом, поразили слепотою, от малого до большого, так что они измучились, искав входа.
\end{tcolorbox}
\begin{tcolorbox}
\textsubscript{12} Сказали мужи те Лоту: кто у тебя есть еще здесь? зять ли, сыновья ли твои, дочери ли твои, и кто бы ни был у тебя в городе, всех выведи из сего места,
\end{tcolorbox}
\begin{tcolorbox}
\textsubscript{13} ибо мы истребим сие место, потому что велик вопль на жителей его к Господу, и Господь послал нас истребить его.
\end{tcolorbox}
\begin{tcolorbox}
\textsubscript{14} И вышел Лот, и говорил с зятьями своими, которые брали за себя дочерей его, и сказал: встаньте, выйдите из сего места, ибо Господь истребит сей город. Но зятьям его показалось, что он шутит.
\end{tcolorbox}
\begin{tcolorbox}
\textsubscript{15} Когда взошла заря, Ангелы начали торопить Лота, говоря: встань, возьми жену твою и двух дочерей твоих, которые у тебя, чтобы не погибнуть тебе за беззакония города.
\end{tcolorbox}
\begin{tcolorbox}
\textsubscript{16} И как он медлил, то мужи те, по милости к нему Господней, взяли за руку его и жену его, и двух дочерей его, и вывели его и поставили его вне города.
\end{tcolorbox}
\begin{tcolorbox}
\textsubscript{17} Когда же вывели их вон, [то один из них] сказал: спасай душу свою; не оглядывайся назад и нигде не останавливайся в окрестности сей; спасайся на гору, чтобы тебе не погибнуть.
\end{tcolorbox}
\begin{tcolorbox}
\textsubscript{18} Но Лот сказал им: нет, Владыка!
\end{tcolorbox}
\begin{tcolorbox}
\textsubscript{19} вот, раб Твой обрел благоволение пред очами Твоими, и велика милость Твоя, которую Ты сделал со мною, что спас жизнь мою; но я не могу спасаться на гору, чтоб не застигла меня беда и мне не умереть;
\end{tcolorbox}
\begin{tcolorbox}
\textsubscript{20} вот, ближе бежать в сей город, он же мал; побегу я туда, --он же мал; и сохранится жизнь моя.
\end{tcolorbox}
\begin{tcolorbox}
\textsubscript{21} И сказал ему: вот, в угодность тебе Я сделаю и это: не ниспровергну города, о котором ты говоришь;
\end{tcolorbox}
\begin{tcolorbox}
\textsubscript{22} поспешай, спасайся туда, ибо Я не могу сделать дела, доколе ты не придешь туда. Потому и назван город сей: Сигор.
\end{tcolorbox}
\begin{tcolorbox}
\textsubscript{23} Солнце взошло над землею, и Лот пришел в Сигор.
\end{tcolorbox}
\begin{tcolorbox}
\textsubscript{24} И пролил Господь на Содом и Гоморру дождем серу и огонь от Господа с неба,
\end{tcolorbox}
\begin{tcolorbox}
\textsubscript{25} и ниспроверг города сии, и всю окрестность сию, и всех жителей городов сих, и произрастания земли.
\end{tcolorbox}
\begin{tcolorbox}
\textsubscript{26} Жена же [Лотова] оглянулась позади его, и стала соляным столпом.
\end{tcolorbox}
\begin{tcolorbox}
\textsubscript{27} И встал Авраам рано утром и [пошел] на место, где стоял пред лицем Господа,
\end{tcolorbox}
\begin{tcolorbox}
\textsubscript{28} и посмотрел к Содому и Гоморре и на все пространство окрестности и увидел: вот, дым поднимается с земли, как дым из печи.
\end{tcolorbox}
\begin{tcolorbox}
\textsubscript{29} И было, когда Бог истреблял города окрестности сей, вспомнил Бог об Аврааме и выслал Лота из среды истребления, когда ниспровергал города, в которых жил Лот.
\end{tcolorbox}
\begin{tcolorbox}
\textsubscript{30} И вышел Лот из Сигора и стал жить в горе, и с ним две дочери его, ибо он боялся жить в Сигоре. И жил в пещере, и с ним две дочери его.
\end{tcolorbox}
\begin{tcolorbox}
\textsubscript{31} И сказала старшая младшей: отец наш стар, и нет человека на земле, который вошел бы к нам по обычаю всей земли;
\end{tcolorbox}
\begin{tcolorbox}
\textsubscript{32} итак напоим отца нашего вином, и переспим с ним, и восставим от отца нашего племя.
\end{tcolorbox}
\begin{tcolorbox}
\textsubscript{33} И напоили отца своего вином в ту ночь; и вошла старшая и спала с отцом своим: а он не знал, когда она легла и когда встала.
\end{tcolorbox}
\begin{tcolorbox}
\textsubscript{34} На другой день старшая сказала младшей: вот, я спала вчера с отцом моим; напоим его вином и в эту ночь; и ты войди, спи с ним, и восставим от отца нашего племя.
\end{tcolorbox}
\begin{tcolorbox}
\textsubscript{35} И напоили отца своего вином и в эту ночь; и вошла младшая и спала с ним; и он не знал, когда она легла и когда встала.
\end{tcolorbox}
\begin{tcolorbox}
\textsubscript{36} И сделались обе дочери Лотовы беременными от отца своего,
\end{tcolorbox}
\begin{tcolorbox}
\textsubscript{37} и родила старшая сына, и нарекла ему имя: Моав. Он отец Моавитян доныне.
\end{tcolorbox}
\begin{tcolorbox}
\textsubscript{38} И младшая также родила сына, и нарекла ему имя: Бен-Амми. Он отец Аммонитян доныне.
\end{tcolorbox}
\subsection{CHAPTER 20}
\begin{tcolorbox}
\textsubscript{1} Авраам поднялся оттуда к югу и поселился между Кадесом и между Суром; и был на время в Гераре.
\end{tcolorbox}
\begin{tcolorbox}
\textsubscript{2} И сказал Авраам о Сарре, жене своей: она сестра моя. И послал Авимелех, царь Герарский, и взял Сарру.
\end{tcolorbox}
\begin{tcolorbox}
\textsubscript{3} И пришел Бог к Авимелеху ночью во сне и сказал ему: вот, ты умрешь за женщину, которую ты взял, ибо она имеет мужа.
\end{tcolorbox}
\begin{tcolorbox}
\textsubscript{4} Авимелех же не прикасался к ней и сказал: Владыка! неужели ты погубишь и невинный народ?
\end{tcolorbox}
\begin{tcolorbox}
\textsubscript{5} Не сам ли он сказал мне: она сестра моя? И она сама сказала: он брат мой. Я сделал это в простоте сердца моего и в чистоте рук моих.
\end{tcolorbox}
\begin{tcolorbox}
\textsubscript{6} И сказал ему Бог во сне: и Я знаю, что ты сделал сие в простоте сердца твоего, и удержал тебя от греха предо Мною, потому и не допустил тебя прикоснуться к ней;
\end{tcolorbox}
\begin{tcolorbox}
\textsubscript{7} теперь же возврати жену мужу, ибо он пророк и помолится о тебе, и ты будешь жив; а если не возвратишь, то знай, что непременно умрешь ты и все твои.
\end{tcolorbox}
\begin{tcolorbox}
\textsubscript{8} И встал Авимелех утром рано, и призвал всех рабов своих, и пересказал все слова сии в уши их; и люди сии весьма испугались.
\end{tcolorbox}
\begin{tcolorbox}
\textsubscript{9} И призвал Авимелех Авраама и сказал ему: что ты с нами сделал? чем согрешил я против тебя, что ты навел было на меня и на царство мое великий грех? Ты сделал со мною дела, каких не делают.
\end{tcolorbox}
\begin{tcolorbox}
\textsubscript{10} И сказал Авимелех Аврааму: что ты имел в виду, когда делал это дело?
\end{tcolorbox}
\begin{tcolorbox}
\textsubscript{11} Авраам сказал: я подумал, что нет на месте сем страха Божия, и убьют меня за жену мою;
\end{tcolorbox}
\begin{tcolorbox}
\textsubscript{12} да она и подлинно сестра мне: она дочь отца моего, только не дочь матери моей; и сделалась моею женою;
\end{tcolorbox}
\begin{tcolorbox}
\textsubscript{13} когда Бог повел меня странствовать из дома отца моего, то я сказал ей: сделай со мною сию милость, в какое ни придем мы место, везде говори обо мне: это брат мой.
\end{tcolorbox}
\begin{tcolorbox}
\textsubscript{14} И взял Авимелех мелкого и крупного скота, и рабов и рабынь, и дал Аврааму; и возвратил ему Сарру, жену его.
\end{tcolorbox}
\begin{tcolorbox}
\textsubscript{15} И сказал Авимелех: вот, земля моя пред тобою; живи, где тебе угодно.
\end{tcolorbox}
\begin{tcolorbox}
\textsubscript{16} И Сарре сказал: вот, я дал брату твоему тысячу [сиклей] серебра; вот, это тебе покрывало для очей пред всеми, которые с тобою, и пред всеми ты оправдана.
\end{tcolorbox}
\begin{tcolorbox}
\textsubscript{17} И помолился Авраам Богу, и исцелил Бог Авимелеха, и жену его, и рабынь его, и они стали рождать;
\end{tcolorbox}
\begin{tcolorbox}
\textsubscript{18} ибо заключил Господь всякое чрево в доме Авимелеха за Сарру, жену Авраамову.
\end{tcolorbox}
\subsection{CHAPTER 21}
\begin{tcolorbox}
\textsubscript{1} И призрел Господь на Сарру, как сказал; и сделал Господь Сарре, как говорил.
\end{tcolorbox}
\begin{tcolorbox}
\textsubscript{2} Сарра зачала и родила Аврааму сына в старости его во время, о котором говорил ему Бог;
\end{tcolorbox}
\begin{tcolorbox}
\textsubscript{3} и нарек Авраам имя сыну своему, родившемуся у него, которого родила ему Сарра, Исаак;
\end{tcolorbox}
\begin{tcolorbox}
\textsubscript{4} и обрезал Авраам Исаака, сына своего, в восьмой день, как заповедал ему Бог.
\end{tcolorbox}
\begin{tcolorbox}
\textsubscript{5} Авраам был ста лет, когда родился у него Исаак, сын его.
\end{tcolorbox}
\begin{tcolorbox}
\textsubscript{6} И сказала Сарра: смех сделал мне Бог; кто ни услышит обо мне, рассмеется.
\end{tcolorbox}
\begin{tcolorbox}
\textsubscript{7} И сказала: кто сказал бы Аврааму: Сарра будет кормить детей грудью? ибо в старости его я родила сына.
\end{tcolorbox}
\begin{tcolorbox}
\textsubscript{8} Дитя выросло и отнято от груди; и Авраам сделал большой пир в тот день, когда Исаак отнят был от груди.
\end{tcolorbox}
\begin{tcolorbox}
\textsubscript{9} И увидела Сарра, что сын Агари Египтянки, которого она родила Аврааму, насмехается,
\end{tcolorbox}
\begin{tcolorbox}
\textsubscript{10} и сказала Аврааму: выгони эту рабыню и сына ее, ибо не наследует сын рабыни сей с сыном моим Исааком.
\end{tcolorbox}
\begin{tcolorbox}
\textsubscript{11} И показалось это Аврааму весьма неприятным ради сына его.
\end{tcolorbox}
\begin{tcolorbox}
\textsubscript{12} Но Бог сказал Аврааму: не огорчайся ради отрока и рабыни твоей; во всем, что скажет тебе Сарра, слушайся голоса ее, ибо в Исааке наречется тебе семя;
\end{tcolorbox}
\begin{tcolorbox}
\textsubscript{13} и от сына рабыни Я произведу народ, потому что он семя твое.
\end{tcolorbox}
\begin{tcolorbox}
\textsubscript{14} Авраам встал рано утром, и взял хлеба и мех воды, и дал Агари, положив ей на плечи, и отрока, и отпустил ее. Она пошла, и заблудилась в пустыне Вирсавии;
\end{tcolorbox}
\begin{tcolorbox}
\textsubscript{15} и не стало воды в мехе, и она оставила отрока под одним кустом
\end{tcolorbox}
\begin{tcolorbox}
\textsubscript{16} и пошла, села вдали, в расстоянии на [один] выстрел из лука. Ибо она сказала: не [хочу] видеть смерти отрока. И она села против, и подняла вопль, и плакала;
\end{tcolorbox}
\begin{tcolorbox}
\textsubscript{17} и услышал Бог голос отрока; и Ангел Божий с неба воззвал к Агари и сказал ей: что с тобою, Агарь? не бойся; Бог услышал голос отрока оттуда, где он находится;
\end{tcolorbox}
\begin{tcolorbox}
\textsubscript{18} встань, подними отрока и возьми его за руку, ибо Я произведу от него великий народ.
\end{tcolorbox}
\begin{tcolorbox}
\textsubscript{19} И Бог открыл глаза ее, и она увидела колодезь с водою, и пошла, наполнила мех водою и напоила отрока.
\end{tcolorbox}
\begin{tcolorbox}
\textsubscript{20} И Бог был с отроком; и он вырос, и стал жить в пустыне, и сделался стрелком из лука.
\end{tcolorbox}
\begin{tcolorbox}
\textsubscript{21} Он жил в пустыне Фаран; и мать его взяла ему жену из земли Египетской.
\end{tcolorbox}
\begin{tcolorbox}
\textsubscript{22} И было в то время, Авимелех с Фихолом, военачальником своим, сказал Аврааму: с тобою Бог во всем, что ты ни делаешь;
\end{tcolorbox}
\begin{tcolorbox}
\textsubscript{23} и теперь поклянись мне здесь Богом, что ты не обидишь ни меня, ни сына моего, ни внука моего; и как я хорошо поступал с тобою, так и ты будешь поступать со мною и землею, в которой ты гостишь.
\end{tcolorbox}
\begin{tcolorbox}
\textsubscript{24} И сказал Авраам: я клянусь.
\end{tcolorbox}
\begin{tcolorbox}
\textsubscript{25} И Авраам упрекал Авимелеха за колодезь с водою, который отняли рабы Авимелеховы.
\end{tcolorbox}
\begin{tcolorbox}
\textsubscript{26} Авимелех же сказал: не знаю, кто это сделал, и ты не сказал мне; я даже и не слыхал [о том] доныне.
\end{tcolorbox}
\begin{tcolorbox}
\textsubscript{27} И взял Авраам мелкого и крупного скота и дал Авимелеху, и они оба заключили союз.
\end{tcolorbox}
\begin{tcolorbox}
\textsubscript{28} И поставил Авраам семь агниц из [стада] мелкого скота особо.
\end{tcolorbox}
\begin{tcolorbox}
\textsubscript{29} Авимелех же сказал Аврааму: на что здесь сии семь агниц, которых ты поставил особо?
\end{tcolorbox}
\begin{tcolorbox}
\textsubscript{30} [он] сказал: семь агниц сих возьми от руки моей, чтобы они были мне свидетельством, что я выкопал этот колодезь.
\end{tcolorbox}
\begin{tcolorbox}
\textsubscript{31} Потому и назвал он сие место: Вирсавия, ибо тут оба они клялись
\end{tcolorbox}
\begin{tcolorbox}
\textsubscript{32} и заключили союз в Вирсавии. И встал Авимелех, и Фихол, военачальник его, и возвратились в землю Филистимскую.
\end{tcolorbox}
\begin{tcolorbox}
\textsubscript{33} И насадил [Авраам] при Вирсавии рощу и призвал там имя Господа, Бога вечного.
\end{tcolorbox}
\begin{tcolorbox}
\textsubscript{34} И жил Авраам в земле Филистимской, как странник, дни многие.
\end{tcolorbox}
\subsection{CHAPTER 22}
\begin{tcolorbox}
\textsubscript{1} И было, после сих происшествий Бог искушал Авраама и сказал ему: Авраам! Он сказал: вот я.
\end{tcolorbox}
\begin{tcolorbox}
\textsubscript{2} [Бог] сказал: возьми сына твоего, единственного твоего, которого ты любишь, Исаака; и пойди в землю Мориа и там принеси его во всесожжение на одной из гор, о которой Я скажу тебе.
\end{tcolorbox}
\begin{tcolorbox}
\textsubscript{3} Авраам встал рано утром, оседлал осла своего, взял с собою двоих из отроков своих и Исаака, сына своего; наколол дров для всесожжения, и встав пошел на место, о котором сказал ему Бог.
\end{tcolorbox}
\begin{tcolorbox}
\textsubscript{4} На третий день Авраам возвел очи свои, и увидел то место издалека.
\end{tcolorbox}
\begin{tcolorbox}
\textsubscript{5} И сказал Авраам отрокам своим: останьтесь вы здесь с ослом, а я и сын пойдем туда и поклонимся, и возвратимся к вам.
\end{tcolorbox}
\begin{tcolorbox}
\textsubscript{6} И взял Авраам дрова для всесожжения, и возложил на Исаака, сына своего; взял в руки огонь и нож, и пошли оба вместе.
\end{tcolorbox}
\begin{tcolorbox}
\textsubscript{7} И начал Исаак говорить Аврааму, отцу своему, и сказал: отец мой! Он отвечал: вот я, сын мой. Он сказал: вот огонь и дрова, где же агнец для всесожжения?
\end{tcolorbox}
\begin{tcolorbox}
\textsubscript{8} Авраам сказал: Бог усмотрит Себе агнца для всесожжения, сын мой. И шли [далее] оба вместе.
\end{tcolorbox}
\begin{tcolorbox}
\textsubscript{9} И пришли на место, о котором сказал ему Бог; и устроил там Авраам жертвенник, разложил дрова и, связав сына своего Исаака, положил его на жертвенник поверх дров.
\end{tcolorbox}
\begin{tcolorbox}
\textsubscript{10} И простер Авраам руку свою и взял нож, чтобы заколоть сына своего.
\end{tcolorbox}
\begin{tcolorbox}
\textsubscript{11} Но Ангел Господень воззвал к нему с неба и сказал: Авраам! Авраам! Он сказал: вот я.
\end{tcolorbox}
\begin{tcolorbox}
\textsubscript{12} [Ангел] сказал: не поднимай руки твоей на отрока и не делай над ним ничего, ибо теперь Я знаю, что боишься ты Бога и не пожалел сына твоего, единственного твоего, для Меня.
\end{tcolorbox}
\begin{tcolorbox}
\textsubscript{13} И возвел Авраам очи свои и увидел: и вот, позади овен, запутавшийся в чаще рогами своими. Авраам пошел, взял овна и принес его во всесожжение вместо сына своего.
\end{tcolorbox}
\begin{tcolorbox}
\textsubscript{14} И нарек Авраам имя месту тому: Иегова-ире. Посему [и] ныне говорится: на горе Иеговы усмотрится.
\end{tcolorbox}
\begin{tcolorbox}
\textsubscript{15} И вторично воззвал к Аврааму Ангел Господень с неба
\end{tcolorbox}
\begin{tcolorbox}
\textsubscript{16} и сказал: Мною клянусь, говорит Господь, что, так как ты сделал сие дело, и не пожалел сына твоего, единственного твоего,
\end{tcolorbox}
\begin{tcolorbox}
\textsubscript{17} то Я благословляя благословлю тебя и умножая умножу семя твое, как звезды небесные и как песок на берегу моря; и овладеет семя твое городами врагов своих;
\end{tcolorbox}
\begin{tcolorbox}
\textsubscript{18} и благословятся в семени твоем все народы земли за то, что ты послушался гласа Моего.
\end{tcolorbox}
\begin{tcolorbox}
\textsubscript{19} И возвратился Авраам к отрокам своим, и встали и пошли вместе в Вирсавию; и жил Авраам в Вирсавии.
\end{tcolorbox}
\begin{tcolorbox}
\textsubscript{20} После сих происшествий Аврааму возвестили, сказав: вот, и Милка родила Нахору, брату твоему, сынов:
\end{tcolorbox}
\begin{tcolorbox}
\textsubscript{21} Уца, первенца его, Вуза, брата сему, Кемуила, отца Арамова,
\end{tcolorbox}
\begin{tcolorbox}
\textsubscript{22} Кеседа, Хазо, Пилдаша, Идлафа и Вафуила;
\end{tcolorbox}
\begin{tcolorbox}
\textsubscript{23} от Вафуила родилась Ревекка. Восьмерых сих родила Милка Нахору, брату Авраамову;
\end{tcolorbox}
\begin{tcolorbox}
\textsubscript{24} и наложница его, именем Реума, также родила Теваха, Гахама, Тахаша и Мааху.
\end{tcolorbox}
\subsection{CHAPTER 23}
\begin{tcolorbox}
\textsubscript{1} Жизни Сарриной было сто двадцать семь лет: [вот] лета жизни Сарриной;
\end{tcolorbox}
\begin{tcolorbox}
\textsubscript{2} и умерла Сарра в Кириаф-Арбе, что [ныне] Хеврон, в земле Ханаанской. И пришел Авраам рыдать по Сарре и оплакивать ее.
\end{tcolorbox}
\begin{tcolorbox}
\textsubscript{3} И отошел Авраам от умершей своей, и говорил сынам Хетовым, и сказал:
\end{tcolorbox}
\begin{tcolorbox}
\textsubscript{4} я у вас пришлец и поселенец; дайте мне в собственность [место] [для] гроба между вами, чтобы мне умершую мою схоронить от глаз моих.
\end{tcolorbox}
\begin{tcolorbox}
\textsubscript{5} Сыны Хета отвечали Аврааму и сказали ему:
\end{tcolorbox}
\begin{tcolorbox}
\textsubscript{6} послушай нас, господин наш; ты князь Божий посреди нас; в лучшем из наших погребальных мест похорони умершую твою; никто из нас не откажет тебе в погребальном месте, для погребения умершей твоей.
\end{tcolorbox}
\begin{tcolorbox}
\textsubscript{7} Авраам встал и поклонился народу земли той, сынам Хетовым;
\end{tcolorbox}
\begin{tcolorbox}
\textsubscript{8} и говорил им и сказал: если вы согласны, чтобы я похоронил умершую мою, то послушайте меня, попросите за меня Ефрона, сына Цохарова,
\end{tcolorbox}
\begin{tcolorbox}
\textsubscript{9} чтобы он отдал мне пещеру Махпелу, которая у него на конце поля его, чтобы за довольную цену отдал ее мне посреди вас, в собственность для погребения.
\end{tcolorbox}
\begin{tcolorbox}
\textsubscript{10} Ефрон же сидел посреди сынов Хетовых; и отвечал Ефрон Хеттеянин Аврааму вслух сынов Хета, всех входящих во врата города его, и сказал:
\end{tcolorbox}
\begin{tcolorbox}
\textsubscript{11} нет, господин мой, послушай меня: я даю тебе поле и пещеру, которая на нем, даю тебе, пред очами сынов народа моего дарю тебе ее, похорони умершую твою.
\end{tcolorbox}
\begin{tcolorbox}
\textsubscript{12} Авраам поклонился пред народом земли той
\end{tcolorbox}
\begin{tcolorbox}
\textsubscript{13} и говорил Ефрону вслух народа земли той и сказал: если послушаешь, я даю тебе за поле серебро; возьми у меня, и я похороню там умершую мою.
\end{tcolorbox}
\begin{tcolorbox}
\textsubscript{14} Ефрон отвечал Аврааму и сказал ему:
\end{tcolorbox}
\begin{tcolorbox}
\textsubscript{15} господин мой! послушай меня: земля [стоит] четыреста сиклей серебра; для меня и для тебя что это? похорони умершую твою.
\end{tcolorbox}
\begin{tcolorbox}
\textsubscript{16} Авраам выслушал Ефрона; и отвесил Авраам Ефрону серебра, сколько он объявил вслух сынов Хетовых, четыреста сиклей серебра, какое ходит у купцов.
\end{tcolorbox}
\begin{tcolorbox}
\textsubscript{17} И стало поле Ефроново, которое при Махпеле, против Мамре, поле и пещера, которая на нем, и все деревья, которые на поле, во всех пределах его вокруг,
\end{tcolorbox}
\begin{tcolorbox}
\textsubscript{18} владением Авраамовым пред очами сынов Хета, всех входящих во врата города его.
\end{tcolorbox}
\begin{tcolorbox}
\textsubscript{19} После сего Авраам похоронил Сарру, жену свою, в пещере поля в Махпеле, против Мамре, что [ныне] Хеврон, в земле Ханаанской.
\end{tcolorbox}
\begin{tcolorbox}
\textsubscript{20} Так достались Аврааму от сынов Хетовых поле и пещера, которая на нем, в собственность для погребения.
\end{tcolorbox}
\subsection{CHAPTER 24}
\begin{tcolorbox}
\textsubscript{1} Авраам был уже стар и в летах преклонных. Господь благословил Авраама всем.
\end{tcolorbox}
\begin{tcolorbox}
\textsubscript{2} И сказал Авраам рабу своему, старшему в доме его, управлявшему всем, что у него было: положи руку твою под стегно мое
\end{tcolorbox}
\begin{tcolorbox}
\textsubscript{3} и клянись мне Господом, Богом неба и Богом земли, что ты не возьмешь сыну моему жены из дочерей Хананеев, среди которых я живу,
\end{tcolorbox}
\begin{tcolorbox}
\textsubscript{4} но пойдешь в землю мою, на родину мою, и возьмешь жену сыну моему Исааку.
\end{tcolorbox}
\begin{tcolorbox}
\textsubscript{5} Раб сказал ему: может быть, не захочет женщина идти со мною в эту землю, должен ли я возвратить сына твоего в землю, из которой ты вышел?
\end{tcolorbox}
\begin{tcolorbox}
\textsubscript{6} Авраам сказал ему: берегись, не возвращай сына моего туда;
\end{tcolorbox}
\begin{tcolorbox}
\textsubscript{7} Господь, Бог неба, Который взял меня из дома отца моего и из земли рождения моего, Который говорил мне и Который клялся мне, говоря: 'потомству твоему дам сию землю', --Он пошлет Ангела Своего пред тобою, и ты возьмешь жену сыну моему оттуда;
\end{tcolorbox}
\begin{tcolorbox}
\textsubscript{8} если же не захочет женщина идти с тобою, ты будешь свободен от сей клятвы моей; только сына моего не возвращай туда.
\end{tcolorbox}
\begin{tcolorbox}
\textsubscript{9} И положил раб руку свою под стегно Авраама, господина своего, и клялся ему в сем.
\end{tcolorbox}
\begin{tcolorbox}
\textsubscript{10} И взял раб из верблюдов господина своего десять верблюдов и пошел. В руках у него были также всякие сокровища господина его. Он встал и пошел в Месопотамию, в город Нахора,
\end{tcolorbox}
\begin{tcolorbox}
\textsubscript{11} и остановил верблюдов вне города, у колодезя воды, под вечер, в то время, когда выходят женщины черпать,
\end{tcolorbox}
\begin{tcolorbox}
\textsubscript{12} и сказал: Господи, Боже господина моего Авраама! пошли [ее] сегодня навстречу мне и сотвори милость с господином моим Авраамом;
\end{tcolorbox}
\begin{tcolorbox}
\textsubscript{13} вот, я стою у источника воды, и дочери жителей города выходят черпать воду;
\end{tcolorbox}
\begin{tcolorbox}
\textsubscript{14} и девица, которой я скажу: 'наклони кувшин твой, я напьюсь', и которая скажет: 'пей, я и верблюдам твоим дам пить', --вот та, которую Ты назначил рабу Твоему Исааку; и по сему узнаю я, что Ты творишь милость с господином моим.
\end{tcolorbox}
\begin{tcolorbox}
\textsubscript{15} Еще не перестал он говорить, и вот, вышла Ревекка, которая родилась от Вафуила, сына Милки, жены Нахора, брата Авраамова, и кувшин ее на плече ее;
\end{tcolorbox}
\begin{tcolorbox}
\textsubscript{16} девица [была] прекрасна видом, дева, которой не познал муж. Она сошла к источнику, наполнила кувшин свой и пошла вверх.
\end{tcolorbox}
\begin{tcolorbox}
\textsubscript{17} И побежал раб навстречу ей и сказал: дай мне испить немного воды из кувшина твоего.
\end{tcolorbox}
\begin{tcolorbox}
\textsubscript{18} Она сказала: пей, господин мой. И тотчас спустила кувшин свой на руку свою и напоила его.
\end{tcolorbox}
\begin{tcolorbox}
\textsubscript{19} И, когда напоила его, сказала: я стану черпать и для верблюдов твоих, пока не напьются.
\end{tcolorbox}
\begin{tcolorbox}
\textsubscript{20} И тотчас вылила воду из кувшина своего в поило и побежала опять к колодезю почерпнуть, и начерпала для всех верблюдов его.
\end{tcolorbox}
\begin{tcolorbox}
\textsubscript{21} Человек тот смотрел на нее с изумлением в молчании, желая уразуметь, благословил ли Господь путь его, или нет.
\end{tcolorbox}
\begin{tcolorbox}
\textsubscript{22} Когда верблюды перестали пить, тогда человек тот взял золотую серьгу, весом полсикля, и два запястья на руки ей, весом в десять [сиклей] золота;
\end{tcolorbox}
\begin{tcolorbox}
\textsubscript{23} И сказал: чья ты дочь? скажи мне, есть ли в доме отца твоего место нам ночевать?
\end{tcolorbox}
\begin{tcolorbox}
\textsubscript{24} Она сказала ему: я дочь Вафуила, сына Милки, которого она родила Нахору.
\end{tcolorbox}
\begin{tcolorbox}
\textsubscript{25} И еще сказала ему: у нас много соломы и корму, и [есть] место для ночлега.
\end{tcolorbox}
\begin{tcolorbox}
\textsubscript{26} И преклонился человек тот и поклонился Господу,
\end{tcolorbox}
\begin{tcolorbox}
\textsubscript{27} и сказал: благословен Господь Бог господина моего Авраама, Который не оставил господина моего милостью Своею и истиною Своею! Господь прямым путем привел меня к дому брата господина моего.
\end{tcolorbox}
\begin{tcolorbox}
\textsubscript{28} Девица побежала и рассказала об этом в доме матери своей.
\end{tcolorbox}
\begin{tcolorbox}
\textsubscript{29} У Ревекки был брат, именем Лаван. Лаван выбежал к тому человеку, к источнику.
\end{tcolorbox}
\begin{tcolorbox}
\textsubscript{30} И когда он увидел серьгу и запястья на руках у сестры своей и услышал слова Ревекки, сестры своей, которая говорила: так говорил со мною этот человек, --то пришел к человеку, и вот, он стоит при верблюдах у источника;
\end{tcolorbox}
\begin{tcolorbox}
\textsubscript{31} и сказал: войди, благословенный Господом; зачем ты стоишь вне? я приготовил дом и место для верблюдов.
\end{tcolorbox}
\begin{tcolorbox}
\textsubscript{32} И вошел человек. [Лаван] расседлал верблюдов и дал соломы и корму верблюдам, и воды умыть ноги ему и людям, которые были с ним;
\end{tcolorbox}
\begin{tcolorbox}
\textsubscript{33} и предложена была ему пища; но он сказал: не стану есть, доколе не скажу дела своего. И сказали: говори.
\end{tcolorbox}
\begin{tcolorbox}
\textsubscript{34} Он сказал: я раб Авраамов;
\end{tcolorbox}
\begin{tcolorbox}
\textsubscript{35} Господь весьма благословил господина моего, и он сделался великим: Он дал ему овец и волов, серебро и золото, рабов и рабынь, верблюдов и ослов;
\end{tcolorbox}
\begin{tcolorbox}
\textsubscript{36} Сарра, жена господина моего, уже состарившись, родила господину моему сына, которому он отдал все, что у него;
\end{tcolorbox}
\begin{tcolorbox}
\textsubscript{37} и взял с меня клятву господин мой, сказав: не бери жены сыну моему из дочерей Хананеев, в земле которых я живу,
\end{tcolorbox}
\begin{tcolorbox}
\textsubscript{38} а пойди в дом отца моего и к родственникам моим, и возьмешь жену сыну моему.
\end{tcolorbox}
\begin{tcolorbox}
\textsubscript{39} Я сказал господину моему: может быть, не пойдет женщина со мною.
\end{tcolorbox}
\begin{tcolorbox}
\textsubscript{40} Он сказал мне: Господь, пред лицем Которого я хожу, пошлет с тобою Ангела Своего и благоустроит путь твой, и возьмешь жену сыну моему из родных моих и из дома отца моего;
\end{tcolorbox}
\begin{tcolorbox}
\textsubscript{41} тогда будешь ты свободен от клятвы моей, когда сходишь к родственникам моим; и если они не дадут тебе, то будешь свободен от клятвы моей.
\end{tcolorbox}
\begin{tcolorbox}
\textsubscript{42} И пришел я ныне к источнику, и сказал: Господи, Боже господина моего Авраама! Если Ты благоустроишь путь, который я совершаю,
\end{tcolorbox}
\begin{tcolorbox}
\textsubscript{43} то вот, я стою у источника воды, и девица, которая выйдет почерпать, и которой я скажу: дай мне испить немного из кувшина твоего,
\end{tcolorbox}
\begin{tcolorbox}
\textsubscript{44} и которая скажет мне: 'и ты пей, и верблюдам твоим я начерпаю' --вот жена, которую Господь назначил сыну господина моего.
\end{tcolorbox}
\begin{tcolorbox}
\textsubscript{45} Еще не перестал я говорить в уме моем, и вот вышла Ревекка, и кувшин ее на плече ее, и сошла к источнику и почерпнула; и я сказал ей: напой меня.
\end{tcolorbox}
\begin{tcolorbox}
\textsubscript{46} Она тотчас спустила с себя кувшин свой и сказала: пей, и верблюдов твоих я напою. И я пил, и верблюдов она напоила.
\end{tcolorbox}
\begin{tcolorbox}
\textsubscript{47} Я спросил ее и сказал: чья ты дочь? Она сказала: дочь Вафуила, сына Нахорова, которого родила ему Милка. И дал я серьги ей и запястья на руки ее.
\end{tcolorbox}
\begin{tcolorbox}
\textsubscript{48} И преклонился я и поклонился Господу, и благословил Господа, Бога господина моего Авраама, Который прямым путем привел меня, чтобы взять дочь брата господина моего за сына его.
\end{tcolorbox}
\begin{tcolorbox}
\textsubscript{49} И ныне скажите мне: намерены ли вы оказать милость и правду господину моему или нет? скажите мне, и я обращусь направо, или налево.
\end{tcolorbox}
\begin{tcolorbox}
\textsubscript{50} И отвечали Лаван и Вафуил и сказали: от Господа пришло это дело; мы не можем сказать тебе вопреки ни худого, ни доброго;
\end{tcolorbox}
\begin{tcolorbox}
\textsubscript{51} вот Ревекка пред тобою; возьми и пойди; пусть будет она женою сыну господина твоего, как сказал Господь.
\end{tcolorbox}
\begin{tcolorbox}
\textsubscript{52} Когда раб Авраамов услышал слова их, то поклонился Господу до земли.
\end{tcolorbox}
\begin{tcolorbox}
\textsubscript{53} И вынул раб серебряные вещи и золотые вещи и одежды и дал Ревекке; также и брату ее и матери ее дал богатые подарки.
\end{tcolorbox}
\begin{tcolorbox}
\textsubscript{54} И ели и пили он и люди, бывшие с ним, и переночевали. Когда же встали поутру, то он сказал: отпустите меня к господину моему.
\end{tcolorbox}
\begin{tcolorbox}
\textsubscript{55} Но брат ее и мать ее сказали: пусть побудет с нами девица дней хотя десять, потом пойдешь.
\end{tcolorbox}
\begin{tcolorbox}
\textsubscript{56} Он сказал им: не удерживайте меня, ибо Господь благоустроил путь мой; отпустите меня, и я пойду к господину моему.
\end{tcolorbox}
\begin{tcolorbox}
\textsubscript{57} Они сказали: призовем девицу и спросим, что она скажет.
\end{tcolorbox}
\begin{tcolorbox}
\textsubscript{58} И призвали Ревекку и сказали ей: пойдешь ли с этим человеком? Она сказала: пойду.
\end{tcolorbox}
\begin{tcolorbox}
\textsubscript{59} И отпустили Ревекку, сестру свою, и кормилицу ее, и раба Авраамова, и людей его.
\end{tcolorbox}
\begin{tcolorbox}
\textsubscript{60} И благословили Ревекку и сказали ей: сестра наша! да родятся от тебя тысячи тысяч, и да владеет потомство твое жилищами врагов твоих!
\end{tcolorbox}
\begin{tcolorbox}
\textsubscript{61} И встала Ревекка и служанки ее, и сели на верблюдов, и поехали за тем человеком. И раб взял Ревекку и пошел.
\end{tcolorbox}
\begin{tcolorbox}
\textsubscript{62} А Исаак пришел из Беэр-лахай-рои, ибо жил он в земле полуденной.
\end{tcolorbox}
\begin{tcolorbox}
\textsubscript{63} При наступлении вечера Исаак вышел в поле поразмыслить, и возвел очи свои, и увидел: вот, идут верблюды.
\end{tcolorbox}
\begin{tcolorbox}
\textsubscript{64} Ревекка взглянула, и увидела Исаака, и спустилась с верблюда.
\end{tcolorbox}
\begin{tcolorbox}
\textsubscript{65} И сказала рабу: кто этот человек, который идет по полю навстречу нам? Раб сказал: это господин мой. И она взяла покрывало и покрылась.
\end{tcolorbox}
\begin{tcolorbox}
\textsubscript{66} Раб же сказал Исааку все, что сделал.
\end{tcolorbox}
\begin{tcolorbox}
\textsubscript{67} И ввел ее Исаак в шатер Сарры, матери своей, и взял Ревекку, и она сделалась ему женою, и он возлюбил ее; и утешился Исаак в [печали] по матери своей.
\end{tcolorbox}
\subsection{CHAPTER 25}
\begin{tcolorbox}
\textsubscript{1} И взял Авраам еще жену, именем Хеттуру.
\end{tcolorbox}
\begin{tcolorbox}
\textsubscript{2} Она родила ему Зимрана, Иокшана, Медана, Мадиана, Ишбака и Шуаха.
\end{tcolorbox}
\begin{tcolorbox}
\textsubscript{3} Иокшан родил Шеву и Дедана. Сыны Дедана были: Ашурим, Летушим и Леюмим.
\end{tcolorbox}
\begin{tcolorbox}
\textsubscript{4} Сыны Мадиана: Ефа, Ефер, Ханох, Авида и Елдага. Все сии сыны Хеттуры.
\end{tcolorbox}
\begin{tcolorbox}
\textsubscript{5} И отдал Авраам все, что было у него, Исааку,
\end{tcolorbox}
\begin{tcolorbox}
\textsubscript{6} а сынам наложниц, которые были у Авраама, дал Авраам подарки и отослал их от Исаака, сына своего, еще при жизни своей, на восток, в землю восточную.
\end{tcolorbox}
\begin{tcolorbox}
\textsubscript{7} Дней жизни Авраамовой, которые он прожил, было сто семьдесят пять лет;
\end{tcolorbox}
\begin{tcolorbox}
\textsubscript{8} и скончался Авраам, и умер в старости доброй, престарелый и насыщенный [жизнью], и приложился к народу своему.
\end{tcolorbox}
\begin{tcolorbox}
\textsubscript{9} И погребли его Исаак и Измаил, сыновья его, в пещере Махпеле, на поле Ефрона, сына Цохара, Хеттеянина, которое против Мамре,
\end{tcolorbox}
\begin{tcolorbox}
\textsubscript{10} на поле, которые Авраам приобрел от сынов Хетовых. Там погребены Авраам и Сарра, жена его.
\end{tcolorbox}
\begin{tcolorbox}
\textsubscript{11} По смерти Авраама Бог благословил Исаака, сына его. Исаак жил при Беэр-лахай-рои.
\end{tcolorbox}
\begin{tcolorbox}
\textsubscript{12} Вот родословие Измаила, сына Авраамова, которого родила Аврааму Агарь Египтянка, служанка Саррина;
\end{tcolorbox}
\begin{tcolorbox}
\textsubscript{13} и вот имена сынов Измаиловых, имена их по родословию их: первенец Измаилов Наваиоф, [за ним] Кедар, Адбеел, Мивсам,
\end{tcolorbox}
\begin{tcolorbox}
\textsubscript{14} Мишма, Дума, Масса,
\end{tcolorbox}
\begin{tcolorbox}
\textsubscript{15} Хадад, Фема, Иетур, Нафиш и Кедма.
\end{tcolorbox}
\begin{tcolorbox}
\textsubscript{16} Сии суть сыны Измаиловы, и сии имена их, в селениях их, в кочевьях их. [Это] двенадцать князей племен их.
\end{tcolorbox}
\begin{tcolorbox}
\textsubscript{17} Лет же жизни Измаиловой было сто тридцать семь лет; и скончался он, и умер, и приложился к народу своему.
\end{tcolorbox}
\begin{tcolorbox}
\textsubscript{18} Они жили от Хавилы до Сура, что пред Египтом, как идешь к Ассирии. Они поселились пред лицем всех братьев своих.
\end{tcolorbox}
\begin{tcolorbox}
\textsubscript{19} Вот родословие Исаака, сына Авраамова. Авраам родил Исаака.
\end{tcolorbox}
\begin{tcolorbox}
\textsubscript{20} Исаак был сорока лет, когда он взял себе в жену Ревекку, дочь Вафуила Арамеянина из Месопотамии, сестру Лавана Арамеянина.
\end{tcolorbox}
\begin{tcolorbox}
\textsubscript{21} И молился Исаак Господу о жене своей, потому что она была неплодна; и Господь услышал его, и зачала Ревекка, жена его.
\end{tcolorbox}
\begin{tcolorbox}
\textsubscript{22} Сыновья в утробе ее стали биться, и она сказала: если так будет, то для чего мне это? И пошла вопросить Господа.
\end{tcolorbox}
\begin{tcolorbox}
\textsubscript{23} Господь сказал ей: два племени во чреве твоем, и два различных народа произойдут из утробы твоей; один народ сделается сильнее другого, и больший будет служить меньшему.
\end{tcolorbox}
\begin{tcolorbox}
\textsubscript{24} И настало время родить ей: и вот близнецы в утробе ее.
\end{tcolorbox}
\begin{tcolorbox}
\textsubscript{25} Первый вышел красный, весь, как кожа, косматый; и нарекли ему имя Исав.
\end{tcolorbox}
\begin{tcolorbox}
\textsubscript{26} Потом вышел брат его, держась рукою своею за пяту Исава; и наречено ему имя Иаков. Исаак же был шестидесяти лет, когда они родились.
\end{tcolorbox}
\begin{tcolorbox}
\textsubscript{27} Дети выросли, и стал Исав человеком искусным в звероловстве, человеком полей; а Иаков человеком кротким, живущим в шатрах.
\end{tcolorbox}
\begin{tcolorbox}
\textsubscript{28} Исаак любил Исава, потому что дичь его была по вкусу его, а Ревекка любила Иакова.
\end{tcolorbox}
\begin{tcolorbox}
\textsubscript{29} И сварил Иаков кушанье; а Исав пришел с поля усталый.
\end{tcolorbox}
\begin{tcolorbox}
\textsubscript{30} И сказал Исав Иакову: дай мне поесть красного, красного этого, ибо я устал. От сего дано ему прозвание: Едом.
\end{tcolorbox}
\begin{tcolorbox}
\textsubscript{31} Но Иаков сказал: продай мне теперь же свое первородство.
\end{tcolorbox}
\begin{tcolorbox}
\textsubscript{32} Исав сказал: вот, я умираю, что мне в этом первородстве?
\end{tcolorbox}
\begin{tcolorbox}
\textsubscript{33} Иаков сказал: поклянись мне теперь же. Он поклялся ему, и продал первородство свое Иакову.
\end{tcolorbox}
\begin{tcolorbox}
\textsubscript{34} И дал Иаков Исаву хлеба и кушанья из чечевицы; и он ел и пил, и встал и пошел; и пренебрег Исав первородство.
\end{tcolorbox}
\subsection{CHAPTER 26}
\begin{tcolorbox}
\textsubscript{1} Был голод в земле, сверх прежнего голода, который был во дни Авраама; и пошел Исаак к Авимелеху, царю Филистимскому, в Герар.
\end{tcolorbox}
\begin{tcolorbox}
\textsubscript{2} Господь явился ему и сказал: не ходи в Египет; живи в земле, о которой Я скажу тебе,
\end{tcolorbox}
\begin{tcolorbox}
\textsubscript{3} странствуй по сей земле, и Я буду с тобою и благословлю тебя, ибо тебе и потомству твоему дам все земли сии и исполню клятву, которою Я клялся Аврааму, отцу твоему;
\end{tcolorbox}
\begin{tcolorbox}
\textsubscript{4} умножу потомство твое, как звезды небесные, и дам потомству твоему все земли сии; благословятся в семени твоем все народы земные,
\end{tcolorbox}
\begin{tcolorbox}
\textsubscript{5} за то, что Авраам послушался гласа Моего и соблюдал, что Мною [заповедано] было соблюдать: повеления Мои, уставы Мои и законы Мои.
\end{tcolorbox}
\begin{tcolorbox}
\textsubscript{6} Исаак поселился в Гераре.
\end{tcolorbox}
\begin{tcolorbox}
\textsubscript{7} Жители места того спросили о жене его, и он сказал: это сестра моя; потому что боялся сказать: жена моя, чтобы не убили меня, [думал он], жители места сего за Ревекку, потому что она прекрасна видом.
\end{tcolorbox}
\begin{tcolorbox}
\textsubscript{8} Но когда уже много времени он там прожил, Авимелех, царь Филистимский, посмотрев в окно, увидел, что Исаак играет с Ревеккою, женою своею.
\end{tcolorbox}
\begin{tcolorbox}
\textsubscript{9} И призвал Авимелех Исаака и сказал: вот, это жена твоя; как же ты сказал: она сестра моя? Исаак сказал ему: потому что я думал, не умереть бы мне ради ее.
\end{tcolorbox}
\begin{tcolorbox}
\textsubscript{10} Но Авимелех сказал: что это ты сделал с нами? едва один из народа не совокупился с женою твоею, и ты ввел бы нас в грех.
\end{tcolorbox}
\begin{tcolorbox}
\textsubscript{11} И дал Авимелех повеление всему народу, сказав: кто прикоснется к сему человеку и к жене его, тот предан будет смерти.
\end{tcolorbox}
\begin{tcolorbox}
\textsubscript{12} И сеял Исаак в земле той и получил в тот год ячменя во сто крат: так благословил его Господь.
\end{tcolorbox}
\begin{tcolorbox}
\textsubscript{13} И стал великим человек сей и возвеличивался больше и больше до того, что стал весьма великим.
\end{tcolorbox}
\begin{tcolorbox}
\textsubscript{14} У него были стада мелкого и стада крупного скота и множество пахотных полей, и Филистимляне стали завидовать ему.
\end{tcolorbox}
\begin{tcolorbox}
\textsubscript{15} И все колодези, которые выкопали рабы отца его при жизни отца его Авраама, Филистимляне завалили и засыпали землею.
\end{tcolorbox}
\begin{tcolorbox}
\textsubscript{16} И Авимелех сказал Исааку: удались от нас, ибо ты сделался гораздо сильнее нас.
\end{tcolorbox}
\begin{tcolorbox}
\textsubscript{17} И Исаак удалился оттуда, и расположился шатрами в долине Герарской, и поселился там.
\end{tcolorbox}
\begin{tcolorbox}
\textsubscript{18} И вновь выкопал Исаак колодези воды, которые выкопаны были во дни Авраама, отца его, и которые завалили Филистимляне по смерти Авраама; и назвал их теми же именами, которыми назвал их отец его.
\end{tcolorbox}
\begin{tcolorbox}
\textsubscript{19} И копали рабы Исааковы в долине и нашли там колодезь воды живой.
\end{tcolorbox}
\begin{tcolorbox}
\textsubscript{20} И спорили пастухи Герарские с пастухами Исаака, говоря: наша вода. И он нарек колодезю имя: Есек, потому что спорили с ним.
\end{tcolorbox}
\begin{tcolorbox}
\textsubscript{21} выкопали другой колодезь; спорили также и о нем; и он нарек ему имя: Ситна.
\end{tcolorbox}
\begin{tcolorbox}
\textsubscript{22} И он двинулся отсюда и выкопал иной колодезь, о котором уже не спорили, и нарек ему имя: Реховоф, ибо, сказал он, теперь Господь дал нам пространное место, и мы размножимся на земле.
\end{tcolorbox}
\begin{tcolorbox}
\textsubscript{23} Оттуда перешел он в Вирсавию.
\end{tcolorbox}
\begin{tcolorbox}
\textsubscript{24} И в ту ночь явился ему Господь и сказал: Я Бог Авраама, отца твоего; не бойся, ибо Я с тобою; и благословлю тебя и умножу потомство твое, ради Авраама, раба Моего.
\end{tcolorbox}
\begin{tcolorbox}
\textsubscript{25} И он устроил там жертвенник и призвал имя Господа. И раскинул там шатер свой, и выкопали там рабы Исааковы колодезь.
\end{tcolorbox}
\begin{tcolorbox}
\textsubscript{26} Пришел к нему из Герара Авимелех и Ахузаф, друг его, и Фихол, военачальник его.
\end{tcolorbox}
\begin{tcolorbox}
\textsubscript{27} Исаак сказал им: для чего вы пришли ко мне, когда вы возненавидели меня и выслали меня от себя?
\end{tcolorbox}
\begin{tcolorbox}
\textsubscript{28} Они сказали: мы ясно увидели, что Господь с тобою, и потому мы сказали: поставим между нами и тобою клятву и заключим с тобою союз,
\end{tcolorbox}
\begin{tcolorbox}
\textsubscript{29} чтобы ты не делал нам зла, как и мы не коснулись до тебя, а делали тебе одно доброе и отпустили тебя с миром; теперь ты благословен Господом.
\end{tcolorbox}
\begin{tcolorbox}
\textsubscript{30} Он сделал им пиршество, и они ели и пили.
\end{tcolorbox}
\begin{tcolorbox}
\textsubscript{31} И встав рано утром, поклялись друг другу; и отпустил их Исаак, и они пошли от него с миром.
\end{tcolorbox}
\begin{tcolorbox}
\textsubscript{32} В тот же день пришли рабы Исааковы и известили его о колодезе, который копали они, и сказали ему: мы нашли воду.
\end{tcolorbox}
\begin{tcolorbox}
\textsubscript{33} И он назвал его: Шива. Посему имя городу тому Беэршива до сего дня.
\end{tcolorbox}
\begin{tcolorbox}
\textsubscript{34} И был Исав сорока лет, и взял себе в жены Иегудифу, дочь Беэра Хеттеянина, и Васемафу, дочь Елона Хеттеянина;
\end{tcolorbox}
\begin{tcolorbox}
\textsubscript{35} и они были в тягость Исааку и Ревекке.
\end{tcolorbox}
\subsection{CHAPTER 27}
\begin{tcolorbox}
\textsubscript{1} Когда Исаак состарился и притупилось зрение глаз его, он призвал старшего сына своего Исава и сказал ему: сын мой! Тот сказал ему: вот я.
\end{tcolorbox}
\begin{tcolorbox}
\textsubscript{2} Он сказал: вот, я состарился; не знаю дня смерти моей;
\end{tcolorbox}
\begin{tcolorbox}
\textsubscript{3} возьми теперь орудия твои, колчан твой и лук твой, пойди в поле, и налови мне дичи,
\end{tcolorbox}
\begin{tcolorbox}
\textsubscript{4} и приготовь мне кушанье, какое я люблю, и принеси мне есть, чтобы благословила тебя душа моя, прежде нежели я умру.
\end{tcolorbox}
\begin{tcolorbox}
\textsubscript{5} Ревекка слышала, когда Исаак говорил сыну своему Исаву. И пошел Исав в поле достать и принести дичи;
\end{tcolorbox}
\begin{tcolorbox}
\textsubscript{6} а Ревекка сказала сыну своему Иакову: вот, я слышала, как отец твой говорил брату твоему Исаву:
\end{tcolorbox}
\begin{tcolorbox}
\textsubscript{7} принеси мне дичи и приготовь мне кушанье; я поем и благословлю тебя пред лицем Господним, пред смертью моею.
\end{tcolorbox}
\begin{tcolorbox}
\textsubscript{8} Теперь, сын мой, послушайся слов моих в том, что я прикажу тебе:
\end{tcolorbox}
\begin{tcolorbox}
\textsubscript{9} пойди в [стадо] и возьми мне оттуда два козленка хороших, и я приготовлю из них отцу твоему кушанье, какое он любит,
\end{tcolorbox}
\begin{tcolorbox}
\textsubscript{10} а ты принесешь отцу твоему, и он поест, чтобы благословить тебя пред смертью своею.
\end{tcolorbox}
\begin{tcolorbox}
\textsubscript{11} Иаков сказал Ревекке, матери своей: Исав, брат мой, человек косматый, а я человек гладкий;
\end{tcolorbox}
\begin{tcolorbox}
\textsubscript{12} может статься, ощупает меня отец мой, и я буду в глазах его обманщиком и наведу на себя проклятие, а не благословение.
\end{tcolorbox}
\begin{tcolorbox}
\textsubscript{13} Мать его сказала ему: на мне пусть будет проклятие твое, сын мой, только послушайся слов моих и пойди, принеси мне.
\end{tcolorbox}
\begin{tcolorbox}
\textsubscript{14} Он пошел, и взял, и принес матери своей; и мать его сделала кушанье, какое любил отец его.
\end{tcolorbox}
\begin{tcolorbox}
\textsubscript{15} И взяла Ревекка богатую одежду старшего сына своего Исава, бывшую у ней в доме, и одела [в нее] младшего сына своего Иакова;
\end{tcolorbox}
\begin{tcolorbox}
\textsubscript{16} а руки его и гладкую шею его обложила кожею козлят;
\end{tcolorbox}
\begin{tcolorbox}
\textsubscript{17} и дала кушанье и хлеб, которые она приготовила, в руки Иакову, сыну своему.
\end{tcolorbox}
\begin{tcolorbox}
\textsubscript{18} Он вошел к отцу своему и сказал: отец мой! Тот сказал: вот я; кто ты, сын мой?
\end{tcolorbox}
\begin{tcolorbox}
\textsubscript{19} Иаков сказал отцу своему: я Исав, первенец твой; я сделал, как ты сказал мне; встань, сядь и поешь дичи моей, чтобы благословила меня душа твоя.
\end{tcolorbox}
\begin{tcolorbox}
\textsubscript{20} И сказал Исаак сыну своему: что так скоро нашел ты, сын мой? Он сказал: потому что Господь Бог твой послал мне навстречу.
\end{tcolorbox}
\begin{tcolorbox}
\textsubscript{21} И сказал Исаак Иакову: подойди, я ощупаю тебя, сын мой, ты ли сын мой Исав, или нет?
\end{tcolorbox}
\begin{tcolorbox}
\textsubscript{22} Иаков подошел к Исааку, отцу своему, и он ощупал его и сказал: голос, голос Иакова; а руки, руки Исавовы.
\end{tcolorbox}
\begin{tcolorbox}
\textsubscript{23} И не узнал его, потому что руки его были, как руки Исава, брата его, косматые; и благословил его
\end{tcolorbox}
\begin{tcolorbox}
\textsubscript{24} и сказал: ты ли сын мой Исав? Он отвечал: я.
\end{tcolorbox}
\begin{tcolorbox}
\textsubscript{25} [Исаак] сказал: подай мне, я поем дичи сына моего, чтобы благословила тебя душа моя. [Иаков] подал ему, и он ел; принес ему и вина, и он пил.
\end{tcolorbox}
\begin{tcolorbox}
\textsubscript{26} Исаак, отец его, сказал ему: подойди, поцелуй меня, сын мой.
\end{tcolorbox}
\begin{tcolorbox}
\textsubscript{27} Он подошел и поцеловал его. И ощутил [Исаак] запах от одежды его и благословил его и сказал: вот, запах от сына моего, как запах от поля, которое благословил Господь;
\end{tcolorbox}
\begin{tcolorbox}
\textsubscript{28} да даст тебе Бог от росы небесной и от тука земли, и множество хлеба и вина;
\end{tcolorbox}
\begin{tcolorbox}
\textsubscript{29} да послужат тебе народы, и да поклонятся тебе племена; будь господином над братьями твоими, и да поклонятся тебе сыны матери твоей; проклинающие тебя--прокляты; благословляющие тебя--благословенны!
\end{tcolorbox}
\begin{tcolorbox}
\textsubscript{30} Как скоро совершил Исаак благословение над Иаковом, и как только вышел Иаков от лица Исаака, отца своего, Исав, брат его, пришел с ловли своей.
\end{tcolorbox}
\begin{tcolorbox}
\textsubscript{31} Приготовил и он кушанье, и принес отцу своему, и сказал отцу своему: встань, отец мой, и поешь дичи сына твоего, чтобы благословила меня душа твоя.
\end{tcolorbox}
\begin{tcolorbox}
\textsubscript{32} Исаак же, отец его, сказал ему: кто ты? Он сказал: я сын твой, первенец твой, Исав.
\end{tcolorbox}
\begin{tcolorbox}
\textsubscript{33} И вострепетал Исаак весьма великим трепетом, и сказал: кто ж это, который достал дичи и принес мне, и я ел от всего, прежде нежели ты пришел, и я благословил его? он и будет благословен.
\end{tcolorbox}
\begin{tcolorbox}
\textsubscript{34} Исав, выслушав слова отца своего, поднял громкий и весьма горький вопль и сказал отцу своему: отец мой! благослови и меня.
\end{tcolorbox}
\begin{tcolorbox}
\textsubscript{35} Но он сказал: брат твой пришел с хитростью и взял благословение твое.
\end{tcolorbox}
\begin{tcolorbox}
\textsubscript{36} И сказал он: не потому ли дано ему имя: Иаков, что он запнул меня уже два раза? Он взял первородство мое, и вот, теперь взял благословение мое. И [еще] сказал: неужели ты не оставил мне благословения?
\end{tcolorbox}
\begin{tcolorbox}
\textsubscript{37} Исаак отвечал Исаву: вот, я поставил его господином над тобою и всех братьев его отдал ему в рабы; одарил его хлебом и вином; что же я сделаю для тебя, сын мой?
\end{tcolorbox}
\begin{tcolorbox}
\textsubscript{38} Но Исав сказал отцу своему: неужели, отец мой, одно у тебя благословение? благослови и меня, отец мой! И возвысил Исав голос свой и заплакал.
\end{tcolorbox}
\begin{tcolorbox}
\textsubscript{39} И отвечал Исаак, отец его, и сказал ему: вот, от тука земли будет обитание твое и от росы небесной свыше;
\end{tcolorbox}
\begin{tcolorbox}
\textsubscript{40} и ты будешь жить мечом твоим и будешь служить брату твоему; будет же [время], когда воспротивишься и свергнешь иго его с выи твоей.
\end{tcolorbox}
\begin{tcolorbox}
\textsubscript{41} И возненавидел Исав Иакова за благословение, которым благословил его отец его; и сказал Исав в сердце своем: приближаются дни плача по отце моем, и я убью Иакова, брата моего.
\end{tcolorbox}
\begin{tcolorbox}
\textsubscript{42} И пересказаны были Ревекке слова Исава, старшего сына ее; и она послала, и призвала младшего сына своего Иакова, и сказала ему: вот, Исав, брат твой, грозит убить тебя;
\end{tcolorbox}
\begin{tcolorbox}
\textsubscript{43} и теперь, сын мой, послушайся слов моих, встань, беги к Лавану, брату моему, в Харран,
\end{tcolorbox}
\begin{tcolorbox}
\textsubscript{44} и поживи у него несколько времени, пока утолится ярость брата твоего,
\end{tcolorbox}
\begin{tcolorbox}
\textsubscript{45} пока утолится гнев брата твоего на тебя, и он позабудет, что ты сделал ему: тогда я пошлю и возьму тебя оттуда; для чего мне в один день лишиться обоих вас?
\end{tcolorbox}
\begin{tcolorbox}
\textsubscript{46} И сказала Ревекка Исааку: я жизни не рада от дочерей Хеттейских; если Иаков возьмет жену из дочерей Хеттейских, каковы эти, из дочерей этой земли, то к чему мне и жизнь?
\end{tcolorbox}
\subsection{CHAPTER 28}
\begin{tcolorbox}
\textsubscript{1} И призвал Исаак Иакова и благословил его, и заповедал ему и сказал: не бери себе жены из дочерей Ханаанских;
\end{tcolorbox}
\begin{tcolorbox}
\textsubscript{2} встань, пойди в Месопотамию, в дом Вафуила, отца матери твоей, и возьми себе жену оттуда, из дочерей Лавана, брата матери твоей;
\end{tcolorbox}
\begin{tcolorbox}
\textsubscript{3} Бог же Всемогущий да благословит тебя, да расплодит тебя и да размножит тебя, и да будет от тебя множество народов,
\end{tcolorbox}
\begin{tcolorbox}
\textsubscript{4} и да даст тебе благословение Авраама, тебе и потомству твоему с тобою, чтобы тебе наследовать землю странствования твоего, которую Бог дал Аврааму!
\end{tcolorbox}
\begin{tcolorbox}
\textsubscript{5} И отпустил Исаак Иакова, и он пошел в Месопотамию к Лавану, сыну Вафуила Арамеянина, к брату Ревекки, матери Иакова и Исава.
\end{tcolorbox}
\begin{tcolorbox}
\textsubscript{6} Исав увидел, что Исаак благословил Иакова и благословляя послал его в Месопотамию, взять себе жену оттуда, и заповедал ему, сказав: не бери жены из дочерей Ханаанских;
\end{tcolorbox}
\begin{tcolorbox}
\textsubscript{7} и что Иаков послушался отца своего и матери своей и пошел в Месопотамию.
\end{tcolorbox}
\begin{tcolorbox}
\textsubscript{8} И увидел Исав, что дочери Ханаанские не угодны Исааку, отцу его;
\end{tcolorbox}
\begin{tcolorbox}
\textsubscript{9} и пошел Исав к Измаилу и взял себе жену Махалафу, дочь Измаила, сына Авраамова, сестру Наваиофову, сверх [других] жен своих.
\end{tcolorbox}
\begin{tcolorbox}
\textsubscript{10} Иаков же вышел из Вирсавии и пошел в Харран,
\end{tcolorbox}
\begin{tcolorbox}
\textsubscript{11} и пришел на [одно] место, и [остался] там ночевать, потому что зашло солнце. И взял [один] из камней того места, и положил себе изголовьем, и лег на том месте.
\end{tcolorbox}
\begin{tcolorbox}
\textsubscript{12} И увидел во сне: вот, лестница стоит на земле, а верх ее касается неба; и вот, Ангелы Божии восходят и нисходят по ней.
\end{tcolorbox}
\begin{tcolorbox}
\textsubscript{13} И вот, Господь стоит на ней и говорит: Я Господь, Бог Авраама, отца твоего, и Бог Исаака. Землю, на которой ты лежишь, Я дам тебе и потомству твоему;
\end{tcolorbox}
\begin{tcolorbox}
\textsubscript{14} и будет потомство твое, как песок земной; и распространишься к морю и к востоку, и к северу и к полудню; и благословятся в тебе и в семени твоем все племена земные;
\end{tcolorbox}
\begin{tcolorbox}
\textsubscript{15} и вот Я с тобою, и сохраню тебя везде, куда ты ни пойдешь; и возвращу тебя в сию землю, ибо Я не оставлю тебя, доколе не исполню того, что Я сказал тебе.
\end{tcolorbox}
\begin{tcolorbox}
\textsubscript{16} Иаков пробудился от сна своего и сказал: истинно Господь присутствует на месте сем; а я не знал!
\end{tcolorbox}
\begin{tcolorbox}
\textsubscript{17} И убоялся и сказал: как страшно сие место! это не иное что, как дом Божий, это врата небесные.
\end{tcolorbox}
\begin{tcolorbox}
\textsubscript{18} И встал Иаков рано утром, и взял камень, который он положил себе изголовьем, и поставил его памятником, и возлил елей на верх его.
\end{tcolorbox}
\begin{tcolorbox}
\textsubscript{19} И нарек имя месту тому: Вефиль, а прежнее имя того города было: Луз.
\end{tcolorbox}
\begin{tcolorbox}
\textsubscript{20} И положил Иаков обет, сказав: если Бог будет со мною и сохранит меня в пути сем, в который я иду, и даст мне хлеб есть и одежду одеться,
\end{tcolorbox}
\begin{tcolorbox}
\textsubscript{21} и я в мире возвращусь в дом отца моего, и будет Господь моим Богом, --
\end{tcolorbox}
\begin{tcolorbox}
\textsubscript{22} то этот камень, который я поставил памятником, будет домом Божиим; и из всего, что Ты, [Боже], даруешь мне, я дам Тебе десятую часть.
\end{tcolorbox}
\subsection{CHAPTER 29}
\begin{tcolorbox}
\textsubscript{1} И встал Иаков и пошел в землю сынов востока.
\end{tcolorbox}
\begin{tcolorbox}
\textsubscript{2} И увидел: вот, на поле колодезь, и там три стада мелкого скота, лежавшие около него, потому что из того колодезя поили стада. Над устьем колодезя был большой камень.
\end{tcolorbox}
\begin{tcolorbox}
\textsubscript{3} Когда собирались туда все стада, отваливали камень от устья колодезя и поили овец; потом опять клали камень на свое место, на устье колодезя.
\end{tcolorbox}
\begin{tcolorbox}
\textsubscript{4} Иаков сказал им: братья мои! откуда вы? Они сказали: мы из Харрана.
\end{tcolorbox}
\begin{tcolorbox}
\textsubscript{5} Он сказал им: знаете ли вы Лавана, сына Нахорова? Они сказали: знаем.
\end{tcolorbox}
\begin{tcolorbox}
\textsubscript{6} Он еще сказал им: здравствует ли он? Они сказали: здравствует; и вот, Рахиль, дочь его, идет с овцами.
\end{tcolorbox}
\begin{tcolorbox}
\textsubscript{7} И сказал: вот, дня еще много; не время собирать скот; напойте овец и пойдите, пасите.
\end{tcolorbox}
\begin{tcolorbox}
\textsubscript{8} Они сказали: не можем, пока не соберутся все стада, и не отвалят камня от устья колодезя; тогда будем мы поить овец.
\end{tcolorbox}
\begin{tcolorbox}
\textsubscript{9} Еще он говорил с ними, как пришла Рахиль с мелким скотом отца своего, потому что она пасла.
\end{tcolorbox}
\begin{tcolorbox}
\textsubscript{10} Когда Иаков увидел Рахиль, дочь Лавана, брата матери своей, и овец Лавана, брата матери своей, то подошел Иаков, отвалил камень от устья колодезя и напоил овец Лавана, брата матери своей.
\end{tcolorbox}
\begin{tcolorbox}
\textsubscript{11} И поцеловал Иаков Рахиль и возвысил голос свой и заплакал.
\end{tcolorbox}
\begin{tcolorbox}
\textsubscript{12} И сказал Иаков Рахили, что он родственник отцу ее и что он сын Ревеккин. А она побежала и сказала отцу своему.
\end{tcolorbox}
\begin{tcolorbox}
\textsubscript{13} Лаван, услышав о Иакове, сыне сестры своей, выбежал ему навстречу, обнял его и поцеловал его, и ввел его в дом свой; и он рассказал Лавану всё сие.
\end{tcolorbox}
\begin{tcolorbox}
\textsubscript{14} Лаван же сказал ему: подлинно ты кость моя и плоть моя. И жил у него [Иаков] целый месяц.
\end{tcolorbox}
\begin{tcolorbox}
\textsubscript{15} И Лаван сказал Иакову: неужели ты даром будешь служить мне, потому что ты родственник? скажи мне, что заплатить тебе?
\end{tcolorbox}
\begin{tcolorbox}
\textsubscript{16} У Лавана же было две дочери; имя старшей: Лия; имя младшей: Рахиль.
\end{tcolorbox}
\begin{tcolorbox}
\textsubscript{17} Лия была слаба глазами, а Рахиль была красива станом и красива лицем.
\end{tcolorbox}
\begin{tcolorbox}
\textsubscript{18} Иаков полюбил Рахиль и сказал: я буду служить тебе семь лет за Рахиль, младшую дочь твою.
\end{tcolorbox}
\begin{tcolorbox}
\textsubscript{19} Лаван сказал: лучше отдать мне ее за тебя, нежели отдать ее за другого кого; живи у меня.
\end{tcolorbox}
\begin{tcolorbox}
\textsubscript{20} И служил Иаков за Рахиль семь лет; и они показались ему за несколько дней, потому что он любил ее.
\end{tcolorbox}
\begin{tcolorbox}
\textsubscript{21} И сказал Иаков Лавану: дай жену мою, потому что мне уже исполнилось время, чтобы войти к ней.
\end{tcolorbox}
\begin{tcolorbox}
\textsubscript{22} Лаван созвал всех людей того места и сделал пир.
\end{tcolorbox}
\begin{tcolorbox}
\textsubscript{23} Вечером же взял дочь свою Лию и ввел ее к нему; и вошел к ней [Иаков].
\end{tcolorbox}
\begin{tcolorbox}
\textsubscript{24} И дал Лаван служанку свою Зелфу в служанки дочери своей Лии.
\end{tcolorbox}
\begin{tcolorbox}
\textsubscript{25} Утром же оказалось, что это Лия. И сказал Лавану: что это сделал ты со мною? не за Рахиль ли я служил у тебя? зачем ты обманул меня?
\end{tcolorbox}
\begin{tcolorbox}
\textsubscript{26} Лаван сказал: в нашем месте так не делают, чтобы младшую выдать прежде старшей;
\end{tcolorbox}
\begin{tcolorbox}
\textsubscript{27} окончи неделю этой, потом дадим тебе и ту за службу, которую ты будешь служить у меня еще семь лет других.
\end{tcolorbox}
\begin{tcolorbox}
\textsubscript{28} Иаков так и сделал и окончил неделю этой. И [Лаван] дал Рахиль, дочь свою, ему в жену.
\end{tcolorbox}
\begin{tcolorbox}
\textsubscript{29} И дал Лаван служанку свою Валлу в служанки дочери своей Рахили.
\end{tcolorbox}
\begin{tcolorbox}
\textsubscript{30} [Иаков] вошел и к Рахили, и любил Рахиль больше, нежели Лию; и служил у него еще семь лет других.
\end{tcolorbox}
\begin{tcolorbox}
\textsubscript{31} Господь узрел, что Лия была нелюбима, и отверз утробу ее, а Рахиль была неплодна.
\end{tcolorbox}
\begin{tcolorbox}
\textsubscript{32} Лия зачала и родила сына, и нарекла ему имя: Рувим, потому что сказала она: Господь призрел на мое бедствие; ибо теперь будет любить меня муж мой.
\end{tcolorbox}
\begin{tcolorbox}
\textsubscript{33} И зачала опять и родила сына, и сказала: Господь услышал, что я нелюбима, и дал мне и сего. И нарекла ему имя: Симеон.
\end{tcolorbox}
\begin{tcolorbox}
\textsubscript{34} И зачала еще и родила сына, и сказала: теперь-то прилепится ко мне муж мой, ибо я родила ему трех сынов. От сего наречено ему имя: Левий.
\end{tcolorbox}
\begin{tcolorbox}
\textsubscript{35} И еще зачала и родила сына, и сказала: теперь-то я восхвалю Господа. Посему нарекла ему имя Иуда. И перестала рождать.
\end{tcolorbox}
\subsection{CHAPTER 30}
\begin{tcolorbox}
\textsubscript{1} И увидела Рахиль, что она не рождает детей Иакову, и позавидовала Рахиль сестре своей, и сказала Иакову: дай мне детей, а если не так, я умираю.
\end{tcolorbox}
\begin{tcolorbox}
\textsubscript{2} Иаков разгневался на Рахиль и сказал: разве я Бог, Который не дал тебе плода чрева?
\end{tcolorbox}
\begin{tcolorbox}
\textsubscript{3} Она сказала: вот служанка моя Валла; войди к ней; пусть она родит на колени мои, чтобы и я имела детей от нее.
\end{tcolorbox}
\begin{tcolorbox}
\textsubscript{4} И дала она Валлу, служанку свою, в жену ему; и вошел к ней Иаков.
\end{tcolorbox}
\begin{tcolorbox}
\textsubscript{5} Валла зачала и родила Иакову сына.
\end{tcolorbox}
\begin{tcolorbox}
\textsubscript{6} И сказала Рахиль: судил мне Бог, и услышал голос мой, и дал мне сына. Посему нарекла ему имя: Дан.
\end{tcolorbox}
\begin{tcolorbox}
\textsubscript{7} И еще зачала и родила Валла, служанка Рахилина, другого сына Иакову.
\end{tcolorbox}
\begin{tcolorbox}
\textsubscript{8} И сказала Рахиль: борьбою сильною боролась я с сестрою моею и превозмогла. И нарекла ему имя: Неффалим.
\end{tcolorbox}
\begin{tcolorbox}
\textsubscript{9} Лия увидела, что перестала рождать, и взяла служанку свою Зелфу, и дала ее Иакову в жену.
\end{tcolorbox}
\begin{tcolorbox}
\textsubscript{10} И Зелфа, служанка Лиина, родила Иакову сына.
\end{tcolorbox}
\begin{tcolorbox}
\textsubscript{11} И сказала Лия: прибавилось. И нарекла ему имя: Гад.
\end{tcolorbox}
\begin{tcolorbox}
\textsubscript{12} И родила Зелфа, служанка Лии, другого сына Иакову.
\end{tcolorbox}
\begin{tcolorbox}
\textsubscript{13} И сказала Лия: к благу моему, ибо блаженною будут называть меня женщины. И нарекла ему имя: Асир.
\end{tcolorbox}
\begin{tcolorbox}
\textsubscript{14} Рувим пошел во время жатвы пшеницы, и нашел мандрагоровые яблоки в поле, и принес их Лии, матери своей. И Рахиль сказала Лии: дай мне мандрагоров сына твоего.
\end{tcolorbox}
\begin{tcolorbox}
\textsubscript{15} Но она сказала ей: неужели мало тебе завладеть мужем моим, что ты домогаешься и мандрагоров сына моего? Рахиль сказала: так пусть он ляжет с тобою эту ночь, за мандрагоры сына твоего.
\end{tcolorbox}
\begin{tcolorbox}
\textsubscript{16} Иаков пришел с поля вечером, и Лия вышла ему навстречу и сказала: войди ко мне; ибо я купила тебя за мандрагоры сына моего. И лег он с нею в ту ночь.
\end{tcolorbox}
\begin{tcolorbox}
\textsubscript{17} И услышал Бог Лию, и она зачала и родила Иакову пятого сына.
\end{tcolorbox}
\begin{tcolorbox}
\textsubscript{18} И сказала Лия: Бог дал возмездие мне за то, что я отдала служанку мою мужу моему. И нарекла ему имя: Иссахар.
\end{tcolorbox}
\begin{tcolorbox}
\textsubscript{19} И еще зачала Лия и родила Иакову шестого сына.
\end{tcolorbox}
\begin{tcolorbox}
\textsubscript{20} И сказала Лия: Бог дал мне прекрасный дар; теперь будет жить у меня муж мой, ибо я родила ему шесть сынов. И нарекла ему имя: Завулон.
\end{tcolorbox}
\begin{tcolorbox}
\textsubscript{21} Потом родила дочь и нарекла ей имя: Дина.
\end{tcolorbox}
\begin{tcolorbox}
\textsubscript{22} И вспомнил Бог о Рахили, и услышал ее Бог, и отверз утробу ее.
\end{tcolorbox}
\begin{tcolorbox}
\textsubscript{23} Она зачала и родила сына, и сказала: снял Бог позор мой.
\end{tcolorbox}
\begin{tcolorbox}
\textsubscript{24} И нарекла ему имя: Иосиф, сказав: Господь даст мне и другого сына.
\end{tcolorbox}
\begin{tcolorbox}
\textsubscript{25} После того, как Рахиль родила Иосифа, Иаков сказал Лавану: отпусти меня, и пойду я в свое место, и в свою землю;
\end{tcolorbox}
\begin{tcolorbox}
\textsubscript{26} отдай жен моих и детей моих, за которых я служил тебе, и я пойду, ибо ты знаешь службу мою, какую я служил тебе.
\end{tcolorbox}
\begin{tcolorbox}
\textsubscript{27} И сказал ему Лаван: о, если бы я нашел благоволение пред очами твоими! я примечаю, что за тебя Господь благословил меня.
\end{tcolorbox}
\begin{tcolorbox}
\textsubscript{28} И сказал: назначь себе награду от меня, и я дам.
\end{tcolorbox}
\begin{tcolorbox}
\textsubscript{29} И сказал ему [Иаков]: ты знаешь, как я служил тебе, и каков стал скот твой при мне;
\end{tcolorbox}
\begin{tcolorbox}
\textsubscript{30} ибо мало было у тебя до меня, а стало много; Господь благословил тебя с приходом моим; когда же я буду работать для своего дома?
\end{tcolorbox}
\begin{tcolorbox}
\textsubscript{31} И сказал [Лаван]: что дать тебе? Иаков сказал: не давай мне ничего. Если только сделаешь мне, что я скажу, то я опять буду пасти и стеречь овец твоих.
\end{tcolorbox}
\begin{tcolorbox}
\textsubscript{32} Я пройду сегодня по всему [стаду] овец твоих; отдели из него всякий скот с крапинами и с пятнами, всякую скотину черную из овец, также с пятнами и с крапинами из коз. [Такой скот] будет наградою мне.
\end{tcolorbox}
\begin{tcolorbox}
\textsubscript{33} И будет говорить за меня пред тобою справедливость моя в следующее время, когда придешь посмотреть награду мою. Всякая из коз не с крапинами и не с пятнами, и из овец не черная, краденое это у меня.
\end{tcolorbox}
\begin{tcolorbox}
\textsubscript{34} Лаван сказал: хорошо, пусть будет по твоему слову.
\end{tcolorbox}
\begin{tcolorbox}
\textsubscript{35} И отделил в тот день козлов пестрых и с пятнами, и всех коз с крапинами и с пятнами, всех, на которых было [несколько] белого, и всех черных овец, и отдал на руки сыновьям своим;
\end{tcolorbox}
\begin{tcolorbox}
\textsubscript{36} и назначил расстояние между собою и между Иаковом на три дня пути. Иаков же пас остальной мелкий скот Лаванов.
\end{tcolorbox}
\begin{tcolorbox}
\textsubscript{37} И взял Иаков свежих прутьев тополевых, миндальных и яворовых, и вырезал на них белые полосы, сняв кору до белизны, которая на прутьях,
\end{tcolorbox}
\begin{tcolorbox}
\textsubscript{38} и положил прутья с нарезкою перед скотом в водопойных корытах, куда скот приходил пить, и где, приходя пить, зачинал пред прутьями.
\end{tcolorbox}
\begin{tcolorbox}
\textsubscript{39} И зачинал скот пред прутьями, и рождался скот пестрый, и с крапинами, и с пятнами.
\end{tcolorbox}
\begin{tcolorbox}
\textsubscript{40} И отделял Иаков ягнят и ставил скот лицем к пестрому и всему черному скоту Лаванову; и держал свои стада особо и не ставил их вместе со скотом Лавана.
\end{tcolorbox}
\begin{tcolorbox}
\textsubscript{41} Каждый раз, когда зачинал скот крепкий, Иаков клал прутья в корытах пред глазами скота, чтобы он зачинал пред прутьями.
\end{tcolorbox}
\begin{tcolorbox}
\textsubscript{42} А когда зачинал скот слабый, тогда он не клал. И доставался слабый [скот] Лавану, а крепкий Иакову.
\end{tcolorbox}
\begin{tcolorbox}
\textsubscript{43} И сделался этот человек весьма, весьма богатым, и было у него множество мелкого скота, и рабынь, и рабов, и верблюдов, и ослов.
\end{tcolorbox}
\subsection{CHAPTER 31}
\begin{tcolorbox}
\textsubscript{1} И услышал [Иаков] слова сынов Лавановых, которые говорили: Иаков завладел всем, что было у отца нашего, и из имения отца нашего составил все богатство сие.
\end{tcolorbox}
\begin{tcolorbox}
\textsubscript{2} И увидел Иаков лице Лавана, и вот, оно не таково к нему, как было вчера и третьего дня.
\end{tcolorbox}
\begin{tcolorbox}
\textsubscript{3} И сказал Господь Иакову: возвратись в землю отцов твоих и на родину твою; и Я буду с тобою.
\end{tcolorbox}
\begin{tcolorbox}
\textsubscript{4} И послал Иаков, и призвал Рахиль и Лию в поле, к [стаду] мелкого скота своего,
\end{tcolorbox}
\begin{tcolorbox}
\textsubscript{5} и сказал им: я вижу лице отца вашего, что оно ко мне не таково, как было вчера и третьего дня; но Бог отца моего был со мною;
\end{tcolorbox}
\begin{tcolorbox}
\textsubscript{6} вы сами знаете, что я всеми силами служил отцу вашему,
\end{tcolorbox}
\begin{tcolorbox}
\textsubscript{7} а отец ваш обманывал меня и раз десять переменял награду мою; но Бог не попустил ему сделать мне зло.
\end{tcolorbox}
\begin{tcolorbox}
\textsubscript{8} Когда сказал он, что [скот] с крапинами будет тебе в награду, то скот весь родил с крапинами. А когда он сказал: пестрые будут тебе в награду, то скот весь и родил пестрых.
\end{tcolorbox}
\begin{tcolorbox}
\textsubscript{9} И отнял Бог скот у отца вашего и дал мне.
\end{tcolorbox}
\begin{tcolorbox}
\textsubscript{10} Однажды в такое время, когда скот зачинает, я взглянул и увидел во сне, и вот козлы, поднявшиеся на скот, пестрые с крапинами и пятнами.
\end{tcolorbox}
\begin{tcolorbox}
\textsubscript{11} Ангел Божий сказал мне во сне: Иаков! Я сказал: вот я.
\end{tcolorbox}
\begin{tcolorbox}
\textsubscript{12} Он сказал: возведи очи твои и посмотри: все козлы, поднявшиеся на скот, пестрые, с крапинами и с пятнами, ибо Я вижу все, что Лаван делает с тобою;
\end{tcolorbox}
\begin{tcolorbox}
\textsubscript{13} Я Бог [явившийся тебе] в Вефиле, где ты возлил елей на памятник и где ты дал Мне обет; теперь встань, выйди из земли сей и возвратись в землю родины твоей.
\end{tcolorbox}
\begin{tcolorbox}
\textsubscript{14} Рахиль и Лия сказали ему в ответ: есть ли еще нам доля и наследство в доме отца нашего?
\end{tcolorbox}
\begin{tcolorbox}
\textsubscript{15} не за чужих ли он нас почитает? ибо он продал нас и съел даже серебро наше;
\end{tcolorbox}
\begin{tcolorbox}
\textsubscript{16} посему все богатство, которое Бог отнял у отца нашего, есть наше и детей наших; итак делай все, что Бог сказал тебе.
\end{tcolorbox}
\begin{tcolorbox}
\textsubscript{17} И встал Иаков, и посадил детей своих и жен своих на верблюдов,
\end{tcolorbox}
\begin{tcolorbox}
\textsubscript{18} и взял с собою весь скот свой и все богатство свое, которое приобрел, скот собственный его, который он приобрел в Месопотамии, чтобы идти к Исааку, отцу своему, в землю Ханаанскую.
\end{tcolorbox}
\begin{tcolorbox}
\textsubscript{19} И как Лаван пошел стричь скот свой, то Рахиль похитила идолов, которые были у отца ее.
\end{tcolorbox}
\begin{tcolorbox}
\textsubscript{20} Иаков же похитил сердце у Лавана Арамеянина, потому что не известил его, что удаляется.
\end{tcolorbox}
\begin{tcolorbox}
\textsubscript{21} И ушел со всем, что у него было; и, встав, перешел реку и направился к горе Галаад.
\end{tcolorbox}
\begin{tcolorbox}
\textsubscript{22} На третий день сказали Лавану, что Иаков ушел.
\end{tcolorbox}
\begin{tcolorbox}
\textsubscript{23} Тогда он взял с собою родственников своих, и гнался за ним семь дней, и догнал его на горе Галаад.
\end{tcolorbox}
\begin{tcolorbox}
\textsubscript{24} И пришел Бог к Лавану Арамеянину ночью во сне и сказал ему: берегись, не говори Иакову ни доброго, ни худого.
\end{tcolorbox}
\begin{tcolorbox}
\textsubscript{25} И догнал Лаван Иакова; Иаков же поставил шатер свой на горе, и Лаван со сродниками своими поставил на горе Галаад.
\end{tcolorbox}
\begin{tcolorbox}
\textsubscript{26} И сказал Лаван Иакову: что ты сделал? для чего ты обманул меня, и увел дочерей моих, как плененных оружием?
\end{tcolorbox}
\begin{tcolorbox}
\textsubscript{27} зачем ты убежал тайно, и укрылся от меня, и не сказал мне? я отпустил бы тебя с веселием и с песнями, с тимпаном и с гуслями;
\end{tcolorbox}
\begin{tcolorbox}
\textsubscript{28} ты не позволил мне даже поцеловать внуков моих и дочерей моих; безрассудно ты сделал.
\end{tcolorbox}
\begin{tcolorbox}
\textsubscript{29} Есть в руке моей сила сделать вам зло; но Бог отца вашего вчера говорил ко мне и сказал: берегись, не говори Иакову ни хорошего, ни худого.
\end{tcolorbox}
\begin{tcolorbox}
\textsubscript{30} Но пусть бы ты ушел, потому что ты нетерпеливо захотел быть в доме отца твоего, --зачем ты украл богов моих?
\end{tcolorbox}
\begin{tcolorbox}
\textsubscript{31} Иаков отвечал Лавану и сказал: [я] боялся, ибо я думал, не отнял бы ты у меня дочерей своих.
\end{tcolorbox}
\begin{tcolorbox}
\textsubscript{32} у кого найдешь богов твоих, тот не будет жив; при родственниках наших узнавай, что у меня, и возьми себе. Иаков не знал, что Рахиль украла их.
\end{tcolorbox}
\begin{tcolorbox}
\textsubscript{33} И ходил Лаван в шатер Иакова, и в шатер Лии, и в шатер двух рабынь, но не нашел. И, выйдя из шатра Лии, вошел в шатер Рахили.
\end{tcolorbox}
\begin{tcolorbox}
\textsubscript{34} Рахиль же взяла идолов, и положила их под верблюжье седло и села на них. И обыскал Лаван весь шатер; но не нашел.
\end{tcolorbox}
\begin{tcolorbox}
\textsubscript{35} Она же сказала отцу своему: да не прогневается господин мой, что я не могу встать пред тобою, ибо у меня обыкновенное женское. И он искал, но не нашел идолов.
\end{tcolorbox}
\begin{tcolorbox}
\textsubscript{36} Иаков рассердился и вступил в спор с Лаваном. И начал Иаков говорить и сказал Лавану: какая вина моя, какой грех мой, что ты преследуешь меня?
\end{tcolorbox}
\begin{tcolorbox}
\textsubscript{37} ты осмотрел у меня все вещи, что нашел ты из всех вещей твоего дома? покажи здесь пред родственниками моими и пред родственниками твоими; пусть они рассудят между нами обоими.
\end{tcolorbox}
\begin{tcolorbox}
\textsubscript{38} Вот, двадцать лет я [был] у тебя; овцы твои и козы твои не выкидывали; овнов стада твоего я не ел;
\end{tcolorbox}
\begin{tcolorbox}
\textsubscript{39} растерзанного зверем я не приносил к тебе, это был мой убыток; ты с меня взыскивал, днем ли что пропадало, ночью ли пропадало;
\end{tcolorbox}
\begin{tcolorbox}
\textsubscript{40} я томился днем от жара, а ночью от стужи, и сон мой убегал от глаз моих.
\end{tcolorbox}
\begin{tcolorbox}
\textsubscript{41} Таковы мои двадцать лет в доме твоем. Я служил тебе четырнадцать лет за двух дочерей твоих и шесть лет за скот твой, а ты десять раз переменял награду мою.
\end{tcolorbox}
\begin{tcolorbox}
\textsubscript{42} Если бы не был со мною Бог отца моего, Бог Авраама и страх Исаака, ты бы теперь отпустил меня ни с чем. Бог увидел бедствие мое и труд рук моих и вступился [за меня] вчера.
\end{tcolorbox}
\begin{tcolorbox}
\textsubscript{43} И отвечал Лаван и сказал Иакову: дочери--мои дочери; дети--мои дети; скот--мой скот, и все, что ты видишь, это мое: могу ли я что сделать теперь с дочерями моими и с детьми их, которые рождены ими?
\end{tcolorbox}
\begin{tcolorbox}
\textsubscript{44} Теперь заключим союз я и ты, и это будет свидетельством между мною и тобою.
\end{tcolorbox}
\begin{tcolorbox}
\textsubscript{45} И взял Иаков камень и поставил его памятником.
\end{tcolorbox}
\begin{tcolorbox}
\textsubscript{46} И сказал Иаков родственникам своим: наберите камней. Они взяли камни, и сделали холм, и ели там на холме.
\end{tcolorbox}
\begin{tcolorbox}
\textsubscript{47} И назвал его Лаван: Иегар-Сагадуфа; а Иаков назвал его Галаадом.
\end{tcolorbox}
\begin{tcolorbox}
\textsubscript{48} И сказал Лаван: сегодня этот холм между мною и тобою свидетель. Посему и наречено ему имя: Галаад,
\end{tcolorbox}
\begin{tcolorbox}
\textsubscript{49} [также]: Мицпа, оттого, что Лаван сказал: да надзирает Господь надо мною и над тобою, когда мы скроемся друг от друга;
\end{tcolorbox}
\begin{tcolorbox}
\textsubscript{50} если ты будешь худо поступать с дочерями моими, или если возьмешь жен сверх дочерей моих, то, хотя нет человека между нами, но смотри, Бог свидетель между мною и между тобою.
\end{tcolorbox}
\begin{tcolorbox}
\textsubscript{51} И сказал Лаван Иакову: вот холм сей и вот памятник, который я поставил между мною и тобою;
\end{tcolorbox}
\begin{tcolorbox}
\textsubscript{52} этот холм свидетель, и этот памятник свидетель, что ни я не перейду к тебе за этот холм, ни ты не перейдешь ко мне за этот холм и за этот памятник, для зла;
\end{tcolorbox}
\begin{tcolorbox}
\textsubscript{53} Бог Авраамов и Бог Нахоров да судит между нами, Бог отца их. Иаков поклялся страхом отца своего Исаака.
\end{tcolorbox}
\begin{tcolorbox}
\textsubscript{54} И заколол Иаков жертву на горе и позвал родственников своих есть хлеб; и они ели хлеб и ночевали на горе.
\end{tcolorbox}
\begin{tcolorbox}
\textsubscript{55} И встал Лаван рано утром и поцеловал внуков своих и дочерей своих, и благословил их. И пошел и возвратился Лаван в свое место.
\end{tcolorbox}
\subsection{CHAPTER 32}
\begin{tcolorbox}
\textsubscript{1} А Иаков пошел путем своим. И встретили его Ангелы Божии.
\end{tcolorbox}
\begin{tcolorbox}
\textsubscript{2} Иаков, увидев их, сказал: это ополчение Божие. И нарек имя месту тому: Маханаим.
\end{tcolorbox}
\begin{tcolorbox}
\textsubscript{3} И послал Иаков пред собою вестников к брату своему Исаву в землю Сеир, в область Едом,
\end{tcolorbox}
\begin{tcolorbox}
\textsubscript{4} и приказал им, сказав: так скажите господину моему Исаву: вот что говорит раб твой Иаков: я жил у Лавана и прожил доныне;
\end{tcolorbox}
\begin{tcolorbox}
\textsubscript{5} и есть у меня волы и ослы и мелкий скот, и рабы и рабыни; и я послал известить [о себе] господина моего, дабы приобрести благоволение пред очами твоими.
\end{tcolorbox}
\begin{tcolorbox}
\textsubscript{6} И возвратились вестники к Иакову и сказали: мы ходили к брату твоему Исаву; он идет навстречу тебе, и с ним четыреста человек.
\end{tcolorbox}
\begin{tcolorbox}
\textsubscript{7} Иаков очень испугался и смутился; и разделил людей, бывших с ним, и скот мелкий и крупный и верблюдов на два стана.
\end{tcolorbox}
\begin{tcolorbox}
\textsubscript{8} И сказал: если Исав нападет на один стан и побьет его, то остальной стан может спастись.
\end{tcolorbox}
\begin{tcolorbox}
\textsubscript{9} И сказал Иаков: Боже отца моего Авраама и Боже отца моего Исаака, Господи, сказавший мне: возвратись в землю твою, на родину твою, и Я буду благотворить тебе!
\end{tcolorbox}
\begin{tcolorbox}
\textsubscript{10} Недостоин я всех милостей и всех благодеяний, которые Ты сотворил рабу Твоему, ибо я с посохом моим перешел этот Иордан, а теперь у меня два стана.
\end{tcolorbox}
\begin{tcolorbox}
\textsubscript{11} Избавь меня от руки брата моего, от руки Исава, ибо я боюсь его, чтобы он, придя, не убил меня [и] матери с детьми.
\end{tcolorbox}
\begin{tcolorbox}
\textsubscript{12} Ты сказал: Я буду благотворить тебе и сделаю потомство твое, как песок морской, которого не исчислить от множества.
\end{tcolorbox}
\begin{tcolorbox}
\textsubscript{13} И ночевал там [Иаков] в ту ночь. И взял из того, что у него было, в подарок Исаву, брату своему:
\end{tcolorbox}
\begin{tcolorbox}
\textsubscript{14} двести коз, двадцать козлов, двести овец, двадцать овнов,
\end{tcolorbox}
\begin{tcolorbox}
\textsubscript{15} тридцать верблюдиц дойных с жеребятами их, сорок коров, десять волов, двадцать ослиц, десять ослов.
\end{tcolorbox}
\begin{tcolorbox}
\textsubscript{16} И дал в руки рабам своим каждое стадо особо и сказал рабам своим: пойдите предо мною и оставляйте расстояние от стада до стада.
\end{tcolorbox}
\begin{tcolorbox}
\textsubscript{17} И приказал первому, сказав: когда брат мой Исав встретится тебе и спросит тебя, говоря: чей ты? и куда идешь? и чье это [стадо] пред тобою?
\end{tcolorbox}
\begin{tcolorbox}
\textsubscript{18} то скажи: раба твоего Иакова; это подарок, посланный господину моему Исаву; вот, и сам он за нами.
\end{tcolorbox}
\begin{tcolorbox}
\textsubscript{19} То же приказал он и второму, и третьему, и всем, которые шли за стадами, говоря: так скажите Исаву, когда встретите его;
\end{tcolorbox}
\begin{tcolorbox}
\textsubscript{20} и скажите: вот, и раб твой Иаков за нами. Ибо он сказал [сам в себе]: умилостивлю его дарами, которые идут предо мною, и потом увижу лице его; может быть, и примет меня.
\end{tcolorbox}
\begin{tcolorbox}
\textsubscript{21} И пошли дары пред ним, а он ту ночь ночевал в стане.
\end{tcolorbox}
\begin{tcolorbox}
\textsubscript{22} И встал в ту ночь, и, взяв двух жен своих и двух рабынь своих, и одиннадцать сынов своих, перешел через Иавок вброд;
\end{tcolorbox}
\begin{tcolorbox}
\textsubscript{23} и, взяв их, перевел через поток, и перевел все, что у него [было].
\end{tcolorbox}
\begin{tcolorbox}
\textsubscript{24} И остался Иаков один. И боролся Некто с ним до появления зари;
\end{tcolorbox}
\begin{tcolorbox}
\textsubscript{25} и, увидев, что не одолевает его, коснулся состава бедра его и повредил состав бедра у Иакова, когда он боролся с Ним.
\end{tcolorbox}
\begin{tcolorbox}
\textsubscript{26} И сказал: отпусти Меня, ибо взошла заря. Иаков сказал: не отпущу Тебя, пока не благословишь меня.
\end{tcolorbox}
\begin{tcolorbox}
\textsubscript{27} И сказал: как имя твое? Он сказал: Иаков.
\end{tcolorbox}
\begin{tcolorbox}
\textsubscript{28} И сказал: отныне имя тебе будет не Иаков, а Израиль, ибо ты боролся с Богом, и человеков одолевать будешь.
\end{tcolorbox}
\begin{tcolorbox}
\textsubscript{29} Спросил и Иаков, говоря: скажи имя Твое. И Он сказал: на что ты спрашиваешь о имени Моем? И благословил его там.
\end{tcolorbox}
\begin{tcolorbox}
\textsubscript{30} И нарек Иаков имя месту тому: Пенуэл; ибо, [говорил он], я видел Бога лицем к лицу, и сохранилась душа моя.
\end{tcolorbox}
\begin{tcolorbox}
\textsubscript{31} И взошло солнце, когда он проходил Пенуэл; и хромал он на бедро свое.
\end{tcolorbox}
\begin{tcolorbox}
\textsubscript{32} Поэтому и доныне сыны Израилевы не едят жилы, которая на составе бедра, потому что [Боровшийся] коснулся жилы на составе бедра Иакова.
\end{tcolorbox}
\subsection{CHAPTER 33}
\begin{tcolorbox}
\textsubscript{1} Взглянул Иаков и увидел, и вот, идет Исав, и с ним четыреста человек. И разделил детей Лии, Рахили и двух служанок.
\end{tcolorbox}
\begin{tcolorbox}
\textsubscript{2} И поставил служанок и детей их впереди, Лию и детей ее за ними, а Рахиль и Иосифа позади.
\end{tcolorbox}
\begin{tcolorbox}
\textsubscript{3} А сам пошел пред ними и поклонился до земли семь раз, подходя к брату своему.
\end{tcolorbox}
\begin{tcolorbox}
\textsubscript{4} И побежал Исав к нему навстречу и обнял его, и пал на шею его и целовал его, и плакали.
\end{tcolorbox}
\begin{tcolorbox}
\textsubscript{5} И взглянул и увидел жен и детей и сказал: кто это у тебя? [Иаков] сказал: дети, которых Бог даровал рабу твоему.
\end{tcolorbox}
\begin{tcolorbox}
\textsubscript{6} И подошли служанки и дети их и поклонились;
\end{tcolorbox}
\begin{tcolorbox}
\textsubscript{7} подошла и Лия и дети ее и поклонились; наконец подошли Иосиф и Рахиль и поклонились.
\end{tcolorbox}
\begin{tcolorbox}
\textsubscript{8} И сказал Исав: для чего у тебя это множество, которое я встретил? И сказал Иаков: дабы приобрести благоволение в очах господина моего.
\end{tcolorbox}
\begin{tcolorbox}
\textsubscript{9} Исав сказал: у меня много, брат мой; пусть будет твое у тебя.
\end{tcolorbox}
\begin{tcolorbox}
\textsubscript{10} Иаков сказал: нет, если я приобрел благоволение в очах твоих, прими дар мой от руки моей, ибо я увидел лице твое, как бы кто увидел лице Божие, и ты был благосклонен ко мне;
\end{tcolorbox}
\begin{tcolorbox}
\textsubscript{11} прими благословение мое, которое я принес тебе, потому что Бог даровал мне, и есть у меня всё. И упросил его, и тот взял
\end{tcolorbox}
\begin{tcolorbox}
\textsubscript{12} и сказал: поднимемся и пойдем; и я пойду пред тобою.
\end{tcolorbox}
\begin{tcolorbox}
\textsubscript{13} Иаков сказал ему: господин мой знает, что дети нежны, а мелкий и крупный скот у меня дойный: если погнать его один день, то помрет весь скот;
\end{tcolorbox}
\begin{tcolorbox}
\textsubscript{14} пусть господин мой пойдет впереди раба своего, а я пойду медленно, как пойдет скот, который предо мною, и как пойдут дети, и приду к господину моему в Сеир.
\end{tcolorbox}
\begin{tcolorbox}
\textsubscript{15} Исав сказал: оставлю я с тобою [несколько] из людей, которые при мне. Иаков сказал: к чему это? только бы мне приобрести благоволение в очах господина моего!
\end{tcolorbox}
\begin{tcolorbox}
\textsubscript{16} И возвратился Исав в тот же день путем своим в Сеир.
\end{tcolorbox}
\begin{tcolorbox}
\textsubscript{17} А Иаков двинулся в Сокхоф, и построил себе дом, и для скота своего сделал шалаши. От сего он нарек имя месту: Сокхоф.
\end{tcolorbox}
\begin{tcolorbox}
\textsubscript{18} Иаков, возвратившись из Месопотамии, благополучно пришел в город Сихем, который в земле Ханаанской, и расположился пред городом.
\end{tcolorbox}
\begin{tcolorbox}
\textsubscript{19} И купил часть поля, на котором раскинул шатер свой, у сынов Еммора, отца Сихемова, за сто монет.
\end{tcolorbox}
\begin{tcolorbox}
\textsubscript{20} И поставил там жертвенник, и призвал имя Господа Бога Израилева.
\end{tcolorbox}
\subsection{CHAPTER 34}
\begin{tcolorbox}
\textsubscript{1} Дина, дочь Лии, которую она родила Иакову, вышла посмотреть на дочерей земли той.
\end{tcolorbox}
\begin{tcolorbox}
\textsubscript{2} И увидел ее Сихем, сын Еммора Евеянина, князя земли той, и взял ее, и спал с нею, и сделал ей насилие.
\end{tcolorbox}
\begin{tcolorbox}
\textsubscript{3} И прилепилась душа его в Дине, дочери Иакова, и он полюбил девицу и говорил по сердцу девицы.
\end{tcolorbox}
\begin{tcolorbox}
\textsubscript{4} И сказал Сихем Еммору, отцу своему, говоря: возьми мне эту девицу в жену.
\end{tcolorbox}
\begin{tcolorbox}
\textsubscript{5} Иаков слышал, что [сын Емморов] обесчестил Дину, дочь его, но как сыновья его были со скотом его в поле, то Иаков молчал, пока не пришли они.
\end{tcolorbox}
\begin{tcolorbox}
\textsubscript{6} И вышел Еммор, отец Сихемов, к Иакову, поговорить с ним.
\end{tcolorbox}
\begin{tcolorbox}
\textsubscript{7} Сыновья же Иакова пришли с поля, и когда услышали, то огорчились мужи те и воспылали гневом, потому что бесчестие сделал он Израилю, переспав с дочерью Иакова, а так не надлежало делать.
\end{tcolorbox}
\begin{tcolorbox}
\textsubscript{8} Еммор стал говорить им, и сказал: Сихем, сын мой, прилепился душею к дочери вашей; дайте же ее в жену ему;
\end{tcolorbox}
\begin{tcolorbox}
\textsubscript{9} породнитесь с нами; отдавайте за нас дочерей ваших, а наших дочерей берите себе.
\end{tcolorbox}
\begin{tcolorbox}
\textsubscript{10} и живите с нами; земля сия пред вами, живите и промышляйте на ней и приобретайте ее во владение.
\end{tcolorbox}
\begin{tcolorbox}
\textsubscript{11} Сихем же сказал отцу ее и братьям ее: только бы мне найти благоволение в очах ваших, я дам, что ни скажете мне;
\end{tcolorbox}
\begin{tcolorbox}
\textsubscript{12} назначьте самое большое вено и дары; я дам, что ни скажете мне, только отдайте мне девицу в жену.
\end{tcolorbox}
\begin{tcolorbox}
\textsubscript{13} И отвечали сыновья Иакова Сихему и Еммору, отцу его, с лукавством; а говорили так потому, что он обесчестил Дину, сестру их;
\end{tcolorbox}
\begin{tcolorbox}
\textsubscript{14} и сказали им: не можем этого сделать, выдать сестру нашу за человека, который необрезан, ибо это бесчестно для нас;
\end{tcolorbox}
\begin{tcolorbox}
\textsubscript{15} только на том условии мы согласимся с вами, если вы будете как мы, чтобы и у вас весь мужеский пол был обрезан;
\end{tcolorbox}
\begin{tcolorbox}
\textsubscript{16} и будем отдавать за вас дочерей наших и брать за себя ваших дочерей, и будем жить с вами, и составим один народ;
\end{tcolorbox}
\begin{tcolorbox}
\textsubscript{17} а если не послушаетесь нас в том, чтобы обрезаться, то мы возьмем дочь нашу и удалимся.
\end{tcolorbox}
\begin{tcolorbox}
\textsubscript{18} И понравились слова сии Еммору и Сихему, сыну Емморову.
\end{tcolorbox}
\begin{tcolorbox}
\textsubscript{19} Юноша не умедлил исполнить это, потому что любил дочь Иакова. А он более всех уважаем был из дома отца своего.
\end{tcolorbox}
\begin{tcolorbox}
\textsubscript{20} И пришел Еммор и Сихем, сын его, к воротам города своего, и стали говорить жителям города своего и сказали:
\end{tcolorbox}
\begin{tcolorbox}
\textsubscript{21} сии люди мирны с нами; пусть они селятся на земле и промышляют на ней; земля же вот пространна пред ними. Станем брать дочерей их себе в жены и наших дочерей выдавать за них.
\end{tcolorbox}
\begin{tcolorbox}
\textsubscript{22} Только на том условии сии люди соглашаются жить с нами и быть одним народом, чтобы и у нас обрезан был весь мужеский пол, как они обрезаны.
\end{tcolorbox}
\begin{tcolorbox}
\textsubscript{23} Не для нас ли стада их, и имение их, и весь скот их? Только согласимся с ними, и будут жить с нами.
\end{tcolorbox}
\begin{tcolorbox}
\textsubscript{24} И послушались Еммора и Сихема, сына его, все выходящие из ворот города его: и обрезан был весь мужеский пол, --все выходящие из ворот города его.
\end{tcolorbox}
\begin{tcolorbox}
\textsubscript{25} На третий день, когда они были в болезни, два сына Иакова, Симеон и Левий, братья Динины, взяли каждый свой меч, и смело напали на город, и умертвили весь мужеский пол;
\end{tcolorbox}
\begin{tcolorbox}
\textsubscript{26} и самого Еммора и Сихема, сына его, убили мечом; и взяли Дину из дома Сихемова и вышли.
\end{tcolorbox}
\begin{tcolorbox}
\textsubscript{27} Сыновья Иакова пришли к убитым и разграбили город за то, что обесчестили сестру их.
\end{tcolorbox}
\begin{tcolorbox}
\textsubscript{28} Они взяли мелкий и крупный скот их, и ослов их, и что ни было в городе, и что ни было в поле;
\end{tcolorbox}
\begin{tcolorbox}
\textsubscript{29} и все богатство их, и всех детей их, и жен их взяли в плен, и разграбили всё, что было в домах.
\end{tcolorbox}
\begin{tcolorbox}
\textsubscript{30} И сказал Иаков Симеону и Левию: вы возмутили меня, сделав меня ненавистным для жителей сей земли, для Хананеев и Ферезеев. У меня людей мало; соберутся против меня, поразят меня, и истреблен буду я и дом мой.
\end{tcolorbox}
\begin{tcolorbox}
\textsubscript{31} Они же сказали: а разве можно поступать с сестрою нашею, как с блудницею!
\end{tcolorbox}
\subsection{CHAPTER 35}
\begin{tcolorbox}
\textsubscript{1} Бог сказал Иакову: встань, пойди в Вефиль и живи там, и устрой там жертвенник Богу, явившемуся тебе, когда ты бежал от лица Исава, брата твоего.
\end{tcolorbox}
\begin{tcolorbox}
\textsubscript{2} И сказал Иаков дому своему и всем бывшим с ним: бросьте богов чужих, находящихся у вас, и очиститесь, и перемените одежды ваши;
\end{tcolorbox}
\begin{tcolorbox}
\textsubscript{3} встанем и пойдем в Вефиль; там устрою я жертвенник Богу, Который услышал меня в день бедствия моего и был со мною в пути, которым я ходил.
\end{tcolorbox}
\begin{tcolorbox}
\textsubscript{4} И отдали Иакову всех богов чужих, бывших в руках их, и серьги, бывшие в ушах у них, и закопал их Иаков под дубом, который близ Сихема.
\end{tcolorbox}
\begin{tcolorbox}
\textsubscript{5} И отправились они. И был ужас Божий на окрестных городах, и не преследовали сынов Иаковлевых.
\end{tcolorbox}
\begin{tcolorbox}
\textsubscript{6} И пришел Иаков в Луз, что в земле Ханаанской, то есть в Вефиль, сам и все люди, бывшие с ним,
\end{tcolorbox}
\begin{tcolorbox}
\textsubscript{7} и устроил там жертвенник, и назвал сие место: Эл-Вефиль, ибо тут явился ему Бог, когда он бежал от лица брата своего.
\end{tcolorbox}
\begin{tcolorbox}
\textsubscript{8} И умерла Девора, кормилица Ревеккина, и погребена ниже Вефиля под дубом, который и назвал [Иаков] дубом плача.
\end{tcolorbox}
\begin{tcolorbox}
\textsubscript{9} И явился Бог Иакову по возвращении его из Месопотамии, и благословил его,
\end{tcolorbox}
\begin{tcolorbox}
\textsubscript{10} и сказал ему Бог: имя твое Иаков; отныне ты не будешь называться Иаковом, но будет имя тебе: Израиль. И нарек ему имя: Израиль.
\end{tcolorbox}
\begin{tcolorbox}
\textsubscript{11} И сказал ему Бог: Я Бог Всемогущий; плодись и умножайся; народ и множество народов будет от тебя, и цари произойдут из чресл твоих;
\end{tcolorbox}
\begin{tcolorbox}
\textsubscript{12} землю, которую Я дал Аврааму и Исааку, Я дам тебе, и потомству твоему по тебе дам землю сию.
\end{tcolorbox}
\begin{tcolorbox}
\textsubscript{13} И восшел от него Бог с места, на котором говорил ему.
\end{tcolorbox}
\begin{tcolorbox}
\textsubscript{14} И поставил Иаков памятник на месте, на котором говорил ему [Бог], памятник каменный, и возлил на него возлияние, и возлил на него елей;
\end{tcolorbox}
\begin{tcolorbox}
\textsubscript{15} и нарек Иаков имя месту, на котором Бог говорил ему: Вефиль.
\end{tcolorbox}
\begin{tcolorbox}
\textsubscript{16} И отправились из Вефиля. И когда еще оставалось некоторое расстояние земли до Ефрафы, Рахиль родила, и роды ее были трудны.
\end{tcolorbox}
\begin{tcolorbox}
\textsubscript{17} Когда же она страдала в родах, повивальная бабка сказала ей: не бойся, ибо и это тебе сын.
\end{tcolorbox}
\begin{tcolorbox}
\textsubscript{18} И когда выходила из нее душа, ибо она умирала, то нарекла ему имя: Бенони. Но отец его назвал его Вениамином.
\end{tcolorbox}
\begin{tcolorbox}
\textsubscript{19} И умерла Рахиль, и погребена на дороге в Ефрафу, то есть Вифлеем.
\end{tcolorbox}
\begin{tcolorbox}
\textsubscript{20} Иаков поставил над гробом ее памятник. Это надгробный памятник Рахили до сего дня.
\end{tcolorbox}
\begin{tcolorbox}
\textsubscript{21} И отправился Израиль и раскинул шатер свой за башнею Гадер.
\end{tcolorbox}
\begin{tcolorbox}
\textsubscript{22} Во время пребывания Израиля в той стране, Рувим пошел и переспал с Валлою, наложницею отца своего. И услышал Израиль. Сынов же у Иакова было двенадцать.
\end{tcolorbox}
\begin{tcolorbox}
\textsubscript{23} Сыновья Лии: первенец Иакова Рувим, [по нем] Симеон, Левий, Иуда, Иссахар и Завулон.
\end{tcolorbox}
\begin{tcolorbox}
\textsubscript{24} Сыновья Рахили: Иосиф и Вениамин.
\end{tcolorbox}
\begin{tcolorbox}
\textsubscript{25} Сыновья Валлы, служанки Рахилиной: Дан и Неффалим.
\end{tcolorbox}
\begin{tcolorbox}
\textsubscript{26} Сыновья Зелфы, служанки Лииной: Гад и Асир. Сии сыновья Иакова, родившиеся ему в Месопотамии.
\end{tcolorbox}
\begin{tcolorbox}
\textsubscript{27} И пришел Иаков к Исааку, отцу своему, в Мамре, в Кириаф-Арбу, то есть Хеврон где странствовал Авраам и Исаак.
\end{tcolorbox}
\begin{tcolorbox}
\textsubscript{28} И было дней [жизни] Исааковой сто восемьдесят лет.
\end{tcolorbox}
\begin{tcolorbox}
\textsubscript{29} И испустил Исаак дух и умер, и приложился к народу своему, будучи стар и насыщен жизнью; и погребли его Исав и Иаков, сыновья его.
\end{tcolorbox}
\subsection{CHAPTER 36}
\begin{tcolorbox}
\textsubscript{1} Вот родословие Исава, он же Едом.
\end{tcolorbox}
\begin{tcolorbox}
\textsubscript{2} Исав взял себе жен из дочерей Ханаанских: Аду, дочь Елона Хеттеянина, и Оливему, дочь Аны, сына Цивеона Евеянина,
\end{tcolorbox}
\begin{tcolorbox}
\textsubscript{3} и Васемафу, дочь Измаила, сестру Наваиофа.
\end{tcolorbox}
\begin{tcolorbox}
\textsubscript{4} Ада родила Исаву Елифаза, Васемафа родила Рагуила,
\end{tcolorbox}
\begin{tcolorbox}
\textsubscript{5} Оливема родила Иеуса, Иеглома и Корея. Это сыновья Исава, родившиеся ему в земле Ханаанской.
\end{tcolorbox}
\begin{tcolorbox}
\textsubscript{6} И взял Исав жен своих и сыновей своих, и дочерей своих, и всех людей дома своего, и стада свои, и весь скот свой, и всё имение свое, которое он приобрел в земле Ханаанской, и пошел в [другую] землю от лица Иакова, брата своего,
\end{tcolorbox}
\begin{tcolorbox}
\textsubscript{7} ибо имение их было так велико, что они не могли жить вместе, и земля странствования их не вмещала их, по множеству стад их.
\end{tcolorbox}
\begin{tcolorbox}
\textsubscript{8} И поселился Исав на горе Сеир, Исав, он же Едом.
\end{tcolorbox}
\begin{tcolorbox}
\textsubscript{9} И вот родословие Исава, отца Идумеев, на горе Сеир.
\end{tcolorbox}
\begin{tcolorbox}
\textsubscript{10} Вот имена сынов Исава: Елифаз, сын Ады, жены Исавовой, и Рагуил, сын Васемафы, жены Исавовой.
\end{tcolorbox}
\begin{tcolorbox}
\textsubscript{11} У Елифаза были сыновья: Феман, Омар, Цефо, Гафам и Кеназ.
\end{tcolorbox}
\begin{tcolorbox}
\textsubscript{12} Фамна же была наложница Елифаза, сына Исавова, и родила Елифазу Амалика. Вот сыновья Ады, жены Исавовой.
\end{tcolorbox}
\begin{tcolorbox}
\textsubscript{13} И вот сыновья Рагуила: Нахаф и Зерах, Шамма и Миза. Это сыновья Васемафы, жены Исавовой.
\end{tcolorbox}
\begin{tcolorbox}
\textsubscript{14} И сии были сыновья Оливемы, дочери Аны, сына Цивеонова, жены Исавовой: она родила Исаву Иеуса, Иеглома и Корея.
\end{tcolorbox}
\begin{tcolorbox}
\textsubscript{15} Вот старейшины сынов Исавовых. Сыновья Елифаза, первенца Исавова: старейшина Феман, старейшина Омар, старейшина Цефо, старейшина Кеназ,
\end{tcolorbox}
\begin{tcolorbox}
\textsubscript{16} старейшина Корей, старейшина Гафам, старейшина Амалик. Сии старейшины Елифазовы в земле Едома; сии сыновья Ады.
\end{tcolorbox}
\begin{tcolorbox}
\textsubscript{17} Сии сыновья Рагуила, сына Исавова: старейшина Нахаф, старейшина Зерах, старейшина Шамма, старейшина Миза. Сии старейшины Рагуиловы в земле Едома; сии сыновья Васемафы, жены Исавовой.
\end{tcolorbox}
\begin{tcolorbox}
\textsubscript{18} Сии сыновья Оливемы, жены Исавовой: старейшина Иеус, старейшина Иеглом, старейшина Корей. Сии старейшины Оливемы, дочери Аны, жены Исавовой.
\end{tcolorbox}
\begin{tcolorbox}
\textsubscript{19} Вот сыновья Исава, и вот старейшины их. Это Едом.
\end{tcolorbox}
\begin{tcolorbox}
\textsubscript{20} Сии сыновья Сеира Хорреянина, жившие в земле той: Лотан, Шовал, Цивеон, Ана,
\end{tcolorbox}
\begin{tcolorbox}
\textsubscript{21} Дишон, Эцер и Дишан. Сии старейшины Хорреев, сынов Сеира, в земле Едома.
\end{tcolorbox}
\begin{tcolorbox}
\textsubscript{22} Сыновья Лотана были: Хори и Геман; а сестра у Лотана: Фамна.
\end{tcolorbox}
\begin{tcolorbox}
\textsubscript{23} Сии сыновья Шовала: Алван, Манахаф, Эвал, Шефо и Онам.
\end{tcolorbox}
\begin{tcolorbox}
\textsubscript{24} Сии сыновья Цивеона: Аиа и Ана. Это тот Ана, который нашел теплые воды в пустыне, когда пас ослов Цивеона, отца своего.
\end{tcolorbox}
\begin{tcolorbox}
\textsubscript{25} Сии дети Аны: Дишон и Оливема, дочь Аны.
\end{tcolorbox}
\begin{tcolorbox}
\textsubscript{26} Сии сыновья Дишона: Хемдан, Эшбан, Ифран и Херан.
\end{tcolorbox}
\begin{tcolorbox}
\textsubscript{27} Сии сыновья Эцера: Билган, Зааван, и Акан.
\end{tcolorbox}
\begin{tcolorbox}
\textsubscript{28} Сии сыновья Дишана: Уц и Аран.
\end{tcolorbox}
\begin{tcolorbox}
\textsubscript{29} Сии старейшины Хорреев: старейшина Лотан, старейшина Шовал, старейшина Цивеон, старейшина Ана,
\end{tcolorbox}
\begin{tcolorbox}
\textsubscript{30} старейшина Дишон, старейшина Эцер, старейшина Дишан. Вот старейшины Хорреев, по старшинствам их в земле Сеир.
\end{tcolorbox}
\begin{tcolorbox}
\textsubscript{31} Вот цари, царствовавшие в земле Едома, прежде царствования царей у сынов Израилевых:
\end{tcolorbox}
\begin{tcolorbox}
\textsubscript{32} царствовал в Едоме Бела, сын Веоров, а имя городу его Дингава.
\end{tcolorbox}
\begin{tcolorbox}
\textsubscript{33} И умер Бела, и воцарился по нем Иовав, сын Зераха, из Восоры.
\end{tcolorbox}
\begin{tcolorbox}
\textsubscript{34} Умер Иовав, и воцарился по нем Хушам, из земли Феманитян.
\end{tcolorbox}
\begin{tcolorbox}
\textsubscript{35} И умер Хушам, и воцарился по нем Гадад, сын Бедадов, который поразил Мадианитян на поле Моава; имя городу его Авиф.
\end{tcolorbox}
\begin{tcolorbox}
\textsubscript{36} И умер Гадад, и воцарился по нем Самла из Масреки.
\end{tcolorbox}
\begin{tcolorbox}
\textsubscript{37} И умер Самла, и воцарился по нем Саул из Реховофа, что при реке.
\end{tcolorbox}
\begin{tcolorbox}
\textsubscript{38} И умер Саул, и воцарился по нем Баал-Ханан, сын Ахбора.
\end{tcolorbox}
\begin{tcolorbox}
\textsubscript{39} И умер Баал-Ханан, сын Ахбора, и воцарился по нем Гадар: имя городу его Пау; имя жене его Мегетавеель, дочь Матреды, сына Мезагава.
\end{tcolorbox}
\begin{tcolorbox}
\textsubscript{40} Сии имена старейшин Исавовых, по племенам их, по местам их, по именам их: старейшина Фимна, старейшина Алва, старейшина Иетеф,
\end{tcolorbox}
\begin{tcolorbox}
\textsubscript{41} старейшина Оливема, старейшина Эла, старейшина Пинон,
\end{tcolorbox}
\begin{tcolorbox}
\textsubscript{42} старейшина Кеназ, старейшина Феман, старейшина Мивцар,
\end{tcolorbox}
\begin{tcolorbox}
\textsubscript{43} старейшина Магдиил, старейшина Ирам. Вот старейшины Идумейские, по их селениям, в земле обладания их. Вот Исав, отец Идумеев.
\end{tcolorbox}
\subsection{CHAPTER 37}
\begin{tcolorbox}
\textsubscript{1} Иаков жил в земле странствования отца своего, в земле Ханаанской.
\end{tcolorbox}
\begin{tcolorbox}
\textsubscript{2} Вот житие Иакова. Иосиф, семнадцати лет, пас скот вместе с братьями своими, будучи отроком, с сыновьями Валлы и с сыновьями Зелфы, жен отца своего. И доводил Иосиф худые о них слухи до отца их.
\end{tcolorbox}
\begin{tcolorbox}
\textsubscript{3} Израиль любил Иосифа более всех сыновей своих, потому что он был сын старости его, --и сделал ему разноцветную одежду.
\end{tcolorbox}
\begin{tcolorbox}
\textsubscript{4} И увидели братья его, что отец их любит его более всех братьев его; и возненавидели его и не могли говорить с ним дружелюбно.
\end{tcolorbox}
\begin{tcolorbox}
\textsubscript{5} И видел Иосиф сон, и рассказал братьям своим: и они возненавидели его еще более.
\end{tcolorbox}
\begin{tcolorbox}
\textsubscript{6} Он сказал им: выслушайте сон, который я видел:
\end{tcolorbox}
\begin{tcolorbox}
\textsubscript{7} вот, мы вяжем снопы посреди поля; и вот, мой сноп встал и стал прямо; и вот, ваши снопы стали кругом и поклонились моему снопу.
\end{tcolorbox}
\begin{tcolorbox}
\textsubscript{8} И сказали ему братья его: неужели ты будешь царствовать над нами? неужели будешь владеть нами? И возненавидели его еще более за сны его и за слова его.
\end{tcolorbox}
\begin{tcolorbox}
\textsubscript{9} И видел он еще другой сон и рассказал его братьям своим, говоря: вот, я видел еще сон: вот, солнце и луна и одиннадцать звезд поклоняются мне.
\end{tcolorbox}
\begin{tcolorbox}
\textsubscript{10} И он рассказал отцу своему и братьям своим; и побранил его отец его и сказал ему: что это за сон, который ты видел? неужели я и твоя мать, и твои братья придем поклониться тебе до земли?
\end{tcolorbox}
\begin{tcolorbox}
\textsubscript{11} Братья его досадовали на него, а отец его заметил это слово.
\end{tcolorbox}
\begin{tcolorbox}
\textsubscript{12} Братья его пошли пасти скот отца своего в Сихем.
\end{tcolorbox}
\begin{tcolorbox}
\textsubscript{13} И сказал Израиль Иосифу: братья твои не пасут ли в Сихеме? пойди, я пошлю тебя к ним. Он отвечал ему: вот я.
\end{tcolorbox}
\begin{tcolorbox}
\textsubscript{14} И сказал ему: пойди, посмотри, здоровы ли братья твои и цел ли скот, и принеси мне ответ. И послал его из долины Хевронской; и он пришел в Сихем.
\end{tcolorbox}
\begin{tcolorbox}
\textsubscript{15} И нашел его некто блуждающим в поле, и спросил его тот человек, говоря: чего ты ищешь?
\end{tcolorbox}
\begin{tcolorbox}
\textsubscript{16} Он сказал: я ищу братьев моих; скажи мне, где они пасут?
\end{tcolorbox}
\begin{tcolorbox}
\textsubscript{17} И сказал тот человек: они ушли отсюда, ибо я слышал, как они говорили: пойдем в Дофан. И пошел Иосиф за братьями своими и нашел их в Дофане.
\end{tcolorbox}
\begin{tcolorbox}
\textsubscript{18} И увидели они его издали, и прежде нежели он приблизился к ним, стали умышлять против него, чтобы убить его.
\end{tcolorbox}
\begin{tcolorbox}
\textsubscript{19} И сказали друг другу: вот, идет сновидец;
\end{tcolorbox}
\begin{tcolorbox}
\textsubscript{20} пойдем теперь, и убьем его, и бросим его в какой-нибудь ров, и скажем, что хищный зверь съел его; и увидим, что будет из его снов.
\end{tcolorbox}
\begin{tcolorbox}
\textsubscript{21} И услышал [сие] Рувим и избавил его от рук их, сказав: не убьем его.
\end{tcolorbox}
\begin{tcolorbox}
\textsubscript{22} И сказал им Рувим: не проливайте крови; бросьте его в ров, который в пустыне, а руки не налагайте на него. [Сие говорил он], чтобы избавить его от рук их и возвратить его к отцу его.
\end{tcolorbox}
\begin{tcolorbox}
\textsubscript{23} Когда Иосиф пришел к братьям своим, они сняли с Иосифа одежду его, одежду разноцветную, которая была на нем,
\end{tcolorbox}
\begin{tcolorbox}
\textsubscript{24} и взяли его и бросили его в ров; ров же тот был пуст; воды в нем не было.
\end{tcolorbox}
\begin{tcolorbox}
\textsubscript{25} И сели они есть хлеб, и, взглянув, увидели, вот, идет из Галаада караван Измаильтян, и верблюды их несут стираксу, бальзам и ладан: идут они отвезти это в Египет.
\end{tcolorbox}
\begin{tcolorbox}
\textsubscript{26} И сказал Иуда братьям своим: что пользы, если мы убьем брата нашего и скроем кровь его?
\end{tcolorbox}
\begin{tcolorbox}
\textsubscript{27} Пойдем, продадим его Измаильтянам, а руки наши да не будут на нем, ибо он брат наш, плоть наша. Братья его послушались
\end{tcolorbox}
\begin{tcolorbox}
\textsubscript{28} и, когда проходили купцы Мадиамские, вытащили Иосифа изо рва и продали Иосифа Измаильтянам за двадцать сребренников; а они отвели Иосифа в Египет.
\end{tcolorbox}
\begin{tcolorbox}
\textsubscript{29} Рувим же пришел опять ко рву; и вот, нет Иосифа во рве. И разодрал он одежды свои,
\end{tcolorbox}
\begin{tcolorbox}
\textsubscript{30} и возвратился к братьям своим, и сказал: отрока нет, а я, куда я денусь?
\end{tcolorbox}
\begin{tcolorbox}
\textsubscript{31} И взяли одежду Иосифа, и закололи козла, и вымарали одежду кровью;
\end{tcolorbox}
\begin{tcolorbox}
\textsubscript{32} и послали разноцветную одежду, и доставили к отцу своему, и сказали: мы это нашли; посмотри, сына ли твоего эта одежда, или нет.
\end{tcolorbox}
\begin{tcolorbox}
\textsubscript{33} Он узнал ее и сказал: [это] одежда сына моего; хищный зверь съел его; верно, растерзан Иосиф.
\end{tcolorbox}
\begin{tcolorbox}
\textsubscript{34} И разодрал Иаков одежды свои, и возложил вретище на чресла свои, и оплакивал сына своего многие дни.
\end{tcolorbox}
\begin{tcolorbox}
\textsubscript{35} И собрались все сыновья его и все дочери его, чтобы утешить его; но он не хотел утешиться и сказал: с печалью сойду к сыну моему в преисподнюю. Так оплакивал его отец его.
\end{tcolorbox}
\begin{tcolorbox}
\textsubscript{36} Мадианитяне же продали его в Египте Потифару, царедворцу фараонову, начальнику телохранителей.
\end{tcolorbox}
\subsection{CHAPTER 38}
\begin{tcolorbox}
\textsubscript{1} В то время Иуда отошел от братьев своих и поселился близ одного Одолламитянина, которому имя: Хира.
\end{tcolorbox}
\begin{tcolorbox}
\textsubscript{2} И увидел там Иуда дочь одного Хананеянина, которому имя: Шуа; и взял ее и вошел к ней.
\end{tcolorbox}
\begin{tcolorbox}
\textsubscript{3} Она зачала и родила сына; и он нарек ему имя: Ир.
\end{tcolorbox}
\begin{tcolorbox}
\textsubscript{4} И зачала опять, и родила сына, и нарекла ему имя: Онан.
\end{tcolorbox}
\begin{tcolorbox}
\textsubscript{5} И еще родила сына и нарекла ему имя: Шела. Иуда был в Хезиве, когда она родила его.
\end{tcolorbox}
\begin{tcolorbox}
\textsubscript{6} И взял Иуда жену Иру, первенцу своему; имя ей Фамарь.
\end{tcolorbox}
\begin{tcolorbox}
\textsubscript{7} Ир, первенец Иудин, был неугоден пред очами Господа, и умертвил его Господь.
\end{tcolorbox}
\begin{tcolorbox}
\textsubscript{8} И сказал Иуда Онану: войди к жене брата твоего, женись на ней, как деверь, и восстанови семя брату твоему.
\end{tcolorbox}
\begin{tcolorbox}
\textsubscript{9} Онан знал, что семя будет не ему, и потому, когда входил к жене брата своего, изливал на землю, чтобы не дать семени брату своему.
\end{tcolorbox}
\begin{tcolorbox}
\textsubscript{10} Зло было пред очами Господа то, что он делал; и Он умертвил и его.
\end{tcolorbox}
\begin{tcolorbox}
\textsubscript{11} И сказал Иуда Фамари, невестке своей: живи вдовою в доме отца твоего, пока подрастет Шела, сын мой. Ибо он сказал: не умер бы и он подобно братьям его. Фамарь пошла и стала жить в доме отца своего.
\end{tcolorbox}
\begin{tcolorbox}
\textsubscript{12} Прошло много времени, и умерла дочь Шуи, жена Иудина. Иуда, утешившись, пошел в Фамну к стригущим скот его, сам и Хира, друг его, Одолламитянин.
\end{tcolorbox}
\begin{tcolorbox}
\textsubscript{13} И уведомили Фамарь, говоря: вот, свекор твой идет в Фамну стричь скот свой.
\end{tcolorbox}
\begin{tcolorbox}
\textsubscript{14} И сняла она с себя одежду вдовства своего, покрыла себя покрывалом и, закрывшись, села у ворот Енаима, что на дороге в Фамну. Ибо видела, что Шела вырос, и она не дана ему в жену.
\end{tcolorbox}
\begin{tcolorbox}
\textsubscript{15} И увидел ее Иуда и почел ее за блудницу, потому что она закрыла лице свое.
\end{tcolorbox}
\begin{tcolorbox}
\textsubscript{16} Он поворотил к ней и сказал: войду я к тебе. Ибо не знал, что это невестка его. Она сказала: что ты дашь мне, если войдешь ко мне?
\end{tcolorbox}
\begin{tcolorbox}
\textsubscript{17} Он сказал: я пришлю тебе козленка из стада. Она сказала: дашь ли ты мне залог, пока пришлешь?
\end{tcolorbox}
\begin{tcolorbox}
\textsubscript{18} Он сказал: какой дать тебе залог? Она сказала: печать твою, и перевязь твою, и трость твою, которая в руке твоей. И дал он ей и вошел к ней; и она зачала от него.
\end{tcolorbox}
\begin{tcolorbox}
\textsubscript{19} И, встав, пошла, сняла с себя покрывало свое и оделась в одежду вдовства своего.
\end{tcolorbox}
\begin{tcolorbox}
\textsubscript{20} Иуда же послал козленка чрез друга своего Одолламитянина, чтобы взять залог из руки женщины, но он не нашел ее.
\end{tcolorbox}
\begin{tcolorbox}
\textsubscript{21} И спросил жителей того места, говоря: где блудница, [которая] [была] в Енаиме при дороге? Но они сказали: здесь не было блудницы.
\end{tcolorbox}
\begin{tcolorbox}
\textsubscript{22} И возвратился он к Иуде и сказал: я не нашел ее; да и жители места того сказали: здесь не было блудницы.
\end{tcolorbox}
\begin{tcolorbox}
\textsubscript{23} Иуда сказал: пусть она возьмет себе, чтобы только не стали над нами смеяться; вот, я посылал этого козленка, но ты не нашел ее.
\end{tcolorbox}
\begin{tcolorbox}
\textsubscript{24} Прошло около трех месяцев, и сказали Иуде, говоря: Фамарь, невестка твоя, впала в блуд, и вот, она беременна от блуда. Иуда сказал: выведите ее, и пусть она будет сожжена.
\end{tcolorbox}
\begin{tcolorbox}
\textsubscript{25} Но когда повели ее, она послала сказать свекру своему: я беременна от того, чьи эти вещи. И сказала: узнавай, чья эта печать и перевязь и трость.
\end{tcolorbox}
\begin{tcolorbox}
\textsubscript{26} Иуда узнал и сказал: она правее меня, потому что я не дал ее Шеле, сыну моему. И не познавал ее более.
\end{tcolorbox}
\begin{tcolorbox}
\textsubscript{27} Во время родов ее оказалось, что близнецы в утробе ее.
\end{tcolorbox}
\begin{tcolorbox}
\textsubscript{28} И во время родов ее показалась рука; и взяла повивальная бабка и навязала ему на руку красную нить, сказав: этот вышел первый.
\end{tcolorbox}
\begin{tcolorbox}
\textsubscript{29} Но он возвратил руку свою; и вот, вышел брат его. И она сказала: как ты расторг себе преграду? И наречено ему имя: Фарес.
\end{tcolorbox}
\begin{tcolorbox}
\textsubscript{30} Потом вышел брат его с красной нитью на руке. И наречено ему имя: Зара.
\end{tcolorbox}
\subsection{CHAPTER 39}
\begin{tcolorbox}
\textsubscript{1} Иосиф же отведен был в Египет, и купил его из рук Измаильтян, приведших его туда, Египтянин Потифар, царедворец фараонов, начальник телохранителей.
\end{tcolorbox}
\begin{tcolorbox}
\textsubscript{2} И был Господь с Иосифом: он был успешен в делах и жил в доме господина своего, Египтянина.
\end{tcolorbox}
\begin{tcolorbox}
\textsubscript{3} И увидел господин его, что Господь с ним и что всему, что он делает, Господь в руках его дает успех.
\end{tcolorbox}
\begin{tcolorbox}
\textsubscript{4} И снискал Иосиф благоволение в очах его и служил ему. И он поставил его над домом своим, и все, что имел, отдал на руки его.
\end{tcolorbox}
\begin{tcolorbox}
\textsubscript{5} И с того времени, как он поставил его над домом своим и над всем, что имел, Господь благословил дом Египтянина ради Иосифа, и было благословение Господне на всем, что имел он в доме и в поле.
\end{tcolorbox}
\begin{tcolorbox}
\textsubscript{6} И оставил он все, что имел, в руках Иосифа и не знал при нем ничего, кроме хлеба, который он ел. Иосиф же был красив станом и красив лицем.
\end{tcolorbox}
\begin{tcolorbox}
\textsubscript{7} И обратила взоры на Иосифа жена господина его и сказала: спи со мною.
\end{tcolorbox}
\begin{tcolorbox}
\textsubscript{8} Но он отказался и сказал жене господина своего: вот, господин мой не знает при мне ничего в доме, и все, что имеет, отдал в мои руки;
\end{tcolorbox}
\begin{tcolorbox}
\textsubscript{9} нет больше меня в доме сем; и он не запретил мне ничего, кроме тебя, потому что ты жена ему; как же сделаю я сие великое зло и согрешу пред Богом?
\end{tcolorbox}
\begin{tcolorbox}
\textsubscript{10} Когда так она ежедневно говорила Иосифу, а он не слушался ее, чтобы спать с нею и быть с нею,
\end{tcolorbox}
\begin{tcolorbox}
\textsubscript{11} случилось в один день, что он вошел в дом делать дело свое, а никого из домашних тут в доме не было;
\end{tcolorbox}
\begin{tcolorbox}
\textsubscript{12} она схватила его за одежду его и сказала: ложись со мной. Но он, оставив одежду свою в руках ее, побежал и выбежал вон.
\end{tcolorbox}
\begin{tcolorbox}
\textsubscript{13} Она же, увидев, что он оставил одежду свою в руках ее и побежал вон,
\end{tcolorbox}
\begin{tcolorbox}
\textsubscript{14} кликнула домашних своих и сказала им так: посмотрите, он привел к нам Еврея ругаться над нами. Он пришел ко мне, чтобы лечь со мною, но я закричала громким голосом,
\end{tcolorbox}
\begin{tcolorbox}
\textsubscript{15} и он, услышав, что я подняла вопль и закричала, оставил у меня одежду свою, и побежал, и выбежал вон.
\end{tcolorbox}
\begin{tcolorbox}
\textsubscript{16} И оставила одежду его у себя до прихода господина его в дом свой.
\end{tcolorbox}
\begin{tcolorbox}
\textsubscript{17} И пересказала ему те же слова, говоря: раб Еврей, которого ты привел к нам, приходил ко мне ругаться надо мною.
\end{tcolorbox}
\begin{tcolorbox}
\textsubscript{18} но, когда я подняла вопль и закричала, он оставил у меня одежду свою и убежал вон.
\end{tcolorbox}
\begin{tcolorbox}
\textsubscript{19} Когда господин его услышал слова жены своей, которые она сказала ему, говоря: так поступил со мною раб твой, то воспылал гневом;
\end{tcolorbox}
\begin{tcolorbox}
\textsubscript{20} и взял Иосифа господин его и отдал его в темницу, где заключены узники царя. И был он там в темнице.
\end{tcolorbox}
\begin{tcolorbox}
\textsubscript{21} И Господь был с Иосифом, и простер к нему милость, и даровал ему благоволение в очах начальника темницы.
\end{tcolorbox}
\begin{tcolorbox}
\textsubscript{22} И отдал начальник темницы в руки Иосифу всех узников, находившихся в темнице, и во всем, что они там ни делали, он был распорядителем.
\end{tcolorbox}
\begin{tcolorbox}
\textsubscript{23} Начальник темницы и не смотрел ни за чем, что было у него в руках, потому что Господь был с [Иосифом], и во всем, что он делал, Господь давал успех.
\end{tcolorbox}
\subsection{CHAPTER 40}
\begin{tcolorbox}
\textsubscript{1} После сего виночерпий царя Египетского и хлебодар провинились пред господином своим, царем Египетским.
\end{tcolorbox}
\begin{tcolorbox}
\textsubscript{2} И прогневался фараон на двух царедворцев своих, на главного виночерпия и на главного хлебодара,
\end{tcolorbox}
\begin{tcolorbox}
\textsubscript{3} и отдал их под стражу в дом начальника телохранителей, в темницу, в место, где заключен был Иосиф.
\end{tcolorbox}
\begin{tcolorbox}
\textsubscript{4} Начальник телохранителей приставил к ним Иосифа, и он служил им. И пробыли они под стражею несколько времени.
\end{tcolorbox}
\begin{tcolorbox}
\textsubscript{5} Однажды виночерпию и хлебодару царя Египетского, заключенным в темнице, виделись сны, каждому свой сон, обоим в одну ночь, каждому сон особенного значения.
\end{tcolorbox}
\begin{tcolorbox}
\textsubscript{6} И пришел к ним Иосиф поутру, увидел их, и вот, они в смущении.
\end{tcolorbox}
\begin{tcolorbox}
\textsubscript{7} И спросил он царедворцев фараоновых, находившихся с ним в доме господина его под стражею, говоря: отчего у вас сегодня печальные лица?
\end{tcolorbox}
\begin{tcolorbox}
\textsubscript{8} Они сказали ему: нам виделись сны; а истолковать их некому. Иосиф сказал им: не от Бога ли истолкования? расскажите мне.
\end{tcolorbox}
\begin{tcolorbox}
\textsubscript{9} И рассказал главный виночерпий Иосифу сон свой и сказал ему: мне снилось, вот виноградная лоза предо мною;
\end{tcolorbox}
\begin{tcolorbox}
\textsubscript{10} на лозе три ветви; она развилась, показался на ней цвет, выросли и созрели на ней ягоды;
\end{tcolorbox}
\begin{tcolorbox}
\textsubscript{11} и чаша фараонова в руке у меня; я взял ягод, выжал их в чашу фараонову и подал чашу в руку фараону.
\end{tcolorbox}
\begin{tcolorbox}
\textsubscript{12} И сказал ему Иосиф: вот истолкование его: три ветви--это три дня;
\end{tcolorbox}
\begin{tcolorbox}
\textsubscript{13} через три дня фараон вознесет главу твою и возвратит тебя на место твое, и ты подашь чашу фараонову в руку его, по прежнему обыкновению, когда ты был у него виночерпием;
\end{tcolorbox}
\begin{tcolorbox}
\textsubscript{14} вспомни же меня, когда хорошо тебе будет, и сделай мне благодеяние, и упомяни обо мне фараону, и выведи меня из этого дома,
\end{tcolorbox}
\begin{tcolorbox}
\textsubscript{15} ибо я украден из земли Евреев; а также и здесь ничего не сделал, за что бы бросить меня в темницу.
\end{tcolorbox}
\begin{tcolorbox}
\textsubscript{16} Главный хлебодар увидел, что истолковал он хорошо, и сказал Иосифу: мне также снилось: вот на голове у меня три корзины решетчатых;
\end{tcolorbox}
\begin{tcolorbox}
\textsubscript{17} в верхней корзине всякая пища фараонова, изделие пекаря, и птицы клевали ее из корзины на голове моей.
\end{tcolorbox}
\begin{tcolorbox}
\textsubscript{18} И отвечал Иосиф и сказал: вот истолкование его: три корзины--это три дня;
\end{tcolorbox}
\begin{tcolorbox}
\textsubscript{19} через три дня фараон снимет с тебя голову твою и повесит тебя на дереве, и птицы будут клевать плоть твою с тебя.
\end{tcolorbox}
\begin{tcolorbox}
\textsubscript{20} На третий день, день рождения фараонова, сделал он пир для всех слуг своих и вспомнил о главном виночерпии и главном хлебодаре среди слуг своих;
\end{tcolorbox}
\begin{tcolorbox}
\textsubscript{21} и возвратил главного виночерпия на прежнее место, и он подал чашу в руку фараону,
\end{tcolorbox}
\begin{tcolorbox}
\textsubscript{22} а главного хлебодара повесил, как истолковал им Иосиф.
\end{tcolorbox}
\begin{tcolorbox}
\textsubscript{23} И не вспомнил главный виночерпий об Иосифе, но забыл его.
\end{tcolorbox}
\subsection{CHAPTER 41}
\begin{tcolorbox}
\textsubscript{1} По прошествии двух лет фараону снилось: вот, он стоит у реки;
\end{tcolorbox}
\begin{tcolorbox}
\textsubscript{2} и вот, вышли из реки семь коров, хороших видом и тучных плотью, и паслись в тростнике;
\end{tcolorbox}
\begin{tcolorbox}
\textsubscript{3} но вот, после них вышли из реки семь коров других, худых видом и тощих плотью, и стали подле тех коров, на берегу реки;
\end{tcolorbox}
\begin{tcolorbox}
\textsubscript{4} и съели коровы худые видом и тощие плотью семь коров хороших видом и тучных. И проснулся фараон,
\end{tcolorbox}
\begin{tcolorbox}
\textsubscript{5} и заснул опять, и снилось ему в другой раз: вот, на одном стебле поднялось семь колосьев тучных и хороших;
\end{tcolorbox}
\begin{tcolorbox}
\textsubscript{6} но вот, после них выросло семь колосьев тощих и иссушенных восточным ветром;
\end{tcolorbox}
\begin{tcolorbox}
\textsubscript{7} и пожрали тощие колосья семь колосьев тучных и полных. И проснулся фараон и [понял, что] это сон.
\end{tcolorbox}
\begin{tcolorbox}
\textsubscript{8} Утром смутился дух его, и послал он, и призвал всех волхвов Египта и всех мудрецов его, и рассказал им фараон сон свой; но не было никого, кто бы истолковал его фараону.
\end{tcolorbox}
\begin{tcolorbox}
\textsubscript{9} И стал говорить главный виночерпий фараону и сказал: грехи мои вспоминаю я ныне;
\end{tcolorbox}
\begin{tcolorbox}
\textsubscript{10} фараон прогневался на рабов своих и отдал меня и главного хлебодара под стражу в дом начальника телохранителей;
\end{tcolorbox}
\begin{tcolorbox}
\textsubscript{11} и снился нам сон в одну ночь, мне и ему, каждому снился сон особенного значения;
\end{tcolorbox}
\begin{tcolorbox}
\textsubscript{12} там же был с нами молодой Еврей, раб начальника телохранителей; мы рассказали ему сны наши, и он истолковал нам каждому соответственно с его сновидением;
\end{tcolorbox}
\begin{tcolorbox}
\textsubscript{13} и как он истолковал нам, так и сбылось: я возвращен на место мое, а тот повешен.
\end{tcolorbox}
\begin{tcolorbox}
\textsubscript{14} И послал фараон и позвал Иосифа. И поспешно вывели его из темницы. Он остригся и переменил одежду свою и пришел к фараону.
\end{tcolorbox}
\begin{tcolorbox}
\textsubscript{15} Фараон сказал Иосифу: мне снился сон, и нет никого, кто бы истолковал его, а о тебе я слышал, что ты умеешь толковать сны.
\end{tcolorbox}
\begin{tcolorbox}
\textsubscript{16} И отвечал Иосиф фараону, говоря: это не мое; Бог даст ответ во благо фараону.
\end{tcolorbox}
\begin{tcolorbox}
\textsubscript{17} И сказал фараон Иосифу: мне снилось: вот, стою я на берегу реки;
\end{tcolorbox}
\begin{tcolorbox}
\textsubscript{18} и вот, вышли из реки семь коров тучных плотью и хороших видом и паслись в тростнике;
\end{tcolorbox}
\begin{tcolorbox}
\textsubscript{19} но вот, после них вышли семь коров других, худых, очень дурных видом и тощих плотью: я не видывал во всей земле Египетской таких худых, как они;
\end{tcolorbox}
\begin{tcolorbox}
\textsubscript{20} и съели тощие и худые коровы прежних семь коров тучных;
\end{tcolorbox}
\begin{tcolorbox}
\textsubscript{21} и вошли [тучные] в утробу их, но не приметно было, что они вошли в утробу их: они были так же худы видом, как и сначала. И я проснулся.
\end{tcolorbox}
\begin{tcolorbox}
\textsubscript{22} [Потом] снилось мне: вот, на одном стебле поднялись семь колосьев полных и хороших;
\end{tcolorbox}
\begin{tcolorbox}
\textsubscript{23} но вот, после них выросло семь колосьев тонких, тощих и иссушенных восточным ветром;
\end{tcolorbox}
\begin{tcolorbox}
\textsubscript{24} и пожрали тощие колосья семь колосьев хороших. Я рассказал это волхвам, но никто не изъяснил мне.
\end{tcolorbox}
\begin{tcolorbox}
\textsubscript{25} И сказал Иосиф фараону: сон фараонов один: что Бог сделает, то Он возвестил фараону.
\end{tcolorbox}
\begin{tcolorbox}
\textsubscript{26} Семь коров хороших, это семь лет; и семь колосьев хороших, это семь лет: сон один;
\end{tcolorbox}
\begin{tcolorbox}
\textsubscript{27} и семь коров тощих и худых, вышедших после тех, это семь лет, также и семь колосьев тощих и иссушенных восточным ветром, это семь лет голода.
\end{tcolorbox}
\begin{tcolorbox}
\textsubscript{28} Вот почему сказал я фараону: что Бог сделает, то Он показал фараону.
\end{tcolorbox}
\begin{tcolorbox}
\textsubscript{29} Вот, наступает семь лет великого изобилия во всей земле Египетской;
\end{tcolorbox}
\begin{tcolorbox}
\textsubscript{30} после них настанут семь лет голода, и забудется все то изобилие в земле Египетской, и истощит голод землю,
\end{tcolorbox}
\begin{tcolorbox}
\textsubscript{31} и неприметно будет прежнее изобилие на земле, по причине голода, который последует, ибо он будет очень тяжел.
\end{tcolorbox}
\begin{tcolorbox}
\textsubscript{32} А что сон повторился фараону дважды, [это значит], что сие истинно слово Божие, и что вскоре Бог исполнит сие.
\end{tcolorbox}
\begin{tcolorbox}
\textsubscript{33} И ныне да усмотрит фараон мужа разумного и мудрого и да поставит его над землею Египетскою.
\end{tcolorbox}
\begin{tcolorbox}
\textsubscript{34} Да повелит фараон поставить над землею надзирателей и собирать в семь лет изобилия пятую часть с земли Египетской;
\end{tcolorbox}
\begin{tcolorbox}
\textsubscript{35} пусть они берут всякий хлеб этих наступающих хороших годов и соберут в городах хлеб под ведение фараона в пищу, и пусть берегут;
\end{tcolorbox}
\begin{tcolorbox}
\textsubscript{36} и будет сия пища в запас для земли на семь лет голода, которые будут в земле Египетской, дабы земля не погибла от голода.
\end{tcolorbox}
\begin{tcolorbox}
\textsubscript{37} Сие понравилось фараону и всем слугам его.
\end{tcolorbox}
\begin{tcolorbox}
\textsubscript{38} И сказал фараон слугам своим: найдем ли мы такого, как он, человека, в котором был бы Дух Божий?
\end{tcolorbox}
\begin{tcolorbox}
\textsubscript{39} И сказал фараон Иосифу: так как Бог открыл тебе все сие, то нет столь разумного и мудрого, как ты;
\end{tcolorbox}
\begin{tcolorbox}
\textsubscript{40} ты будешь над домом моим, и твоего слова держаться будет весь народ мой; только престолом я буду больше тебя.
\end{tcolorbox}
\begin{tcolorbox}
\textsubscript{41} И сказал фараон Иосифу: вот, я поставляю тебя над всею землею Египетскою.
\end{tcolorbox}
\begin{tcolorbox}
\textsubscript{42} И снял фараон перстень свой с руки своей и надел его на руку Иосифа; одел его в виссонные одежды, возложил золотую цепь на шею ему;
\end{tcolorbox}
\begin{tcolorbox}
\textsubscript{43} велел везти его на второй из своих колесниц и провозглашать пред ним: преклоняйтесь! И поставил его над всею землею Египетскою.
\end{tcolorbox}
\begin{tcolorbox}
\textsubscript{44} И сказал фараон Иосифу: я фараон; без тебя никто не двинет ни руки своей, ни ноги своей во всей земле Египетской.
\end{tcolorbox}
\begin{tcolorbox}
\textsubscript{45} И нарек фараон Иосифу имя: Цафнаф-панеах, и дал ему в жену Асенефу, дочь Потифера, жреца Илиопольского. И пошел Иосиф по земле Египетской.
\end{tcolorbox}
\begin{tcolorbox}
\textsubscript{46} Иосифу было тридцать лет от рождения, когда он предстал пред лице фараона, царя Египетского. И вышел Иосиф от лица фараонова и прошел по всей земле Египетской.
\end{tcolorbox}
\begin{tcolorbox}
\textsubscript{47} Земля же в семь лет изобилия приносила [из зерна] по горсти.
\end{tcolorbox}
\begin{tcolorbox}
\textsubscript{48} И собрал он всякий хлеб семи лет, которые были [плодородны] в земле Египетской, и положил хлеб в городах; в [каждом] городе положил хлеб полей, окружающих его.
\end{tcolorbox}
\begin{tcolorbox}
\textsubscript{49} И скопил Иосиф хлеба весьма много, как песку морского, так что перестал и считать, ибо не стало счета.
\end{tcolorbox}
\begin{tcolorbox}
\textsubscript{50} До наступления годов голода, у Иосифа родились два сына, которых родила ему Асенефа, дочь Потифера, жреца Илиопольского.
\end{tcolorbox}
\begin{tcolorbox}
\textsubscript{51} И нарек Иосиф имя первенцу: Манассия, потому что [говорил он] Бог дал мне забыть все несчастья мои и весь дом отца моего.
\end{tcolorbox}
\begin{tcolorbox}
\textsubscript{52} А другому нарек имя: Ефрем, потому что [говорил он] Бог сделал меня плодовитым в земле страдания моего.
\end{tcolorbox}
\begin{tcolorbox}
\textsubscript{53} И прошли семь лет изобилия, которое было в земле Египетской,
\end{tcolorbox}
\begin{tcolorbox}
\textsubscript{54} и наступили семь лет голода, как сказал Иосиф. И был голод во всех землях, а во всей земле Египетской был хлеб.
\end{tcolorbox}
\begin{tcolorbox}
\textsubscript{55} Но когда и вся земля Египетская начала терпеть голод, то народ начал вопиять к фараону о хлебе. И сказал фараон всем Египтянам: пойдите к Иосифу и делайте, что он вам скажет.
\end{tcolorbox}
\begin{tcolorbox}
\textsubscript{56} И был голод по всей земле; и отворил Иосиф все житницы, и стал продавать хлеб Египтянам. Голод же усиливался в земле Египетской.
\end{tcolorbox}
\begin{tcolorbox}
\textsubscript{57} И из всех стран приходили в Египет покупать хлеб у Иосифа, ибо голод усилился по всей земле.
\end{tcolorbox}
\subsection{CHAPTER 42}
\begin{tcolorbox}
\textsubscript{1} И узнал Иаков, что в Египте есть хлеб, и сказал Иаков сыновьям своим: что вы смотрите?
\end{tcolorbox}
\begin{tcolorbox}
\textsubscript{2} И сказал: вот, я слышал, что есть хлеб в Египте; пойдите туда и купите нам оттуда хлеба, чтобы нам жить и не умереть.
\end{tcolorbox}
\begin{tcolorbox}
\textsubscript{3} Десять братьев Иосифовых пошли купить хлеба в Египте,
\end{tcolorbox}
\begin{tcolorbox}
\textsubscript{4} а Вениамина, брата Иосифова, не послал Иаков с братьями его, ибо сказал: не случилось бы с ним беды.
\end{tcolorbox}
\begin{tcolorbox}
\textsubscript{5} И пришли сыны Израилевы покупать хлеб, вместе с другими пришедшими, ибо в земле Ханаанской был голод.
\end{tcolorbox}
\begin{tcolorbox}
\textsubscript{6} Иосиф же был начальником в земле той; он и продавал хлеб всему народу земли. Братья Иосифа пришли и поклонились ему лицем до земли.
\end{tcolorbox}
\begin{tcolorbox}
\textsubscript{7} И увидел Иосиф братьев своих и узнал их; но показал, будто не знает их, и говорил с ними сурово и сказал им: откуда вы пришли? Они сказали: из земли Ханаанской, купить пищи.
\end{tcolorbox}
\begin{tcolorbox}
\textsubscript{8} Иосиф узнал братьев своих, но они не узнали его.
\end{tcolorbox}
\begin{tcolorbox}
\textsubscript{9} И вспомнил Иосиф сны, которые снились ему о них; и сказал им: вы соглядатаи, вы пришли высмотреть наготу земли сей.
\end{tcolorbox}
\begin{tcolorbox}
\textsubscript{10} Они сказали ему: нет, господин наш; рабы твои пришли купить пищи;
\end{tcolorbox}
\begin{tcolorbox}
\textsubscript{11} мы все дети одного человека; мы люди честные; рабы твои не бывали соглядатаями.
\end{tcolorbox}
\begin{tcolorbox}
\textsubscript{12} Он сказал им: нет, вы пришли высмотреть наготу земли сей.
\end{tcolorbox}
\begin{tcolorbox}
\textsubscript{13} Они сказали: нас, рабов твоих, двенадцать братьев; мы сыновья одного человека в земле Ханаанской, и вот, меньший теперь с отцом нашим, а одного не стало.
\end{tcolorbox}
\begin{tcolorbox}
\textsubscript{14} И сказал им Иосиф: это самое я и говорил вам, сказав: вы соглядатаи;
\end{tcolorbox}
\begin{tcolorbox}
\textsubscript{15} вот как вы будете испытаны: [клянусь] жизнью фараона, вы не выйдете отсюда, если не придет сюда меньший брат ваш;
\end{tcolorbox}
\begin{tcolorbox}
\textsubscript{16} пошлите одного из вас, и пусть он приведет брата вашего, а вы будете задержаны; и откроется, правда ли у вас; и если нет, [то клянусь] жизнью фараона, что вы соглядатаи.
\end{tcolorbox}
\begin{tcolorbox}
\textsubscript{17} И отдал их под стражу на три дня.
\end{tcolorbox}
\begin{tcolorbox}
\textsubscript{18} И сказал им Иосиф в третий день: вот что сделайте, и останетесь живы, ибо я боюсь Бога:
\end{tcolorbox}
\begin{tcolorbox}
\textsubscript{19} если вы люди честные, то один брат из вас пусть содержится в доме, где вы заключены; а вы пойдите, отвезите хлеб, ради голода семейств ваших;
\end{tcolorbox}
\begin{tcolorbox}
\textsubscript{20} брата же вашего меньшого приведите ко мне, чтобы оправдались слова ваши и чтобы не умереть вам. Так они и сделали.
\end{tcolorbox}
\begin{tcolorbox}
\textsubscript{21} И говорили они друг другу: точно мы наказываемся за грех против брата нашего; мы видели страдание души его, когда он умолял нас, но не послушали; за то и постигло нас горе сие.
\end{tcolorbox}
\begin{tcolorbox}
\textsubscript{22} Рувим отвечал им и сказал: не говорил ли я вам: не грешите против отрока? но вы не послушались; вот, кровь его взыскивается.
\end{tcolorbox}
\begin{tcolorbox}
\textsubscript{23} А того не знали они, что Иосиф понимает; ибо между ними был переводчик.
\end{tcolorbox}
\begin{tcolorbox}
\textsubscript{24} И отошел от них, и заплакал. И возвратился к ним, и говорил с ними, и, взяв из них Симеона, связал его пред глазами их.
\end{tcolorbox}
\begin{tcolorbox}
\textsubscript{25} И приказал Иосиф наполнить мешки их хлебом, а серебро их возвратить каждому в мешок его, и дать им запасов на дорогу. Так и сделано с ними.
\end{tcolorbox}
\begin{tcolorbox}
\textsubscript{26} Они положили хлеб свой на ослов своих, и пошли оттуда.
\end{tcolorbox}
\begin{tcolorbox}
\textsubscript{27} И открыл один [из них] мешок свой, чтобы дать корму ослу своему на ночлеге, и увидел серебро свое в отверстии мешка его,
\end{tcolorbox}
\begin{tcolorbox}
\textsubscript{28} и сказал своим братьям: серебро мое возвращено; вот оно в мешке у меня. И смутилось сердце их, и они с трепетом друг другу говорили: что это Бог сделал с нами?
\end{tcolorbox}
\begin{tcolorbox}
\textsubscript{29} И пришли к Иакову, отцу своему, в землю Ханаанскую и рассказали ему всё случившееся с ними, говоря:
\end{tcolorbox}
\begin{tcolorbox}
\textsubscript{30} начальствующий над тою землею говорил с нами сурово и принял нас за соглядатаев земли той.
\end{tcolorbox}
\begin{tcolorbox}
\textsubscript{31} И сказали мы ему: мы люди честные; мы не бывали соглядатаями;
\end{tcolorbox}
\begin{tcolorbox}
\textsubscript{32} нас двенадцать братьев, сыновей у отца нашего; одного не стало, а меньший теперь с отцом нашим в земле Ханаанской.
\end{tcolorbox}
\begin{tcolorbox}
\textsubscript{33} И сказал нам начальствующий над тою землею: вот как узнаю я, честные ли вы люди: оставьте у меня одного брата из вас, а вы возьмите хлеб ради голода семейств ваших и пойдите,
\end{tcolorbox}
\begin{tcolorbox}
\textsubscript{34} и приведите ко мне меньшого брата вашего; и узнаю я, что вы не соглядатаи, но люди честные; отдам вам брата вашего, и вы можете промышлять в этой земле.
\end{tcolorbox}
\begin{tcolorbox}
\textsubscript{35} Когда же они опорожняли мешки свои, вот, у каждого узел серебра его в мешке его. И увидели они узлы серебра своего, они и отец их, и испугались.
\end{tcolorbox}
\begin{tcolorbox}
\textsubscript{36} И сказал им Иаков, отец их: вы лишили меня детей: Иосифа нет, и Симеона нет, и Вениамина взять хотите, --все это на меня!
\end{tcolorbox}
\begin{tcolorbox}
\textsubscript{37} И сказал Рувим отцу своему, говоря: убей двух моих сыновей, если я не приведу его к тебе; отдай его на мои руки; я возвращу его тебе.
\end{tcolorbox}
\begin{tcolorbox}
\textsubscript{38} Он сказал: не пойдет сын мой с вами; потому что брат его умер, и он один остался; если случится с ним несчастье на пути, в который вы пойдете, то сведете вы седину мою с печалью во гроб.
\end{tcolorbox}
\subsection{CHAPTER 43}
\begin{tcolorbox}
\textsubscript{1} Голод усилился на земле.
\end{tcolorbox}
\begin{tcolorbox}
\textsubscript{2} И когда они съели хлеб, который привезли из Египта, тогда отец их сказал им: пойдите опять, купите нам немного пищи.
\end{tcolorbox}
\begin{tcolorbox}
\textsubscript{3} И сказал ему Иуда, говоря: тот человек решительно объявил нам, сказав: не являйтесь ко мне на лице, если брата вашего не будет с вами.
\end{tcolorbox}
\begin{tcolorbox}
\textsubscript{4} Если пошлешь с нами брата нашего, то пойдем и купим тебе пищи,
\end{tcolorbox}
\begin{tcolorbox}
\textsubscript{5} а если не пошлешь, то не пойдем, ибо тот человек сказал нам: не являйтесь ко мне на лице, если брата вашего не будет с вами.
\end{tcolorbox}
\begin{tcolorbox}
\textsubscript{6} Израиль сказал: для чего вы сделали мне такое зло, сказав тому человеку, что у вас есть еще брат?
\end{tcolorbox}
\begin{tcolorbox}
\textsubscript{7} Они сказали: расспрашивал тот человек о нас и о родстве нашем, говоря: жив ли еще отец ваш? есть ли у вас брат? Мы и рассказали ему по этим расспросам. Могли ли мы знать, что он скажет: приведите брата вашего?
\end{tcolorbox}
\begin{tcolorbox}
\textsubscript{8} Иуда же сказал Израилю, отцу своему: отпусти отрока со мною, и мы встанем и пойдем, и живы будем и не умрем и мы, и ты, и дети наши;
\end{tcolorbox}
\begin{tcolorbox}
\textsubscript{9} я отвечаю за него, из моих рук потребуешь его; если я не приведу его к тебе и не поставлю его пред лицем твоим, то останусь я виновным пред тобою во все дни жизни;
\end{tcolorbox}
\begin{tcolorbox}
\textsubscript{10} если бы мы не медлили, то уже сходили бы два раза.
\end{tcolorbox}
\begin{tcolorbox}
\textsubscript{11} Израиль, отец их, сказал им: если так, то вот что сделайте: возьмите с собою плодов земли сей и отнесите в дар тому человеку несколько бальзама и несколько меду, стираксы и ладану, фисташков и миндальных орехов;
\end{tcolorbox}
\begin{tcolorbox}
\textsubscript{12} возьмите и другое серебро в руки ваши; а серебро, обратно положенное в отверстие мешков ваших, возвратите руками вашими: может быть, это недосмотр;
\end{tcolorbox}
\begin{tcolorbox}
\textsubscript{13} и брата вашего возьмите и, встав, пойдите опять к человеку тому;
\end{tcolorbox}
\begin{tcolorbox}
\textsubscript{14} Бог же Всемогущий да даст вам найти милость у человека того, чтобы он отпустил вам и другого брата вашего и Вениамина, а мне если уже быть бездетным, то пусть буду бездетным.
\end{tcolorbox}
\begin{tcolorbox}
\textsubscript{15} И взяли те люди дары эти, и серебра вдвое взяли в руки свои, и Вениамина, и встали, пошли в Египет и предстали пред лице Иосифа.
\end{tcolorbox}
\begin{tcolorbox}
\textsubscript{16} Иосиф, увидев между ними Вениамина, сказал начальнику дома своего: введи сих людей в дом и заколи что-нибудь из скота, и приготовь, потому что со мною будут есть эти люди в полдень.
\end{tcolorbox}
\begin{tcolorbox}
\textsubscript{17} И сделал человек тот, как сказал Иосиф, и ввел человек тот людей сих в дом Иосифов.
\end{tcolorbox}
\begin{tcolorbox}
\textsubscript{18} И испугались люди эти, что ввели их в дом Иосифов, и сказали: это за серебро, возвращенное прежде в мешки наши, ввели нас, чтобы придраться к нам и напасть на нас, и взять нас в рабство, и ослов наших.
\end{tcolorbox}
\begin{tcolorbox}
\textsubscript{19} И подошли они к начальнику дома Иосифова, и стали говорить ему у дверей дома,
\end{tcolorbox}
\begin{tcolorbox}
\textsubscript{20} и сказали: послушай, господин наш, мы приходили уже прежде покупать пищи,
\end{tcolorbox}
\begin{tcolorbox}
\textsubscript{21} и случилось, что, когда пришли мы на ночлег и открыли мешки наши, --вот серебро каждого в отверстии мешка его, серебро наше по весу его, и мы возвращаем его своими руками;
\end{tcolorbox}
\begin{tcolorbox}
\textsubscript{22} а для покупки пищи мы принесли другое серебро в руках наших, мы не знаем, кто положил серебро наше в мешки наши.
\end{tcolorbox}
\begin{tcolorbox}
\textsubscript{23} Он сказал: будьте спокойны, не бойтесь; Бог ваш и Бог отца вашего дал вам клад в мешках ваших; серебро ваше дошло до меня. И привел к ним Симеона.
\end{tcolorbox}
\begin{tcolorbox}
\textsubscript{24} И ввел тот человек людей сих в дом Иосифов и дал воды, и они омыли ноги свои; и дал корму ослам их.
\end{tcolorbox}
\begin{tcolorbox}
\textsubscript{25} И они приготовили дары к приходу Иосифа в полдень, ибо слышали, что там будут есть хлеб.
\end{tcolorbox}
\begin{tcolorbox}
\textsubscript{26} И пришел Иосиф домой; и они принесли ему в дом дары, которые были на руках их, и поклонились ему до земли.
\end{tcolorbox}
\begin{tcolorbox}
\textsubscript{27} Он спросил их о здоровье и сказал: здоров ли отец ваш старец, о котором вы говорили? жив ли еще он?
\end{tcolorbox}
\begin{tcolorbox}
\textsubscript{28} Они сказали: здоров раб твой, отец наш; еще жив. И преклонились они и поклонились.
\end{tcolorbox}
\begin{tcolorbox}
\textsubscript{29} И поднял глаза свои, и увидел Вениамина, брата своего, сына матери своей, и сказал: это брат ваш меньший, о котором вы сказывали мне? И сказал: да будет милость Божия с тобою, сын мой!
\end{tcolorbox}
\begin{tcolorbox}
\textsubscript{30} И поспешно удалился Иосиф, потому что воскипела любовь к брату его, и он готов был заплакать, и вошел он во внутреннюю комнату и плакал там.
\end{tcolorbox}
\begin{tcolorbox}
\textsubscript{31} И умыв лице свое, вышел, и скрепился и сказал: подавайте кушанье.
\end{tcolorbox}
\begin{tcolorbox}
\textsubscript{32} И подали ему особо, и им особо, и Египтянам, обедавшим с ним, особо, ибо Египтяне не могут есть с Евреями, потому что это мерзость для Египтян.
\end{tcolorbox}
\begin{tcolorbox}
\textsubscript{33} И сели они пред ним, первородный по первородству его, и младший по молодости его, и дивились эти люди друг пред другом.
\end{tcolorbox}
\begin{tcolorbox}
\textsubscript{34} И посылались им кушанья от него, и доля Вениамина была впятеро больше долей каждого из них. И пили, и довольно пили они с ним.
\end{tcolorbox}
\subsection{CHAPTER 44}
\begin{tcolorbox}
\textsubscript{1} И приказал [Иосиф] начальнику дома своего, говоря: наполни мешки этих людей пищею, сколько они могут нести, и серебро каждого положи в отверстие мешка его,
\end{tcolorbox}
\begin{tcolorbox}
\textsubscript{2} а чашу мою, чашу серебряную, положи в отверстие мешка к младшему вместе с серебром за купленный им хлеб. И сделал тот по слову Иосифа, которое сказал он.
\end{tcolorbox}
\begin{tcolorbox}
\textsubscript{3} Утром, когда рассвело, эти люди были отпущены, они и ослы их.
\end{tcolorbox}
\begin{tcolorbox}
\textsubscript{4} Еще не далеко отошли они от города, как Иосиф сказал начальнику дома своего: ступай, догоняй этих людей и, когда догонишь, скажи им: для чего вы заплатили злом за добро?
\end{tcolorbox}
\begin{tcolorbox}
\textsubscript{5} Не та ли это, из которой пьет господин мой и он гадает на ней? Худо это вы сделали.
\end{tcolorbox}
\begin{tcolorbox}
\textsubscript{6} Он догнал их и сказал им эти слова.
\end{tcolorbox}
\begin{tcolorbox}
\textsubscript{7} Они сказали ему: для чего господин наш говорит такие слова? Нет, рабы твои не сделают такого дела.
\end{tcolorbox}
\begin{tcolorbox}
\textsubscript{8} Вот, серебро, найденное нами в отверстии мешков наших, мы обратно принесли тебе из земли Ханаанской: как же нам украсть из дома господина твоего серебро или золото?
\end{tcolorbox}
\begin{tcolorbox}
\textsubscript{9} У кого из рабов твоих найдется, тому смерть, и мы будем рабами господину нашему.
\end{tcolorbox}
\begin{tcolorbox}
\textsubscript{10} Он сказал: хорошо; как вы сказали, так пусть и будет: у кого найдется [чаша], тот будет мне рабом, а вы будете не виноваты.
\end{tcolorbox}
\begin{tcolorbox}
\textsubscript{11} Они поспешно спустили каждый свой мешок на землю и открыли каждый свой мешок.
\end{tcolorbox}
\begin{tcolorbox}
\textsubscript{12} Он обыскал, начал со старшего и окончил младшим; и нашлась чаша в мешке Вениаминовом.
\end{tcolorbox}
\begin{tcolorbox}
\textsubscript{13} И разодрали они одежды свои, и, возложив каждый на осла своего ношу, возвратились в город.
\end{tcolorbox}
\begin{tcolorbox}
\textsubscript{14} И пришли Иуда и братья его в дом Иосифа, который был еще дома, и пали пред ним на землю.
\end{tcolorbox}
\begin{tcolorbox}
\textsubscript{15} Иосиф сказал им: что это вы сделали? разве вы не знали, что такой человек, как я, конечно угадает?
\end{tcolorbox}
\begin{tcolorbox}
\textsubscript{16} Иуда сказал: что нам сказать господину нашему? что говорить? чем оправдываться? Бог нашел неправду рабов твоих; вот, мы рабы господину нашему, и мы, и тот, в чьих руках нашлась чаша.
\end{tcolorbox}
\begin{tcolorbox}
\textsubscript{17} Но [Иосиф] сказал: нет, я этого не сделаю; тот, в чьих руках нашлась чаша, будет мне рабом, а вы пойдите с миром к отцу вашему.
\end{tcolorbox}
\begin{tcolorbox}
\textsubscript{18} И подошел Иуда к нему и сказал: господин мой, позволь рабу твоему сказать слово в уши господина моего, и не прогневайся на раба твоего, ибо ты то же, что фараон.
\end{tcolorbox}
\begin{tcolorbox}
\textsubscript{19} Господин мой спрашивал рабов своих, говоря: есть ли у вас отец или брат?
\end{tcolorbox}
\begin{tcolorbox}
\textsubscript{20} Мы сказали господину нашему, что у нас есть отец престарелый, и младший сын, сын старости, которого брат умер, а он остался один [от] матери своей, и отец любит его.
\end{tcolorbox}
\begin{tcolorbox}
\textsubscript{21} Ты же сказал рабам твоим: приведите его ко мне, чтобы мне взглянуть на него.
\end{tcolorbox}
\begin{tcolorbox}
\textsubscript{22} Мы сказали господину нашему: отрок не может оставить отца своего, и если он оставит отца своего, то сей умрет.
\end{tcolorbox}
\begin{tcolorbox}
\textsubscript{23} Но ты сказал рабам твоим: если не придет с вами меньший брат ваш, то вы более не являйтесь ко мне на лице.
\end{tcolorbox}
\begin{tcolorbox}
\textsubscript{24} Когда мы пришли к рабу твоему, отцу нашему, то пересказали ему слова господина моего.
\end{tcolorbox}
\begin{tcolorbox}
\textsubscript{25} И сказал отец наш: пойдите опять, купите нам немного пищи.
\end{tcolorbox}
\begin{tcolorbox}
\textsubscript{26} Мы сказали: нельзя нам идти; а если будет с нами меньший брат наш, то пойдем; потому что нельзя нам видеть лица того человека, если не будет с нами меньшого брата нашего.
\end{tcolorbox}
\begin{tcolorbox}
\textsubscript{27} И сказал нам раб твой, отец наш: вы знаете, что жена моя родила мне двух [сынов];
\end{tcolorbox}
\begin{tcolorbox}
\textsubscript{28} один пошел от меня, и я сказал: верно он растерзан; и я не видал его доныне;
\end{tcolorbox}
\begin{tcolorbox}
\textsubscript{29} если и сего возьмете от глаз моих, и случится с ним несчастье, то сведете вы седину мою с горестью во гроб.
\end{tcolorbox}
\begin{tcolorbox}
\textsubscript{30} Теперь если я приду к рабу твоему, отцу нашему, и не будет с нами отрока, с душею которого связана душа его,
\end{tcolorbox}
\begin{tcolorbox}
\textsubscript{31} то он, увидев, что нет отрока, умрет; и сведут рабы твои седину раба твоего, отца нашего, с печалью во гроб.
\end{tcolorbox}
\begin{tcolorbox}
\textsubscript{32} Притом я, раб твой, взялся отвечать за отрока отцу моему, сказав: если не приведу его к тебе, то останусь я виновным пред отцом моим во все дни жизни.
\end{tcolorbox}
\begin{tcolorbox}
\textsubscript{33} Итак пусть я, раб твой, вместо отрока останусь рабом у господина моего, а отрок пусть идет с братьями своими:
\end{tcolorbox}
\begin{tcolorbox}
\textsubscript{34} ибо как пойду я к отцу моему, когда отрока не будет со мною? я увидел бы бедствие, которое постигло бы отца моего.
\end{tcolorbox}
\subsection{CHAPTER 45}
\begin{tcolorbox}
\textsubscript{1} Иосиф не мог более удерживаться при всех стоявших около него и закричал: удалите от меня всех. И не оставалось при Иосифе никого, когда он открылся братьям своим.
\end{tcolorbox}
\begin{tcolorbox}
\textsubscript{2} И громко зарыдал он, и услышали Египтяне, и услышал дом фараонов.
\end{tcolorbox}
\begin{tcolorbox}
\textsubscript{3} И сказал Иосиф братьям своим: я--Иосиф, жив ли еще отец мой? Но братья его не могли отвечать ему, потому что они смутились пред ним.
\end{tcolorbox}
\begin{tcolorbox}
\textsubscript{4} И сказал Иосиф братьям своим: подойдите ко мне. Они подошли. Он сказал: я--Иосиф, брат ваш, которого вы продали в Египет;
\end{tcolorbox}
\begin{tcolorbox}
\textsubscript{5} но теперь не печальтесь и не жалейте о том, что вы продали меня сюда, потому что Бог послал меня перед вами для сохранения вашей жизни;
\end{tcolorbox}
\begin{tcolorbox}
\textsubscript{6} ибо теперь два года голода на земле: еще пять лет, в которые ни орать, ни жать не будут;
\end{tcolorbox}
\begin{tcolorbox}
\textsubscript{7} Бог послал меня перед вами, чтобы оставить вас на земле и сохранить вашу жизнь великим избавлением.
\end{tcolorbox}
\begin{tcolorbox}
\textsubscript{8} Итак не вы послали меня сюда, но Бог, Который и поставил меня отцом фараону и господином во всем доме его и владыкою во всей земле Египетской.
\end{tcolorbox}
\begin{tcolorbox}
\textsubscript{9} Идите скорее к отцу моему и скажите ему: так говорит сын твой Иосиф: Бог поставил меня господином над всем Египтом; приди ко мне, не медли;
\end{tcolorbox}
\begin{tcolorbox}
\textsubscript{10} ты будешь жить в земле Гесем; и будешь близ меня, ты, и сыны твои, и сыны сынов твоих, и мелкий и крупный скот твой, и все твое;
\end{tcolorbox}
\begin{tcolorbox}
\textsubscript{11} и прокормлю тебя там, ибо голод будет еще пять лет, чтобы не обнищал ты и дом твой и все твое.
\end{tcolorbox}
\begin{tcolorbox}
\textsubscript{12} И вот, очи ваши и очи брата моего Вениамина видят, что это мои уста говорят с вами;
\end{tcolorbox}
\begin{tcolorbox}
\textsubscript{13} скажите же отцу моему о всей славе моей в Египте и о всем, что вы видели, и приведите скорее отца моего сюда.
\end{tcolorbox}
\begin{tcolorbox}
\textsubscript{14} И пал он на шею Вениамину, брату своему, и плакал; и Вениамин плакал на шее его.
\end{tcolorbox}
\begin{tcolorbox}
\textsubscript{15} И целовал всех братьев своих и плакал, обнимая их. Потом говорили с ним братья его.
\end{tcolorbox}
\begin{tcolorbox}
\textsubscript{16} Дошел в дом фараона слух, что пришли братья Иосифа; и приятно было фараону и рабам его.
\end{tcolorbox}
\begin{tcolorbox}
\textsubscript{17} И сказал фараон Иосифу: скажи братьям твоим: вот что сделайте: навьючьте скот ваш, и ступайте в землю Ханаанскую;
\end{tcolorbox}
\begin{tcolorbox}
\textsubscript{18} и возьмите отца вашего и семейства ваши и придите ко мне; я дам вам лучшее в земле Египетской, и вы будете есть тук земли.
\end{tcolorbox}
\begin{tcolorbox}
\textsubscript{19} Тебе же повелеваю сказать им: сделайте сие: возьмите себе из земли Египетской колесниц для детей ваших и для жен ваших, и привезите отца вашего и придите;
\end{tcolorbox}
\begin{tcolorbox}
\textsubscript{20} и не жалейте вещей ваших, ибо лучшее из всей земли Египетской [дам] вам.
\end{tcolorbox}
\begin{tcolorbox}
\textsubscript{21} Так и сделали сыны Израилевы. И дал им Иосиф колесницы по приказанию фараона, и дал им путевой запас,
\end{tcolorbox}
\begin{tcolorbox}
\textsubscript{22} каждому из них он дал перемену одежд, а Вениамину дал триста сребренников и пять перемен одежд;
\end{tcolorbox}
\begin{tcolorbox}
\textsubscript{23} также и отцу своему послал десять ослов, навьюченных лучшими [произведениями] Египетскими, и десять ослиц, навьюченных зерном, хлебом и припасами отцу своему на путь.
\end{tcolorbox}
\begin{tcolorbox}
\textsubscript{24} И отпустил братьев своих, и они пошли. И сказал им: не ссорьтесь на дороге.
\end{tcolorbox}
\begin{tcolorbox}
\textsubscript{25} И пошли они из Египта, и пришли в землю Ханаанскую к Иакову, отцу своему,
\end{tcolorbox}
\begin{tcolorbox}
\textsubscript{26} и известили его, сказав: Иосиф жив, и теперь владычествует над всею землею Египетскою. Но сердце его смутилось, ибо он не верил им.
\end{tcolorbox}
\begin{tcolorbox}
\textsubscript{27} Когда же они пересказали ему все слова Иосифа, которые он говорил им, и когда увидел колесницы, которые прислал Иосиф, чтобы везти его, тогда ожил дух Иакова, отца их,
\end{tcolorbox}
\begin{tcolorbox}
\textsubscript{28} и сказал Израиль: довольно, еще жив сын мой Иосиф; пойду и увижу его, пока не умру.
\end{tcolorbox}
\subsection{CHAPTER 46}
\begin{tcolorbox}
\textsubscript{1} И отправился Израиль со всем, что у него было, и пришел в Вирсавию, и принес жертвы Богу отца своего Исаака.
\end{tcolorbox}
\begin{tcolorbox}
\textsubscript{2} И сказал Бог Израилю в видении ночном: Иаков! Иаков! Он сказал: вот я.
\end{tcolorbox}
\begin{tcolorbox}
\textsubscript{3} [Бог] сказал: Я Бог, Бог отца твоего; не бойся идти в Египет, ибо там произведу от тебя народ великий;
\end{tcolorbox}
\begin{tcolorbox}
\textsubscript{4} Я пойду с тобою в Египет, Я и выведу тебя обратно. Иосиф своею рукою закроет глаза [твои].
\end{tcolorbox}
\begin{tcolorbox}
\textsubscript{5} Иаков отправился из Вирсавии; и повезли сыны Израилевы Иакова отца своего, и детей своих, и жен своих на колесницах, которые послал фараон, чтобы привезти его.
\end{tcolorbox}
\begin{tcolorbox}
\textsubscript{6} И взяли они скот свой и имущество свое, которое приобрели в земле Ханаанской, и пришли в Египет, --Иаков и весь род его с ним.
\end{tcolorbox}
\begin{tcolorbox}
\textsubscript{7} Сынов своих и внуков своих с собою, дочерей своих и внучек своих и весь род свой привел он с собою в Египет.
\end{tcolorbox}
\begin{tcolorbox}
\textsubscript{8} Вот имена сынов Израилевых, пришедших в Египет: Иаков и сыновья его. Первенец Иакова Рувим.
\end{tcolorbox}
\begin{tcolorbox}
\textsubscript{9} Сыны Рувима: Ханох и Фаллу, Хецрон и Харми.
\end{tcolorbox}
\begin{tcolorbox}
\textsubscript{10} Сыны Симеона: Иемуил и Иамин, и Огад, и Иахин, и Цохар, и Саул, сын Хананеянки.
\end{tcolorbox}
\begin{tcolorbox}
\textsubscript{11} Сыны Левия: Гирсон, Кааф и Мерари.
\end{tcolorbox}
\begin{tcolorbox}
\textsubscript{12} Сыны Иуды: Ир и Онан, и Шела, и Фарес, и Зара; но Ир и Онан умерли в земле Ханаанской. Сыны Фареса были: Есром и Хамул.
\end{tcolorbox}
\begin{tcolorbox}
\textsubscript{13} Сыны Иссахара: Фола и Фува, Иов и Шимрон.
\end{tcolorbox}
\begin{tcolorbox}
\textsubscript{14} Сыны Завулона: Серед и Елон, и Иахлеил.
\end{tcolorbox}
\begin{tcolorbox}
\textsubscript{15} Это сыны Лии, которых она родила Иакову в Месопотамии, и Дину, дочь его. Всех душ сынов его и дочерей его--тридцать три.
\end{tcolorbox}
\begin{tcolorbox}
\textsubscript{16} Сыны Гада: Цифион и Хагги, Шуни и Эцбон, Ери и Ароди и Арели.
\end{tcolorbox}
\begin{tcolorbox}
\textsubscript{17} Сыны Асира: Имна и Ишва, и Ишви, и Бриа, и Серах, сестра их. Сыны Брии: Хевер и Малхиил.
\end{tcolorbox}
\begin{tcolorbox}
\textsubscript{18} Это сыны Зелфы, которую Лаван дал Лии, дочери своей; она родила их Иакову шестнадцать душ.
\end{tcolorbox}
\begin{tcolorbox}
\textsubscript{19} Сыны Рахили, жены Иакова: Иосиф и Вениамин.
\end{tcolorbox}
\begin{tcolorbox}
\textsubscript{20} И родились у Иосифа в земле Египетской Манассия и Ефрем, которых родила ему Асенефа, дочь Потифера, жреца Илиопольского.
\end{tcolorbox}
\begin{tcolorbox}
\textsubscript{21} Сыны Вениамина: Бела и Бехер и Ашбел; Гера и Нааман, Эхи и Рош, Муппим и Хуппим и Ард.
\end{tcolorbox}
\begin{tcolorbox}
\textsubscript{22} Это сыны Рахили, которые родились у Иакова, всего четырнадцать душ.
\end{tcolorbox}
\begin{tcolorbox}
\textsubscript{23} Сын Дана: Хушим.
\end{tcolorbox}
\begin{tcolorbox}
\textsubscript{24} Сыны Неффалима: Иахцеил и Гуни, и Иецер, и Шиллем.
\end{tcolorbox}
\begin{tcolorbox}
\textsubscript{25} Это сыны Валлы, которую дал Лаван дочери своей Рахили; она родила их Иакову всего семь душ.
\end{tcolorbox}
\begin{tcolorbox}
\textsubscript{26} Всех душ, пришедших с Иаковом в Египет, которые произошли из чресл его, кроме жен сынов Иаковлевых, всего шестьдесят шесть душ.
\end{tcolorbox}
\begin{tcolorbox}
\textsubscript{27} Сынов Иосифа, которые родились у него в Египте, две души. Всех душ дома Иаковлева, перешедших в Египет, семьдесят.
\end{tcolorbox}
\begin{tcolorbox}
\textsubscript{28} Иуду послал он пред собою к Иосифу, чтобы он указал [путь] в Гесем. И пришли в землю Гесем.
\end{tcolorbox}
\begin{tcolorbox}
\textsubscript{29} Иосиф запряг колесницу свою и выехал навстречу Израилю, отцу своему, в Гесем, и, увидев его, пал на шею его, и долго плакал на шее его.
\end{tcolorbox}
\begin{tcolorbox}
\textsubscript{30} И сказал Израиль Иосифу: умру я теперь, увидев лице твое, ибо ты еще жив.
\end{tcolorbox}
\begin{tcolorbox}
\textsubscript{31} И сказал Иосиф братьям своим и дому отца своего: я пойду, извещу фараона и скажу ему: братья мои и дом отца моего, которые были в земле Ханаанской, пришли ко мне;
\end{tcolorbox}
\begin{tcolorbox}
\textsubscript{32} эти люди пастухи овец, ибо скотоводы они; и мелкий и крупный скот свой, и все, что у них, привели они.
\end{tcolorbox}
\begin{tcolorbox}
\textsubscript{33} Если фараон призовет вас и скажет: какое занятие ваше?
\end{tcolorbox}
\begin{tcolorbox}
\textsubscript{34} то вы скажите: [мы], рабы твои, скотоводами были от юности нашей доныне, и мы и отцы наши, чтобы вас поселили в земле Гесем. Ибо мерзость для Египтян всякий пастух овец.
\end{tcolorbox}
\subsection{CHAPTER 47}
\begin{tcolorbox}
\textsubscript{1} И пришел Иосиф и известил фараона и сказал: отец мой и братья мои, с мелким и крупным скотом своим и со всем, что у них, пришли из земли Ханаанской; и вот, они в земле Гесем.
\end{tcolorbox}
\begin{tcolorbox}
\textsubscript{2} И из братьев своих он взял пять человек и представил их фараону.
\end{tcolorbox}
\begin{tcolorbox}
\textsubscript{3} И сказал фараон братьям его: какое ваше занятие? Они сказали фараону: пастухи овец рабы твои, и мы и отцы наши.
\end{tcolorbox}
\begin{tcolorbox}
\textsubscript{4} И сказали они фараону: мы пришли пожить в этой земле, потому что нет пажити для скота рабов твоих, ибо в земле Ханаанской сильный голод; итак позволь поселиться рабам твоим в земле Гесем.
\end{tcolorbox}
\begin{tcolorbox}
\textsubscript{5} И сказал фараон Иосифу: отец твой и братья твои пришли к тебе;
\end{tcolorbox}
\begin{tcolorbox}
\textsubscript{6} земля Египетская пред тобою; на лучшем месте земли посели отца твоего и братьев твоих; пусть живут они в земле Гесем; и если знаешь, что между ними есть способные люди, поставь их смотрителями над моим скотом.
\end{tcolorbox}
\begin{tcolorbox}
\textsubscript{7} И привел Иосиф Иакова, отца своего, и представил его фараону; и благословил Иаков фараона.
\end{tcolorbox}
\begin{tcolorbox}
\textsubscript{8} Фараон сказал Иакову: сколько лет жизни твоей?
\end{tcolorbox}
\begin{tcolorbox}
\textsubscript{9} Иаков сказал фараону: дней странствования моего сто тридцать лет; малы и несчастны дни жизни моей и не достигли до лет жизни отцов моих во днях странствования их.
\end{tcolorbox}
\begin{tcolorbox}
\textsubscript{10} И благословил фараона Иаков и вышел от фараона.
\end{tcolorbox}
\begin{tcolorbox}
\textsubscript{11} И поселил Иосиф отца своего и братьев своих, и дал им владение в земле Египетской, в лучшей части земли, в земле Раамсес, как повелел фараон.
\end{tcolorbox}
\begin{tcolorbox}
\textsubscript{12} И снабжал Иосиф отца своего и братьев своих и весь дом отца своего хлебом, по потребностям каждого семейства.
\end{tcolorbox}
\begin{tcolorbox}
\textsubscript{13} И не было хлеба по всей земле, потому что голод весьма усилился, и изнурены были от голода земля Египетская и земля Ханаанская.
\end{tcolorbox}
\begin{tcolorbox}
\textsubscript{14} Иосиф собрал все серебро, какое было в земле Египетской и в земле Ханаанской, за хлеб, который покупали, и внес Иосиф серебро в дом фараонов.
\end{tcolorbox}
\begin{tcolorbox}
\textsubscript{15} И серебро истощилось в земле Египетской и в земле Ханаанской. Все Египтяне пришли к Иосифу и говорили: дай нам хлеба; зачем нам умирать пред тобою, потому что серебро вышло у нас?
\end{tcolorbox}
\begin{tcolorbox}
\textsubscript{16} Иосиф сказал: пригоняйте скот ваш, и я буду давать вам за скот ваш, если серебро вышло у вас.
\end{tcolorbox}
\begin{tcolorbox}
\textsubscript{17} И пригоняли они к Иосифу скот свой; и давал им Иосиф хлеб за лошадей, и за стада мелкого скота, и за стада крупного скота, и за ослов; и снабжал их хлебом в тот год за весь скот их.
\end{tcolorbox}
\begin{tcolorbox}
\textsubscript{18} И прошел этот год; и пришли к нему на другой год и сказали ему: не скроем от господина нашего, что серебро истощилось и стада скота нашего у господина нашего; ничего не осталось у нас пред господином нашим, кроме тел наших и земель наших;
\end{tcolorbox}
\begin{tcolorbox}
\textsubscript{19} для чего нам погибать в глазах твоих, и нам и землям нашим? купи нас и земли наши за хлеб, и мы с землями нашими будем рабами фараону, а ты дай нам семян, чтобы нам быть живыми и не умереть, и чтобы не опустела земля.
\end{tcolorbox}
\begin{tcolorbox}
\textsubscript{20} И купил Иосиф всю землю Египетскую для фараона, потому что продали Египтяне каждый свое поле, ибо голод одолевал их. И досталась земля фараону.
\end{tcolorbox}
\begin{tcolorbox}
\textsubscript{21} И народ сделал он рабами от одного конца Египта до другого.
\end{tcolorbox}
\begin{tcolorbox}
\textsubscript{22} Только земли жрецов не купил, ибо жрецам от фараона положен был участок, и они питались своим участком, который дал им фараон; посему и не продали земли своей.
\end{tcolorbox}
\begin{tcolorbox}
\textsubscript{23} И сказал Иосиф народу: вот, я купил теперь для фараона вас и землю вашу; вот вам семена, и засевайте землю;
\end{tcolorbox}
\begin{tcolorbox}
\textsubscript{24} когда будет жатва, давайте пятую часть фараону, а четыре части останутся вам на засеяние полей, на пропитание вам и тем, кто в домах ваших, и на пропитание детям вашим.
\end{tcolorbox}
\begin{tcolorbox}
\textsubscript{25} Они сказали: ты спас нам жизнь; да обретем милость в очах господина нашего и да будем рабами фараону.
\end{tcolorbox}
\begin{tcolorbox}
\textsubscript{26} И поставил Иосиф в закон земле Египетской, даже до сего дня: пятую часть давать фараону, исключая только землю жрецов, которая не принадлежала фараону.
\end{tcolorbox}
\begin{tcolorbox}
\textsubscript{27} И жил Израиль в земле Египетской, в земле Гесем, и владели они ею, и плодились, и весьма умножились.
\end{tcolorbox}
\begin{tcolorbox}
\textsubscript{28} И жил Иаков в земле Египетской семнадцать лет; и было дней Иакова, годов жизни его, сто сорок семь лет.
\end{tcolorbox}
\begin{tcolorbox}
\textsubscript{29} И пришло время Израилю умереть, и призвал он сына своего Иосифа и сказал ему: если я нашел благоволение в очах твоих, положи руку твою под стегно мое и [клянись], что ты окажешь мне милость и правду, не похоронишь меня в Египте,
\end{tcolorbox}
\begin{tcolorbox}
\textsubscript{30} дабы мне лечь с отцами моими; вынесешь меня из Египта и похоронишь меня в их гробнице. [Иосиф] сказал: сделаю по слову твоему.
\end{tcolorbox}
\begin{tcolorbox}
\textsubscript{31} И сказал: клянись мне. И клялся ему. И поклонился Израиль на возглавие постели.
\end{tcolorbox}
\subsection{CHAPTER 48}
\begin{tcolorbox}
\textsubscript{1} После того Иосифу сказали: вот, отец твой болен. И он взял с собою двух сынов своих, Манассию и Ефрема.
\end{tcolorbox}
\begin{tcolorbox}
\textsubscript{2} Иакова известили и сказали: вот, сын твой Иосиф идет к тебе. Израиль собрал силы свои и сел на постели.
\end{tcolorbox}
\begin{tcolorbox}
\textsubscript{3} И сказал Иаков Иосифу: Бог Всемогущий явился мне в Лузе, в земле Ханаанской, и благословил меня,
\end{tcolorbox}
\begin{tcolorbox}
\textsubscript{4} и сказал мне: вот, Я распложу тебя, и размножу тебя, и произведу от тебя множество народов, и дам землю сию потомству твоему после тебя, в вечное владение.
\end{tcolorbox}
\begin{tcolorbox}
\textsubscript{5} И ныне два сына твои, родившиеся тебе в земле Египетской, до моего прибытия к тебе в Египет, мои они; Ефрем и Манассия, как Рувим и Симеон, будут мои;
\end{tcolorbox}
\begin{tcolorbox}
\textsubscript{6} дети же твои, которые родятся от тебя после них, будут твои; они под именем братьев своих будут именоваться в их уделе.
\end{tcolorbox}
\begin{tcolorbox}
\textsubscript{7} Когда я шел из Месопотамии, умерла у меня Рахиль в земле Ханаанской, по дороге, не доходя несколько до Ефрафы, и я похоронил ее там на дороге к Ефрафе, что [ныне] Вифлеем.
\end{tcolorbox}
\begin{tcolorbox}
\textsubscript{8} И увидел Израиль сыновей Иосифа и сказал: кто это?
\end{tcolorbox}
\begin{tcolorbox}
\textsubscript{9} И сказал Иосиф отцу своему: это сыновья мои, которых Бог дал мне здесь. Иаков сказал: подведи их ко мне, и я благословлю их.
\end{tcolorbox}
\begin{tcolorbox}
\textsubscript{10} Глаза же Израилевы притупились от старости; не мог он видеть [ясно. Иосиф] подвел их к нему, и он поцеловал их и обнял их.
\end{tcolorbox}
\begin{tcolorbox}
\textsubscript{11} И сказал Израиль Иосифу: не надеялся я видеть твое лице; но вот, Бог показал мне и детей твоих.
\end{tcolorbox}
\begin{tcolorbox}
\textsubscript{12} И отвел их Иосиф от колен его и поклонился ему лицем своим до земли.
\end{tcolorbox}
\begin{tcolorbox}
\textsubscript{13} И взял Иосиф обоих, Ефрема в правую свою руку против левой Израиля, а Манассию в левую против правой Израиля, и подвел к нему.
\end{tcolorbox}
\begin{tcolorbox}
\textsubscript{14} Но Израиль простер правую руку свою и положил на голову Ефрему, хотя сей был меньший, а левую на голову Манассии. С намерением положил он так руки свои, хотя Манассия был первенец.
\end{tcolorbox}
\begin{tcolorbox}
\textsubscript{15} И благословил Иосифа и сказал: Бог, пред Которым ходили отцы мои Авраам и Исаак, Бог, пасущий меня с тех пор, как я существую, до сего дня,
\end{tcolorbox}
\begin{tcolorbox}
\textsubscript{16} Ангел, избавляющий меня от всякого зла, да благословит отроков сих; да будет на них наречено имя мое и имя отцов моих Авраама и Исаака, и да возрастут они во множество посреди земли.
\end{tcolorbox}
\begin{tcolorbox}
\textsubscript{17} И увидел Иосиф, что отец его положил правую руку свою на голову Ефрема; и прискорбно было ему это. И взял он руку отца своего, чтобы переложить ее с головы Ефрема на голову Манассии,
\end{tcolorbox}
\begin{tcolorbox}
\textsubscript{18} и сказал Иосиф отцу своему: не так, отец мой, ибо это--первенец; положи на его голову правую руку твою.
\end{tcolorbox}
\begin{tcolorbox}
\textsubscript{19} Но отец его не согласился и сказал: знаю, сын мой, знаю; и от него произойдет народ, и он будет велик; но меньший его брат будет больше его, и от семени его произойдет многочисленный народ.
\end{tcolorbox}
\begin{tcolorbox}
\textsubscript{20} И благословил их в тот день, говоря: тобою будет благословлять Израиль, говоря: Бог да сотворит тебе, как Ефрему и Манассии. И поставил Ефрема выше Манассии.
\end{tcolorbox}
\begin{tcolorbox}
\textsubscript{21} И сказал Израиль Иосифу: вот, я умираю; и Бог будет с вами и возвратит вас в землю отцов ваших;
\end{tcolorbox}
\begin{tcolorbox}
\textsubscript{22} я даю тебе, преимущественно пред братьями твоими, один участок, который я взял из рук Аморреев мечом моим и луком моим.
\end{tcolorbox}
\subsection{CHAPTER 49}
\begin{tcolorbox}
\textsubscript{1} И призвал Иаков сыновей своих и сказал: соберитесь, и я возвещу вам, что будет с вами в грядущие дни;
\end{tcolorbox}
\begin{tcolorbox}
\textsubscript{2} сойдитесь и послушайте, сыны Иакова, послушайте Израиля, отца вашего.
\end{tcolorbox}
\begin{tcolorbox}
\textsubscript{3} Рувим, первенец мой! ты--крепость моя и начаток силы моей, верх достоинства и верх могущества;
\end{tcolorbox}
\begin{tcolorbox}
\textsubscript{4} но ты бушевал, как вода, --не будешь преимуществовать, ибо ты взошел на ложе отца твоего, ты осквернил постель мою, взошел.
\end{tcolorbox}
\begin{tcolorbox}
\textsubscript{5} Симеон и Левий братья, орудия жестокости мечи их;
\end{tcolorbox}
\begin{tcolorbox}
\textsubscript{6} в совет их да не внидет душа моя, и к собранию их да не приобщится слава моя, ибо они во гневе своем убили мужа и по прихоти своей перерезали жилы тельца;
\end{tcolorbox}
\begin{tcolorbox}
\textsubscript{7} проклят гнев их, ибо жесток, и ярость их, ибо свирепа; разделю их в Иакове и рассею их в Израиле.
\end{tcolorbox}
\begin{tcolorbox}
\textsubscript{8} Иуда! тебя восхвалят братья твои. Рука твоя на хребте врагов твоих; поклонятся тебе сыны отца твоего.
\end{tcolorbox}
\begin{tcolorbox}
\textsubscript{9} Молодой лев Иуда, с добычи, сын мой, поднимается. Преклонился он, лег, как лев и как львица: кто поднимет его?
\end{tcolorbox}
\begin{tcolorbox}
\textsubscript{10} Не отойдет скипетр от Иуды и законодатель от чресл его, доколе не приидет Примиритель, и Ему покорность народов.
\end{tcolorbox}
\begin{tcolorbox}
\textsubscript{11} Он привязывает к виноградной лозе осленка своего и к лозе лучшего винограда сына ослицы своей; моет в вине одежду свою и в крови гроздов одеяние свое;
\end{tcolorbox}
\begin{tcolorbox}
\textsubscript{12} блестящи очи [его] от вина, и белы зубы от молока.
\end{tcolorbox}
\begin{tcolorbox}
\textsubscript{13} Завулон при береге морском будет жить и у пристани корабельной, и предел его до Сидона.
\end{tcolorbox}
\begin{tcolorbox}
\textsubscript{14} Иссахар осел крепкий, лежащий между протоками вод;
\end{tcolorbox}
\begin{tcolorbox}
\textsubscript{15} и увидел он, что покой хорош, и что земля приятна: и преклонил плечи свои для ношения бремени и стал работать в уплату дани.
\end{tcolorbox}
\begin{tcolorbox}
\textsubscript{16} Дан будет судить народ свой, как одно из колен Израиля;
\end{tcolorbox}
\begin{tcolorbox}
\textsubscript{17} Дан будет змеем на дороге, аспидом на пути, уязвляющим ногу коня, так что всадник его упадет назад.
\end{tcolorbox}
\begin{tcolorbox}
\textsubscript{18} На помощь твою надеюсь, Господи!
\end{tcolorbox}
\begin{tcolorbox}
\textsubscript{19} Гад, --толпа будет теснить его, но он оттеснит ее по пятам.
\end{tcolorbox}
\begin{tcolorbox}
\textsubscript{20} Для Асира--слишком тучен хлеб его, и он будет доставлять царские яства.
\end{tcolorbox}
\begin{tcolorbox}
\textsubscript{21} Неффалим--теревинф рослый, распускающий прекрасные ветви.
\end{tcolorbox}
\begin{tcolorbox}
\textsubscript{22} Иосиф--отрасль плодоносного [дерева], отрасль плодоносного [дерева] над источником; ветви его простираются над стеною;
\end{tcolorbox}
\begin{tcolorbox}
\textsubscript{23} огорчали его, и стреляли и враждовали на него стрельцы,
\end{tcolorbox}
\begin{tcolorbox}
\textsubscript{24} но тверд остался лук его, и крепки мышцы рук его, от рук мощного [Бога] Иаковлева. Оттуда Пастырь и твердыня Израилева,
\end{tcolorbox}
\begin{tcolorbox}
\textsubscript{25} от Бога отца твоего, [Который] и да поможет тебе, и от Всемогущего, Который и да благословит тебя благословениями небесными свыше, благословениями бездны, лежащей долу, благословениями сосцов и утробы,
\end{tcolorbox}
\begin{tcolorbox}
\textsubscript{26} благословениями отца твоего, которые превышают благословения гор древних и приятности холмов вечных; да будут они на голове Иосифа и на темени избранного между братьями своими.
\end{tcolorbox}
\begin{tcolorbox}
\textsubscript{27} Вениамин, хищный волк, утром будет есть ловитву и вечером будет делить добычу.
\end{tcolorbox}
\begin{tcolorbox}
\textsubscript{28} Вот все двенадцать колен Израилевых; и вот что сказал им отец их; и благословил их, и дал им благословение, каждому свое.
\end{tcolorbox}
\begin{tcolorbox}
\textsubscript{29} И заповедал он им и сказал им: я прилагаюсь к народу моему; похороните меня с отцами моими в пещере, которая на поле Ефрона Хеттеянина,
\end{tcolorbox}
\begin{tcolorbox}
\textsubscript{30} в пещере, которая на поле Махпела, что пред Мамре, в земле Ханаанской, которую купил Авраам с полем у Ефрона Хеттеянина в собственность для погребения;
\end{tcolorbox}
\begin{tcolorbox}
\textsubscript{31} там похоронили Авраама и Сарру, жену его; там похоронили Исаака и Ревекку, жену его; и там похоронил я Лию;
\end{tcolorbox}
\begin{tcolorbox}
\textsubscript{32} это поле и пещера, которая на нем, куплена у сынов Хеттеевых.
\end{tcolorbox}
\begin{tcolorbox}
\textsubscript{33} И окончил Иаков завещание сыновьям своим, и положил ноги свои на постель, и скончался, и приложился к народу своему.
\end{tcolorbox}
\subsection{CHAPTER 50}
\begin{tcolorbox}
\textsubscript{1} Иосиф пал на лице отца своего, и плакал над ним, и целовал его.
\end{tcolorbox}
\begin{tcolorbox}
\textsubscript{2} И повелел Иосиф слугам своим--врачам, бальзамировать отца его; и врачи набальзамировали Израиля.
\end{tcolorbox}
\begin{tcolorbox}
\textsubscript{3} И исполнилось ему сорок дней, ибо столько дней употребляется на бальзамирование, и оплакивали его Египтяне семьдесят дней.
\end{tcolorbox}
\begin{tcolorbox}
\textsubscript{4} Когда же прошли дни плача по нем, Иосиф сказал придворным фараона, говоря: если я обрел благоволение в очах ваших, то скажите фараону так:
\end{tcolorbox}
\begin{tcolorbox}
\textsubscript{5} отец мой заклял меня, сказав: вот, я умираю; во гробе моем, который я выкопал себе в земле Ханаанской, там похорони меня. И теперь хотел бы я пойти и похоронить отца моего и возвратиться.
\end{tcolorbox}
\begin{tcolorbox}
\textsubscript{6} И сказал фараон: пойди и похорони отца твоего, как он заклял тебя.
\end{tcolorbox}
\begin{tcolorbox}
\textsubscript{7} И пошел Иосиф хоронить отца своего. И пошли с ним все слуги фараона, старейшины дома его и все старейшины земли Египетской,
\end{tcolorbox}
\begin{tcolorbox}
\textsubscript{8} и весь дом Иосифа, и братья его, и дом отца его. Только детей своих и мелкий и крупный скот свой оставили в земле Гесем.
\end{tcolorbox}
\begin{tcolorbox}
\textsubscript{9} С ним отправились также колесницы и всадники, так что сонм был весьма велик.
\end{tcolorbox}
\begin{tcolorbox}
\textsubscript{10} И дошли они до Горен-гаатада при Иордане и плакали там плачем великим и весьма сильным; и сделал [Иосиф] плач по отце своем семь дней.
\end{tcolorbox}
\begin{tcolorbox}
\textsubscript{11} И видели жители земли той, Хананеи, плач в Горен-гаатаде, и сказали: велик плач этот у Египтян! Посему наречено имя [месту] тому: плач Египтян, что при Иордане.
\end{tcolorbox}
\begin{tcolorbox}
\textsubscript{12} И сделали сыновья [Иакова] с ним, как он заповедал им;
\end{tcolorbox}
\begin{tcolorbox}
\textsubscript{13} и отнесли его сыновья его в землю Ханаанскую и похоронили его в пещере на поле Махпела, которую купил Авраам с полем в собственность для погребения у Ефрона Хеттеянина, пред Мамре.
\end{tcolorbox}
\begin{tcolorbox}
\textsubscript{14} И возвратился Иосиф в Египет, сам и братья его и все ходившие с ним хоронить отца его, после погребения им отца своего.
\end{tcolorbox}
\begin{tcolorbox}
\textsubscript{15} И увидели братья Иосифовы, что умер отец их, и сказали: что, если Иосиф возненавидит нас и захочет отмстить нам за всё зло, которое мы ему сделали?
\end{tcolorbox}
\begin{tcolorbox}
\textsubscript{16} И послали они сказать Иосифу: отец твой пред смертью своею завещал, говоря:
\end{tcolorbox}
\begin{tcolorbox}
\textsubscript{17} так скажите Иосифу: прости братьям твоим вину и грех их, так как они сделали тебе зло. И ныне прости вины рабов Бога отца твоего. Иосиф плакал, когда ему говорили это.
\end{tcolorbox}
\begin{tcolorbox}
\textsubscript{18} Пришли и сами братья его, и пали пред лицем его, и сказали: вот, мы рабы тебе.
\end{tcolorbox}
\begin{tcolorbox}
\textsubscript{19} И сказал Иосиф: не бойтесь, ибо я боюсь Бога;
\end{tcolorbox}
\begin{tcolorbox}
\textsubscript{20} вот, вы умышляли против меня зло; но Бог обратил это в добро, чтобы сделать то, что теперь есть: сохранить жизнь великому числу людей;
\end{tcolorbox}
\begin{tcolorbox}
\textsubscript{21} итак не бойтесь: я буду питать вас и детей ваших. И успокоил их и говорил по сердцу их.
\end{tcolorbox}
\begin{tcolorbox}
\textsubscript{22} И жил Иосиф в Египте сам и дом отца его; жил же Иосиф всего сто десять лет.
\end{tcolorbox}
\begin{tcolorbox}
\textsubscript{23} И видел Иосиф детей у Ефрема до третьего рода, также и сыновья Махира, сына Манассиина, родились на колени Иосифа.
\end{tcolorbox}
\begin{tcolorbox}
\textsubscript{24} И сказал Иосиф братьям своим: я умираю, но Бог посетит вас и выведет вас из земли сей в землю, о которой клялся Аврааму, Исааку и Иакову.
\end{tcolorbox}
\begin{tcolorbox}
\textsubscript{25} И заклял Иосиф сынов Израилевых, говоря: Бог посетит вас, и вынесите кости мои отсюда.
\end{tcolorbox}
\begin{tcolorbox}
\textsubscript{26} И умер Иосиф ста десяти лет. И набальзамировали его и положили в ковчег в Египте.
\end{tcolorbox}
