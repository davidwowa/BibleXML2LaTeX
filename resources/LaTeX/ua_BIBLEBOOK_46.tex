\section{BOOK 45}
\subsection{CHAPTER 1}
\begin{tcolorbox}
\textsubscript{1} Павло, волею Божою покликаний за апостола Ісуса Христа, і брат Состен,
\end{tcolorbox}
\begin{tcolorbox}
\textsubscript{2} Божій Церкві, що в Коринті, посвяченим у Христі Ісусі, покликаним святим, зо всіма, що на всякому місті прикликають Ім'я Господа нашого Ісуса Христа, їхнього і нашого,
\end{tcolorbox}
\begin{tcolorbox}
\textsubscript{3} благодать вам і мир від Бога Отця нашого й Господа Ісуса Христа!
\end{tcolorbox}
\begin{tcolorbox}
\textsubscript{4} Я завжди дякую моєму Богові за вас, через Божу благодать, що була вам дана в Христі Ісусі,
\end{tcolorbox}
\begin{tcolorbox}
\textsubscript{5} бо ви всім збагатилися в Ньому, словом усяким і всяким знанням,
\end{tcolorbox}
\begin{tcolorbox}
\textsubscript{6} бо свідоцтво Христове між вами утвердилось,
\end{tcolorbox}
\begin{tcolorbox}
\textsubscript{7} так що не маєте недостачі в жаднім дарі благодаті ви, що очікуєте з'явлення Господа нашого Ісуса Христа.
\end{tcolorbox}
\begin{tcolorbox}
\textsubscript{8} Він вас утвердить до кінця неповинними бути дня Господа нашого Ісуса Христа!
\end{tcolorbox}
\begin{tcolorbox}
\textsubscript{9} Вірний Бог, що ви через Нього покликані до спільноти Сина Його Ісуса Христа, Господа нашого.
\end{tcolorbox}
\begin{tcolorbox}
\textsubscript{10} Тож благаю вас, браття, Ім'ям Господа нашого Ісуса Христа, щоб ви всі говорили те саме, і щоб не було поміж вами поділення, але щоб були ви поєднані в однім розумінні та в думці одній!
\end{tcolorbox}
\begin{tcolorbox}
\textsubscript{11} Бо стало відомо мені про вас, мої браття, від Хлоїних, що між вами суперечки.
\end{tcolorbox}
\begin{tcolorbox}
\textsubscript{12} А кажу я про те, що з вас кожен говорить: я ж Павлів, а я Аполлосів, а я Кифин, а я Христів.
\end{tcolorbox}
\begin{tcolorbox}
\textsubscript{13} Чи ж Христос поділився? Чи ж Павло був розп'ятий за вас? Чи в Павлове ім'я ви христились?
\end{tcolorbox}
\begin{tcolorbox}
\textsubscript{14} Дякую Богові, що я ані одного з вас не христив, окрім Кріспа та Гая,
\end{tcolorbox}
\begin{tcolorbox}
\textsubscript{15} щоб ніхто не сказав, ніби я охристив був у ймення своє.
\end{tcolorbox}
\begin{tcolorbox}
\textsubscript{16} Охристив же був я й дім Степанів; більш не знаю, чи христив кого іншого я.
\end{tcolorbox}
\begin{tcolorbox}
\textsubscript{17} Бо Христос не послав мене, щоб христити, а звіщати Євангелію, і то не в мудрості слова, щоб безсилим не став хрест Христа.
\end{tcolorbox}
\begin{tcolorbox}
\textsubscript{18} Бож слово про хреста тим, що гинуть, то глупота, а для нас, що спасаємось, Сила Божа!
\end{tcolorbox}
\begin{tcolorbox}
\textsubscript{19} Бо написано: Я погублю мудрість премудрих, а розум розумних відкину!
\end{tcolorbox}
\begin{tcolorbox}
\textsubscript{20} Де мудрий? Де книжник? Де дослідувач віку цього? Хіба Бог мудрість світу цього не змінив на глупоту?
\end{tcolorbox}
\begin{tcolorbox}
\textsubscript{21} Через те ж, що світ мудрістю не зрозумів Бога в мудрості Божій, то Богові вгодно було спасти віруючих через дурість проповіді.
\end{tcolorbox}
\begin{tcolorbox}
\textsubscript{22} Бо й юдеї жадають ознак, і греки пошукують мудрости,
\end{tcolorbox}
\begin{tcolorbox}
\textsubscript{23} а ми проповідуємо Христа розп'ятого, для юдеїв згіршення, а для греків безумство,
\end{tcolorbox}
\begin{tcolorbox}
\textsubscript{24} а для самих покликаних юдеїв та греків Христа, Божу силу та Божую мудрість!
\end{tcolorbox}
\begin{tcolorbox}
\textsubscript{25} Бо Боже й немудре розумніше воно від людей, а Боже немічне сильніше воно від людей!
\end{tcolorbox}
\begin{tcolorbox}
\textsubscript{26} Дивіться бо, браття, на ваших покликаних, що небагато-хто мудрі за тілом, небагато-хто сильні, небагато-хто шляхетні.
\end{tcolorbox}
\begin{tcolorbox}
\textsubscript{27} Але Бог вибрав немудре світу, щоб засоромити мудрих, і немічне світу Бог вибрав, щоб засоромити сильне,
\end{tcolorbox}
\begin{tcolorbox}
\textsubscript{28} і простих світу, і погорджених, і незначних вибрав Бог, щоб значне знівечити,
\end{tcolorbox}
\begin{tcolorbox}
\textsubscript{29} так щоб не хвалилося перед Богом жадне тіло.
\end{tcolorbox}
\begin{tcolorbox}
\textsubscript{30} А з Нього ви в Христі Ісусі, що став нам мудрістю від Бога, праведністю ж, і освяченням, і відкупленням,
\end{tcolorbox}
\begin{tcolorbox}
\textsubscript{31} щоб було, як написано: Хто хвалиться, нехай хвалиться Господом!
\end{tcolorbox}
\subsection{CHAPTER 2}
\begin{tcolorbox}
\textsubscript{1} А я, як прийшов до вас, браття, не прийшов вам звіщати про Боже свідоцтво з добірною мовою або мудрістю,
\end{tcolorbox}
\begin{tcolorbox}
\textsubscript{2} бо я надумавсь нічого між вами не знати, крім Ісуса Христа, і Того розп'ятого...
\end{tcolorbox}
\begin{tcolorbox}
\textsubscript{3} І я в вас був у немочі, і в страху, і в великім тремтінні.
\end{tcolorbox}
\begin{tcolorbox}
\textsubscript{4} І слово моє й моя проповідь не в словах переконливих людської мудрости, але в доказі духа та сили,
\end{tcolorbox}
\begin{tcolorbox}
\textsubscript{5} щоб була віра ваша не в мудрості людській, але в силі Божій!
\end{tcolorbox}
\begin{tcolorbox}
\textsubscript{6} А ми говоримо про мудрість між досконалими, але мудрість не віку цього, ані володарів цього віку, що гинуть,
\end{tcolorbox}
\begin{tcolorbox}
\textsubscript{7} але ми говоримо Божу мудрість у таємниці, приховану, яку Бог перед віками призначив нам на славу,
\end{tcolorbox}
\begin{tcolorbox}
\textsubscript{8} яку ніхто з володарів цього віку не пізнав; коли б бо пізнали були, то не розп'яли б вони Господа слави!
\end{tcolorbox}
\begin{tcolorbox}
\textsubscript{9} Але, як написано: Чого око не бачило й вухо не чуло, і що на серце людині не впало, те Бог приготував був тим, хто любить Його!
\end{tcolorbox}
\begin{tcolorbox}
\textsubscript{10} А нам Бог відкрив це Своїм Духом, усе бо досліджує Дух, навіть Божі глибини.
\end{tcolorbox}
\begin{tcolorbox}
\textsubscript{11} Хто бо з людей знає речі людські, окрім людського духа, що в нім проживає? Так само не знає ніхто й речей Божих, окрім Духа Божого.
\end{tcolorbox}
\begin{tcolorbox}
\textsubscript{12} А ми прийняли духа не світу, але Духа, що з Бога, щоб знати про речі, від Бога даровані нам,
\end{tcolorbox}
\begin{tcolorbox}
\textsubscript{13} що й говоримо не вивченими словами людської мудрости, але вивченими від Духа Святого, порівнюючи духовне до духовного.
\end{tcolorbox}
\begin{tcolorbox}
\textsubscript{14} А людина тілесна не приймає речей, що від Божого Духа, бо їй це глупота, і вона зрозуміти їх не може, бо вони розуміються тільки духовно.
\end{tcolorbox}
\begin{tcolorbox}
\textsubscript{15} Духовна ж людина судить усе, а її судити не може ніхто.
\end{tcolorbox}
\begin{tcolorbox}
\textsubscript{16} Бо хто розум Господній пізнав, який би його міг навчати? А ми маємо розум Христів!
\end{tcolorbox}
\subsection{CHAPTER 3}
\begin{tcolorbox}
\textsubscript{1} І я, браття, не міг говорити до вас, як до духовних, але як до тілесних, як до немовлят у Христі.
\end{tcolorbox}
\begin{tcolorbox}
\textsubscript{2} Я вас годував молоком, а не твердою їжею, бо ви не могли її їсти, та й тепер ще не можете,
\end{tcolorbox}
\begin{tcolorbox}
\textsubscript{3} бо ви ще тілесні. Бо коли заздрість та суперечки між вами, то чи ж ви не тілесні, і хіба не полюдському робите?
\end{tcolorbox}
\begin{tcolorbox}
\textsubscript{4} Бо коли хто каже: Я ж Павлів, а інший: Я Аполлосів, то чи ж ви не тілесні?
\end{tcolorbox}
\begin{tcolorbox}
\textsubscript{5} Бо хто ж Аполлос? Або хто то Павло? Вони тільки служителі, що ви ввірували через них, і то скільки кому дав Господь.
\end{tcolorbox}
\begin{tcolorbox}
\textsubscript{6} Я посадив, Аполлос поливав, Бог же зростив,
\end{tcolorbox}
\begin{tcolorbox}
\textsubscript{7} тому ані той, хто садить, ані хто поливає, є щось, але Бог, що родить!
\end{tcolorbox}
\begin{tcolorbox}
\textsubscript{8} І хто садить, і хто поливає одне, і кожен одержить свою нагороду за працею своєю!
\end{tcolorbox}
\begin{tcolorbox}
\textsubscript{9} Бо ми співробітники Божі, а ви Боже поле, Божа будівля.
\end{tcolorbox}
\begin{tcolorbox}
\textsubscript{10} Я за благодаттю Божою, що дана мені, як мудрий будівничий, основу поклав, а інший будує на ній; але нехай кожен пильнує, як він будує на ній!
\end{tcolorbox}
\begin{tcolorbox}
\textsubscript{11} Ніхто бо не може покласти іншої основи, окрім покладеної, а вона Ісус Христос.
\end{tcolorbox}
\begin{tcolorbox}
\textsubscript{12} А коли хто на цій основі будує з золота, срібла, дорогоцінного каміння, із дерева, сіна, соломи,
\end{tcolorbox}
\begin{tcolorbox}
\textsubscript{13} то буде виявлене діло кожного, бо виявить день, тому що він огнем об'являється, і огонь діло кожного випробує, яке воно є.
\end{tcolorbox}
\begin{tcolorbox}
\textsubscript{14} І коли чиє діло, яке збудував хто, устоїть, то той нагороду одержить;
\end{tcolorbox}
\begin{tcolorbox}
\textsubscript{15} коли ж діло згорить, той матиме шкоду, та сам він спасеться, але так, як через огонь.
\end{tcolorbox}
\begin{tcolorbox}
\textsubscript{16} Чи не знаєте ви, що ви Божий храм, і Дух Божий у вас пробуває?
\end{tcolorbox}
\begin{tcolorbox}
\textsubscript{17} Як хто нівечить Божого храма, того знівечить Бог, бо храм Божий святий, а храм той то ви!
\end{tcolorbox}
\begin{tcolorbox}
\textsubscript{18} Хай не зводить ніхто сам себе. Як кому з вас здається, що він мудрий в цім віці, нехай стане нерозумним, щоб бути премудрим.
\end{tcolorbox}
\begin{tcolorbox}
\textsubscript{19} Цьогосвітня бо мудрість у Бога глупота, бо написано: Він ловить премудрих у хитрощах їхніх!
\end{tcolorbox}
\begin{tcolorbox}
\textsubscript{20} І знову: Знає Господь думки мудрих, що марнотні вони!
\end{tcolorbox}
\begin{tcolorbox}
\textsubscript{21} Тож нехай ніхто не хвалиться людьми, бо все ваше:
\end{tcolorbox}
\begin{tcolorbox}
\textsubscript{22} чи Павло, чи Аполлос, чи Кифа, чи світ, чи життя, чи смерть, чи теперішнє, чи майбутнє усе ваше,
\end{tcolorbox}
\begin{tcolorbox}
\textsubscript{23} ви ж Христові, а Христос Божий!
\end{tcolorbox}
\subsection{CHAPTER 4}
\begin{tcolorbox}
\textsubscript{1} Нехай кожен нас так уважає, якби служителів Христових і доморядників Божих таємниць;
\end{tcolorbox}
\begin{tcolorbox}
\textsubscript{2} а що ще шукається в доморядниках, щоб кожен був знайдений вірним.
\end{tcolorbox}
\begin{tcolorbox}
\textsubscript{3} А для мене то найменше, щоб судили мене ви чи суд людський, бо я й сам не суджу себе.
\end{tcolorbox}
\begin{tcolorbox}
\textsubscript{4} Я бо проти себе нічого не знаю, але цим не виправдуюсь; Той же, Хто судить мене, то Господь.
\end{tcolorbox}
\begin{tcolorbox}
\textsubscript{5} Тому не судіть передчасно нічого, аж поки не прийде Господь, що й висвітлить таємниці темряви та виявить задуми сердець, і тоді кожному буде похвала від Бога.
\end{tcolorbox}
\begin{tcolorbox}
\textsubscript{6} Оце ж усе, браття, приклав я до себе й Аполлоса ради вас, щоб від нас ви навчилися думати не більш, як написано, щоб ви не чванились один за одним перед іншим.
\end{tcolorbox}
\begin{tcolorbox}
\textsubscript{7} Хто бо тебе вирізняє? Що ти маєш, чого б ти не взяв? А коли ж бо ти взяв, чого чванишся, ніби не взяв?
\end{tcolorbox}
\begin{tcolorbox}
\textsubscript{8} Ви вже нагодовані, ви вже збагатилися, без нас ви царюєте. І коли б то ви стали царювати, щоб і ми царювали з вами!
\end{tcolorbox}
\begin{tcolorbox}
\textsubscript{9} Бо я думаю, що Бог нас, апостолів, поставив за найостанніших, мов на смерть засуджених, бо ми стали дивовищем світові, і Анголам, і людям.
\end{tcolorbox}
\begin{tcolorbox}
\textsubscript{10} Ми нерозумні Христа ради, а ви мудрі в Христі; ми слабі, ви ж міцні; ви славні, а ми безчесні!
\end{tcolorbox}
\begin{tcolorbox}
\textsubscript{11} Ми до цього часу і голодуємо, і прагнемо, і нагі ми, і катовані, і тиняємось,
\end{tcolorbox}
\begin{tcolorbox}
\textsubscript{12} і трудимось, працюючи своїми руками. Коли нас лихословлять, ми благословляємо; як нас переслідують, ми терпимо;
\end{tcolorbox}
\begin{tcolorbox}
\textsubscript{13} як лають, ми молимось; ми стали, як сміття те для світу, аж досі ми всім, як ті викидки!
\end{tcolorbox}
\begin{tcolorbox}
\textsubscript{14} Не пишу це для того, щоб вас осоромити, але остерігаю, як своїх любих дітей.
\end{tcolorbox}
\begin{tcolorbox}
\textsubscript{15} Бо хоч би ви мали десять тисяч наставників у Христі, та отців не багато; а я вас породив у Христі Ісусі через Євангелію...
\end{tcolorbox}
\begin{tcolorbox}
\textsubscript{16} Тож благаю я вас: будьте наслідувачами мене!
\end{tcolorbox}
\begin{tcolorbox}
\textsubscript{17} Для цього послав я до вас Тимофія, що для мене улюблений і вірний син у Господі, він вам нагадає шляхи мої в Христі Ісусі, як навчаю я скрізь у кожній Церкві.
\end{tcolorbox}
\begin{tcolorbox}
\textsubscript{18} Деякі згорділи, так немов би не мав я прийти до вас.
\end{tcolorbox}
\begin{tcolorbox}
\textsubscript{19} Та небавом прийду до вас, як захоче Господь, і пізнаю не слово згорділих, але силу.
\end{tcolorbox}
\begin{tcolorbox}
\textsubscript{20} Бо Царство Боже не в слові, а в силі.
\end{tcolorbox}
\begin{tcolorbox}
\textsubscript{21} Чого хочете? Чи прийти до вас з києм, чи з любов'ю та з духом лагідности?
\end{tcolorbox}
\subsection{CHAPTER 5}
\begin{tcolorbox}
\textsubscript{1} Всюди чути, що між вами перелюб, і то такий перелюб, який і між поганами незнаний, що хтось має за дружину собі дружину батькову...
\end{tcolorbox}
\begin{tcolorbox}
\textsubscript{2} І ви завеличалися, а не засмутились радніш, щоб був вилучений з-поміж вас, хто цей учинок зробив.
\end{tcolorbox}
\begin{tcolorbox}
\textsubscript{3} Отож я, відсутній тілом, та присутній духом, уже розсудив, як присутній між вами: того, хто так учинив це,
\end{tcolorbox}
\begin{tcolorbox}
\textsubscript{4} у Ім'я Господа Ісуса, як зберетеся ви та мій дух, із силою Господа нашого Ісуса,
\end{tcolorbox}
\begin{tcolorbox}
\textsubscript{5} віддати такого сатані на погибіль тіла, щоб дух спасся Господнього дня!
\end{tcolorbox}
\begin{tcolorbox}
\textsubscript{6} Величання ваше не добре. Хіба ви не знаєте, що мала розчина все тісто заквашує?
\end{tcolorbox}
\begin{tcolorbox}
\textsubscript{7} Отож, очистьте стару розчину, щоб стати вам новим тістом, бо ви прісні, бо наша Пасха, Христос, за нас у жертву принесений.
\end{tcolorbox}
\begin{tcolorbox}
\textsubscript{8} Тому святкуймо не в давній розчині, ані в розчині злоби й лукавства, але в опрісноках чистости та правди!
\end{tcolorbox}
\begin{tcolorbox}
\textsubscript{9} Я писав вам у листі не єднатися з перелюбниками,
\end{tcolorbox}
\begin{tcolorbox}
\textsubscript{10} але не взагалі з цьогосвітніми перелюбниками, чи з користолюбцями, чи з хижаками, чи з ідолянами, бо ви мусіли були б відійти від світу.
\end{tcolorbox}
\begin{tcolorbox}
\textsubscript{11} А тепер я писав вам не єднатися з тим, хто зветься братом, та є перелюбник, чи користолюбець, чи ідолянин, чи злоріка, чи п'яниця, чи хижак, із такими навіть не їсти!
\end{tcolorbox}
\begin{tcolorbox}
\textsubscript{12} Бо що ж мені судити й чужих? Чи ви не судите своїх?
\end{tcolorbox}
\begin{tcolorbox}
\textsubscript{13} А чужих судить Бог. Тож вилучіть лукавого з-поміж себе самих!
\end{tcolorbox}
\subsection{CHAPTER 6}
\begin{tcolorbox}
\textsubscript{1} Чи посміє хто з вас, маючи справу до іншого, судитися в неправедних, а не в святих?
\end{tcolorbox}
\begin{tcolorbox}
\textsubscript{2} Хіба ви не знаєте, що святі світ судитимуть? Коли ж будете ви світ судити, то чи ж ви негідні судити незначні справи?
\end{tcolorbox}
\begin{tcolorbox}
\textsubscript{3} Хіба ви не знаєте, що ми будем судити Анголів, а не тільки життєве?
\end{tcolorbox}
\begin{tcolorbox}
\textsubscript{4} А ви, коли маєте суд за життєве, то ставите суддями тих, хто нічого не значить у Церкві.
\end{tcolorbox}
\begin{tcolorbox}
\textsubscript{5} Я на сором це вам говорю. Чи ж між вами немає ні одного мудрого, щоб він міг розсудити між братами своїми?
\end{tcolorbox}
\begin{tcolorbox}
\textsubscript{6} Та брат судиться з братом, і то перед невірними!
\end{tcolorbox}
\begin{tcolorbox}
\textsubscript{7} Тож уже для вас сором зовсім, що суди між собою ви маєте. Чому краще не терпите кривди? Чому краще не маєте шкоди?
\end{tcolorbox}
\begin{tcolorbox}
\textsubscript{8} Але ви самі кривду чините та обдираєте, та ще братів...
\end{tcolorbox}
\begin{tcolorbox}
\textsubscript{9} Хіба ви не знаєте, що неправедні не вспадкують Божого Царства? Не обманюйте себе: ні розпусники, ні ідоляни, ні перелюбники, ні блудодійники, ні мужоложники,
\end{tcolorbox}
\begin{tcolorbox}
\textsubscript{10} ні злодії, ні користолюбці, ні п'яниці, ні злоріки, ні хижаки Царства Божого не вспадкують вони!
\end{tcolorbox}
\begin{tcolorbox}
\textsubscript{11} І такими були дехто з вас, але ви обмились, але освятились, але виправдались Іменем Господа Ісуса Христа й Духом нашого Бога.
\end{tcolorbox}
\begin{tcolorbox}
\textsubscript{12} Усе мені можна, та не все на пожиток. Усе мені можна, але мною ніщо володіти не повинно.
\end{tcolorbox}
\begin{tcolorbox}
\textsubscript{13} Їжа для черева, і черево для їжі, але Бог одне й друге понищить. А тіло не для розпусти, але для Господа, і Господь для тіла.
\end{tcolorbox}
\begin{tcolorbox}
\textsubscript{14} Бог же й Господа воскресив, воскресить Він і нас Своєю силою!
\end{tcolorbox}
\begin{tcolorbox}
\textsubscript{15} Хіба ви не знаєте, що ваші тіла то члени Христові? Отож, узявши члени Христові, зроблю їх членами розпусниці? Зовсім ні!
\end{tcolorbox}
\begin{tcolorbox}
\textsubscript{16} Хіба ви не знаєте, що той, хто злучується з розпусницею, стає одним тілом із нею? Бо каже: Обидва ви будете тілом одним.
\end{tcolorbox}
\begin{tcolorbox}
\textsubscript{17} А хто з Господом злучується, стає одним духом із Ним.
\end{tcolorbox}
\begin{tcolorbox}
\textsubscript{18} Утікайте від розпусти. Усякий бо гріх, що його чинить людина, є поза тілом. А хто чинить розпусту, той грішить проти власного тіла.
\end{tcolorbox}
\begin{tcolorbox}
\textsubscript{19} Хіба ви не знаєте, що ваше тіло то храм Духа Святого, що живе Він у вас, якого від Бога ви маєте, і ви не свої?
\end{tcolorbox}
\begin{tcolorbox}
\textsubscript{20} Бо дорого куплені ви. Отож прославляйте Бога в тілі своєму та в дусі своєму, що Божі вони!
\end{tcolorbox}
\subsection{CHAPTER 7}
\begin{tcolorbox}
\textsubscript{1} А про що ви писали мені, то добре було б чоловікові не дотикатися жінки.
\end{tcolorbox}
\begin{tcolorbox}
\textsubscript{2} Але щоб уникнути розпусти, нехай кожен муж має дружину свою, і кожна жінка хай має свого чоловіка.
\end{tcolorbox}
\begin{tcolorbox}
\textsubscript{3} Нехай віддає чоловік своїй дружині потрібну любов, так же само й чоловікові дружина.
\end{tcolorbox}
\begin{tcolorbox}
\textsubscript{4} Дружина не володіє над тілом своїм, але чоловік; так же само й чоловік не володіє над тілом своїм, але дружина.
\end{tcolorbox}
\begin{tcolorbox}
\textsubscript{5} Не вхиляйтесь одне від одного, хібащо дочасно за згодою, щоб бути в пості та молитві, та й сходьтеся знову докупи, щоб вас сатана не спокушував вашим нестриманням.
\end{tcolorbox}
\begin{tcolorbox}
\textsubscript{6} А це говорю вам як раду, а не як наказа.
\end{tcolorbox}
\begin{tcolorbox}
\textsubscript{7} Бо хочу, щоб усі чоловіки були, як і я; але кожен має від Бога свій дар, один так, інший так.
\end{tcolorbox}
\begin{tcolorbox}
\textsubscript{8} Говорю ж неодруженим і вдовам: добре їм, як вони позостануться так, як і я.
\end{tcolorbox}
\begin{tcolorbox}
\textsubscript{9} Коли ж не втримаються, нехай одружуються, бо краще женитися, ніж розпалятися.
\end{tcolorbox}
\begin{tcolorbox}
\textsubscript{10} А тим, що побрались, наказую не я, а Господь: Нехай не розлучається дружина з своїм чоловіком!
\end{tcolorbox}
\begin{tcolorbox}
\textsubscript{11} А коли ж і розлучиться, хай зостається незаміжня, або з чоловіком своїм хай помириться, і не відпускати чоловікові дружини!
\end{tcolorbox}
\begin{tcolorbox}
\textsubscript{12} Іншим же я говорю, не Господь: коли який брат має дружину невіруючу, і згідна вона жити з ним, нехай він не лишає її.
\end{tcolorbox}
\begin{tcolorbox}
\textsubscript{13} І жінка, як має чоловіка невіруючого, а той згоден жити з нею, нехай не лишає його.
\end{tcolorbox}
\begin{tcolorbox}
\textsubscript{14} Чоловік бо невіруючий освячується в дружині, а дружина невіруюча освячується в чоловікові. А інакше нечисті були б ваші діти, тепер же святі.
\end{tcolorbox}
\begin{tcolorbox}
\textsubscript{15} А як хоче невіруючий розлучитися, хай розлучиться, не неволиться брат чи сестра в такім разі, бо покликав нас Бог до миру.
\end{tcolorbox}
\begin{tcolorbox}
\textsubscript{16} Звідки знаєш ти, дружино, чи не спасеш чоловіка? Або звідки знаєш, чоловіче, чи не спасеш дружини?
\end{tcolorbox}
\begin{tcolorbox}
\textsubscript{17} Нехай тільки так ходить кожен, як кому Бог призначив, як Господь покликав його. І так усім Церквам я наказую.
\end{tcolorbox}
\begin{tcolorbox}
\textsubscript{18} Хто покликаний був в обрізанні, нехай він того не цурається; чи покликаний хто в необрізанні, нехай не обрізується.
\end{tcolorbox}
\begin{tcolorbox}
\textsubscript{19} Обрізання ніщо, і ніщо необрізання, а важливе дотримування Божих заповідей.
\end{tcolorbox}
\begin{tcolorbox}
\textsubscript{20} Нехай кожен лишається в стані такому, в якому покликаний був.
\end{tcolorbox}
\begin{tcolorbox}
\textsubscript{21} Чи покликаний був ти рабом? Не турбуйся про те. Але коли й можеш стати вільним, то використай краще це.
\end{tcolorbox}
\begin{tcolorbox}
\textsubscript{22} Бо покликаний в Господі раб визволенець Господній; так само покликаний і визволенець він раб Христа.
\end{tcolorbox}
\begin{tcolorbox}
\textsubscript{23} Ви дорого куплені, тож не ставайте рабами людей!
\end{tcolorbox}
\begin{tcolorbox}
\textsubscript{24} Браття, кожен із вас, в якім стані покликаний був, хай у тім перед Богом лишається!
\end{tcolorbox}
\begin{tcolorbox}
\textsubscript{25} Про дівчат же не маю наказу Господнього, але даю раду як той, хто одержав від Господа милість буть вірним.
\end{tcolorbox}
\begin{tcolorbox}
\textsubscript{26} Отож за сучасного утиску добрим уважаю я те, що чоловікові добре лишатися так.
\end{tcolorbox}
\begin{tcolorbox}
\textsubscript{27} Ти зв'язаний з дружиною? Не шукай розв'язання. Розв'язався від дружини? Не шукай дружини.
\end{tcolorbox}
\begin{tcolorbox}
\textsubscript{28} А коли ти й оженишся, то не згрішив; і як дівчина заміж піде, вона не згрішить. Та муку тілесну такі будуть мати, а мені шкода вас.
\end{tcolorbox}
\begin{tcolorbox}
\textsubscript{29} А це, браття, кажу я, бо час позосталий короткий, щоб і ті, що мають дружин, були, як ті, що не мають,
\end{tcolorbox}
\begin{tcolorbox}
\textsubscript{30} а хто плаче, як ті, хто не плаче, а хто тішиться, як ті, хто не тішиться; і хто купує, як би не набули,
\end{tcolorbox}
\begin{tcolorbox}
\textsubscript{31} а хто цьогосвітнім користується, як би не користувались, бо минає стан світу цього.
\end{tcolorbox}
\begin{tcolorbox}
\textsubscript{32} А я хочу, щоб ви безклопітні були. Неодружений про речі Господні клопочеться, як догодити Господеві,
\end{tcolorbox}
\begin{tcolorbox}
\textsubscript{33} а одружений про речі життєві клопочеться, як догодити своїй дружині,
\end{tcolorbox}
\begin{tcolorbox}
\textsubscript{34} і він поділений. Незаміжня ж жінка та дівчина про речі Господні клопочеться, щоб бути святою ті тілом, і духом. А заміжня про речі життєві клопочеться, як догодити чоловікові.
\end{tcolorbox}
\begin{tcolorbox}
\textsubscript{35} А це я кажу вам самим на пожиток, а не щоб сильце вам накинути, але щоб пристойно й горливо держались ви Господа.
\end{tcolorbox}
\begin{tcolorbox}
\textsubscript{36} А як думає хто про дівчину свою, що соромно, як вона переросте, і так мала б лишатись, нехай робить, що хоче, не згрішить: нехай заміж виходять.
\end{tcolorbox}
\begin{tcolorbox}
\textsubscript{37} А хто в серці своїм стоїть міцно, не має конечности, владу ж має над своєю волею, і це постановив він у серці своєму берегти свою дівчину, той робить добре.
\end{tcolorbox}
\begin{tcolorbox}
\textsubscript{38} Тому й той, хто віддає свою дівчину заміж, добре робить, а хто не віддає робить краще.
\end{tcolorbox}
\begin{tcolorbox}
\textsubscript{39} Дружина законом прив'язана, поки живе чоловік її; коли ж чоловік її вмре, вона вільна виходити заміж, за кого захоче, аби тільки в Господі.
\end{tcolorbox}
\begin{tcolorbox}
\textsubscript{40} Блаженніша вона, коли так позостанеться за моєю порадою, бо міркую, що й я маю Божого Духа.
\end{tcolorbox}
\subsection{CHAPTER 8}
\begin{tcolorbox}
\textsubscript{1} А щодо ідольських жертов, то ми знаємо, що всі маємо знання. Знання ж надимає, любов же будує!
\end{tcolorbox}
\begin{tcolorbox}
\textsubscript{2} Коли хто думає, ніби щось знає, той нічого не знає ще так, як знати повинно.
\end{tcolorbox}
\begin{tcolorbox}
\textsubscript{3} Коли ж любить хто Бога, той пізнаний Ним.
\end{tcolorbox}
\begin{tcolorbox}
\textsubscript{4} Тож про споживання ідольських жертов ми знаємо, що ідол у світі ніщо, і що іншого Бога нема, окрім Бога Одного.
\end{tcolorbox}
\begin{tcolorbox}
\textsubscript{5} Бо хоч і існують так звані боги чи на небі, чи то на землі, як існує багато богів і багато панів,
\end{tcolorbox}
\begin{tcolorbox}
\textsubscript{6} та для нас один Бог Отець, що з Нього походить усе, ми ж для Нього, і один Господь Ісус Христос, що все сталося Ним, і ми Ним.
\end{tcolorbox}
\begin{tcolorbox}
\textsubscript{7} Та не всі таке мають знання, бо деякі мають призвичаєння до ідола й досі, і їдять, як ідольську жертву, і їхнє сумління, бувши недуже, споганюється.
\end{tcolorbox}
\begin{tcolorbox}
\textsubscript{8} Їжа ж нас до Бога не зближує: бо коли не їмо, то нічого не тратимо, а коли ми їмо, то не набуваєм нічого.
\end{tcolorbox}
\begin{tcolorbox}
\textsubscript{9} Але стережіться, щоб ця ваша воля не стала якось за спотикання слабим!
\end{tcolorbox}
\begin{tcolorbox}
\textsubscript{10} Коли бо хто бачить тебе, маючого знання, як ти в ідольській божниці сидиш за столом, чи ж сумління його, бувши слабе, не буде спонукане їсти ідольські жертви?
\end{tcolorbox}
\begin{tcolorbox}
\textsubscript{11} І через знання твоє згине недужий твій брат, що за нього Христос був умер!
\end{tcolorbox}
\begin{tcolorbox}
\textsubscript{12} Грішачи так проти братів та вражаючи їхнє слабе сумління, ви проти Христа грішите.
\end{tcolorbox}
\begin{tcolorbox}
\textsubscript{13} Ось тому, коли їжа спокушує брата мого, то повік я не їстиму м'яса, щоб не спокусити брата свого!
\end{tcolorbox}
\subsection{CHAPTER 9}
\begin{tcolorbox}
\textsubscript{1} Хіба ж я не вільний? Чи ж я не апостол? Хіба я не бачив Ісуса Христа, Господа нашого? Хіба ви, то не справа моя перед Господом?
\end{tcolorbox}
\begin{tcolorbox}
\textsubscript{2} Коли я не апостол для інших, то для вас я апостол, ви бо печать мого апостольства в Господі.
\end{tcolorbox}
\begin{tcolorbox}
\textsubscript{3} Оце оборона моя перед тими, хто судить мене.
\end{tcolorbox}
\begin{tcolorbox}
\textsubscript{4} Чи ми права не маємо їсти та пити?
\end{tcolorbox}
\begin{tcolorbox}
\textsubscript{5} Чи ми права не маємо водити з собою сестру, дружину, як і інші апостоли, і Господні брати, і Кифа?
\end{tcolorbox}
\begin{tcolorbox}
\textsubscript{6} Хіба я один і Варнава не маємо права, щоб не працювати?
\end{tcolorbox}
\begin{tcolorbox}
\textsubscript{7} Хто коштом своїм коли служить у війську? Або хто виноградника садить, і не їсть з його плоду? Або хто отару пасе, і не їсть молока від отари?
\end{tcolorbox}
\begin{tcolorbox}
\textsubscript{8} Чи я тільки по-людському це говорю? Хіба ж і Закон не говорить цього?
\end{tcolorbox}
\begin{tcolorbox}
\textsubscript{9} Бо в Законі Мойсеєвім писано: Не в'яжи рота волові, що молотить. Хіба за волів Бог турбується?
\end{tcolorbox}
\begin{tcolorbox}
\textsubscript{10} Чи говорить Він зовсім для нас? Для нас, бо написано, що з надією мусить орати орач, а молотник молотити з надією мати частку в своїм сподіванні.
\end{tcolorbox}
\begin{tcolorbox}
\textsubscript{11} Коли ми сіяли вам духовне, чи ж велика то річ, як пожнемо ми ваше тілесне?
\end{tcolorbox}
\begin{tcolorbox}
\textsubscript{12} Як право на вас мають інші, то тим більше ми. Але ми не вжили цього права, та все терпимо, аби перешкоди якої Христовій Євангелії ми не вчинили.
\end{tcolorbox}
\begin{tcolorbox}
\textsubscript{13} Хіба ви не знаєте, що священнослужителі від святині годуються? Що ті, хто служить вівтареві, із вівтаря мають частку?
\end{tcolorbox}
\begin{tcolorbox}
\textsubscript{14} Так і Господь наказав проповідникам Євангелії жити з Євангелії.
\end{tcolorbox}
\begin{tcolorbox}
\textsubscript{15} Але з того нічого не вжив я. А цього не писав я для того, щоб для мене так було. Бо мені краще вмерти, аніж щоб хто знівечив хвалу мою!
\end{tcolorbox}
\begin{tcolorbox}
\textsubscript{16} Бо коли я звіщаю Євангелію, то нема чим хвалитись мені, це бо повинність моя. І горе мені, коли я не звіщаю Євангелії!
\end{tcolorbox}
\begin{tcolorbox}
\textsubscript{17} Тож коли це роблю добровільно, я маю нагороду; коли ж недобровільно, то виконую службу доручену.
\end{tcolorbox}
\begin{tcolorbox}
\textsubscript{18} Яка ж нагорода мені? Та, що, благовістячи, я безкорисливо проповідував Христову Євангелію, не використовуючи особистих прав щодо благовістя.
\end{tcolorbox}
\begin{tcolorbox}
\textsubscript{19} Від усіх бувши вільний, я зробився рабом для всіх, щоб найбільше придбати.
\end{tcolorbox}
\begin{tcolorbox}
\textsubscript{20} Для юдеїв я був, як юдей, щоб юдеїв придбати; для підзаконних був, як підзаконний, хоч сам підзаконним не бувши, щоб придбати підзаконних.
\end{tcolorbox}
\begin{tcolorbox}
\textsubscript{21} Для тих, хто без Закону, я був беззаконний, не бувши беззаконний Богові, а законний Христові, щоб придбати беззаконних.
\end{tcolorbox}
\begin{tcolorbox}
\textsubscript{22} Для слабих, як слабий, щоб придбати слабих. Для всіх я був усе, щоб спасти бодай деяких.
\end{tcolorbox}
\begin{tcolorbox}
\textsubscript{23} А це я роблю для Євангелії, щоб стати її спільником.
\end{tcolorbox}
\begin{tcolorbox}
\textsubscript{24} Хіба ви не знаєте, що ті, хто на перегонах біжить, усі біжать, але нагороду приймає один? Біжіть так, щоб одержали ви!
\end{tcolorbox}
\begin{tcolorbox}
\textsubscript{25} І кожен змагун від усього стримується; вони ж щоб тлінний прийняти вінок, але ми щоб нетлінний.
\end{tcolorbox}
\begin{tcolorbox}
\textsubscript{26} Тож біжу я не так, немов на непевне, борюся не так, немов би повітря б'ючи.
\end{tcolorbox}
\begin{tcolorbox}
\textsubscript{27} Але вмертвляю й неволю я тіло своє, щоб, звіщаючи іншим, не стати самому негідним.
\end{tcolorbox}
\subsection{CHAPTER 10}
\begin{tcolorbox}
\textsubscript{1} Не хочу я, браття, щоб ви не знали, що під хмарою всі отці наші були, і всі перейшли через море,
\end{tcolorbox}
\begin{tcolorbox}
\textsubscript{2} і всі охристилися в хмарі та в морі в Мойсея,
\end{tcolorbox}
\begin{tcolorbox}
\textsubscript{3} і всі їли ту саму поживу духовну,
\end{tcolorbox}
\begin{tcolorbox}
\textsubscript{4} і пили всі той самий духовний напій, бо пили від духовної скелі, що йшла вслід за ними, а та скеля був Христос!
\end{tcolorbox}
\begin{tcolorbox}
\textsubscript{5} Але їх багатьох не вподобав був Бог, бо понищив Він їх у пустині.
\end{tcolorbox}
\begin{tcolorbox}
\textsubscript{6} А це були приклади нам, щоб ми пожадливі на зле не були, як були пожадливі й вони.
\end{tcolorbox}
\begin{tcolorbox}
\textsubscript{7} Не будьте також ідолянами, як деякі з них, як написано: Люди сіли, щоб їсти та пити, і встали, щоб грати.
\end{tcolorbox}
\begin{tcolorbox}
\textsubscript{8} Не станьмо чинити блуду, як деякі з них блудодіяли, і полягло їх одного дня двадцять три тисячі.
\end{tcolorbox}
\begin{tcolorbox}
\textsubscript{9} Ані не випробовуймо Христа, як деякі з них випробовували, та й від зміїв загинули.
\end{tcolorbox}
\begin{tcolorbox}
\textsubscript{10} Ані не нарікайте, як деякі з них нарікали, і загинули від погубителя.
\end{tcolorbox}
\begin{tcolorbox}
\textsubscript{11} Усе це трапилось з ними, як приклади, а написане нам на науку, бо за нашого часу кінець віку прийшов.
\end{tcolorbox}
\begin{tcolorbox}
\textsubscript{12} Тому то, хто думає, ніби стоїть він, нехай стережеться, щоб не впасти!
\end{tcolorbox}
\begin{tcolorbox}
\textsubscript{13} Досягла вас спроба не інша, тільки людська; але вірний Бог, Який не попустить, щоб ви випробовувалися більше, ніж можете, але при спробі й полегшення дасть, щоб знести могли ви її.
\end{tcolorbox}
\begin{tcolorbox}
\textsubscript{14} Тому, мої любі, утікайте від служіння ідолам.
\end{tcolorbox}
\begin{tcolorbox}
\textsubscript{15} Кажу, як розумним; судіть самі, що кажу я.
\end{tcolorbox}
\begin{tcolorbox}
\textsubscript{16} Чаша благословення, яку благословляємо, чи не спільнота то крови Христової? Хліб, який ломимо, чи не спільнота він тіла Христового?
\end{tcolorbox}
\begin{tcolorbox}
\textsubscript{17} Тому що один хліб, тіло одне нас багато, бо ми всі спільники хліба одного.
\end{tcolorbox}
\begin{tcolorbox}
\textsubscript{18} Погляньте на Ізраїля за тілом: чи ж ті, що жертви їдять, не спільники вівтаря?
\end{tcolorbox}
\begin{tcolorbox}
\textsubscript{19} Тож що я кажу? Що ідольська жертва є щось? Чи що ідол є щось?
\end{tcolorbox}
\begin{tcolorbox}
\textsubscript{20} Ні, але те, що в жертву приносять, демонам, а не Богові в жертву приносять. Я ж не хочу, щоб ви спільниками для демонів стали.
\end{tcolorbox}
\begin{tcolorbox}
\textsubscript{21} Бо не можете пити чаші Господньої та чаші демонської; не можете бути спільниками Господнього столу й столу демонського.
\end{tcolorbox}
\begin{tcolorbox}
\textsubscript{22} Чи ми дратуватимем Господа? Хіба ми потужніші за Нього?
\end{tcolorbox}
\begin{tcolorbox}
\textsubscript{23} Усе мені можна, та не все на пожиток. Усе мені можна, та будує не все!
\end{tcolorbox}
\begin{tcolorbox}
\textsubscript{24} Нехай не шукає ніхто свого власного, але кожен для ближнього!
\end{tcolorbox}
\begin{tcolorbox}
\textsubscript{25} Їжте все, що на ятках м'ясних продається, за сумління зовсім не турбуючись,
\end{tcolorbox}
\begin{tcolorbox}
\textsubscript{26} Бо Господня земля, і все, що на ній!
\end{tcolorbox}
\begin{tcolorbox}
\textsubscript{27} Як покличе вас хтось із невіруючих, і ви захочете піти, їжте все, що дадуть вам, за сумління зовсім не турбуючись.
\end{tcolorbox}
\begin{tcolorbox}
\textsubscript{28} Коли ж скаже вам хтось: Це ідольська жертва, не їжте тоді через того, хто сказав, та через сумління!
\end{tcolorbox}
\begin{tcolorbox}
\textsubscript{29} Говорю ж не про власне сумління, але іншого, чого б моя воля судилась сумлінням чужим?
\end{tcolorbox}
\begin{tcolorbox}
\textsubscript{30} Коли я стаю спільником їжі з подякою, чому мене зневажають за те, за що дякую я?
\end{tcolorbox}
\begin{tcolorbox}
\textsubscript{31} Тож, коли ви їсте, чи коли ви п'єте, або коли інше що робите, усе на Божу славу робіть!
\end{tcolorbox}
\begin{tcolorbox}
\textsubscript{32} Не робіть спокуси юдеям та гелленам, та Церкві Божій,
\end{tcolorbox}
\begin{tcolorbox}
\textsubscript{33} як догоджую й я всім у всьому, не шукаючи в тому пожитку свого, але пожитку для багатьох, щоб спаслися вони.
\end{tcolorbox}
\subsection{CHAPTER 11}
\begin{tcolorbox}
\textsubscript{1} Будьте наслідувачами мене, як і я Христа!
\end{tcolorbox}
\begin{tcolorbox}
\textsubscript{2} Похваляю ж вас, браття, що ви все моє пам'ятаєте, і заховуєте так передання, як я вам передав.
\end{tcolorbox}
\begin{tcolorbox}
\textsubscript{3} Хочу ж я, щоб ви знали, що всякому чоловікові голова Христос, а жінці голова чоловік, голова ж Христові Бог.
\end{tcolorbox}
\begin{tcolorbox}
\textsubscript{4} Кожен чоловік, що молиться чи пророкує з головою покритою, осоромлює він свою голову.
\end{tcolorbox}
\begin{tcolorbox}
\textsubscript{5} І кожна жінка, що молиться чи пророкує з головою відкритою, осоромлює тим свою голову, бо це є те саме, як була б вона виголена.
\end{tcolorbox}
\begin{tcolorbox}
\textsubscript{6} Бо коли жінка не покривається, хай стрижеться вона; коли ж жінці сором стригтися чи голитися, нехай покривається!
\end{tcolorbox}
\begin{tcolorbox}
\textsubscript{7} Отож, чоловік покривати голови не повинен, бо він образ і слава Бога, а жінка чоловікові слава.
\end{tcolorbox}
\begin{tcolorbox}
\textsubscript{8} Бо чоловік не походить від жінки, але жінка від чоловіка,
\end{tcolorbox}
\begin{tcolorbox}
\textsubscript{9} не створений бо чоловік ради жінки, але жінка ради чоловіка.
\end{tcolorbox}
\begin{tcolorbox}
\textsubscript{10} Тому жінка повина мати на голові знака влади над нею, ради Анголів.
\end{tcolorbox}
\begin{tcolorbox}
\textsubscript{11} Одначе в Господі ані чоловік без жінки, ані жінка без чоловіка.
\end{tcolorbox}
\begin{tcolorbox}
\textsubscript{12} Бо як жінка від чоловіка, так і чоловік через жінку; а все від Бога.
\end{tcolorbox}
\begin{tcolorbox}
\textsubscript{13} Поміркуйте самі між собою, чи пристойне воно, щоб жінка молилася Богові непокрита?
\end{tcolorbox}
\begin{tcolorbox}
\textsubscript{14} Чи ж природа сама вас не вчить, що коли чоловік запускає волосся, то безчестя для нього?
\end{tcolorbox}
\begin{tcolorbox}
\textsubscript{15} Коли ж жінка косу запускає, це слава для неї, бо замість покривала дана коса їй.
\end{tcolorbox}
\begin{tcolorbox}
\textsubscript{16} Коли ж хто сперечатися хоче, ми такого звичаю не маємо, ані Церкви Божі.
\end{tcolorbox}
\begin{tcolorbox}
\textsubscript{17} Пропонуючи це вам, я не хвалю, що збираєтесь ви не на ліпше, а на гірше.
\end{tcolorbox}
\begin{tcolorbox}
\textsubscript{18} Бо найперше, я чую, що як сходитесь ви на збори, то між вами бувають поділення, у що почасти я й вірю.
\end{tcolorbox}
\begin{tcolorbox}
\textsubscript{19} Бо мусять між вами й поділи бути, щоб відкрились між вами й досвідчені.
\end{tcolorbox}
\begin{tcolorbox}
\textsubscript{20} А далі, коли ви збираєтесь разом, то не на те, щоб їсти Господню Вечерю.
\end{tcolorbox}
\begin{tcolorbox}
\textsubscript{21} Бо кожен спішить з'їсти власну вечерю, і один голодує, а другий впивається.
\end{tcolorbox}
\begin{tcolorbox}
\textsubscript{22} Хіба ж ви не маєте хат, щоб їсти та пити? Чи ви зневажаєте Божу Церкву, і осоромлюєте немаючих? Що маю сказати вам? Чи за це похвалю вас? Не похвалю!
\end{tcolorbox}
\begin{tcolorbox}
\textsubscript{23} Бо прийняв я від Господа, що й вам передав, що Господь Ісус ночі тієї, як виданий був, узяв хліб,
\end{tcolorbox}
\begin{tcolorbox}
\textsubscript{24} подяку віддав, і переломив, і сказав: Прийміть, споживайте, це тіло Моє, що за вас ломається. Це робіть на спомин про Мене!
\end{tcolorbox}
\begin{tcolorbox}
\textsubscript{25} Так само і чашу взяв Він по Вечері й сказав: Ця чаша Новий Заповіт у Моїй крові. Це робіть, коли тільки будете пити, на спомин про Мене!
\end{tcolorbox}
\begin{tcolorbox}
\textsubscript{26} Бо кожного разу, як будете їсти цей хліб та чашу цю пити, смерть Господню звіщаєте, аж доки Він прийде.
\end{tcolorbox}
\begin{tcolorbox}
\textsubscript{27} Тому то, хто їстиме хліб цей чи питиме чашу Господню негідно, буде винний супроти тіла та крови Господньої!
\end{tcolorbox}
\begin{tcolorbox}
\textsubscript{28} Нехай же людина випробовує себе, і так нехай хліб їсть і з чаші хай п'є.
\end{tcolorbox}
\begin{tcolorbox}
\textsubscript{29} Бо хто їсть і п'є негідно, не розважаючи про тіло, той суд собі їсть і п'є!
\end{tcolorbox}
\begin{tcolorbox}
\textsubscript{30} Через це поміж вами багато недужих та хворих, і багато-хто заснули.
\end{tcolorbox}
\begin{tcolorbox}
\textsubscript{31} Бо коли б ми самі судили себе, то засуджені ми не були б.
\end{tcolorbox}
\begin{tcolorbox}
\textsubscript{32} Та засуджені від Господа, караємося, щоб нас не засуджено з світом.
\end{tcolorbox}
\begin{tcolorbox}
\textsubscript{33} Ось тому, мої браття, сходячись на поживу, чекайте один одного.
\end{tcolorbox}
\begin{tcolorbox}
\textsubscript{34} А коли хто голодний, нехай вдома він їсть, щоб не сходилися ви на осуд. А про інше, як прийду, заряджу.
\end{tcolorbox}
\subsection{CHAPTER 12}
\begin{tcolorbox}
\textsubscript{1} А щодо духовних дарів, то не хочу я, браття, щоб не відали ви.
\end{tcolorbox}
\begin{tcolorbox}
\textsubscript{2} Знаєте, що коли ви поганами були, то ходили до німих ідолів, ніби воджено вас.
\end{tcolorbox}
\begin{tcolorbox}
\textsubscript{3} Тому то кажу вам, що ніхто, хто говорить Духом Божим, не скаже: Нехай анатема буде на Ісуса, і не може сказати ніхто: Ісус то Господь, як тільки Духом Святим.
\end{tcolorbox}
\begin{tcolorbox}
\textsubscript{4} Є різниця між дарами милости, Дух же той Самий.
\end{tcolorbox}
\begin{tcolorbox}
\textsubscript{5} Є й різниця між служіннями, та Господь той же Самий.
\end{tcolorbox}
\begin{tcolorbox}
\textsubscript{6} Є різниця й між діями, але Бог той же Самий, що в усіх робить усе.
\end{tcolorbox}
\begin{tcolorbox}
\textsubscript{7} І кожному дається виявлення Духа на користь.
\end{tcolorbox}
\begin{tcolorbox}
\textsubscript{8} Одному бо Духом дається слово мудрости, а другому слово знання тим же Духом,
\end{tcolorbox}
\begin{tcolorbox}
\textsubscript{9} а іншому віра тим же Духом, а іншому дари вздоровлення тим же Духом,
\end{tcolorbox}
\begin{tcolorbox}
\textsubscript{10} а іншому роблення чуд, а іншому пророкування, а іншому розпізнавання духів, а тому різні мови, а іншому вияснення мов.
\end{tcolorbox}
\begin{tcolorbox}
\textsubscript{11} А все оце чинить один і той Самий Дух, уділяючи кожному осібно, як Він хоче.
\end{tcolorbox}
\begin{tcolorbox}
\textsubscript{12} Бо як тіло одне, але має членів багато, усі ж члени тіла, хоч їх багато, то тіло одне, так і Христос.
\end{tcolorbox}
\begin{tcolorbox}
\textsubscript{13} Бо ми всі одним Духом охрищені в тіло одне, чи то юдеї, чи геллени, чи раби, чи то вільні, і всі ми напоєні Духом одним.
\end{tcolorbox}
\begin{tcolorbox}
\textsubscript{14} Бо тіло не є один член, а багато.
\end{tcolorbox}
\begin{tcolorbox}
\textsubscript{15} Коли скаже нога, що я не від тіла, бо я не рука, то хіба через це не від тіла вона?
\end{tcolorbox}
\begin{tcolorbox}
\textsubscript{16} І коли скаже вухо, що я не від тіла, бо я не око, то хіба через це не від тіла воно?
\end{tcolorbox}
\begin{tcolorbox}
\textsubscript{17} Коли б оком було ціле тіло, то де був би слух? А коли б усе слух, то де був би нюх?
\end{tcolorbox}
\begin{tcolorbox}
\textsubscript{18} Та нині Бог розклав члени в тілі, кожного з них, як хотів.
\end{tcolorbox}
\begin{tcolorbox}
\textsubscript{19} Якби всі одним членом були, то де тіло було б?
\end{tcolorbox}
\begin{tcolorbox}
\textsubscript{20} Отож, тепер членів багато, та тіло одне.
\end{tcolorbox}
\begin{tcolorbox}
\textsubscript{21} Бо око не може сказати руці: Ти мені непотрібна; або голова знов ногам: Ви мені непотрібні.
\end{tcolorbox}
\begin{tcolorbox}
\textsubscript{22} Але члени тіла, що здаються слабіші, значно більше потрібні.
\end{tcolorbox}
\begin{tcolorbox}
\textsubscript{23} А тим, що вважаємо їх за зовсім нешановані в тілі, таким честь найбільшу приносимо, і бридкі наші члени отримують пристойність найбільшу,
\end{tcolorbox}
\begin{tcolorbox}
\textsubscript{24} а нашим пристойним того не потрібно. Та Бог змішав тіло, і честь більшу дав нижчому членові,
\end{tcolorbox}
\begin{tcolorbox}
\textsubscript{25} щоб поділення в тілі не було, а щоб члени однаково дбали один про одного.
\end{tcolorbox}
\begin{tcolorbox}
\textsubscript{26} І коли терпить один член, то всі члени з ним терплять; і коли один член пошанований, то всі члени з ним тішаться.
\end{tcolorbox}
\begin{tcolorbox}
\textsubscript{27} І ви тіло Христове, а зосібна ви члени!
\end{tcolorbox}
\begin{tcolorbox}
\textsubscript{28} А інших поставив Бог у Церкві поперше апостолами, подруге пророками, потретє учителями, потім дав сили, також дари вздоровлення, допомоги, управління, різні мови.
\end{tcolorbox}
\begin{tcolorbox}
\textsubscript{29} Чи ж усі апостоли? Чи ж усі пророки? Чи ж усі вчителі? Чи ж усі сили чудодійні?
\end{tcolorbox}
\begin{tcolorbox}
\textsubscript{30} Чи ж усі мають дари вздоровлення? Чи ж мовами всі розмовляють? Чи ж усі виясняють?
\end{tcolorbox}
\begin{tcolorbox}
\textsubscript{31} Тож дбайте ревно про ліпші дари, а я вам покажу путь іще кращу!
\end{tcolorbox}
\subsection{CHAPTER 13}
\begin{tcolorbox}
\textsubscript{1} Коли я говорю мовами людськими й ангольськими, та любови не маю, то став я як мідь та дзвінка або бубон гудячий!
\end{tcolorbox}
\begin{tcolorbox}
\textsubscript{2} І коли маю дара пророкувати, і знаю всі таємниці й усе знання, і коли маю всю віру, щоб навіть гори переставляти, та любови не маю, то я ніщо!
\end{tcolorbox}
\begin{tcolorbox}
\textsubscript{3} І коли я роздам усі маєтки свої, і коли я віддам своє тіло на спалення, та любови не маю, то пожитку не матиму жадного!
\end{tcolorbox}
\begin{tcolorbox}
\textsubscript{4} Любов довготерпить, любов милосердствує, не заздрить, любов не величається, не надимається,
\end{tcolorbox}
\begin{tcolorbox}
\textsubscript{5} не поводиться нечемно, не шукає тільки свого, не рветься до гніву, не думає лихого,
\end{tcolorbox}
\begin{tcolorbox}
\textsubscript{6} не радіє з неправди, але тішиться правдою,
\end{tcolorbox}
\begin{tcolorbox}
\textsubscript{7} усе зносить, вірить у все, сподівається всього, усе терпить!
\end{tcolorbox}
\begin{tcolorbox}
\textsubscript{8} Ніколи любов не перестає! Хоч пророцтва й існують, та припиняться, хоч мови існують, замовкнуть, хоч існує знання, та скасується.
\end{tcolorbox}
\begin{tcolorbox}
\textsubscript{9} Бо ми знаємо частинно, і пророкуємо частинно;
\end{tcolorbox}
\begin{tcolorbox}
\textsubscript{10} коли ж досконале настане, тоді зупиниться те, що частинне.
\end{tcolorbox}
\begin{tcolorbox}
\textsubscript{11} Коли я дитиною був, то я говорив, як дитина, як дитина я думав, розумів, як дитина. Коли ж мужем я став, то відкинув дитяче.
\end{tcolorbox}
\begin{tcolorbox}
\textsubscript{12} Отож, тепер бачимо ми ніби у дзеркалі, у загадці, але потім обличчям в обличчя; тепер розумію частинно, а потім пізнаю, як і пізнаний я.
\end{tcolorbox}
\begin{tcolorbox}
\textsubscript{13} А тепер залишаються віра, надія, любов, оці три. А найбільша між ними любов!
\end{tcolorbox}
\subsection{CHAPTER 14}
\begin{tcolorbox}
\textsubscript{1} Дбайте про любов, і про духовне пильнуйте, а найбільше щоб пророкувати.
\end{tcolorbox}
\begin{tcolorbox}
\textsubscript{2} Як говорить хто чужою мовою, той не людям говорить, а Богові, бо ніхто його не розуміє, і він духом говорить таємне.
\end{tcolorbox}
\begin{tcolorbox}
\textsubscript{3} А хто пророкує, той людям говорить на збудування, і на умовлення, і на розраду.
\end{tcolorbox}
\begin{tcolorbox}
\textsubscript{4} Як говорить хто чужою мовою, той будує тільки самого себе, а хто пророкує, той Церкву будує.
\end{tcolorbox}
\begin{tcolorbox}
\textsubscript{5} Я ж хочу, щоб мовами говорили всі, а ліпше щоб пророкували: більший бо той, хто пророкує, аніж той, хто говорить мовами, хібащо пояснює, щоб будувалася Церква.
\end{tcolorbox}
\begin{tcolorbox}
\textsubscript{6} А тепер, як прийду я до вас, браття, і до вас говорити буду чужою мовою, то який вам пожиток зроблю, коли не поясню вам чи то відкриттям, чи знанням, чи пророцтвом, чи наукою?
\end{tcolorbox}
\begin{tcolorbox}
\textsubscript{7} Бо навіть і речі бездушні, що звук видають, як сопілка чи лютня, коли б не видавали вони різних звуків, як пізнати б тоді, що бринить або грає?
\end{tcolorbox}
\begin{tcolorbox}
\textsubscript{8} Бо коли сурма звук невиразний дає, хто до бою готовитись буде?
\end{tcolorbox}
\begin{tcolorbox}
\textsubscript{9} Так і ви, коли мовою не подасте зрозумілого слова, як пізнати, що кажете? Ви говоритимете на вітер!
\end{tcolorbox}
\begin{tcolorbox}
\textsubscript{10} Як багато, наприклад, різних мов є на світі, і жадна з них не без значення!
\end{tcolorbox}
\begin{tcolorbox}
\textsubscript{11} І коли я не знатиму значення слів, то я буду чужинцем промовцеві, і промовець чужинцем мені.
\end{tcolorbox}
\begin{tcolorbox}
\textsubscript{12} Так і ви, що пильнуєте про духовні дари, дбайте, щоб збагачуватись через них на збудування Церкви!
\end{tcolorbox}
\begin{tcolorbox}
\textsubscript{13} Ось тому, хто говорить чужою мовою, нехай молиться, щоб умів виясняти.
\end{tcolorbox}
\begin{tcolorbox}
\textsubscript{14} Бо коли я молюся чужою мовою, то молиться дух мій, а мій розум без плоду!
\end{tcolorbox}
\begin{tcolorbox}
\textsubscript{15} Ну, то що ж? Буду молитися духом, і буду молитися й розумом, співатиму духом, і співатиму й розумом.
\end{tcolorbox}
\begin{tcolorbox}
\textsubscript{16} Бо коли благословлятимеш духом, то як той, що займає місце простої людини, промовить амінь на подяку твою? Не знає бо він, що ти кажеш.
\end{tcolorbox}
\begin{tcolorbox}
\textsubscript{17} Ти дякуєш добре, але не будується інший.
\end{tcolorbox}
\begin{tcolorbox}
\textsubscript{18} Дякую Богові моєму, розмовляю я мовами більше всіх вас.
\end{tcolorbox}
\begin{tcolorbox}
\textsubscript{19} Але в Церкві волію п'ять слів зрозумілих сказати, щоб і інших навчити, аніж десять тисяч слів чужою мовою!
\end{tcolorbox}
\begin{tcolorbox}
\textsubscript{20} Браття, не будьте дітьми своїм розумом, будьте в лихому дітьми, а в розумі досконалими будьте!
\end{tcolorbox}
\begin{tcolorbox}
\textsubscript{21} У Законі написано: Іншими мовами й іншими устами Я говоритиму людям оцим, та Мене вони й так не послухають, каже Господь.
\end{tcolorbox}
\begin{tcolorbox}
\textsubscript{22} Отож, мови існують на знак не для віруючих, але для невіруючих, а пророцтво для віруючих, а не для невіруючих.
\end{tcolorbox}
\begin{tcolorbox}
\textsubscript{23} А як зійдеться Церква вся разом, і всі говоритимуть чужими мовами, і ввійдуть туди й сторонні чи невіруючі, чи ж не скажуть вони, що біснуєтесь ви?
\end{tcolorbox}
\begin{tcolorbox}
\textsubscript{24} Коли ж усі пророкують, а ввійде якийсь невіруючий чи сторонній, то всі докоряють йому, усі судять його,
\end{tcolorbox}
\begin{tcolorbox}
\textsubscript{25} і так таємниці серця його виявляються, і так він падає ницьма і вклоняється Богові й каже: Бог справді між вами!
\end{tcolorbox}
\begin{tcolorbox}
\textsubscript{26} То що ж, браття? Коли сходитесь ви, то кожен із вас псалом має, має науку, має мову, об'явлення має, має вияснення, нехай буде все це на збудування!
\end{tcolorbox}
\begin{tcolorbox}
\textsubscript{27} Як говорить хто чужою мовою, говоріть по двох, чи найбільше по трьох, і то за чергою, а один нехай перекладає!
\end{tcolorbox}
\begin{tcolorbox}
\textsubscript{28} А коли б не було перекладача, то нехай він у Церкві мовчить, а говорить нехай собі й Богові!
\end{tcolorbox}
\begin{tcolorbox}
\textsubscript{29} А пророки нехай промовляють по двох чи по трьох, а інші нехай розпізнають.
\end{tcolorbox}
\begin{tcolorbox}
\textsubscript{30} Коли ж відкриття буде іншому з тих, хто сидить, нехай перший замовкне!
\end{tcolorbox}
\begin{tcolorbox}
\textsubscript{31} Бо можете пророкувати ви всі по одному, щоб училися всі й усі тішилися!
\end{tcolorbox}
\begin{tcolorbox}
\textsubscript{32} І коряться духи пророчі пророкам,
\end{tcolorbox}
\begin{tcolorbox}
\textsubscript{33} бо Бог не є Богом безладу, але миру. Як по всіх Церквах у святих,
\end{tcolorbox}
\begin{tcolorbox}
\textsubscript{34} нехай у Церкві мовчать жінки ваші! Бо їм говорити не позволено, тільки коритись, як каже й Закон.
\end{tcolorbox}
\begin{tcolorbox}
\textsubscript{35} Коли ж вони хочуть навчитись чогось, нехай вдома питають своїх чоловіків, непристойно бо жінці говорити в Церкві!
\end{tcolorbox}
\begin{tcolorbox}
\textsubscript{36} Хіба вийшло від вас Слово Боже? Чи прийшло воно тільки до вас?
\end{tcolorbox}
\begin{tcolorbox}
\textsubscript{37} Коли хто вважає себе за пророка або за духовного, нехай розуміє, що я пишу вам, бо Господня це заповідь!
\end{tcolorbox}
\begin{tcolorbox}
\textsubscript{38} Коли б же хто не розумів, нехай не розуміє!
\end{tcolorbox}
\begin{tcolorbox}
\textsubscript{39} Отож, браття мої, майте ревність пророкувати, та не бороніть говорити й мовами!
\end{tcolorbox}
\begin{tcolorbox}
\textsubscript{40} Але все нехай буде добропристойно і статечно!
\end{tcolorbox}
\subsection{CHAPTER 15}
\begin{tcolorbox}
\textsubscript{1} Звіщаю ж вам, браття, Євангелію, яку я вам благовістив, і яку прийняли ви, в якій і стоїте,
\end{tcolorbox}
\begin{tcolorbox}
\textsubscript{2} Якою й спасаєтесь, коли пам'ятаєте, яким словом я благовістив вам, якщо тільки ви ввірували не наосліп.
\end{tcolorbox}
\begin{tcolorbox}
\textsubscript{3} Бо я передав вам найперш, що й прийняв, що Христос був умер ради наших гріхів за Писанням,
\end{tcolorbox}
\begin{tcolorbox}
\textsubscript{4} і що Він був похований, і що третього дня Він воскрес за Писанням,
\end{tcolorbox}
\begin{tcolorbox}
\textsubscript{5} і що з'явився Він Кифі, потім Дванадцятьом.
\end{tcolorbox}
\begin{tcolorbox}
\textsubscript{6} А потім з'явився нараз більше як п'ятистам браттям, що більшість із них живе й досі, а дехто й спочили.
\end{tcolorbox}
\begin{tcolorbox}
\textsubscript{7} Потому з'явився Він Якову, опісля усім апостолам.
\end{tcolorbox}
\begin{tcolorbox}
\textsubscript{8} А по всіх Він з'явився й мені, мов якому недородкові.
\end{tcolorbox}
\begin{tcolorbox}
\textsubscript{9} Я бо найменший з апостолів, що негідний зватись апостолом, бо я переслідував був Божу Церкву.
\end{tcolorbox}
\begin{tcolorbox}
\textsubscript{10} Та благодаттю Божою я те, що є, і благодать Його, що в мені, не даремна була, але я працював більше всіх їх, правда не я, але Божа благодать, що зо мною вона.
\end{tcolorbox}
\begin{tcolorbox}
\textsubscript{11} Тож чи я, чи вони, ми так проповідуємо, і так ви ввірували.
\end{tcolorbox}
\begin{tcolorbox}
\textsubscript{12} Коли ж про Христа проповідується, що воскрес Він із мертвих, як же дехто між вами говорять, що немає воскресення мертвих?
\end{tcolorbox}
\begin{tcolorbox}
\textsubscript{13} Як немає ж воскресення мертвих, то й Христос не воскрес!
\end{tcolorbox}
\begin{tcolorbox}
\textsubscript{14} оли ж бо Христос не воскрес, то проповідь наша даремна, даремна також віра ваша!
\end{tcolorbox}
\begin{tcolorbox}
\textsubscript{15} Ми знайшлися б тоді неправдивими свідками Божими, бо про Бога ми свідчили, що воскресив Він Христа, Якого Він не воскресив, якщо не воскресають померлі.
\end{tcolorbox}
\begin{tcolorbox}
\textsubscript{16} Бо як мертві не воскресають, то й Христос не воскрес!
\end{tcolorbox}
\begin{tcolorbox}
\textsubscript{17} Коли ж бо Христос не воскрес, тоді віра ваша даремна, ви в своїх ще гріхах,
\end{tcolorbox}
\begin{tcolorbox}
\textsubscript{18} тоді то загинули й ті, що в Христі упокоїлись!
\end{tcolorbox}
\begin{tcolorbox}
\textsubscript{19} Коли ми надіємося на Христа тільки в цьому житті, то ми найнещасніші від усіх людей!
\end{tcolorbox}
\begin{tcolorbox}
\textsubscript{20} Та нині Христос воскрес із мертвих, первісток серед покійних.
\end{tcolorbox}
\begin{tcolorbox}
\textsubscript{21} Смерть бо через людину, і через Людину воскресення мертвих.
\end{tcolorbox}
\begin{tcolorbox}
\textsubscript{22} Бо так, як в Адамі вмирають усі, так само в Христі всі оживуть,
\end{tcolorbox}
\begin{tcolorbox}
\textsubscript{23} кожен у своєму порядку: первісток Христос, потім ті, що Христові, під час Його приходу.
\end{tcolorbox}
\begin{tcolorbox}
\textsubscript{24} А потому кінець, коли Він передасть царство Богові й Отцеві, коли Він зруйнує всякий уряд, і владу всяку та силу.
\end{tcolorbox}
\begin{tcolorbox}
\textsubscript{25} Бо належить Йому царювати, аж доки Він не покладе всіх Своїх ворогів під ногами Своїми!
\end{tcolorbox}
\begin{tcolorbox}
\textsubscript{26} Як ворог останній смерть знищиться,
\end{tcolorbox}
\begin{tcolorbox}
\textsubscript{27} бо під ноги Його Він усе впокорив. Коли ж каже, що впокорено все, то ясно, що все, окрім Того, Хто впокорив Йому все.
\end{tcolorbox}
\begin{tcolorbox}
\textsubscript{28} А коли Йому все Він упокорить, тоді й Сам Син упокориться Тому, Хто все впокорив Йому, щоб Бог був у всьому все.
\end{tcolorbox}
\begin{tcolorbox}
\textsubscript{29} Бо що зроблять ті, хто христяться ради мертвих? Коли мертві не воскресають зовсім, то нащо вони ради мертвих і христяться?
\end{tcolorbox}
\begin{tcolorbox}
\textsubscript{30} Для чого й ми повсякчас наражаємось на небезпеки?
\end{tcolorbox}
\begin{tcolorbox}
\textsubscript{31} Я щодень умираю. Так свідчу, браття, вашою хвалою, що маю її в Христі Ісусі, Господі нашім.
\end{tcolorbox}
\begin{tcolorbox}
\textsubscript{32} Коли я зо звірами боровся в Ефесі, яка мені по-людському користь, коли мертві не воскресають? Будем їсти та пити, бо ми взавтра вмрем!...
\end{tcolorbox}
\begin{tcolorbox}
\textsubscript{33} Не дайте себе звести, товариство лихе псує добрі звичаї!
\end{tcolorbox}
\begin{tcolorbox}
\textsubscript{34} Протверезіться правдиво, та й не грішіть, бо деякі Бога не знають, говорю вам на сором!
\end{tcolorbox}
\begin{tcolorbox}
\textsubscript{35} Але дехто скаже: Як мертві воскреснуть? І в якім тілі прийдуть?
\end{tcolorbox}
\begin{tcolorbox}
\textsubscript{36} Нерозумний, що ти сієш, те не оживе, як не вмре.
\end{tcolorbox}
\begin{tcolorbox}
\textsubscript{37} І коли сієш, то сієш не тіло майбутнє, але голе зерно, яке трапиться, пшениці або чого іншого,
\end{tcolorbox}
\begin{tcolorbox}
\textsubscript{38} і Бог йому тіло дає, як захоче, і кожному зерняті тіло його.
\end{tcolorbox}
\begin{tcolorbox}
\textsubscript{39} Не кожне тіло однакове тіло, але ж інше в людей, та інше тіло в скотини, та інше тіло в пташок, та інше у риб.
\end{tcolorbox}
\begin{tcolorbox}
\textsubscript{40} Є небесні тіла й тіла земні, але ж інша слава небесним, а інша земним.
\end{tcolorbox}
\begin{tcolorbox}
\textsubscript{41} Інша слава для сонця, та інша слава для місяця, та інша слава для зір, бо зоря від зорі відрізняється славою!
\end{tcolorbox}
\begin{tcolorbox}
\textsubscript{42} Так само й воскресення мертвих: сіється в тління, в нетління встає,
\end{tcolorbox}
\begin{tcolorbox}
\textsubscript{43} сіється в неславу, у славі встає, сіється в немочі, у силі встає,
\end{tcolorbox}
\begin{tcolorbox}
\textsubscript{44} сіється тіло звичайне, встає тіло духовне. Є тіло звичайне, є й тіло духовне.
\end{tcolorbox}
\begin{tcolorbox}
\textsubscript{45} Так і написано: Перша людина Адам став душею живою, а останній Адам то дух оживляючий.
\end{tcolorbox}
\begin{tcolorbox}
\textsubscript{46} Та не перше духовне, але звичайне, а потім духовне.
\end{tcolorbox}
\begin{tcolorbox}
\textsubscript{47} Перша людина з землі, земна, друга Людина із неба Господь.
\end{tcolorbox}
\begin{tcolorbox}
\textsubscript{48} Який земний, такі й земні, і Який небесний, такі й небесні.
\end{tcolorbox}
\begin{tcolorbox}
\textsubscript{49} І, як носили ми образ земного, так і образ небесного будемо носити.
\end{tcolorbox}
\begin{tcolorbox}
\textsubscript{50} І це скажу, браття, що тіло й кров посісти Божого Царства не можуть, ані тління нетління не посяде.
\end{tcolorbox}
\begin{tcolorbox}
\textsubscript{51} Ось кажу я вам таємницю: не всі ми заснемо, та всі перемінимось,
\end{tcolorbox}
\begin{tcolorbox}
\textsubscript{52} раптом, як оком змигнути, при останній сурмі: бо засурмить вона і мертві воскреснуть, а ми перемінимось!...
\end{tcolorbox}
\begin{tcolorbox}
\textsubscript{53} Мусить бо тлінне оце зодягнутись в нетління, а смертне оце зодягтися в безсмертя.
\end{tcolorbox}
\begin{tcolorbox}
\textsubscript{54} А коли оце тлінне в нетління зодягнеться, і оце смертне в безсмертя зодягнеться, тоді збудеться слово написане: Поглинута смерть перемогою!
\end{tcolorbox}
\begin{tcolorbox}
\textsubscript{55} Де, смерте, твоя перемога? Де твоє, смерте, жало?
\end{tcolorbox}
\begin{tcolorbox}
\textsubscript{56} Жало ж смерти то гріх, а сила гріха то Закон.
\end{tcolorbox}
\begin{tcolorbox}
\textsubscript{57} А Богові дяка, що Він Господом нашим Ісусом Христом перемогу нам дав.
\end{tcolorbox}
\begin{tcolorbox}
\textsubscript{58} Отож, брати любі мої, будьте міцні, непохитні, збагачуйтесь завжди в Господньому ділі, знаючи, що ваша праця не марнотна у Господі!
\end{tcolorbox}
\subsection{CHAPTER 16}
\begin{tcolorbox}
\textsubscript{1} А щодо складок на святих, то й ви робіть так, як я постановив для Церков галатійських.
\end{tcolorbox}
\begin{tcolorbox}
\textsubscript{2} А першого дня в тижні нехай кожен із вас відкладає собі та збирає, згідно з тим, як ведеться йому, щоб складок не робити тоді, аж коли я прийду.
\end{tcolorbox}
\begin{tcolorbox}
\textsubscript{3} А коли я прийду, тоді тих, кого виберете, тих пошлю я з листами, щоб вони ваш дар любови віднесли до Єрусалиму.
\end{tcolorbox}
\begin{tcolorbox}
\textsubscript{4} А коли ж і мені випадатиме йти, то зо мною підуть.
\end{tcolorbox}
\begin{tcolorbox}
\textsubscript{5} Я прибуду до вас, коли перейду Македонію, бо проходжу через Македонію.
\end{tcolorbox}
\begin{tcolorbox}
\textsubscript{6} А в вас, коли трапиться, я поживу або й перезимую, щоб мене провели ви, куди я піду.
\end{tcolorbox}
\begin{tcolorbox}
\textsubscript{7} Не хочу я бачитись з вами тепер мимохідь, але сподіваюся деякий час перебути у вас, як дозволить Господь.
\end{tcolorbox}
\begin{tcolorbox}
\textsubscript{8} А в Ефесі пробуду я до П'ятдесятниці,
\end{tcolorbox}
\begin{tcolorbox}
\textsubscript{9} бо двері великі й широкі мені відчинилися, та багато противників...
\end{tcolorbox}
\begin{tcolorbox}
\textsubscript{10} Коли ж прийде до вас Тимофій, то пильнуйте, щоб він був безпечний у вас, бо діло Господнє він робить, як і я.
\end{tcolorbox}
\begin{tcolorbox}
\textsubscript{11} Тому то нехай ним ніхто не погорджує, але відпровадьте його з миром, щоб прийшов він до мене, бо чекаю його з братами.
\end{tcolorbox}
\begin{tcolorbox}
\textsubscript{12} А щодо брата Аполлоса, то я дуже благав був його, щоб прийшов до вас з братами, та охоти не мав він прибути тепер, але прийде, як матиме час відповідний.
\end{tcolorbox}
\begin{tcolorbox}
\textsubscript{13} Пильнуйте, стійте у вірі, будьте мужні, будьте міцні,
\end{tcolorbox}
\begin{tcolorbox}
\textsubscript{14} хай з любов'ю все робиться в вас!
\end{tcolorbox}
\begin{tcolorbox}
\textsubscript{15} Благаю ж вас, браття, знаєте ви дім Степанів, що в Ахаї він первісток, і що службі святим присвятились вони,
\end{tcolorbox}
\begin{tcolorbox}
\textsubscript{16} і ви підкоряйтесь таким, також кожному, хто помагає та працює.
\end{tcolorbox}
\begin{tcolorbox}
\textsubscript{17} Я тішусь з приходу Степана, і Фортуната, і Ахаїка, бо вашу відсутність вони заступили,
\end{tcolorbox}
\begin{tcolorbox}
\textsubscript{18} бо вони заспокоїли духа мого й вашого. Тож шануйте таких!
\end{tcolorbox}
\begin{tcolorbox}
\textsubscript{19} Вітають вас азійські Церкви; Акила й Прискилла з домашньою Церквою їхньою гаряче вітають у Господі вас.
\end{tcolorbox}
\begin{tcolorbox}
\textsubscript{20} Вітають вас усі брати. Вітайте один одного святим поцілунком.
\end{tcolorbox}
\begin{tcolorbox}
\textsubscript{21} Привітання моєю рукою Павловою.
\end{tcolorbox}
\begin{tcolorbox}
\textsubscript{22} Коли хто не любить Господа, нехай буде проклятий! Марана та!
\end{tcolorbox}
\begin{tcolorbox}
\textsubscript{23} Благодать Господа нашого Ісуса нехай буде з вами!
\end{tcolorbox}
\begin{tcolorbox}
\textsubscript{24} Любов моя з вами всіма у Христі Ісусі, амінь!
\end{tcolorbox}
