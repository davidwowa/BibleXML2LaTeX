\section{BOOK 150}
\subsection{CHAPTER 1}
\begin{tcolorbox}
\textsubscript{1} Павел, волею Божиею Апостол Иисуса Христа, и Тимофей брат,
\end{tcolorbox}
\begin{tcolorbox}
\textsubscript{2} находящимся в Колоссах святым и верным братиям во Христе Иисусе:
\end{tcolorbox}
\begin{tcolorbox}
\textsubscript{3} благодать вам и мир от Бога Отца нашего и Господа Иисуса Христа. Благодарим Бога и Отца Господа нашего Иисуса Христа, всегда молясь о вас,
\end{tcolorbox}
\begin{tcolorbox}
\textsubscript{4} услышав о вере вашей во Христа Иисуса и о любви ко всем святым,
\end{tcolorbox}
\begin{tcolorbox}
\textsubscript{5} в надежде на уготованное вам на небесах, о чем вы прежде слышали в истинном слове благовествования,
\end{tcolorbox}
\begin{tcolorbox}
\textsubscript{6} которое пребывает у вас, как и во всем мире, и приносит плод, и возрастает, как и между вами, с того дня, как вы услышали и познали благодать Божию в истине,
\end{tcolorbox}
\begin{tcolorbox}
\textsubscript{7} как и научились от Епафраса, возлюбленного сотрудника нашего, верного для вас служителя Христова,
\end{tcolorbox}
\begin{tcolorbox}
\textsubscript{8} который и известил нас о вашей любви в духе.
\end{tcolorbox}
\begin{tcolorbox}
\textsubscript{9} Посему и мы с того дня, как [о сем] услышали, не перестаем молиться о вас и просить, чтобы вы исполнялись познанием воли Его, во всякой премудрости и разумении духовном,
\end{tcolorbox}
\begin{tcolorbox}
\textsubscript{10} чтобы поступали достойно Бога, во всем угождая [Ему], принося плод во всяком деле благом и возрастая в познании Бога,
\end{tcolorbox}
\begin{tcolorbox}
\textsubscript{11} укрепляясь всякою силою по могуществу славы Его, во всяком терпении и великодушии с радостью,
\end{tcolorbox}
\begin{tcolorbox}
\textsubscript{12} благодаря Бога и Отца, призвавшего нас к участию в наследии святых во свете,
\end{tcolorbox}
\begin{tcolorbox}
\textsubscript{13} избавившего нас от власти тьмы и введшего в Царство возлюбленного Сына Своего,
\end{tcolorbox}
\begin{tcolorbox}
\textsubscript{14} в Котором мы имеем искупление Кровию Его и прощение грехов,
\end{tcolorbox}
\begin{tcolorbox}
\textsubscript{15} Который есть образ Бога невидимого, рожденный прежде всякой твари;
\end{tcolorbox}
\begin{tcolorbox}
\textsubscript{16} ибо Им создано всё, что на небесах и что на земле, видимое и невидимое: престолы ли, господства ли, начальства ли, власти ли, --все Им и для Него создано;
\end{tcolorbox}
\begin{tcolorbox}
\textsubscript{17} и Он есть прежде всего, и все Им стоит.
\end{tcolorbox}
\begin{tcolorbox}
\textsubscript{18} И Он есть глава тела Церкви; Он--начаток, первенец из мертвых, дабы иметь Ему во всем первенство,
\end{tcolorbox}
\begin{tcolorbox}
\textsubscript{19} ибо благоугодно было [Отцу], чтобы в Нем обитала всякая полнота,
\end{tcolorbox}
\begin{tcolorbox}
\textsubscript{20} и чтобы посредством Его примирить с Собою все, умиротворив через Него, Кровию креста Его, и земное и небесное.
\end{tcolorbox}
\begin{tcolorbox}
\textsubscript{21} И вас, бывших некогда отчужденными и врагами, по расположению к злым делам,
\end{tcolorbox}
\begin{tcolorbox}
\textsubscript{22} ныне примирил в теле Плоти Его, смертью Его, [чтобы] представить вас святыми и непорочными и неповинными пред Собою,
\end{tcolorbox}
\begin{tcolorbox}
\textsubscript{23} если только пребываете тверды и непоколебимы в вере и не отпадаете от надежды благовествования, которое вы слышали, которое возвещено всей твари поднебесной, которого я, Павел, сделался служителем.
\end{tcolorbox}
\begin{tcolorbox}
\textsubscript{24} Ныне радуюсь в страданиях моих за вас и восполняю недостаток в плоти моей скорбей Христовых за Тело Его, которое есть Церковь,
\end{tcolorbox}
\begin{tcolorbox}
\textsubscript{25} которой сделался я служителем по домостроительству Божию, вверенному мне для вас, [чтобы] исполнить слово Божие,
\end{tcolorbox}
\begin{tcolorbox}
\textsubscript{26} тайну, сокрытую от веков и родов, ныне же открытую святым Его,
\end{tcolorbox}
\begin{tcolorbox}
\textsubscript{27} Которым благоволил Бог показать, какое богатство славы в тайне сей для язычников, которая есть Христос в вас, упование славы,
\end{tcolorbox}
\begin{tcolorbox}
\textsubscript{28} Которого мы проповедуем, вразумляя всякого человека и научая всякой премудрости, чтобы представить всякого человека совершенным во Христе Иисусе;
\end{tcolorbox}
\begin{tcolorbox}
\textsubscript{29} для чего я и тружусь и подвизаюсь силою Его, действующею во мне могущественно.
\end{tcolorbox}
\subsection{CHAPTER 2}
\begin{tcolorbox}
\textsubscript{1} Желаю, чтобы вы знали, какой подвиг имею я ради вас и ради тех, которые в Лаодикии и Иераполе, и ради всех, кто не видел лица моего в плоти,
\end{tcolorbox}
\begin{tcolorbox}
\textsubscript{2} дабы утешились сердца их, соединенные в любви для всякого богатства совершенного разумения, для познания тайны Бога и Отца и Христа,
\end{tcolorbox}
\begin{tcolorbox}
\textsubscript{3} в Котором сокрыты все сокровища премудрости и ведения.
\end{tcolorbox}
\begin{tcolorbox}
\textsubscript{4} Это говорю я для того, чтобы кто-нибудь не прельстил вас вкрадчивыми словами;
\end{tcolorbox}
\begin{tcolorbox}
\textsubscript{5} ибо хотя я и отсутствую телом, но духом нахожусь с вами, радуясь и видя ваше благоустройство и твердость веры вашей во Христа.
\end{tcolorbox}
\begin{tcolorbox}
\textsubscript{6} Посему, как вы приняли Христа Иисуса Господа, [так] и ходите в Нем,
\end{tcolorbox}
\begin{tcolorbox}
\textsubscript{7} будучи укоренены и утверждены в Нем и укреплены в вере, как вы научены, преуспевая в ней с благодарением.
\end{tcolorbox}
\begin{tcolorbox}
\textsubscript{8} Смотрите, братия, чтобы кто не увлек вас философиею и пустым обольщением, по преданию человеческому, по стихиям мира, а не по Христу;
\end{tcolorbox}
\begin{tcolorbox}
\textsubscript{9} ибо в Нем обитает вся полнота Божества телесно,
\end{tcolorbox}
\begin{tcolorbox}
\textsubscript{10} и вы имеете полноту в Нем, Который есть глава всякого начальства и власти.
\end{tcolorbox}
\begin{tcolorbox}
\textsubscript{11} В Нем вы и обрезаны обрезанием нерукотворенным, совлечением греховного тела плоти, обрезанием Христовым;
\end{tcolorbox}
\begin{tcolorbox}
\textsubscript{12} быв погребены с Ним в крещении, в Нем вы и совоскресли верою в силу Бога, Который воскресил Его из мертвых,
\end{tcolorbox}
\begin{tcolorbox}
\textsubscript{13} и вас, которые были мертвы во грехах и в необрезании плоти вашей, оживил вместе с Ним, простив нам все грехи,
\end{tcolorbox}
\begin{tcolorbox}
\textsubscript{14} истребив учением бывшее о нас рукописание, которое было против нас, и Он взял его от среды и пригвоздил ко кресту;
\end{tcolorbox}
\begin{tcolorbox}
\textsubscript{15} отняв силы у начальств и властей, властно подверг их позору, восторжествовав над ними Собою.
\end{tcolorbox}
\begin{tcolorbox}
\textsubscript{16} Итак никто да не осуждает вас за пищу, или питие, или за какой-нибудь праздник, или новомесячие, или субботу:
\end{tcolorbox}
\begin{tcolorbox}
\textsubscript{17} это есть тень будущего, а тело--во Христе.
\end{tcolorbox}
\begin{tcolorbox}
\textsubscript{18} Никто да не обольщает вас самовольным смиренномудрием и служением Ангелов, вторгаясь в то, чего не видел, безрассудно надмеваясь плотским своим умом
\end{tcolorbox}
\begin{tcolorbox}
\textsubscript{19} и не держась главы, от которой все тело, составами и связями будучи соединяемо и скрепляемо, растет возрастом Божиим.
\end{tcolorbox}
\begin{tcolorbox}
\textsubscript{20} Итак, если вы со Христом умерли для стихий мира, то для чего вы, как живущие в мире, держитесь постановлений:
\end{tcolorbox}
\begin{tcolorbox}
\textsubscript{21} 'не прикасайся', 'не вкушай', 'не дотрагивайся' --
\end{tcolorbox}
\begin{tcolorbox}
\textsubscript{22} что все истлевает от употребления, --по заповедям и учению человеческому?
\end{tcolorbox}
\begin{tcolorbox}
\textsubscript{23} Это имеет только вид мудрости в самовольном служении, смиренномудрии и изнурении тела, в некотором небрежении о насыщении плоти.
\end{tcolorbox}
\subsection{CHAPTER 3}
\begin{tcolorbox}
\textsubscript{1} Итак, если вы воскресли со Христом, то ищите горнего, где Христос сидит одесную Бога;
\end{tcolorbox}
\begin{tcolorbox}
\textsubscript{2} о горнем помышляйте, а не о земном.
\end{tcolorbox}
\begin{tcolorbox}
\textsubscript{3} Ибо вы умерли, и жизнь ваша сокрыта со Христом в Боге.
\end{tcolorbox}
\begin{tcolorbox}
\textsubscript{4} Когда же явится Христос, жизнь ваша, тогда и вы явитесь с Ним во славе.
\end{tcolorbox}
\begin{tcolorbox}
\textsubscript{5} Итак, умертвите земные члены ваши: блуд, нечистоту, страсть, злую похоть и любостяжание, которое есть идолослужение,
\end{tcolorbox}
\begin{tcolorbox}
\textsubscript{6} за которые гнев Божий грядет на сынов противления,
\end{tcolorbox}
\begin{tcolorbox}
\textsubscript{7} в которых и вы некогда обращались, когда жили между ними.
\end{tcolorbox}
\begin{tcolorbox}
\textsubscript{8} А теперь вы отложите все: гнев, ярость, злобу, злоречие, сквернословие уст ваших;
\end{tcolorbox}
\begin{tcolorbox}
\textsubscript{9} не говорите лжи друг другу, совлекшись ветхого человека с делами его
\end{tcolorbox}
\begin{tcolorbox}
\textsubscript{10} и облекшись в нового, который обновляется в познании по образу Создавшего его,
\end{tcolorbox}
\begin{tcolorbox}
\textsubscript{11} где нет ни Еллина, ни Иудея, ни обрезания, ни необрезания, варвара, Скифа, раба, свободного, но все и во всем Христос.
\end{tcolorbox}
\begin{tcolorbox}
\textsubscript{12} Итак облекитесь, как избранные Божии, святые и возлюбленные, в милосердие, благость, смиренномудрие, кротость, долготерпение,
\end{tcolorbox}
\begin{tcolorbox}
\textsubscript{13} снисходя друг другу и прощая взаимно, если кто на кого имеет жалобу: как Христос простил вас, так и вы.
\end{tcolorbox}
\begin{tcolorbox}
\textsubscript{14} Более же всего [облекитесь] в любовь, которая есть совокупность совершенства.
\end{tcolorbox}
\begin{tcolorbox}
\textsubscript{15} И да владычествует в сердцах ваших мир Божий, к которому вы и призваны в одном теле, и будьте дружелюбны.
\end{tcolorbox}
\begin{tcolorbox}
\textsubscript{16} Слово Христово да вселяется в вас обильно, со всякою премудростью; научайте и вразумляйте друг друга псалмами, славословием и духовными песнями, во благодати воспевая в сердцах ваших Господу.
\end{tcolorbox}
\begin{tcolorbox}
\textsubscript{17} И всё, что вы делаете, словом или делом, всё [делайте] во имя Господа Иисуса Христа, благодаря через Него Бога и Отца.
\end{tcolorbox}
\begin{tcolorbox}
\textsubscript{18} Жены, повинуйтесь мужьям своим, как прилично в Господе.
\end{tcolorbox}
\begin{tcolorbox}
\textsubscript{19} Мужья, любите своих жен и не будьте к ним суровы.
\end{tcolorbox}
\begin{tcolorbox}
\textsubscript{20} Дети, будьте послушны родителям вашим во всем, ибо это благоугодно Господу.
\end{tcolorbox}
\begin{tcolorbox}
\textsubscript{21} Отцы, не раздражайте детей ваших, дабы они не унывали.
\end{tcolorbox}
\begin{tcolorbox}
\textsubscript{22} Рабы, во всем повинуйтесь господам вашим по плоти, не в глазах только служа [им], как человекоугодники, но в простоте сердца, боясь Бога.
\end{tcolorbox}
\begin{tcolorbox}
\textsubscript{23} И всё, что делаете, делайте от души, как для Господа, а не для человеков,
\end{tcolorbox}
\begin{tcolorbox}
\textsubscript{24} зная, что в воздаяние от Господа получите наследие, ибо вы служите Господу Христу.
\end{tcolorbox}
\begin{tcolorbox}
\textsubscript{25} А кто неправо поступит, тот получит по своей неправде, [у Него] нет лицеприятия.
\end{tcolorbox}
\subsection{CHAPTER 4}
\begin{tcolorbox}
\textsubscript{1} Господа, оказывайте рабам должное и справедливое, зная, что и вы имеете Господа на небесах.
\end{tcolorbox}
\begin{tcolorbox}
\textsubscript{2} Будьте постоянны в молитве, бодрствуя в ней с благодарением.
\end{tcolorbox}
\begin{tcolorbox}
\textsubscript{3} Молитесь также и о нас, чтобы Бог отверз нам дверь для слова, возвещать тайну Христову, за которую я и в узах,
\end{tcolorbox}
\begin{tcolorbox}
\textsubscript{4} дабы я открыл ее, как должно мне возвещать.
\end{tcolorbox}
\begin{tcolorbox}
\textsubscript{5} Со внешними обходитесь благоразумно, пользуясь временем.
\end{tcolorbox}
\begin{tcolorbox}
\textsubscript{6} Слово ваше [да будет] всегда с благодатию, приправлено солью, дабы вы знали, как отвечать каждому.
\end{tcolorbox}
\begin{tcolorbox}
\textsubscript{7} О мне всё скажет вам Тихик, возлюбленный брат и верный служитель и сотрудник в Господе,
\end{tcolorbox}
\begin{tcolorbox}
\textsubscript{8} которого я для того послал к вам, чтобы он узнал о ваших [обстоятельствах] и утешил сердца ваши,
\end{tcolorbox}
\begin{tcolorbox}
\textsubscript{9} с Онисимом, верным и возлюбленным братом нашим, который от вас. Они расскажут вам о всем здешнем.
\end{tcolorbox}
\begin{tcolorbox}
\textsubscript{10} Приветствует вас Аристарх, заключенный вместе со мною, и Марк, племянник Варнавы--о котором вы получили приказания: если придет к вам, примите его, --
\end{tcolorbox}
\begin{tcolorbox}
\textsubscript{11} также Иисус, прозываемый Иустом, оба из обрезанных. Они--единственные сотрудники для Царствия Божия, бывшие мне отрадою.
\end{tcolorbox}
\begin{tcolorbox}
\textsubscript{12} Приветствует вас Епафрас ваш, раб Иисуса Христа, всегда подвизающийся за вас в молитвах, чтобы вы пребыли совершенны и исполнены всем, что угодно Богу.
\end{tcolorbox}
\begin{tcolorbox}
\textsubscript{13} Свидетельствую о нем, что он имеет великую ревность и заботу о вас и о находящихся в Лаодикии и Иераполе.
\end{tcolorbox}
\begin{tcolorbox}
\textsubscript{14} Приветствует вас Лука, врач возлюбленный, и Димас.
\end{tcolorbox}
\begin{tcolorbox}
\textsubscript{15} Приветствуйте братьев в Лаодикии, и Нимфана, и домашнюю церковь его.
\end{tcolorbox}
\begin{tcolorbox}
\textsubscript{16} Когда это послание прочитано будет у вас, то распорядитесь, чтобы оно было прочитано и в Лаодикийской церкви; а то, которое из Лаодикии, прочитайте и вы.
\end{tcolorbox}
\begin{tcolorbox}
\textsubscript{17} Скажите Архиппу: смотри, чтобы тебе исполнить служение, которое ты принял в Господе.
\end{tcolorbox}
\begin{tcolorbox}
\textsubscript{18} Приветствие моею рукою, Павловою. Помните мои узы. Благодать со всеми вами. Аминь.
\end{tcolorbox}
