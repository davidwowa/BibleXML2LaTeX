\section{BOOK 19}
\subsection{CHAPTER 1}
\begin{tcolorbox}
\textsubscript{1} Приповісті Соломона, сина Давидового, царя Ізраїлевого,
\end{tcolorbox}
\begin{tcolorbox}
\textsubscript{2} щоб пізнати премудрість і карність, щоб зрозуміти розсудні слова,
\end{tcolorbox}
\begin{tcolorbox}
\textsubscript{3} щоб прийняти напоумлення мудрости, праведности, і права й простоти,
\end{tcolorbox}
\begin{tcolorbox}
\textsubscript{4} щоб мудрости дати простодушним, юнакові пізнання й розважність.
\end{tcolorbox}
\begin{tcolorbox}
\textsubscript{5} Хай послухає мудрий і примножить науку, а розумний здобуде хай мудрих думок,
\end{tcolorbox}
\begin{tcolorbox}
\textsubscript{6} щоб пізнати ту приповість та загадкове говорення, слова мудреців та їхні загадки.
\end{tcolorbox}
\begin{tcolorbox}
\textsubscript{7} Страх Господній початок премудрости, нерозумні погорджують мудрістю та напучуванням.
\end{tcolorbox}
\begin{tcolorbox}
\textsubscript{8} Послухай, мій сину, напучення батька свого, і не відкидай науки матері своєї,
\end{tcolorbox}
\begin{tcolorbox}
\textsubscript{9} вони бо хороший вінок для твоєї голови, і прикраса на шию твою.
\end{tcolorbox}
\begin{tcolorbox}
\textsubscript{10} Мій сину, як грішники будуть тебе намовляти, то з ними не згоджуйся ти!
\end{tcolorbox}
\begin{tcolorbox}
\textsubscript{11} Якщо скажуть вони: Ходи з нами, чатуймо на кров, безпричинно засядьмо на неповинного,
\end{tcolorbox}
\begin{tcolorbox}
\textsubscript{12} живих поковтаймо ми їх, як шеол, та здорових, як тих, які сходять до гробу!
\end{tcolorbox}
\begin{tcolorbox}
\textsubscript{13} Ми знайдемо всіляке багатство цінне, переповнимо здобиччю наші хати.
\end{tcolorbox}
\begin{tcolorbox}
\textsubscript{14} Жеребок свій ти кинеш із нами, буде саква одна для всіх нас,
\end{tcolorbox}
\begin{tcolorbox}
\textsubscript{15} сину мій, не ходи ти дорогою з ними, спини ногу свою від їхньої стежки,
\end{tcolorbox}
\begin{tcolorbox}
\textsubscript{16} бо біжать їхні ноги на зло, і поспішають, щоб кров проливати!
\end{tcolorbox}
\begin{tcolorbox}
\textsubscript{17} Бож надармо поставлена сітка на очах усього крилатого:
\end{tcolorbox}
\begin{tcolorbox}
\textsubscript{18} то вони на кров власну чатують, засідають на душу свою!
\end{tcolorbox}
\begin{tcolorbox}
\textsubscript{19} Такі то дороги усіх, хто заздрий чужого добра: воно бере душу свого власника!
\end{tcolorbox}
\begin{tcolorbox}
\textsubscript{20} Кличе мудрість на вулиці, на площах свій голос дає,
\end{tcolorbox}
\begin{tcolorbox}
\textsubscript{21} на шумливих місцях проповідує, у місті при входах до брам вона каже слова свої:
\end{tcolorbox}
\begin{tcolorbox}
\textsubscript{22} Доки ви, нерозумні, глупоту любитимете? Аж доки насмішники будуть кохатись собі в глузуванні, а безглузді ненавидіти будуть знання?
\end{tcolorbox}
\begin{tcolorbox}
\textsubscript{23} Зверніться но ви до картання мого, ось я виллю вам духа свого, сповіщу вам слова свої!
\end{tcolorbox}
\begin{tcolorbox}
\textsubscript{24} Бо кликала я, та відмовились ви, простягла була руку свою, та ніхто не прислухувався!
\end{tcolorbox}
\begin{tcolorbox}
\textsubscript{25} І всю раду мою ви відкинули, картання ж мого не схотіли!
\end{tcolorbox}
\begin{tcolorbox}
\textsubscript{26} Тож у вашім нещасті сміятися буду і я, насміхатися буду, як прийде ваш страх.
\end{tcolorbox}
\begin{tcolorbox}
\textsubscript{27} Коли прийде ваш страх, немов вихор, і привалиться ваше нещастя, мов буря, як прийде недоля та утиск на вас,
\end{tcolorbox}
\begin{tcolorbox}
\textsubscript{28} тоді кликати будуть мене, але не відповім, будуть шукати мене, та не знайдуть мене,
\end{tcolorbox}
\begin{tcolorbox}
\textsubscript{29} за те, що науку зненавиділи, і не вибрали страху Господнього,
\end{tcolorbox}
\begin{tcolorbox}
\textsubscript{30} не хотіли поради моєї, погорджували всіма моїми докорами!
\end{tcolorbox}
\begin{tcolorbox}
\textsubscript{31} І тому хай їдять вони з плоду дороги своєї, а з порад своїх хай насищаються,
\end{tcolorbox}
\begin{tcolorbox}
\textsubscript{32} бо відступство безумних заб'є їх, і безпечність безтямних їх вигубить!
\end{tcolorbox}
\begin{tcolorbox}
\textsubscript{33} А хто мене слухає, той буде жити безпечно, і буде спокійний від страху перед злом!
\end{tcolorbox}
\subsection{CHAPTER 2}
\begin{tcolorbox}
\textsubscript{1} Сину мій, якщо приймеш слова мої ти, а накази мої при собі заховаєш,
\end{tcolorbox}
\begin{tcolorbox}
\textsubscript{2} щоб слухало мудрости вухо твоє, своє серце прихилиш до розуму,
\end{tcolorbox}
\begin{tcolorbox}
\textsubscript{3} якщо до розсудку ти кликати будеш, до розуму кликатимеш своїм голосом,
\end{tcolorbox}
\begin{tcolorbox}
\textsubscript{4} якщо будеш шукати його, немов срібла, і будеш його ти пошукувати, як тих схованих скарбів,
\end{tcolorbox}
\begin{tcolorbox}
\textsubscript{5} тоді зрозумієш страх Господній, і знайдеш ти Богопізнання,
\end{tcolorbox}
\begin{tcolorbox}
\textsubscript{6} бо Господь дає мудрість, з Його уст знання й розум!
\end{tcolorbox}
\begin{tcolorbox}
\textsubscript{7} Він спасіння ховає для щирих, мов щит той для тих, хто в невинності ходить,
\end{tcolorbox}
\begin{tcolorbox}
\textsubscript{8} щоб справедливих стежок стерегти, і береже Він дорогу Своїх богобійних!
\end{tcolorbox}
\begin{tcolorbox}
\textsubscript{9} Тоді ти збагнеш справедливість та право, і простоту, всіляку дорогу добра,
\end{tcolorbox}
\begin{tcolorbox}
\textsubscript{10} бо мудрість увійде до серця твого, і буде приємне знання для твоєї душі!
\end{tcolorbox}
\begin{tcolorbox}
\textsubscript{11} розважність тоді тебе пильнуватиме, розум тебе стерегтиме,
\end{tcolorbox}
\begin{tcolorbox}
\textsubscript{12} щоб тебе врятувати від злої дороги, від людини, що каже лукаве,
\end{tcolorbox}
\begin{tcolorbox}
\textsubscript{13} від тих, хто стежки простоти покидає, щоб ходити дорогами темряви,
\end{tcolorbox}
\begin{tcolorbox}
\textsubscript{14} що тішаться, роблячи зло, що радіють крутійствами злого,
\end{tcolorbox}
\begin{tcolorbox}
\textsubscript{15} що стежки їхні круті, і відходять своїми путями,
\end{tcolorbox}
\begin{tcolorbox}
\textsubscript{16} щоб тебе врятувати від блудниці, від чужинки, що мовить м'якенькі слова,
\end{tcolorbox}
\begin{tcolorbox}
\textsubscript{17} що покинула друга юнацтва свого, а про заповіт свого Бога забула,
\end{tcolorbox}
\begin{tcolorbox}
\textsubscript{18} вона бо із домом своїм западеться у смерть, а стежки її до померлих,
\end{tcolorbox}
\begin{tcolorbox}
\textsubscript{19} ніхто, хто входить до неї, не вернеться, і стежки життя не досягне,
\end{tcolorbox}
\begin{tcolorbox}
\textsubscript{20} щоб ходив ти дорогою добрих, і стежки справедливих беріг!
\end{tcolorbox}
\begin{tcolorbox}
\textsubscript{21} Бо замешкають праведні землю, і невинні зостануться в ній,
\end{tcolorbox}
\begin{tcolorbox}
\textsubscript{22} а безбожні з землі будуть вигублені, і повириваються з неї невірні!
\end{tcolorbox}
\subsection{CHAPTER 3}
\begin{tcolorbox}
\textsubscript{1} Сину мій, не забудь ти моєї науки, і нехай мої заповіді стережуть твоє серце,
\end{tcolorbox}
\begin{tcolorbox}
\textsubscript{2} бо примножать для тебе вони довготу твоїх днів, і років життя та спокою!
\end{tcolorbox}
\begin{tcolorbox}
\textsubscript{3} Милість та правда нехай не залишать тебе, прив'яжи їх до шиї своєї, напиши їх на таблиці серця свого,
\end{tcolorbox}
\begin{tcolorbox}
\textsubscript{4} і знайдеш ти ласку та добру премудрість в очах Бога й людини!
\end{tcolorbox}
\begin{tcolorbox}
\textsubscript{5} Надійся на Господа всім своїм серцем, а на розум свій не покладайся!
\end{tcolorbox}
\begin{tcolorbox}
\textsubscript{6} Пізнавай ти Його на всіх дорогах своїх, і Він випростує твої стежки.
\end{tcolorbox}
\begin{tcolorbox}
\textsubscript{7} Не будь мудрий у власних очах, бійся Господа та ухиляйся від злого!
\end{tcolorbox}
\begin{tcolorbox}
\textsubscript{8} Це буде ліком для тіла твого, напоєм для костей твоїх.
\end{tcolorbox}
\begin{tcolorbox}
\textsubscript{9} Шануй Господа із маєтку свого, і з початку всіх плодів своїх,
\end{tcolorbox}
\begin{tcolorbox}
\textsubscript{10} і будуть комори твої переповнені ситістю, а чавила твої будуть переливатись вином молодим!
\end{tcolorbox}
\begin{tcolorbox}
\textsubscript{11} Мій сину, карання Господнього не відкидай, і картання Його не вважай тягарем,
\end{tcolorbox}
\begin{tcolorbox}
\textsubscript{12} бо кого Господь любить, картає того, і кохає, немов батько сина!
\end{tcolorbox}
\begin{tcolorbox}
\textsubscript{13} Блаженна людина, що мудрість знайшла, і людина, що розум одержала,
\end{tcolorbox}
\begin{tcolorbox}
\textsubscript{14} бо ліпше надбання її від надбання срібла, і від щирого золота ліпший прибуток її,
\end{tcolorbox}
\begin{tcolorbox}
\textsubscript{15} дорожча за перли вона, і всіляке жадання твоє не зрівняється з нею.
\end{tcolorbox}
\begin{tcolorbox}
\textsubscript{16} Довгість днів у правиці її, багатство та слава в лівиці її.
\end{tcolorbox}
\begin{tcolorbox}
\textsubscript{17} Дороги її то дороги приємности, всі стежки її мир.
\end{tcolorbox}
\begin{tcolorbox}
\textsubscript{18} Вона дерево життя для тих, хто тримається міцно її, і блаженний, хто держить її!
\end{tcolorbox}
\begin{tcolorbox}
\textsubscript{19} Господь мудрістю землю заклав, небо розумом міцно поставив.
\end{tcolorbox}
\begin{tcolorbox}
\textsubscript{20} Знанням Його порозкривались безодні, і кроплять росою ті хмари.
\end{tcolorbox}
\begin{tcolorbox}
\textsubscript{21} Мій сину, нехай від очей твоїх це не відходить, стережи добрий розум і розважність,
\end{tcolorbox}
\begin{tcolorbox}
\textsubscript{22} і вони будуть життям для твоєї душі, і прикрасою шиї твоєї,
\end{tcolorbox}
\begin{tcolorbox}
\textsubscript{23} Тоді підеш безпечно своєю дорогою, а нога твоя не спотикнеться!
\end{tcolorbox}
\begin{tcolorbox}
\textsubscript{24} Якщо покладешся не будеш боятись, а ляжеш, то буде приємний твій сон.
\end{tcolorbox}
\begin{tcolorbox}
\textsubscript{25} Не будеш боятися наглого страху, ні бурі безбожних, як прийде,
\end{tcolorbox}
\begin{tcolorbox}
\textsubscript{26} бо твоєю надією буде Господь, і Він пильнуватиме ногу твою, щоб вона не зловилась у пастку!
\end{tcolorbox}
\begin{tcolorbox}
\textsubscript{27} Не стримуй добра потребуючому, коли в силі твоєї руки це вчинити,
\end{tcolorbox}
\begin{tcolorbox}
\textsubscript{28} не кажи своїм ближнім: Іди, і знову прийди, а взавтра я дам, коли маєш з собою.
\end{tcolorbox}
\begin{tcolorbox}
\textsubscript{29} Не виорюй лихого на свого ближнього, коли він безпечно з тобою сидить.
\end{tcolorbox}
\begin{tcolorbox}
\textsubscript{30} Не сварися з людиною дармо, якщо злого вона не вчинила тобі.
\end{tcolorbox}
\begin{tcolorbox}
\textsubscript{31} Не заздри насильникові, і ні однієї з доріг його не вибирай,
\end{tcolorbox}
\begin{tcolorbox}
\textsubscript{32} бо бридить Господь крутіями, а з праведними в Нього дружба.
\end{tcolorbox}
\begin{tcolorbox}
\textsubscript{33} Прокляття Господнє на домі безбожного, а мешкання праведних Він благословить,
\end{tcolorbox}
\begin{tcolorbox}
\textsubscript{34} з насмішників Він насміхається, а покірливим милість дає.
\end{tcolorbox}
\begin{tcolorbox}
\textsubscript{35} Мудрі славу вспадковують, а нерозумні носитимуть сором.
\end{tcolorbox}
\subsection{CHAPTER 4}
\begin{tcolorbox}
\textsubscript{1} Послухайте, діти, напучення батькового, і прислухайтеся, щоб навчитися розуму,
\end{tcolorbox}
\begin{tcolorbox}
\textsubscript{2} бо даю я вам добру науку: закона мого не кидайте,
\end{tcolorbox}
\begin{tcolorbox}
\textsubscript{3} бо сином у батька свого я був, пещений й єдиний у неньки своєї.
\end{tcolorbox}
\begin{tcolorbox}
\textsubscript{4} І навчав він мене, і мені говорив: Нехай держиться серце твоє моїх слів, стережи мої заповіді та й живи!
\end{tcolorbox}
\begin{tcolorbox}
\textsubscript{5} Здобудь мудрість, здобудь собі розум, не забудь, і не цурайся слів моїх уст,
\end{tcolorbox}
\begin{tcolorbox}
\textsubscript{6} не кидай її й вона буде тебе стерегти! Кохай ти її й вона буде тебе пильнувати!
\end{tcolorbox}
\begin{tcolorbox}
\textsubscript{7} Початок премудрости мудрість здобудь, а за ввесь свій маєток здобудь собі розуму!
\end{tcolorbox}
\begin{tcolorbox}
\textsubscript{8} Тримай її високо і підійме тебе, ушанує тебе, як її ти пригорнеш:
\end{tcolorbox}
\begin{tcolorbox}
\textsubscript{9} вона дасть голові твоїй гарний вінок, пишну корону тобі подарує!
\end{tcolorbox}
\begin{tcolorbox}
\textsubscript{10} Послухай, мій сину, й бери ти слова мої, і помножаться роки твойого життя,
\end{tcolorbox}
\begin{tcolorbox}
\textsubscript{11} дороги премудрости вчу я тебе, стежками прямими проваджу тебе:
\end{tcolorbox}
\begin{tcolorbox}
\textsubscript{12} коли підеш, то крок твій не буде тісний, а коли побіжиш не спіткнешся!
\end{tcolorbox}
\begin{tcolorbox}
\textsubscript{13} Міцно тримайся напучування, не лишай, його стережи, воно бо життя твоє!
\end{tcolorbox}
\begin{tcolorbox}
\textsubscript{14} На стежку безбожних не йди, і не ходи на дорогу лихих,
\end{tcolorbox}
\begin{tcolorbox}
\textsubscript{15} покинь ти її, не йди нею, усунься від неї й мини,
\end{tcolorbox}
\begin{tcolorbox}
\textsubscript{16} бо вони не заснуть, якщо злого не вчинять, відійметься сон їм, як не зроблять кому, щоб спіткнувся!...
\end{tcolorbox}
\begin{tcolorbox}
\textsubscript{17} Бо вони хліб безбожжя їдять, і вино грабежу попивають.
\end{tcolorbox}
\begin{tcolorbox}
\textsubscript{18} А путь праведних ніби те світло ясне, що світить все більше та більш аж до повного дня!
\end{tcolorbox}
\begin{tcolorbox}
\textsubscript{19} Дорога ж безбожних як темність: не знають, об що спотикнуться...
\end{tcolorbox}
\begin{tcolorbox}
\textsubscript{20} Мій сину, прислухуйся до моїх слів, до речей моїх ухо своє нахили!
\end{tcolorbox}
\begin{tcolorbox}
\textsubscript{21} Нехай не відійдуть вони від очей твоїх, бережи їх в середині серця свого!
\end{tcolorbox}
\begin{tcolorbox}
\textsubscript{22} Бо життя вони тим, хто їх знайде, а для тіла усього його лікування.
\end{tcolorbox}
\begin{tcolorbox}
\textsubscript{23} Над усе, що лише стережеться, серце своє стережи, бо з нього походить життя.
\end{tcolorbox}
\begin{tcolorbox}
\textsubscript{24} Відкинь ти від себе лукавство уст, віддали ти від себе крутійство губ.
\end{tcolorbox}
\begin{tcolorbox}
\textsubscript{25} Нехай дивляться очі твої уперед, а повіки твої нехай перед тобою простують.
\end{tcolorbox}
\begin{tcolorbox}
\textsubscript{26} Стежку ніг своїх вирівняй, і стануть міцні всі дороги твої:
\end{tcolorbox}
\begin{tcolorbox}
\textsubscript{27} не вступайся ні вправо, ні вліво, усунь свою ногу від зла!
\end{tcolorbox}
\subsection{CHAPTER 5}
\begin{tcolorbox}
\textsubscript{1} Мій сину, на мудрість мою уважай, нахили своє ухо до мого розуму,
\end{tcolorbox}
\begin{tcolorbox}
\textsubscript{2} щоб розважність ти міг стерегти, а пізнання хай уста твої стережуть!
\end{tcolorbox}
\begin{tcolorbox}
\textsubscript{3} Бо крапають солодощ губи блудниці, а уста її від оливи масніші,
\end{tcolorbox}
\begin{tcolorbox}
\textsubscript{4} та гіркий їй кінець, мов полин, гострий, як меч обосічний,
\end{tcolorbox}
\begin{tcolorbox}
\textsubscript{5} її ноги до смерти спускаються, шеолу тримаються кроки її!
\end{tcolorbox}
\begin{tcolorbox}
\textsubscript{6} Вона путь життя не урівнює, її стежки непевні, і цього не знає вона.
\end{tcolorbox}
\begin{tcolorbox}
\textsubscript{7} Тож тепер, мої діти, мене ви послухайте, не відходьте від слів моїх уст:
\end{tcolorbox}
\begin{tcolorbox}
\textsubscript{8} віддали ти від неї дорогу свою, і не зближайсь до дверей її дому,
\end{tcolorbox}
\begin{tcolorbox}
\textsubscript{9} щоб слави своєї ти іншим не дав, а роки свої для жорстокого,
\end{tcolorbox}
\begin{tcolorbox}
\textsubscript{10} щоб чужі не наситились сили твоєї й маєтку твого в чужім домі!...
\end{tcolorbox}
\begin{tcolorbox}
\textsubscript{11} І будеш стогнати при своєму кінці, як знеможеться тіло твоє й твої сили,
\end{tcolorbox}
\begin{tcolorbox}
\textsubscript{12} і скажеш: Як ненавидів я те напучування, а картання те серце моє відкидало!
\end{tcolorbox}
\begin{tcolorbox}
\textsubscript{13} І не слухав я голосу своїх учителів, і уха свого не схиляв до наставників...
\end{tcolorbox}
\begin{tcolorbox}
\textsubscript{14} Трохи не був я при кожному злому, в середині збору й громади!...
\end{tcolorbox}
\begin{tcolorbox}
\textsubscript{15} Пий воду з криниці своєї, і текуче з свого колодязя:
\end{tcolorbox}
\begin{tcolorbox}
\textsubscript{16} чи ж мають на вулицю вилиті бути джерела твої, а на площі потоки твоєї води?
\end{tcolorbox}
\begin{tcolorbox}
\textsubscript{17} Нехай вони будуть для тебе, для тебе самого, а не для чужих із тобою!
\end{tcolorbox}
\begin{tcolorbox}
\textsubscript{18} Хай твоє джерело буде благословенне, і радій через жінку твоїх юних літ,
\end{tcolorbox}
\begin{tcolorbox}
\textsubscript{19} вона ланя любовна та серна прекрасна, її перса напоять тебе кожночасно, впивайся ж назавжди коханням її!
\end{tcolorbox}
\begin{tcolorbox}
\textsubscript{20} І нащо, мій сину, ти маєш впиватись блудницею, і нащо ти будеш пригортати груди чужинки?
\end{tcolorbox}
\begin{tcolorbox}
\textsubscript{21} Бож перед очима Господніми всі дороги людини, і стежки її всі Він рівняє:
\end{tcolorbox}
\begin{tcolorbox}
\textsubscript{22} власні провини безбожного схоплять його, і повороззям свого гріха буде зв'язаний він,
\end{tcolorbox}
\begin{tcolorbox}
\textsubscript{23} помиратиме він без напучування, і буде блукати в великій глупоті своїй!...
\end{tcolorbox}
\subsection{CHAPTER 6}
\begin{tcolorbox}
\textsubscript{1} Мій сину, якщо поручився ти за свого ближнього, дав руку свою за чужого,
\end{tcolorbox}
\begin{tcolorbox}
\textsubscript{2} ти попався до пастки з-за слів своїх уст, схоплений ти із-за слів своїх уст!
\end{tcolorbox}
\begin{tcolorbox}
\textsubscript{3} Учини тоді це, сину мій, та рятуйсь, бо впав ти до рук свого ближнього: іди, впади в порох, і на ближніх своїх напирай,
\end{tcolorbox}
\begin{tcolorbox}
\textsubscript{4} не дай сну своїм очам, і дрімання повікам своїм,
\end{tcolorbox}
\begin{tcolorbox}
\textsubscript{5} рятуйся, як серна, з руки, і як птах із руки птахолова!
\end{tcolorbox}
\begin{tcolorbox}
\textsubscript{6} Іди до мурашки, лінюху, поглянь на дороги її й помудрій:
\end{tcolorbox}
\begin{tcolorbox}
\textsubscript{7} нема в неї володаря, ані урядника, ані правителя;
\end{tcolorbox}
\begin{tcolorbox}
\textsubscript{8} вона заготовлює літом свій хліб, збирає в жнива свою їжу.
\end{tcolorbox}
\begin{tcolorbox}
\textsubscript{9} Аж доки, лінюху, ти будеш вилежуватись, коли ти зо сну свого встанеш?
\end{tcolorbox}
\begin{tcolorbox}
\textsubscript{10} Ще трохи поспати, подрімати ще трохи, руки трохи зложити, щоб полежати,
\end{tcolorbox}
\begin{tcolorbox}
\textsubscript{11} і прийде, немов волоцюга, твоя незаможність, і злидні твої, як озброєний муж!...
\end{tcolorbox}
\begin{tcolorbox}
\textsubscript{12} Людина нікчемна, чоловік злочинний, він ходить з лукавими устами,
\end{tcolorbox}
\begin{tcolorbox}
\textsubscript{13} він моргає очима своїми, шургає своїми ногами, знаки подає пальцями своїми,
\end{tcolorbox}
\begin{tcolorbox}
\textsubscript{14} в його серці лукавство виорює зло кожночасно, сварки розсіває,
\end{tcolorbox}
\begin{tcolorbox}
\textsubscript{15} тому нагло приходить погибіль його, буде раптом побитий і ліку нема!
\end{tcolorbox}
\begin{tcolorbox}
\textsubscript{16} Оцих шість ненавидить Господь, а ці сім то гидота душі Його:
\end{tcolorbox}
\begin{tcolorbox}
\textsubscript{17} очі пишні, брехливий язик, і руки, що кров неповинну ллють,
\end{tcolorbox}
\begin{tcolorbox}
\textsubscript{18} серце, що плекає злочинні думки, ноги, що сквапно біжать на лихе,
\end{tcolorbox}
\begin{tcolorbox}
\textsubscript{19} свідок брехливий, що брехні роздмухує, і хто розсіває сварки між братів!
\end{tcolorbox}
\begin{tcolorbox}
\textsubscript{20} Стережи, сину мій, заповідь батька свого, і не відкидай науки матері своєї!
\end{tcolorbox}
\begin{tcolorbox}
\textsubscript{21} Прив'яжи їх на серці своєму назавжди, повісь їх на шиї своїй!
\end{tcolorbox}
\begin{tcolorbox}
\textsubscript{22} Вона буде провадити тебе у ході, стерегтиме тебе, коли будеш лежати, а пробудишся мовити буде до тебе!
\end{tcolorbox}
\begin{tcolorbox}
\textsubscript{23} Бо заповідь Божа світильник, а наука то світло, дорога ж життя то навчальні картання,
\end{tcolorbox}
\begin{tcolorbox}
\textsubscript{24} щоб тебе стерегти від злосливої жінки, від облесливого язика чужинки.
\end{tcolorbox}
\begin{tcolorbox}
\textsubscript{25} Не жадай її вроди у серці своїм, і тебе хай не візьме своїми повіками,
\end{tcolorbox}
\begin{tcolorbox}
\textsubscript{26} бо вартість розпусної жінки то боханець хліба, а жінка заміжня вловлює душу цінну...
\end{tcolorbox}
\begin{tcolorbox}
\textsubscript{27} Чи візьме людина огонь на лоно своє, і одіж її не згорить?
\end{tcolorbox}
\begin{tcolorbox}
\textsubscript{28} Чи буде людина ходити по вугіллю розпаленому, і не попаляться ноги її?
\end{tcolorbox}
\begin{tcolorbox}
\textsubscript{29} Так і той, хто вчащає до жінки свого ближнього: не буде некараним кожен, хто доторкнеться до неї!
\end{tcolorbox}
\begin{tcolorbox}
\textsubscript{30} Не погорджують злодієм, якщо він украде, щоб рятувати життя своє, коли він голодує,
\end{tcolorbox}
\begin{tcolorbox}
\textsubscript{31} та як буде він знайдений, всемеро він відшкодує, віддасть все майно свого дому!
\end{tcolorbox}
\begin{tcolorbox}
\textsubscript{32} Хто чинить перелюб, не має той розуму, він знищує душу свою,
\end{tcolorbox}
\begin{tcolorbox}
\textsubscript{33} побої та сором він знайде, а ганьба його не зітреться,
\end{tcolorbox}
\begin{tcolorbox}
\textsubscript{34} бо заздрощі лютість мужчини, і не змилосердиться він у день помсти:
\end{tcolorbox}
\begin{tcolorbox}
\textsubscript{35} він не зверне уваги на жоден твій викуп, і не схоче, коли ти гостинця прибільшиш!
\end{tcolorbox}
\subsection{CHAPTER 7}
\begin{tcolorbox}
\textsubscript{1} Сину мій, бережи ти слова мої, мої ж заповіді заховай при собі,
\end{tcolorbox}
\begin{tcolorbox}
\textsubscript{2} бережи мої заповіді та й живи, а наука моя немов в очах твоїх та зіниця,
\end{tcolorbox}
\begin{tcolorbox}
\textsubscript{3} прив'яжи їх на пальцях своїх, напиши на таблиці тій серця свого!
\end{tcolorbox}
\begin{tcolorbox}
\textsubscript{4} На мудрість скажи: Ти сестра моя! а розум назви: Мій довірений!
\end{tcolorbox}
\begin{tcolorbox}
\textsubscript{5} щоб тебе стерегти від блудниці, від чужинки, що мовить м'якенькі слова.
\end{tcolorbox}
\begin{tcolorbox}
\textsubscript{6} Бо я визирав був в вікно свого дому, через ґрати мого вікна,
\end{tcolorbox}
\begin{tcolorbox}
\textsubscript{7} і приглядавсь до невіж, розглядався між молоддю. І юнак ось, позбавлений розуму,
\end{tcolorbox}
\begin{tcolorbox}
\textsubscript{8} проходив по ринку при розі його, і ступив по дорозі до дому її,
\end{tcolorbox}
\begin{tcolorbox}
\textsubscript{9} коли вітерець повівав був увечорі дня, у темряві ночі та мороку.
\end{tcolorbox}
\begin{tcolorbox}
\textsubscript{10} Аж ось жінка в убранні блудниці назустріч йому, із серцем підступним,
\end{tcolorbox}
\begin{tcolorbox}
\textsubscript{11} галаслива та непогамована, її ноги у домі своїм не бувають:
\end{tcolorbox}
\begin{tcolorbox}
\textsubscript{12} раз на вулиці, раз на майданах, і при кожному розі чатує вона...
\end{tcolorbox}
\begin{tcolorbox}
\textsubscript{13} І вхопила вона його міцно та й поцілувала його, безсоромним зробила обличчя своє та й сказала йому:
\end{tcolorbox}
\begin{tcolorbox}
\textsubscript{14} У мене тепер мирні жертви, виповнила я сьогодні обіти свої!
\end{tcolorbox}
\begin{tcolorbox}
\textsubscript{15} Тому то я вийшла назустріч тобі, пошукати обличчя твого, і знайшла я тебе!
\end{tcolorbox}
\begin{tcolorbox}
\textsubscript{16} Килимами я вистелила своє ложе, тканинами різних кольорів з єгипетського полотна,
\end{tcolorbox}
\begin{tcolorbox}
\textsubscript{17} постелю свою я посипала миррою, алоєм та цинамоном...
\end{tcolorbox}
\begin{tcolorbox}
\textsubscript{18} Ходи ж, аж до ранку впиватися будем коханням, любов'ю натішимось ми!
\end{tcolorbox}
\begin{tcolorbox}
\textsubscript{19} Бо вдома нема чоловіка, пішов у далеку дорогу:
\end{tcolorbox}
\begin{tcolorbox}
\textsubscript{20} вузлик срібла він узяв в свою руку, хіба на день повні поверне до дому свого...
\end{tcolorbox}
\begin{tcolorbox}
\textsubscript{21} Прихилила його велемовством своїм, облесливістю своїх губ його звабила,
\end{tcolorbox}
\begin{tcolorbox}
\textsubscript{22} він раптом за нею пішов, немов віл, до зарізу проваджений, і немов пес, що ведуть його на ланцюгу до ув'язнення,
\end{tcolorbox}
\begin{tcolorbox}
\textsubscript{23} як той птах, поспішає до сітки, і не знає, що це на життя його пастка...
\end{tcolorbox}
\begin{tcolorbox}
\textsubscript{24} А тепер, мої діти, мене ви послухайте, і на слова моїх уст уважайте:
\end{tcolorbox}
\begin{tcolorbox}
\textsubscript{25} Хай не збочує серце твоє на дороги її, не блукай ти стежками її,
\end{tcolorbox}
\begin{tcolorbox}
\textsubscript{26} бо вона багатьох уже трупами кинула, і численні всі, нею забиті!
\end{tcolorbox}
\begin{tcolorbox}
\textsubscript{27} Її дім до шеолу дороги, що провадять до смертних кімнат...
\end{tcolorbox}
\subsection{CHAPTER 8}
\begin{tcolorbox}
\textsubscript{1} Чи ж мудрість не кличе, і не подає свого голосу розум?
\end{tcolorbox}
\begin{tcolorbox}
\textsubscript{2} На верхів'ях холмів, при дорозі та на перехрестях стоїть он вона!
\end{tcolorbox}
\begin{tcolorbox}
\textsubscript{3} При брамах, при вході до міста, де входиться в двері, там голосно кличе вона:
\end{tcolorbox}
\begin{tcolorbox}
\textsubscript{4} До вас, мужі, я кличу, а мій голос до людських синів:
\end{tcolorbox}
\begin{tcolorbox}
\textsubscript{5} Зрозумійте но, неуки, мудрість, зрозумійте ви розум, безглузді!
\end{tcolorbox}
\begin{tcolorbox}
\textsubscript{6} Послухайте, я бо шляхетне кажу, і відкриття моїх губ то простота.
\end{tcolorbox}
\begin{tcolorbox}
\textsubscript{7} Бо правду говорять уста мої, а лукавство гидота для губ моїх.
\end{tcolorbox}
\begin{tcolorbox}
\textsubscript{8} Всі слова моїх уст справедливі, нема в них крутійства й лукавства.
\end{tcolorbox}
\begin{tcolorbox}
\textsubscript{9} Усі вони прості, хто їх розуміє, і щирі для тих, хто знаходить знання.
\end{tcolorbox}
\begin{tcolorbox}
\textsubscript{10} Візьміть ви картання моє, а не срібло, і знання, добірніше від щирого золота:
\end{tcolorbox}
\begin{tcolorbox}
\textsubscript{11} ліпша бо мудрість за перли, і не рівняються їй всі клейноди!
\end{tcolorbox}
\begin{tcolorbox}
\textsubscript{12} Я, мудрість, живу разом з розумом, і знаходжу пізнання розважне.
\end{tcolorbox}
\begin{tcolorbox}
\textsubscript{13} Страх Господній лихе все ненавидіти: я ненавиджу пиху та гордість, і дорогу лиху та лукаві уста!
\end{tcolorbox}
\begin{tcolorbox}
\textsubscript{14} В мене рада й оглядність, я розум, і сила у мене.
\end{tcolorbox}
\begin{tcolorbox}
\textsubscript{15} Мною царюють царі, а законодавці права справедливі встановлюють.
\end{tcolorbox}
\begin{tcolorbox}
\textsubscript{16} Мною правлять владики й вельможні, всі праведні судді.
\end{tcolorbox}
\begin{tcolorbox}
\textsubscript{17} Я кохаю всіх тих, хто кохає мене, хто ж шукає мене мене знайде!
\end{tcolorbox}
\begin{tcolorbox}
\textsubscript{18} Зо мною багатство та слава, тривалий маєток та правда:
\end{tcolorbox}
\begin{tcolorbox}
\textsubscript{19} ліпший плід мій від щирого золота й золота чистого, а прибуток мій ліпший за срібло добірне!
\end{tcolorbox}
\begin{tcolorbox}
\textsubscript{20} Путтю праведною я ходжу, поміж правних стежок,
\end{tcolorbox}
\begin{tcolorbox}
\textsubscript{21} щоб дати багатство в спадщину для тих, хто кохає мене, і я понаповнюю їхні скарбниці!
\end{tcolorbox}
\begin{tcolorbox}
\textsubscript{22} Господь мене мав на початку Своєї дороги, перше чинів Своїх, спервовіку,
\end{tcolorbox}
\begin{tcolorbox}
\textsubscript{23} відвіку була я встановлена, від початку, від правіку землі.
\end{tcolorbox}
\begin{tcolorbox}
\textsubscript{24} Народжена я, як безодень іще не було, коли не було ще джерел, водою обтяжених.
\end{tcolorbox}
\begin{tcolorbox}
\textsubscript{25} Народжена я, поки гори поставлені ще не були, давніше за пагірки,
\end{tcolorbox}
\begin{tcolorbox}
\textsubscript{26} коли ще землі не вчинив Він, ні піль, ні початкового пороху всесвіту.
\end{tcolorbox}
\begin{tcolorbox}
\textsubscript{27} Коли приправляв небеса я була там, коли круга вставляв на поверхні безодні,
\end{tcolorbox}
\begin{tcolorbox}
\textsubscript{28} коли хмари уміцнював Він нагорі, як джерела безодні зміцняв,
\end{tcolorbox}
\begin{tcolorbox}
\textsubscript{29} коли клав Він для моря устава його, щоб його берегів вода не переходила, коли ставив основи землі,
\end{tcolorbox}
\begin{tcolorbox}
\textsubscript{30} то я майстром у Нього була, і була я веселощами день-у-день, радіючи перед обличчям Його кожночасно,
\end{tcolorbox}
\begin{tcolorbox}
\textsubscript{31} радіючи на земнім крузі Його, а забава моя із синами людськими!
\end{tcolorbox}
\begin{tcolorbox}
\textsubscript{32} Тепер же, послухайте, діти, мене, і блаженні, хто буде дороги мої стерегти!
\end{tcolorbox}
\begin{tcolorbox}
\textsubscript{33} Навчання послухайте й мудрими станьте, і не відступайте від нього!
\end{tcolorbox}
\begin{tcolorbox}
\textsubscript{34} Блаженна людина, яка мене слухає, щоб пильнувати при дверях моїх день-у-день, щоб одвірки мої берегти!
\end{tcolorbox}
\begin{tcolorbox}
\textsubscript{35} Хто бо знаходить мене, той знаходить життя, і одержує милість від Господа.
\end{tcolorbox}
\begin{tcolorbox}
\textsubscript{36} А хто проти мене грішить, ограбовує душу свою; всі, хто мене ненавидить, ті смерть покохали!
\end{tcolorbox}
\subsection{CHAPTER 9}
\begin{tcolorbox}
\textsubscript{1} Мудрість свій дім збудувала, сім стовпів своїх витесала.
\end{tcolorbox}
\begin{tcolorbox}
\textsubscript{2} Зарізала те, що було на заріз, змішала вино своє, і трапезу свою приготовила.
\end{tcolorbox}
\begin{tcolorbox}
\textsubscript{3} Дівчат своїх вислала, і кличе вона на висотах міських:
\end{tcolorbox}
\begin{tcolorbox}
\textsubscript{4} Хто бідний на розум, хай прийде сюди, а хто нерозумний, говорить йому:
\end{tcolorbox}
\begin{tcolorbox}
\textsubscript{5} Ходіть, споживайте із хліба мого, та пийте з вина, що його я змішала!
\end{tcolorbox}
\begin{tcolorbox}
\textsubscript{6} Покиньте глупоту і будете жити, і ходіте дорогою розуму!
\end{tcolorbox}
\begin{tcolorbox}
\textsubscript{7} Хто картає насмішника, той собі ганьбу бере, хто ж безбожникові виговорює, сором собі набуває.
\end{tcolorbox}
\begin{tcolorbox}
\textsubscript{8} Не дорікай пересмішникові, щоб тебе не зненавидів він, викартай мудрого й він покохає тебе.
\end{tcolorbox}
\begin{tcolorbox}
\textsubscript{9} Дай мудрому й він помудріє іще, навчи праведного і прибільшить він мудрости!
\end{tcolorbox}
\begin{tcolorbox}
\textsubscript{10} Страх Господній початок премудрости, а пізнання Святого це розум,
\end{tcolorbox}
\begin{tcolorbox}
\textsubscript{11} бо мною помножаться дні твої, і додадуть тобі років життя.
\end{tcolorbox}
\begin{tcolorbox}
\textsubscript{12} Якщо ти змудрів то для себе змудрів, а як станеш насмішником, сам понесеш!
\end{tcolorbox}
\begin{tcolorbox}
\textsubscript{13} Жінка безглузда криклива, нерозумна, і нічого не знає!
\end{tcolorbox}
\begin{tcolorbox}
\textsubscript{14} Сідає вона на сидінні при вході до дому свого, на високостях міста,
\end{tcolorbox}
\begin{tcolorbox}
\textsubscript{15} щоб кликати тих, хто дорогою йде, хто путтю своєю простує:
\end{tcolorbox}
\begin{tcolorbox}
\textsubscript{16} Хто бідний на розум, хай прийде сюди, а хто нерозумний, то каже йому:
\end{tcolorbox}
\begin{tcolorbox}
\textsubscript{17} Вода крадена солодка, і приємний прихований хліб...
\end{tcolorbox}
\begin{tcolorbox}
\textsubscript{18} І не відає він, що самі там мерці, у глибинах шеолу запрошені нею!...
\end{tcolorbox}
\subsection{CHAPTER 10}
\begin{tcolorbox}
\textsubscript{1} Син мудрий потіха для батька, а син нерозумний то смуток для неньки його.
\end{tcolorbox}
\begin{tcolorbox}
\textsubscript{2} Не поможуть неправедні скарби, а справедливість від смерти визволює.
\end{tcolorbox}
\begin{tcolorbox}
\textsubscript{3} Не допустить Господь голодувати душу праведного, а набуток безбожників згине.
\end{tcolorbox}
\begin{tcolorbox}
\textsubscript{4} Ледача рука до убозтва веде, рука ж роботяща збагачує.
\end{tcolorbox}
\begin{tcolorbox}
\textsubscript{5} Хто літом збирає син мудрий, хто ж дрімає в жнива син безпутній.
\end{tcolorbox}
\begin{tcolorbox}
\textsubscript{6} Благословенства на голову праведного, а уста безбожним прикриє насильство.
\end{tcolorbox}
\begin{tcolorbox}
\textsubscript{7} Пам'ять про праведного на благословення, а ймення безбожних загине.
\end{tcolorbox}
\begin{tcolorbox}
\textsubscript{8} Заповіді мудросердий приймає, але дурногубий впаде.
\end{tcolorbox}
\begin{tcolorbox}
\textsubscript{9} Хто в невинності ходить, той ходить безпечно, а хто кривить дороги свої, буде виявлений.
\end{tcolorbox}
\begin{tcolorbox}
\textsubscript{10} Хто оком моргає, той смуток дає, але дурногубий впаде.
\end{tcolorbox}
\begin{tcolorbox}
\textsubscript{11} Уста праведного то джерело життя, а уста безбожним прикриє насильство.
\end{tcolorbox}
\begin{tcolorbox}
\textsubscript{12} Ненависть побуджує сварки, а любов покриває всі вини.
\end{tcolorbox}
\begin{tcolorbox}
\textsubscript{13} В устах розумного мудрість знаходиться, а різка на спину безтямного.
\end{tcolorbox}
\begin{tcolorbox}
\textsubscript{14} Приховують мудрі знання, а уста нерозумного близькі до загибелі.
\end{tcolorbox}
\begin{tcolorbox}
\textsubscript{15} Маєток багатого місто твердинне його, погибіль убогих їхні злидні.
\end{tcolorbox}
\begin{tcolorbox}
\textsubscript{16} Дорібок праведного на життя, прибуток безбожного в гріх.
\end{tcolorbox}
\begin{tcolorbox}
\textsubscript{17} Хто напучування стереже той на стежці життя, а хто нехтує картання, той блудить.
\end{tcolorbox}
\begin{tcolorbox}
\textsubscript{18} Хто ненависть ховає, в того губи брехливі, а хто наклепи ширить, той дурноверхий.
\end{tcolorbox}
\begin{tcolorbox}
\textsubscript{19} Не бракує гріха в многомовності, а хто стримує губи свої, той розумний.
\end{tcolorbox}
\begin{tcolorbox}
\textsubscript{20} Язик праведного то добірне срібло, а розум безбожних мізерний.
\end{tcolorbox}
\begin{tcolorbox}
\textsubscript{21} Пасуть багатьох губи праведного, безглузді ж умирають з нерозуму.
\end{tcolorbox}
\begin{tcolorbox}
\textsubscript{22} Благословення Господнє воно збагачає, і смутку воно не приносить з собою.
\end{tcolorbox}
\begin{tcolorbox}
\textsubscript{23} Нешляхетне робити забава невігласа, а мудрість людині розумній.
\end{tcolorbox}
\begin{tcolorbox}
\textsubscript{24} Чого нечестивий боїться, те прийде на нього, а прагнення праведних сповняться.
\end{tcolorbox}
\begin{tcolorbox}
\textsubscript{25} Як буря, яка пронесеться, то й гине безбожний, а праведний має довічну основу.
\end{tcolorbox}
\begin{tcolorbox}
\textsubscript{26} Як оцет зубам, і як дим для очей, так лінивий для тих, хто його посилає.
\end{tcolorbox}
\begin{tcolorbox}
\textsubscript{27} Страх Господній примножує днів, а роки безбожних вкоротяться.
\end{tcolorbox}
\begin{tcolorbox}
\textsubscript{28} Сподівання для праведних радість, а надія безбожних загине.
\end{tcolorbox}
\begin{tcolorbox}
\textsubscript{29} Дорога Господня твердиня невинним, а загибіль злочинцям.
\end{tcolorbox}
\begin{tcolorbox}
\textsubscript{30} Повік праведний не захитається, а безбожники не поживуть на землі.
\end{tcolorbox}
\begin{tcolorbox}
\textsubscript{31} Уста праведного дають мудрість, а лукавий язик буде втятий.
\end{tcolorbox}
\begin{tcolorbox}
\textsubscript{32} Уста праведного уподобання знають, а уста безбожних лукавство.
\end{tcolorbox}
\subsection{CHAPTER 11}
\begin{tcolorbox}
\textsubscript{1} Обманливі шальки огида для Господа, а повна вага це Його уподоба.
\end{tcolorbox}
\begin{tcolorbox}
\textsubscript{2} Прийде пишність, та прийде і ганьба, а з сумирними мудрість.
\end{tcolorbox}
\begin{tcolorbox}
\textsubscript{3} Невинність простосердих веде їх, а лукавство зрадливих їх вигубить.
\end{tcolorbox}
\begin{tcolorbox}
\textsubscript{4} Не поможе багатство в день гніву, а справедливість від смерти визволює.
\end{tcolorbox}
\begin{tcolorbox}
\textsubscript{5} Справедливість невинного дорогу йому випростовує, безбожний же падає через безбожність свою.
\end{tcolorbox}
\begin{tcolorbox}
\textsubscript{6} Справедливість прямих їх рятує, а зрадливі захоплені будуть своєю захланністю.
\end{tcolorbox}
\begin{tcolorbox}
\textsubscript{7} При смерті людини безбожної гине надія, зникає чекання людини нікчемної.
\end{tcolorbox}
\begin{tcolorbox}
\textsubscript{8} Виривається праведний з утиску, і замість нього безбожний іде.
\end{tcolorbox}
\begin{tcolorbox}
\textsubscript{9} Свого ближнього нищить лукавий устами, а знанням визволяються праведні.
\end{tcolorbox}
\begin{tcolorbox}
\textsubscript{10} Добром праведних місто радіє, а як гинуть безбожні співає.
\end{tcolorbox}
\begin{tcolorbox}
\textsubscript{11} Благословенням чесних підноситься місто, а устами безбожних руйнується.
\end{tcolorbox}
\begin{tcolorbox}
\textsubscript{12} Хто погорджує ближнім своїм, той позбавлений розуму, а розумна людина мовчить.
\end{tcolorbox}
\begin{tcolorbox}
\textsubscript{13} Виявляє обмовник таємне, вірнодухий же справу ховає.
\end{tcolorbox}
\begin{tcolorbox}
\textsubscript{14} Народ падає з браку розумного проводу, при численності ж радників спасіння буває.
\end{tcolorbox}
\begin{tcolorbox}
\textsubscript{15} Зле робить, як хто за чужого поручується, хто ж поруку ненавидить, той безпечний.
\end{tcolorbox}
\begin{tcolorbox}
\textsubscript{16} Жінка чеснотна осягує слави, і пильні багатства здобудуть.
\end{tcolorbox}
\begin{tcolorbox}
\textsubscript{17} Людина ласкава душі своїй чинить добро, а жорстока замучує тіло своє.
\end{tcolorbox}
\begin{tcolorbox}
\textsubscript{18} Чинить діло безвартне безбожний, хто ж праведність сіє заплату правдиву одержує.
\end{tcolorbox}
\begin{tcolorbox}
\textsubscript{19} Отак праведність є на життя, хто ж женеться за злом, той до смерти зближається.
\end{tcolorbox}
\begin{tcolorbox}
\textsubscript{20} Серцем лукаві огида для Господа, а хто в неповинності ходить Його уподоба.
\end{tcolorbox}
\begin{tcolorbox}
\textsubscript{21} Ручаюсь: не буде невинним лихий, а нащадок правдивих захований буде.
\end{tcolorbox}
\begin{tcolorbox}
\textsubscript{22} Золотая сережка в свині на ніздрі це жінка гарна, позбавлена розуму.
\end{tcolorbox}
\begin{tcolorbox}
\textsubscript{23} Жадання у праведних тільки добро, надія безбожних то гнів.
\end{tcolorbox}
\begin{tcolorbox}
\textsubscript{24} Дехто щедро дає, та ще додається йому, а дехто ховає над міру, та тільки бідніє.
\end{tcolorbox}
\begin{tcolorbox}
\textsubscript{25} Душа, яка благословляє, насичена буде, а хто поїть інших, напоєний буде і він.
\end{tcolorbox}
\begin{tcolorbox}
\textsubscript{26} Хто задержує збіжжя, того проклинає народ, хто ж поживу випродує, тому благословення на голову.
\end{tcolorbox}
\begin{tcolorbox}
\textsubscript{27} Хто прагне добра, той шукає вподобання, хто ж лихого жадає, то й прийде на нього воно.
\end{tcolorbox}
\begin{tcolorbox}
\textsubscript{28} Хто надію кладе на багатство своє, той впаде, а праведники зеленіють, як листя.
\end{tcolorbox}
\begin{tcolorbox}
\textsubscript{29} Хто неряд уносить до дому свого, той вітер посяде, а дурноголовий розумному стане рабом.
\end{tcolorbox}
\begin{tcolorbox}
\textsubscript{30} Плід праведного дерево життя, і мудрий життя набуває.
\end{tcolorbox}
\begin{tcolorbox}
\textsubscript{31} Коли праведний ось надолужується на землі, то тим більше безбожний та грішний!
\end{tcolorbox}
\subsection{CHAPTER 12}
\begin{tcolorbox}
\textsubscript{1} Хто любить навчання, той любить пізнання, а хто докір ненавидить, той нерозумний.
\end{tcolorbox}
\begin{tcolorbox}
\textsubscript{2} Добрий від Господа має вподобання, а людину злих замірів осудить Господь.
\end{tcolorbox}
\begin{tcolorbox}
\textsubscript{3} Не зміцниться людина безбожністю, корінь же праведних не захитається.
\end{tcolorbox}
\begin{tcolorbox}
\textsubscript{4} Жінка чеснотна корона для чоловіка свого, а засоромлююча мов та гниль в його костях.
\end{tcolorbox}
\begin{tcolorbox}
\textsubscript{5} Думки праведних право, підступні заміри безбожних омана.
\end{tcolorbox}
\begin{tcolorbox}
\textsubscript{6} Безбожних слова чатування на кров, а уста невинних урятовують їх.
\end{tcolorbox}
\begin{tcolorbox}
\textsubscript{7} Перевернути безбожних і вже їх нема, а дім праведних буде стояти.
\end{tcolorbox}
\begin{tcolorbox}
\textsubscript{8} Хвалять людину за розум її, а кривосердий стає на погорду.
\end{tcolorbox}
\begin{tcolorbox}
\textsubscript{9} Ліпше простий, але роботящий на себе, від того, хто поважним себе видає, та хліба позбавлений.
\end{tcolorbox}
\begin{tcolorbox}
\textsubscript{10} Піклується праведний життям худоби своєї, а серце безбожних жорстоке.
\end{tcolorbox}
\begin{tcolorbox}
\textsubscript{11} Хто оброблює землю свою, той хлібом насичується, хто ж за марницею гониться, той позбавлений розуму.
\end{tcolorbox}
\begin{tcolorbox}
\textsubscript{12} Безбожний жадає ловити у сітку лихих, а в праведних корень приносить плоди.
\end{tcolorbox}
\begin{tcolorbox}
\textsubscript{13} Пастка злого в гріху його уст, а праведний з утиску вийде.
\end{tcolorbox}
\begin{tcolorbox}
\textsubscript{14} Людина насичується добром з плоду уст, і зроблене рук чоловіка до нього впаде.
\end{tcolorbox}
\begin{tcolorbox}
\textsubscript{15} Дорога безумця пряма в його очах, а мудрий послухає ради.
\end{tcolorbox}
\begin{tcolorbox}
\textsubscript{16} Нерозумного гнів пізнається відразу, розумний же мовчки ховає зневагу.
\end{tcolorbox}
\begin{tcolorbox}
\textsubscript{17} Хто правду говорить, той виявлює праведність, а свідок брехливий оману.
\end{tcolorbox}
\begin{tcolorbox}
\textsubscript{18} Дехто говорить, мов коле мечем, язик же премудрих то ліки.
\end{tcolorbox}
\begin{tcolorbox}
\textsubscript{19} Уста правдиві стоятимуть вічно, а брехливий язик лиш на хвилю.
\end{tcolorbox}
\begin{tcolorbox}
\textsubscript{20} В серці тих, хто зло оре, омана, а радість у тих, хто дораджує мир.
\end{tcolorbox}
\begin{tcolorbox}
\textsubscript{21} Жодна кривда не трапиться праведному, а безбожні наповняться лихом.
\end{tcolorbox}
\begin{tcolorbox}
\textsubscript{22} Уста брехливі огида у Господа, а чинячі правду Його уподоба.
\end{tcolorbox}
\begin{tcolorbox}
\textsubscript{23} Приховує мудра людина знання, а серце безумних глупоту викликує.
\end{tcolorbox}
\begin{tcolorbox}
\textsubscript{24} Роботяща рука пануватиме, а лінива даниною стане.
\end{tcolorbox}
\begin{tcolorbox}
\textsubscript{25} Туга на серці людини чавить її, добре ж слово її веселить.
\end{tcolorbox}
\begin{tcolorbox}
\textsubscript{26} Праведний вивідає свою путь, а дорога безбожних зведе їх самих.
\end{tcolorbox}
\begin{tcolorbox}
\textsubscript{27} Не буде ледачий пекти свого полову, а людина трудяща набуде маєток цінний.
\end{tcolorbox}
\begin{tcolorbox}
\textsubscript{28} В путі праведности є життя, і на стежці її нема смерти.
\end{tcolorbox}
\subsection{CHAPTER 13}
\begin{tcolorbox}
\textsubscript{1} Син мудрий приймає картання від батька, а насмішник докору не слухає.
\end{tcolorbox}
\begin{tcolorbox}
\textsubscript{2} З плоду уст чоловік споживає добро, а жадоба зрадливих насильство.
\end{tcolorbox}
\begin{tcolorbox}
\textsubscript{3} Хто уста свої стереже, той душу свою береже, а хто губи свої розпускає, на того погибіль.
\end{tcolorbox}
\begin{tcolorbox}
\textsubscript{4} Пожадає душа лінюха, та даремно, душа ж роботящих насититься.
\end{tcolorbox}
\begin{tcolorbox}
\textsubscript{5} Ненавидить праведний слово брехливе, безбожний же чинить лихе, і себе засоромлює.
\end{tcolorbox}
\begin{tcolorbox}
\textsubscript{6} Праведність оберігає невинного на дорозі його, а безбожність погублює грішника.
\end{tcolorbox}
\begin{tcolorbox}
\textsubscript{7} Дехто вдає багача, хоч нічого не має, а дехто вдає бідака, хоч маєток великий у нього.
\end{tcolorbox}
\begin{tcolorbox}
\textsubscript{8} Викуп за душу людини багатство її, а вбогий й докору не чує.
\end{tcolorbox}
\begin{tcolorbox}
\textsubscript{9} Світло праведних весело світить, а світильник безбожних погасне.
\end{tcolorbox}
\begin{tcolorbox}
\textsubscript{10} Тільки сварка пихою зчиняється, а мудрість із тими, хто радиться.
\end{tcolorbox}
\begin{tcolorbox}
\textsubscript{11} Багатство, заскоро здобуте, поменшується, хто ж збирає помалу примножує.
\end{tcolorbox}
\begin{tcolorbox}
\textsubscript{12} Задовга надія недуга для серця, а бажання, що сповнюється, це дерево життя.
\end{tcolorbox}
\begin{tcolorbox}
\textsubscript{13} Хто погорджує словом Господнім, той шкодить собі, хто ж страх має до заповіді, тому надолужиться.
\end{tcolorbox}
\begin{tcolorbox}
\textsubscript{14} Наука премудрого криниця життя, щоб віддалитися від пасток смерти.
\end{tcolorbox}
\begin{tcolorbox}
\textsubscript{15} Добрий розум приносить приємність, а дорога зрадливих погуба для них.
\end{tcolorbox}
\begin{tcolorbox}
\textsubscript{16} Кожен розумний за мудрістю робить, а безумний глупоту показує.
\end{tcolorbox}
\begin{tcolorbox}
\textsubscript{17} Безбожний посол у нещастя впаде, а вірний посол немов лік.
\end{tcolorbox}
\begin{tcolorbox}
\textsubscript{18} Хто ламає поуку убозтво та ганьба тому, а хто береже осторогу шанований він.
\end{tcolorbox}
\begin{tcolorbox}
\textsubscript{19} Виконане побажання приємне душі, а вступитись від зла то огида безумним.
\end{tcolorbox}
\begin{tcolorbox}
\textsubscript{20} Хто з мудрими ходить, той мудрим стає, а хто товаришує з безумним, той лиха набуде.
\end{tcolorbox}
\begin{tcolorbox}
\textsubscript{21} Грішників зло доганяє, а праведним Бог надолужить добром.
\end{tcolorbox}
\begin{tcolorbox}
\textsubscript{22} Добрий лишає спадок і онукам, маєток же грішника схований буде для праведного.
\end{tcolorbox}
\begin{tcolorbox}
\textsubscript{23} Убогому буде багато поживи і з поля невправного, та деякі гинуть з безправ'я.
\end{tcolorbox}
\begin{tcolorbox}
\textsubscript{24} Хто стримує різку свою, той ненавидить сина свого, хто ж кохає його, той шукає для нього картання.
\end{tcolorbox}
\begin{tcolorbox}
\textsubscript{25} Праведний їсть, скільки схоче душа, живіт же безбожників завсіди брак відчуває.
\end{tcolorbox}
\subsection{CHAPTER 14}
\begin{tcolorbox}
\textsubscript{1} Мудра жінка будує свій дім, а безумна своєю рукою руйнує його.
\end{tcolorbox}
\begin{tcolorbox}
\textsubscript{2} Хто ходить в простоті своїй, боїться той Господа, а в кого дороги криві, той погорджує Ним.
\end{tcolorbox}
\begin{tcolorbox}
\textsubscript{3} На устах безумця галузка пихи, а губи премудрих їх стережуть.
\end{tcolorbox}
\begin{tcolorbox}
\textsubscript{4} Де немає биків, там ясла порожні, а щедрість врожаю у силі вола.
\end{tcolorbox}
\begin{tcolorbox}
\textsubscript{5} Свідок правдивий не лже, а свідок брехливий говорить неправду.
\end{tcolorbox}
\begin{tcolorbox}
\textsubscript{6} Насмішник шукає премудрости, та надаремно, пізнання легке для розумного.
\end{tcolorbox}
\begin{tcolorbox}
\textsubscript{7} Ходи здалека від людини безумної, і від того, в кого мудрих уст ти не бачив.
\end{tcolorbox}
\begin{tcolorbox}
\textsubscript{8} Мудрість розумного то розуміння дороги своєї, а глупота дурних то омана.
\end{tcolorbox}
\begin{tcolorbox}
\textsubscript{9} Нерозумні сміються з гріха, а між праведними уподобання.
\end{tcolorbox}
\begin{tcolorbox}
\textsubscript{10} Серце знає гіркоту своєї душі, і в радість його не втручається інший.
\end{tcolorbox}
\begin{tcolorbox}
\textsubscript{11} Буде вигублений дім безбожних, а намет безневинних розквітне.
\end{tcolorbox}
\begin{tcolorbox}
\textsubscript{12} Буває, дорога людині здається простою, та кінець її стежка до смерти.
\end{tcolorbox}
\begin{tcolorbox}
\textsubscript{13} Також іноді і від сміху болить серце, і закінчення радости смуток.
\end{tcolorbox}
\begin{tcolorbox}
\textsubscript{14} Хто підступного серця, насититься той із доріг своїх, а добра людина із чинів своїх.
\end{tcolorbox}
\begin{tcolorbox}
\textsubscript{15} Вірить безглуздий в кожнісіньке слово, а мудрий зважає на кроки свої.
\end{tcolorbox}
\begin{tcolorbox}
\textsubscript{16} Мудрий боїться й від злого вступає, нерозумний же гнівається та сміливий.
\end{tcolorbox}
\begin{tcolorbox}
\textsubscript{17} Скорий на гнів учиняє глупоту, а людина лукава зненавиджена.
\end{tcolorbox}
\begin{tcolorbox}
\textsubscript{18} Нерозумні глупоту вспадковують, а мудрі знанням коронуються.
\end{tcolorbox}
\begin{tcolorbox}
\textsubscript{19} Поклоняться злі перед добрими, а безбожники при брамах праведного.
\end{tcolorbox}
\begin{tcolorbox}
\textsubscript{20} Убогий зненавиджений навіть ближнім своїм, а в багатого друзі численні.
\end{tcolorbox}
\begin{tcolorbox}
\textsubscript{21} Хто погорджує ближнім своїм, той грішить, а ласкавий до вбогих блаженний.
\end{tcolorbox}
\begin{tcolorbox}
\textsubscript{22} Чи ж не блудять, хто оре лихе? А милість та правда для тих, хто оре добро.
\end{tcolorbox}
\begin{tcolorbox}
\textsubscript{23} Кожна праця приносить достаток, але праця уст в недостаток веде.
\end{tcolorbox}
\begin{tcolorbox}
\textsubscript{24} Корона премудрих їхня мудрість, а вінець нерозумних глупота.
\end{tcolorbox}
\begin{tcolorbox}
\textsubscript{25} Свідок правдивий визволює душі, а свідок обманливий брехні торочить.
\end{tcolorbox}
\begin{tcolorbox}
\textsubscript{26} У Господньому страхові сильна надія, і Він пристановище дітям Своїм.
\end{tcolorbox}
\begin{tcolorbox}
\textsubscript{27} Страх Господній криниця життя, щоб віддалятися від пасток смерти.
\end{tcolorbox}
\begin{tcolorbox}
\textsubscript{28} У численності люду величність царя, а в браку народу погибіль володаря.
\end{tcolorbox}
\begin{tcolorbox}
\textsubscript{29} Терпеливий у гніві багаторозумний, а гнівливий вчиняє глупоту.
\end{tcolorbox}
\begin{tcolorbox}
\textsubscript{30} Лагідне серце життя то для тіла, а заздрість гнилизна костей.
\end{tcolorbox}
\begin{tcolorbox}
\textsubscript{31} Хто тисне нужденного, той ображає свого Творця, а хто милостивий до вбогого, той поважає Його.
\end{tcolorbox}
\begin{tcolorbox}
\textsubscript{32} Безбожний у зло своє падає, а праведний повний надії й при смерті своїй.
\end{tcolorbox}
\begin{tcolorbox}
\textsubscript{33} Мудрість має спочинок у серці розумного, а що в нутрі безумних, те виявиться.
\end{tcolorbox}
\begin{tcolorbox}
\textsubscript{34} Праведність люд підіймає, а беззаконня то сором народів.
\end{tcolorbox}
\begin{tcolorbox}
\textsubscript{35} Ласка царева рабові розумному, гнів же його проти того, хто соромить його.
\end{tcolorbox}
\subsection{CHAPTER 15}
\begin{tcolorbox}
\textsubscript{1} Лагідна відповідь гнів відвертає, а слово вразливе гнів підіймає.
\end{tcolorbox}
\begin{tcolorbox}
\textsubscript{2} Язик мудрих то добре знання, а уста нерозумних глупоту висловлюють.
\end{tcolorbox}
\begin{tcolorbox}
\textsubscript{3} Очі Господні на кожному місці, позирають на злих та на добрих.
\end{tcolorbox}
\begin{tcolorbox}
\textsubscript{4} Язик лагідний то дерево життя, а лукавство його заламання на дусі.
\end{tcolorbox}
\begin{tcolorbox}
\textsubscript{5} Зневажає безумний напучення батькове, а хто береже осторогу, стає розумніший.
\end{tcolorbox}
\begin{tcolorbox}
\textsubscript{6} Дім праведного скарб великий, а в плоді безбожного безлад.
\end{tcolorbox}
\begin{tcolorbox}
\textsubscript{7} Уста мудрих знання розсівають, а серце безглуздих не так.
\end{tcolorbox}
\begin{tcolorbox}
\textsubscript{8} Жертва безбожних огида для Господа, а молитва невинних Його уподоба.
\end{tcolorbox}
\begin{tcolorbox}
\textsubscript{9} Господеві огида дорога безбожного, а того, хто женеться за праведністю, Він кохає.
\end{tcolorbox}
\begin{tcolorbox}
\textsubscript{10} Люта кара на того, хто путь оставляє, а хто осторогу ненавидить, той умирає.
\end{tcolorbox}
\begin{tcolorbox}
\textsubscript{11} Шеол й Аваддон перед Господом, тим більше серця синів людських!
\end{tcolorbox}
\begin{tcolorbox}
\textsubscript{12} Насмішник не любить картання собі, він до мудрих не піде.
\end{tcolorbox}
\begin{tcolorbox}
\textsubscript{13} Радісне серце лице веселить, а при смутку сердечному дух приголомшений.
\end{tcolorbox}
\begin{tcolorbox}
\textsubscript{14} Серце розумне шукає знання, а уста безумних глупоту пасуть.
\end{tcolorbox}
\begin{tcolorbox}
\textsubscript{15} Нужденному всі дні лихі, кому ж добре на серці, у того гостина постійно.
\end{tcolorbox}
\begin{tcolorbox}
\textsubscript{16} Ліпше мале у Господньому страху, ані ж скарб великий, та тривога при тому.
\end{tcolorbox}
\begin{tcolorbox}
\textsubscript{17} Ліпша пожива яринна, і при тому любов, аніж тучний віл, та ненависть при тому.
\end{tcolorbox}
\begin{tcolorbox}
\textsubscript{18} Гнівлива людина роздражнює сварку, терпелива ж у гніві вспокоює заколот.
\end{tcolorbox}
\begin{tcolorbox}
\textsubscript{19} Дорога лінивого то терновиння, а путь щирих дорога гладка.
\end{tcolorbox}
\begin{tcolorbox}
\textsubscript{20} Мудрий син тішить батька свого, а людина безумна погорджує матір'ю своєю.
\end{tcolorbox}
\begin{tcolorbox}
\textsubscript{21} Глупота то радість для нерозумного, а людина розумна дорогою простою ходить.
\end{tcolorbox}
\begin{tcolorbox}
\textsubscript{22} Ламаються задуми з браку поради, при численності ж радників сповняться.
\end{tcolorbox}
\begin{tcolorbox}
\textsubscript{23} Радість людині у відповіді його уст, а слово на часі своєму яке воно добре!
\end{tcolorbox}
\begin{tcolorbox}
\textsubscript{24} Путь життя для премудрого угору, щоб віддалюватись від шеолу внизу.
\end{tcolorbox}
\begin{tcolorbox}
\textsubscript{25} Дім пишних руйнує Господь, але ставить межу для вдови.
\end{tcolorbox}
\begin{tcolorbox}
\textsubscript{26} Думки злого огида для Господа, але чисті для Нього приємні слова.
\end{tcolorbox}
\begin{tcolorbox}
\textsubscript{27} Зажерливий робить нещасним свій дім, хто ж дарунки ненавидить, той буде жити.
\end{tcolorbox}
\begin{tcolorbox}
\textsubscript{28} Серце праведного розмірковує про відповідь, а уста безбожних вибризкують зло.
\end{tcolorbox}
\begin{tcolorbox}
\textsubscript{29} Далекий Господь від безбожних, але справедливих молитву Він чує.
\end{tcolorbox}
\begin{tcolorbox}
\textsubscript{30} Світло очей тішить серце, добра звістка підкріплює кості.
\end{tcolorbox}
\begin{tcolorbox}
\textsubscript{31} Ухо, що навчання життя вислуховує, буде перебувати між мудрими.
\end{tcolorbox}
\begin{tcolorbox}
\textsubscript{32} Хто напучування не приймає, той не дбає про душу свою, а хто слухається остороги, здобуде той розум.
\end{tcolorbox}
\begin{tcolorbox}
\textsubscript{33} Страх Господній навчання премудрости, а перед славою скромність іде.
\end{tcolorbox}
\subsection{CHAPTER 16}
\begin{tcolorbox}
\textsubscript{1} Заміри серця належать людині, та від Господа відповідь язика.
\end{tcolorbox}
\begin{tcolorbox}
\textsubscript{2} Всі дороги людини чисті в очах її, та зважує душі Господь.
\end{tcolorbox}
\begin{tcolorbox}
\textsubscript{3} Поклади свої чини на Господа, і будуть поставлені міцно думки твої.
\end{tcolorbox}
\begin{tcolorbox}
\textsubscript{4} Все Господь учинив ради цілей Своїх, і безбожного на днину зла.
\end{tcolorbox}
\begin{tcolorbox}
\textsubscript{5} Огида для Господа всякий бундючний, ручуся: не буде такий без вини!
\end{tcolorbox}
\begin{tcolorbox}
\textsubscript{6} Провина викуплюється через милість та правду, і страх Господній відводить від злого.
\end{tcolorbox}
\begin{tcolorbox}
\textsubscript{7} Як дороги людини Господь уподобає, то й її ворогів Він замирює з нею.
\end{tcolorbox}
\begin{tcolorbox}
\textsubscript{8} Ліпше мале справедливе, аніж великі прибутки з безправ'я.
\end{tcolorbox}
\begin{tcolorbox}
\textsubscript{9} Розум людини обдумує путь її, але кроки її наставляє Господь.
\end{tcolorbox}
\begin{tcolorbox}
\textsubscript{10} Вирішальне слово в царя на губах, тому в суді уста його не спроневіряться.
\end{tcolorbox}
\begin{tcolorbox}
\textsubscript{11} Вага й шальки правдиві від Господа, все каміння вагове в торбинці то діло Його.
\end{tcolorbox}
\begin{tcolorbox}
\textsubscript{12} Чинити безбожне огида царям, бо трон зміцнюється справедливістю.
\end{tcolorbox}
\begin{tcolorbox}
\textsubscript{13} Уподоба царям губи праведности, і він любить того, хто правдиве говорить.
\end{tcolorbox}
\begin{tcolorbox}
\textsubscript{14} Гнів царя вісник смерти, та мудра людина злагіднить його.
\end{tcolorbox}
\begin{tcolorbox}
\textsubscript{15} У світлі царського обличчя життя, а його уподоба мов хмара дощева весною.
\end{tcolorbox}
\begin{tcolorbox}
\textsubscript{16} Набування премудрости як же це ліпше від золота, набування ж розуму добірніше від срібла!
\end{tcolorbox}
\begin{tcolorbox}
\textsubscript{17} Путь справедливих ухилятись від зла; хто дорогу свою береже, той душу свою охоронює.
\end{tcolorbox}
\begin{tcolorbox}
\textsubscript{18} Перед загибіллю гордість буває, а перед упадком бундючність.
\end{tcolorbox}
\begin{tcolorbox}
\textsubscript{19} Ліпше бути покірливим із лагідними, ніж здобич ділити з бундючними.
\end{tcolorbox}
\begin{tcolorbox}
\textsubscript{20} Хто вважає на слово, той знайде добро, хто ж надію складає на Господа буде блаженний.
\end{tcolorbox}
\begin{tcolorbox}
\textsubscript{21} Мудросердого кличуть розумний, а солодощ уст прибавляє науки.
\end{tcolorbox}
\begin{tcolorbox}
\textsubscript{22} Розум джерело життя власникові його, а картання безумних глупота.
\end{tcolorbox}
\begin{tcolorbox}
\textsubscript{23} Серце мудрого чинить розумними уста його, і на уста його прибавляє навчання.
\end{tcolorbox}
\begin{tcolorbox}
\textsubscript{24} Приємні слова щільниковий то мед, солодкий душі й лік на кості.
\end{tcolorbox}
\begin{tcolorbox}
\textsubscript{25} Буває, дорога людині здається простою, та кінець її стежка до смерти.
\end{tcolorbox}
\begin{tcolorbox}
\textsubscript{26} Людина трудяща працює для себе, бо до того примушує рот її.
\end{tcolorbox}
\begin{tcolorbox}
\textsubscript{27} Нікчемна людина копає лихе, а на устах її як палючий огонь.
\end{tcolorbox}
\begin{tcolorbox}
\textsubscript{28} Лукава людина сварки розсіває, а обмовник розділює друзів.
\end{tcolorbox}
\begin{tcolorbox}
\textsubscript{29} Насильник підмовлює друга свого, і провадить його по недобрій дорозі.
\end{tcolorbox}
\begin{tcolorbox}
\textsubscript{30} Хто прижмурює очі свої, той крутійства видумує, хто губами знаки подає, той виконує зло.
\end{tcolorbox}
\begin{tcolorbox}
\textsubscript{31} Сивизна то пишна корона, знаходять її на дорозі праведности.
\end{tcolorbox}
\begin{tcolorbox}
\textsubscript{32} Ліпший від силача, хто не скорий до гніву, хто ж панує над собою самим, ліпший від завойовника міста.
\end{tcolorbox}
\begin{tcolorbox}
\textsubscript{33} За пазуху жереб вкладається, та ввесь його вирок від Господа.
\end{tcolorbox}
\subsection{CHAPTER 17}
\begin{tcolorbox}
\textsubscript{1} Ліпший черствий кусок зо спокоєм, ніж дім, повний учти м'ясної зо сваркою.
\end{tcolorbox}
\begin{tcolorbox}
\textsubscript{2} Раб розумний панує над сином безпутнім, і серед братів він поділить спадок.
\end{tcolorbox}
\begin{tcolorbox}
\textsubscript{3} Для срібла топильна посудина, а горно для золота, Господь же серця випробовує.
\end{tcolorbox}
\begin{tcolorbox}
\textsubscript{4} Лиходій слухається уст безбожних, слухає неправдомов язика лиходійного.
\end{tcolorbox}
\begin{tcolorbox}
\textsubscript{5} Хто сміється з убогого, той ображає свого Творця, хто радіє з нещастя, не буде такий без вини.
\end{tcolorbox}
\begin{tcolorbox}
\textsubscript{6} Корона для старших онуки, а пишнота дітей їхні батьки.
\end{tcolorbox}
\begin{tcolorbox}
\textsubscript{7} Не пристойна безумному мова поважна, а тим більше шляхетному мова брехлива.
\end{tcolorbox}
\begin{tcolorbox}
\textsubscript{8} Хабар в очах його власника самоцвіт: до всього, до чого повернеться, буде щастити йому.
\end{tcolorbox}
\begin{tcolorbox}
\textsubscript{9} Хто шукає любови провину ховає, хто ж про неї повторює, розгонює друзів.
\end{tcolorbox}
\begin{tcolorbox}
\textsubscript{10} На розумного більше впливає одне остереження, як на глупака сто ударів.
\end{tcolorbox}
\begin{tcolorbox}
\textsubscript{11} Злий шукає лише неслухняности, та вісник жорстокий на нього пошлеться.
\end{tcolorbox}
\begin{tcolorbox}
\textsubscript{12} Ліпше спіткати обездітнену ведмедицю, що кидається на людину, аніж нерозумного в глупоті його.
\end{tcolorbox}
\begin{tcolorbox}
\textsubscript{13} Хто відплачує злом за добро, не відступить лихе з його дому.
\end{tcolorbox}
\begin{tcolorbox}
\textsubscript{14} Почин сварки то прорив води, тому перед вибухом сварки покинь ти її!
\end{tcolorbox}
\begin{tcolorbox}
\textsubscript{15} Хто оправдує несправедливого, і хто засуджує праведного, обидва вони Господеві огидні.
\end{tcolorbox}
\begin{tcolorbox}
\textsubscript{16} Нащо ті гроші в руці нерозумного, щоб мудрість купити, як мозку нема?
\end{tcolorbox}
\begin{tcolorbox}
\textsubscript{17} Правдивий друг любить за всякого часу, в недолі ж він робиться братом.
\end{tcolorbox}
\begin{tcolorbox}
\textsubscript{18} Людина, позбавлена розуму, ручиться, поруку дає за друга свого.
\end{tcolorbox}
\begin{tcolorbox}
\textsubscript{19} Хто сварку кохає, той любить гріх; хто ж підвищує уста свої, той шукає нещастя.
\end{tcolorbox}
\begin{tcolorbox}
\textsubscript{20} Людина лукавого серця не знайде добра, хто ж лукавить своїм язиком, упаде в зло.
\end{tcolorbox}
\begin{tcolorbox}
\textsubscript{21} Хто родить безумного, родить на смуток собі, і не потішиться батько безглуздого.
\end{tcolorbox}
\begin{tcolorbox}
\textsubscript{22} Серце радісне добре лікує, а пригноблений дух сушить кості.
\end{tcolorbox}
\begin{tcolorbox}
\textsubscript{23} Безбожний таємно бере хабара, щоб зігнути путі правосуддя.
\end{tcolorbox}
\begin{tcolorbox}
\textsubscript{24} З обличчям розумного мудрість, а очі глупця аж на кінці землі.
\end{tcolorbox}
\begin{tcolorbox}
\textsubscript{25} Нерозумний син смуток для батька, для своєї ж родительки гіркість.
\end{tcolorbox}
\begin{tcolorbox}
\textsubscript{26} Не добре карати справедливого, бити шляхетних за щирість!
\end{tcolorbox}
\begin{tcolorbox}
\textsubscript{27} Хто слова свої стримує, той знає пізнання, і холоднокровний розумна людина.
\end{tcolorbox}
\begin{tcolorbox}
\textsubscript{28} І глупак, як мовчить, уважається мудрим, а як уста свої закриває розумним.
\end{tcolorbox}
\subsection{CHAPTER 18}
\begin{tcolorbox}
\textsubscript{1} Примхливий шукає сваволі, стає проти всього розумного.
\end{tcolorbox}
\begin{tcolorbox}
\textsubscript{2} Нерозумний не хоче навчатися, а тільки свій ум показати.
\end{tcolorbox}
\begin{tcolorbox}
\textsubscript{3} З приходом безбожного й ганьба приходить, а з легковаженням сором.
\end{tcolorbox}
\begin{tcolorbox}
\textsubscript{4} Слова уст людини глибока вода, джерело премудрости бризкотливий потік.
\end{tcolorbox}
\begin{tcolorbox}
\textsubscript{5} Не добре вважати на обличчя безбожного, щоб праведного повалити на суді.
\end{tcolorbox}
\begin{tcolorbox}
\textsubscript{6} Уста нерозумного тягнуть до сварки, а слова його кличуть бійки.
\end{tcolorbox}
\begin{tcolorbox}
\textsubscript{7} Язик нерозумного загибіль для нього, а уста його то тенета на душу його.
\end{tcolorbox}
\begin{tcolorbox}
\textsubscript{8} Слова обмовника мов ті присмаки, і вони сходять у нутро утроби.
\end{tcolorbox}
\begin{tcolorbox}
\textsubscript{9} Теж недбалий у праці своїй то брат марнотратнику.
\end{tcolorbox}
\begin{tcolorbox}
\textsubscript{10} Господнє Ім'я сильна башта: до неї втече справедливий і буде безпечний.
\end{tcolorbox}
\begin{tcolorbox}
\textsubscript{11} Маєток багатому місто твердинне його, і немов міцний мур ув уяві його.
\end{tcolorbox}
\begin{tcolorbox}
\textsubscript{12} Перед загибіллю серце людини високо несеться, перед славою ж скромність.
\end{tcolorbox}
\begin{tcolorbox}
\textsubscript{13} Хто відповідає на слово, ще поки почув, то глупота та сором йому!
\end{tcolorbox}
\begin{tcolorbox}
\textsubscript{14} Дух дійсного мужа виносить терпіння своє, а духа прибитого хто піднесе?
\end{tcolorbox}
\begin{tcolorbox}
\textsubscript{15} Серце розумне знання набуває, і вухо премудрих шукає знання.
\end{tcolorbox}
\begin{tcolorbox}
\textsubscript{16} Дарунок людини виводить із утиску, і провадить її до великих людей.
\end{tcolorbox}
\begin{tcolorbox}
\textsubscript{17} Перший у сварці своїй уважає себе справедливим, але прийде противник його та й дослідить його.
\end{tcolorbox}
\begin{tcolorbox}
\textsubscript{18} Жереб перериває сварки, та відділює сильних один від одного.
\end{tcolorbox}
\begin{tcolorbox}
\textsubscript{19} Розлючений брат протиставиться більше за місто твердинне, а сварки, немов засуви замку.
\end{tcolorbox}
\begin{tcolorbox}
\textsubscript{20} Із плоду уст людини насичується її шлунок, вона насичується плодом уст своїх.
\end{tcolorbox}
\begin{tcolorbox}
\textsubscript{21} Смерть та життя у владі язика, хто ж кохає його, його плід поїдає.
\end{tcolorbox}
\begin{tcolorbox}
\textsubscript{22} Хто жінку чеснотну знайшов, знайшов той добро, і милість отримав від Господа.
\end{tcolorbox}
\begin{tcolorbox}
\textsubscript{23} Убогий говорить благально, багатий же відповідає зухвало.
\end{tcolorbox}
\begin{tcolorbox}
\textsubscript{24} Є товариші на розбиття, та є й приятель, більше від брата прив'язаний.
\end{tcolorbox}
\subsection{CHAPTER 19}
\begin{tcolorbox}
\textsubscript{1} Ліпший убогий, що ходить в своїй неповинності, ніж лукавий устами та нерозумний.
\end{tcolorbox}
\begin{tcolorbox}
\textsubscript{2} Теж не добра душа без знання, а хто наглить ногами, спіткнеться.
\end{tcolorbox}
\begin{tcolorbox}
\textsubscript{3} Глупота людини дорогу її викривляє, і на Господа гнівається її серце.
\end{tcolorbox}
\begin{tcolorbox}
\textsubscript{4} Маєток примножує друзів численних, а від бідака відпадає й товариш його...
\end{tcolorbox}
\begin{tcolorbox}
\textsubscript{5} Свідок брехливий не буде без кари, а хто брехні говорить, не буде врятований.
\end{tcolorbox}
\begin{tcolorbox}
\textsubscript{6} Багато-хто годять тому, хто гостинці дає, і кожен товариш людині, яка не скупиться на дари.
\end{tcolorbox}
\begin{tcolorbox}
\textsubscript{7} Бідаря ненавидять всі браття його, а тимбільш його приятелі відпадають від нього; а коли за словами поради женеться, нема їх!
\end{tcolorbox}
\begin{tcolorbox}
\textsubscript{8} Хто ума набуває, кохає той душу свою, а хто розум стереже, той знаходить добро.
\end{tcolorbox}
\begin{tcolorbox}
\textsubscript{9} Свідок брехливий не буде без кари, хто ж неправду говорить, загине.
\end{tcolorbox}
\begin{tcolorbox}
\textsubscript{10} Не лицює пишнота безумному, тим більше рабові панувати над зверхником.
\end{tcolorbox}
\begin{tcolorbox}
\textsubscript{11} Розум людини припинює гнів її, а величність її перейти над провиною.
\end{tcolorbox}
\begin{tcolorbox}
\textsubscript{12} Гнів царя немов рик левчука, а ласкавість його як роса на траву.
\end{tcolorbox}
\begin{tcolorbox}
\textsubscript{13} Син безумний погибіль для батька свого, а жінка сварлива як ринва, що з неї вода тече завжди.
\end{tcolorbox}
\begin{tcolorbox}
\textsubscript{14} Хата й маєток спадщина батьків, а жінка розумна від Господа.
\end{tcolorbox}
\begin{tcolorbox}
\textsubscript{15} Лінощі сон накидають, і лінива душа голодує.
\end{tcolorbox}
\begin{tcolorbox}
\textsubscript{16} Хто заповідь охороняє, той душу свою стереже; хто дороги свої легковажить, помре.
\end{tcolorbox}
\begin{tcolorbox}
\textsubscript{17} Хто милостивий до вбогого, той позичає для Господа, і чин його Він надолужить йому.
\end{tcolorbox}
\begin{tcolorbox}
\textsubscript{18} Картай свого сина, коли є надія навчити, та забити його не піднось свою душу.
\end{tcolorbox}
\begin{tcolorbox}
\textsubscript{19} Людина великого гніву хай кару несе, бо якщо ти врятуєш її, то вчиниш ще гірше.
\end{tcolorbox}
\begin{tcolorbox}
\textsubscript{20} Слухай ради й картання приймай, щоб мудрим ти став при своєму кінці.
\end{tcolorbox}
\begin{tcolorbox}
\textsubscript{21} У серці людини багато думок, але виповниться тільки задум Господній.
\end{tcolorbox}
\begin{tcolorbox}
\textsubscript{22} Здобуток людині то милість її, але ліпший бідар за людину брехливу.
\end{tcolorbox}
\begin{tcolorbox}
\textsubscript{23} Страх Господній веде до життя, і хто його має, той ситим ночує, і зло не досягне його.
\end{tcolorbox}
\begin{tcolorbox}
\textsubscript{24} У миску стромляє лінюх свою руку, до уст же своїх не підійме її.
\end{tcolorbox}
\begin{tcolorbox}
\textsubscript{25} Як битимеш нерозважного, то помудріє й немудрий, а будеш розумного остерігати, то він зрозуміє поуку.
\end{tcolorbox}
\begin{tcolorbox}
\textsubscript{26} Хто батька грабує, хто матір жене? Це син, що застиджує та осоромлює,
\end{tcolorbox}
\begin{tcolorbox}
\textsubscript{27} перестань же, мій сину, навчатися від нерозумних, щоб відступитися від слів знання!
\end{tcolorbox}
\begin{tcolorbox}
\textsubscript{28} Свідок нікчемний висміює суд, а уста безбожних вибризкують кривду.
\end{tcolorbox}
\begin{tcolorbox}
\textsubscript{29} На насмішників кари готові постійно, і вдари на спину безумним.
\end{tcolorbox}
\subsection{CHAPTER 20}
\begin{tcolorbox}
\textsubscript{1} Вино то насмішник, напій п'янкий галасун, і кожен, хто блудить у ньому, немудрий.
\end{tcolorbox}
\begin{tcolorbox}
\textsubscript{2} Страх царя як рик лева; хто до гніву доводить його, проти свого життя прогрішає.
\end{tcolorbox}
\begin{tcolorbox}
\textsubscript{3} Слава людині, що гнів покидає, а кожен глупак вибухає.
\end{tcolorbox}
\begin{tcolorbox}
\textsubscript{4} Лінивий не оре із осени, а захоче в жнива і нічого нема.
\end{tcolorbox}
\begin{tcolorbox}
\textsubscript{5} Рада в серці людини глибока вода, і розумна людина її повичерпує.
\end{tcolorbox}
\begin{tcolorbox}
\textsubscript{6} Багато людей себе звуть милосердними, та вірну людину хто знайде?
\end{tcolorbox}
\begin{tcolorbox}
\textsubscript{7} У своїй неповинності праведний ходить, блаженні по ньому сини його!
\end{tcolorbox}
\begin{tcolorbox}
\textsubscript{8} Цар сидить на суддевім престолі, всяке зло розганяє своїми очима.
\end{tcolorbox}
\begin{tcolorbox}
\textsubscript{9} Хто скаже: Очистив я серце своє, очистився я від свого гріха?
\end{tcolorbox}
\begin{tcolorbox}
\textsubscript{10} Вага неоднакова, неоднакова міра, обоє вони то огида для Господа.
\end{tcolorbox}
\begin{tcolorbox}
\textsubscript{11} Навіть юнак буде пізнаний з чинів своїх, чи чин його чистий й чи простий.
\end{tcolorbox}
\begin{tcolorbox}
\textsubscript{12} Ухо, що слухає, й око, що бачить, Господь учинив їх обоє.
\end{tcolorbox}
\begin{tcolorbox}
\textsubscript{13} Не кохайся в спанні, щоб не збідніти; розплющ свої очі та хлібом наситься!
\end{tcolorbox}
\begin{tcolorbox}
\textsubscript{14} Зле, зле! каже той, хто купує, а як піде собі, тоді хвалиться купном.
\end{tcolorbox}
\begin{tcolorbox}
\textsubscript{15} Є золото й перел багато, та розумні уста найцінніший то посуд.
\end{tcolorbox}
\begin{tcolorbox}
\textsubscript{16} Візьми його одіж, бо він поручивсь за чужого, і за чужинку візьми його застав.
\end{tcolorbox}
\begin{tcolorbox}
\textsubscript{17} Хліб з неправди солодкий людині, та піском потім будуть наповнені уста її.
\end{tcolorbox}
\begin{tcolorbox}
\textsubscript{18} Тримаються заміри радою, і війну провадь мудрими радами.
\end{tcolorbox}
\begin{tcolorbox}
\textsubscript{19} Виявляє обмовник таємне, а ти не втручайся до того, легко хто розтулює уста свої.
\end{tcolorbox}
\begin{tcolorbox}
\textsubscript{20} Хто кляне свого батька та матір свою, погасне світильник йому серед темряви!
\end{tcolorbox}
\begin{tcolorbox}
\textsubscript{21} Спадок спочатку заскоро набутий, не буде кінець його поблагословлений!
\end{tcolorbox}
\begin{tcolorbox}
\textsubscript{22} Не кажи: Надолужу я зло! май надію на Господа, і Він допоможе тобі.
\end{tcolorbox}
\begin{tcolorbox}
\textsubscript{23} Вага неоднакова то огида для Господа, а оманливі шальки не добрі.
\end{tcolorbox}
\begin{tcolorbox}
\textsubscript{24} Від Господа кроки людини, а людина як вона зрозуміє дорогу свою?
\end{tcolorbox}
\begin{tcolorbox}
\textsubscript{25} Тенета людині казати святе нерозважно, а згодом свої обітниці досліджувати.
\end{tcolorbox}
\begin{tcolorbox}
\textsubscript{26} Мудрий цар розпорошить безбожних, і зверне на них своє коло для мук.
\end{tcolorbox}
\begin{tcolorbox}
\textsubscript{27} Дух людини світильник Господній, що все нутро обшукує.
\end{tcolorbox}
\begin{tcolorbox}
\textsubscript{28} Милість та правда царя стережуть, і трона свого він підтримує милістю.
\end{tcolorbox}
\begin{tcolorbox}
\textsubscript{29} Окраса юнацтва їхня сила, а пишність старих сивина.
\end{tcolorbox}
\begin{tcolorbox}
\textsubscript{30} Синяки від побоїв то масть лікувальна на злого, та вдари нутру живота.
\end{tcolorbox}
\subsection{CHAPTER 21}
\begin{tcolorbox}
\textsubscript{1} Водні потоки цареве це серце в Господній руці: куди тільки захоче, його Він скеровує.
\end{tcolorbox}
\begin{tcolorbox}
\textsubscript{2} Всяка дорога людини пряма в її очах, та керує серцями Господь.
\end{tcolorbox}
\begin{tcolorbox}
\textsubscript{3} Справедливість та правду чинити для Господа це добірніше за жертву.
\end{tcolorbox}
\begin{tcolorbox}
\textsubscript{4} Муж гордого ока та серця надутого несправедливий, а світильник безбожних це гріх.
\end{tcolorbox}
\begin{tcolorbox}
\textsubscript{5} Думки пильного лиш на достаток ведуть, а всякий квапливий на збиток.
\end{tcolorbox}
\begin{tcolorbox}
\textsubscript{6} Набування майна язиком неправдивим це скороминуща марнота шукаючих смерти.
\end{tcolorbox}
\begin{tcolorbox}
\textsubscript{7} Насильство безбожних прямує на них, бо права чинити не хочуть.
\end{tcolorbox}
\begin{tcolorbox}
\textsubscript{8} Дорога злочинця крута, а чистий прямий його чин.
\end{tcolorbox}
\begin{tcolorbox}
\textsubscript{9} Ліпше жити в куті на даху, ніж з сварливою жінкою в спільному домі.
\end{tcolorbox}
\begin{tcolorbox}
\textsubscript{10} Лихого жадає душа нечестивого, і в очах його ближній його не отримає милости.
\end{tcolorbox}
\begin{tcolorbox}
\textsubscript{11} Як карають глумливця мудріє безумний, а як мудрого вчать, знання набуває.
\end{tcolorbox}
\begin{tcolorbox}
\textsubscript{12} До дому свого приглядається праведний, а безбожний доводить до зла.
\end{tcolorbox}
\begin{tcolorbox}
\textsubscript{13} Хто вухо своє затикає від зойку убогого, то й він буде кликати, та не отримає відповіді.
\end{tcolorbox}
\begin{tcolorbox}
\textsubscript{14} Таємний дарунок погашує гнів, а неявний гостинець лють сильну.
\end{tcolorbox}
\begin{tcolorbox}
\textsubscript{15} Радість праведному правосуддя чинити, а злочинцеві страх.
\end{tcolorbox}
\begin{tcolorbox}
\textsubscript{16} Людина, що зблуджує від путі розуму, у зборі померлих спочине.
\end{tcolorbox}
\begin{tcolorbox}
\textsubscript{17} Хто любить веселощі, той немаючий, хто любить вино та оливу, той не збагатіє.
\end{tcolorbox}
\begin{tcolorbox}
\textsubscript{18} Безбожний то викуп за праведного, а лукавий за щирого.
\end{tcolorbox}
\begin{tcolorbox}
\textsubscript{19} Ліпше сидіти в пустинній країні, ніж з сварливою та сердитою жінкою.
\end{tcolorbox}
\begin{tcolorbox}
\textsubscript{20} Скарб цінний та олива в мешканні премудрого, та нищить безумна людина його.
\end{tcolorbox}
\begin{tcolorbox}
\textsubscript{21} Хто женеться за праведністю та за милістю, той знаходить життя, справедливість та славу.
\end{tcolorbox}
\begin{tcolorbox}
\textsubscript{22} До міста хоробрих увійде премудрий, і твердиню надії його поруйнує.
\end{tcolorbox}
\begin{tcolorbox}
\textsubscript{23} Хто стереже свої уста й свого язика, той душу свою зберігає від лиха.
\end{tcolorbox}
\begin{tcolorbox}
\textsubscript{24} Надутий пихою насмішник ім'я йому, він робить усе із бундючним зухвальством.
\end{tcolorbox}
\begin{tcolorbox}
\textsubscript{25} Пожадання лінивого вб'є його, бо руки його відмовляють робити,
\end{tcolorbox}
\begin{tcolorbox}
\textsubscript{26} він кожного дня пожадливо жадає, а справедливий дає та не жалує.
\end{tcolorbox}
\begin{tcolorbox}
\textsubscript{27} Жертва безбожних огида, а надто тоді, як за діло безчесне приноситься.
\end{tcolorbox}
\begin{tcolorbox}
\textsubscript{28} Свідок брехливий загине, а людина, що слухає Боже, говоритиме завжди.
\end{tcolorbox}
\begin{tcolorbox}
\textsubscript{29} Безбожна людина жорстока обличчям своїм, а невинний зміцняє дорогу свою.
\end{tcolorbox}
\begin{tcolorbox}
\textsubscript{30} Нема мудрости, ані розуму, ані ради насупроти Господа.
\end{tcolorbox}
\begin{tcolorbox}
\textsubscript{31} Приготовлений кінь на день бою, але перемога від Господа!
\end{tcolorbox}
\subsection{CHAPTER 22}
\begin{tcolorbox}
\textsubscript{1} Ліпше добре ім'я за багатство велике, і ліпша милість за срібло та золото.
\end{tcolorbox}
\begin{tcolorbox}
\textsubscript{2} Багатий та вбогий стрічаються, Господь їх обох створив.
\end{tcolorbox}
\begin{tcolorbox}
\textsubscript{3} Мудрий бачить лихе і ховається, а безумні йдуть і караються.
\end{tcolorbox}
\begin{tcolorbox}
\textsubscript{4} Заплата покори і страху Господнього, це багатство, і слава, й життя.
\end{tcolorbox}
\begin{tcolorbox}
\textsubscript{5} Тернина й пастки на дорозі лукавого, а хто стереже свою душу, відійде далеко від них.
\end{tcolorbox}
\begin{tcolorbox}
\textsubscript{6} Привчай юнака до дороги його, і він, як постаріється, не уступиться з неї.
\end{tcolorbox}
\begin{tcolorbox}
\textsubscript{7} Багатий панує над бідними, а боржник раб позичальника.
\end{tcolorbox}
\begin{tcolorbox}
\textsubscript{8} Хто сіє кривду, той жатиме лихо, а бич гніву його покінчиться.
\end{tcolorbox}
\begin{tcolorbox}
\textsubscript{9} Хто доброго ока, той поблагословлений буде, бо дає він убогому з хліба свого.
\end{tcolorbox}
\begin{tcolorbox}
\textsubscript{10} Глумливого вижени, й вийде з ним сварка, і суперечка та ганьба припиняться.
\end{tcolorbox}
\begin{tcolorbox}
\textsubscript{11} Хто чистість серця кохає, той має хороше на устах, і другом йому буде цар.
\end{tcolorbox}
\begin{tcolorbox}
\textsubscript{12} Очі Господа оберігають знання, а лукаві слова Він відкине.
\end{tcolorbox}
\begin{tcolorbox}
\textsubscript{13} Лінивий говорить: На вулиці лев, серед майдану я буду забитий!
\end{tcolorbox}
\begin{tcolorbox}
\textsubscript{14} Уста коханки яма глибока: на кого Господь має гнів, той впадає туди.
\end{tcolorbox}
\begin{tcolorbox}
\textsubscript{15} До юнакового серця глупота прив'язана, та різка картання віддалить від нього її.
\end{tcolorbox}
\begin{tcolorbox}
\textsubscript{16} Хто тисне убогого, щоб собі збагатитись, і хто багачеві дає, той певно збідніє.
\end{tcolorbox}
\begin{tcolorbox}
\textsubscript{17} Нахили своє вухо, і послухай слів мудрих, і серце зверни до мого знання,
\end{tcolorbox}
\begin{tcolorbox}
\textsubscript{18} бо гарне воно, коли будеш ти їх у своєму нутрі стерегти, хай стануть на устах твоїх вони разом!
\end{tcolorbox}
\begin{tcolorbox}
\textsubscript{19} Щоб надія твоя була в Господі, я й сьогодні навчаю тебе.
\end{tcolorbox}
\begin{tcolorbox}
\textsubscript{20} Хіба ж не писав тобі тричі з порадами та із знанням,
\end{tcolorbox}
\begin{tcolorbox}
\textsubscript{21} щоб тобі завідомити правду, правдиві слова, щоб ти істину міг відповісти тому, хто тебе запитає.
\end{tcolorbox}
\begin{tcolorbox}
\textsubscript{22} Не грабуй незаможнього, бо він незаможній, і не тисни убогого в брамі,
\end{tcolorbox}
\begin{tcolorbox}
\textsubscript{23} бо Господь за їхню справу судитиметься, і грабіжникам їхнім ограбує Він душу.
\end{tcolorbox}
\begin{tcolorbox}
\textsubscript{24} Не дружись із чоловіком гнівливим, і не ходи із людиною лютою,
\end{tcolorbox}
\begin{tcolorbox}
\textsubscript{25} щоб доріг її ти не навчився, і тенета не взяв для своєї душі.
\end{tcolorbox}
\begin{tcolorbox}
\textsubscript{26} Не будь серед тих, хто поруку дає, серед тих, хто поручується за борги:
\end{tcolorbox}
\begin{tcolorbox}
\textsubscript{27} коли ти не матимеш чим заплатити, нащо візьмуть з-під тебе постелю твою?
\end{tcolorbox}
\begin{tcolorbox}
\textsubscript{28} Не пересувай вікової границі, яку встановили батьки твої.
\end{tcolorbox}
\begin{tcolorbox}
\textsubscript{29} Ти бачив людину, моторну в занятті своїм? Вона перед царями спокійно стоятиме, та не встоїть вона перед простими.
\end{tcolorbox}
\subsection{CHAPTER 23}
\begin{tcolorbox}
\textsubscript{1} Коли сядеш хліб їсти з володарем, то пильно вважай, що перед тобою,
\end{tcolorbox}
\begin{tcolorbox}
\textsubscript{2} і поклади собі в горло ножа, якщо ти ненажера:
\end{tcolorbox}
\begin{tcolorbox}
\textsubscript{3} не жадай його ласощів, бо вони хліб обманливий!
\end{tcolorbox}
\begin{tcolorbox}
\textsubscript{4} Не мордуйся, щоб мати багатство, відступися від думки своєї про це,
\end{tcolorbox}
\begin{tcolorbox}
\textsubscript{5} свої очі ти звернеш на нього, й нема вже його: бо конче змайструє воно собі крила, і полетить, мов орел той, до неба...
\end{tcolorbox}
\begin{tcolorbox}
\textsubscript{6} Не їж хліба в злоокого, і не пожадай лакоминок його,
\end{tcolorbox}
\begin{tcolorbox}
\textsubscript{7} бо як у душі своїй він обраховує, такий є. Він скаже тобі: Їж та пий! але серце його не з тобою,
\end{tcolorbox}
\begin{tcolorbox}
\textsubscript{8} той кавалок, якого ти з'їв, із себе викинеш, і свої гарні слова надаремно потратиш!
\end{tcolorbox}
\begin{tcolorbox}
\textsubscript{9} Не кажи до ушей нерозумному, бо погордить він мудрістю слів твоїх.
\end{tcolorbox}
\begin{tcolorbox}
\textsubscript{10} Не пересувай вікової границі, і не входь на сирітські поля,
\end{tcolorbox}
\begin{tcolorbox}
\textsubscript{11} бо їхній Визволитель міцний, Він за справу їхню буде судитись з тобою!
\end{tcolorbox}
\begin{tcolorbox}
\textsubscript{12} Своє серце зверни до навчання, а уші свої до розумних речей.
\end{tcolorbox}
\begin{tcolorbox}
\textsubscript{13} Не стримуй напучування юнака, коли різкою виб'єш його, не помре:
\end{tcolorbox}
\begin{tcolorbox}
\textsubscript{14} ти різкою виб'єш його, і душу його від шеолу врятуєш.
\end{tcolorbox}
\begin{tcolorbox}
\textsubscript{15} Мій сину, якщо твоє серце змудріло, то буде радіти також моє серце,
\end{tcolorbox}
\begin{tcolorbox}
\textsubscript{16} і нутро моє буде тішитись, коли уста твої говоритимуть слушне.
\end{tcolorbox}
\begin{tcolorbox}
\textsubscript{17} Нехай серце твоє не завидує грішним, і повсякчас пильнуй тільки страху Господнього,
\end{tcolorbox}
\begin{tcolorbox}
\textsubscript{18} бо існує майбутнє, і надія твоя не загине.
\end{tcolorbox}
\begin{tcolorbox}
\textsubscript{19} Послухай, мій сину, та й помудрій, і нехай твоє серце ступає дорогою рівною.
\end{tcolorbox}
\begin{tcolorbox}
\textsubscript{20} Не будь поміж тими, що жлуктять вино, поміж тими, що м'ясо собі пожирають,
\end{tcolorbox}
\begin{tcolorbox}
\textsubscript{21} бо п'яниця й жерун збідніють, а сонливий одягне лахміття.
\end{tcolorbox}
\begin{tcolorbox}
\textsubscript{22} Слухай батька свого, він тебе породив, і не гордуй, як постаріла мати твоя.
\end{tcolorbox}
\begin{tcolorbox}
\textsubscript{23} Купи собі й не продавай правду, мудрість, і картання та розум.
\end{tcolorbox}
\begin{tcolorbox}
\textsubscript{24} Буде вельми радіти батько праведного, і родитель премудрого втішиться ним.
\end{tcolorbox}
\begin{tcolorbox}
\textsubscript{25} Хай радіє твій батько та мати твоя, хай потішиться та, що тебе породила.
\end{tcolorbox}
\begin{tcolorbox}
\textsubscript{26} Дай мені, сину мій, своє серце, і очі твої хай кохають дороги мої.
\end{tcolorbox}
\begin{tcolorbox}
\textsubscript{27} Бо блудниця то яма глибока, а криниця тісна чужа жінка.
\end{tcolorbox}
\begin{tcolorbox}
\textsubscript{28} І вона, мов грабіжник, чатує, і примножує зрадників поміж людьми.
\end{tcolorbox}
\begin{tcolorbox}
\textsubscript{29} В кого ой, в кого ай, в кого сварки, в кого клопіт, в кого рани даремні, в кого очі червоні?
\end{tcolorbox}
\begin{tcolorbox}
\textsubscript{30} У тих, хто запізнюється над вином, у тих, хто приходить попробувати вина змішаного.
\end{tcolorbox}
\begin{tcolorbox}
\textsubscript{31} Не дивись на вино, як воно рум'яніє, як виблискує в келіху й рівненько ллється,
\end{tcolorbox}
\begin{tcolorbox}
\textsubscript{32} кінець його буде кусати, як гад, і вжалить, немов та гадюка,
\end{tcolorbox}
\begin{tcolorbox}
\textsubscript{33} пантруватимуть очі твої на чужі жінки, і серце твоє говоритиме дурощі...
\end{tcolorbox}
\begin{tcolorbox}
\textsubscript{34} І ти будеш, як той, хто лежить у середині моря, й як той, хто лежить на щогловім верху.
\end{tcolorbox}
\begin{tcolorbox}
\textsubscript{35} І скажеш: Побили мене, та мені не боліло, мене штурхали, я ж не почув, коли я прокинусь, шукатиму далі того ж...
\end{tcolorbox}
\subsection{CHAPTER 24}
\begin{tcolorbox}
\textsubscript{1} Не завидуй злим людям, не бажай бути з ними,
\end{tcolorbox}
\begin{tcolorbox}
\textsubscript{2} бо їхне серце говорить про здирство, а уста їхні мовлять про зло.
\end{tcolorbox}
\begin{tcolorbox}
\textsubscript{3} Дім будується мудрістю, і розумом ставиться міцно.
\end{tcolorbox}
\begin{tcolorbox}
\textsubscript{4} А через пізнання кімнати наповнюються усіляким маєтком цінним та приємним.
\end{tcolorbox}
\begin{tcolorbox}
\textsubscript{5} Мудрий сильніший від сильного, а людина розумна від повносилого.
\end{tcolorbox}
\begin{tcolorbox}
\textsubscript{6} Тому то провадь війну мудрими радами, бо спасіння в численності радників.
\end{tcolorbox}
\begin{tcolorbox}
\textsubscript{7} Для безумного мудрість занадто висока, своїх уст не розкриє при брамі.
\end{tcolorbox}
\begin{tcolorbox}
\textsubscript{8} Хто чинити лихе заміряє, того звуть лукавим.
\end{tcolorbox}
\begin{tcolorbox}
\textsubscript{9} Замір глупоти то гріх, а насмішник огида людині.
\end{tcolorbox}
\begin{tcolorbox}
\textsubscript{10} Якщо ти в день недолі знесилився, то мала твоя сила.
\end{tcolorbox}
\begin{tcolorbox}
\textsubscript{11} Рятуй узятих на смерть, також тих, хто на страчення хилиться, хіба не підтримаєш їх?
\end{tcolorbox}
\begin{tcolorbox}
\textsubscript{12} Якщо скажеш: Цього ми не знали! чи ж Той, хто серця випробовує, знати не буде? Він Сторож твоєї душі, і він знає про це, і поверне людині за чином її.
\end{tcolorbox}
\begin{tcolorbox}
\textsubscript{13} Їж, сину мій, мед, бо він добрий, а мед щільниковий солодкий він на піднебінні твоїм,
\end{tcolorbox}
\begin{tcolorbox}
\textsubscript{14} отак мудрість пізнай для своєї душі: якщо знайдеш її, то ти маєш майбутність, і надія твоя не понищиться!
\end{tcolorbox}
\begin{tcolorbox}
\textsubscript{15} Не чатуй на помешкання праведного, ти безбожнику, не ограблюй мешкання його,
\end{tcolorbox}
\begin{tcolorbox}
\textsubscript{16} бо праведний сім раз впаде та зведеться, а безбожний в погибіль впаде!
\end{tcolorbox}
\begin{tcolorbox}
\textsubscript{17} Не тішся, як ворог твій падає, а коли він спіткнеться, хай серце твоє не радіє,
\end{tcolorbox}
\begin{tcolorbox}
\textsubscript{18} щоб Господь не побачив, і це не було в Його очах лихим, і щоб Він не звернув Свого гніву від нього на тебе!
\end{tcolorbox}
\begin{tcolorbox}
\textsubscript{19} Не пались на злочинців, не заздри безбожним,
\end{tcolorbox}
\begin{tcolorbox}
\textsubscript{20} бо злому не буде майбутности, світильник безбожних погасне.
\end{tcolorbox}
\begin{tcolorbox}
\textsubscript{21} Бійся, сину мій, Господа та царя, не водися з непевними,
\end{tcolorbox}
\begin{tcolorbox}
\textsubscript{22} бо погибіль їхня нагло постане, а біду від обох тих хто знає?
\end{tcolorbox}
\begin{tcolorbox}
\textsubscript{23} І оце ось походить від мудрих: Звертати увагу в суді на обличчя не добре.
\end{tcolorbox}
\begin{tcolorbox}
\textsubscript{24} Хто буде казати безбожному: Праведний ти! того проклинатимуть люди, і гніватись будуть на того народи.
\end{tcolorbox}
\begin{tcolorbox}
\textsubscript{25} А тим, хто картає його, буде миле оце, і прийде на них благословення добра!
\end{tcolorbox}
\begin{tcolorbox}
\textsubscript{26} Мов у губи цілує, хто відповідає правдиве.
\end{tcolorbox}
\begin{tcolorbox}
\textsubscript{27} Приготуй свою працю надворі, й оброби собі поле, а потім збудуєш свій дім.
\end{tcolorbox}
\begin{tcolorbox}
\textsubscript{28} Не будь ложним свідком на свого ближнього, і не підговорюй устами своїми.
\end{tcolorbox}
\begin{tcolorbox}
\textsubscript{29} Не кажи: Як зробив він мені, так зроблю я йому, верну людині за чином її!
\end{tcolorbox}
\begin{tcolorbox}
\textsubscript{30} Я проходив край поля людини лінивої, та край виноградника недоумкуватого,
\end{tcolorbox}
\begin{tcolorbox}
\textsubscript{31} і ось все воно позаростало терням, будяками покрита поверхня його, камінний же мур його був поруйнований...
\end{tcolorbox}
\begin{tcolorbox}
\textsubscript{32} І бачив я те, і увагу звернув, і взяв я поуку собі:
\end{tcolorbox}
\begin{tcolorbox}
\textsubscript{33} Ще трохи поспати, подрімати ще трохи, руки трохи зложити, щоб полежати,
\end{tcolorbox}
\begin{tcolorbox}
\textsubscript{34} і приходить, немов мандрівник, незаможність твоя, і нужда твоя, як озброєний муж!...
\end{tcolorbox}
\subsection{CHAPTER 25}
\begin{tcolorbox}
\textsubscript{1} І оце Соломонові приповісті, що зібрали люди Єзекії, Юдиного царя.
\end{tcolorbox}
\begin{tcolorbox}
\textsubscript{2} Слава Божа щоб справу сховати, а слава царів щоб розвідати справу.
\end{tcolorbox}
\begin{tcolorbox}
\textsubscript{3} Небо високістю, і земля глибиною, і серце царів недослідимі.
\end{tcolorbox}
\begin{tcolorbox}
\textsubscript{4} Як відкинути жужель від срібла, то золотареві виходить посудина,
\end{tcolorbox}
\begin{tcolorbox}
\textsubscript{5} коли віддалити безбожного з-перед обличчя царевого, то справедливістю міцно поставиться трон його.
\end{tcolorbox}
\begin{tcolorbox}
\textsubscript{6} Перед царем не пишайся, а на місці великих не стій,
\end{tcolorbox}
\begin{tcolorbox}
\textsubscript{7} бо ліпше, як скажуть тобі: Ходи вище сюди! аніж тебе знизити перед шляхетним, що бачили очі твої.
\end{tcolorbox}
\begin{tcolorbox}
\textsubscript{8} Не спішися ставати до позову, бо що будеш робити в кінці його, як тебе засоромить твій ближній?
\end{tcolorbox}
\begin{tcolorbox}
\textsubscript{9} Судися за сварку свою з своїм ближнім, але не виявляй таємниці іншого,
\end{tcolorbox}
\begin{tcolorbox}
\textsubscript{10} щоб тебе не образив, хто слухати буде, і щоб не вернулась на тебе обмова твоя.
\end{tcolorbox}
\begin{tcolorbox}
\textsubscript{11} Золоті яблука на срібнім тарелі це слово, проказане часу свого.
\end{tcolorbox}
\begin{tcolorbox}
\textsubscript{12} Золотая сережка й оздоба зо щирого золота це мудрий картач для уважного уха.
\end{tcolorbox}
\begin{tcolorbox}
\textsubscript{13} Немов снігова прохолода в день жнив посол вірний для тих, хто його посилає, і він душу пана свого оживляє.
\end{tcolorbox}
\begin{tcolorbox}
\textsubscript{14} Хмари та вітер, а немає дощу це людина, що чваниться даром, та його не дає.
\end{tcolorbox}
\begin{tcolorbox}
\textsubscript{15} Володар зм'якшується терпеливістю, а м'якенький язик ломить кістку.
\end{tcolorbox}
\begin{tcolorbox}
\textsubscript{16} Якщо мед ти знайшов, то спожий, скільки досить тобі, щоб ним не пересититися та не звернути.
\end{tcolorbox}
\begin{tcolorbox}
\textsubscript{17} Здержуй ногу свою від дому твого товариша, щоб тобою він не переситивсь, і не зненавидів тебе.
\end{tcolorbox}
\begin{tcolorbox}
\textsubscript{18} Молот, і меч, і гостра стріла, людина, що говорить на ближнього свого, як свідок брехливий.
\end{tcolorbox}
\begin{tcolorbox}
\textsubscript{19} Гнилий зуб та кульгава нога це надія на зрадливого радника в день твого утиску.
\end{tcolorbox}
\begin{tcolorbox}
\textsubscript{20} Що здіймати одежу холодного дня, що лити оцет на соду, це співати пісні серцю засмученому.
\end{tcolorbox}
\begin{tcolorbox}
\textsubscript{21} Якщо голодує твій ворог нагодуй його хлібом, а як спрагнений він водою напій ти його,
\end{tcolorbox}
\begin{tcolorbox}
\textsubscript{22} бо цим пригортаєш ти жар на його голову, і Господь надолужить тобі!
\end{tcolorbox}
\begin{tcolorbox}
\textsubscript{23} Вітер північний народжує дощ, а таємний язик сердите обличчя.
\end{tcolorbox}
\begin{tcolorbox}
\textsubscript{24} Ліпше жити в куті на даху, ніж з сварливою жінкою в спільному домі.
\end{tcolorbox}
\begin{tcolorbox}
\textsubscript{25} Добра звістка з далекого краю це холодна водиця на спрагнену душу.
\end{tcolorbox}
\begin{tcolorbox}
\textsubscript{26} Джерело скаламучене чи зіпсутий потік це праведний, що схиляється перед безбожним.
\end{tcolorbox}
\begin{tcolorbox}
\textsubscript{27} Їсти меду багато не добре, так досліджувати власну славу неслава.
\end{tcolorbox}
\begin{tcolorbox}
\textsubscript{28} Людина, що стриму немає для духу свого, це зруйноване місто без муру.
\end{tcolorbox}
\subsection{CHAPTER 26}
\begin{tcolorbox}
\textsubscript{1} Як літом той сніг, і як дощ у жнива, так не лицює глупцеві пошана.
\end{tcolorbox}
\begin{tcolorbox}
\textsubscript{2} Як пташка літає, як ластівка лине, так невинне прокляття не сповниться.
\end{tcolorbox}
\begin{tcolorbox}
\textsubscript{3} Батіг на коня, оброть на осла, а різка на спину глупців.
\end{tcolorbox}
\begin{tcolorbox}
\textsubscript{4} Нерозумному відповіді не давай за нерозум його, щоб і ти не став рівний йому.
\end{tcolorbox}
\begin{tcolorbox}
\textsubscript{5} Нерозумному відповідь дай за безумством його, щоб він в очах своїх не став мудрим.
\end{tcolorbox}
\begin{tcolorbox}
\textsubscript{6} Хто через глупця посилає слова, той ноги собі обтинає, отруту він п'є.
\end{tcolorbox}
\begin{tcolorbox}
\textsubscript{7} Як волочаться ноги в кульгавого, так у безумних устах приповістка.
\end{tcolorbox}
\begin{tcolorbox}
\textsubscript{8} Як прив'язувати камінь коштовний до пращі, так глупцеві пошану давати.
\end{tcolorbox}
\begin{tcolorbox}
\textsubscript{9} Як терен, що влізе у руку, отак приповістка в устах нерозумного.
\end{tcolorbox}
\begin{tcolorbox}
\textsubscript{10} Як стрілець, що все ранить, так і той, хто наймає глупця, і наймає усяких прохожих.
\end{tcolorbox}
\begin{tcolorbox}
\textsubscript{11} Як вертається пес до своєї блювотини, так глупоту свою повторяє глупак.
\end{tcolorbox}
\begin{tcolorbox}
\textsubscript{12} Чи ти бачив людину, що мудра в очах своїх? Більша надія глупцеві, ніж їй.
\end{tcolorbox}
\begin{tcolorbox}
\textsubscript{13} Лінивий говорить: Лев на дорозі! Лев на майдані!
\end{tcolorbox}
\begin{tcolorbox}
\textsubscript{14} Двері обертаються на своєму чопі, а лінивий на ліжку своїм.
\end{tcolorbox}
\begin{tcolorbox}
\textsubscript{15} Свою руку лінивий стромляє до миски, та піднести до рота її йому тяжко.
\end{tcolorbox}
\begin{tcolorbox}
\textsubscript{16} Лінивий мудріший ув очах своїх за сімох, що відповідають розумно.
\end{tcolorbox}
\begin{tcolorbox}
\textsubscript{17} Пса за вуха хапає, хто, йдучи, устряває до сварки чужої.
\end{tcolorbox}
\begin{tcolorbox}
\textsubscript{18} Як той, хто вдає божевільного, кидає іскри, стріли та смерть,
\end{tcolorbox}
\begin{tcolorbox}
\textsubscript{19} так і людина, що обманює друга свого та каже: Таж це я жартую!...
\end{tcolorbox}
\begin{tcolorbox}
\textsubscript{20} З браку дров огонь гасне, а без пліткаря мовкне сварка.
\end{tcolorbox}
\begin{tcolorbox}
\textsubscript{21} Вугілля для жару, а дрова огневі, а людина сварлива щоб сварку розпалювати.
\end{tcolorbox}
\begin{tcolorbox}
\textsubscript{22} Слова обмовника мов ті присмаки, й у нутро живота вони сходять.
\end{tcolorbox}
\begin{tcolorbox}
\textsubscript{23} Як срібло з жужелицею, на горшкові накладене, так полум'яні уста, а серце лихе,
\end{tcolorbox}
\begin{tcolorbox}
\textsubscript{24} устами своїми маскується ворог, і ховає оману в своєму нутрі:
\end{tcolorbox}
\begin{tcolorbox}
\textsubscript{25} коли він говорить лагідно не вір ти йому, бо в серці його сім огид!
\end{tcolorbox}
\begin{tcolorbox}
\textsubscript{26} Як ненависть прикрита оманою, її зло відкривається в зборі.
\end{tcolorbox}
\begin{tcolorbox}
\textsubscript{27} Хто яму копає, той в неї впаде, а хто котить каміння на нього воно повертається.
\end{tcolorbox}
\begin{tcolorbox}
\textsubscript{28} Брехливий язик ненавидить своїх утискуваних, і уста гладенькі до згуби провадять.
\end{tcolorbox}
\subsection{CHAPTER 27}
\begin{tcolorbox}
\textsubscript{1} Не вихвалюйся завтрішнім днем, бо не знаєш, що день той породить.
\end{tcolorbox}
\begin{tcolorbox}
\textsubscript{2} Нехай інший тебе вихваляє, а не уста твої, чужий, а не губи твої.
\end{tcolorbox}
\begin{tcolorbox}
\textsubscript{3} Каміння тягар, і пісок важка річ, та гнів нерозумного тяжчий від них від обох.
\end{tcolorbox}
\begin{tcolorbox}
\textsubscript{4} Лютість жорстокість, а гнів то затоплення, та хто перед заздрістю встоїть?
\end{tcolorbox}
\begin{tcolorbox}
\textsubscript{5} Ліпше відкрите картання, ніж таємна любов.
\end{tcolorbox}
\begin{tcolorbox}
\textsubscript{6} Побої коханого вірність показують, а в ненависника поцілунки численні.
\end{tcolorbox}
\begin{tcolorbox}
\textsubscript{7} Сита душа топче й мед щільниковий, а голодній душі все гірке то солодке.
\end{tcolorbox}
\begin{tcolorbox}
\textsubscript{8} Як птах, що гніздо своє кинув, так і людина, що з місця свого мандрує.
\end{tcolorbox}
\begin{tcolorbox}
\textsubscript{9} Олива й кадило потішують серце, і солодкий нам друг за душевну пораду.
\end{tcolorbox}
\begin{tcolorbox}
\textsubscript{10} Друга свого й друга батька свого не кидай, а в дім брата свого не приходь в день нещастя свого, ліпший сусіда близький за далекого брата!
\end{tcolorbox}
\begin{tcolorbox}
\textsubscript{11} Будь мудрий, мій сину, й потіш моє серце, і я матиму що відповісти, як мені докорятиме хто.
\end{tcolorbox}
\begin{tcolorbox}
\textsubscript{12} Мудрий бачить лихе і ховається, а безумні йдуть і караються.
\end{tcolorbox}
\begin{tcolorbox}
\textsubscript{13} Візьми його одіж, бо він поручивсь за чужого, і за чужинку заставу візьми.
\end{tcolorbox}
\begin{tcolorbox}
\textsubscript{14} Хто сильним голосом благословляє із раннього ранку свого товариша, за прокляття залічується це йому.
\end{tcolorbox}
\begin{tcolorbox}
\textsubscript{15} Ринва, постійно текуча слотливого дня та жінка сварлива однакове:
\end{tcolorbox}
\begin{tcolorbox}
\textsubscript{16} хто хоче сховати її той вітра ховає, чи оливу пахучу правиці своєї, що видасть себе.
\end{tcolorbox}
\begin{tcolorbox}
\textsubscript{17} Як гострить залізо залізо, так гострить людина лице свого друга.
\end{tcolorbox}
\begin{tcolorbox}
\textsubscript{18} Сторож фіґовниці плоди її споживає, а хто пана свого стереже, той шанований.
\end{tcolorbox}
\begin{tcolorbox}
\textsubscript{19} Як лице до лиця у воді, так серце людини до серця людини.
\end{tcolorbox}
\begin{tcolorbox}
\textsubscript{20} Шеол й Аваддон не наситяться, не наситяться й очі людини.
\end{tcolorbox}
\begin{tcolorbox}
\textsubscript{21} Що для срібла топильна посудина, і горно для золота, те для людини уста, які хвалять її.
\end{tcolorbox}
\begin{tcolorbox}
\textsubscript{22} Хоч нерозумного будеш товкти товкачем поміж зернами в ступі, не відійде від нього глупота його!
\end{tcolorbox}
\begin{tcolorbox}
\textsubscript{23} Добре знай вигляд своєї отари, поклади своє серце на череди,
\end{tcolorbox}
\begin{tcolorbox}
\textsubscript{24} бо багатство твоє не навіки, і чи корона твоя з роду в рід?
\end{tcolorbox}
\begin{tcolorbox}
\textsubscript{25} Появилася зелень, і трава показалась, і збирається сіно із гір,
\end{tcolorbox}
\begin{tcolorbox}
\textsubscript{26} будуть вівці тобі на вбрання, і козли ціна поля,
\end{tcolorbox}
\begin{tcolorbox}
\textsubscript{27} і молока твоїх кіз буде досить на їжу тобі, на їду твого дому, і на життя для служниць твоїх.
\end{tcolorbox}
\subsection{CHAPTER 28}
\begin{tcolorbox}
\textsubscript{1} Безбожні втікають, коли й не женуться за ними, а справедливий безпечний, немов той левчук.
\end{tcolorbox}
\begin{tcolorbox}
\textsubscript{2} Коли край провиниться, то має багато володарів, коли ж є людина розумна й знаюча, то держиться довго.
\end{tcolorbox}
\begin{tcolorbox}
\textsubscript{3} Людина убога, що гнобить нужденних, це злива рвучка, що хліба по ній не буває.
\end{tcolorbox}
\begin{tcolorbox}
\textsubscript{4} Ті, хто Закон залишає, хвалять безбожних, а ті, хто Закон береже, на них буряться.
\end{tcolorbox}
\begin{tcolorbox}
\textsubscript{5} Люди лихі правосуддя не розуміють, а шукаючі Господа все розуміють.
\end{tcolorbox}
\begin{tcolorbox}
\textsubscript{6} Ліпше убогий, що ходить в своїй неповинності, ніж криводорогий, хоч він і багач.
\end{tcolorbox}
\begin{tcolorbox}
\textsubscript{7} Хто Закон береже, розумний той син, а хто водиться із гультяями, засоромлює батька свого.
\end{tcolorbox}
\begin{tcolorbox}
\textsubscript{8} Хто множить лихварським відсотком багатство своє, той для того громадить його, хто ласкавий для бідних.
\end{tcolorbox}
\begin{tcolorbox}
\textsubscript{9} Хто відхилює вухо своє, щоб не слухати Закона, то буде огидна й молитва того.
\end{tcolorbox}
\begin{tcolorbox}
\textsubscript{10} Хто простих доводить блудити дорогою зла, сам до ями своєї впаде, а невинні посядуть добро.
\end{tcolorbox}
\begin{tcolorbox}
\textsubscript{11} Багата людина в очах своїх мудра, та розумний убогий розслідить її.
\end{tcolorbox}
\begin{tcolorbox}
\textsubscript{12} Велика пишнота, як тішаться праведні, коли ж несправедливі зростають, то треба шукати людину.
\end{tcolorbox}
\begin{tcolorbox}
\textsubscript{13} Хто ховає провини свої, тому не ведеться, а хто признається та кидає їх, той буде помилуваний.
\end{tcolorbox}
\begin{tcolorbox}
\textsubscript{14} Блаженна людина, що завжди обачна, а хто ожорсточує серце своє, той впадає в лихе.
\end{tcolorbox}
\begin{tcolorbox}
\textsubscript{15} Лев ричучий й ведмідь ненажерливий це безбожний володар над людом убогим.
\end{tcolorbox}
\begin{tcolorbox}
\textsubscript{16} Володар, позбавлений розуму, тисне дошкульно, а ненависник зажерливости буде мати дні довгі.
\end{tcolorbox}
\begin{tcolorbox}
\textsubscript{17} Людина, обтяжена за душогубство, втікає до гробу, нехай її не підпирають!
\end{tcolorbox}
\begin{tcolorbox}
\textsubscript{18} Хто ходить невинний, той буде спасений, а криводорогий впаде на одній із доріг.
\end{tcolorbox}
\begin{tcolorbox}
\textsubscript{19} Хто землю свою обробляє, той насититься хлібом, а хто за марнотним женеться, насититься вбогістю.
\end{tcolorbox}
\begin{tcolorbox}
\textsubscript{20} Вірна людина багата на благословення, а хто спішно збагачується, непокараним той не залишиться.
\end{tcolorbox}
\begin{tcolorbox}
\textsubscript{21} Увагу звертати на особу не добре, бо й за кус хліба людина згрішить.
\end{tcolorbox}
\begin{tcolorbox}
\textsubscript{22} Завидюща людина спішить до багатства, і не знає, що прийде на неї нужда.
\end{tcolorbox}
\begin{tcolorbox}
\textsubscript{23} Хто напоумляє людину, той знаходить вкінці більшу ласку, ніж той, хто лестить язиком.
\end{tcolorbox}
\begin{tcolorbox}
\textsubscript{24} Хто батька свого й свою матір грабує і каже: Це не гріх, той розбійнику друг.
\end{tcolorbox}
\begin{tcolorbox}
\textsubscript{25} Захланний викликує сварку, хто ж має надію на Господа, буде насичений.
\end{tcolorbox}
\begin{tcolorbox}
\textsubscript{26} Хто надію кладе на свій розум, то він нерозумний, а хто мудрістю ходить, той буде врятований.
\end{tcolorbox}
\begin{tcolorbox}
\textsubscript{27} Хто дає немаючому, той недостатку не знатиме, хто ж свої очі ховає від нього, той зазнає багато проклять.
\end{tcolorbox}
\begin{tcolorbox}
\textsubscript{28} Коли підіймаються люди безбожні, людина ховається, а як гинуть вони, то множаться праведні.
\end{tcolorbox}
\subsection{CHAPTER 29}
\begin{tcolorbox}
\textsubscript{1} Чоловік остережуваний, та твердошиїй, буде зламаний нагло, і ліку не буде йому.
\end{tcolorbox}
\begin{tcolorbox}
\textsubscript{2} Коли множаться праведні, радіє народ, як панує ж безбожний то стогне народ.
\end{tcolorbox}
\begin{tcolorbox}
\textsubscript{3} Людина, що мудрість кохає, потішує батька свого, а хто попасає блудниць, той губить маєток.
\end{tcolorbox}
\begin{tcolorbox}
\textsubscript{4} Цар утримує край правосуддям, а людина хабарна руйнує його.
\end{tcolorbox}
\begin{tcolorbox}
\textsubscript{5} Людина, що другові своєму підлещує, на стопах його пастку ставить.
\end{tcolorbox}
\begin{tcolorbox}
\textsubscript{6} У провині людини лихої знаходиться пастка, а справедливий радіє та тішиться.
\end{tcolorbox}
\begin{tcolorbox}
\textsubscript{7} Праведний знає про право вбогих, безбожний же не розуміє пізнання про це.
\end{tcolorbox}
\begin{tcolorbox}
\textsubscript{8} Люди глузливі підбурюють місто, а мудрі утишують гнів.
\end{tcolorbox}
\begin{tcolorbox}
\textsubscript{9} Мудра людина, що правується із нерозумним, то чи гнівається, чи сміється, спокою не знає.
\end{tcolorbox}
\begin{tcolorbox}
\textsubscript{10} Кровожерці ненавидять праведного, справедливі ж шукають спасти його душу.
\end{tcolorbox}
\begin{tcolorbox}
\textsubscript{11} Глупак увесь свій гнів увиявляє, а мудрий назад його стримує.
\end{tcolorbox}
\begin{tcolorbox}
\textsubscript{12} Володар, що слухає слова брехливого, безбожні всі слуги його!
\end{tcolorbox}
\begin{tcolorbox}
\textsubscript{13} Убогий й гнобитель стрічаються, їм обом Господь очі освітлює.
\end{tcolorbox}
\begin{tcolorbox}
\textsubscript{14} Як цар правдою судить убогих, стоятиме трон його завжди.
\end{tcolorbox}
\begin{tcolorbox}
\textsubscript{15} Різка й поука премудрість дають, а дитина, залишена тільки собі, засоромлює матір свою.
\end{tcolorbox}
\begin{tcolorbox}
\textsubscript{16} Як множаться несправедливі провина розмножується, але праведні бачитимуть їхній упадок.
\end{tcolorbox}
\begin{tcolorbox}
\textsubscript{17} Карай сина свого й він тебе заспокоїть, і приємнощі дасть для твоєї душі.
\end{tcolorbox}
\begin{tcolorbox}
\textsubscript{18} Без пророчих видінь люд розбещений, коли ж стереже він Закона блаженний.
\end{tcolorbox}
\begin{tcolorbox}
\textsubscript{19} Раб словами не буде покараний, хоч він розуміє, але не послухає.
\end{tcolorbox}
\begin{tcolorbox}
\textsubscript{20} Чи бачив людину, квапливу в словах своїх? Більша надія глупцеві, ніж їй!
\end{tcolorbox}
\begin{tcolorbox}
\textsubscript{21} Хто розпещує змалку свого раба, то кінець його буде невдячний.
\end{tcolorbox}
\begin{tcolorbox}
\textsubscript{22} Гнівлива людина викликує сварку, а лютий вчиняє багато провин.
\end{tcolorbox}
\begin{tcolorbox}
\textsubscript{23} Гординя людини її понижає, а чести набуває покірливий духом.
\end{tcolorbox}
\begin{tcolorbox}
\textsubscript{24} Хто ділиться з злодієм, той ненавидить душу свою, він чує прокляття, та не виявляє.
\end{tcolorbox}
\begin{tcolorbox}
\textsubscript{25} Страх перед людиною пастку дає, хто ж надію складає на Господа, буде безпечний.
\end{tcolorbox}
\begin{tcolorbox}
\textsubscript{26} Багато шукають для себе обличчя володаря, та від Господа суд для людини.
\end{tcolorbox}
\begin{tcolorbox}
\textsubscript{27} Насильник огида для праведних, а простодорогий огида безбожному.
\end{tcolorbox}
\subsection{CHAPTER 30}
\begin{tcolorbox}
\textsubscript{1} Слова Агура, Якеєвого сина, массеянина: Слово мужчини: Трудився я, Боже, трудився я, Боже, і змучився я!
\end{tcolorbox}
\begin{tcolorbox}
\textsubscript{2} Бо думаю, що немудріший за кожного я, і не маю я людського розуму,
\end{tcolorbox}
\begin{tcolorbox}
\textsubscript{3} і не навчився я мудрости, і не знаю пізнання святих...
\end{tcolorbox}
\begin{tcolorbox}
\textsubscript{4} Хто на небо ввійшов і зійшов? Хто у жмені свої зібрав вітер? Хто воду в одежу зв'язав? Хто поставив усі кінці землі? Яке ймення його, і яке ймення сина його, коли знаєш?
\end{tcolorbox}
\begin{tcolorbox}
\textsubscript{5} Кожне Боже слово очищене, щит Він для тих, хто в Нім пристановище має.
\end{tcolorbox}
\begin{tcolorbox}
\textsubscript{6} До слів Його не додавай, щоб тебе не скартав Він, і щоб неправдомовцем не став ти.
\end{tcolorbox}
\begin{tcolorbox}
\textsubscript{7} Двох речей я від Тебе просив, не відмов мені, поки помру:
\end{tcolorbox}
\begin{tcolorbox}
\textsubscript{8} віддали Ти від мене марноту та слово брехливе, убозтва й багатства мені не давай! Годуй мене хлібом, для мене призначеним,
\end{tcolorbox}
\begin{tcolorbox}
\textsubscript{9} щоб я не переситився та й не відрікся, і не сказав: Хто Господь? і щоб я не збіднів і не крав, і не зневажив Ім'я мого Бога.
\end{tcolorbox}
\begin{tcolorbox}
\textsubscript{10} Раба не обмовляй перед паном його, щоб тебе не прокляв він, і ти винуватим не став.
\end{tcolorbox}
\begin{tcolorbox}
\textsubscript{11} Оце покоління, що батька свого проклинає, і неньки своєї не благословляє,
\end{tcolorbox}
\begin{tcolorbox}
\textsubscript{12} покоління, що чисте в очах своїх, та від бруду свого не обмите,
\end{tcolorbox}
\begin{tcolorbox}
\textsubscript{13} покоління, які гордісні очі його, а повіки його як піднеслися!
\end{tcolorbox}
\begin{tcolorbox}
\textsubscript{14} Покоління, що в нього мечі його зуби, а гострі ножі його щелепи, щоб пожерти убогих із краю й нужденних з землі!
\end{tcolorbox}
\begin{tcolorbox}
\textsubscript{15} Дві дочки в кровожерця: Дай, дай! Оці три не наситяться, чотири не скажуть досить:
\end{tcolorbox}
\begin{tcolorbox}
\textsubscript{16} шеол та утроба неплідна, водою земля не насититься, і не скаже досить огонь!
\end{tcolorbox}
\begin{tcolorbox}
\textsubscript{17} Око, що з батька сміється й погорджує послухом матері, нехай видзьобають його круки поточні, і нехай орленята його пожеруть!
\end{tcolorbox}
\begin{tcolorbox}
\textsubscript{18} Три речі оці дивовижні для мене, і чотири, яких я не знаю:
\end{tcolorbox}
\begin{tcolorbox}
\textsubscript{19} дорога орлина в повітрі, дорога зміїна на скелі, корабельна дорога в середині моря, і дорога мужчини при дівчині!...
\end{tcolorbox}
\begin{tcolorbox}
\textsubscript{20} Така ось дорога блудливої жінки: наїлась та витерла уста свої й повіла: Не вчинила я злого!...
\end{tcolorbox}
\begin{tcolorbox}
\textsubscript{21} Трясеться земля під трьома, і під чотирма, яких знести не може вона:
\end{tcolorbox}
\begin{tcolorbox}
\textsubscript{22} під рабом, коли він зацарює, і під нерозумним, як хліба наїсться,
\end{tcolorbox}
\begin{tcolorbox}
\textsubscript{23} під розпустницею, коли взята за жінку, і невільницею, коли вижене пані свою!...
\end{tcolorbox}
\begin{tcolorbox}
\textsubscript{24} Оці ось чотири малі на землі, та вони вельми мудрі:
\end{tcolorbox}
\begin{tcolorbox}
\textsubscript{25} мурашки, не сильний народ, та поживу свою заготовлюють літом;
\end{tcolorbox}
\begin{tcolorbox}
\textsubscript{26} борсуки, люд не сильний, та в скелі свій дім вони ставлять;
\end{tcolorbox}
\begin{tcolorbox}
\textsubscript{27} немає царя в сарани, але вся вона в строї бойовім виходить;
\end{tcolorbox}
\begin{tcolorbox}
\textsubscript{28} павук тільки лапками пнеться, та він і в палатах царських!
\end{tcolorbox}
\begin{tcolorbox}
\textsubscript{29} Добре ступають ці троє, і добре ходять чотири:
\end{tcolorbox}
\begin{tcolorbox}
\textsubscript{30} лев, найсильніший поміж звіриною, який не вступається ні перед ким,
\end{tcolorbox}
\begin{tcolorbox}
\textsubscript{31} осідланий кінь, і козел, та той цар, що з ним військо!
\end{tcolorbox}
\begin{tcolorbox}
\textsubscript{32} Якщо ти допустився глупоти пихою, й якщо заміряєш лихе, то руку на уста!
\end{tcolorbox}
\begin{tcolorbox}
\textsubscript{33} Бо збивання молока дає масло, і дає кров вдар по носі, тиск же на гнів дає сварку.
\end{tcolorbox}
\subsection{CHAPTER 31}
\begin{tcolorbox}
\textsubscript{1} Слова Лемуїла, царя Масси, що ними навчала його його мати:
\end{tcolorbox}
\begin{tcolorbox}
\textsubscript{2} Що, сину мій, і що, сину утроби моєї, і що, сину обітниць моїх?
\end{tcolorbox}
\begin{tcolorbox}
\textsubscript{3} Не давай жінкам сили своєї, ні доріг своїх для руйнувальниць царів!
\end{tcolorbox}
\begin{tcolorbox}
\textsubscript{4} Не царям, Лемуїле, вино, не царям, і напій той п'янкий не князям,
\end{tcolorbox}
\begin{tcolorbox}
\textsubscript{5} щоб не впився він та не забув про Закона, і щоб не змінив для всіх гноблених права!
\end{tcolorbox}
\begin{tcolorbox}
\textsubscript{6} Дайте напою п'янкого тому, хто гине, а вина гіркодухим:
\end{tcolorbox}
\begin{tcolorbox}
\textsubscript{7} він вип'є й забуде за бідність свою, і муки своєї вже не пам'ятатиме!
\end{tcolorbox}
\begin{tcolorbox}
\textsubscript{8} Відкривай свої уста немові, для суда всім нещасним.
\end{tcolorbox}
\begin{tcolorbox}
\textsubscript{9} Відкривай свої уста, й суди справедливо, і правосуддя зроби для убогого та для нужденного.
\end{tcolorbox}
\begin{tcolorbox}
\textsubscript{10} Хто жінку чеснотну знайде? а ціна її більша від перел:
\end{tcolorbox}
\begin{tcolorbox}
\textsubscript{11} довіряє їй серце її чоловіка, і йому не забракне прибутку!
\end{tcolorbox}
\begin{tcolorbox}
\textsubscript{12} Вона чинить для нього добро, а не зло, по всі дні свого життя.
\end{tcolorbox}
\begin{tcolorbox}
\textsubscript{13} Шукає вона вовни й льону, і робить охоче своїми руками.
\end{tcolorbox}
\begin{tcolorbox}
\textsubscript{14} Вона, немов кораблі ті купецькі, здалека спроваджує хліб свій.
\end{tcolorbox}
\begin{tcolorbox}
\textsubscript{15} І встане вона ще вночі, і видасть для дому свого поживу, а порядок служницям своїм.
\end{tcolorbox}
\begin{tcolorbox}
\textsubscript{16} Про поле вона намишляла, і його набула, із плоду долоней своїх засадила вона виноградника.
\end{tcolorbox}
\begin{tcolorbox}
\textsubscript{17} Вона підперізує силою стегна свої та зміцняє рамена свої.
\end{tcolorbox}
\begin{tcolorbox}
\textsubscript{18} Вона розуміє, що добра робота її, і світильник її не погасне вночі.
\end{tcolorbox}
\begin{tcolorbox}
\textsubscript{19} Вона руки свої простягає до прядки, а долоні її веретено тримають.
\end{tcolorbox}
\begin{tcolorbox}
\textsubscript{20} Долоню свою відкриває для вбогого, а руки свої простягає до бідного.
\end{tcolorbox}
\begin{tcolorbox}
\textsubscript{21} Холоду в домі своїм не боїться вона, бо подвійно одягнений ввесь її дім.
\end{tcolorbox}
\begin{tcolorbox}
\textsubscript{22} Килими поробила собі, віссон та кармазин убрання її.
\end{tcolorbox}
\begin{tcolorbox}
\textsubscript{23} Чоловік її знаний при брамах, як сидить він із старшими краю.
\end{tcolorbox}
\begin{tcolorbox}
\textsubscript{24} Тонку туніку робить вона й продає, і купцеві дає пояси.
\end{tcolorbox}
\begin{tcolorbox}
\textsubscript{25} Сила та пишність одежа її, і сміється вона до прийдещнього дня.
\end{tcolorbox}
\begin{tcolorbox}
\textsubscript{26} Свої уста вона відкриває на мудрість, і милостива наука їй на язиці.
\end{tcolorbox}
\begin{tcolorbox}
\textsubscript{27} Доглядає вона ходи дому свого, і хліба з лінивства не їсть.
\end{tcolorbox}
\begin{tcolorbox}
\textsubscript{28} Устають її діти, і хвалять її, чоловік її й він похваляє її:
\end{tcolorbox}
\begin{tcolorbox}
\textsubscript{29} Багато було тих чеснотних дочок, та ти їх усіх перевищила!
\end{tcolorbox}
\begin{tcolorbox}
\textsubscript{30} Краса то омана, а врода марнота, жінка ж богобоязна вона буде хвалена!
\end{tcolorbox}
\begin{tcolorbox}
\textsubscript{31} Дайте їй з плоду рук її, і нехай її вчинки її вихваляють при брамах!
\end{tcolorbox}
