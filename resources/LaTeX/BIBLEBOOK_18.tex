Book 17
\subsection{CHAPTER 1}
\begin{tcolorbox}
\textsubscript{1} Був чоловік у країні Уц, на ім'я йому Йов. І був чоловік цей невинний та праведний, і він Бога боявся, а від злого втікав.
\end{tcolorbox}
\begin{tcolorbox}
\textsubscript{2} І народилися йому семеро синів та три дочки.
\end{tcolorbox}
\begin{tcolorbox}
\textsubscript{3} А маєток його був: сім тисяч худоби дрібної, і три тисячі верблюдів, і п'ять сотень пар худоби великої, і п'ять сотень ослиць та дуже багато рабів. І був цей чоловік більший від усіх синів сходу.
\end{tcolorbox}
\begin{tcolorbox}
\textsubscript{4} А сини його ходили один до одного, і справляли гостину в домі того, чий був день. І посилали вони, і кликали трьох своїх сестер, щоб їсти та пити із ними.
\end{tcolorbox}
\begin{tcolorbox}
\textsubscript{5} І бувало, як миналося коло бенкетних днів, то Йов посилав за дітьми й освячував їх, і вставав він рано вранці, і приносив цілопалення за числом їх усіх, бо Йов казав: Може згрішили сини мої, і зневажили Бога в серці своєму. Так робив Йов по всі дні.
\end{tcolorbox}
\begin{tcolorbox}
\textsubscript{6} І сталося одного дня, і поприходили Божі сини, щоб стати при Господі. І прийшов поміж ними й сатана.
\end{tcolorbox}
\begin{tcolorbox}
\textsubscript{7} І сказав Господь до сатани: Звідки ти йдеш? А сатана відповів Господеві й сказав: Я мандрував по землі та й перейшов її.
\end{tcolorbox}
\begin{tcolorbox}
\textsubscript{8} І сказав Господь до сатани: Чи звернув ти увагу на раба Мого Йова? Бо немає такого, як він, на землі: муж він невинний та праведний, що Бога боїться, а від злого втікає.
\end{tcolorbox}
\begin{tcolorbox}
\textsubscript{9} І відповів сатана Господеві й сказав: Чи ж Йов дармо боїться Бога?
\end{tcolorbox}
\begin{tcolorbox}
\textsubscript{10} Чи ж Ти не забезпечив його, і дім його, і все, що його? Чин його рук Ти поблагословив, а маєток його поширився по краю.
\end{tcolorbox}
\begin{tcolorbox}
\textsubscript{11} Але простягни тільки руку Свою, і доторкнися до всього, що його, чи він не зневажить Тебе перед лицем Твоїм?
\end{tcolorbox}
\begin{tcolorbox}
\textsubscript{12} І сказав Господь до сатани: Ось усе, що його, у твоїй руці, тільки на нього самого не простягай своєї руки! І пішов сатана від лиця Господнього.
\end{tcolorbox}
\begin{tcolorbox}
\textsubscript{13} І сталося одного дня, коли сини його та дочки його їли та вино пили в домі свого первородженого брата,
\end{tcolorbox}
\begin{tcolorbox}
\textsubscript{14} то прибіг до Йова посланець та й сказав: Худоба велика орала, а ослиці паслися при них.
\end{tcolorbox}
\begin{tcolorbox}
\textsubscript{15} Аж тут напали сабеї й позабирали їх, а слуг повбивали вістрям меча. І втік тільки я сам, щоб донести тобі...
\end{tcolorbox}
\begin{tcolorbox}
\textsubscript{16} Він ще говорив, аж прибігає інший та й каже: З неба спав Божий огонь, і спалив отару та слуг, та й пожер їх... А втік тільки я сам, щоб донести тобі...
\end{tcolorbox}
\begin{tcolorbox}
\textsubscript{17} Він ще говорив, аж біжить ще інший та й каже: Халдеї поділилися на три відділи, і напали на верблюдів, та й позабирали їх, а слуг повбивали вістрям меча... І втік тільки я сам, щоб донести тобі...
\end{tcolorbox}
\begin{tcolorbox}
\textsubscript{18} Поки він говорив, аж надбігає ще інший та й каже: Сини твої та дочки твої їли та вино пили в домі свого первородженого брата.
\end{tcolorbox}
\begin{tcolorbox}
\textsubscript{19} Аж раптово надійшов великий вітер з боку пустині, та й ударив на чотири роги дому, і він упав на юнаків, і вони повмирали... І втік тільки я сам, щоб донести тобі...
\end{tcolorbox}
\begin{tcolorbox}
\textsubscript{20} І встав Йов, і роздер плаща свого, й обстриг свою голову, та й упав на землю, і поклонився,
\end{tcolorbox}
\begin{tcolorbox}
\textsubscript{21} та й сказав: Я вийшов нагий із утроби матері своєї, і нагий повернусь туди, в землю! Господь дав, і Господь узяв... Нехай буде благословенне Господнє Ім'я!
\end{tcolorbox}
\begin{tcolorbox}
\textsubscript{22} При всьому цьому Йов не згрішив, і не сказав на Бога нічого безумного!
\end{tcolorbox}
\subsection{CHAPTER 2}
\begin{tcolorbox}
\textsubscript{1} І сталося одного дня, і поприходили Божі сини, щоб стати перед Господом: і прийшов також сатана поміж ними, щоб стати перед Господом.
\end{tcolorbox}
\begin{tcolorbox}
\textsubscript{2} І сказав Господь до сатани: Звідки ти йдеш? А сатана відповів Господеві й сказав:
\end{tcolorbox}
\begin{tcolorbox}
\textsubscript{3} І сказав Господь до сатани: Чи звернув ти увагу на раба Мого Йова? Бо немає такого, як він, на землі: муж він невинний та праведний, який Бога боїться, а від злого втікає. І він ще тримається міцно в своїй невинності, а ти намовляв був Мене на нього, щоб без приводу його зруйнувати...
\end{tcolorbox}
\begin{tcolorbox}
\textsubscript{4} І відповів сатана Господеві й сказав: Шкіра за шкіру, і все, що хто має, віддасть він за душу свою.
\end{tcolorbox}
\begin{tcolorbox}
\textsubscript{5} Але простягни но Ти руку Свою, і доторкнись до костей його та до тіла його, чи він не зневажить Тебе перед лицем Твоїм?
\end{tcolorbox}
\begin{tcolorbox}
\textsubscript{6} І сказав Господь до сатани: Ось він у руці твоїй, тільки душу його бережи!
\end{tcolorbox}
\begin{tcolorbox}
\textsubscript{7} І вийшов сатана від лиця Господнього, та й ударив Йова злим гнояком від стопи ноги його аж до його черепа...
\end{tcolorbox}
\begin{tcolorbox}
\textsubscript{8} А той узяв собі черепка, щоб шкребти себе. І він сидів серед попелу...
\end{tcolorbox}
\begin{tcolorbox}
\textsubscript{9} І сказала йому його жінка: Ти ще міцно тримаєшся в невинності своїй? Прокляни Бога і помреш!...
\end{tcolorbox}
\begin{tcolorbox}
\textsubscript{10} А він до неї відказав: Ти говориш отак, як говорить яка з божевільних!... Чи ж ми будем приймати від Бога добре, а злого не приймем? При всьому тому Йов не згрішив своїми устами...
\end{tcolorbox}
\begin{tcolorbox}
\textsubscript{11} І почули троє приятелів Йовових про все те нещастя, що прийшло на нього, і поприходили кожен з місця свого: теманянин Еліфаз, шух'янин Біддад та нааматянин Цофар. І вмовилися вони прийти разом, щоб похитати головою над ним та потішити його.
\end{tcolorbox}
\begin{tcolorbox}
\textsubscript{12} І звели вони здалека очі свої, і не пізнали його... І піднесли вони голос свій, та й заголосили, і роздерли кожен одежу свою, і кидали порох над своїми головами аж до неба...
\end{tcolorbox}
\begin{tcolorbox}
\textsubscript{13} І сиділи вони з ним на землі сім день та сім ночей, і ніхто не промовив до нього ні слова, бо вони бачили, що біль його вельми великий...
\end{tcolorbox}
\subsection{CHAPTER 3}
\begin{tcolorbox}
\textsubscript{1} По цьому відкрив Йов уста свої та й прокляв був свій день народження.
\end{tcolorbox}
\begin{tcolorbox}
\textsubscript{2} І Йов заговорив та й сказав:
\end{tcolorbox}
\begin{tcolorbox}
\textsubscript{3} Хай загине той день, що я в ньому родився, і та ніч, що сказала: Зачавсь чоловік!
\end{tcolorbox}
\begin{tcolorbox}
\textsubscript{4} Нехай стане цей день темнотою, нехай Бог з висоти не згадає його, і нехай не являється світло над ним!...
\end{tcolorbox}
\begin{tcolorbox}
\textsubscript{5} Бодай темрява й морок його заступили, бодай хмара над ним пробувала, бодай темнощі денні лякали його!...
\end{tcolorbox}
\begin{tcolorbox}
\textsubscript{6} Оця ніч бодай темність її обгорнула, нехай у днях року не буде названа вона, хай не ввійде вона в число місяців!...
\end{tcolorbox}
\begin{tcolorbox}
\textsubscript{7} Тож ця ніч нехай буде самітна, хай не прийде до неї співання!
\end{tcolorbox}
\begin{tcolorbox}
\textsubscript{8} Бодай її ті проклинали, що день проклинають, що левіятана готові збудити!
\end{tcolorbox}
\begin{tcolorbox}
\textsubscript{9} Хай потемніють зорі поранку її, нехай має надію на світло й не буде його, і хай вона не побачить тремтячих повік зорі ранньої,
\end{tcolorbox}
\begin{tcolorbox}
\textsubscript{10} бо вона не замкнула дверей нутра матернього, і не сховала страждання з очей моїх!...
\end{tcolorbox}
\begin{tcolorbox}
\textsubscript{11} Чому я не згинув в утробі? Як вийшов, із нутра то чому я не вмер?
\end{tcolorbox}
\begin{tcolorbox}
\textsubscript{12} Чого прийняли ті коліна мене? І нащо ті перса, які я був ссав?
\end{tcolorbox}
\begin{tcolorbox}
\textsubscript{13} Бо тепер я лежав би спокійно, я спав би, та був би мені відпочинок
\end{tcolorbox}
\begin{tcolorbox}
\textsubscript{14} з царями та з земними радниками, що гробниці будують собі,
\end{tcolorbox}
\begin{tcolorbox}
\textsubscript{15} або із князями, що золото мали, що доми свої сріблом наповнювали!...
\end{tcolorbox}
\begin{tcolorbox}
\textsubscript{16} Або чом я не ставсь недоноском прихованим, немов ті немовлята, що світла не бачили?
\end{tcolorbox}
\begin{tcolorbox}
\textsubscript{17} Там же безбожники перестають докучати, і спочивають там змученосилі,
\end{tcolorbox}
\begin{tcolorbox}
\textsubscript{18} разом з тим мають спокій ув'язнені, вони не почують вже крику гнобителя!...
\end{tcolorbox}
\begin{tcolorbox}
\textsubscript{19} Малий та великий там рівні, а раб вільний від пана свого...
\end{tcolorbox}
\begin{tcolorbox}
\textsubscript{20} І нащо Він струдженому дає світло, і життя гіркодухим,
\end{tcolorbox}
\begin{tcolorbox}
\textsubscript{21} що вичікують смерти й немає її, що її відкопали б, як скарби заховані,
\end{tcolorbox}
\begin{tcolorbox}
\textsubscript{22} тим, що радісно тішилися б, веселились, коли б знайшли гроба,
\end{tcolorbox}
\begin{tcolorbox}
\textsubscript{23} мужчині, якому дорога закрита, що Бог тінню закрив перед ним?...
\end{tcolorbox}
\begin{tcolorbox}
\textsubscript{24} Бо зідхання моє випереджує хліб мій, а зойки мої полились, як вода,
\end{tcolorbox}
\begin{tcolorbox}
\textsubscript{25} бо страх, що його я жахався, до мене прибув, і чого я боявся прийшло те мені...
\end{tcolorbox}
\begin{tcolorbox}
\textsubscript{26} Не знав я спокою й не був втихомирений, і я не відпочив, та нещастя прийшло!...
\end{tcolorbox}
\subsection{CHAPTER 4}
\begin{tcolorbox}
\textsubscript{1} І відповів теманянин Еліфаз та й сказав:
\end{tcolorbox}
\begin{tcolorbox}
\textsubscript{2} Коли спробувать слово до тебе, чи мука не буде ще більша? Та хто стримати зможе слова?
\end{tcolorbox}
\begin{tcolorbox}
\textsubscript{3} Таж ти багатьох був навчав, а руки ослаблі зміцняв,
\end{tcolorbox}
\begin{tcolorbox}
\textsubscript{4} того, хто спотикавсь, підіймали слова твої, а коліна тремткі ти зміцняв!
\end{tcolorbox}
\begin{tcolorbox}
\textsubscript{5} А тепер, як нещастя на тебе найшло, то ти змучився, тебе досягло воно і ти налякався...
\end{tcolorbox}
\begin{tcolorbox}
\textsubscript{6} Хіба не була богобійність твоя за надію твою, за твоє сподівання невинність доріг твоїх?
\end{tcolorbox}
\begin{tcolorbox}
\textsubscript{7} Пригадай но, чи гинув невинний, і де праведні вигублені?
\end{tcolorbox}
\begin{tcolorbox}
\textsubscript{8} Як я бачив таких, що орали були беззаконня, та сіяли кривду, то й жали її:
\end{tcolorbox}
\begin{tcolorbox}
\textsubscript{9} вони гинуть від подиху Божого, і від духу гнівного Його погибають!
\end{tcolorbox}
\begin{tcolorbox}
\textsubscript{10} Левине ричання й рик лютого лева минає, і левчукам вилущаються зуби.
\end{tcolorbox}
\begin{tcolorbox}
\textsubscript{11} Гине лев, як немає здобичі, і левенята левиці втікають.
\end{tcolorbox}
\begin{tcolorbox}
\textsubscript{12} І закрадається слово до мене, і моє ухо почуло ось дещо від нього.
\end{tcolorbox}
\begin{tcolorbox}
\textsubscript{13} у роздумуваннях над нічними видіннями, коли міцний сон обіймає людей,
\end{tcolorbox}
\begin{tcolorbox}
\textsubscript{14} спіткав мене жах та тремтіння, і багато костей моїх він струсонув,
\end{tcolorbox}
\begin{tcolorbox}
\textsubscript{15} і дух перейшов по обличчі моїм, стало дуба волосся на тілі моїм...
\end{tcolorbox}
\begin{tcolorbox}
\textsubscript{16} Він стояв, але я не пізнав його вигляду, образ навпроти очей моїх був, і тихий голос почув я:
\end{tcolorbox}
\begin{tcolorbox}
\textsubscript{17} Хіба праведніша людина за Бога, хіба чоловік за свойого Творця є чистіший?
\end{tcolorbox}
\begin{tcolorbox}
\textsubscript{18} Таж рабам Своїм Він не йме віри, і накладає вину й на Своїх Анголів!
\end{tcolorbox}
\begin{tcolorbox}
\textsubscript{19} Що ж тоді мешканці глиняних хат, що в поросі їхня основа? Як міль, вони будуть розчавлені!
\end{tcolorbox}
\begin{tcolorbox}
\textsubscript{20} Вони товчені зранку до вечора, і без помочі гинуть назавжди...
\end{tcolorbox}
\begin{tcolorbox}
\textsubscript{21} Слава їхня минається з ними, вони помирають не в мудрості!...
\end{tcolorbox}
\subsection{CHAPTER 5}
\begin{tcolorbox}
\textsubscript{1} Ану клич, чи є хто, щоб тобі відповів? І до кого з святих ти вдасися?
\end{tcolorbox}
\begin{tcolorbox}
\textsubscript{2} Бо гнів побиває безглуздого, а заздрощі смерть завдають нерозумному!
\end{tcolorbox}
\begin{tcolorbox}
\textsubscript{3} Я бачив безумного, як він розсівся, та зараз оселя його спорохнявіла...
\end{tcolorbox}
\begin{tcolorbox}
\textsubscript{4} Від спасіння далекі сини його, вони без рятунку почавлені будуть у брамі!
\end{tcolorbox}
\begin{tcolorbox}
\textsubscript{5} Його жниво голодний поїсть, і з-між терну його забере, і спрагнені ось поковтають маєток його!
\end{tcolorbox}
\begin{tcolorbox}
\textsubscript{6} Бо нещастя виходить не з пороху, а горе росте не з землі,
\end{tcolorbox}
\begin{tcolorbox}
\textsubscript{7} бо людина народжується на страждання, як іскри, щоб угору летіти...
\end{tcolorbox}
\begin{tcolorbox}
\textsubscript{8} А я б удавався до Бога, і на Бога б поклав свою справу,
\end{tcolorbox}
\begin{tcolorbox}
\textsubscript{9} Він чинить велике та недослідиме, предивне, якому немає числа,
\end{tcolorbox}
\begin{tcolorbox}
\textsubscript{10} бо Він дає дощ на поверхню землі, і на поля посилає Він воду,
\end{tcolorbox}
\begin{tcolorbox}
\textsubscript{11} щоб поставить низьких на високе, і зміцнити спасіння засмучених.
\end{tcolorbox}
\begin{tcolorbox}
\textsubscript{12} Він розвіює задуми хитрих, і не виконують плану їх руки,
\end{tcolorbox}
\begin{tcolorbox}
\textsubscript{13} Він мудрих лукавством їх ловить, і рада крутійська марною стає,
\end{tcolorbox}
\begin{tcolorbox}
\textsubscript{14} вдень знаходять вони темноту, а в полудень мацають, мов уночі!...
\end{tcolorbox}
\begin{tcolorbox}
\textsubscript{15} І Він від меча урятовує бідного, а з міцної руки бідаря,
\end{tcolorbox}
\begin{tcolorbox}
\textsubscript{16} і стається надія нужденному, і замкнула уста свої кривда!
\end{tcolorbox}
\begin{tcolorbox}
\textsubscript{17} Тож блаженна людина, яку Бог картає, і ти не цурайсь Всемогутнього кари:
\end{tcolorbox}
\begin{tcolorbox}
\textsubscript{18} Бо Він рану завдасть і перев'яже, Він ламає й вигоюють руки Його!
\end{tcolorbox}
\begin{tcolorbox}
\textsubscript{19} В шістьох лихах спасає тебе, а в сімох не діткне тебе зло:
\end{tcolorbox}
\begin{tcolorbox}
\textsubscript{20} Викупляє тебе Він від смерти за голоду, а в бою з рук меча.
\end{tcolorbox}
\begin{tcolorbox}
\textsubscript{21} Як бич язика запанує, сховаєшся ти, і не будеш боятись руїни, як прийде вона.
\end{tcolorbox}
\begin{tcolorbox}
\textsubscript{22} З насилля та з голоду будеш сміятись, а земної звірини не бійся.
\end{tcolorbox}
\begin{tcolorbox}
\textsubscript{23} Бо з камінням на полі є в тебе умова, і звір польовий примирився з тобою.
\end{tcolorbox}
\begin{tcolorbox}
\textsubscript{24} І довідаєшся, що намет твій спокійний, і переглянеш домівку свою, і не знайдеш у ній недостатку.
\end{tcolorbox}
\begin{tcolorbox}
\textsubscript{25} І довідаєшся, що численне насіння твоє, а нащадки твої як трава на землі!
\end{tcolorbox}
\begin{tcolorbox}
\textsubscript{26} І в дозрілому віці до гробу ти зійдеш, як збіжжя доспіле ввіходить до клуні за часу свого!
\end{tcolorbox}
\begin{tcolorbox}
\textsubscript{27} Отож, дослідили ми це й воно так, послухай цього, й зрозумій собі все!
\end{tcolorbox}
\subsection{CHAPTER 6}
\begin{tcolorbox}
\textsubscript{1} А Йов відповів та й сказав:
\end{tcolorbox}
\begin{tcolorbox}
\textsubscript{2} Коли б смуток мій вірно був зважений, а з ним разом нещастя моє підняли на вазі,
\end{tcolorbox}
\begin{tcolorbox}
\textsubscript{3} то тепер воно тяжче було б від морського піску, тому нерозважне слова мої кажуть!...
\end{tcolorbox}
\begin{tcolorbox}
\textsubscript{4} Бо в мені Всемогутнього стріли, і їхня отрута п'є духа мого, страхи Божі шикуються в бій проти мене...
\end{tcolorbox}
\begin{tcolorbox}
\textsubscript{5} Чи дикий осел над травою реве? Хіба реве віл, коли ясла повні?
\end{tcolorbox}
\begin{tcolorbox}
\textsubscript{6} Чи без соли їдять несмачне, чи є смак у білкові яйця?
\end{tcolorbox}
\begin{tcolorbox}
\textsubscript{7} Чого не хотіла торкнутись душа моя, все те стало мені за поживу в хворобі...
\end{tcolorbox}
\begin{tcolorbox}
\textsubscript{8} О, коли б же збулося прохання моє, а моє сподівання дав Бог!
\end{tcolorbox}
\begin{tcolorbox}
\textsubscript{9} О, коли б зволив Бог розчавити мене, простягнув Свою руку й мене поламав,
\end{tcolorbox}
\begin{tcolorbox}
\textsubscript{10} то була б ще потіха мені, і скакав би я в немилосердному болі, бо я не зрікався слів Святого!...
\end{tcolorbox}
\begin{tcolorbox}
\textsubscript{11} Яка сила моя, що надію я матиму? І який мій кінець, щоб продовжити життя моє це?
\end{tcolorbox}
\begin{tcolorbox}
\textsubscript{12} Чи сила камінна то сила моя? Чи тіло моє мідяне?
\end{tcolorbox}
\begin{tcolorbox}
\textsubscript{13} Чи не поміч для мене в мені, чи спасіння від мене відсунене?
\end{tcolorbox}
\begin{tcolorbox}
\textsubscript{14} Для того, хто гине, товариш то ласка, хоча б опустив того страх Всемогутнього...
\end{tcolorbox}
\begin{tcolorbox}
\textsubscript{15} Брати мої зраджують, мов той потік, мов річище потоків, минають вони,
\end{tcolorbox}
\begin{tcolorbox}
\textsubscript{16} темніші від льоду вони, в них ховається сніг.
\end{tcolorbox}
\begin{tcolorbox}
\textsubscript{17} Коли сонце їх гріє, вони висихають, у теплі гинуть з місця свого.
\end{tcolorbox}
\begin{tcolorbox}
\textsubscript{18} Каравани дорогу свою відхиляють, уходять в пустиню й щезають.
\end{tcolorbox}
\begin{tcolorbox}
\textsubscript{19} Каравани з Теми поглядають, походи з Шеви покладають надії на них.
\end{tcolorbox}
\begin{tcolorbox}
\textsubscript{20} І засоромилися, що вони сподівались; до нього прийшли та й збентежились.
\end{tcolorbox}
\begin{tcolorbox}
\textsubscript{21} Так і ви тепер стали ніщо, побачили страх і злякались!
\end{tcolorbox}
\begin{tcolorbox}
\textsubscript{22} Чи я говорив коли: Дайте мені, а з маєтку свого дайте підкуп за мене,
\end{tcolorbox}
\begin{tcolorbox}
\textsubscript{23} і врятуйте мене з руки ворога, і з рук гнобителевих мене викупіть?
\end{tcolorbox}
\begin{tcolorbox}
\textsubscript{24} Навчіть ви мене і я буду мовчати, а в чім я невмисне згрішив розтлумачте мені...
\end{tcolorbox}
\begin{tcolorbox}
\textsubscript{25} Які гострі слова справедливі, та що то доводить догана від вас?
\end{tcolorbox}
\begin{tcolorbox}
\textsubscript{26} Чи ви думаєте докоряти словами? Бо на вітер слова одчайдушного,
\end{tcolorbox}
\begin{tcolorbox}
\textsubscript{27} і на сироту нападаєте ви, і копаєте яму для друга свого!...
\end{tcolorbox}
\begin{tcolorbox}
\textsubscript{28} Та звольте поглянути на мене тепер, а я не скажу перед вами неправди.
\end{tcolorbox}
\begin{tcolorbox}
\textsubscript{29} Верніться ж, хай кривди не буде, і верніться, ще в тім моя правда!
\end{tcolorbox}
\begin{tcolorbox}
\textsubscript{30} Хіба в мене на язиці є неправда? чи ж не маю смаку, щоб розпізнати нещастя?
\end{tcolorbox}
\subsection{CHAPTER 7}
\begin{tcolorbox}
\textsubscript{1} Хіба чоловік на землі не на службі військовій? І його дні як дні наймита!...
\end{tcolorbox}
\begin{tcolorbox}
\textsubscript{2} Як раб, спрагнений тіні, і як наймит чекає заплати за працю свою,
\end{tcolorbox}
\begin{tcolorbox}
\textsubscript{3} так місяці марности дано в спадок мені, та ночі терпіння мені відлічили...
\end{tcolorbox}
\begin{tcolorbox}
\textsubscript{4} Коли я кладусь, то кажу: Коли встану? І тягнеться вечір, і перевертання із боку на бік їм до ранку...
\end{tcolorbox}
\begin{tcolorbox}
\textsubscript{5} Зодяглось моє тіло червою та струпами в поросі, шкіра моя затверділа й бридка...
\end{tcolorbox}
\begin{tcolorbox}
\textsubscript{6} А дні мої стали швидчіші за ткацького човника, і в марнотній надії минають вони...
\end{tcolorbox}
\begin{tcolorbox}
\textsubscript{7} Пам'ятай, що життя моє вітер, моє око вже більш не побачить добра...
\end{tcolorbox}
\begin{tcolorbox}
\textsubscript{8} Не побачить мене око того, хто бачив мене, Твої очі поглянуть на мене та немає мене...
\end{tcolorbox}
\begin{tcolorbox}
\textsubscript{9} Як хмара зникає й проходить, так хто сходить в шеол, не виходить,
\end{tcolorbox}
\begin{tcolorbox}
\textsubscript{10} не вертається вже той до дому свого, та й його не пізнає вже місце його...
\end{tcolorbox}
\begin{tcolorbox}
\textsubscript{11} Тож не стримаю я своїх уст, говоритиму в утиску духа свого, нарікати я буду в гіркоті своєї душі:
\end{tcolorbox}
\begin{tcolorbox}
\textsubscript{12} Чи я море чи морська потвора, що Ти надо мною сторожу поставив?
\end{tcolorbox}
\begin{tcolorbox}
\textsubscript{13} Коли я кажу: Нехай постіль потішить мене, хай думки мої ложе моє забере,
\end{tcolorbox}
\begin{tcolorbox}
\textsubscript{14} то Ти снами лякаєш мене, і видіннями страшиш мене...
\end{tcolorbox}
\begin{tcolorbox}
\textsubscript{15} І душа моя прагне задушення, смерти хочуть мої кості.
\end{tcolorbox}
\begin{tcolorbox}
\textsubscript{16} Я обридив життям... Не повіки ж я житиму!... Відпусти ж Ти мене, бо марнота оці мої дні!...
\end{tcolorbox}
\begin{tcolorbox}
\textsubscript{17} Що таке чоловік, що його Ти підносиш, що серце Своє прикладаєш до нього?
\end{tcolorbox}
\begin{tcolorbox}
\textsubscript{18} Ти щоранку за ним назираєш, щохвилі його Ти досліджуєш...
\end{tcolorbox}
\begin{tcolorbox}
\textsubscript{19} Як довго від мене ще Ти не відвернешся, не пустиш мене проковтнути хоч слину свою?
\end{tcolorbox}
\begin{tcolorbox}
\textsubscript{20} Я згрішив... Що ж я маю робити, о Стороже людський? Чому Ти поклав мене ціллю для Себе, і я стався собі тягарем?
\end{tcolorbox}
\begin{tcolorbox}
\textsubscript{21} І чому Ти не простиш мойого гріха, і не відкинеш провини моєї? А тепер я до пороху ляжу, і Ти будеш шукати мене, та немає мене...
\end{tcolorbox}
\subsection{CHAPTER 8}
\begin{tcolorbox}
\textsubscript{1} І заговорив шух'янин Білдад та й сказав:
\end{tcolorbox}
\begin{tcolorbox}
\textsubscript{2} Аж доки ти будеш таке теревенити? І доки слова твоїх уст будуть вітром бурхливим?
\end{tcolorbox}
\begin{tcolorbox}
\textsubscript{3} Чи Бог скривлює суд, і хіба Всемогутній викривлює правду?
\end{tcolorbox}
\begin{tcolorbox}
\textsubscript{4} Якщо твої діти згрішили Йому, то Він їх віддав в руку їх беззаконня!
\end{tcolorbox}
\begin{tcolorbox}
\textsubscript{5} Якщо ти звертатися будеш до Бога, і будеш благати Всемогутнього,
\end{tcolorbox}
\begin{tcolorbox}
\textsubscript{6} якщо чистий ти та безневинний, то тепер Він тобі Свою милість пробудить, і наповнить оселю твою справедливістю,
\end{tcolorbox}
\begin{tcolorbox}
\textsubscript{7} і хоч твій початок нужденний, але твій кінець буде вельми великий!
\end{tcolorbox}
\begin{tcolorbox}
\textsubscript{8} Поспитай в покоління давнішого, і міцно збагни батьків їхніх,
\end{tcolorbox}
\begin{tcolorbox}
\textsubscript{9} бо ми ж учорашні, й нічого не знаєм, бо тінь наші дні на землі,
\end{tcolorbox}
\begin{tcolorbox}
\textsubscript{10} отож вони навчать тебе, тобі скажуть, і з серця свойого слова подадуть:
\end{tcolorbox}
\begin{tcolorbox}
\textsubscript{11} Чи папірус росте без болота? Чи росте очерет без води?
\end{tcolorbox}
\begin{tcolorbox}
\textsubscript{12} Він іще в доспіванні своїм, не зривається, але сохне раніш за всіляку траву:
\end{tcolorbox}
\begin{tcolorbox}
\textsubscript{13} отакі то дороги всіх тих, хто забуває про Бога! І згине надія безбожного,
\end{tcolorbox}
\begin{tcolorbox}
\textsubscript{14} бо його сподівання як те павутиння, і як дім павуків його певність...
\end{tcolorbox}
\begin{tcolorbox}
\textsubscript{15} На свій дім опирається, та не встоїть, тримається міцно за нього, й не вдержиться він...
\end{tcolorbox}
\begin{tcolorbox}
\textsubscript{16} Він зеленіє на сонці, й галузки його випинаються понад садка його,
\end{tcolorbox}
\begin{tcolorbox}
\textsubscript{17} на купі каміння сплелося коріння його, воно між каміння вросло:
\end{tcolorbox}
\begin{tcolorbox}
\textsubscript{18} Якщо вирвуть його з його місця, то зречеться його: тебе я не бачило!...
\end{tcolorbox}
\begin{tcolorbox}
\textsubscript{19} Така радість дороги його, а з пороху інші ростуть.
\end{tcolorbox}
\begin{tcolorbox}
\textsubscript{20} Тож невинного Бог не цурається, і не буде тримати за руку злочинців,
\end{tcolorbox}
\begin{tcolorbox}
\textsubscript{21} аж наповнить уста твої сміхом, а губи твої криком радости...
\end{tcolorbox}
\begin{tcolorbox}
\textsubscript{22} Твої ненависники в сором зодягнуться, і намету безбожних не буде!
\end{tcolorbox}
\subsection{CHAPTER 9}
\begin{tcolorbox}
\textsubscript{1} А Йов відповів та й сказав:
\end{tcolorbox}
\begin{tcolorbox}
\textsubscript{2} Справді пізнав я, що так... Та як оправдатись людині земній перед Богом?
\end{tcolorbox}
\begin{tcolorbox}
\textsubscript{3} Якщо вона схоче на прю стати з Ним, Він відповіді їй не дасть ні на одне із тисячі скаржень...
\end{tcolorbox}
\begin{tcolorbox}
\textsubscript{4} Він мудрого серця й могутньої сили; хто був проти Нього упертий і цілим зостався?
\end{tcolorbox}
\begin{tcolorbox}
\textsubscript{5} Він гори зриває, й не знають вони, що в гніві Своїм Він їх перевернув.
\end{tcolorbox}
\begin{tcolorbox}
\textsubscript{6} Він землю трясе з її місця, і стовпи її трусяться.
\end{tcolorbox}
\begin{tcolorbox}
\textsubscript{7} Він сонцеві скаже, й не сходить воно, і Він запечатує зорі.
\end{tcolorbox}
\begin{tcolorbox}
\textsubscript{8} Розтягує небо Він Сам, і ходить по морських висотах,
\end{tcolorbox}
\begin{tcolorbox}
\textsubscript{9} Він Воза створив, Оріона та Волосожара, та зорі південні.
\end{tcolorbox}
\begin{tcolorbox}
\textsubscript{10} Він чинить велике та недослідиме, предивне, якому немає числа!...
\end{tcolorbox}
\begin{tcolorbox}
\textsubscript{11} Ось Він надо мною проходить, та я не побачу, і Він перейде, а я не приглянусь до Нього...
\end{tcolorbox}
\begin{tcolorbox}
\textsubscript{12} Ось Він схопить кого, хто заверне Його, хто скаже Йому: що Ти робиш?
\end{tcolorbox}
\begin{tcolorbox}
\textsubscript{13} Бог гніву Свойого не спинить, під Ним гнуться Рагавові помічники,
\end{tcolorbox}
\begin{tcolorbox}
\textsubscript{14} що ж тоді відповім я Йому? Які я слова підберу проти Нього,
\end{tcolorbox}
\begin{tcolorbox}
\textsubscript{15} я, який коли б був справедливий, то не відповідав би, я, що благаю свойого Суддю?
\end{tcolorbox}
\begin{tcolorbox}
\textsubscript{16} Коли б я взивав, а Він мені відповідь дав, не повірю, що вчув би мій голос,
\end{tcolorbox}
\begin{tcolorbox}
\textsubscript{17} Він, що бурею може розтерти мене та помножити рани мої безневинно...
\end{tcolorbox}
\begin{tcolorbox}
\textsubscript{18} Не дає Він мені й звести духа мого, бо мене насичає гіркотою.
\end{tcolorbox}
\begin{tcolorbox}
\textsubscript{19} Коли ходить про силу, то Він Всемогутній, коли ж ходить про суд, хто посвідчить мені?
\end{tcolorbox}
\begin{tcolorbox}
\textsubscript{20} Якщо б справедливим я був, то осудять мене мої уста, якщо я безневинний, то вчинять мене винуватим...
\end{tcolorbox}
\begin{tcolorbox}
\textsubscript{21} Я невинний, проте своєї душі я не знаю, і не радий життям своїм я...
\end{tcolorbox}
\begin{tcolorbox}
\textsubscript{22} Це одне, а тому я кажу: невинного як і лукавого Він вигубляє...
\end{tcolorbox}
\begin{tcolorbox}
\textsubscript{23} Якщо нагло бич смерть заподіює, Він з проби невинних сміється...
\end{tcolorbox}
\begin{tcolorbox}
\textsubscript{24} У руку безбожного дана земля, та Він лиця суддів її закриває... Як не Він, тоді хто?
\end{tcolorbox}
\begin{tcolorbox}
\textsubscript{25} А дні мої стали швидкіші, як той скороход, повтікали, не бачили доброго,
\end{tcolorbox}
\begin{tcolorbox}
\textsubscript{26} проминули, немов ті човни очеретяні, мов орел, що несеться на здобич...
\end{tcolorbox}
\begin{tcolorbox}
\textsubscript{27} Якщо я скажу: Хай забуду своє нарікання, хай зміню я обличчя своє й підбадьорюся,
\end{tcolorbox}
\begin{tcolorbox}
\textsubscript{28} то боюся всіх смутків своїх, і я знаю, що Ти не очистиш мене...
\end{tcolorbox}
\begin{tcolorbox}
\textsubscript{29} Все одно буду я винуватий, то нащо надармо я мучитися буду?
\end{tcolorbox}
\begin{tcolorbox}
\textsubscript{30} Коли б я умився сніговою водою, і почистив би лугом долоні свої,
\end{tcolorbox}
\begin{tcolorbox}
\textsubscript{31} то й тоді Ти до гробу опустиш мене, і учинить бридким мене одіж моя...
\end{tcolorbox}
\begin{tcolorbox}
\textsubscript{32} Бо Він не людина, як я, й Йому відповіді я не дам, і не підемо разом на суд,
\end{tcolorbox}
\begin{tcolorbox}
\textsubscript{33} поміж нами нема посередника, що поклав би на нас на обох свою руку...
\end{tcolorbox}
\begin{tcolorbox}
\textsubscript{34} Нехай забере Він від мене Свойого бича, Його ж страх хай мене не жахає,
\end{tcolorbox}
\begin{tcolorbox}
\textsubscript{35} тоді буду казати, й не буду боятись Його, бо я не такий сам з собою!...
\end{tcolorbox}
\subsection{CHAPTER 10}
\begin{tcolorbox}
\textsubscript{1} Життя моє стало бридке для моєї душі... Нехай нарікання своє я на себе пущу, нехай говорю я в гіркоті своєї душі!
\end{tcolorbox}
\begin{tcolorbox}
\textsubscript{2} Скажу Богові я: Не осуджуй мене! Повідом же мене, чого став Ти зо мною на прю?
\end{tcolorbox}
\begin{tcolorbox}
\textsubscript{3} Чи це добре Тобі, що Ти гнобиш мене, що погорджуєш творивом рук Своїх, а раду безбожних освітлюєш?
\end{tcolorbox}
\begin{tcolorbox}
\textsubscript{4} Хіба маєш Ти очі тілесні? Чи Ти бачиш так само, як бачить людина людину?
\end{tcolorbox}
\begin{tcolorbox}
\textsubscript{5} Хіба Твої дні як дні людські, чи літа Твої як дні мужа,
\end{tcolorbox}
\begin{tcolorbox}
\textsubscript{6} що шукаєш провини моєї й вивідуєш гріх мій,
\end{tcolorbox}
\begin{tcolorbox}
\textsubscript{7} хоч відаєш Ти, що я не беззаконник, та нема, хто б мене врятував від Твоєї руки?
\end{tcolorbox}
\begin{tcolorbox}
\textsubscript{8} Твої руки створили мене і вчинили мене, потім Ти обернувся і губиш мене...
\end{tcolorbox}
\begin{tcolorbox}
\textsubscript{9} Пам'ятай, що мов глину мене обробив Ти, і в порох мене обертаєш.
\end{tcolorbox}
\begin{tcolorbox}
\textsubscript{10} Чи не ллєш мене, мов молоко, і не згустив Ти мене, мов на сир?
\end{tcolorbox}
\begin{tcolorbox}
\textsubscript{11} Ти шкірою й тілом мене зодягаєш, і сплів Ти мене із костей та із жил.
\end{tcolorbox}
\begin{tcolorbox}
\textsubscript{12} Життя й милість подав Ти мені, а опіка Твоя стерегла мого духа.
\end{tcolorbox}
\begin{tcolorbox}
\textsubscript{13} А оце заховав Ти у серці Своєму, я знаю, що є воно в Тебе:
\end{tcolorbox}
\begin{tcolorbox}
\textsubscript{14} якщо я грішу, Ти мене стережеш, та з провини моєї мене не очищуєш...
\end{tcolorbox}
\begin{tcolorbox}
\textsubscript{15} Якщо я провинюся, то горе мені! А якщо я невинний, не смію підняти свою голову, ситий стидом та напоєний горем своїм!...
\end{tcolorbox}
\begin{tcolorbox}
\textsubscript{16} А коли піднесеться вона, то Ти ловиш мене, як той лев, і знову предивно зо мною поводишся:
\end{tcolorbox}
\begin{tcolorbox}
\textsubscript{17} поновлюєш свідків Своїх проти мене, помножуєш гнів Свій на мене, військо за військом на мене Ти шлеш...
\end{tcolorbox}
\begin{tcolorbox}
\textsubscript{18} І нащо з утроби Ти вивів мене? Я був би помер, і жоднісіньке око мене не побачило б,
\end{tcolorbox}
\begin{tcolorbox}
\textsubscript{19} як нібито не існував був би я, перейшов би з утроби до гробу...
\end{tcolorbox}
\begin{tcolorbox}
\textsubscript{20} Отож, дні мої нечисленні, перестань же, й від мене вступись, і нехай не турбуюся я бодай трохи,
\end{tcolorbox}
\begin{tcolorbox}
\textsubscript{21} поки я не піду й не вернуся! до краю темноти та смертної тіні,
\end{tcolorbox}
\begin{tcolorbox}
\textsubscript{22} до темного краю, як морок, до тьмяного краю, в якому порядків нема, і де світло, як темрява...
\end{tcolorbox}
\subsection{CHAPTER 11}
\begin{tcolorbox}
\textsubscript{1} І заговорив нааматянин Цофар та й сказав:
\end{tcolorbox}
\begin{tcolorbox}
\textsubscript{2} Чи має зостатись без відповіді безліч слів? І хіба язиката людина невинною буде?
\end{tcolorbox}
\begin{tcolorbox}
\textsubscript{3} Чи мужі замовчать твої теревені, й не буде кому засоромити тебе?
\end{tcolorbox}
\begin{tcolorbox}
\textsubscript{4} Ось говориш ти: Чисте моє міркування, і я чистий в очах Твоїх, Боже!
\end{tcolorbox}
\begin{tcolorbox}
\textsubscript{5} О, коли б говорити став Бог, і відкрив Свої уста до тебе,
\end{tcolorbox}
\begin{tcolorbox}
\textsubscript{6} і представив тобі таємниці премудрости, бо вони як ті чуда роздумування! І знай, вимагає Бог менше від тебе, ніж провини твої того варті!
\end{tcolorbox}
\begin{tcolorbox}
\textsubscript{7} Чи ти Божу глибінь дослідиш, чи знаєш ти аж до кінця Всемогутнього?
\end{tcolorbox}
\begin{tcolorbox}
\textsubscript{8} Вона вища від неба, що зможеш зробити? І глибша вона за шеол, як пізнаєш її?
\end{tcolorbox}
\begin{tcolorbox}
\textsubscript{9} Її міра довша за землю, і ширша за море вона!
\end{tcolorbox}
\begin{tcolorbox}
\textsubscript{10} Якщо Він перейде й замкне щось, і згромадить, то хто заборонить Йому?
\end{tcolorbox}
\begin{tcolorbox}
\textsubscript{11} Бо Він знає нікчемності людські та бачить насилля, і Він не догляне?
\end{tcolorbox}
\begin{tcolorbox}
\textsubscript{12} Тож людина порожня мудрішає, хоч народжується, як те дике осля!
\end{tcolorbox}
\begin{tcolorbox}
\textsubscript{13} Якщо ти зміцниш своє серце, і свої руки до Нього простягнеш,
\end{tcolorbox}
\begin{tcolorbox}
\textsubscript{14} якщо є беззаконня в руці твоїй, то прожени ти його, і кривда в наметах твоїх нехай не пробуває,
\end{tcolorbox}
\begin{tcolorbox}
\textsubscript{15} тож тоді ти підіймеш обличчя невинне своє, і будеш міцний, і не будеш боятись!
\end{tcolorbox}
\begin{tcolorbox}
\textsubscript{16} Бо забудеш страждання, про них будеш згадувати, як про воду, яка пропливла...
\end{tcolorbox}
\begin{tcolorbox}
\textsubscript{17} Від півдня повстане життя, а темрява буде, як ранок.
\end{tcolorbox}
\begin{tcolorbox}
\textsubscript{18} І будеш ти певний, бо маєш надію, і викопаєш собі яму та й будеш безпечно лежати,
\end{tcolorbox}
\begin{tcolorbox}
\textsubscript{19} і будеш лежати, й ніхто не сполошить, і багато-хто будуть підлещуватися до обличчя твого...
\end{tcolorbox}
\begin{tcolorbox}
\textsubscript{20} А очі безбожних минуться, і згине притулок у них, а їхня надія то стогін душі!
\end{tcolorbox}
\subsection{CHAPTER 12}
\begin{tcolorbox}
\textsubscript{1} А Йов відповів та й сказав:
\end{tcolorbox}
\begin{tcolorbox}
\textsubscript{2} Справді, то ж ви тільки люди, і мудрість із вами помре!...
\end{tcolorbox}
\begin{tcolorbox}
\textsubscript{3} Таж і я маю розум, як ви, я не нижчий від вас! І в кого немає такого, як це?
\end{tcolorbox}
\begin{tcolorbox}
\textsubscript{4} Посміховищем став я для друга свого, я, що кликав до Бога, і Він мені відповідав, посміховищем став справедливий, невинний...
\end{tcolorbox}
\begin{tcolorbox}
\textsubscript{5} Нещасливцю погорда, на думку спокійного, приготовлена для спотикання ноги!
\end{tcolorbox}
\begin{tcolorbox}
\textsubscript{6} Спокійні намети грабіжників, і безпечність у тих, хто Бога гнівить, у того, хто ніби то Бога провадить рукою своєю.
\end{tcolorbox}
\begin{tcolorbox}
\textsubscript{7} Але запитай хоч худобу і навчить тебе, і птаство небесне й тобі розповість.
\end{tcolorbox}
\begin{tcolorbox}
\textsubscript{8} Або говори до землі й вона вивчить тебе, і розкажуть тобі риби морські.
\end{tcolorbox}
\begin{tcolorbox}
\textsubscript{9} Хто б із цього всього не пізнав, що Господня рука це вчинила?
\end{tcolorbox}
\begin{tcolorbox}
\textsubscript{10} Що в Нього в руці душа всього живого й дух кожного людського тіла?
\end{tcolorbox}
\begin{tcolorbox}
\textsubscript{11} Чи ж не ухо слова розбирає, піднебіння ж смакує для себе поживу?
\end{tcolorbox}
\begin{tcolorbox}
\textsubscript{12} Мудрість у старших, бо довгість днів розум.
\end{tcolorbox}
\begin{tcolorbox}
\textsubscript{13} Мудрість та сила у Нього, Його рада та розум.
\end{tcolorbox}
\begin{tcolorbox}
\textsubscript{14} Ось Він зруйнує й не буде воно відбудоване, замкне чоловіка й не буде він випущений.
\end{tcolorbox}
\begin{tcolorbox}
\textsubscript{15} Ось Він стримає води і висохнуть, Він їх пустить то землю вони перевернуть.
\end{tcolorbox}
\begin{tcolorbox}
\textsubscript{16} В Нього сила та задум, у Нього заблуджений і той, хто призводить до блуду.
\end{tcolorbox}
\begin{tcolorbox}
\textsubscript{17} Він уводить у помилку радників, і обезумлює суддів,
\end{tcolorbox}
\begin{tcolorbox}
\textsubscript{18} Він розв'язує пута царів і приперізує пояса на їхні стегна.
\end{tcolorbox}
\begin{tcolorbox}
\textsubscript{19} Він провадить священиків босо, і потужних повалює,
\end{tcolorbox}
\begin{tcolorbox}
\textsubscript{20} Він надійним уста відіймає й забирає від старших розумність.
\end{tcolorbox}
\begin{tcolorbox}
\textsubscript{21} На достойників ллє Він погорду, а пояса можним ослаблює.
\end{tcolorbox}
\begin{tcolorbox}
\textsubscript{22} Відкриває Він речі глибокі із темряви, а темне провадить на світло.
\end{tcolorbox}
\begin{tcolorbox}
\textsubscript{23} Він робить народи потужними й знову їх нищить, Він народи поширює, й потім виводить в неволю.
\end{tcolorbox}
\begin{tcolorbox}
\textsubscript{24} Відіймає Він розум в народніх голів на землі та блукати їх змушує по бездорожній пустелі,
\end{tcolorbox}
\begin{tcolorbox}
\textsubscript{25} вони ходять навпомацки в темряві темній, і Він упроваджує їх в блуканину, мов п'яного!
\end{tcolorbox}
\subsection{CHAPTER 13}
\begin{tcolorbox}
\textsubscript{1} Ось усе оце бачило око моє, чуло ухо моє, та й усе зауважило...
\end{tcolorbox}
\begin{tcolorbox}
\textsubscript{2} Як знаєте ви знаю й я, я не нижчий від вас,
\end{tcolorbox}
\begin{tcolorbox}
\textsubscript{3} і я говоритиму до Всемогутнього, і переконувати хочу Бога!
\end{tcolorbox}
\begin{tcolorbox}
\textsubscript{4} Та неправду куєте тут ви, лікарі непутящі ви всі!
\end{tcolorbox}
\begin{tcolorbox}
\textsubscript{5} О, коли б ви насправді мовчали, то вам це за мудрість було б!...
\end{tcolorbox}
\begin{tcolorbox}
\textsubscript{6} Послухайте но переконань моїх: і вислухайте заперечення уст моїх.
\end{tcolorbox}
\begin{tcolorbox}
\textsubscript{7} Чи будете ви говорити неправду про Бога, чи будете ви говорити оману про Нього?
\end{tcolorbox}
\begin{tcolorbox}
\textsubscript{8} Чи будете ви уважати на Нього? Чи за Бога на прю постаєте?
\end{tcolorbox}
\begin{tcolorbox}
\textsubscript{9} Чи добре, що вас Він дослідить? Чи як з людини сміються, так будете ви насміхатися з Нього?
\end{tcolorbox}
\begin{tcolorbox}
\textsubscript{10} Насправді Він вас покарає, якщо будете ви потурати таємно особі!
\end{tcolorbox}
\begin{tcolorbox}
\textsubscript{11} Чи ж велич Його не настрашує вас, і не нападає на вас Його страх?
\end{tcolorbox}
\begin{tcolorbox}
\textsubscript{12} Ваші нагадування це прислів'я із попелу, ваші башти це глиняні башти!
\end{tcolorbox}
\begin{tcolorbox}
\textsubscript{13} Мовчіть передо мною, а я говоритиму, і нехай щобудь прийде на мене!
\end{tcolorbox}
\begin{tcolorbox}
\textsubscript{14} Нащо дертиму я своє тіло зубами своїми, а душу свою покладу в свою руку?
\end{tcolorbox}
\begin{tcolorbox}
\textsubscript{15} Ось Він мене вб'є, і я надії не матиму, але перед обличчям Його про дороги свої сперечатися буду!
\end{tcolorbox}
\begin{tcolorbox}
\textsubscript{16} І це мені буде спасінням, бо перед обличчя Його не підійде безбожний.
\end{tcolorbox}
\begin{tcolorbox}
\textsubscript{17} Направду послухайте слова мого, а моє це освідчення в ваших ушах нехай буде.
\end{tcolorbox}
\begin{tcolorbox}
\textsubscript{18} Ось я суд спорядив, бо я справедливий, те знаю!
\end{tcolorbox}
\begin{tcolorbox}
\textsubscript{19} Хто той, що буде зо мною провадити прю? Бо тепер я замовк би й помер би...
\end{tcolorbox}
\begin{tcolorbox}
\textsubscript{20} Тільки двох цих речей не роби Ти зо мною, тоді від обличчя Твого я не буду ховатись:
\end{tcolorbox}
\begin{tcolorbox}
\textsubscript{21} віддали Свою руку від мене, а Твій страх хай мене не жахає!...
\end{tcolorbox}
\begin{tcolorbox}
\textsubscript{22} Тоді клич, а я відповідатиму, або я говоритиму, Ти ж мені відповідь дай!
\end{tcolorbox}
\begin{tcolorbox}
\textsubscript{23} Скільки в мене провин та гріхів? Покажи Ти мені мій переступ та гріх мій!
\end{tcolorbox}
\begin{tcolorbox}
\textsubscript{24} Чому Ти ховаєш обличчя Своє і вважаєш мене Собі ворогом?
\end{tcolorbox}
\begin{tcolorbox}
\textsubscript{25} Чи Ти будеш страхати завіяний вітром листок? Чи Ти соломину суху будеш гнати?
\end{tcolorbox}
\begin{tcolorbox}
\textsubscript{26} Бо Ти пишеш на мене гіркоти й провини мого молодечого віку даєш на спадок мені,
\end{tcolorbox}
\begin{tcolorbox}
\textsubscript{27} і в кайдани заковуєш ноги мої, і всі дороги мої стережеш, назирці ходиш за мною,
\end{tcolorbox}
\begin{tcolorbox}
\textsubscript{28} і він розпадається, мов та трухлявина, немов та одежа, що міль її з'їла!...
\end{tcolorbox}
\subsection{CHAPTER 14}
\begin{tcolorbox}
\textsubscript{1} Людина, що від жінки народжена, короткоденна та повна печалями:
\end{tcolorbox}
\begin{tcolorbox}
\textsubscript{2} вона виходить, як квітка й зів'яне, і втікає, мов тінь, і не зостається...
\end{tcolorbox}
\begin{tcolorbox}
\textsubscript{3} І на такого Ти очі Свої відкриваєш, і водиш на суд із Собою його!
\end{tcolorbox}
\begin{tcolorbox}
\textsubscript{4} Хто чистого вивести може з нечистого? Ані один!
\end{tcolorbox}
\begin{tcolorbox}
\textsubscript{5} Якщо визначені його дні, число його місяців в Тебе, якщо Ти призначив для нього мету, що її не перейде,
\end{tcolorbox}
\begin{tcolorbox}
\textsubscript{6} відвернися від нього і він заспокоїться, і буде він тішитися своїм днем, як той наймит...
\end{tcolorbox}
\begin{tcolorbox}
\textsubscript{7} Бо дерево має надію: якщо буде стяте, то силу отримає знову, і парост його не загине;
\end{tcolorbox}
\begin{tcolorbox}
\textsubscript{8} якщо постаріє в землі його корінь і в поросі вмре його пень,
\end{tcolorbox}
\begin{tcolorbox}
\textsubscript{9} то від водного запаху знов зацвіте, і пустить галуззя, немов саджанець!
\end{tcolorbox}
\begin{tcolorbox}
\textsubscript{10} А помре чоловік і зникає, а сконає людина то де ж вона є?...
\end{tcolorbox}
\begin{tcolorbox}
\textsubscript{11} Як вода витікає із озера, а річка спадає та сохне,
\end{tcolorbox}
\begin{tcolorbox}
\textsubscript{12} так і та людина покладеться й не встане, аж до закінчення неба не збудяться люди та не прокинуться зо сну свого...
\end{tcolorbox}
\begin{tcolorbox}
\textsubscript{13} О, якби Ти в шеолі мене заховав, коли б Ти мене приховав, аж поки минеться Твій гнів, коли б час Ти призначив мені, та й про мене згадав!
\end{tcolorbox}
\begin{tcolorbox}
\textsubscript{14} Як помре чоловік, то чи він оживе? Буду мати надію по всі дні свойого життя, аж поки не прийде заміна для мене!
\end{tcolorbox}
\begin{tcolorbox}
\textsubscript{15} Кликав би Ти, то я відповів би Тобі, за чин Своїх рук сумував би,
\end{tcolorbox}
\begin{tcolorbox}
\textsubscript{16} бо кроки мої рахував би тепер, а мойого гріха не стеріг би,
\end{tcolorbox}
\begin{tcolorbox}
\textsubscript{17} провина моя була б запечатана в вузлику, і Ти закрив би моє беззаконня...
\end{tcolorbox}
\begin{tcolorbox}
\textsubscript{18} Але гора справді впаде, а скеля зсувається з місця свого,
\end{tcolorbox}
\begin{tcolorbox}
\textsubscript{19} каміння стирає вода, її злива сполощує порох землі, так надію того Ти губиш...
\end{tcolorbox}
\begin{tcolorbox}
\textsubscript{20} Ти силою схопиш назавжди його, і відходить, Ти міняєш обличчя його й відсилаєш його...
\end{tcolorbox}
\begin{tcolorbox}
\textsubscript{21} Чи сини його славні, того він не знає, чи в прикрому стані того він не відає...
\end{tcolorbox}
\begin{tcolorbox}
\textsubscript{22} Боліє він тільки тоді, коли тіло на ньому, коли в ньому душа тоді тужить..
\end{tcolorbox}
\subsection{CHAPTER 15}
\begin{tcolorbox}
\textsubscript{1} І відповів теманянин Еліфаз та й сказав:
\end{tcolorbox}
\begin{tcolorbox}
\textsubscript{2} Чи відповідатиме мудра людина знанням вітряним, і східнім вітром наповнить утробу свою?
\end{tcolorbox}
\begin{tcolorbox}
\textsubscript{3} Буде виправдуватися тим словом, що не надається, чи тими речами, що пожитку немає від них?
\end{tcolorbox}
\begin{tcolorbox}
\textsubscript{4} Ти страх Божий руйнуєш також, і пустошиш молитву до Бога,
\end{tcolorbox}
\begin{tcolorbox}
\textsubscript{5} бо навчає провина твоя твої уста, і ти вибираєш собі язика хитрунів.
\end{tcolorbox}
\begin{tcolorbox}
\textsubscript{6} Оскаржають тебе твої уста, не я, й твої губи свідкують на тебе:
\end{tcolorbox}
\begin{tcolorbox}
\textsubscript{7} Чи ти народився людиною першою, чи раніше, ніж згір'я, ти створений?
\end{tcolorbox}
\begin{tcolorbox}
\textsubscript{8} Чи ти слухав у Божій таємній нараді, та мудрість для себе забрав?
\end{tcolorbox}
\begin{tcolorbox}
\textsubscript{9} Що ти знаєш, чого б ми не знали? Що ти зрозумів, і не з нами воно?
\end{tcolorbox}
\begin{tcolorbox}
\textsubscript{10} Поміж нами і сивий, отой і старий, старший днями від батька твого.
\end{tcolorbox}
\begin{tcolorbox}
\textsubscript{11} Чи мало для тебе потішення Божі та слово, яке Він сховав у тобі?
\end{tcolorbox}
\begin{tcolorbox}
\textsubscript{12} Чого то підносить тебе твоє серце, й які то знаки твої очі дають,
\end{tcolorbox}
\begin{tcolorbox}
\textsubscript{13} що на Бога звертаєш ти духа свого, і з своїх уст випускаєш подібні слова?
\end{tcolorbox}
\begin{tcolorbox}
\textsubscript{14} Що таке чоловік, щоб оправданим бути, і щоб був справедливим від жінки народжений?
\end{tcolorbox}
\begin{tcolorbox}
\textsubscript{15} Таж Він навіть святим Своїм не довіряє, і не оправдані в очах Його небеса,
\end{tcolorbox}
\begin{tcolorbox}
\textsubscript{16} що ж тоді чоловік той бридкий та зіпсутий, що п'є кривду, як воду?
\end{tcolorbox}
\begin{tcolorbox}
\textsubscript{17} Я тобі розповім, ти послухай мене, а що бачив, то те розкажу,
\end{tcolorbox}
\begin{tcolorbox}
\textsubscript{18} про що мудрі донесли та від батьків своїх не затаїли того,
\end{tcolorbox}
\begin{tcolorbox}
\textsubscript{19} їм самим була дана земля, і не приходив чужий поміж них.
\end{tcolorbox}
\begin{tcolorbox}
\textsubscript{20} Безбожний тремтить по всі дні, а насильникові мало років заховано.
\end{tcolorbox}
\begin{tcolorbox}
\textsubscript{21} Вереск жахів у нього в ушах, серед миру приходить на нього грабіжник.
\end{tcolorbox}
\begin{tcolorbox}
\textsubscript{22} Він не вірить, що вернеться від темноти, й він вичікується для меча.
\end{tcolorbox}
\begin{tcolorbox}
\textsubscript{23} Він мандрує за хлібом, та де він? Знає він, що для нього встановлений день темноти...
\end{tcolorbox}
\begin{tcolorbox}
\textsubscript{24} Страшать його утиск та гноблення, хапають його, немов цар, що готовий до бою,
\end{tcolorbox}
\begin{tcolorbox}
\textsubscript{25} бо руку свою простягав він на Бога, і повставав на Всемогутнього,
\end{tcolorbox}
\begin{tcolorbox}
\textsubscript{26} проти Нього твердою він шиєю бігав, товстими хребтами щитів своїх.
\end{tcolorbox}
\begin{tcolorbox}
\textsubscript{27} Бо закрив він обличчя своє своїм салом, і боки обклав своїм жиром,
\end{tcolorbox}
\begin{tcolorbox}
\textsubscript{28} і сидів у містах поруйнованих, у домах тих, що в них не сидять, що на купи каміння призначені.
\end{tcolorbox}
\begin{tcolorbox}
\textsubscript{29} Він не буде багатий, і не встоїться сила його, і по землі не поширяться їхні маєтки.
\end{tcolorbox}
\begin{tcolorbox}
\textsubscript{30} Не вступиться з темности він, полум'я висушить парост його, й духом уст Його буде він схоплений.
\end{tcolorbox}
\begin{tcolorbox}
\textsubscript{31} Хай не вірить в марноту заблуканий, бо марнотою буде заплата йому,
\end{tcolorbox}
\begin{tcolorbox}
\textsubscript{32} вона виповниться не за днів його, а його верховіття не буде зелене!
\end{tcolorbox}
\begin{tcolorbox}
\textsubscript{33} Поскидає насиллям, немов виноград, недозрілість свою, поронить він квіття своє, як оливка,
\end{tcolorbox}
\begin{tcolorbox}
\textsubscript{34} бо збори безбожних спустошені будуть, а огонь пожере дім хабарника:
\end{tcolorbox}
\begin{tcolorbox}
\textsubscript{35} він злом вагітніє, й породить марноту, й оману готує утроба його...
\end{tcolorbox}
\subsection{CHAPTER 16}
\begin{tcolorbox}
\textsubscript{1} А Йов відповів та й сказав:
\end{tcolorbox}
\begin{tcolorbox}
\textsubscript{2} Чув я такого багато, даремні розрадники всі ви!
\end{tcolorbox}
\begin{tcolorbox}
\textsubscript{3} Чи настане кінець вітряним цим словам? Або що зміцнило тебе, що так відповідаєш?
\end{tcolorbox}
\begin{tcolorbox}
\textsubscript{4} І я говорив би, як ви, якби ви на місці моєму були, я додав би словами на вас, і головою своєю кивав би на вас,
\end{tcolorbox}
\begin{tcolorbox}
\textsubscript{5} устами своїми зміцняв би я вас, і не стримав би рух своїх губ на розраду!
\end{tcolorbox}
\begin{tcolorbox}
\textsubscript{6} Якщо я говоритиму, біль мій не стримається, а якщо перестану, що відійде від мене?
\end{tcolorbox}
\begin{tcolorbox}
\textsubscript{7} Та тепер ось Він змучив мене: Всю громаду мою Ти спустошив,
\end{tcolorbox}
\begin{tcolorbox}
\textsubscript{8} і поморщив мене, і це стало за свідчення, і змарнілість моя проти мене повстала, і очевидьки мені докоряє!
\end{tcolorbox}
\begin{tcolorbox}
\textsubscript{9} Його гнів мене шарпає та ненавидить мене, скрегоче на мене зубами своїми, мій ворог вигострює очі свої проти мене...
\end{tcolorbox}
\begin{tcolorbox}
\textsubscript{10} Вони пащі свої роззявляють на мене, б'ють ганебно по щоках мене, збираються разом на мене:
\end{tcolorbox}
\begin{tcolorbox}
\textsubscript{11} Бог злочинцеві видав мене, і кинув у руки безбожних мене...
\end{tcolorbox}
\begin{tcolorbox}
\textsubscript{12} Спокійний я був, та тремтячим мене Він зробив... І за шию вхопив Він мене й розторощив мене, та й поставив мене Собі ціллю:
\end{tcolorbox}
\begin{tcolorbox}
\textsubscript{13} Його стрільці мене оточили, розриває нирки мої Він не жалівши, мою жовч виливає на землю...
\end{tcolorbox}
\begin{tcolorbox}
\textsubscript{14} Він робить пролім на проломі в мені, Він на мене біжить, як силач...
\end{tcolorbox}
\begin{tcolorbox}
\textsubscript{15} Верету пошив я на шкіру свою та під порох знизив свою голову...
\end{tcolorbox}
\begin{tcolorbox}
\textsubscript{16} Зашарілось обличчя моє від плачу, й на повіках моїх залягла смертна тінь,
\end{tcolorbox}
\begin{tcolorbox}
\textsubscript{17} хоч насильства немає в долонях моїх, і чиста молитва моя!
\end{tcolorbox}
\begin{tcolorbox}
\textsubscript{18} Не прикрий, земле, крови моєї, і хай місця не буде для зойку мого,
\end{tcolorbox}
\begin{tcolorbox}
\textsubscript{19} бо тепер ось на небі мій Свідок, Самовидець мій на висоті...
\end{tcolorbox}
\begin{tcolorbox}
\textsubscript{20} Глузливці мої, мої друзі, моє око до Бога сльозить,
\end{tcolorbox}
\begin{tcolorbox}
\textsubscript{21} і нехай Він дозволить людині змагання із Богом, як між сином людським і ближнім його,
\end{tcolorbox}
\begin{tcolorbox}
\textsubscript{22} бо почислені роки минуть, і піду я дорогою, та й не вернусь...
\end{tcolorbox}
\subsection{CHAPTER 17}
\begin{tcolorbox}
\textsubscript{1} Мій дух заламавсь, мої дні погасають, зостались мені самі гроби!...
\end{tcolorbox}
\begin{tcolorbox}
\textsubscript{2} Дійсно, насмішки зо мною, й моє око в розгірченні їхнім ночує...
\end{tcolorbox}
\begin{tcolorbox}
\textsubscript{3} Поклади, дай заставу за мене Ти Сам, хто ж то той, що умову зо мною заб'є по руках?
\end{tcolorbox}
\begin{tcolorbox}
\textsubscript{4} Бо від розуміння закрив Ти їх серце тому не звеличуєш їх.
\end{tcolorbox}
\begin{tcolorbox}
\textsubscript{5} Він призначує ближніх на поділ, а очі синів його темніють,
\end{tcolorbox}
\begin{tcolorbox}
\textsubscript{6} Він поставив мене за прислів'я в народів, і став я таким, на якого плюють...
\end{tcolorbox}
\begin{tcolorbox}
\textsubscript{7} З безталання потемніло око моє, а всі члени мої як та тінь...
\end{tcolorbox}
\begin{tcolorbox}
\textsubscript{8} Праведники остовпіють на це, і невинний встає на безбожного.
\end{tcolorbox}
\begin{tcolorbox}
\textsubscript{9} І праведний буде держатись дороги своєї, а хто чисторукий побільшиться в силі.
\end{tcolorbox}
\begin{tcolorbox}
\textsubscript{10} Але всі ви повернетеся, і приходьте, та я не знаходжу між вами розумного...
\end{tcolorbox}
\begin{tcolorbox}
\textsubscript{11} Мої дні проминули, порвалися думи мої, мого серця маєток,
\end{tcolorbox}
\begin{tcolorbox}
\textsubscript{12} вони мені ніч обертають на день, наближують світло при темряві!
\end{tcolorbox}
\begin{tcolorbox}
\textsubscript{13} Якщо сподіваюсь, то тільки шеолу, як дому свого, в темноті постелю своє ложе...
\end{tcolorbox}
\begin{tcolorbox}
\textsubscript{14} До гробу я кличу: О батьку ти мій! До черви: Моя мамо та сестро моя!...
\end{tcolorbox}
\begin{tcolorbox}
\textsubscript{15} Де ж тоді та надія моя? А надія моя, хто побачить її?
\end{tcolorbox}
\begin{tcolorbox}
\textsubscript{16} До шеолових засувів зійде вона, коли зійдемо разом до пороху...
\end{tcolorbox}
\subsection{CHAPTER 18}
\begin{tcolorbox}
\textsubscript{1} І заговорив шух'янин Білдад та й сказав:
\end{tcolorbox}
\begin{tcolorbox}
\textsubscript{2} Як довго ви будете пастками класти слова? Розміркуйте, а потім собі поговоримо!
\end{tcolorbox}
\begin{tcolorbox}
\textsubscript{3} Чому пораховані ми, як худоба? Чому в ваших очах ми безумні?
\end{tcolorbox}
\begin{tcolorbox}
\textsubscript{4} О ти, що розшарпуєш душу свою в своїм гніві, чи для тебе земля опустіє, а скеля осунеться з місця свого?
\end{tcolorbox}
\begin{tcolorbox}
\textsubscript{5} Таж світильник безбожних погасне, і не буде світитися іскра огню його:
\end{tcolorbox}
\begin{tcolorbox}
\textsubscript{6} його світло стемніє в наметі, і згасне на ньому світильник його,
\end{tcolorbox}
\begin{tcolorbox}
\textsubscript{7} стануть тісні кроки сили його, і вдарить його власна рада!...
\end{tcolorbox}
\begin{tcolorbox}
\textsubscript{8} Бо він кинений в пастку ногами своїми, і на ґраті він буде ходити:
\end{tcolorbox}
\begin{tcolorbox}
\textsubscript{9} пастка схопить за стопу його, зміцниться сітка на ньому,
\end{tcolorbox}
\begin{tcolorbox}
\textsubscript{10} на нього захований шнур на землі, а пастка на нього на стежці...
\end{tcolorbox}
\begin{tcolorbox}
\textsubscript{11} Страхіття жахають його звідусіль, і женуться за ним по слідах.
\end{tcolorbox}
\begin{tcolorbox}
\textsubscript{12} Його сила голодною буде, а нещастя при боці його приготовлене.
\end{tcolorbox}
\begin{tcolorbox}
\textsubscript{13} Його шкіра поїджена буде хворобою, поїсть члени його первороджений смерти.
\end{tcolorbox}
\begin{tcolorbox}
\textsubscript{14} Відірвана буде безпека його від намету його, а Ти до царя жахів його приведеш...
\end{tcolorbox}
\begin{tcolorbox}
\textsubscript{15} Він перебуває в наметі своєму, який не його, на мешкання його буде кинена сірка.
\end{tcolorbox}
\begin{tcolorbox}
\textsubscript{16} Здолу посохнуть коріння його, а згори його віття зів'яне.
\end{tcolorbox}
\begin{tcolorbox}
\textsubscript{17} Його пам'ять загине з землі, а на вулиці ймення не буде йому.
\end{tcolorbox}
\begin{tcolorbox}
\textsubscript{18} Заженуть його з світла до темряви, і ввесь світ проганяє його.
\end{tcolorbox}
\begin{tcolorbox}
\textsubscript{19} У нього немає в народі нащадка, ні внука, і немає останку в місцях його мешкання.
\end{tcolorbox}
\begin{tcolorbox}
\textsubscript{20} На згадку про день його остовпівали останні, за волосся ж хапались давніші...
\end{tcolorbox}
\begin{tcolorbox}
\textsubscript{21} Ось такі то мешкання неправедного, і це місце того, хто Бога не знає!
\end{tcolorbox}
\subsection{CHAPTER 19}
\begin{tcolorbox}
\textsubscript{1} А Йов відповів та й сказав:
\end{tcolorbox}
\begin{tcolorbox}
\textsubscript{2} Аж доки смутити ви будете душу мою, та душити словами мене?
\end{tcolorbox}
\begin{tcolorbox}
\textsubscript{3} Десять раз це мене ви соромите, гнобити мене не стидаєтесь!...
\end{tcolorbox}
\begin{tcolorbox}
\textsubscript{4} Якщо справді зблудив я, то мій гріх при мені позостане.
\end{tcolorbox}
\begin{tcolorbox}
\textsubscript{5} Чи ви величаєтесь справді над мною, і виказуєте мою ганьбу на мене?
\end{tcolorbox}
\begin{tcolorbox}
\textsubscript{6} Знайте тоді, що Бог скривдив мене, і тенета Свої розточив надо мною!
\end{tcolorbox}
\begin{tcolorbox}
\textsubscript{7} Ось ґвалт! я кричу, та не відповідає ніхто, голошу, та немає суду!...
\end{tcolorbox}
\begin{tcolorbox}
\textsubscript{8} Він дорогу мою оточив і я не перейду, Він поклав на стежки мої темряву!
\end{tcolorbox}
\begin{tcolorbox}
\textsubscript{9} Він стягнув з мене славу мою і вінця зняв мені з голови!
\end{tcolorbox}
\begin{tcolorbox}
\textsubscript{10} Звідусіль Він ламає мене, і я йду, надію мою, як те дерево, вивернув Він...
\end{tcolorbox}
\begin{tcolorbox}
\textsubscript{11} І на мене Свій гнів запалив, і зарахував Він мене до Своїх ворогів:
\end{tcolorbox}
\begin{tcolorbox}
\textsubscript{12} полки Його разом приходять, і торують на мене дорогу свою, і таборують навколо намету мого...
\end{tcolorbox}
\begin{tcolorbox}
\textsubscript{13} Віддалив Він від мене братів моїх, а знайомі мої почужіли для мене,
\end{tcolorbox}
\begin{tcolorbox}
\textsubscript{14} мої ближні відстали, і забули про мене знайомі мої...
\end{tcolorbox}
\begin{tcolorbox}
\textsubscript{15} Мешканці дому мого, і служниці мої за чужого вважають мене, чужаком я став в їхніх очах...
\end{tcolorbox}
\begin{tcolorbox}
\textsubscript{16} Я кличу свойого раба і він відповіді не дає, хоч своїми устами благаю його...
\end{tcolorbox}
\begin{tcolorbox}
\textsubscript{17} Мій дух став бридкий для моєї дружини, а мій запах синам моєї утроби...
\end{tcolorbox}
\begin{tcolorbox}
\textsubscript{18} Навіть діти малі зневажають мене, коли я встаю, то глузують із мене...
\end{tcolorbox}
\begin{tcolorbox}
\textsubscript{19} Мої всі повірники бридяться мною, а кого я кохав обернулись на мене...
\end{tcolorbox}
\begin{tcolorbox}
\textsubscript{20} До шкіри моєї й до тіла мого приліпилися кості мої, ще біля зубів лиш зосталася шкіра моя...
\end{tcolorbox}
\begin{tcolorbox}
\textsubscript{21} Змилуйтеся надо мною, о, змилуйтеся надо мною ви, ближні мої, бо Божа рука доторкнулась мене!...
\end{tcolorbox}
\begin{tcolorbox}
\textsubscript{22} Чого ви мене переслідуєте, немов Бог, і не насичуєтесь моїм тілом?
\end{tcolorbox}
\begin{tcolorbox}
\textsubscript{23} О, коли б записати слова мої, о, коли б були в книжці вони позазначувані,
\end{tcolorbox}
\begin{tcolorbox}
\textsubscript{24} коли б рильцем залізним та оливом в скелі навіки вони були витесані!
\end{tcolorbox}
\begin{tcolorbox}
\textsubscript{25} Та я знаю, що мій Викупитель живий, і останнього дня Він підійме із пороху
\end{tcolorbox}
\begin{tcolorbox}
\textsubscript{26} цю шкіру мою, яка розпадається, і з тіла свойого я Бога побачу,
\end{tcolorbox}
\begin{tcolorbox}
\textsubscript{27} сам я побачу Його, й мої очі побачать, а не очі чужі... Тануть нирки мої в моїм нутрі!...
\end{tcolorbox}
\begin{tcolorbox}
\textsubscript{28} Коли скажете ви: Нащо будемо гнати його, коли корень справи знаходиться в ньому!
\end{tcolorbox}
\begin{tcolorbox}
\textsubscript{29} то побійтесь меча собі ви, бо гнів за провину то меч, щоб ви знали, що є ще Суддя!...
\end{tcolorbox}
\subsection{CHAPTER 20}
\begin{tcolorbox}
\textsubscript{1} І відповів нааматянин Цофар та й сказав:
\end{tcolorbox}
\begin{tcolorbox}
\textsubscript{2} Тому то думки мої відповідати мене навертають, і тому то в мені цей мій поспіх!
\end{tcolorbox}
\begin{tcolorbox}
\textsubscript{3} Соромливу нагану собі я почув, та дух з мого розуму відповідає мені.
\end{tcolorbox}
\begin{tcolorbox}
\textsubscript{4} Чи знаєш ти те, що від вічности, відколи людина на землі була поставлена,
\end{tcolorbox}
\begin{tcolorbox}
\textsubscript{5} то спів несправедливих короткий, а радість безбожного тільки на хвилю?
\end{tcolorbox}
\begin{tcolorbox}
\textsubscript{6} Якщо піднесеться величність його аж до неба, а його голова аж до хмари досягне,
\end{tcolorbox}
\begin{tcolorbox}
\textsubscript{7} проте він загине навіки, немов його гній, хто бачив його, запитає: де він?
\end{tcolorbox}
\begin{tcolorbox}
\textsubscript{8} Немов сон улетить і не знайдуть його, мов видіння нічне, він сполошений буде:
\end{tcolorbox}
\begin{tcolorbox}
\textsubscript{9} його бачило око, та бачити більше не буде, і вже не побачить його його місце...
\end{tcolorbox}
\begin{tcolorbox}
\textsubscript{10} Сини його запобігатимуть ласки в нужденних, а руки його позвертають маєток його...
\end{tcolorbox}
\begin{tcolorbox}
\textsubscript{11} Повні кості його молодечости, та до пороху з ним вона ляже!
\end{tcolorbox}
\begin{tcolorbox}
\textsubscript{12} Якщо в устах його зло солодке, його він таїть під своїм язиком,
\end{tcolorbox}
\begin{tcolorbox}
\textsubscript{13} над ним милосердиться та не пускає його, і тримає його в своїх устах,
\end{tcolorbox}
\begin{tcolorbox}
\textsubscript{14} то цей хліб в його нутрощах зміниться, стане він жовчю зміїною в нутрі його!...
\end{tcolorbox}
\begin{tcolorbox}
\textsubscript{15} Він маєток чужого ковтав, але його виблює: Бог виганяє його із утроби його...
\end{tcolorbox}
\begin{tcolorbox}
\textsubscript{16} Отруту зміїну він ссатиме, гадючий язик його вб'є!
\end{tcolorbox}
\begin{tcolorbox}
\textsubscript{17} Він річкових джерел не побачить, струмків меду та молока.
\end{tcolorbox}
\begin{tcolorbox}
\textsubscript{18} Позвертає він працю чужу, і її не ковтне, як і маєток, набутий з виміни своєї, жувати не буде...
\end{tcolorbox}
\begin{tcolorbox}
\textsubscript{19} Бо він переслідував, кидав убогих, він дім грабував, хоч не ставив його!
\end{tcolorbox}
\begin{tcolorbox}
\textsubscript{20} Бо спокою не знав він у нутрі своїм, і свого наймилішого не збереже.
\end{tcolorbox}
\begin{tcolorbox}
\textsubscript{21} Немає останку з обжирства його, тому нетривале добро його все:
\end{tcolorbox}
\begin{tcolorbox}
\textsubscript{22} за повні достатку його буде тісно йому, рука кожного скривдженого прийде на нього!
\end{tcolorbox}
\begin{tcolorbox}
\textsubscript{23} Хай наповнена буде утроба його, та пошле Він на нього жар гніву Свого, і буде дощити на нього недугами його...
\end{tcolorbox}
\begin{tcolorbox}
\textsubscript{24} Він буде втікати від зброї залізної, та прониже його мідний лук...
\end{tcolorbox}
\begin{tcolorbox}
\textsubscript{25} Він стане меча витягати, і вийде він із тіла, та держак його вийде із жовчі його, і перестрах на нього впаде!
\end{tcolorbox}
\begin{tcolorbox}
\textsubscript{26} При скарбах його всі нещастя заховані, його буде жерти огонь не роздмухуваний, позостале в наметі його буде знищене...
\end{tcolorbox}
\begin{tcolorbox}
\textsubscript{27} Небо відкриє його беззаконня, а земля проти нього повстане,
\end{tcolorbox}
\begin{tcolorbox}
\textsubscript{28} урожай його дому втече, розпливеться в день гніву Його...
\end{tcolorbox}
\begin{tcolorbox}
\textsubscript{29} Оце доля від Бога людині безбожній, і спадщина, обіцяна Богом для неї!
\end{tcolorbox}
\subsection{CHAPTER 21}
\begin{tcolorbox}
\textsubscript{1} А Йов відповів та й сказав:
\end{tcolorbox}
\begin{tcolorbox}
\textsubscript{2} Уважно послухайте слово моє, і нехай буде мені це розрадою вашою!
\end{tcolorbox}
\begin{tcolorbox}
\textsubscript{3} Перетерпіть мені, а я промовлятиму, по промові ж моїй насміхатися будеш.
\end{tcolorbox}
\begin{tcolorbox}
\textsubscript{4} Хіба до людини моє нарікання? Чи не мав би чого стати нетерпеливим мій дух?
\end{tcolorbox}
\begin{tcolorbox}
\textsubscript{5} Оберніться до мене й жахніться, та руку на уста свої покладіть...
\end{tcolorbox}
\begin{tcolorbox}
\textsubscript{6} І якщо я згадаю про це, то жахаюсь, і морозом проймається тіло моє...
\end{tcolorbox}
\begin{tcolorbox}
\textsubscript{7} Чого несправедливі живуть, доживають до віку, й багатством зміцняються?
\end{tcolorbox}
\begin{tcolorbox}
\textsubscript{8} Насіння їх міцно стоїть перед ними, при них, а їхні нащадки на їхніх очах...
\end{tcolorbox}
\begin{tcolorbox}
\textsubscript{9} Доми їхні то спокій від страху, і над ними нема бича Божого.
\end{tcolorbox}
\begin{tcolorbox}
\textsubscript{10} Спинається бик його, і не даремно, зачинає корова його, й не скидає.
\end{tcolorbox}
\begin{tcolorbox}
\textsubscript{11} Вони випускають своїх молодят, як отару, а їх діти вибрикують.
\end{tcolorbox}
\begin{tcolorbox}
\textsubscript{12} Вони голос здіймають при бубні та цитрі, і веселяться при звуку сопілки.
\end{tcolorbox}
\begin{tcolorbox}
\textsubscript{13} Провадять в добрі свої дні, і сходять в спокої в шеол.
\end{tcolorbox}
\begin{tcolorbox}
\textsubscript{14} А до Бога говорять вони: Уступися від нас, ми ж доріг Твоїх знати не хочем!
\end{tcolorbox}
\begin{tcolorbox}
\textsubscript{15} Що таке Всемогутній, що будем служити Йому? І що скористаєм, як будем благати Його?
\end{tcolorbox}
\begin{tcolorbox}
\textsubscript{16} Та не в їхній руці добро їхнє, далека від мене порада безбожних...
\end{tcolorbox}
\begin{tcolorbox}
\textsubscript{17} Як часто світильник безбожним згасає, і приходить на них їх нещастя? Він приділює в гніві Своїм на них пастки!
\end{tcolorbox}
\begin{tcolorbox}
\textsubscript{18} Вони будуть, немов та солома на вітрі, і немов та полова, що буря схопила її!
\end{tcolorbox}
\begin{tcolorbox}
\textsubscript{19} Бог ховає синам його кривду Свою та нехай надолужить самому йому, і він знатиме!
\end{tcolorbox}
\begin{tcolorbox}
\textsubscript{20} Нехай його очі побачать нещастя його, й бодай сам він пив гнів Всемогутнього!
\end{tcolorbox}
\begin{tcolorbox}
\textsubscript{21} Яке бо старання його про родину по ньому, як для нього число його місяців вже перелічене?
\end{tcolorbox}
\begin{tcolorbox}
\textsubscript{22} Чи буде хто Бога навчати знання, Його, що й небесних судитиме?
\end{tcolorbox}
\begin{tcolorbox}
\textsubscript{23} Оцей в повній силі своїй помирає, увесь він спокійний та мирний,
\end{tcolorbox}
\begin{tcolorbox}
\textsubscript{24} діжки його повні були молока, а мізок костей його свіжий.
\end{tcolorbox}
\begin{tcolorbox}
\textsubscript{25} А цей помирає з душею огірченою, і доброго не споживав він,
\end{tcolorbox}
\begin{tcolorbox}
\textsubscript{26} та порохом будуть лежати обоє вони, і черва їх покриє...
\end{tcolorbox}
\begin{tcolorbox}
\textsubscript{27} Тож я знаю думки ваші й задуми, що хочете кривдити ними мене.
\end{tcolorbox}
\begin{tcolorbox}
\textsubscript{28} Бож питаєте ви: Де князів дім, і де намет пробування безбожних?
\end{tcolorbox}
\begin{tcolorbox}
\textsubscript{29} Тож спитайтеся тих, що дорогою йдуть, а їхніх ознак не затаюйте:
\end{tcolorbox}
\begin{tcolorbox}
\textsubscript{30} що буває врятований злий в день загибелі, на день гніву відводиться в захист!
\end{tcolorbox}
\begin{tcolorbox}
\textsubscript{31} Хто йому розповість у лице про дорогу його? А коли наробив, хто йому надолужить?
\end{tcolorbox}
\begin{tcolorbox}
\textsubscript{32} І на кладовище буде проваджений він, і про могилу подбають...
\end{tcolorbox}
\begin{tcolorbox}
\textsubscript{33} Скиби долини солодкі йому, і тягнеться кожна людина за ним, а тим, хто попереду нього, немає числа...
\end{tcolorbox}
\begin{tcolorbox}
\textsubscript{34} І як ви мене потішаєте марністю, коли з ваших відповідей зостається сама тільки фальш?...
\end{tcolorbox}
\subsection{CHAPTER 22}
\begin{tcolorbox}
\textsubscript{1} І заговорив теманянин Еліфаз та й сказав:
\end{tcolorbox}
\begin{tcolorbox}
\textsubscript{2} Чи для Бога людина корисна? Бо мудрий корисний самому собі!
\end{tcolorbox}
\begin{tcolorbox}
\textsubscript{3} Хіба Всемогутній бажає, щоб ти ніби праведним був? І що за користь Йому, як дороги свої ти вважаєш невинними сам?
\end{tcolorbox}
\begin{tcolorbox}
\textsubscript{4} Чи Він буде карати, тебе боячись, і чи піде з тобою на суд?
\end{tcolorbox}
\begin{tcolorbox}
\textsubscript{5} Хіба твоє зло не велике? Таж твоїм беззаконням немає кінця!
\end{tcolorbox}
\begin{tcolorbox}
\textsubscript{6} Таж з братів своїх брав ти заставу даремно, а з нагого одежу стягав!
\end{tcolorbox}
\begin{tcolorbox}
\textsubscript{7} Не поїв ти водою знеможеного, і від голодного стримував хліб...
\end{tcolorbox}
\begin{tcolorbox}
\textsubscript{8} А сильна людина то їй оцей край, і почесний у ньому сидітиме.
\end{tcolorbox}
\begin{tcolorbox}
\textsubscript{9} Ти напорожньо вдів відсилав, і сирітські рамена гнобились,
\end{tcolorbox}
\begin{tcolorbox}
\textsubscript{10} тому пастки тебе оточили, і жахає тебе наглий страх,
\end{tcolorbox}
\begin{tcolorbox}
\textsubscript{11} твоє світло стемніло, нічого не бачиш, і велика вода закриває тебе...
\end{tcolorbox}
\begin{tcolorbox}
\textsubscript{12} Чи ж Бог не високий, як небо? Та на зорі угору поглянь, які стали високі вони!
\end{tcolorbox}
\begin{tcolorbox}
\textsubscript{13} А ти кажеш: Що відає Бог? Чи судитиме Він через млу?
\end{tcolorbox}
\begin{tcolorbox}
\textsubscript{14} Хмари завіса Йому, й Він не бачить, і ходить по крузі небесному.
\end{tcolorbox}
\begin{tcolorbox}
\textsubscript{15} Чи ти будеш триматись дороги відвічної, що нею ступали безбожні,
\end{tcolorbox}
\begin{tcolorbox}
\textsubscript{16} що невчасно були вони згублені, що річка розлита, підвалина їх,
\end{tcolorbox}
\begin{tcolorbox}
\textsubscript{17} що до Бога казали вони: Відступися від нас! та: Що зробить для нас Всемогутній?
\end{tcolorbox}
\begin{tcolorbox}
\textsubscript{18} А Він доми їхні наповнив добром!... Але віддалилась від мене порада безбожних!
\end{tcolorbox}
\begin{tcolorbox}
\textsubscript{19} Справедливі це бачать та тішаться, і насміхається з нього невинний:
\end{tcolorbox}
\begin{tcolorbox}
\textsubscript{20} Справді вигублений наш противник, а останок їх вижер огонь!
\end{tcolorbox}
\begin{tcolorbox}
\textsubscript{21} Заприязнися із Ним, та й май спокій, цим прийде на тебе добро.
\end{tcolorbox}
\begin{tcolorbox}
\textsubscript{22} Закона візьми з Його уст, а слова Його в серце своє поклади.
\end{tcolorbox}
\begin{tcolorbox}
\textsubscript{23} Якщо вернешся до Всемогутнього, будеш збудований, і віддалиш беззаконня з наметів своїх.
\end{tcolorbox}
\begin{tcolorbox}
\textsubscript{24} І викинь до пороху золото, і мов камінь з потоку офірське те золото,
\end{tcolorbox}
\begin{tcolorbox}
\textsubscript{25} і буде тобі Всемогутній за золото та за срібло блискуче тобі!
\end{tcolorbox}
\begin{tcolorbox}
\textsubscript{26} Бо тоді Всемогутнього ти покохаєш і до Бога підіймеш обличчя своє,
\end{tcolorbox}
\begin{tcolorbox}
\textsubscript{27} будеш благати Його й Він почує тебе, і ти обітниці свої надолужиш.
\end{tcolorbox}
\begin{tcolorbox}
\textsubscript{28} А що постановиш, то виповниться те тобі, й на дорогах твоїх буде сяяти світло.
\end{tcolorbox}
\begin{tcolorbox}
\textsubscript{29} Бо знижує Він спину пишного, хто ж смиренний, тому помагає.
\end{tcolorbox}
\begin{tcolorbox}
\textsubscript{30} Рятує Він і небезвинного, і той чистотою твоїх рук урятований буде.
\end{tcolorbox}
\subsection{CHAPTER 23}
\begin{tcolorbox}
\textsubscript{1} А Йов відповів та й сказав:
\end{tcolorbox}
\begin{tcolorbox}
\textsubscript{2} Моя мова й сьогодні гірка, тяжче страждання моє за стогнання мої...
\end{tcolorbox}
\begin{tcolorbox}
\textsubscript{3} О, якби то я знав, де Його я знайду, то прийшов би до місця Його пробування!
\end{tcolorbox}
\begin{tcolorbox}
\textsubscript{4} Я б перед обличчям Його свою справу поклав, а уста свої я наповнив би доводами,
\end{tcolorbox}
\begin{tcolorbox}
\textsubscript{5} розізнав би слова, що мені відповість, і я зрозумів би, що скаже мені.
\end{tcolorbox}
\begin{tcolorbox}
\textsubscript{6} Чи зо мною на прю Він з великою силою стане? О ні, тільки б увагу звернув Він на мене!
\end{tcolorbox}
\begin{tcolorbox}
\textsubscript{7} Справедливий судився б там з Ним, я ж назавжди б звільнивсь від свойого Судді.
\end{tcolorbox}
\begin{tcolorbox}
\textsubscript{8} Та піду я на схід і немає Його, а на захід удамся Його не побачу,
\end{tcolorbox}
\begin{tcolorbox}
\textsubscript{9} на півночі шукаю Його й не вхоплю, збочу на південь і не добачаю...
\end{tcolorbox}
\begin{tcolorbox}
\textsubscript{10} А Він знає дорогу, яка при мені, хай би випробував Він мене, мов те золото, вийду!
\end{tcolorbox}
\begin{tcolorbox}
\textsubscript{11} Трималась нога моя коло стопи Його, дороги Його я держався й не збочив.
\end{tcolorbox}
\begin{tcolorbox}
\textsubscript{12} Я не відступався від заповідей Його губ, над уставу свою я ховав слова уст Його.
\end{tcolorbox}
\begin{tcolorbox}
\textsubscript{13} Але Він при одному, й хто заверне Його? Як чого зажадає душа Його, те Він учинить:
\end{tcolorbox}
\begin{tcolorbox}
\textsubscript{14} бо Він виконає, що про мене призначив, і в Нього багато такого, як це!
\end{tcolorbox}
\begin{tcolorbox}
\textsubscript{15} Тому перед обличчям Його я тремчу, розважаю й жахаюсь Його...
\end{tcolorbox}
\begin{tcolorbox}
\textsubscript{16} А Бог пом'якшив моє серце, і Всемогутній мене настрашив,
\end{tcolorbox}
\begin{tcolorbox}
\textsubscript{17} бо не знищений я від темноти, ані від обличчя свого, що темність закрила його!
\end{tcolorbox}
\subsection{CHAPTER 24}
\begin{tcolorbox}
\textsubscript{1} Для чого часи не заховані від Всемогутнього? Ті ж, що знають Його, Його днів не побачать!
\end{tcolorbox}
\begin{tcolorbox}
\textsubscript{2} Пересовують межі безбожні, стадо грабують вони та пасуть,
\end{tcolorbox}
\begin{tcolorbox}
\textsubscript{3} займають осла в сиротини, беруть у заставу вола від удовиць,
\end{tcolorbox}
\begin{tcolorbox}
\textsubscript{4} вони бідних з дороги спихають, разом мусять ховатися збіджені краю...
\end{tcolorbox}
\begin{tcolorbox}
\textsubscript{5} Тож вони, бідарі, немов дикі осли на пустині, виходять на працю свою, здобичі шукаючи, степ йому хліба дає для дітей...
\end{tcolorbox}
\begin{tcolorbox}
\textsubscript{6} На полі вночі вони жнуть, і збирають собі виноград у безбожного,
\end{tcolorbox}
\begin{tcolorbox}
\textsubscript{7} наго ночують вони, без одежі, і не мають вкриття собі в холоді,
\end{tcolorbox}
\begin{tcolorbox}
\textsubscript{8} мокнуть від зливи гірської, а заслони не маючи, скелю вони обіймають...
\end{tcolorbox}
\begin{tcolorbox}
\textsubscript{9} Сироту відривають від перс, і в заставу беруть від убогого...
\end{tcolorbox}
\begin{tcolorbox}
\textsubscript{10} Ходять наго вони, без вбрання, і голодними носять снопи.
\end{tcolorbox}
\begin{tcolorbox}
\textsubscript{11} Хоч між мурами їхніми роблять оливу, топчуть чавила, та прагнуть вони!
\end{tcolorbox}
\begin{tcolorbox}
\textsubscript{12} Стогнуть люди із міста, і кричить душа вбиваних, а Бог на це зло не звертає уваги...
\end{tcolorbox}
\begin{tcolorbox}
\textsubscript{13} Вони проти світла бунтують, не знають доріг Його, і на стежках Його не сидять.
\end{tcolorbox}
\begin{tcolorbox}
\textsubscript{14} На світанку встає душогуб, замордовує бідного та злидаря, а ніч він проводить, як злодій...
\end{tcolorbox}
\begin{tcolorbox}
\textsubscript{15} А перелюбника око чекає смеркання, говорячи: Не побачить мене жодне око! і заслону кладе на обличчя...
\end{tcolorbox}
\begin{tcolorbox}
\textsubscript{16} Підкопуються під доми в темноті, замикаються вдень, світла не знають вони,
\end{tcolorbox}
\begin{tcolorbox}
\textsubscript{17} бо ранок для них усіх разом то темрява, і знають вони жахи темряви...
\end{tcolorbox}
\begin{tcolorbox}
\textsubscript{18} Такий легкий він на поверхні води, на землі їхня частка проклята, не вернеться він на дорогу садів-виноградів...
\end{tcolorbox}
\begin{tcolorbox}
\textsubscript{19} Як посуха та спека їдять сніжну воду, так шеол поїсть грішників!
\end{tcolorbox}
\begin{tcolorbox}
\textsubscript{20} Забуде його лоно матері, буде жерти черва його, мов солодощі, більше не буде він згадуваний, і безбожник поламаний буде, мов дерево!...
\end{tcolorbox}
\begin{tcolorbox}
\textsubscript{21} Чинить зло для бездітної він, щоб вона не родила, і вдовиці не зробить добра.
\end{tcolorbox}
\begin{tcolorbox}
\textsubscript{22} А міццю своєю він тягне могутніх, коли він встає, то ніхто вже не певний свойого життя!
\end{tcolorbox}
\begin{tcolorbox}
\textsubscript{23} Бог дає йому все на безпеку, і на те він спирається, та очі Його бачать їхні дороги:
\end{tcolorbox}
\begin{tcolorbox}
\textsubscript{24} підіймуться трохи й немає вже їх, бо понижені... Як усе, вони гинуть, і зрізуються, немов та колоскова головка...
\end{tcolorbox}
\begin{tcolorbox}
\textsubscript{25} Якщо ж ні, то хто зробить мене неправдомовцем, а слово моє на марноту оберне?
\end{tcolorbox}
\subsection{CHAPTER 25}
\begin{tcolorbox}
\textsubscript{1} І заговорив шух'янин Білдад та й сказав:
\end{tcolorbox}
\begin{tcolorbox}
\textsubscript{2} Панування та острах у Нього, Який на висотах Своїх чинить мир.
\end{tcolorbox}
\begin{tcolorbox}
\textsubscript{3} Чи війську Його є число? І над ким Його світло не сходить?
\end{tcolorbox}
\begin{tcolorbox}
\textsubscript{4} І як може людина бути праведною перед Богом, і як може бути чистим, від жінки народжений?
\end{tcolorbox}
\begin{tcolorbox}
\textsubscript{5} Таж Йому навіть місяць не світить, і в очах Його й зорі не ясні!
\end{tcolorbox}
\begin{tcolorbox}
\textsubscript{6} Що ж тоді людина ота, червяк, чи син людський хробак?...
\end{tcolorbox}
\subsection{CHAPTER 26}
\begin{tcolorbox}
\textsubscript{1} А Йов відповів та й сказав:
\end{tcolorbox}
\begin{tcolorbox}
\textsubscript{2} Як безсилому ти допоміг, як рамено підпер ти неможному?
\end{tcolorbox}
\begin{tcolorbox}
\textsubscript{3} Що ти радив немудрому, й яку раду подав багатьом?
\end{tcolorbox}
\begin{tcolorbox}
\textsubscript{4} Кому ти слова говорив, і чий дух вийшов з тебе?
\end{tcolorbox}
\begin{tcolorbox}
\textsubscript{5} Рефаїми тремтять під водою й всі її мешканці.
\end{tcolorbox}
\begin{tcolorbox}
\textsubscript{6} Голий шеол перед Ним, і нема покриття Аваддону.
\end{tcolorbox}
\begin{tcolorbox}
\textsubscript{7} Він над порожнечею північ простяг, на нічому Він землю повісив.
\end{tcolorbox}
\begin{tcolorbox}
\textsubscript{8} Він зав'язує воду в Своїх облаках, і не розбивається хмара під ними.
\end{tcolorbox}
\begin{tcolorbox}
\textsubscript{9} Він поставив престола Свого, розтягнув над ним хмару Свою.
\end{tcolorbox}
\begin{tcolorbox}
\textsubscript{10} На поверхні води Він зазначив межу аж до границі між світлом та темрявою.
\end{tcolorbox}
\begin{tcolorbox}
\textsubscript{11} Стовпи неба тремтять та страшаться від гніву Його.
\end{tcolorbox}
\begin{tcolorbox}
\textsubscript{12} Він міццю Своєю вспокоює море, і Своїм розумом нищить Рагава.
\end{tcolorbox}
\begin{tcolorbox}
\textsubscript{13} Своїм Духом Він небо прикрасив, рука Його в ньому створила втікаючого Скорпіона.
\end{tcolorbox}
\begin{tcolorbox}
\textsubscript{14} Таж це все самі кінці дороги Його, бо ми тільки слабке шепотіння чували про Нього, грім потуги ж Його хто його зрозуміє?...
\end{tcolorbox}
\subsection{CHAPTER 27}
\begin{tcolorbox}
\textsubscript{1} І Йов далі вів мову свою та й казав:
\end{tcolorbox}
\begin{tcolorbox}
\textsubscript{2} Як живий Бог, відкинув Він право моє, і душу мою засмутив Всемогутній,
\end{tcolorbox}
\begin{tcolorbox}
\textsubscript{3} і як довго в мені ще душа моя, і дух Божий у ніздрях моїх,
\end{tcolorbox}
\begin{tcolorbox}
\textsubscript{4} неправди уста мої не говоритимуть, а язик мій не скаже омани!
\end{tcolorbox}
\begin{tcolorbox}
\textsubscript{5} Борони мене, Боже, признати вас за справедливих! Доки я не помру, своєї невинности я не відкину від себе,
\end{tcolorbox}
\begin{tcolorbox}
\textsubscript{6} за свою справедливість тримаюся міцно, й її не пущу, моє серце не буде ганьбити ні одного з днів моїх,
\end{tcolorbox}
\begin{tcolorbox}
\textsubscript{7} нехай буде мій ворог немов той безбожник, а хто повстає проти мене як кривдник!
\end{tcolorbox}
\begin{tcolorbox}
\textsubscript{8} Яка ж бо надія лукавому, коли відірве, коли візьме Бог душу його?
\end{tcolorbox}
\begin{tcolorbox}
\textsubscript{9} Чи Бог вислухає його крик, коли прийде на нього нещастя?
\end{tcolorbox}
\begin{tcolorbox}
\textsubscript{10} Чи буде втішатися він Всемогутнім? Буде кликати Бога за кожного часу?
\end{tcolorbox}
\begin{tcolorbox}
\textsubscript{11} Я вас буду навчати про Божую руку, що є у Всемогутнього я не сховаю,
\end{tcolorbox}
\begin{tcolorbox}
\textsubscript{12} таж самі ви це бачили всі, то чого ж нісенітниці плещете?
\end{tcolorbox}
\begin{tcolorbox}
\textsubscript{13} Така доля людини безбожної, це спадщина насильників, що отримають від Всемогутнього:
\end{tcolorbox}
\begin{tcolorbox}
\textsubscript{14} Як розмножаться діти його то хіба для меча, а нащадки його не наситяться хлібом!
\end{tcolorbox}
\begin{tcolorbox}
\textsubscript{15} Позосталих по нім моровиця сховає, і вдовиці його не заплачуть...
\end{tcolorbox}
\begin{tcolorbox}
\textsubscript{16} Якщо накопичить він срібла, немов того пороху, і наготує одежі, як глини,
\end{tcolorbox}
\begin{tcolorbox}
\textsubscript{17} то він наготує, а праведний вдягне, а срібло невинний поділить...
\end{tcolorbox}
\begin{tcolorbox}
\textsubscript{18} Він будує свій дім, як та міль, й як той сторож, що ставить собі куреня,
\end{tcolorbox}
\begin{tcolorbox}
\textsubscript{19} він лягає багатим, та більше не зробить того: свої очі відкриє й немає його...
\end{tcolorbox}
\begin{tcolorbox}
\textsubscript{20} Страхіття досягнуть його, мов вода, вночі буря украде його,
\end{tcolorbox}
\begin{tcolorbox}
\textsubscript{21} східній вітер його понесе і минеться, і бурею схопить його з його місця...
\end{tcolorbox}
\begin{tcolorbox}
\textsubscript{22} Оце все Він кине на нього, і не змилосердиться, і від руки Його мусить той спішно втікати!
\end{tcolorbox}
\begin{tcolorbox}
\textsubscript{23} Своїми долонями сплесне над ним, і свисне над ним з свого місця...
\end{tcolorbox}
\subsection{CHAPTER 28}
\begin{tcolorbox}
\textsubscript{1} Отож, має срібло своє джерело, і є місце для золота, де його чистять,
\end{tcolorbox}
\begin{tcolorbox}
\textsubscript{2} залізо береться із пороху, з каменя мідь виплавляється.
\end{tcolorbox}
\begin{tcolorbox}
\textsubscript{3} Людина кладе для темноти кінця, і докраю досліджує все, і шукає каміння у темряві та в смертній тіні:
\end{tcolorbox}
\begin{tcolorbox}
\textsubscript{4} ламає в копальні далеко від мешканця; забуті ногою людини, висять місця, віддалені від чоловіка.
\end{tcolorbox}
\begin{tcolorbox}
\textsubscript{5} Земля хліб із неї походить, а під нею порито, немов би огнем,
\end{tcolorbox}
\begin{tcolorbox}
\textsubscript{6} місце сапфіру каміння її, й порох золота в ній.
\end{tcolorbox}
\begin{tcolorbox}
\textsubscript{7} Стежка туди не знає її хижий птах, її око орлине не бачило,
\end{tcolorbox}
\begin{tcolorbox}
\textsubscript{8} не ступала по ній молода звірина, не ходив нею лев.
\end{tcolorbox}
\begin{tcolorbox}
\textsubscript{9} Чоловік свою руку по кремінь витягує, гори від кореня перевертає,
\end{tcolorbox}
\begin{tcolorbox}
\textsubscript{10} пробиває у скелях канали, і все дороге бачить око його!
\end{tcolorbox}
\begin{tcolorbox}
\textsubscript{11} Він загачує ріки від виливу, а заховані речі виводить на світло.
\end{tcolorbox}
\begin{tcolorbox}
\textsubscript{12} Та де мудрість знаходиться, і де місце розуму?
\end{tcolorbox}
\begin{tcolorbox}
\textsubscript{13} Людина не знає ціни їй, і вона у країні живих не знаходиться.
\end{tcolorbox}
\begin{tcolorbox}
\textsubscript{14} Безодня говорить: Вона не в мені! і море звіщає: Вона не зо мною!
\end{tcolorbox}
\begin{tcolorbox}
\textsubscript{15} Щирого золота дати за неї не можна, і не важиться срібло ціною за неї.
\end{tcolorbox}
\begin{tcolorbox}
\textsubscript{16} Не важать за неї офірського золота, ні дорогого оніксу й сапфіру.
\end{tcolorbox}
\begin{tcolorbox}
\textsubscript{17} Золото й скло не рівняються в вартості їй, і її не зміняти на посуд із щирого золота.
\end{tcolorbox}
\begin{tcolorbox}
\textsubscript{18} Коралі й кришталь і не згадуються, а набуток премудрости ліпший за перли!
\end{tcolorbox}
\begin{tcolorbox}
\textsubscript{19} Не рівняється їй етіопський топаз, і не важиться золото щире за неї.
\end{tcolorbox}
\begin{tcolorbox}
\textsubscript{20} А мудрість ізвідки проходить, і де місце розуму?
\end{tcolorbox}
\begin{tcolorbox}
\textsubscript{21} Бо вона від очей усього живого захована, і від птаства небесного скрита вона.
\end{tcolorbox}
\begin{tcolorbox}
\textsubscript{22} Аваддон той і смерть промовляють: Ушима своїми ми чули про неї лиш чутку!
\end{tcolorbox}
\begin{tcolorbox}
\textsubscript{23} Тільки Бог розуміє дорогу її, й тільки Він знає місце її!
\end{tcolorbox}
\begin{tcolorbox}
\textsubscript{24} Бо Він аж на кінці землі придивляється, бачить під небом усім.
\end{tcolorbox}
\begin{tcolorbox}
\textsubscript{25} Коли Він чинив вагу вітрові, а воду утворював мірою,
\end{tcolorbox}
\begin{tcolorbox}
\textsubscript{26} коли Він уставу складав для дощу та дороги для блискавки грому,
\end{tcolorbox}
\begin{tcolorbox}
\textsubscript{27} тоді Він побачив її та про неї повів, міцно поставив її та її дослідив!
\end{tcolorbox}
\begin{tcolorbox}
\textsubscript{28} І сказав Він людині тоді: Таж страх Господній це мудрість, а відступ від злого це розум!
\end{tcolorbox}
\subsection{CHAPTER 29}
\begin{tcolorbox}
\textsubscript{1} І Йов далі вів мову свою та й сказав:
\end{tcolorbox}
\begin{tcolorbox}
\textsubscript{2} О, коли б я був той, як за місяців давніх, як за днів тих, коли боронив мене Бог,
\end{tcolorbox}
\begin{tcolorbox}
\textsubscript{3} коли над головою моєю світився світильник Його, і при світлі його я ходив в темноті,
\end{tcolorbox}
\begin{tcolorbox}
\textsubscript{4} як був я за днів тих своєї погожої осени, коли Божа милість була над наметом моїм,
\end{tcolorbox}
\begin{tcolorbox}
\textsubscript{5} коли Всемогутній зо мною ще був, а навколо мене мої діти,
\end{tcolorbox}
\begin{tcolorbox}
\textsubscript{6} коли мої кроки купалися в маслі, а скеля оливні струмки біля мене лила!...
\end{tcolorbox}
\begin{tcolorbox}
\textsubscript{7} Коли я виходив до брами при місті, і ставив на площі сидіння своє,
\end{tcolorbox}
\begin{tcolorbox}
\textsubscript{8} як тільки вбачали мене юнаки то ховались, а старші вставали й стояли,
\end{tcolorbox}
\begin{tcolorbox}
\textsubscript{9} зверхники стримували свою мову та клали долоню на уста свої,
\end{tcolorbox}
\begin{tcolorbox}
\textsubscript{10} ховався тоді голос володарів, а їхній язик приліпав їм був до піднебіння...
\end{tcolorbox}
\begin{tcolorbox}
\textsubscript{11} Бо яке ухо чуло про мене, то звало блаженним мене, і яке око бачило, то свідкувало за мене,
\end{tcolorbox}
\begin{tcolorbox}
\textsubscript{12} бо я рятував бідаря, що про поміч кричав, і сироту та безпомічного.
\end{tcolorbox}
\begin{tcolorbox}
\textsubscript{13} Благословення гинучого на мене приходило, а серце вдовиці чинив я співаючим!
\end{tcolorbox}
\begin{tcolorbox}
\textsubscript{14} Зодягавсь я у праведність, і вона зодягала мене, немов плащ та завій було право моє.
\end{tcolorbox}
\begin{tcolorbox}
\textsubscript{15} Очима я був для сліпого, а кривому ногами я був.
\end{tcolorbox}
\begin{tcolorbox}
\textsubscript{16} Бідарям я був батьком, суперечку ж, якої не знав, я досліджував.
\end{tcolorbox}
\begin{tcolorbox}
\textsubscript{17} Й я торощив злочинцеві щелепи, і виривав із зубів його схоплене.
\end{tcolorbox}
\begin{tcolorbox}
\textsubscript{18} І я говорив: Умру я в своєму гнізді, і свої дні я помножу, немов той пісок:
\end{tcolorbox}
\begin{tcolorbox}
\textsubscript{19} для води був відкритий мій корень, а роса зоставалась на вітці моїй...
\end{tcolorbox}
\begin{tcolorbox}
\textsubscript{20} Моя слава була при мені все нова, і в руці моїй лук мій відновлював силу.
\end{tcolorbox}
\begin{tcolorbox}
\textsubscript{21} Мене слухалися й дожидали, і мовчали на раду мою.
\end{tcolorbox}
\begin{tcolorbox}
\textsubscript{22} По слові моїм уже не говорили, і падала мова моя на них краплями.
\end{tcolorbox}
\begin{tcolorbox}
\textsubscript{23} І чекали мене, як дощу, і уста свої відкривали, немов на весінній той дощик...
\end{tcolorbox}
\begin{tcolorbox}
\textsubscript{24} Коли я, бувало, сміявся до них, то не вірили, та світла обличчя мого не гасили.
\end{tcolorbox}
\begin{tcolorbox}
\textsubscript{25} Вибирав я дорогу для них і сидів на чолі, і пробував, немов цар той у війську, коли тішить засмучених він!
\end{tcolorbox}
\subsection{CHAPTER 30}
\begin{tcolorbox}
\textsubscript{1} А тепер насміхаються з мене молодші від мене літами, ті, що їхніх батьків я бридився б покласти із псами отари моєї...
\end{tcolorbox}
\begin{tcolorbox}
\textsubscript{2} Та й сила рук їхніх для чого бувала мені? Повня сил їх минулась!
\end{tcolorbox}
\begin{tcolorbox}
\textsubscript{3} Самотні були в недостатку та голоді, ссали вони суху землю, зруйновану та опустілу!
\end{tcolorbox}
\begin{tcolorbox}
\textsubscript{4} рвали вони лободу на кущах, ялівцеве ж коріння було їхнім хлібом...
\end{tcolorbox}
\begin{tcolorbox}
\textsubscript{5} Вони були вигнані з-поміж людей, кричали на них, немов на злодіїв,
\end{tcolorbox}
\begin{tcolorbox}
\textsubscript{6} так що вони пробували в яругах долин, по ямах підземних та скелях,
\end{tcolorbox}
\begin{tcolorbox}
\textsubscript{7} ревіли вони між кущами, збирались під терням,
\end{tcolorbox}
\begin{tcolorbox}
\textsubscript{8} сини нерозумного й діти неславного, вони були вигнані з краю!
\end{tcolorbox}
\begin{tcolorbox}
\textsubscript{9} А тепер я став піснею їм, і зробився для них поговором...
\end{tcolorbox}
\begin{tcolorbox}
\textsubscript{10} Вони обридили мене, віддалились від мене, і від мойого обличчя не стримали слини,
\end{tcolorbox}
\begin{tcolorbox}
\textsubscript{11} бо Він розв'язав мого пояса й мучить мене, то й вони ось вуздечку із себе відкинули перед обличчям моїм...
\end{tcolorbox}
\begin{tcolorbox}
\textsubscript{12} По правиці встають жовтодзюбі, ноги мені підставляють, і топчуть на мене дороги нещастя свого...
\end{tcolorbox}
\begin{tcolorbox}
\textsubscript{13} Порили вони мою стежку, хочуть мати користь із мойого життя, немає кому їх затримати,
\end{tcolorbox}
\begin{tcolorbox}
\textsubscript{14} немов через вилім широкий приходять, валяються попід румовищем...
\end{tcolorbox}
\begin{tcolorbox}
\textsubscript{15} Обернулось страхіття на мене, моя слава пронеслась, як вітер, і, як хмара, минулося щастя моє...
\end{tcolorbox}
\begin{tcolorbox}
\textsubscript{16} А тепер розливається в мене душа моя, хапають мене дні нещастя!
\end{tcolorbox}
\begin{tcolorbox}
\textsubscript{17} Вночі мої кості від мене віддовбуються, а жили мої не вспокоюються...
\end{tcolorbox}
\begin{tcolorbox}
\textsubscript{18} З великої Божої сили змінилося тіло моє, і недуга мене оперізує, мов той хітон.
\end{tcolorbox}
\begin{tcolorbox}
\textsubscript{19} Він укинув мене до болота, і став я подібний до пороху й попелу.
\end{tcolorbox}
\begin{tcolorbox}
\textsubscript{20} Я кличу до Тебе, та Ти мені відповіді не даєш, я перед Тобою стою, Ти ж на мене лише придивляєшся...
\end{tcolorbox}
\begin{tcolorbox}
\textsubscript{21} Ти змінився мені на жорстокого, мене Ти женеш силою Своєї руки...
\end{tcolorbox}
\begin{tcolorbox}
\textsubscript{22} На вітер підняв Ти мене, на нього мене посадив, і робиш, щоб я розтопивсь на спустошення!
\end{tcolorbox}
\begin{tcolorbox}
\textsubscript{23} Знаю я: Ти до смерти провадиш мене, і до дому зібрання, якого призначив для всього живого...
\end{tcolorbox}
\begin{tcolorbox}
\textsubscript{24} Хіба не простягає руки потопельник, чи він у нещасті своїм не кричить?
\end{tcolorbox}
\begin{tcolorbox}
\textsubscript{25} Чи ж не плакав я за бідарем? Чи за вбогим душа моя не сумувала?
\end{tcolorbox}
\begin{tcolorbox}
\textsubscript{26} Бо чекав я добра, але лихо прийшло, сподівався я світла, та темнота прийшла...
\end{tcolorbox}
\begin{tcolorbox}
\textsubscript{27} Киплять мої нутрощі й не замовкають, зустріли мене дні нещастя,
\end{tcolorbox}
\begin{tcolorbox}
\textsubscript{28} ходжу почорнілий без сонця, на зборі встаю та кричу...
\end{tcolorbox}
\begin{tcolorbox}
\textsubscript{29} Я став братом шакалам, а струсятам товаришем,
\end{tcolorbox}
\begin{tcolorbox}
\textsubscript{30} моя шкіра зчорніла та й лупиться з мене, від спекоти спалилися кості мої...
\end{tcolorbox}
\begin{tcolorbox}
\textsubscript{31} І стала жалобою арфа моя, а сопілка моя зойком плачливим...
\end{tcolorbox}
\subsection{CHAPTER 31}
\begin{tcolorbox}
\textsubscript{1} Умову я склав був з очима своїми, то як буду дивитись на дівчину?
\end{tcolorbox}
\begin{tcolorbox}
\textsubscript{2} І зверху яка доля від Бога, чи спадщина від Всемогутнього із висот?
\end{tcolorbox}
\begin{tcolorbox}
\textsubscript{3} Хіба не загибіль для кривдника, і хіба не нещастя злочинцям?
\end{tcolorbox}
\begin{tcolorbox}
\textsubscript{4} Хіба ж Він не бачить дороги мої, і не лічить усі мої кроки?
\end{tcolorbox}
\begin{tcolorbox}
\textsubscript{5} Якщо я ходив у марноті, і на оману спішила нога моя,
\end{tcolorbox}
\begin{tcolorbox}
\textsubscript{6} то нехай на вазі справедливости зважить мене, і невинність мою Бог пізнає!
\end{tcolorbox}
\begin{tcolorbox}
\textsubscript{7} Якщо збочує крок мій з дороги, і за очима моїми пішло моє серце, і до рук моїх нечисть приліпла,
\end{tcolorbox}
\begin{tcolorbox}
\textsubscript{8} то нехай сію я, а їсть інший, а рослинність моя нехай вирвана буде з корінням!
\end{tcolorbox}
\begin{tcolorbox}
\textsubscript{9} Якщо моє серце зваблялось до жінки чужої, і причаювався я при дверях мойого товариша,
\end{tcolorbox}
\begin{tcolorbox}
\textsubscript{10} то хай меле для іншого жінка моя, і над нею нехай нахиляються інші!
\end{tcolorbox}
\begin{tcolorbox}
\textsubscript{11} Бо гидота оце, й це провина підсудна,
\end{tcolorbox}
\begin{tcolorbox}
\textsubscript{12} бо огонь це, який буде жерти аж до Аваддону, і вирве з корінням увесь урожай мій!...
\end{tcolorbox}
\begin{tcolorbox}
\textsubscript{13} Якщо я понехтував правом свойого раба чи своєї невільниці в їх суперечці зо мною,
\end{tcolorbox}
\begin{tcolorbox}
\textsubscript{14} то що я зроблю, як підійметься Бог? А коли Він приглянеться, що Йому відповім?
\end{tcolorbox}
\begin{tcolorbox}
\textsubscript{15} Чи ж не Той, Хто мене учинив у нутрі, учинив і його, і Один утворив нас в утробі?
\end{tcolorbox}
\begin{tcolorbox}
\textsubscript{16} Чи бажання убогих я стримував, а очі вдовицям засмучував?
\end{tcolorbox}
\begin{tcolorbox}
\textsubscript{17} Чи я сам поїдав свій шматок, і з нього не їв сирота?
\end{tcolorbox}
\begin{tcolorbox}
\textsubscript{18} Таж від днів молодечих моїх виростав він у мене, як в батька, і від утроби матері моєї я провадив його!
\end{tcolorbox}
\begin{tcolorbox}
\textsubscript{19} Якщо бачив я гинучого без одежі, і вбрання не було в сіромахи,
\end{tcolorbox}
\begin{tcolorbox}
\textsubscript{20} чи ж не благословляли мене його стегна, і руном овечок моїх він не грівся?
\end{tcolorbox}
\begin{tcolorbox}
\textsubscript{21} Якщо на сироту я порушував руку свою, коли бачив у брамі собі допомогу,
\end{tcolorbox}
\begin{tcolorbox}
\textsubscript{22} хай рамено моє відпаде від свойого плеча, а рука моя від суглобу свого нехай буде відламана!
\end{tcolorbox}
\begin{tcolorbox}
\textsubscript{23} Бо острах на мене нещастя від Бога, а перед величчям Його я не можу встояти...
\end{tcolorbox}
\begin{tcolorbox}
\textsubscript{24} Чи я золото клав за надію собі, чи до щирого золота я говорив: Ти, безпеко моя?
\end{tcolorbox}
\begin{tcolorbox}
\textsubscript{25} Чи тішився я, що велике багатство моє, й що рука моя стільки надбала?
\end{tcolorbox}
\begin{tcolorbox}
\textsubscript{26} Коли бачив я сонце, як сяє воно, а місяць велично пливе,
\end{tcolorbox}
\begin{tcolorbox}
\textsubscript{27} то коли б потаємно повабилось серце моє, і цілунки рукою я їм посилав,
\end{tcolorbox}
\begin{tcolorbox}
\textsubscript{28} це так само провина підсудна була б, бо відрікся б я Бога Всевишнього!
\end{tcolorbox}
\begin{tcolorbox}
\textsubscript{29} Чи я тішивсь упадком свойого ненависника, чи порушувавсь я, коли зло спотикало його?
\end{tcolorbox}
\begin{tcolorbox}
\textsubscript{30} Таки ні, не давав я на гріх піднебіння свого, щоб прокляттям жадати душі його.
\end{tcolorbox}
\begin{tcolorbox}
\textsubscript{31} Хіба люди намету мого не казали: Хто покаже такого, хто з м'яса його не наситився?
\end{tcolorbox}
\begin{tcolorbox}
\textsubscript{32} Чужинець на вулиці не ночував, я двері свої відчиняв подорожньому.
\end{tcolorbox}
\begin{tcolorbox}
\textsubscript{33} Чи ховав свої прогріхи я, як людина, щоб у своєму нутрі затаїти провину свою?
\end{tcolorbox}
\begin{tcolorbox}
\textsubscript{34} Бо тоді я боявся б великого натовпу, і сором від родів жахав би мене, я мовчав би, й з дверей не виходив...
\end{tcolorbox}
\begin{tcolorbox}
\textsubscript{35} О, якби мене вислухав хто! Оце підпис моєї руки: Нехай Всемогутній мені відповість, а ось звій, зо скаргою, що його написав мій противник...
\end{tcolorbox}
\begin{tcolorbox}
\textsubscript{36} Чи ж я не носив би його на своєму плечі, не обвинувся б ним, як вінками?
\end{tcolorbox}
\begin{tcolorbox}
\textsubscript{37} Число кроків своїх я представлю йому; мов до князя, наближусь до нього.
\end{tcolorbox}
\begin{tcolorbox}
\textsubscript{38} Якщо проти мене голосить земля моя, й її борозни плачуть із нею,
\end{tcolorbox}
\begin{tcolorbox}
\textsubscript{39} якщо без грошей я їв плоди її, а її власника я стогнати примушував,
\end{tcolorbox}
\begin{tcolorbox}
\textsubscript{40} то замість пшениці хай виросте терен, а замість ячменю кукіль!... Слова Йова скінчилися.
\end{tcolorbox}
\subsection{CHAPTER 32}
\begin{tcolorbox}
\textsubscript{1} І перестали ті троє мужів відповідати Йову, бо він був справедливий в очах своїх.
\end{tcolorbox}
\begin{tcolorbox}
\textsubscript{2} І запалився гнів Елігу, сина Барах'їлового, бузянина, з роду Рамового, на Йова запалився гнів його за те, що той уважав душу свою справедливішою за Бога.
\end{tcolorbox}
\begin{tcolorbox}
\textsubscript{3} Також на трьох приятелів його запалився його гнів за те, що не знайшли вони відповіді, а зробили тільки Йова винним.
\end{tcolorbox}
\begin{tcolorbox}
\textsubscript{4} А Елігу вичікував Йова та їх із словами, бо вони були старші віком за нього.
\end{tcolorbox}
\begin{tcolorbox}
\textsubscript{5} І побачив Елігу, що нема належної відповіді в устах тих трьох людей, і запалився його гнів!
\end{tcolorbox}
\begin{tcolorbox}
\textsubscript{6} І відповів бузянин Елігу, син Барах'їлів, та й сказав: Молодий я літами, ви ж старші, тому то я стримувався та боявся знання своє висловити вам.
\end{tcolorbox}
\begin{tcolorbox}
\textsubscript{7} Я подумав: Хай вік промовляє, і хай розуму вчить многоліття!
\end{tcolorbox}
\begin{tcolorbox}
\textsubscript{8} Справді, дух він у людині, та Всемогутнього подих їх мудрими чинить.
\end{tcolorbox}
\begin{tcolorbox}
\textsubscript{9} Многолітні не завжди розумні, і не все розуміються в праві старі.
\end{tcolorbox}
\begin{tcolorbox}
\textsubscript{10} Тому я кажу: Послухай мене, хай знання своє висловлю й я!
\end{tcolorbox}
\begin{tcolorbox}
\textsubscript{11} Тож слів ваших вичікував я, наставляв свої уші до вашої мудрости, поки справу ви дослідите.
\end{tcolorbox}
\begin{tcolorbox}
\textsubscript{12} І я приглядався до вас, й ось немає між вами, хто б Йову довів, хто б відповідь дав на слова його!
\end{tcolorbox}
\begin{tcolorbox}
\textsubscript{13} Щоб ви не сказали: Ми мудрість знайшли: не людина, а Бог переможе його!
\end{tcolorbox}
\begin{tcolorbox}
\textsubscript{14} Не на мене слова він скеровував, і я не відповім йому мовою вашою.
\end{tcolorbox}
\begin{tcolorbox}
\textsubscript{15} Полякались вони, вже не відповідають, не мають вже слів...
\end{tcolorbox}
\begin{tcolorbox}
\textsubscript{16} Я чекав, що не будуть вони говорити, що спинились, не відповідають уже.
\end{tcolorbox}
\begin{tcolorbox}
\textsubscript{17} Відповім також я свою частку, і висловлю й я свою думку.
\end{tcolorbox}
\begin{tcolorbox}
\textsubscript{18} Бо я повний словами, дух мойого нутра докучає мені...
\end{tcolorbox}
\begin{tcolorbox}
\textsubscript{19} Ось утроба моя, мов вино невідкрите, вона тріскається, як нові бурдюки!
\end{tcolorbox}
\begin{tcolorbox}
\textsubscript{20} Нехай я скажу й буде легше мені, нехай уста відкрию свої й відповім!
\end{tcolorbox}
\begin{tcolorbox}
\textsubscript{21} На особу не буду уваги звертати, не буду підлещуватись до людини,
\end{tcolorbox}
\begin{tcolorbox}
\textsubscript{22} бо не вмію підлещуватись! Коли ж ні, нехай зараз візьме мене мій Творець!
\end{tcolorbox}
\subsection{CHAPTER 33}
\begin{tcolorbox}
\textsubscript{1} Але слухай но, Йове, промови мої, і візьми до ушей всі слова мої.
\end{tcolorbox}
\begin{tcolorbox}
\textsubscript{2} Ось я уста свої відкриваю, в моїх устах говорить язик мій.
\end{tcolorbox}
\begin{tcolorbox}
\textsubscript{3} Простота мого серця слова мої, і висловлять ясно знання мої уста.
\end{tcolorbox}
\begin{tcolorbox}
\textsubscript{4} Дух Божий мене учинив, й оживляє мене Всемогутнього подих.
\end{tcolorbox}
\begin{tcolorbox}
\textsubscript{5} Якщо можеш, то дай мені відповідь, вишикуйсь передо мною, постався!
\end{tcolorbox}
\begin{tcolorbox}
\textsubscript{6} Тож Божий і я, як і ти, з глини витиснений теж і я!
\end{tcolorbox}
\begin{tcolorbox}
\textsubscript{7} Ото страх мій тебе не настрашить, і не буде тяжкою рука моя на тобі.
\end{tcolorbox}
\begin{tcolorbox}
\textsubscript{8} Отож, говорив до моїх ушей ти, і я чув голос слів:
\end{tcolorbox}
\begin{tcolorbox}
\textsubscript{9} Чистий я, без гріха, я невинний, і немає провини в мені!
\end{tcolorbox}
\begin{tcolorbox}
\textsubscript{10} Оце Сам Він причини на мене знаходить, уважає мене Собі ворогом.
\end{tcolorbox}
\begin{tcolorbox}
\textsubscript{11} У кайдани закув мої ноги, усі стежки мої Він стереже...
\end{tcolorbox}
\begin{tcolorbox}
\textsubscript{12} Ось у цьому ти не справедливий! Відповім я тобі, бо більший же Бог за людину!
\end{tcolorbox}
\begin{tcolorbox}
\textsubscript{13} Чого Ти із Ним сперечаєшся, що про всі Свої справи Він відповіді не дає?
\end{tcolorbox}
\begin{tcolorbox}
\textsubscript{14} Бо Бог промовляє і раз, і два рази, та людина не бачить того:
\end{tcolorbox}
\begin{tcolorbox}
\textsubscript{15} у сні, у видінні нічному, коли міцний сон на людей нападає, в дрімотах на ложі,
\end{tcolorbox}
\begin{tcolorbox}
\textsubscript{16} тоді відкриває Він ухо людей, і настрашує їх осторогою,
\end{tcolorbox}
\begin{tcolorbox}
\textsubscript{17} щоб відвести людину від чину її, і Він гордість від мужа ховає,
\end{tcolorbox}
\begin{tcolorbox}
\textsubscript{18} щоб від гробу повстримати душу його, а живая його щоб не впала на ратище.
\end{tcolorbox}
\begin{tcolorbox}
\textsubscript{19} І карається хворістю він на постелі своїй, а в костях його сварка міцна.
\end{tcolorbox}
\begin{tcolorbox}
\textsubscript{20} І жива його бридиться хлібом, а душа його стравою влюбленою.
\end{tcolorbox}
\begin{tcolorbox}
\textsubscript{21} Гине тіло його, аж не видно його, і вистають його кості, що перше не видні були.
\end{tcolorbox}
\begin{tcolorbox}
\textsubscript{22} І до гробу душа його зближується, а живая його до померлих іде.
\end{tcolorbox}
\begin{tcolorbox}
\textsubscript{23} Якщо ж Ангол-заступник при нім, один з тисячі, щоб представити людині її правоту,
\end{tcolorbox}
\begin{tcolorbox}
\textsubscript{24} то Він буде йому милосердний та й скаже: Звільни ти його, щоб до гробу не йшов він, Я викуп знайшов.
\end{tcolorbox}
\begin{tcolorbox}
\textsubscript{25} Тоді відмолодиться тіло його, поверне до днів його юности.
\end{tcolorbox}
\begin{tcolorbox}
\textsubscript{26} Він благатиме Бога, й його Собі Він уподобає, і обличчя його буде бачити з окликом радости, і чоловікові верне його справедливість.
\end{tcolorbox}
\begin{tcolorbox}
\textsubscript{27} Він дивитиметься на людей й говоритиме: Я грішив був і правду кривив, та мені не відплачено.
\end{tcolorbox}
\begin{tcolorbox}
\textsubscript{28} Він викупив душу мою, щоб до гробу не йшла, і буде бачити світло живая моя.
\end{tcolorbox}
\begin{tcolorbox}
\textsubscript{29} Бог робить це все двічі-тричі з людиною,
\end{tcolorbox}
\begin{tcolorbox}
\textsubscript{30} щоб душу її відвернути від гробу, щоб він був освітлений світлом живих.
\end{tcolorbox}
\begin{tcolorbox}
\textsubscript{31} Уважай, Йове, слухай мене, мовчи, а я промовлятиму!
\end{tcolorbox}
\begin{tcolorbox}
\textsubscript{32} Коли маєш слова, то дай мені відповідь, говори, бо бажаю твого оправдання.
\end{tcolorbox}
\begin{tcolorbox}
\textsubscript{33} Якщо ні ти послухай мене; помовчи, й я навчу тебе мудрости!
\end{tcolorbox}
\subsection{CHAPTER 34}
\begin{tcolorbox}
\textsubscript{1} І говорив Елігу та й сказав:
\end{tcolorbox}
\begin{tcolorbox}
\textsubscript{2} Слухайте, мудрі, слова ці мої, ви ж, розважні, почуйте мене!
\end{tcolorbox}
\begin{tcolorbox}
\textsubscript{3} Бо ухо слова випробовує, а піднебіння їжу куштує.
\end{tcolorbox}
\begin{tcolorbox}
\textsubscript{4} Виберім право собі, між собою пізнаймо, що добре.
\end{tcolorbox}
\begin{tcolorbox}
\textsubscript{5} Бо Йов говорив: Я був справедливий, та відкинув Бог право моє.
\end{tcolorbox}
\begin{tcolorbox}
\textsubscript{6} Чи буду неправду казати за право своє? Без вини небезпечна стріла моя...
\end{tcolorbox}
\begin{tcolorbox}
\textsubscript{7} Чи є такий муж, як цей Йов, що п'є глузування, як воду,
\end{tcolorbox}
\begin{tcolorbox}
\textsubscript{8} і товаришує з злочинцями, і ходить з людьми беззаконними?
\end{tcolorbox}
\begin{tcolorbox}
\textsubscript{9} Бо він каже: Нема людині користи, коли її Бог уподобає.
\end{tcolorbox}
\begin{tcolorbox}
\textsubscript{10} Тож вислухайте, ви розумні, мене: Бог далекий від несправедливости, і Всемогутній від кривди!
\end{tcolorbox}
\begin{tcolorbox}
\textsubscript{11} Бо за чином людини Він їй надолужить, і згідно з своєю дорогою знайде людина заплату!
\end{tcolorbox}
\begin{tcolorbox}
\textsubscript{12} Тож поправді, не чинить Бог несправедливого, і Всемогутній не скривлює права.
\end{tcolorbox}
\begin{tcolorbox}
\textsubscript{13} Хто землю довірив Йому, і хто на Нього вселенну поклав?
\end{tcolorbox}
\begin{tcolorbox}
\textsubscript{14} Коли б Він до Себе забрав Своє серце, Свій дух, і Свій подих до Себе забрав,
\end{tcolorbox}
\begin{tcolorbox}
\textsubscript{15} всяке тіло погинуло б вмить, а людина повернулася б на порох!...
\end{tcolorbox}
\begin{tcolorbox}
\textsubscript{16} Коли маєш ти розум, послухай же це, почуй голос оцих моїх слів:
\end{tcolorbox}
\begin{tcolorbox}
\textsubscript{17} Хіба стримувати може ненависник право? І хіба осудити ти зможеш Всеправедного?
\end{tcolorbox}
\begin{tcolorbox}
\textsubscript{18} Хіба можна сказати цареві: Негідний, а вельможним: Безбожний?
\end{tcolorbox}
\begin{tcolorbox}
\textsubscript{19} Таж Він не звертає уваги на зверхників, і не вирізнює можного перед убогим, бо всі вони чин Його рук,
\end{tcolorbox}
\begin{tcolorbox}
\textsubscript{20} за хвилину вони помирають, опівночі... Доторкнеться Він можних і гинуть вони, сильний усунений буде рукою не людською.
\end{tcolorbox}
\begin{tcolorbox}
\textsubscript{21} Бо очі Його на дорогах людини, і Він бачить всі кроки її,
\end{tcolorbox}
\begin{tcolorbox}
\textsubscript{22} немає темноти, немає і темряви, де б злочинці сховались.
\end{tcolorbox}
\begin{tcolorbox}
\textsubscript{23} Бо людині Він не призначає означений час, щоб ходила до Бога на суд.
\end{tcolorbox}
\begin{tcolorbox}
\textsubscript{24} Він сильних ламає без досліду, і ставить на місце їх інших.
\end{tcolorbox}
\begin{tcolorbox}
\textsubscript{25} Бож знає Він їхні діла, оберне вночі і почавлені будуть!
\end{tcolorbox}
\begin{tcolorbox}
\textsubscript{26} Як несправедливих уразить Він їх, на видному місці,
\end{tcolorbox}
\begin{tcolorbox}
\textsubscript{27} за те, що вони відступили від Нього, і не розуміли доріг Його всіх,
\end{tcolorbox}
\begin{tcolorbox}
\textsubscript{28} щоб зойк сіромахи спровадити до Нього, бо Він чує благання пригнічених.
\end{tcolorbox}
\begin{tcolorbox}
\textsubscript{29} Коли Він заспокоїть, то хто винуватити буде? Коли Він закриє лице, хто побачить Його? А це робиться і над народом, і над людиною разом,
\end{tcolorbox}
\begin{tcolorbox}
\textsubscript{30} щоб не панував чоловік нечестивий із тих, що правлять за пастку народові.
\end{tcolorbox}
\begin{tcolorbox}
\textsubscript{31} Бо Богові треба отак говорити: Несу я заслужене, злого робити не буду!
\end{tcolorbox}
\begin{tcolorbox}
\textsubscript{32} Чого я не бачу, навчи Ти мене; коли кривду зробив я, то більше не буду чинити!
\end{tcolorbox}
\begin{tcolorbox}
\textsubscript{33} Чи на думку твою надолужить Він це, бо відкинув ти те? Бо вибереш ти, а не я, а що знаєш, кажи!
\end{tcolorbox}
\begin{tcolorbox}
\textsubscript{34} Мені скажуть розумні та муж мудрий, який мене слухає:
\end{tcolorbox}
\begin{tcolorbox}
\textsubscript{35} Йов говорить немудро, а слова його без розуміння.
\end{tcolorbox}
\begin{tcolorbox}
\textsubscript{36} О, коли б Йов досліджений був аж навіки за відповіді, як злі люди,
\end{tcolorbox}
\begin{tcolorbox}
\textsubscript{37} бо він додає до свойого гріха ще провину, між нами він плеще в долоні та множить на Бога промови свої...
\end{tcolorbox}
\subsection{CHAPTER 35}
\begin{tcolorbox}
\textsubscript{1} І говорив Елігу та й сказав:
\end{tcolorbox}
\begin{tcolorbox}
\textsubscript{2} Чи це полічив ти за право, як кажеш: Моя праведність більша за Божу?
\end{tcolorbox}
\begin{tcolorbox}
\textsubscript{3} Бо ти говорив: Що поможе тобі? Яку користь із цього я матиму більшу, аніж від свойого гріха?
\end{tcolorbox}
\begin{tcolorbox}
\textsubscript{4} Я тобі відповім, а з тобою і ближнім твоїм.
\end{tcolorbox}
\begin{tcolorbox}
\textsubscript{5} Подивися на небо й побач, і на хмари споглянь, вони вищі за тебе.
\end{tcolorbox}
\begin{tcolorbox}
\textsubscript{6} Як ти будеш грішити, що зробиш Йому? А стануть численні провини твої, що ти вчиниш Йому?
\end{tcolorbox}
\begin{tcolorbox}
\textsubscript{7} Коли праведним станеш, що даси ти Йому? Або що Він візьме з твоєї руки?
\end{tcolorbox}
\begin{tcolorbox}
\textsubscript{8} Для людини, як ти, беззаконня твоє, і для людського сина твоя справедливість!...
\end{tcolorbox}
\begin{tcolorbox}
\textsubscript{9} Від безлічі гноблення стогнуть вони, кричать від твердого плеча багатьох...
\end{tcolorbox}
\begin{tcolorbox}
\textsubscript{10} Та не скаже ніхто: Де ж той Бог, що мене Він створив, що вночі дає співи,
\end{tcolorbox}
\begin{tcolorbox}
\textsubscript{11} що нас над худобу земну Він навчає, і над птаство небесне вчиняє нас мудрими?
\end{tcolorbox}
\begin{tcolorbox}
\textsubscript{12} Вони там кричать, але через бундючність злочинців Він відповіді не дає.
\end{tcolorbox}
\begin{tcolorbox}
\textsubscript{13} Тільки марноти не слухає Бог, і Всемогутній не бачить її.
\end{tcolorbox}
\begin{tcolorbox}
\textsubscript{14} Що ж тоді, коли кажеш: Не бачив Його! Та є суд перед Ним, і чекай ти його!
\end{tcolorbox}
\begin{tcolorbox}
\textsubscript{15} А тепер, коли гнів Його не покарав, і не дуже пізнав про глупоту,
\end{tcolorbox}
\begin{tcolorbox}
\textsubscript{16} то намарно Йов уста свої відкриває та множить слова без знання...
\end{tcolorbox}
\subsection{CHAPTER 36}
\begin{tcolorbox}
\textsubscript{1} І далі Елігу казав:
\end{tcolorbox}
\begin{tcolorbox}
\textsubscript{2} Почекай мені трохи, й тобі покажу, бо ще є про Бога слова.
\end{tcolorbox}
\begin{tcolorbox}
\textsubscript{3} Зачну викладати я здалека, і Творцеві своєму віддам справедливість.
\end{tcolorbox}
\begin{tcolorbox}
\textsubscript{4} Бо справді слова мої не неправдиві, я з тобою безвадний в знанні.
\end{tcolorbox}
\begin{tcolorbox}
\textsubscript{5} Таж Бог сильний, і не відкидає нікого, Він міцний в силі серця.
\end{tcolorbox}
\begin{tcolorbox}
\textsubscript{6} Не лишає безбожного Він при житті, але право для бідних дає.
\end{tcolorbox}
\begin{tcolorbox}
\textsubscript{7} Від праведного Він очей Своїх не відвертає, але їх садовить з царями на троні назавжди, і вони підвищаються.
\end{tcolorbox}
\begin{tcolorbox}
\textsubscript{8} А як тільки вони ланцюгами пов'язані, і тримаються в путах біди,
\end{tcolorbox}
\begin{tcolorbox}
\textsubscript{9} то Він їм представляє їх вчинок та їхні провини, що багато їх стало.
\end{tcolorbox}
\begin{tcolorbox}
\textsubscript{10} Відкриває Він ухо їх для остороги, та велить, щоб вернулися від беззаконня.
\end{tcolorbox}
\begin{tcolorbox}
\textsubscript{11} Якщо тільки послухаються, та стануть служити Йому, покінчать вони свої дні у добрі, а роки свої у приємнощах.
\end{tcolorbox}
\begin{tcolorbox}
\textsubscript{12} Коли ж не послухаються, то наскочать на ратище, і покінчать життя без знання.
\end{tcolorbox}
\begin{tcolorbox}
\textsubscript{13} А злосерді кладуть гнів на себе, не кричать, коли в'яже Він їх.
\end{tcolorbox}
\begin{tcolorbox}
\textsubscript{14} У молодості помирає душа їх, а їхня живая поміж блудниками.
\end{tcolorbox}
\begin{tcolorbox}
\textsubscript{15} Він визволяє убогого з горя його, а в переслідуванні відкриває їм ухо.
\end{tcolorbox}
\begin{tcolorbox}
\textsubscript{16} Також і тебе Він би вибавив був із тісноти на широкість, що в ній нема утиску, а те, що на стіл твій поклалося б, повне товщу було б.
\end{tcolorbox}
\begin{tcolorbox}
\textsubscript{17} Та правом безбожного ти переповнений, право ж та суд підпирають людину.
\end{tcolorbox}
\begin{tcolorbox}
\textsubscript{18} Отож лютість нехай не намовить тебе до плескання в долоні, а окуп великий нехай не заверне з дороги тебе.
\end{tcolorbox}
\begin{tcolorbox}
\textsubscript{19} Чи в біді допоможе твій зойк та всі зміцнення сили?
\end{tcolorbox}
\begin{tcolorbox}
\textsubscript{20} Не квапся до ночі тієї, коли вирвані будуть народи із місця свого.
\end{tcolorbox}
\begin{tcolorbox}
\textsubscript{21} Стережись, не звертайся до зла, яке замість біди ти обрав.
\end{tcolorbox}
\begin{tcolorbox}
\textsubscript{22} Отож, Бог найвищий у силі Своїй, хто навчає, як Він?
\end{tcolorbox}
\begin{tcolorbox}
\textsubscript{23} Хто дорогу Його Йому вказувати буде? І хто скаже: Ти кривду зробив?
\end{tcolorbox}
\begin{tcolorbox}
\textsubscript{24} Пам'ятай, щоб звеличувати Його вчинок, про якого виспівують люди,
\end{tcolorbox}
\begin{tcolorbox}
\textsubscript{25} що його бачить всяка людина, чоловік приглядається здалека.
\end{tcolorbox}
\begin{tcolorbox}
\textsubscript{26} Отож, Бог великий та недовідомий, і недослідиме число Його літ!
\end{tcolorbox}
\begin{tcolorbox}
\textsubscript{27} Бо стягає Він краплі води, і дощем вони падають з хмари Його,
\end{tcolorbox}
\begin{tcolorbox}
\textsubscript{28} що хмари спускають його, і спадають дощем на багато людей.
\end{tcolorbox}
\begin{tcolorbox}
\textsubscript{29} Також хто зрозуміє розтягнення хмари, грім намету Його?
\end{tcolorbox}
\begin{tcolorbox}
\textsubscript{30} Отож, розтягає Він світло Своє над Собою і морську глибінь закриває,
\end{tcolorbox}
\begin{tcolorbox}
\textsubscript{31} бо ними Він судить народи, багато поживи дає.
\end{tcolorbox}
\begin{tcolorbox}
\textsubscript{32} Він тримає в руках Своїх блискавку, і керує її проти цілі.
\end{tcolorbox}
\begin{tcolorbox}
\textsubscript{33} Її гуркіт звіщає про неї, і прихід її відчуває й худоба.
\end{tcolorbox}
\subsection{CHAPTER 37}
\begin{tcolorbox}
\textsubscript{1} Отож, і від цього тремтить моє серце і зрушилось з місця свого.
\end{tcolorbox}
\begin{tcolorbox}
\textsubscript{2} Уважливо слухайте гук Його голосу, і грім, що несеться із уст Його,
\end{tcolorbox}
\begin{tcolorbox}
\textsubscript{3} його Він пускає попід усім небом, а світло Своє аж на кінці землі.
\end{tcolorbox}
\begin{tcolorbox}
\textsubscript{4} За Ним грім ричить левом, гримить гуком своєї величности, і його Він не стримує, почується голос Його.
\end{tcolorbox}
\begin{tcolorbox}
\textsubscript{5} Бог предивно гримить Своїм голосом, вчиняє великі діла, яких не розуміємо ми.
\end{tcolorbox}
\begin{tcolorbox}
\textsubscript{6} До снігу говорить Він: Падай на землю! а дощеві та зливі: Будьте сильні!
\end{tcolorbox}
\begin{tcolorbox}
\textsubscript{7} Він руку печатає кожній людині, щоб пізнали всі люди про діло Його.
\end{tcolorbox}
\begin{tcolorbox}
\textsubscript{8} І звір входить у сховище, і живе в своїх лігвищах.
\end{tcolorbox}
\begin{tcolorbox}
\textsubscript{9} Із кімнати південної буря приходить, а з вітру північного холод.
\end{tcolorbox}
\begin{tcolorbox}
\textsubscript{10} Від Божого подиху лід повстає, і водна широкість тужавіє.
\end{tcolorbox}
\begin{tcolorbox}
\textsubscript{11} Також Він обтяжує вільгістю тучу, і світло своє розпорошує хмара,
\end{tcolorbox}
\begin{tcolorbox}
\textsubscript{12} і вона по околицях ходить та блукає за Його проводом, щоб чинити все те, що накаже Він їй на поверхні вселенної,
\end{tcolorbox}
\begin{tcolorbox}
\textsubscript{13} він наводить її чи на кару для краю Свого, чи на милість.
\end{tcolorbox}
\begin{tcolorbox}
\textsubscript{14} Бери, Йове, оце до ушей, уставай і розваж Божі чуда!
\end{tcolorbox}
\begin{tcolorbox}
\textsubscript{15} Чи ти знаєш, що Бог накладає на них, і заяснює світло із хмари Своєї?
\end{tcolorbox}
\begin{tcolorbox}
\textsubscript{16} Чи ти знаєш, як носиться хмара в повітрі, про чуда Того, Який має безвадне знання,
\end{tcolorbox}
\begin{tcolorbox}
\textsubscript{17} ти, що шати твої стають теплі, як стишується земля з полудня?
\end{tcolorbox}
\begin{tcolorbox}
\textsubscript{18} Чи ти розтягав із Ним хмару, міцну, немов дзеркало лите?
\end{tcolorbox}
\begin{tcolorbox}
\textsubscript{19} Навчи нас, що скажем Йому? Через темність ми не впорядкуємо слова.
\end{tcolorbox}
\begin{tcolorbox}
\textsubscript{20} Чи Йому оповісться, що буду казати? Чи зміг хто сказати, що Він знищений буде?
\end{tcolorbox}
\begin{tcolorbox}
\textsubscript{21} І тепер ми не бачимо світла, щоб світило у хмарах, та вітер перейде і вичистить їх.
\end{tcolorbox}
\begin{tcolorbox}
\textsubscript{22} Із півночі приходить воно, немов золото те, та над Богом величність страшна.
\end{tcolorbox}
\begin{tcolorbox}
\textsubscript{23} Всемогутній, Його не знайшли ми, Він могутній у силі, але Він не мучить нікого судом та великою правдою.
\end{tcolorbox}
\begin{tcolorbox}
\textsubscript{24} Тому нехай люди бояться Його, бо на всіх мудросердих не дивиться Він.
\end{tcolorbox}
\subsection{CHAPTER 38}
\begin{tcolorbox}
\textsubscript{1} Тоді відповів Господь Йову із бурі й сказав:
\end{tcolorbox}
\begin{tcolorbox}
\textsubscript{2} Хто то такий, що затемнює раду словами без розуму?
\end{tcolorbox}
\begin{tcolorbox}
\textsubscript{3} Підпережи но ти стегна свої, як мужчина, а Я буду питати тебе, ти ж Мені поясни!
\end{tcolorbox}
\begin{tcolorbox}
\textsubscript{4} Де ти був, коли землю основував Я? Розкажи, якщо маєш знання!
\end{tcolorbox}
\begin{tcolorbox}
\textsubscript{5} Хто основи її положив, чи ти знаєш? Або хто розтягнув по ній шнура?
\end{tcolorbox}
\begin{tcolorbox}
\textsubscript{6} У що підстави її позапущувані, або хто поклав камінь наріжний її,
\end{tcolorbox}
\begin{tcolorbox}
\textsubscript{7} коли разом співали всі зорі поранні та радісний окрик здіймали всі Божі сини?
\end{tcolorbox}
\begin{tcolorbox}
\textsubscript{8} І хто море воротами загородив, як воно виступало, немов би з утроби виходило,
\end{tcolorbox}
\begin{tcolorbox}
\textsubscript{9} коли хмари поклав Я за одіж йому, а імлу за його пелюшки,
\end{tcolorbox}
\begin{tcolorbox}
\textsubscript{10} і призначив йому Я границю Свою та поставив засува й ворота,
\end{tcolorbox}
\begin{tcolorbox}
\textsubscript{11} і сказав: Аж досі ти дійдеш, не далі, і тут ось межа твоїх хвиль гордовитих?
\end{tcolorbox}
\begin{tcolorbox}
\textsubscript{12} Чи за своїх днів ти наказував ранкові? Чи досвітній зорі показав її місце,
\end{tcolorbox}
\begin{tcolorbox}
\textsubscript{13} щоб хапалась за кінці землі та посипались з неї безбожні?
\end{tcolorbox}
\begin{tcolorbox}
\textsubscript{14} Земля змінюється, мов та глина печатки, і стають, немов одіж, вони!
\end{tcolorbox}
\begin{tcolorbox}
\textsubscript{15} І нехай від безбожних їх світло відійметься, а високе рамено зламається!
\end{tcolorbox}
\begin{tcolorbox}
\textsubscript{16} Чи ти сходив коли аж до морських джерел, і чи ти переходжувався дном безодні?
\end{tcolorbox}
\begin{tcolorbox}
\textsubscript{17} Чи для тебе відкриті були брами смерти, і чи бачив ти брами смертельної тіні?
\end{tcolorbox}
\begin{tcolorbox}
\textsubscript{18} Чи широкість землі ти оглянув? Розкажи, якщо знаєш це все!
\end{tcolorbox}
\begin{tcolorbox}
\textsubscript{19} Де та дорога, що світло на ній пробуває? А темрява де її місце,
\end{tcolorbox}
\begin{tcolorbox}
\textsubscript{20} щоб узяти її до границі її, і щоб знати стежки її дому?
\end{tcolorbox}
\begin{tcolorbox}
\textsubscript{21} Знаєш ти, бо тоді народився ж ти був, і велике число твоїх днів!
\end{tcolorbox}
\begin{tcolorbox}
\textsubscript{22} Чи доходив коли ти до схованок снігу, і схованки граду ти бачив,
\end{tcolorbox}
\begin{tcolorbox}
\textsubscript{23} які Я тримаю на час лихоліття, на день бою й війни?
\end{tcolorbox}
\begin{tcolorbox}
\textsubscript{24} Якою дорогою ділиться вітер, розпорошується по землі вітерець?
\end{tcolorbox}
\begin{tcolorbox}
\textsubscript{25} Хто для зливи протоку провів, а для громовиці дорогу,
\end{tcolorbox}
\begin{tcolorbox}
\textsubscript{26} щоб дощити на землю безлюдну, на пустиню, в якій чоловіка нема,
\end{tcolorbox}
\begin{tcolorbox}
\textsubscript{27} щоб пустиню та пущу насичувати, і щоб забезпечити вихід траві?
\end{tcolorbox}
\begin{tcolorbox}
\textsubscript{28} Чи є батько в доща, чи хто краплі роси породив?
\end{tcolorbox}
\begin{tcolorbox}
\textsubscript{29} Із чиєї утроби лід вийшов, а іній небесний хто його породив?
\end{tcolorbox}
\begin{tcolorbox}
\textsubscript{30} Як камінь, тужавіють води, а поверхня безодні ховається.
\end{tcolorbox}
\begin{tcolorbox}
\textsubscript{31} Чи зв'яжеш ти зав'язки Волосожару, чи розв'яжеш віжки в Оріона?
\end{tcolorbox}
\begin{tcolorbox}
\textsubscript{32} Чи виведеш часу свого Зодіяка, чи Воза з синами його попровадиш?
\end{tcolorbox}
\begin{tcolorbox}
\textsubscript{33} Чи ти знаєш устави небес? Чи ти покладеш на землі їхню владу?
\end{tcolorbox}
\begin{tcolorbox}
\textsubscript{34} Чи підіймеш свій голос до хмар, і багато води тебе вкриє?
\end{tcolorbox}
\begin{tcolorbox}
\textsubscript{35} Чи блискавки ти посилаєш, і підуть вони, й тобі скажуть Ось ми?
\end{tcolorbox}
\begin{tcolorbox}
\textsubscript{36} Хто мудрість вкладає людині в нутро? Або хто дає серцеві розум?
\end{tcolorbox}
\begin{tcolorbox}
\textsubscript{37} Хто мудрістю хмари зрахує, і хто може затримати небесні посуди,
\end{tcolorbox}
\begin{tcolorbox}
\textsubscript{38} коли порох зливається в зливки, а кавалки злипаються?
\end{tcolorbox}
\begin{tcolorbox}
\textsubscript{39} Чи здобич левиці ти зловиш, і заспокоїш життя левчуків,
\end{tcolorbox}
\begin{tcolorbox}
\textsubscript{40} як вони по леговищах туляться, на чатах сидять по кущах?
\end{tcolorbox}
\begin{tcolorbox}
\textsubscript{41} Хто готує для крука поживу його, як до Бога кричать його діти, як без їжі блукають вони?
\end{tcolorbox}
\subsection{CHAPTER 39}
\begin{tcolorbox}
\textsubscript{1} Хіба ти пізнав час народження скельних козиць? Хіба ти пильнував час мук породу лані?
\end{tcolorbox}
\begin{tcolorbox}
\textsubscript{2} Чи на місяці лічиш, що сповнитись мусять, і відаєш час їх народження,
\end{tcolorbox}
\begin{tcolorbox}
\textsubscript{3} коли приклякають вони, випускають дітей своїх, і звільняються від болів породу?
\end{tcolorbox}
\begin{tcolorbox}
\textsubscript{4} Набираються сил їхні діти, на полі зростають, відходять і більше до них не вертаються.
\end{tcolorbox}
\begin{tcolorbox}
\textsubscript{5} Хто пустив осла дикого вільним, і хто розв'язав ослу дикому пута,
\end{tcolorbox}
\begin{tcolorbox}
\textsubscript{6} якому призначив Я степ його домом, а місцем його пробування солону пустиню?
\end{tcolorbox}
\begin{tcolorbox}
\textsubscript{7} Він сміється із галасу міста, не чує він крику погонича.
\end{tcolorbox}
\begin{tcolorbox}
\textsubscript{8} Що знаходить по горах, то паша його, і шукає він усього зеленого.
\end{tcolorbox}
\begin{tcolorbox}
\textsubscript{9} Чи захоче служити тобі одноріг? Чи при яслах твоїх ночуватиме він?
\end{tcolorbox}
\begin{tcolorbox}
\textsubscript{10} Чи ти однорога прив'яжеш до його борозни повороззям? Чи буде він боронувати за тобою долини?
\end{tcolorbox}
\begin{tcolorbox}
\textsubscript{11} Чи повіриш йому через те, що має він силу велику, і свою працю на нього попустиш?
\end{tcolorbox}
\begin{tcolorbox}
\textsubscript{12} Чи повіриш йому, що він верне насіння твоє, і збере тобі тік?
\end{tcolorbox}
\begin{tcolorbox}
\textsubscript{13} Крило струсеве радісно б'ється, чи ж крило це й пір'їна лелеки?
\end{tcolorbox}
\begin{tcolorbox}
\textsubscript{14} Бо яйця свої він на землю кладе та в поросі їх вигріває,
\end{tcolorbox}
\begin{tcolorbox}
\textsubscript{15} і забува, що нога може їх розчавити, а звір польовий може їх розтоптати.
\end{tcolorbox}
\begin{tcolorbox}
\textsubscript{16} Він жорстокий відносно дітей своїх, ніби вони не його, а що праця його може бути надаремна, того не боїться,
\end{tcolorbox}
\begin{tcolorbox}
\textsubscript{17} бо Бог учинив, щоб забув він про мудрість, і не наділив його розумом.
\end{tcolorbox}
\begin{tcolorbox}
\textsubscript{18} А за часу надходу стрільців ударяє він крильми повітря, і сміється з коня та з його верхівця!
\end{tcolorbox}
\begin{tcolorbox}
\textsubscript{19} Чи ти силу коневі даси, чи шию його ти зодягнеш у гриву?
\end{tcolorbox}
\begin{tcolorbox}
\textsubscript{20} Чи ти зробиш, що буде скакати він, мов сарана? Величне іржання його страшелезне!
\end{tcolorbox}
\begin{tcolorbox}
\textsubscript{21} Б'є ногою в долині та тішиться силою, іде він насупроти зброї,
\end{tcolorbox}
\begin{tcolorbox}
\textsubscript{22} сміється з страху й не жахається, і не вертається з-перед меча,
\end{tcolorbox}
\begin{tcolorbox}
\textsubscript{23} хоч дзвонить над ним сагайдак, вістря списове та ратище!
\end{tcolorbox}
\begin{tcolorbox}
\textsubscript{24} Він із шаленістю та лютістю землю ковтає, і не вірить, що чути гук рогу.
\end{tcolorbox}
\begin{tcolorbox}
\textsubscript{25} При кожному розі кричить він: І-га! і винюхує здалека бій, грім гетьманів та крик.
\end{tcolorbox}
\begin{tcolorbox}
\textsubscript{26} Чи яструб літає твоєю премудрістю, на південь простягує крила свої?
\end{tcolorbox}
\begin{tcolorbox}
\textsubscript{27} Чи з твойого наказу орел підіймається, і мостить кубло своє на висоті?
\end{tcolorbox}
\begin{tcolorbox}
\textsubscript{28} На скелі замешкує він та ночує, на скельнім вершку та твердині,
\end{tcolorbox}
\begin{tcolorbox}
\textsubscript{29} ізвідти визорює їжу, далеко вдивляються очі його,
\end{tcolorbox}
\begin{tcolorbox}
\textsubscript{30} а його пташенята п'ють кров. Де ж забиті, там він.
\end{tcolorbox}
\subsection{CHAPTER 40}
\begin{tcolorbox}
\textsubscript{1} І говорив Господь Йову й сказав:
\end{tcolorbox}
\begin{tcolorbox}
\textsubscript{2} Чи буде ставати на прю з Всемогутнім огудник? Хто сперечається з Богом, хай на це відповість!
\end{tcolorbox}
\begin{tcolorbox}
\textsubscript{3} І Йов відповів Господеві й сказав:
\end{tcolorbox}
\begin{tcolorbox}
\textsubscript{4} Оце я знікчемнів, що ж маю Тобі відповісти? Я кладу свою руку на уста свої...
\end{tcolorbox}
\begin{tcolorbox}
\textsubscript{5} Я раз говорив був, і вже не скажу, а вдруге і більш не додам!...
\end{tcolorbox}
\begin{tcolorbox}
\textsubscript{6} І відповів Господь Йову із бурі й сказав:
\end{tcolorbox}
\begin{tcolorbox}
\textsubscript{7} Підпережи но ти стегна свої, як мужчина: Я буду питати тебе, ти ж пояснюй Мені!
\end{tcolorbox}
\begin{tcolorbox}
\textsubscript{8} Чи ти хочеш порушити право Моє, винуватити Мене, щоб оправданим бути?
\end{tcolorbox}
\begin{tcolorbox}
\textsubscript{9} Коли маєш рамено, як Бог, і голосом ти загримиш, немов Він,
\end{tcolorbox}
\begin{tcolorbox}
\textsubscript{10} то окрась Ти себе пишнотою й величністю, зодягнися у славу й красу!
\end{tcolorbox}
\begin{tcolorbox}
\textsubscript{11} Розпорош лютість гніву свого, і поглянь на все горде й принизь ти його!
\end{tcolorbox}
\begin{tcolorbox}
\textsubscript{12} Поглянь на все горде й його впокори, поспихай нечестивих на їхньому місці,
\end{tcolorbox}
\begin{tcolorbox}
\textsubscript{13} поховай їх у поросі разом, а їхні обличчя обвий в укритті.
\end{tcolorbox}
\begin{tcolorbox}
\textsubscript{14} Тоді й Я тебе славити буду, як правиця твоя допоможе тобі!
\end{tcolorbox}
\begin{tcolorbox}
\textsubscript{15} А ось бегемот, що його Я створив, як тебе, траву, як худоба велика, він їсть.
\end{tcolorbox}
\begin{tcolorbox}
\textsubscript{16} Ото сила його в його стегнах, його ж міцність у м'язах його живота.
\end{tcolorbox}
\begin{tcolorbox}
\textsubscript{17} Випростовує він, немов кедра, свойого хвоста, жили стегон його посплітались.
\end{tcolorbox}
\begin{tcolorbox}
\textsubscript{18} Його кості немов мідяні оті рури, костомахи його як ті пруття залізні.
\end{tcolorbox}
\begin{tcolorbox}
\textsubscript{19} Голова оце Божих доріг; і тільки Творець його може зблизити до нього меча...
\end{tcolorbox}
\begin{tcolorbox}
\textsubscript{20} Бо гори приносять поживу йому, і там грається вся звірина польова.
\end{tcolorbox}
\begin{tcolorbox}
\textsubscript{21} Під лотосами він вилежується, в укритті очерету й болота.
\end{tcolorbox}
\begin{tcolorbox}
\textsubscript{22} Лотоси тінню своєю вкривають його, тополі поточні його обгортають.
\end{tcolorbox}
\begin{tcolorbox}
\textsubscript{23} Ось підіймається річка, та він не боїться її, він безпечний, хоча б сам Йордан йому в пащу впливав!
\end{tcolorbox}
\begin{tcolorbox}
\textsubscript{24} Хто може схопити його в його очах, гаками ніздрю продіравити?
\end{tcolorbox}
\subsection{CHAPTER 41}
\begin{tcolorbox}
\textsubscript{1} (40-25) Чи левіятана потягнеш гачком, і йому язика стягнеш шнуром?
\end{tcolorbox}
\begin{tcolorbox}
\textsubscript{2} (40-26) Чи очеретину вкладеш йому в ніздря, чи терниною щоку йому продіравиш?
\end{tcolorbox}
\begin{tcolorbox}
\textsubscript{3} (40-27) Чи він буде багато благати тебе, чи буде тобі говорити лагідне?
\end{tcolorbox}
\begin{tcolorbox}
\textsubscript{4} (40-28) Чи складе він умову з тобою, і ти візьмеш його за раба собі вічного?
\end{tcolorbox}
\begin{tcolorbox}
\textsubscript{5} (40-29) Чи ним бавитись будеш, як птахом, і прив'яжеш його для дівчаток своїх?
\end{tcolorbox}
\begin{tcolorbox}
\textsubscript{6} (40-30) Чи ним спільники торгуватимуть, чи поділять його між купців-хананеїв?
\end{tcolorbox}
\begin{tcolorbox}
\textsubscript{7} (40-31) Чи шпильками проколиш ти шкіру його, а острогою риб'ячою його голову?
\end{tcolorbox}
\begin{tcolorbox}
\textsubscript{8} (40-32) Поклади ж свою руку на нього, й згадай про війну, і більше того не чини!
\end{tcolorbox}
\begin{tcolorbox}
\textsubscript{9} (41-1) Тож надія твоя неправдива, на сам вигляд його упадеш.
\end{tcolorbox}
\begin{tcolorbox}
\textsubscript{10} (41-2) Нема смільчака, щоб його він збудив, а хто ж перед обличчям Моїм зможе стати?
\end{tcolorbox}
\begin{tcolorbox}
\textsubscript{11} (41-3) Хто вийде навпроти Мене й буде цілий? Що під небом усім це Моє!
\end{tcolorbox}
\begin{tcolorbox}
\textsubscript{12} (41-4) Не буду мовчати про члени його, про стан його сили й красу його складу.
\end{tcolorbox}
\begin{tcolorbox}
\textsubscript{13} (41-5) Хто відкриє поверхню одежі його? Хто підійде коли до двійних його щелепів?
\end{tcolorbox}
\begin{tcolorbox}
\textsubscript{14} (41-6) Двері обличчя його хто відчинить? Навколо зубів його жах!
\end{tcolorbox}
\begin{tcolorbox}
\textsubscript{15} (41-7) Його спина канали щитів, поєднання їх крем'яная печать.
\end{tcolorbox}
\begin{tcolorbox}
\textsubscript{16} (41-8) Одне до одного доходить, а вітер між ними не пройде.
\end{tcolorbox}
\begin{tcolorbox}
\textsubscript{17} (41-9) Одне до одного притверджені, сполучені, і не відділяться.
\end{tcolorbox}
\begin{tcolorbox}
\textsubscript{18} (41-10) Його чхання засвічує світло, а очі його як повіки зорі світової!
\end{tcolorbox}
\begin{tcolorbox}
\textsubscript{19} (41-11) Бухає полум'я з пащі його, вириваються іскри огненні!
\end{tcolorbox}
\begin{tcolorbox}
\textsubscript{20} (41-12) Із ніздер його валить дим, немов з того горшка, що кипить та біжить.
\end{tcolorbox}
\begin{tcolorbox}
\textsubscript{21} (41-13) Його подих розпалює вугіль, і бухає полум'я з пащі його.
\end{tcolorbox}
\begin{tcolorbox}
\textsubscript{22} (41-14) Сила ночує на шиї його, а страх перед ним утікає.
\end{tcolorbox}
\begin{tcolorbox}
\textsubscript{23} (41-15) М'ясо нутра його міцно тримається, воно в ньому тверде, не хитається.
\end{tcolorbox}
\begin{tcolorbox}
\textsubscript{24} (41-16) Його серце, мов з каменя вилите, і тверде, як те долішнє жорно!
\end{tcolorbox}
\begin{tcolorbox}
\textsubscript{25} (41-17) Як підводиться він, перелякуються силачі, та й ховаються з жаху.
\end{tcolorbox}
\begin{tcolorbox}
\textsubscript{26} (41-18) Той меч, що досягне його, не встоїть, ані спис, ані ратище й панцер.
\end{tcolorbox}
\begin{tcolorbox}
\textsubscript{27} (41-19) За солому залізо вважає, а мідь за гнилу деревину!
\end{tcolorbox}
\begin{tcolorbox}
\textsubscript{28} (41-20) Син лука, стріла, не примусит увтікати його, каміння із пращі для нього зміняється в сіно.
\end{tcolorbox}
\begin{tcolorbox}
\textsubscript{29} (41-21) Булаву уважає він за соломинку, і сміється із посвисту ратища.
\end{tcolorbox}
\begin{tcolorbox}
\textsubscript{30} (41-22) Під ним гостре череп'я, лягає на гостре, немов у болото.
\end{tcolorbox}
\begin{tcolorbox}
\textsubscript{31} (41-23) Чинить він, що кипить глибочінь, мов горня, і обертає море в окріп.
\end{tcolorbox}
\begin{tcolorbox}
\textsubscript{32} (41-24) Стежка світить за ним, а безодня здається йому сивиною.
\end{tcolorbox}
\begin{tcolorbox}
\textsubscript{33} (41-25) Немає подоби йому на землі, він безстрашним створений,
\end{tcolorbox}
\begin{tcolorbox}
\textsubscript{34} (41-26) він бачить усе, що високе, він цар над усім пишним звір'ям!
\end{tcolorbox}
\subsection{CHAPTER 42}
\begin{tcolorbox}
\textsubscript{1} А Йов відповів Господеві й сказав:
\end{tcolorbox}
\begin{tcolorbox}
\textsubscript{2} Я знаю, що можеш Ти все, і не спиняється задум у Тебе!
\end{tcolorbox}
\begin{tcolorbox}
\textsubscript{3} Хто ж то такий, що ховає пораду немудру? Тому я говорив, але не розумів... Це чудніше від мене, й не знаю його:
\end{tcolorbox}
\begin{tcolorbox}
\textsubscript{4} Слухай же ти, а Я буду казати, запитаю тебе, ти ж Мені поясни...
\end{tcolorbox}
\begin{tcolorbox}
\textsubscript{5} Тільки послухом уха я чув був про Тебе, а тепер моє око ось бачить Тебе...
\end{tcolorbox}
\begin{tcolorbox}
\textsubscript{6} Тому я зрікаюсь говореного, і каюсь у поросі й попелі!...
\end{tcolorbox}
\begin{tcolorbox}
\textsubscript{7} І сталося по тому, як Господь промовив ці слова до Йова, сказав Господь теманянину Еліфазові: Запалився Мій гнів на тебе та на двох твоїх приятелів, бо ви не говорили слушного про Мене, як раб Мій Йов.
\end{tcolorbox}
\begin{tcolorbox}
\textsubscript{8} А тепер візьміть собі сім бичків та сім баранів, і йдіть до Мого раба Йова, і принесете цілопалення за себе, а Мій раб Йов помолиться за вас, бо тільки з ним Я буду рахуватися, щоб не вчинити вам злої речі, бо ви не говорили слушного про Мене, як раб Мій Йов.
\end{tcolorbox}
\begin{tcolorbox}
\textsubscript{9} І пішли теманянин Еліфаз, і шух'янин Білдад, та нааматянин Цофар, і зробили, як говорив їм Господь. І споглянув Господь на Йова.
\end{tcolorbox}
\begin{tcolorbox}
\textsubscript{10} І Господь привернув Йова до першого стану, коли він помолився за своїх приятелів. І помножив Господь усе, що Йов мав, удвоє.
\end{tcolorbox}
\begin{tcolorbox}
\textsubscript{11} І поприходили до нього всі брати його, і всі сестри його та всі попередні знайомі його, і їли з ним хліб у його домі. І вони головою хитали над ним, та потішали його за все зле, що Господь був спровадив на нього. І дали вони йому кожен по одній кеситі, і кожен по одній золотій обручці.
\end{tcolorbox}
\begin{tcolorbox}
\textsubscript{12} А Господь поблагословив останок днів Йова більше від початку його, і було в нього чотирнадцять тисяч дрібної худоби, і шість тисяч верблюдів, тисяча пар худоби великої та тисяча ослиць.
\end{tcolorbox}
\begin{tcolorbox}
\textsubscript{13} І було в нього семеро синів та три дочки.
\end{tcolorbox}
\begin{tcolorbox}
\textsubscript{14} І назвав він ім'я першій: Єміма, і ім'я другій: Кеція, а ім'я третій: Керен-Гаппух.
\end{tcolorbox}
\begin{tcolorbox}
\textsubscript{15} І таких вродливих жінок, як Йовові дочки, не знайшлося по всій землі. І дав їм їх батько спадщину поміж їхніми братами.
\end{tcolorbox}
\begin{tcolorbox}
\textsubscript{16} А Йов жив по тому сотню й сорок років, і побачив синів своїх та синів синів своїх, чотири поколінні.
\end{tcolorbox}
\begin{tcolorbox}
\textsubscript{17} І впокоївся Йов старим та насиченим днями.
\end{tcolorbox}
