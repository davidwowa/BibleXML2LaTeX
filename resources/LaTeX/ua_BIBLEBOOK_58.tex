\section{BOOK 57}
\subsection{CHAPTER 1}
\begin{tcolorbox}
\textsubscript{1} Багато разів і багатьма способами в давнину промовляв був Бог до отців через пророків,
\end{tcolorbox}
\begin{tcolorbox}
\textsubscript{2} а в останні ці дні промовляв Він до нас через Сина, що Його настановив за Наслідника всього, що Ним і віки Він створив.
\end{tcolorbox}
\begin{tcolorbox}
\textsubscript{3} Він був сяєвом слави та образом істоти Його, тримав усе словом сили Своєї, учинив Собою очищення наших гріхів, і засів на правиці величности на висоті.
\end{tcolorbox}
\begin{tcolorbox}
\textsubscript{4} Він остільки був ліпший понад Анголів, оскільки славніше за них успадкував Ім'я.
\end{tcolorbox}
\begin{tcolorbox}
\textsubscript{5} Кому бо коли з Анголів Він промовив: Ти Мій Син, Я сьогодні Тебе породив! І знову: Я буду Йому за Отця, а Він Мені буде за Сина!
\end{tcolorbox}
\begin{tcolorbox}
\textsubscript{6} І коли знов Він уводить на світ Перворідного, то говорить: І нехай Йому вклоняться всі Анголи Божі.
\end{tcolorbox}
\begin{tcolorbox}
\textsubscript{7} А про Анголів Він говорить: Ти чиниш духів Анголами Своїми, а палючий огонь Своїми слугами.
\end{tcolorbox}
\begin{tcolorbox}
\textsubscript{8} А про Сина: Престол Твій, о Боже, навік віку; берло Твого царювання берло праведности.
\end{tcolorbox}
\begin{tcolorbox}
\textsubscript{9} Ти полюбив праведність, а беззаконня зненавидів; через це намастив Тебе, Боже, Твій Бог оливою радости більше, ніж друзів Твоїх.
\end{tcolorbox}
\begin{tcolorbox}
\textsubscript{10} І: Ти, Господи, землю колись заклав, а небо то чин Твоїх рук.
\end{tcolorbox}
\begin{tcolorbox}
\textsubscript{11} Загинуть вони, а Ти будеш стояти, всі вони, як той одяг, постаріють.
\end{tcolorbox}
\begin{tcolorbox}
\textsubscript{12} Як одежу, їх зміниш, і минуться вони, а Ти завжди Той Самий, і роки Твої не закінчаться!
\end{tcolorbox}
\begin{tcolorbox}
\textsubscript{13} Кому з Анголів Він промовив коли: Сядь праворуч Мене, доки не покладу Я Твоїх ворогів підніжком ногам Твоїм!
\end{tcolorbox}
\begin{tcolorbox}
\textsubscript{14} Чи не всі вони духи служебні, що їх посилають на службу для тих, хто має спасіння вспадкувати?
\end{tcolorbox}
\subsection{CHAPTER 2}
\begin{tcolorbox}
\textsubscript{1} Через це подобає нам більше вважати на почуте, щоб ми не відпали коли.
\end{tcolorbox}
\begin{tcolorbox}
\textsubscript{2} Коли бо те слово, що сказали його Анголи, було певне, а всякий переступ та непослух прийняли справедливу заплату,
\end{tcolorbox}
\begin{tcolorbox}
\textsubscript{3} то як ми втечемо, коли ми не дбали про таке велике спасіння? Воно проповідувалося спочатку від Господа, ствердилося нам через тих, хто почув,
\end{tcolorbox}
\begin{tcolorbox}
\textsubscript{4} коли Бог був засвідчив ознаками й чудами, і різними силами та обдаруванням Духом Святим із волі Своєї.
\end{tcolorbox}
\begin{tcolorbox}
\textsubscript{5} Бо Він не піддав Анголам світ майбутній, що про нього говоримо.
\end{tcolorbox}
\begin{tcolorbox}
\textsubscript{6} Але хтось десь засвідчив був, кажучи: Що є чоловік, що Ти пам'ятаєш про нього, і син людський, якого відвідуєш?
\end{tcolorbox}
\begin{tcolorbox}
\textsubscript{7} Ти його вчинив мало меншим від Анголів, і честю й величністю Ти вінчаєш його, і поставив його над ділами рук Своїх,
\end{tcolorbox}
\begin{tcolorbox}
\textsubscript{8} усе піддав Ти під ноги йому! А коли Він піддав йому все, то не залишив нічого йому непідданого. А тепер ще не бачимо, щоб піддане було йому все.
\end{tcolorbox}
\begin{tcolorbox}
\textsubscript{9} Але бачимо Ісуса, мало чим уменшеним від Анголів, що за перетерплення смерти Він увінчаний честю й величністю, щоб за благодаттю Божою смерть скуштувати за всіх.
\end{tcolorbox}
\begin{tcolorbox}
\textsubscript{10} Бо належало, щоб Той, що все ради Нього й усе від Нього, Хто до слави привів багато синів, Провідника їхнього спасіння вчинив досконалим через страждання.
\end{tcolorbox}
\begin{tcolorbox}
\textsubscript{11} Бо Хто освячує, і ті, хто освячується усі від Одного. З цієї причини не соромиться Він звати братами їх, кажучи:
\end{tcolorbox}
\begin{tcolorbox}
\textsubscript{12} Сповіщу про Ім'я Твоє браттям Своїм, буду хвалити Тебе серед Церкви!
\end{tcolorbox}
\begin{tcolorbox}
\textsubscript{13} І ще: На Нього я буду надіятися! І ще: Ото Я та діти, яких Бог Мені дав.
\end{tcolorbox}
\begin{tcolorbox}
\textsubscript{14} А що діти стали спільниками тіла та крови, то й Він став учасником їхнім, щоб смертю знищити того, хто має владу смерти, цебто диявола,
\end{tcolorbox}
\begin{tcolorbox}
\textsubscript{15} та визволити тих усіх, хто все життя страхом смерти тримався в неволі.
\end{tcolorbox}
\begin{tcolorbox}
\textsubscript{16} Бо приймає Він не Анголів, але Авраамове насіння.
\end{tcolorbox}
\begin{tcolorbox}
\textsubscript{17} Тому мусів бути Він у всьому подібний братам, щоб стати милостивим та вірним Первосвящеником у Божих справах, для вблагання за гріхи людей.
\end{tcolorbox}
\begin{tcolorbox}
\textsubscript{18} Бо в чому був Сам постраждав, випробовуваний, у тому Він може й випробовуваним помогти.
\end{tcolorbox}
\subsection{CHAPTER 3}
\begin{tcolorbox}
\textsubscript{1} Отож, святі брати, учасники небесного покликання, уважайте на Апостола й Первосвященика нашого ісповідання, Ісуса,
\end{tcolorbox}
\begin{tcolorbox}
\textsubscript{2} що вірний Тому, Хто настановив Його, як був і Мойсей у всім домі Його,
\end{tcolorbox}
\begin{tcolorbox}
\textsubscript{3} бо гідний Він вищої слави понад Мойсея, поскільки будівничий має більшу честь, аніж дім.
\end{tcolorbox}
\begin{tcolorbox}
\textsubscript{4} Усякий бо дім хтось будує, а Той, хто все збудував, то Бог.
\end{tcolorbox}
\begin{tcolorbox}
\textsubscript{5} І Мойсей вірний був у всім домі Його, як слуга, на свідоцтво того, що сказати повинно було.
\end{tcolorbox}
\begin{tcolorbox}
\textsubscript{6} Христос же, як Син, у Його домі. А дім Його ми, коли тільки відвагу й похвалу надії додержимо певними аж до кінця.
\end{tcolorbox}
\begin{tcolorbox}
\textsubscript{7} Тому то, як каже Дух Святий: Сьогодні, як голос Його ви почуєте,
\end{tcolorbox}
\begin{tcolorbox}
\textsubscript{8} не робіть затверділими ваших сердець, як під час нарікань, за дня випробовування на пустині,
\end{tcolorbox}
\begin{tcolorbox}
\textsubscript{9} де Мене випробовували отці ваші, Мене випробовували, і бачили працю Мою сорок років.
\end{tcolorbox}
\begin{tcolorbox}
\textsubscript{10} Через це Я розгнівався був на той рід і сказав: Постійно вони блудять серцем, вони не пізнали доріг Моїх,
\end{tcolorbox}
\begin{tcolorbox}
\textsubscript{11} тому Я присягнув був у гніві Своїм, що вони до Мого відпочинку не ввійдуть!
\end{tcolorbox}
\begin{tcolorbox}
\textsubscript{12} Стережіться, брати, щоб у комусь із вас не було злого серця невірства, що воно відступало б від Бога Живого!
\end{tcolorbox}
\begin{tcolorbox}
\textsubscript{13} Але кожного дня заохочуйте один одного, доки зветься Сьогодні, щоб запеклим не став котрий з вас через підступ гріха.
\end{tcolorbox}
\begin{tcolorbox}
\textsubscript{14} Бо ми стали учасниками Христа, коли тільки почате життя ми затримаємо певним аж до кінця,
\end{tcolorbox}
\begin{tcolorbox}
\textsubscript{15} аж поки говориться: Сьогодні, як голос Його ви почуєте, не робіть затверділими ваших сердець, як під час нарікань!
\end{tcolorbox}
\begin{tcolorbox}
\textsubscript{16} Котрі бо, почувши, розгнівали Бога? Чи не всі, хто з Єгипту вийшов з Мойсеєм?
\end{tcolorbox}
\begin{tcolorbox}
\textsubscript{17} На кого ж Він гнівався був сорок років? Хіба не на тих, хто згрішив, що їхні кості в пустині полягли?
\end{tcolorbox}
\begin{tcolorbox}
\textsubscript{18} Проти кого Він був присягався, що не ввійдуть вони до Його відпочинку, як не проти неслухняних?
\end{tcolorbox}
\begin{tcolorbox}
\textsubscript{19} І ми бачимо, що вони не змогли ввійти за невірство.
\end{tcolorbox}
\subsection{CHAPTER 4}
\begin{tcolorbox}
\textsubscript{1} Отже, біймося, коли зостається обітниця входу до Його відпочинку, щоб не виявилось, що хтось із вас опізнився.
\end{tcolorbox}
\begin{tcolorbox}
\textsubscript{2} Бо Євангелія була звіщена нам, як і тим. Але не принесло пожитку їм слово почуте, бо воно не злучилося з вірою слухачів.
\end{tcolorbox}
\begin{tcolorbox}
\textsubscript{3} Бо до Його відпочинку входимо ми, що ввірували, як Він провістив: Я присяг був у гніві Своїм, що до місця Мого відпочинку не ввійдуть вони, хоч діла Його були вчинені від закладин світу.
\end{tcolorbox}
\begin{tcolorbox}
\textsubscript{4} Бо колись про день сьомий сказав Він отак: І Бог відпочив сьомого дня від усієї праці Своєї.
\end{tcolorbox}
\begin{tcolorbox}
\textsubscript{5} А ще тут: До Мого відпочинку не ввійдуть вони!
\end{tcolorbox}
\begin{tcolorbox}
\textsubscript{6} Коли ж залишається ото, що деякі ввійдуть до нього, а ті, кому Євангелія була перше звіщена, не ввійшли за непослух,
\end{tcolorbox}
\begin{tcolorbox}
\textsubscript{7} то ще призначає Він деякий день, сьогодні, бо через Давида говорить по такім довгім часі, як вище вже сказано: Сьогодні, як голос Його ви почуєте, не робіть затверділими ваших сердець!
\end{tcolorbox}
\begin{tcolorbox}
\textsubscript{8} Бо коли б Ісус Навин дав їм відпочинок, то про інший день не казав би по цьому.
\end{tcolorbox}
\begin{tcolorbox}
\textsubscript{9} Отож, людові Божому залишається суботство, спочинок.
\end{tcolorbox}
\begin{tcolorbox}
\textsubscript{10} Хто бо ввійшов був у Його відпочинок, то й той відпочив від учинків своїх, як і Бог від Своїх.
\end{tcolorbox}
\begin{tcolorbox}
\textsubscript{11} Отож, попильнуймо ввійти до того відпочинку, щоб ніхто не потрапив у непослух за прикладом тим.
\end{tcolorbox}
\begin{tcolorbox}
\textsubscript{12} Бо Боже Слово живе та діяльне, гостріше від усякого меча обосічного, проходить воно аж до поділу душі й духа, суглобів та мозків, і спосібне судити думки та наміри серця.
\end{tcolorbox}
\begin{tcolorbox}
\textsubscript{13} І немає створіння, щоб сховалось перед Ним, але все наге та відкрите перед очима Його, Йому дамо звіт!
\end{tcolorbox}
\begin{tcolorbox}
\textsubscript{14} Отож, мавши великого Первосвященика, що небо перейшов, Ісуса, Сина Божого, тримаймося ісповідання нашого!
\end{tcolorbox}
\begin{tcolorbox}
\textsubscript{15} Бо ми маємо не такого Первосвященика, що не міг би співчувати слабостям нашим, але випробуваного в усьому, подібно до нас, окрім гріха.
\end{tcolorbox}
\begin{tcolorbox}
\textsubscript{16} Отож, приступаймо з відвагою до престолу благодаті, щоб прийняти милість та для своєчасної допомоги знайти благодать.
\end{tcolorbox}
\subsection{CHAPTER 5}
\begin{tcolorbox}
\textsubscript{1} Кожен бо первосвященик, що з-між людей вибирається, настановляється для людей на служіння Богові, щоб приносити дари та жертви за гріхи,
\end{tcolorbox}
\begin{tcolorbox}
\textsubscript{2} і щоб міг співчувати недосвідченим та заблудженим, бо й сам він перейнятий слабістю.
\end{tcolorbox}
\begin{tcolorbox}
\textsubscript{3} І тому він повинен як за людей, так само й за себе самого приносити жертви за гріхи.
\end{tcolorbox}
\begin{tcolorbox}
\textsubscript{4} А чести цієї ніхто не бере сам собою, а покликаний Богом, як і Аарон.
\end{tcolorbox}
\begin{tcolorbox}
\textsubscript{5} Так і Христос, не Сам Він прославив Себе, щоб Первосвящеником стати, а Той, що до Нього сказав: Ти Мій Син, Я сьогодні Тебе породив.
\end{tcolorbox}
\begin{tcolorbox}
\textsubscript{6} Як і на іншому місці говорить: Ти Священик навіки за чином Мелхиседековим.
\end{tcolorbox}
\begin{tcolorbox}
\textsubscript{7} Він за днів тіла Свого з голосінням великим та слізьми приніс був благання й молитви до Того, хто від смерти Його міг спасти, і був вислуханий за побожність Свою.
\end{tcolorbox}
\begin{tcolorbox}
\textsubscript{8} І хоч Сином Він був, проте навчився послуху з того, що вистраждав був.
\end{tcolorbox}
\begin{tcolorbox}
\textsubscript{9} А вдосконалившися, Він для всіх, хто слухняний Йому, спричинився для вічного спасіння,
\end{tcolorbox}
\begin{tcolorbox}
\textsubscript{10} і від Бога був названий Первосвящеником за чином Мелхиседековим.
\end{tcolorbox}
\begin{tcolorbox}
\textsubscript{11} Про це нам би треба багато казати, та висловити важко його, бо нездібні ви стали, щоб слухати.
\end{tcolorbox}
\begin{tcolorbox}
\textsubscript{12} Ви бо за віком повинні б бути вчителями, але ви потребуєте ще, щоб хтось вас навчав перших початків Божого Слова. І ви стали такими, яким потрібне молоко, а не страва тверда.
\end{tcolorbox}
\begin{tcolorbox}
\textsubscript{13} Бо хто молока вживає, той недосвідчений у слові правди, бо він немовля.
\end{tcolorbox}
\begin{tcolorbox}
\textsubscript{14} А страва тверда для дорослих, що мають чуття, привчені звичкою розрізняти добро й зло.
\end{tcolorbox}
\subsection{CHAPTER 6}
\begin{tcolorbox}
\textsubscript{1} Тому полишімо початки науки Христа, та й звернімося до досконалости, і не кладімо знову засади покаяння від мертвих учинків та про віру в Бога,
\end{tcolorbox}
\begin{tcolorbox}
\textsubscript{2} науки про хрищення, про покладання рук, про воскресіння мертвих та вічний суд.
\end{tcolorbox}
\begin{tcolorbox}
\textsubscript{3} Зробимо й це, коли Бог дозволить.
\end{tcolorbox}
\begin{tcolorbox}
\textsubscript{4} Не можна бо тих, що раз просвітились були, і скуштували небесного дару, і стали причасниками Духа Святого,
\end{tcolorbox}
\begin{tcolorbox}
\textsubscript{5} і скуштували доброго Божого Слова та сили майбутнього віку,
\end{tcolorbox}
\begin{tcolorbox}
\textsubscript{6} та й відпали, знов відновляти покаянням, коли вдруге вони розпинають у собі Сина Божого та зневажають.
\end{tcolorbox}
\begin{tcolorbox}
\textsubscript{7} Бо земля, що п'є дощ, який падає часто на неї, і родить рослини, добрі для тих, хто їх і вирощує, вона благословення від Бога приймає.
\end{tcolorbox}
\begin{tcolorbox}
\textsubscript{8} Але та, що приносить терня й будяччя, непотрібна вона та близька до прокляття, а кінець її спалення.
\end{tcolorbox}
\begin{tcolorbox}
\textsubscript{9} Та ми сподіваємось, любі, кращого про вас, що спасіння тримаєтеся, хоч говоримо й так.
\end{tcolorbox}
\begin{tcolorbox}
\textsubscript{10} Та не є Бог несправедливий, щоб забути діло ваше та працю любови, яку показали в Ім'я Його ви, що святим послужили та служите.
\end{tcolorbox}
\begin{tcolorbox}
\textsubscript{11} Ми ж бажаємо, щоб кожен із вас виявляв таку саму завзятість на певність надії аж до кінця,
\end{tcolorbox}
\begin{tcolorbox}
\textsubscript{12} щоб ви не розлінились, але переймали від тих, хто обітниці вспадковує вірою та терпеливістю.
\end{tcolorbox}
\begin{tcolorbox}
\textsubscript{13} Бо Бог, обітницю давши Авраамові, як не міг ніким вищим поклястися, поклявся Сам Собою,
\end{tcolorbox}
\begin{tcolorbox}
\textsubscript{14} говорячи: Поблагословити Я конче тебе поблагословлю, та розмножити розмножу тебе!
\end{tcolorbox}
\begin{tcolorbox}
\textsubscript{15} І, терплячи довго отак, Авраам одержав обітницю.
\end{tcolorbox}
\begin{tcolorbox}
\textsubscript{16} Бо люди клянуться вищим, і клятва на ствердження кінчає всяку їхню суперечку.
\end{tcolorbox}
\begin{tcolorbox}
\textsubscript{17} Тому й Бог, хотівши переважно показати спадкоємцям обітниці незмінність волі Своєї, учинив те при помочі клятви,
\end{tcolorbox}
\begin{tcolorbox}
\textsubscript{18} щоб у двох тих незмінних речах, що в них не можна сказати неправди Богові, мали потіху міцну ми, хто прибіг прийняти надію, що лежить перед нами,
\end{tcolorbox}
\begin{tcolorbox}
\textsubscript{19} що вони для душі як котвиця, міцна та безпечна, що аж до середини входить за заслону,
\end{tcolorbox}
\begin{tcolorbox}
\textsubscript{20} куди, як предтеча, за нас увійшов був Ісус, ставши навіки Первосвящеником за чином Мелхиседековим.
\end{tcolorbox}
\subsection{CHAPTER 7}
\begin{tcolorbox}
\textsubscript{1} Бо цей Мелхиседек, цар Салиму, священик Бога Всевишнього, що був стрів Авраама, як той вертався по поразці царів, і його поблагословив.
\end{tcolorbox}
\begin{tcolorbox}
\textsubscript{2} Авраам відділив йому й десятину від усього, найперше бо він визначає цар правди, а потім цар Салиму, цебто цар миру.
\end{tcolorbox}
\begin{tcolorbox}
\textsubscript{3} Він без батька, без матері, без родоводу, не мав ані початку днів, ані кінця життя, уподобився Божому Сину, пробуває священиком завжди.
\end{tcolorbox}
\begin{tcolorbox}
\textsubscript{4} Побачте ж, який він великий, що йому й десятину з добичі найліпшої дав патріярх Авраам!
\end{tcolorbox}
\begin{tcolorbox}
\textsubscript{5} Ті з синів Левієвих, що священство приймають, мають заповідь брати за Законом десятину з народу, цебто з братів своїх, хоч і вийшли вони з Авраамових стегон.
\end{tcolorbox}
\begin{tcolorbox}
\textsubscript{6} Але цей, що не походить з їхнього роду, десятину одержав від Авраама, і поблагословив того, хто обітницю мав.
\end{tcolorbox}
\begin{tcolorbox}
\textsubscript{7} І без усякої суперечки більший меншого благословляє.
\end{tcolorbox}
\begin{tcolorbox}
\textsubscript{8} І тут люди смертельні беруть десятину, а там той, про якого засвідчується, що живе.
\end{tcolorbox}
\begin{tcolorbox}
\textsubscript{9} І, щоб сказати отак, через Авраама і Левій, що бере десятини, дав сам десятини.
\end{tcolorbox}
\begin{tcolorbox}
\textsubscript{10} Бо ще в батькових стегнах він був, коли стрів його Мелхиседек.
\end{tcolorbox}
\begin{tcolorbox}
\textsubscript{11} Отож, коли б досконалість була через священство левитське, бо люди Закона одержали з ним, то яка ще потреба була, щоб Інший Священик повстав за чином Мелхиседековим, а не зватися за чином Аароновим?
\end{tcolorbox}
\begin{tcolorbox}
\textsubscript{12} Коли бо священство зміняється, то з потреби буває переміна й Закону.
\end{tcolorbox}
\begin{tcolorbox}
\textsubscript{13} Бо Той, що про Нього говориться це, належав до іншого племени, з якого ніхто не ставав був до жертівника.
\end{tcolorbox}
\begin{tcolorbox}
\textsubscript{14} Бож відомо, що Господь наш походить від Юди, а про це плем'я, про священство його, нічого Мойсей не сказав.
\end{tcolorbox}
\begin{tcolorbox}
\textsubscript{15} І ще більше відомо, коли повстає на подобу Мелхиседека Інший Священик,
\end{tcolorbox}
\begin{tcolorbox}
\textsubscript{16} що був не за законом тілесної заповіді, але з сили незнищального життя.
\end{tcolorbox}
\begin{tcolorbox}
\textsubscript{17} Бо свідчить: Ти Священик навіки за чином Мелхиседековим.
\end{tcolorbox}
\begin{tcolorbox}
\textsubscript{18} Попередня бо заповідь відкладається через неміч її та некорисність.
\end{tcolorbox}
\begin{tcolorbox}
\textsubscript{19} Бо не вдосконалив нічого Закон. Запроваджена ж краща надія, що нею ми наближуємось до Бога.
\end{tcolorbox}
\begin{tcolorbox}
\textsubscript{20} І поскільки воно не без клятви,
\end{tcolorbox}
\begin{tcolorbox}
\textsubscript{21} вони бо без клятви були священиками, Цей же з клятвою через Того, Хто говорить до Нього: Клявся Господь і не буде Він каятися: Ти Священик навіки за чином Мелхиседековим,
\end{tcolorbox}
\begin{tcolorbox}
\textsubscript{22} то постільки Ісус став запорукою кращого Заповіту!
\end{tcolorbox}
\begin{tcolorbox}
\textsubscript{23} І багато було їх священиків, бо смерть боронила лишатися їм,
\end{tcolorbox}
\begin{tcolorbox}
\textsubscript{24} але Цей, що навіки лишається, безперестанне Священство Він має.
\end{tcolorbox}
\begin{tcolorbox}
\textsubscript{25} Тому може Він завжди й спасати тих, хто через Нього до Бога приходить, бо Він завжди живий, щоб за них заступитись.
\end{tcolorbox}
\begin{tcolorbox}
\textsubscript{26} Отакий бо потрібний нам Первосвященик: святий, незлобивий, невинний, відлучений від грішників, що вищий над небеса,
\end{tcolorbox}
\begin{tcolorbox}
\textsubscript{27} що потреби не має щодня, як ті первосвященики, перше приносити жертви за власні гріхи, а потому за людські гріхи, бо Він це раз назавжди вчинив, принісши Самого Себе.
\end{tcolorbox}
\begin{tcolorbox}
\textsubscript{28} Закон бо людей ставить первосвящениками, що немочі мають, але слово клятви, що воно за Законом, ставить Сина, Який досконалий навіки!
\end{tcolorbox}
\subsection{CHAPTER 8}
\begin{tcolorbox}
\textsubscript{1} Головна ж річ у тому, про що я говорю: маємо Первосвященика, що засів на небесах, по правиці престолу величности,
\end{tcolorbox}
\begin{tcolorbox}
\textsubscript{2} що Він Священнослужитель святині й правдивої скинії, що її збудував був Господь, а не людина.
\end{tcolorbox}
\begin{tcolorbox}
\textsubscript{3} Усякий бо первосвященик настановляється, щоб приносити дари та жертви, а тому було треба, щоб і Цей щось мав, що принести.
\end{tcolorbox}
\begin{tcolorbox}
\textsubscript{4} Бо коли б на землі перебував, то не був би Він священиком, бо тут пробувають священики, що дари приносять за Законом.
\end{tcolorbox}
\begin{tcolorbox}
\textsubscript{5} Вони служать образові й тіні небесного, як Мойсеєві сказано, коли мав докінчити скинію: Дивись бо, сказав, зроби все за зразком, що тобі на горі був показаний!
\end{tcolorbox}
\begin{tcolorbox}
\textsubscript{6} А тепер одержав Він краще служіння, поскільки Він посередник і кращого заповіту, який на кращих обітницях був узаконений.
\end{tcolorbox}
\begin{tcolorbox}
\textsubscript{7} Бо коли б отой перший був бездоганний, не шукалося б місця для другого.
\end{tcolorbox}
\begin{tcolorbox}
\textsubscript{8} Бо їм докоряючи, каже: Ото дні надходять, говорить Господь, коли з домом Ізраїля й з Юдиним домом Я складу Заповіта Нового,
\end{tcolorbox}
\begin{tcolorbox}
\textsubscript{9} не за заповітом, що його Я склав був з отцями їхніми дня, коли взяв їх за руку, щоб вивести їх із землі єгипетської. А що вони не залишилися в Моїм заповіті, то й Я їх покинув, говорить Господь!
\end{tcolorbox}
\begin{tcolorbox}
\textsubscript{10} Оце Заповіт, що його Я складу по тих днях із домом Ізраїлевим, говорить Господь: Покладу Я Закони Свої в їхні думки, і на їхніх серцях напишу їх, і буду їм Богом, вони ж будуть народом Моїм!
\end{tcolorbox}
\begin{tcolorbox}
\textsubscript{11} І кожен не буде навчати свого ближнього, і кожен брата свого, промовляючи: Пізнай Господа! Усі бо вони будуть знати Мене від малого та аж до великого з них!
\end{tcolorbox}
\begin{tcolorbox}
\textsubscript{12} Буду бо Я милостивий до їхніх неправд, і їхніх гріхів не згадаю Я більш!
\end{tcolorbox}
\begin{tcolorbox}
\textsubscript{13} Коли ж каже Новий Заповіт, то тим назвав перший старим. А що порохнявіє й старіє, те близьке до зотління.
\end{tcolorbox}
\subsection{CHAPTER 9}
\begin{tcolorbox}
\textsubscript{1} Мав же і перший заповіт постанови богослужби та світську святиню.
\end{tcolorbox}
\begin{tcolorbox}
\textsubscript{2} Була бо уряджена перша скинія, яка зветься святиня, а в ній був свічник, і стіл, і жертвенні хліби.
\end{tcolorbox}
\begin{tcolorbox}
\textsubscript{3} А за другою заслоною скинія, що зветься Святеє Святих.
\end{tcolorbox}
\begin{tcolorbox}
\textsubscript{4} Мала вона золоту кадильницю й ковчега заповіту, усюди обкутого золотом, а в нім золота посудина з манною, і розцвіле жезло Ааронове та таблиці заповіту.
\end{tcolorbox}
\begin{tcolorbox}
\textsubscript{5} А над ним херувими слави, що затінювали престола благодаті, про що говорити докладно тепер не потрібно.
\end{tcolorbox}
\begin{tcolorbox}
\textsubscript{6} При такому ж урядженні до першої скинії входили завжди священики, правлячи служби Богові,
\end{tcolorbox}
\begin{tcolorbox}
\textsubscript{7} а до другої раз на рік сам первосвященик, не без крови, яку він приносить за себе й за людські провини.
\end{tcolorbox}
\begin{tcolorbox}
\textsubscript{8} Святий Дух виявляє оцим, що ще не відкрита дорога в святиню, коли ще стоїть перша скинія.
\end{tcolorbox}
\begin{tcolorbox}
\textsubscript{9} Вона образ для часу теперішнього, за якого приносяться дари та жертви, що того не можуть вдосконалити, щодо сумління того, хто служить,
\end{tcolorbox}
\begin{tcolorbox}
\textsubscript{10} що тільки в потравах та в напоях, та в різних обмиваннях, в уставах тілесних, установлено їх аж до часу направи.
\end{tcolorbox}
\begin{tcolorbox}
\textsubscript{11} Але Христос, Первосвященик майбутнього доброго, прийшов із більшою й досконалішою скинією, нерукотворною, цебто не цього втворення,
\end{tcolorbox}
\begin{tcolorbox}
\textsubscript{12} і не з кров'ю козлів та телят, але з власною кров'ю увійшов до святині один раз, та й набув вічне відкуплення.
\end{tcolorbox}
\begin{tcolorbox}
\textsubscript{13} Бо коли кров козлів та телят та попіл із ялівок, як покропить нечистих, освячує їх на очищення тіла,
\end{tcolorbox}
\begin{tcolorbox}
\textsubscript{14} скільки ж більш кров Христа, що Себе непорочного Богу приніс Святим Духом, очистить наше сумління від мертвих учинків, щоб служити нам Богові Живому!
\end{tcolorbox}
\begin{tcolorbox}
\textsubscript{15} Тому Він Посередник Нового Заповіту, щоб через смерть, що була для відкуплення від переступів, учинених за першого заповіту, покликані прийняли обітницю вічного спадку.
\end{tcolorbox}
\begin{tcolorbox}
\textsubscript{16} Бо де заповіт, там має відбутися смерть заповітника,
\end{tcolorbox}
\begin{tcolorbox}
\textsubscript{17} заповіт бо важливий по мертвих, бо нічого не варт він, як живе заповітник.
\end{tcolorbox}
\begin{tcolorbox}
\textsubscript{18} Тому й перший заповіт освячений був не без крови:
\end{tcolorbox}
\begin{tcolorbox}
\textsubscript{19} Коли бо Мойсей сповістив був усі заповіді за Законом усьому народові, він узяв кров козлів та телят із водою й червоною вовною та з ісопом, та й покропив і саму оту книгу, і людей,
\end{tcolorbox}
\begin{tcolorbox}
\textsubscript{20} проказуючи: Це кров заповіту, що його наказав для вас Бог!
\end{tcolorbox}
\begin{tcolorbox}
\textsubscript{21} Так само і скинію, і ввесь посуд служебний покропив він кров'ю.
\end{tcolorbox}
\begin{tcolorbox}
\textsubscript{22} І майже все за Законом кров'ю очищується, а без пролиття крови не має відпущення.
\end{tcolorbox}
\begin{tcolorbox}
\textsubscript{23} Отож, треба було, щоб образи небесного очищалися цими, а небесне саме кращими від оцих жертвами.
\end{tcolorbox}
\begin{tcolorbox}
\textsubscript{24} Бо Христос увійшов не в рукотворну святиню, що була на взір правдивої, але в саме небо, щоб з'явитись тепер перед Божим лицем за нас,
\end{tcolorbox}
\begin{tcolorbox}
\textsubscript{25} і не тому, щоб часто приносити в жертву Себе, як первосвященик увіходить у святиню кожнорічно із кров'ю чужою,
\end{tcolorbox}
\begin{tcolorbox}
\textsubscript{26} бо інакше Він мусів би часто страждати ще від закладин світу, а тепер Він з'явився один раз на схилку віків, щоб власною жертвою знищити гріх.
\end{tcolorbox}
\begin{tcolorbox}
\textsubscript{27} І як людям призначено вмерти один раз, потім же суд,
\end{tcolorbox}
\begin{tcolorbox}
\textsubscript{28} так і Христос один раз був у жертву принесений, щоб понести гріхи багатьох, і не в справі гріха другий раз з'явитися тим, хто чекає Його на спасіння.
\end{tcolorbox}
\subsection{CHAPTER 10}
\begin{tcolorbox}
\textsubscript{1} Бо Закон, мавши тільки тінь майбутнього добра, а не самий образ речей, тими самими жертвами, що завжди щороку приносяться, не може ніколи вдосконалити тих, хто приступає.
\end{tcolorbox}
\begin{tcolorbox}
\textsubscript{2} Інакше вони перестали б приноситись, бо ті, хто служить, очищені раз, уже б не мали жадної свідомости гріхів.
\end{tcolorbox}
\begin{tcolorbox}
\textsubscript{3} Але в них спомин про гріхи буває щороку,
\end{tcolorbox}
\begin{tcolorbox}
\textsubscript{4} бо тож неможливе, щоб кров биків та козлів здіймала гріхи!
\end{tcolorbox}
\begin{tcolorbox}
\textsubscript{5} Тому то, входячи в світ, Він говорить: Жертви й приношення Ти не схотів, але тіло Мені приготував.
\end{tcolorbox}
\begin{tcolorbox}
\textsubscript{6} Цілопалення й жертви покутної Ти не жадав.
\end{tcolorbox}
\begin{tcolorbox}
\textsubscript{7} Тоді Я сказав: Ось іду, в звої книжки про Мене написано, щоб волю чинити Твою, Боже!
\end{tcolorbox}
\begin{tcolorbox}
\textsubscript{8} Він вище сказав, що жертви й приносу, та цілопалення й жертви покутної, які за Законом приносяться, Ти не жадав і Собі не вподобав.
\end{tcolorbox}
\begin{tcolorbox}
\textsubscript{9} Потому сказав: Ось іду, щоб волю Твою чинити, Боже. Відміняє Він перше, щоб друге поставити.
\end{tcolorbox}
\begin{tcolorbox}
\textsubscript{10} У цій волі ми освячені жертвоприношенням тіла Ісуса Христа один раз.
\end{tcolorbox}
\begin{tcolorbox}
\textsubscript{11} І кожен священик щоденно стоїть, служачи, і часто приносить жертви ті самі, що ніколи не можуть зняти гріхів.
\end{tcolorbox}
\begin{tcolorbox}
\textsubscript{12} А Він за гріхи світу приніс жертву один раз, і назавжди по Божій правиці засів,
\end{tcolorbox}
\begin{tcolorbox}
\textsubscript{13} далі чекаючи, аж вороги Його будуть покладені за підніжка Його ніг.
\end{tcolorbox}
\begin{tcolorbox}
\textsubscript{14} Бо жертвоприношенням одним вдосконалив Він тих, хто освячується.
\end{tcolorbox}
\begin{tcolorbox}
\textsubscript{15} Свідкує ж і Дух Святий нам, як говорить:
\end{tcolorbox}
\begin{tcolorbox}
\textsubscript{16} Оце заповіт, що його по цих днях встановляю Я з ними, говорить Господь, Закони вої Я дам в їхні серця, і в їхніх думках напишу їх.
\end{tcolorbox}
\begin{tcolorbox}
\textsubscript{17} А їхніх гріхів та несправедливостей їхніх Я більш не згадаю!
\end{tcolorbox}
\begin{tcolorbox}
\textsubscript{18} А де їхнє відпущення, там нема вже жертвоприношення за гріхи.
\end{tcolorbox}
\begin{tcolorbox}
\textsubscript{19} Отож, браття, ми маємо відвагу входити до святині кров'ю Ісусовою,
\end{tcolorbox}
\begin{tcolorbox}
\textsubscript{20} новою й живою дорогою, яку нам обновив Він через завісу, цебто через тіло Своє,
\end{tcolorbox}
\begin{tcolorbox}
\textsubscript{21} маємо й Великого Священика над домом Божим,
\end{tcolorbox}
\begin{tcolorbox}
\textsubscript{22} то приступімо з щирим серцем, у повноті віри, окропивши серця від сумління лукавого та обмивши тіла чистою водою!
\end{tcolorbox}
\begin{tcolorbox}
\textsubscript{23} Тримаймо непохитне визнання надії, вірний бо Той, Хто обіцяв.
\end{tcolorbox}
\begin{tcolorbox}
\textsubscript{24} І уважаймо один за одним для заохоти до любови й до добрих учинків.
\end{tcolorbox}
\begin{tcolorbox}
\textsubscript{25} Не кидаймо збору свого, як то звичай у деяких, але заохочуймося, і тим більше, скільки більше ви бачите, що зближається день той.
\end{tcolorbox}
\begin{tcolorbox}
\textsubscript{26} Бо як ми грішимо самовільно, одержавши пізнання правди, то вже за гріхи не знаходиться жертви,
\end{tcolorbox}
\begin{tcolorbox}
\textsubscript{27} а страшливе якесь сподівання суду та гнів палючий, що має пожерти противників.
\end{tcolorbox}
\begin{tcolorbox}
\textsubscript{28} Хто відкидає Закона Мойсея, такий немилосердно вмирає при двох чи трьох свідках,
\end{tcolorbox}
\begin{tcolorbox}
\textsubscript{29} скільки ж більшої муки, додумуєтеся? заслуговує той, хто потоптав Сина Божого, і хто кров заповіту, що нею освячений, за звичайну вважав, і хто Духа благодаті зневажив!
\end{tcolorbox}
\begin{tcolorbox}
\textsubscript{30} Бо знаємо Того, Хто сказав: Мені помста належить, Я відплачу, говорить Господь. І ще: Господь буде судити народа Свого!
\end{tcolorbox}
\begin{tcolorbox}
\textsubscript{31} Страшна річ упасти в руки Бога Живого!
\end{tcolorbox}
\begin{tcolorbox}
\textsubscript{32} Згадайте ж про перші дні ваші, як ви просвітилися й витерпіли запеклу боротьбу страждань.
\end{tcolorbox}
\begin{tcolorbox}
\textsubscript{33} Ви були то видовищем зневаги й знущання, то були учасниками тих, що жили так.
\end{tcolorbox}
\begin{tcolorbox}
\textsubscript{34} Ви бо страждали й з ув'язненими, і грабунок свого майна прийняли з потіхою, відаючи, що маєте в небі для себе майно неминуще та краще.
\end{tcolorbox}
\begin{tcolorbox}
\textsubscript{35} Тож не відкидайте відваги своєї, бо має велику нагороду вона.
\end{tcolorbox}
\begin{tcolorbox}
\textsubscript{36} Бо вам терпеливість потрібна, щоб Божу волю вчинити й прийняти обітницю.
\end{tcolorbox}
\begin{tcolorbox}
\textsubscript{37} Бо ще мало, дуже мало, і Той, хто має прийти, прийде й баритись не буде!
\end{tcolorbox}
\begin{tcolorbox}
\textsubscript{38} А праведний житиме вірою. І: Коли захитається він, то душа Моя його не вподобає.
\end{tcolorbox}
\begin{tcolorbox}
\textsubscript{39} Ми ж не з тих, хто хитається на загибіль, але віруємо на спасіння душі.
\end{tcolorbox}
\subsection{CHAPTER 11}
\begin{tcolorbox}
\textsubscript{1} А віра то підстава сподіваного, доказ небаченого.
\end{tcolorbox}
\begin{tcolorbox}
\textsubscript{2} Бо нею засвідчені старші були.
\end{tcolorbox}
\begin{tcolorbox}
\textsubscript{3} Вірою ми розуміємо, що віки Словом Божим збудовані, так що з невидимого сталось видиме.
\end{tcolorbox}
\begin{tcolorbox}
\textsubscript{4} Вірою Авель приніс Богові жертву кращу, як Каїн; нею засвідчений був, що він праведний, як Бог свідчив про дари його; нею, і вмерши, він ще промовляє.
\end{tcolorbox}
\begin{tcolorbox}
\textsubscript{5} Вірою Енох був перенесений на небо, щоб не бачити смерти; і його не знайшли, бо Бог переніс його. Бо раніш, як його перенесено, він був засвідчений, що Богові він догодив.
\end{tcolorbox}
\begin{tcolorbox}
\textsubscript{6} Догодити ж без віри не можна. І той, хто до Бога приходить, мусить вірувати, що Він є, а тим, хто шукає Його, Він дає нагороду.
\end{tcolorbox}
\begin{tcolorbox}
\textsubscript{7} Вірою Ной, як дістав був об'явлення про те, чого ще не бачив, побоявшись, зробив ковчега, щоб дім свій спасти; нею світ засудив він, і став спадкоємцем праведности, що з віри вона.
\end{tcolorbox}
\begin{tcolorbox}
\textsubscript{8} Вірою Авраам, покликаний на місце, яке мав прийняти в спадщину, послухався та й пішов, не відаючи, куди йде.
\end{tcolorbox}
\begin{tcolorbox}
\textsubscript{9} Вірою він перебував на Землі Обіцяній, як на чужій, і проживав у наметах з Ісаком та Яковом, співспадкоємцями тієї ж обітниці,
\end{tcolorbox}
\begin{tcolorbox}
\textsubscript{10} бо чекав він міста, що має підвалини, що Бог його будівничий та творець.
\end{tcolorbox}
\begin{tcolorbox}
\textsubscript{11} Вірою й Сара сама дістала силу прийняти насіння, і породила понад час свого віку, бо вірним вважала Того, Хто обітницю дав.
\end{tcolorbox}
\begin{tcolorbox}
\textsubscript{12} Тому й від одного, та ще змертвілого, народилось так багато, як зорі небесні й пісок незчисленний край моря.
\end{tcolorbox}
\begin{tcolorbox}
\textsubscript{13} Усі вони повмирали за вірою, не одержавши обітниць, але здалека бачили їх, і повітали, і вірували в них, та визнавали, що вони на землі чужаниці й приходьки.
\end{tcolorbox}
\begin{tcolorbox}
\textsubscript{14} Бо ті, що говорять таке, виявляють, що шукають батьківщини.
\end{tcolorbox}
\begin{tcolorbox}
\textsubscript{15} І коли б вони пам'ятали ту, що вийшли з неї, то мали б були час повернутись.
\end{tcolorbox}
\begin{tcolorbox}
\textsubscript{16} Та бажають вони тепер кращої, цебто небесної, тому й Бог не соромиться їх, щоб звати Себе їхнім Богом, бо Він приготував їм місто.
\end{tcolorbox}
\begin{tcolorbox}
\textsubscript{17} Вірою Авраам, випробовуваний, привів був на жертву Ісака, і, мавши обітницю, приніс однородженого,
\end{tcolorbox}
\begin{tcolorbox}
\textsubscript{18} що йому було сказано: В Ісакові буде насіння тобі.
\end{tcolorbox}
\begin{tcolorbox}
\textsubscript{19} Бо він розумів, що Бог має силу й воскресити з мертвих, тому й одержав його на прообраз.
\end{tcolorbox}
\begin{tcolorbox}
\textsubscript{20} Вірою в майбутнє поблагословив Ісак Якова та Ісава.
\end{tcolorbox}
\begin{tcolorbox}
\textsubscript{21} Вірою Яків, умираючи, поблагословив кожного сина Йосипового, і схилився на верх свого жезла.
\end{tcolorbox}
\begin{tcolorbox}
\textsubscript{22} Вірою Йосип, умираючи, згадав про вихід синів Ізраїлевих та про кості свої заповів.
\end{tcolorbox}
\begin{tcolorbox}
\textsubscript{23} Вірою Мойсей, як родився, переховувався батьками своїми три місяці, бо вони бачили, що гарне дитя, і не злякались наказу царевого.
\end{tcolorbox}
\begin{tcolorbox}
\textsubscript{24} Вірою Мойсей, коли виріс, відрікся зватися сином дочки фараонової.
\end{tcolorbox}
\begin{tcolorbox}
\textsubscript{25} Він хотів краще страждати з народом Божим, аніж мати дочасну гріховну потіху.
\end{tcolorbox}
\begin{tcolorbox}
\textsubscript{26} Він наругу Христову вважав за більше багатство, ніж скарби єгипетські, бо він озирався на Божу нагороду.
\end{tcolorbox}
\begin{tcolorbox}
\textsubscript{27} Вірою він покинув Єгипет, не злякавшися гніву царевого, бо він був непохитний, як той, хто Невидимого бачить.
\end{tcolorbox}
\begin{tcolorbox}
\textsubscript{28} Вірою справив він Пасху й покроплення крови, щоб їх не торкнувся той, хто погубив первороджених.
\end{tcolorbox}
\begin{tcolorbox}
\textsubscript{29} Вірою вони перейшли Червоне море, немов суходолом, на що спокусившись єгиптяни, потопились.
\end{tcolorbox}
\begin{tcolorbox}
\textsubscript{30} Вірою впали єрихонські мури по семиденнім обходженні їх.
\end{tcolorbox}
\begin{tcolorbox}
\textsubscript{31} Вірою блудниця Рахав не згинула з невірними, коли з миром прийняла вивідувачів.
\end{tcolorbox}
\begin{tcolorbox}
\textsubscript{32} І що ще скажу? Бо не стане часу мені, щоб оповідати про Гедеона, Варака, Самсона, Ефтая, Давида й Самуїла та про пророків,
\end{tcolorbox}
\begin{tcolorbox}
\textsubscript{33} що вірою царства побивали, правду чинили, одержували обітниці, пащі левам загороджували,
\end{tcolorbox}
\begin{tcolorbox}
\textsubscript{34} силу огненну гасили, утікали від вістря меча, зміцнялись від слабости, хоробрі були на війні, обертали в розтіч полки чужоземців;
\end{tcolorbox}
\begin{tcolorbox}
\textsubscript{35} жінки діставали померлих своїх із воскресіння; а інші бували скатовані, не прийнявши визволення, щоб отримати краще воскресіння;
\end{tcolorbox}
\begin{tcolorbox}
\textsubscript{36} а інші дізнали наруги та рани, а також кайдани й в'язниці.
\end{tcolorbox}
\begin{tcolorbox}
\textsubscript{37} Камінням побиті бували, допитувані, перепилювані, умирали, зарубані мечем, тинялися в овечих та козячих шкурах, збідовані, засумовані, витерпілі.
\end{tcolorbox}
\begin{tcolorbox}
\textsubscript{38} Ті, що світ не вартий був їх, тинялися по пустинях та горах, і по печерах та проваллях земних.
\end{tcolorbox}
\begin{tcolorbox}
\textsubscript{39} І всі вони, одержавши засвідчення вірою, обітниці не прийняли,
\end{tcolorbox}
\begin{tcolorbox}
\textsubscript{40} бо Бог передбачив щось краще про нас, щоб вони не без нас досконалість одержали.
\end{tcolorbox}
\subsection{CHAPTER 12}
\begin{tcolorbox}
\textsubscript{1} Тож і ми, мавши навколо себе велику таку хмару свідків, скиньмо всякий тягар та гріх, що обплутує нас, та й біжім з терпеливістю до боротьби, яка перед нами,
\end{tcolorbox}
\begin{tcolorbox}
\textsubscript{2} дивлячись на Ісуса, на Начальника й Виконавця віри, що замість радости, яка була перед Ним, перетерпів хреста, не звертавши уваги на сором, і сів по правиці престолу Божого.
\end{tcolorbox}
\begin{tcolorbox}
\textsubscript{3} Тож подумайте про Того, хто перетерпів такий перекір проти Себе від грішних, щоб ви не знемоглись, і не впали на душах своїх.
\end{tcolorbox}
\begin{tcolorbox}
\textsubscript{4} Ви ще не змагались до крови, борючись проти гріха,
\end{tcolorbox}
\begin{tcolorbox}
\textsubscript{5} і забули нагад, що говорить до вас, як синів: Мій сину, не нехтуй Господньої кари, і не знемагай, коли Він докоряє тобі.
\end{tcolorbox}
\begin{tcolorbox}
\textsubscript{6} Бо Господь, кого любить, того Він карає, і б'є кожного сина, якого приймає!
\end{tcolorbox}
\begin{tcolorbox}
\textsubscript{7} Коли терпите кару, то робить Бог вам, як синам. Хіба є такий син, що батько його не карає?
\end{tcolorbox}
\begin{tcolorbox}
\textsubscript{8} А коли ви без кари, що спільна для всіх, то ви діти з перелюбу, а не сини.
\end{tcolorbox}
\begin{tcolorbox}
\textsubscript{9} А до того, ми мали батьків, що карали наше тіло, і боялися їх, то чи ж не далеко більше повинні коритися ми Отцеві духів, щоб жити?
\end{tcolorbox}
\begin{tcolorbox}
\textsubscript{10} Ті нас за короткого часу карали, як їм до вподоби було, Цей же на користь, щоб ми стали учасниками Його святости.
\end{tcolorbox}
\begin{tcolorbox}
\textsubscript{11} Усяка кара в теперішній час не здається потіхою, але смутком, та згодом для навчених нею приносить мирний плід праведности!
\end{tcolorbox}
\begin{tcolorbox}
\textsubscript{12} Тому то опущені руки й коліна знеможені випростуйте,
\end{tcolorbox}
\begin{tcolorbox}
\textsubscript{13} і чиніть прості стежки ногам вашим, щоб кульгаве не збочило, але краще виправилось.
\end{tcolorbox}
\begin{tcolorbox}
\textsubscript{14} Пильнуйте про мир зо всіма, і про святість, без якої ніхто не побачить Господа.
\end{tcolorbox}
\begin{tcolorbox}
\textsubscript{15} Дивіться, щоб хто не зостався без Божої благодаті, щоб не виріс який гіркий корінь і не наробив непокою, і щоб багато-хто не опоганились тим.
\end{tcolorbox}
\begin{tcolorbox}
\textsubscript{16} Щоб не був хто блудник чи безбожник, немов той Ісав, що своє перворідство віддав за поживу саму.
\end{tcolorbox}
\begin{tcolorbox}
\textsubscript{17} Бо знаєте ви, що й після, як схотів він успадкувати благословення, відкинутий був, не знайшов бо був можливости до покаяння, хоч його із слізьми шукав.
\end{tcolorbox}
\begin{tcolorbox}
\textsubscript{18} Бо ви не приступили до гори дотикальної та до палючого огню, і до хмари, і до темряви, та до бурі,
\end{tcolorbox}
\begin{tcolorbox}
\textsubscript{19} і до сурмового звуку, і до голосу слів, що його ті, хто чув, просили, щоб більше не мовилось слово до них.
\end{tcolorbox}
\begin{tcolorbox}
\textsubscript{20} Не могли бо вони того витримати, що наказано: Коли й звірина до гори доторкнеться, то буде камінням побита.
\end{tcolorbox}
\begin{tcolorbox}
\textsubscript{21} І таке страшне те видіння було, що Мойсей проказав: Я боюся й тремчу!...
\end{tcolorbox}
\begin{tcolorbox}
\textsubscript{22} Але ви приступили до гори Сіонської, і до міста Бога Живого, до Єрусалиму небесного, і до десятків тисяч Анголів,
\end{tcolorbox}
\begin{tcolorbox}
\textsubscript{23} і до Церкви первороджених, на небі написаних, і до Судді всіх до Бога, і до духів удосконалених праведників,
\end{tcolorbox}
\begin{tcolorbox}
\textsubscript{24} і до Посередника Нового Заповіту до Ісуса, і до покроплення крови, що краще промовляє, як Авелева.
\end{tcolorbox}
\begin{tcolorbox}
\textsubscript{25} Глядіть, не відвертайтеся від того, хто промовляє. Бо як не повтікали вони, що зреклися того, хто звіщав на землі, то тим більше ми, якщо зрікаємся Того, Хто з неба звіщає,
\end{tcolorbox}
\begin{tcolorbox}
\textsubscript{26} що голос Його захитав тоді землю, а тепер обіцяв та каже: Ще раз захитаю не тільки землею, але й небом.
\end{tcolorbox}
\begin{tcolorbox}
\textsubscript{27} А ще раз визначає заміну захитаного, як створеного, щоб зосталися ті, хто непохитний.
\end{tcolorbox}
\begin{tcolorbox}
\textsubscript{28} Отож ми, що приймаємо царство непохитне, нехай маємо благодать, що нею приємно служитимемо Богові з побожністю й зо страхом.
\end{tcolorbox}
\begin{tcolorbox}
\textsubscript{29} Бо наш Бог то палючий огонь!
\end{tcolorbox}
\subsection{CHAPTER 13}
\begin{tcolorbox}
\textsubscript{1} Братолюбство нехай пробуває між вами!
\end{tcolorbox}
\begin{tcolorbox}
\textsubscript{2} Не забувайте любови до приходнів, бо деякі нею, навіть не відаючи, гостинно були прийняли Анголів.
\end{tcolorbox}
\begin{tcolorbox}
\textsubscript{3} Пам'ятайте про в'язнів, немов із ними були б ви пов'язані, про тих, хто страждає, як такі, що й самі ви знаходитесь в тілі.
\end{tcolorbox}
\begin{tcolorbox}
\textsubscript{4} Нехай буде в усіх чесний шлюб та ложе непорочне, а блудників та перелюбів судитиме Бог.
\end{tcolorbox}
\begin{tcolorbox}
\textsubscript{5} Будьте життям не грошолюбні, задовольняйтеся тим, що маєте. Сам бо сказав: Я тебе не покину, ані не відступлюся від тебе!
\end{tcolorbox}
\begin{tcolorbox}
\textsubscript{6} Тому то ми сміливо говоримо: Господь мені помічник, і я не злякаюсь нікого: що зробить людина мені?
\end{tcolorbox}
\begin{tcolorbox}
\textsubscript{7} Спогадуйте наставників ваших, що вам говорили Слово Боже; і, дивлячися на кінець їхнього життя, переймайте їхню віру.
\end{tcolorbox}
\begin{tcolorbox}
\textsubscript{8} Ісус Христос учора, і сьогодні, і навіки Той Самий!
\end{tcolorbox}
\begin{tcolorbox}
\textsubscript{9} Не захоплюйтеся всілякими та чужими науками. Бо річ добра зміцняти серця благодаттю, а не стравами, що користи від них не одержали ті, хто за ними ходив.
\end{tcolorbox}
\begin{tcolorbox}
\textsubscript{10} Маємо жертівника, що від нього годуватися права не мають ті, хто скинії служить,
\end{tcolorbox}
\begin{tcolorbox}
\textsubscript{11} бо котрих звірят кров первосвященик уносить до святині за гріхи, тих м'ясо палиться поза табором,
\end{tcolorbox}
\begin{tcolorbox}
\textsubscript{12} тому то Ісус, щоб кров'ю Своєю людей освятити, постраждав поза брамою.
\end{tcolorbox}
\begin{tcolorbox}
\textsubscript{13} Тож виходьмо до Нього поза табір, і наругу Його понесімо,
\end{tcolorbox}
\begin{tcolorbox}
\textsubscript{14} бо постійного міста не маємо тут, а шукаємо майбутнього!
\end{tcolorbox}
\begin{tcolorbox}
\textsubscript{15} Отож, завжди приносьмо Богові жертву хвали, цебто плід уст, що Ім'я Його славлять.
\end{tcolorbox}
\begin{tcolorbox}
\textsubscript{16} Не забувайте ж і про доброчинність та спільність, бо жертви такі вгодні Богові.
\end{tcolorbox}
\begin{tcolorbox}
\textsubscript{17} Слухайтесь ваших наставників та коріться їм, вони бо пильнують душ ваших, як ті, хто має здати справу. Нехай вони роблять це з радістю, а не зідхаючи, бо це для вас не корисне.
\end{tcolorbox}
\begin{tcolorbox}
\textsubscript{18} Моліться за нас, бо надіємося, що ми маємо добре сумління, бо хочемо добре в усьому поводитись.
\end{tcolorbox}
\begin{tcolorbox}
\textsubscript{19} А надто прошу це робити, щоб швидше до вас мене вернено.
\end{tcolorbox}
\begin{tcolorbox}
\textsubscript{20} Бог же миру, що з мертвих підняв великого Пастиря вівцям кров'ю вічного заповіту, Господа нашого Ісуса,
\end{tcolorbox}
\begin{tcolorbox}
\textsubscript{21} нехай вас удосконалить у кожному доброму ділі, щоб волю чинити Його, чинячи в вас любе перед лицем Його через Ісуса Христа, Якому слава на віки вічні. Амінь.
\end{tcolorbox}
\begin{tcolorbox}
\textsubscript{22} Благаю ж вас, браття, прийміть слово потіхи, бо коротко я написав вам.
\end{tcolorbox}
\begin{tcolorbox}
\textsubscript{23} Знайте, що наш брат Тимофій вже випущений, і я з ним; коли незабаром він прийде, я вас побачу.
\end{tcolorbox}
\begin{tcolorbox}
\textsubscript{24} Вітайте всіх ваших наставників та всіх святих. Вітають вас ті, хто в Італії.
\end{tcolorbox}
\begin{tcolorbox}
\textsubscript{25} Благодать зо всіма вами! Амінь.
\end{tcolorbox}
