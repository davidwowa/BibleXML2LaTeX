\section{BOOK 144}
\subsection{CHAPTER 1}
\begin{tcolorbox}
\textsubscript{1} Павел, волею Божиею Апостол Иисуса Христа, находящимся в Ефесе святым и верным во Христе Иисусе:
\end{tcolorbox}
\begin{tcolorbox}
\textsubscript{2} благодать вам и мир от Бога Отца нашего и Господа Иисуса Христа.
\end{tcolorbox}
\begin{tcolorbox}
\textsubscript{3} Благословен Бог и Отец Господа нашего Иисуса Христа, благословивший нас во Христе всяким духовным благословением в небесах,
\end{tcolorbox}
\begin{tcolorbox}
\textsubscript{4} так как Он избрал нас в Нем прежде создания мира, чтобы мы были святы и непорочны пред Ним в любви,
\end{tcolorbox}
\begin{tcolorbox}
\textsubscript{5} предопределив усыновить нас Себе чрез Иисуса Христа, по благоволению воли Своей,
\end{tcolorbox}
\begin{tcolorbox}
\textsubscript{6} в похвалу славы благодати Своей, которою Он облагодатствовал нас в Возлюбленном,
\end{tcolorbox}
\begin{tcolorbox}
\textsubscript{7} в Котором мы имеем искупление Кровию Его, прощение грехов, по богатству благодати Его,
\end{tcolorbox}
\begin{tcolorbox}
\textsubscript{8} каковую Он в преизбытке даровал нам во всякой премудрости и разумении,
\end{tcolorbox}
\begin{tcolorbox}
\textsubscript{9} открыв нам тайну Своей воли по Своему благоволению, которое Он прежде положил в Нем,
\end{tcolorbox}
\begin{tcolorbox}
\textsubscript{10} в устроении полноты времен, дабы все небесное и земное соединить под главою Христом.
\end{tcolorbox}
\begin{tcolorbox}
\textsubscript{11} В Нем мы и сделались наследниками, быв предназначены [к тому] по определению Совершающего все по изволению воли Своей,
\end{tcolorbox}
\begin{tcolorbox}
\textsubscript{12} дабы послужить к похвале славы Его нам, которые ранее уповали на Христа.
\end{tcolorbox}
\begin{tcolorbox}
\textsubscript{13} В Нем и вы, услышав слово истины, благовествование вашего спасения, и уверовав в Него, запечатлены обетованным Святым Духом,
\end{tcolorbox}
\begin{tcolorbox}
\textsubscript{14} Который есть залог наследия нашего, для искупления удела [Его], в похвалу славы Его.
\end{tcolorbox}
\begin{tcolorbox}
\textsubscript{15} Посему и я, услышав о вашей вере во Христа Иисуса и о любви ко всем святым,
\end{tcolorbox}
\begin{tcolorbox}
\textsubscript{16} непрестанно благодарю за вас [Бога], вспоминая о вас в молитвах моих,
\end{tcolorbox}
\begin{tcolorbox}
\textsubscript{17} чтобы Бог Господа нашего Иисуса Христа, Отец славы, дал вам Духа премудрости и откровения к познанию Его,
\end{tcolorbox}
\begin{tcolorbox}
\textsubscript{18} и просветил очи сердца вашего, дабы вы познали, в чем состоит надежда призвания Его, и какое богатство славного наследия Его для святых,
\end{tcolorbox}
\begin{tcolorbox}
\textsubscript{19} и как безмерно величие могущества Его в нас, верующих по действию державной силы Его,
\end{tcolorbox}
\begin{tcolorbox}
\textsubscript{20} которою Он воздействовал во Христе, воскресив Его из мертвых и посадив одесную Себя на небесах,
\end{tcolorbox}
\begin{tcolorbox}
\textsubscript{21} превыше всякого Начальства, и Власти, и Силы, и Господства, и всякого имени, именуемого не только в сем веке, но и в будущем,
\end{tcolorbox}
\begin{tcolorbox}
\textsubscript{22} и все покорил под ноги Его, и поставил Его выше всего, главою Церкви,
\end{tcolorbox}
\begin{tcolorbox}
\textsubscript{23} которая есть Тело Его, полнота Наполняющего все во всем.
\end{tcolorbox}
\subsection{CHAPTER 2}
\begin{tcolorbox}
\textsubscript{1} И вас, мертвых по преступлениям и грехам вашим,
\end{tcolorbox}
\begin{tcolorbox}
\textsubscript{2} в которых вы некогда жили, по обычаю мира сего, по [воле] князя, господствующего в воздухе, духа, действующего ныне в сынах противления,
\end{tcolorbox}
\begin{tcolorbox}
\textsubscript{3} между которыми и мы все жили некогда по нашим плотским похотям, исполняя желания плоти и помыслов, и были по природе чадами гнева, как и прочие,
\end{tcolorbox}
\begin{tcolorbox}
\textsubscript{4} Бог, богатый милостью, по Своей великой любви, которою возлюбил нас,
\end{tcolorbox}
\begin{tcolorbox}
\textsubscript{5} и нас, мертвых по преступлениям, оживотворил со Христом, --благодатью вы спасены, --
\end{tcolorbox}
\begin{tcolorbox}
\textsubscript{6} и воскресил с Ним, и посадил на небесах во Христе Иисусе,
\end{tcolorbox}
\begin{tcolorbox}
\textsubscript{7} дабы явить в грядущих веках преизобильное богатство благодати Своей в благости к нам во Христе Иисусе.
\end{tcolorbox}
\begin{tcolorbox}
\textsubscript{8} Ибо благодатью вы спасены через веру, и сие не от вас, Божий дар:
\end{tcolorbox}
\begin{tcolorbox}
\textsubscript{9} не от дел, чтобы никто не хвалился.
\end{tcolorbox}
\begin{tcolorbox}
\textsubscript{10} Ибо мы--Его творение, созданы во Христе Иисусе на добрые дела, которые Бог предназначил нам исполнять.
\end{tcolorbox}
\begin{tcolorbox}
\textsubscript{11} Итак помните, что вы, некогда язычники по плоти, которых называли необрезанными так называемые обрезанные плотским [обрезанием], совершаемым руками,
\end{tcolorbox}
\begin{tcolorbox}
\textsubscript{12} что вы были в то время без Христа, отчуждены от общества Израильского, чужды заветов обетования, не имели надежды и были безбожники в мире.
\end{tcolorbox}
\begin{tcolorbox}
\textsubscript{13} А теперь во Христе Иисусе вы, бывшие некогда далеко, стали близки Кровию Христовою.
\end{tcolorbox}
\begin{tcolorbox}
\textsubscript{14} Ибо Он есть мир наш, соделавший из обоих одно и разрушивший стоявшую посреди преграду,
\end{tcolorbox}
\begin{tcolorbox}
\textsubscript{15} упразднив вражду Плотию Своею, а закон заповедей учением, дабы из двух создать в Себе Самом одного нового человека, устрояя мир,
\end{tcolorbox}
\begin{tcolorbox}
\textsubscript{16} и в одном теле примирить обоих с Богом посредством креста, убив вражду на нем.
\end{tcolorbox}
\begin{tcolorbox}
\textsubscript{17} И, придя, благовествовал мир вам, дальним и близким,
\end{tcolorbox}
\begin{tcolorbox}
\textsubscript{18} потому что через Него и те и другие имеем доступ к Отцу, в одном Духе.
\end{tcolorbox}
\begin{tcolorbox}
\textsubscript{19} Итак вы уже не чужие и не пришельцы, но сограждане святым и свои Богу,
\end{tcolorbox}
\begin{tcolorbox}
\textsubscript{20} быв утверждены на основании Апостолов и пророков, имея Самого Иисуса Христа краеугольным [камнем],
\end{tcolorbox}
\begin{tcolorbox}
\textsubscript{21} на котором все здание, слагаясь стройно, возрастает в святый храм в Господе,
\end{tcolorbox}
\begin{tcolorbox}
\textsubscript{22} на котором и вы устрояетесь в жилище Божие Духом.
\end{tcolorbox}
\subsection{CHAPTER 3}
\begin{tcolorbox}
\textsubscript{1} Для сего-то я, Павел, [сделался] узником Иисуса Христа за вас язычников.
\end{tcolorbox}
\begin{tcolorbox}
\textsubscript{2} Как вы слышали о домостроительстве благодати Божией, данной мне для вас,
\end{tcolorbox}
\begin{tcolorbox}
\textsubscript{3} потому что мне через откровение возвещена тайна (о чем я и выше писал кратко),
\end{tcolorbox}
\begin{tcolorbox}
\textsubscript{4} то вы, читая, можете усмотреть мое разумение тайны Христовой,
\end{tcolorbox}
\begin{tcolorbox}
\textsubscript{5} которая не была возвещена прежним поколениям сынов человеческих, как ныне открыта святым Апостолам Его и пророкам Духом Святым,
\end{tcolorbox}
\begin{tcolorbox}
\textsubscript{6} чтобы и язычникам быть сонаследниками, составляющими одно тело, и сопричастниками обетования Его во Христе Иисусе посредством благовествования,
\end{tcolorbox}
\begin{tcolorbox}
\textsubscript{7} которого служителем сделался я по дару благодати Божией, данной мне действием силы Его.
\end{tcolorbox}
\begin{tcolorbox}
\textsubscript{8} Мне, наименьшему из всех святых, дана благодать сия--благовествовать язычникам неисследимое богатство Христово
\end{tcolorbox}
\begin{tcolorbox}
\textsubscript{9} и открыть всем, в чем состоит домостроительство тайны, сокрывавшейся от вечности в Боге, создавшем все Иисусом Христом,
\end{tcolorbox}
\begin{tcolorbox}
\textsubscript{10} дабы ныне соделалась известною через Церковь начальствам и властям на небесах многоразличная премудрость Божия,
\end{tcolorbox}
\begin{tcolorbox}
\textsubscript{11} по предвечному определению, которое Он исполнил во Христе Иисусе, Господе нашем,
\end{tcolorbox}
\begin{tcolorbox}
\textsubscript{12} в Котором мы имеем дерзновение и надежный доступ через веру в Него.
\end{tcolorbox}
\begin{tcolorbox}
\textsubscript{13} Посему прошу вас не унывать при моих ради вас скорбях, которые суть ваша слава.
\end{tcolorbox}
\begin{tcolorbox}
\textsubscript{14} Для сего преклоняю колени мои пред Отцем Господа нашего Иисуса Христа,
\end{tcolorbox}
\begin{tcolorbox}
\textsubscript{15} от Которого именуется всякое отечество на небесах и на земле,
\end{tcolorbox}
\begin{tcolorbox}
\textsubscript{16} да даст вам, по богатству славы Своей, крепко утвердиться Духом Его во внутреннем человеке,
\end{tcolorbox}
\begin{tcolorbox}
\textsubscript{17} верою вселиться Христу в сердца ваши,
\end{tcolorbox}
\begin{tcolorbox}
\textsubscript{18} чтобы вы, укорененные и утвержденные в любви, могли постигнуть со всеми святыми, что широта и долгота, и глубина и высота,
\end{tcolorbox}
\begin{tcolorbox}
\textsubscript{19} и уразуметь превосходящую разумение любовь Христову, дабы вам исполниться всею полнотою Божиею.
\end{tcolorbox}
\begin{tcolorbox}
\textsubscript{20} А Тому, Кто действующею в нас силою может сделать несравненно больше всего, чего мы просим, или о чем помышляем,
\end{tcolorbox}
\begin{tcolorbox}
\textsubscript{21} Тому слава в Церкви во Христе Иисусе во все роды, от века до века. Аминь.
\end{tcolorbox}
\subsection{CHAPTER 4}
\begin{tcolorbox}
\textsubscript{1} Итак я, узник в Господе, умоляю вас поступать достойно звания, в которое вы призваны,
\end{tcolorbox}
\begin{tcolorbox}
\textsubscript{2} со всяким смиренномудрием и кротостью и долготерпением, снисходя друг ко другу любовью,
\end{tcolorbox}
\begin{tcolorbox}
\textsubscript{3} стараясь сохранять единство духа в союзе мира.
\end{tcolorbox}
\begin{tcolorbox}
\textsubscript{4} Одно тело и один дух, как вы и призваны к одной надежде вашего звания;
\end{tcolorbox}
\begin{tcolorbox}
\textsubscript{5} один Господь, одна вера, одно крещение,
\end{tcolorbox}
\begin{tcolorbox}
\textsubscript{6} один Бог и Отец всех, Который над всеми, и через всех, и во всех нас.
\end{tcolorbox}
\begin{tcolorbox}
\textsubscript{7} Каждому же из нас дана благодать по мере дара Христова.
\end{tcolorbox}
\begin{tcolorbox}
\textsubscript{8} Посему и сказано: восшед на высоту, пленил плен и дал дары человекам.
\end{tcolorbox}
\begin{tcolorbox}
\textsubscript{9} А 'восшел' что означает, как не то, что Он и нисходил прежде в преисподние места земли?
\end{tcolorbox}
\begin{tcolorbox}
\textsubscript{10} Нисшедший, Он же есть и восшедший превыше всех небес, дабы наполнить все.
\end{tcolorbox}
\begin{tcolorbox}
\textsubscript{11} И Он поставил одних Апостолами, других пророками, иных Евангелистами, иных пастырями и учителями,
\end{tcolorbox}
\begin{tcolorbox}
\textsubscript{12} к совершению святых, на дело служения, для созидания Тела Христова,
\end{tcolorbox}
\begin{tcolorbox}
\textsubscript{13} доколе все придем в единство веры и познания Сына Божия, в мужа совершенного, в меру полного возраста Христова;
\end{tcolorbox}
\begin{tcolorbox}
\textsubscript{14} дабы мы не были более младенцами, колеблющимися и увлекающимися всяким ветром учения, по лукавству человеков, по хитрому искусству обольщения,
\end{tcolorbox}
\begin{tcolorbox}
\textsubscript{15} но истинною любовью все возращали в Того, Который есть глава Христос,
\end{tcolorbox}
\begin{tcolorbox}
\textsubscript{16} из Которого все тело, составляемое и совокупляемое посредством всяких взаимно скрепляющих связей, при действии в свою меру каждого члена, получает приращение для созидания самого себя в любви.
\end{tcolorbox}
\begin{tcolorbox}
\textsubscript{17} Посему я говорю и заклинаю Господом, чтобы вы более не поступали, как поступают прочие народы, по суетности ума своего,
\end{tcolorbox}
\begin{tcolorbox}
\textsubscript{18} будучи помрачены в разуме, отчуждены от жизни Божией, по причине их невежества и ожесточения сердца их.
\end{tcolorbox}
\begin{tcolorbox}
\textsubscript{19} Они, дойдя до бесчувствия, предались распутству так, что делают всякую нечистоту с ненасытимостью.
\end{tcolorbox}
\begin{tcolorbox}
\textsubscript{20} Но вы не так познали Христа;
\end{tcolorbox}
\begin{tcolorbox}
\textsubscript{21} потому что вы слышали о Нем и в Нем научились, --так как истина во Иисусе, --
\end{tcolorbox}
\begin{tcolorbox}
\textsubscript{22} отложить прежний образ жизни ветхого человека, истлевающего в обольстительных похотях,
\end{tcolorbox}
\begin{tcolorbox}
\textsubscript{23} а обновиться духом ума вашего
\end{tcolorbox}
\begin{tcolorbox}
\textsubscript{24} и облечься в нового человека, созданного по Богу, в праведности и святости истины.
\end{tcolorbox}
\begin{tcolorbox}
\textsubscript{25} Посему, отвергнув ложь, говорите истину каждый ближнему своему, потому что мы члены друг другу.
\end{tcolorbox}
\begin{tcolorbox}
\textsubscript{26} Гневаясь, не согрешайте: солнце да не зайдет во гневе вашем;
\end{tcolorbox}
\begin{tcolorbox}
\textsubscript{27} и не давайте места диаволу.
\end{tcolorbox}
\begin{tcolorbox}
\textsubscript{28} Кто крал, вперед не кради, а лучше трудись, делая своими руками полезное, чтобы было из чего уделять нуждающемуся.
\end{tcolorbox}
\begin{tcolorbox}
\textsubscript{29} Никакое гнилое слово да не исходит из уст ваших, а только доброе для назидания в вере, дабы оно доставляло благодать слушающим.
\end{tcolorbox}
\begin{tcolorbox}
\textsubscript{30} И не оскорбляйте Святаго Духа Божия, Которым вы запечатлены в день искупления.
\end{tcolorbox}
\begin{tcolorbox}
\textsubscript{31} Всякое раздражение и ярость, и гнев, и крик, и злоречие со всякою злобою да будут удалены от вас;
\end{tcolorbox}
\begin{tcolorbox}
\textsubscript{32} но будьте друг ко другу добры, сострадательны, прощайте друг друга, как и Бог во Христе простил вас.
\end{tcolorbox}
\subsection{CHAPTER 5}
\begin{tcolorbox}
\textsubscript{1} Итак, подражайте Богу, как чада возлюбленные,
\end{tcolorbox}
\begin{tcolorbox}
\textsubscript{2} и живите в любви, как и Христос возлюбил нас и предал Себя за нас в приношение и жертву Богу, в благоухание приятное.
\end{tcolorbox}
\begin{tcolorbox}
\textsubscript{3} А блуд и всякая нечистота и любостяжание не должны даже именоваться у вас, как прилично святым.
\end{tcolorbox}
\begin{tcolorbox}
\textsubscript{4} Также сквернословие и пустословие и смехотворство не приличны [вам], а, напротив, благодарение;
\end{tcolorbox}
\begin{tcolorbox}
\textsubscript{5} ибо знайте, что никакой блудник, или нечистый, или любостяжатель, который есть идолослужитель, не имеет наследия в Царстве Христа и Бога.
\end{tcolorbox}
\begin{tcolorbox}
\textsubscript{6} Никто да не обольщает вас пустыми словами, ибо за это приходит гнев Божий на сынов противления;
\end{tcolorbox}
\begin{tcolorbox}
\textsubscript{7} итак, не будьте сообщниками их.
\end{tcolorbox}
\begin{tcolorbox}
\textsubscript{8} Вы были некогда тьма, а теперь--свет в Господе: поступайте, как чада света,
\end{tcolorbox}
\begin{tcolorbox}
\textsubscript{9} потому что плод Духа состоит во всякой благости, праведности и истине.
\end{tcolorbox}
\begin{tcolorbox}
\textsubscript{10} Испытывайте, что благоугодно Богу,
\end{tcolorbox}
\begin{tcolorbox}
\textsubscript{11} и не участвуйте в бесплодных делах тьмы, но и обличайте.
\end{tcolorbox}
\begin{tcolorbox}
\textsubscript{12} Ибо о том, что они делают тайно, стыдно и говорить.
\end{tcolorbox}
\begin{tcolorbox}
\textsubscript{13} Все же обнаруживаемое делается явным от света, ибо все, делающееся явным, свет есть.
\end{tcolorbox}
\begin{tcolorbox}
\textsubscript{14} Посему сказано: 'встань, спящий, и воскресни из мертвых, и осветит тебя Христос'.
\end{tcolorbox}
\begin{tcolorbox}
\textsubscript{15} Итак, смотрите, поступайте осторожно, не как неразумные, но как мудрые,
\end{tcolorbox}
\begin{tcolorbox}
\textsubscript{16} дорожа временем, потому что дни лукавы.
\end{tcolorbox}
\begin{tcolorbox}
\textsubscript{17} Итак, не будьте нерассудительны, но познавайте, что есть воля Божия.
\end{tcolorbox}
\begin{tcolorbox}
\textsubscript{18} И не упивайтесь вином, от которого бывает распутство; но исполняйтесь Духом,
\end{tcolorbox}
\begin{tcolorbox}
\textsubscript{19} назидая самих себя псалмами и славословиями и песнопениями духовными, поя и воспевая в сердцах ваших Господу,
\end{tcolorbox}
\begin{tcolorbox}
\textsubscript{20} благодаря всегда за все Бога и Отца, во имя Господа нашего Иисуса Христа,
\end{tcolorbox}
\begin{tcolorbox}
\textsubscript{21} повинуясь друг другу в страхе Божием.
\end{tcolorbox}
\begin{tcolorbox}
\textsubscript{22} Жены, повинуйтесь своим мужьям, как Господу,
\end{tcolorbox}
\begin{tcolorbox}
\textsubscript{23} потому что муж есть глава жены, как и Христос глава Церкви, и Он же Спаситель тела.
\end{tcolorbox}
\begin{tcolorbox}
\textsubscript{24} Но как Церковь повинуется Христу, так и жены своим мужьям во всем.
\end{tcolorbox}
\begin{tcolorbox}
\textsubscript{25} Мужья, любите своих жен, как и Христос возлюбил Церковь и предал Себя за нее,
\end{tcolorbox}
\begin{tcolorbox}
\textsubscript{26} чтобы освятить ее, очистив банею водною посредством слова;
\end{tcolorbox}
\begin{tcolorbox}
\textsubscript{27} чтобы представить ее Себе славною Церковью, не имеющею пятна, или порока, или чего-либо подобного, но дабы она была свята и непорочна.
\end{tcolorbox}
\begin{tcolorbox}
\textsubscript{28} Так должны мужья любить своих жен, как свои тела: любящий свою жену любит самого себя.
\end{tcolorbox}
\begin{tcolorbox}
\textsubscript{29} Ибо никто никогда не имел ненависти к своей плоти, но питает и греет ее, как и Господь Церковь,
\end{tcolorbox}
\begin{tcolorbox}
\textsubscript{30} потому что мы члены тела Его, от плоти Его и от костей Его.
\end{tcolorbox}
\begin{tcolorbox}
\textsubscript{31} Посему оставит человек отца своего и мать и прилепится к жене своей, и будут двое одна плоть.
\end{tcolorbox}
\begin{tcolorbox}
\textsubscript{32} Тайна сия велика; я говорю по отношению ко Христу и к Церкви.
\end{tcolorbox}
\begin{tcolorbox}
\textsubscript{33} Так каждый из вас да любит свою жену, как самого себя; а жена да боится своего мужа.
\end{tcolorbox}
\subsection{CHAPTER 6}
\begin{tcolorbox}
\textsubscript{1} Дети, повинуйтесь своим родителям в Господе, ибо сего [требует] справедливость.
\end{tcolorbox}
\begin{tcolorbox}
\textsubscript{2} Почитай отца твоего и мать, это первая заповедь с обетованием:
\end{tcolorbox}
\begin{tcolorbox}
\textsubscript{3} да будет тебе благо, и будешь долголетен на земле.
\end{tcolorbox}
\begin{tcolorbox}
\textsubscript{4} И вы, отцы, не раздражайте детей ваших, но воспитывайте их в учении и наставлении Господнем.
\end{tcolorbox}
\begin{tcolorbox}
\textsubscript{5} Рабы, повинуйтесь господам своим по плоти со страхом и трепетом, в простоте сердца вашего, как Христу,
\end{tcolorbox}
\begin{tcolorbox}
\textsubscript{6} не с видимою только услужливостью, как человекоугодники, но как рабы Христовы, исполняя волю Божию от души,
\end{tcolorbox}
\begin{tcolorbox}
\textsubscript{7} служа с усердием, как Господу, а не как человекам,
\end{tcolorbox}
\begin{tcolorbox}
\textsubscript{8} зная, что каждый получит от Господа по мере добра, которое он сделал, раб ли, или свободный.
\end{tcolorbox}
\begin{tcolorbox}
\textsubscript{9} И вы, господа, поступайте с ними так же, умеряя строгость, зная, что и над вами самими и над ними есть на небесах Господь, у Которого нет лицеприятия.
\end{tcolorbox}
\begin{tcolorbox}
\textsubscript{10} Наконец, братия мои, укрепляйтесь Господом и могуществом силы Его.
\end{tcolorbox}
\begin{tcolorbox}
\textsubscript{11} Облекитесь во всеоружие Божие, чтобы вам можно было стать против козней диавольских,
\end{tcolorbox}
\begin{tcolorbox}
\textsubscript{12} потому что наша брань не против крови и плоти, но против начальств, против властей, против мироправителей тьмы века сего, против духов злобы поднебесной.
\end{tcolorbox}
\begin{tcolorbox}
\textsubscript{13} Для сего приимите всеоружие Божие, дабы вы могли противостать в день злый и, все преодолев, устоять.
\end{tcolorbox}
\begin{tcolorbox}
\textsubscript{14} Итак станьте, препоясав чресла ваши истиною и облекшись в броню праведности,
\end{tcolorbox}
\begin{tcolorbox}
\textsubscript{15} и обув ноги в готовность благовествовать мир;
\end{tcolorbox}
\begin{tcolorbox}
\textsubscript{16} а паче всего возьмите щит веры, которым возможете угасить все раскаленные стрелы лукавого;
\end{tcolorbox}
\begin{tcolorbox}
\textsubscript{17} и шлем спасения возьмите, и меч духовный, который есть Слово Божие.
\end{tcolorbox}
\begin{tcolorbox}
\textsubscript{18} Всякою молитвою и прошением молитесь во всякое время духом, и старайтесь о сем самом со всяким постоянством и молением о всех святых
\end{tcolorbox}
\begin{tcolorbox}
\textsubscript{19} и о мне, дабы мне дано было слово--устами моими открыто с дерзновением возвещать тайну благовествования,
\end{tcolorbox}
\begin{tcolorbox}
\textsubscript{20} для которого я исполняю посольство в узах, дабы я смело проповедывал, как мне должно.
\end{tcolorbox}
\begin{tcolorbox}
\textsubscript{21} А дабы и вы знали о моих обстоятельствах и делах, обо всем известит вас Тихик, возлюбленный брат и верный в Господе служитель,
\end{tcolorbox}
\begin{tcolorbox}
\textsubscript{22} которого я и послал к вам для того самого, чтобы вы узнали о нас и чтобы он утешил сердца ваши.
\end{tcolorbox}
\begin{tcolorbox}
\textsubscript{23} Мир братиям и любовь с верою от Бога Отца и Господа Иисуса Христа.
\end{tcolorbox}
\begin{tcolorbox}
\textsubscript{24} Благодать со всеми, неизменно любящими Господа нашего Иисуса Христа. Аминь.
\end{tcolorbox}
