\section{BOOK 72}
\subsection{CHAPTER 1}
\begin{tcolorbox}
\textsubscript{1} Как одиноко сидит город, некогда многолюдный! он стал, как вдова; великий между народами, князь над областями сделался данником.
\end{tcolorbox}
\begin{tcolorbox}
\textsubscript{2} Горько плачет он ночью, и слезы его на ланитах его. Нет у него утешителя из всех, любивших его; все друзья его изменили ему, сделались врагами ему.
\end{tcolorbox}
\begin{tcolorbox}
\textsubscript{3} Иуда переселился по причине бедствия и тяжкого рабства, поселился среди язычников, и не нашел покоя; все, преследовавшие его, настигли его в тесных местах.
\end{tcolorbox}
\begin{tcolorbox}
\textsubscript{4} Пути Сиона сетуют, потому что нет идущих на праздник; все ворота его опустели; священники его вздыхают, девицы его печальны, горько и ему самому.
\end{tcolorbox}
\begin{tcolorbox}
\textsubscript{5} Враги его стали во главе, неприятели его благоденствуют, потому что Господь наслал на него горе за множество беззаконий его; дети его пошли в плен впереди врага.
\end{tcolorbox}
\begin{tcolorbox}
\textsubscript{6} И отошло от дщери Сиона все ее великолепие; князья ее--как олени, не находящие пажити; обессиленные они пошли вперед погонщика.
\end{tcolorbox}
\begin{tcolorbox}
\textsubscript{7} Вспомнил Иерусалим, во дни бедствия своего и страданий своих, о всех драгоценностях своих, какие были у него в прежние дни, тогда как народ его пал от руки врага, и никто не помогает ему; неприятели смотрят на него и смеются над его субботами.
\end{tcolorbox}
\begin{tcolorbox}
\textsubscript{8} Тяжко согрешил Иерусалим, за то и сделался отвратительным; все, прославлявшие его, смотрят на него с презрением, потому что увидели наготу его; и сам он вздыхает и отворачивается назад.
\end{tcolorbox}
\begin{tcolorbox}
\textsubscript{9} На подоле у него была нечистота, но он не помышлял о будущности своей, и поэтому необыкновенно унизился, и нет у него утешителя. 'Воззри, Господи, на бедствие мое, ибо враг возвеличился!'
\end{tcolorbox}
\begin{tcolorbox}
\textsubscript{10} Враг простер руку свою на все самое драгоценное его; он видит, как язычники входят во святилище его, о котором Ты заповедал, чтобы они не вступали в собрание Твое.
\end{tcolorbox}
\begin{tcolorbox}
\textsubscript{11} Весь народ его вздыхает, ища хлеба, отдает драгоценности свои за пищу, чтобы подкрепить душу. 'Воззри, Господи, и посмотри, как я унижен!'
\end{tcolorbox}
\begin{tcolorbox}
\textsubscript{12} Да не будет этого с вами, все проходящие путем! взгляните и посмотрите, есть ли болезнь, как моя болезнь, какая постигла меня, какую наслал на меня Господь в день пламенного гнева Своего?
\end{tcolorbox}
\begin{tcolorbox}
\textsubscript{13} Свыше послал Он огонь в кости мои, и он овладел ими; раскинул сеть для ног моих, опрокинул меня, сделал меня бедным и томящимся всякий день.
\end{tcolorbox}
\begin{tcolorbox}
\textsubscript{14} Ярмо беззаконий моих связано в руке Его; они сплетены и поднялись на шею мою; Он ослабил силы мои. Господь отдал меня в руки, из которых не могу подняться.
\end{tcolorbox}
\begin{tcolorbox}
\textsubscript{15} Всех сильных моих Господь низложил среди меня, созвал против меня собрание, чтобы истребить юношей моих; как в точиле, истоптал Господь деву, дочь Иуды.
\end{tcolorbox}
\begin{tcolorbox}
\textsubscript{16} Об этом плачу я; око мое, око мое изливает воды, ибо далеко от меня утешитель, который оживил бы душу мою; дети мои разорены, потому что враг превозмог.
\end{tcolorbox}
\begin{tcolorbox}
\textsubscript{17} Сион простирает руки свои, но утешителя нет ему. Господь дал повеление о Иакове врагам его окружить его; Иерусалим сделался мерзостью среди них.
\end{tcolorbox}
\begin{tcolorbox}
\textsubscript{18} Праведен Господь, ибо я непокорен был слову Его. Послушайте, все народы, и взгляните на болезнь мою: девы мои и юноши мои пошли в плен.
\end{tcolorbox}
\begin{tcolorbox}
\textsubscript{19} Зову друзей моих, но они обманули меня; священники мои и старцы мои издыхают в городе, ища пищи себе, чтобы подкрепить душу свою.
\end{tcolorbox}
\begin{tcolorbox}
\textsubscript{20} Воззри, Господи, ибо мне тесно, волнуется во мне внутренность, сердце мое перевернулось во мне за то, что я упорно противился Тебе; отвне обесчадил меня меч, а дома--как смерть.
\end{tcolorbox}
\begin{tcolorbox}
\textsubscript{21} Услышали, что я стенаю, а утешителя у меня нет; услышали все враги мои о бедствии моем и обрадовались, что Ты соделал это: о, если бы Ты повелел наступить дню, предреченному Тобою, и они стали бы подобными мне!
\end{tcolorbox}
\begin{tcolorbox}
\textsubscript{22} Да предстанет пред лице Твое вся злоба их; и поступи с ними так же, как Ты поступил со мною за все грехи мои, ибо тяжки стоны мои, и сердце мое изнемогает.
\end{tcolorbox}
\subsection{CHAPTER 2}
\begin{tcolorbox}
\textsubscript{1} Как помрачил Господь во гневе Своем дщерь Сиона! с небес поверг на землю красу Израиля и не вспомнил о подножии ног Своих в день гнева Своего.
\end{tcolorbox}
\begin{tcolorbox}
\textsubscript{2} Погубил Господь все жилища Иакова, не пощадил, разрушил в ярости Своей укрепления дщери Иудиной, поверг на землю, отверг царство и князей его, как нечистых:
\end{tcolorbox}
\begin{tcolorbox}
\textsubscript{3} в пылу гнева сломил все роги Израилевы, отвел десницу Свою от неприятеля и воспылал в Иакове, как палящий огонь, пожиравший все вокруг;
\end{tcolorbox}
\begin{tcolorbox}
\textsubscript{4} натянул лук Свой, как неприятель, направил десницу Свою, как враг, и убил все, вожделенное для глаз; на скинию дщери Сиона излил ярость Свою, как огонь.
\end{tcolorbox}
\begin{tcolorbox}
\textsubscript{5} Господь стал как неприятель, истребил Израиля, разорил все чертоги его, разрушил укрепления его и распространил у дщери Иудиной сетование и плач.
\end{tcolorbox}
\begin{tcolorbox}
\textsubscript{6} И отнял ограду Свою, как у сада; разорил Свое место собраний, заставил Господь забыть на Сионе празднества и субботы; и в негодовании гнева Своего отверг царя и священника.
\end{tcolorbox}
\begin{tcolorbox}
\textsubscript{7} Отверг Господь жертвенник Свой, отвратил сердце Свое от святилища Своего, предал в руки врагов стены чертогов его; в доме Господнем они шумели, как в праздничный день.
\end{tcolorbox}
\begin{tcolorbox}
\textsubscript{8} Господь определил разрушить стену дщери Сиона, протянул вервь, не отклонил руки Своей от разорения; истребил внешние укрепления, и стены вместе разрушены.
\end{tcolorbox}
\begin{tcolorbox}
\textsubscript{9} Ворота ее вдались в землю; Он разрушил и сокрушил запоры их; царь ее и князья ее--среди язычников; не стало закона, и пророки ее не сподобляются видений от Господа.
\end{tcolorbox}
\begin{tcolorbox}
\textsubscript{10} Сидят на земле безмолвно старцы дщери Сионовой, посыпали пеплом свои головы, препоясались вретищем; опустили к земле головы свои девы Иерусалимские.
\end{tcolorbox}
\begin{tcolorbox}
\textsubscript{11} Истощились от слез глаза мои, волнуется во мне внутренность моя, изливается на землю печень моя от гибели дщери народа моего, когда дети и грудные младенцы умирают от голода среди городских улиц.
\end{tcolorbox}
\begin{tcolorbox}
\textsubscript{12} Матерям своим говорят они: 'где хлеб и вино?', умирая, подобно раненым, на улицах городских, изливая души свои в лоно матерей своих.
\end{tcolorbox}
\begin{tcolorbox}
\textsubscript{13} Что мне сказать тебе, с чем сравнить тебя, дщерь Иерусалима? чему уподобить тебя, чтобы утешить тебя, дева, дщерь Сиона? ибо рана твоя велика, как море; кто может исцелить тебя?
\end{tcolorbox}
\begin{tcolorbox}
\textsubscript{14} Пророки твои провещали тебе пустое и ложное и не раскрывали твоего беззакония, чтобы отвратить твое пленение, и изрекали тебе откровения ложные и приведшие тебя к изгнанию.
\end{tcolorbox}
\begin{tcolorbox}
\textsubscript{15} Руками всплескивают о тебе все проходящие путем, свищут и качают головою своею о дщери Иерусалима, говоря: 'это ли город, который называли совершенством красоты, радостью всей земли?'
\end{tcolorbox}
\begin{tcolorbox}
\textsubscript{16} Разинули на тебя пасть свою все враги твои, свищут и скрежещут зубами, говорят: 'поглотили мы его, только этого дня и ждали мы, дождались, увидели!'
\end{tcolorbox}
\begin{tcolorbox}
\textsubscript{17} Совершил Господь, что определил, исполнил слово Свое, изреченное в древние дни, разорил без пощады и дал врагу порадоваться над тобою, вознес рог неприятелей твоих.
\end{tcolorbox}
\begin{tcolorbox}
\textsubscript{18} Сердце их вопиет к Господу: стена дщери Сиона! лей ручьем слезы день и ночь, не давай себе покоя, не спускай зениц очей твоих.
\end{tcolorbox}
\begin{tcolorbox}
\textsubscript{19} Вставай, взывай ночью, при начале каждой стражи; изливай, как воду, сердце твое пред лицем Господа; простирай к Нему руки твои о душе детей твоих, издыхающих от голода на углах всех улиц.
\end{tcolorbox}
\begin{tcolorbox}
\textsubscript{20} 'Воззри, Господи, и посмотри: кому Ты сделал так, чтобы женщины ели плод свой, младенцев, вскормленных ими? чтобы убиваемы были в святилище Господнем священник и пророк?
\end{tcolorbox}
\begin{tcolorbox}
\textsubscript{21} Дети и старцы лежат на земле по улицам; девы мои и юноши мои пали от меча; Ты убивал их в день гнева Твоего, заколал без пощады.
\end{tcolorbox}
\begin{tcolorbox}
\textsubscript{22} Ты созвал отовсюду, как на праздник, ужасы мои, и в день гнева Господня никто не спасся, никто не уцелел; тех, которые были мною вскормлены и вырощены, враг мой истребил'.
\end{tcolorbox}
\subsection{CHAPTER 3}
\begin{tcolorbox}
\textsubscript{1} Я человек, испытавший горе от жезла гнева Его.
\end{tcolorbox}
\begin{tcolorbox}
\textsubscript{2} Он повел меня и ввел во тьму, а не во свет.
\end{tcolorbox}
\begin{tcolorbox}
\textsubscript{3} Так, Он обратился на меня и весь день обращает руку Свою;
\end{tcolorbox}
\begin{tcolorbox}
\textsubscript{4} измождил плоть мою и кожу мою, сокрушил кости мои;
\end{tcolorbox}
\begin{tcolorbox}
\textsubscript{5} огородил меня и обложил горечью и тяготою;
\end{tcolorbox}
\begin{tcolorbox}
\textsubscript{6} посадил меня в темное место, как давно умерших;
\end{tcolorbox}
\begin{tcolorbox}
\textsubscript{7} окружил меня стеною, чтобы я не вышел, отяготил оковы мои,
\end{tcolorbox}
\begin{tcolorbox}
\textsubscript{8} и когда я взывал и вопиял, задерживал молитву мою;
\end{tcolorbox}
\begin{tcolorbox}
\textsubscript{9} каменьями преградил дороги мои, извратил стези мои.
\end{tcolorbox}
\begin{tcolorbox}
\textsubscript{10} Он стал для меня как бы медведь в засаде, [как бы] лев в скрытном месте;
\end{tcolorbox}
\begin{tcolorbox}
\textsubscript{11} извратил пути мои и растерзал меня, привел меня в ничто;
\end{tcolorbox}
\begin{tcolorbox}
\textsubscript{12} натянул лук Свой и поставил меня как бы целью для стрел;
\end{tcolorbox}
\begin{tcolorbox}
\textsubscript{13} послал в почки мои стрелы из колчана Своего.
\end{tcolorbox}
\begin{tcolorbox}
\textsubscript{14} Я стал посмешищем для всего народа моего, вседневною песнью их.
\end{tcolorbox}
\begin{tcolorbox}
\textsubscript{15} Он пресытил меня горечью, напоил меня полынью.
\end{tcolorbox}
\begin{tcolorbox}
\textsubscript{16} Сокрушил камнями зубы мои, покрыл меня пеплом.
\end{tcolorbox}
\begin{tcolorbox}
\textsubscript{17} И удалился мир от души моей; я забыл о благоденствии,
\end{tcolorbox}
\begin{tcolorbox}
\textsubscript{18} и сказал я: погибла сила моя и надежда моя на Господа.
\end{tcolorbox}
\begin{tcolorbox}
\textsubscript{19} Помысли о моем страдании и бедствии моем, о полыни и желчи.
\end{tcolorbox}
\begin{tcolorbox}
\textsubscript{20} Твердо помнит это душа моя и падает во мне.
\end{tcolorbox}
\begin{tcolorbox}
\textsubscript{21} Вот что я отвечаю сердцу моему и потому уповаю:
\end{tcolorbox}
\begin{tcolorbox}
\textsubscript{22} по милости Господа мы не исчезли, ибо милосердие Его не истощилось.
\end{tcolorbox}
\begin{tcolorbox}
\textsubscript{23} Оно обновляется каждое утро; велика верность Твоя!
\end{tcolorbox}
\begin{tcolorbox}
\textsubscript{24} Господь часть моя, говорит душа моя, итак буду надеяться на Него.
\end{tcolorbox}
\begin{tcolorbox}
\textsubscript{25} Благ Господь к надеющимся на Него, к душе, ищущей Его.
\end{tcolorbox}
\begin{tcolorbox}
\textsubscript{26} Благо тому, кто терпеливо ожидает спасения от Господа.
\end{tcolorbox}
\begin{tcolorbox}
\textsubscript{27} Благо человеку, когда он несет иго в юности своей;
\end{tcolorbox}
\begin{tcolorbox}
\textsubscript{28} сидит уединенно и молчит, ибо Он наложил его на него;
\end{tcolorbox}
\begin{tcolorbox}
\textsubscript{29} полагает уста свои в прах, [помышляя]: 'может быть, еще есть надежда';
\end{tcolorbox}
\begin{tcolorbox}
\textsubscript{30} подставляет ланиту свою биющему его, пресыщается поношением,
\end{tcolorbox}
\begin{tcolorbox}
\textsubscript{31} ибо не навек оставляет Господь.
\end{tcolorbox}
\begin{tcolorbox}
\textsubscript{32} Но послал горе, и помилует по великой благости Своей.
\end{tcolorbox}
\begin{tcolorbox}
\textsubscript{33} Ибо Он не по изволению сердца Своего наказывает и огорчает сынов человеческих.
\end{tcolorbox}
\begin{tcolorbox}
\textsubscript{34} Но, когда попирают ногами своими всех узников земли,
\end{tcolorbox}
\begin{tcolorbox}
\textsubscript{35} когда неправедно судят человека пред лицем Всевышнего,
\end{tcolorbox}
\begin{tcolorbox}
\textsubscript{36} когда притесняют человека в деле его: разве не видит Господь?
\end{tcolorbox}
\begin{tcolorbox}
\textsubscript{37} Кто это говорит: 'и то бывает, чему Господь не повелел быть'?
\end{tcolorbox}
\begin{tcolorbox}
\textsubscript{38} Не от уст ли Всевышнего происходит бедствие и благополучие?
\end{tcolorbox}
\begin{tcolorbox}
\textsubscript{39} Зачем сетует человек живущий? всякий сетуй на грехи свои.
\end{tcolorbox}
\begin{tcolorbox}
\textsubscript{40} Испытаем и исследуем пути свои, и обратимся к Господу.
\end{tcolorbox}
\begin{tcolorbox}
\textsubscript{41} Вознесем сердце наше и руки к Богу, [сущему] на небесах:
\end{tcolorbox}
\begin{tcolorbox}
\textsubscript{42} мы отпали и упорствовали; Ты не пощадил.
\end{tcolorbox}
\begin{tcolorbox}
\textsubscript{43} Ты покрыл Себя гневом и преследовал нас, умерщвлял, не щадил;
\end{tcolorbox}
\begin{tcolorbox}
\textsubscript{44} Ты закрыл Себя облаком, чтобы не доходила молитва наша;
\end{tcolorbox}
\begin{tcolorbox}
\textsubscript{45} сором и мерзостью Ты сделал нас среди народов.
\end{tcolorbox}
\begin{tcolorbox}
\textsubscript{46} Разинули на нас пасть свою все враги наши.
\end{tcolorbox}
\begin{tcolorbox}
\textsubscript{47} Ужас и яма, опустошение и разорение--доля наша.
\end{tcolorbox}
\begin{tcolorbox}
\textsubscript{48} Потоки вод изливает око мое о гибели дщери народа моего.
\end{tcolorbox}
\begin{tcolorbox}
\textsubscript{49} Око мое изливается и не перестает, ибо нет облегчения,
\end{tcolorbox}
\begin{tcolorbox}
\textsubscript{50} доколе не призрит и не увидит Господь с небес.
\end{tcolorbox}
\begin{tcolorbox}
\textsubscript{51} Око мое опечаливает душу мою ради всех дщерей моего города.
\end{tcolorbox}
\begin{tcolorbox}
\textsubscript{52} Всячески усиливались уловить меня, как птичку, враги мои, без всякой причины;
\end{tcolorbox}
\begin{tcolorbox}
\textsubscript{53} повергли жизнь мою в яму и закидали меня камнями.
\end{tcolorbox}
\begin{tcolorbox}
\textsubscript{54} Воды поднялись до головы моей; я сказал: 'погиб я'.
\end{tcolorbox}
\begin{tcolorbox}
\textsubscript{55} Я призывал имя Твое, Господи, из ямы глубокой.
\end{tcolorbox}
\begin{tcolorbox}
\textsubscript{56} Ты слышал голос мой; не закрой уха Твоего от воздыхания моего, от вопля моего.
\end{tcolorbox}
\begin{tcolorbox}
\textsubscript{57} Ты приближался, когда я взывал к Тебе, и говорил: 'не бойся'.
\end{tcolorbox}
\begin{tcolorbox}
\textsubscript{58} Ты защищал, Господи, дело души моей; искуплял жизнь мою.
\end{tcolorbox}
\begin{tcolorbox}
\textsubscript{59} Ты видишь, Господи, обиду мою; рассуди дело мое.
\end{tcolorbox}
\begin{tcolorbox}
\textsubscript{60} Ты видишь всю мстительность их, все замыслы их против меня.
\end{tcolorbox}
\begin{tcolorbox}
\textsubscript{61} Ты слышишь, Господи, ругательство их, все замыслы их против меня,
\end{tcolorbox}
\begin{tcolorbox}
\textsubscript{62} речи восстающих на меня и их ухищрения против меня всякий день.
\end{tcolorbox}
\begin{tcolorbox}
\textsubscript{63} Воззри, сидят ли они, встают ли, я для них--песнь.
\end{tcolorbox}
\begin{tcolorbox}
\textsubscript{64} Воздай им, Господи, по делам рук их;
\end{tcolorbox}
\begin{tcolorbox}
\textsubscript{65} пошли им помрачение сердца и проклятие Твое на них;
\end{tcolorbox}
\begin{tcolorbox}
\textsubscript{66} преследуй их, Господи, гневом, и истреби их из поднебесной.
\end{tcolorbox}
\subsection{CHAPTER 4}
\begin{tcolorbox}
\textsubscript{1} Как потускло золото, изменилось золото наилучшее! камни святилища раскиданы по всем перекресткам.
\end{tcolorbox}
\begin{tcolorbox}
\textsubscript{2} Сыны Сиона драгоценные, равноценные чистейшему золоту, как они сравнены с глиняною посудою, изделием рук горшечника!
\end{tcolorbox}
\begin{tcolorbox}
\textsubscript{3} И чудовища подают сосцы и кормят своих детенышей, а дщерь народа моего стала жестока подобно страусам в пустыне.
\end{tcolorbox}
\begin{tcolorbox}
\textsubscript{4} Язык грудного младенца прилипает к гортани его от жажды; дети просят хлеба, и никто не подает им.
\end{tcolorbox}
\begin{tcolorbox}
\textsubscript{5} Евшие сладкое истаевают на улицах; воспитанные на багрянице жмутся к навозу.
\end{tcolorbox}
\begin{tcolorbox}
\textsubscript{6} Наказание нечестия дщери народа моего превышает казнь за грехи Содома: тот низринут мгновенно, и руки человеческие не касались его.
\end{tcolorbox}
\begin{tcolorbox}
\textsubscript{7} Князья ее [были] в ней чище снега, белее молока; они были телом краше коралла, вид их был, как сапфир;
\end{tcolorbox}
\begin{tcolorbox}
\textsubscript{8} а теперь темнее всего черного лице их; не узнают их на улицах; кожа их прилипла к костям их, стала суха, как дерево.
\end{tcolorbox}
\begin{tcolorbox}
\textsubscript{9} Умерщвляемые мечом счастливее умерщвляемых голодом, потому что сии истаевают, поражаемые недостатком плодов полевых.
\end{tcolorbox}
\begin{tcolorbox}
\textsubscript{10} Руки мягкосердых женщин варили детей своих, чтобы они были для них пищею во время гибели дщери народа моего.
\end{tcolorbox}
\begin{tcolorbox}
\textsubscript{11} Совершил Господь гнев Свой, излил ярость гнева Своего и зажег на Сионе огонь, который пожрал основания его.
\end{tcolorbox}
\begin{tcolorbox}
\textsubscript{12} Не верили цари земли и все живущие во вселенной, чтобы враг и неприятель вошел во врата Иерусалима.
\end{tcolorbox}
\begin{tcolorbox}
\textsubscript{13} [Все это] --за грехи лжепророков его, за беззакония священников его, которые среди него проливали кровь праведников;
\end{tcolorbox}
\begin{tcolorbox}
\textsubscript{14} бродили как слепые по улицам, осквернялись кровью, так что невозможно было прикоснуться к одеждам их.
\end{tcolorbox}
\begin{tcolorbox}
\textsubscript{15} 'Сторонитесь! нечистый!' кричали им; 'сторонитесь, сторонитесь, не прикасайтесь'; и они уходили в смущении; а между народом говорили: 'их более не будет!
\end{tcolorbox}
\begin{tcolorbox}
\textsubscript{16} лице Господне рассеет их; Он уже не призрит на них', потому что они лица священников не уважают, старцев не милуют.
\end{tcolorbox}
\begin{tcolorbox}
\textsubscript{17} Наши глаза истомлены в напрасном ожидании помощи; со сторожевой башни нашей мы ожидали народ, который не мог спасти нас.
\end{tcolorbox}
\begin{tcolorbox}
\textsubscript{18} А они подстерегали шаги наши, чтобы мы не могли ходить по улицам нашим; приблизился конец наш, дни наши исполнились; пришел конец наш.
\end{tcolorbox}
\begin{tcolorbox}
\textsubscript{19} Преследовавшие нас были быстрее орлов небесных; гонялись за нами по горам, ставили засаду для нас в пустыне.
\end{tcolorbox}
\begin{tcolorbox}
\textsubscript{20} Дыхание жизни нашей, помазанник Господень пойман в ямы их, тот, о котором мы говорили: 'под тенью его будем жить среди народов'.
\end{tcolorbox}
\begin{tcolorbox}
\textsubscript{21} Радуйся и веселись, дочь Едома, обитательница земли Уц! И до тебя дойдет чаша; напьешься допьяна и обнажишься.
\end{tcolorbox}
\begin{tcolorbox}
\textsubscript{22} Дщерь Сиона! наказание за беззаконие твое кончилось; Он не будет более изгонять тебя; но твое беззаконие, дочь Едома, Он посетит и обнаружит грехи твои.
\end{tcolorbox}
\subsection{CHAPTER 5}
\begin{tcolorbox}
\textsubscript{1} Вспомни, Господи, что над нами совершилось; призри и посмотри на поругание наше.
\end{tcolorbox}
\begin{tcolorbox}
\textsubscript{2} Наследие наше перешло к чужим, домы наши--к иноплеменным;
\end{tcolorbox}
\begin{tcolorbox}
\textsubscript{3} мы сделались сиротами, без отца; матери наши--как вдовы.
\end{tcolorbox}
\begin{tcolorbox}
\textsubscript{4} Воду свою пьем за серебро, дрова наши достаются нам за деньги.
\end{tcolorbox}
\begin{tcolorbox}
\textsubscript{5} Нас погоняют в шею, мы работаем, [и] не имеем отдыха.
\end{tcolorbox}
\begin{tcolorbox}
\textsubscript{6} Протягиваем руку к Египтянам, к Ассириянам, чтобы насытиться хлебом.
\end{tcolorbox}
\begin{tcolorbox}
\textsubscript{7} Отцы наши грешили: их уже нет, а мы несем наказание за беззакония их.
\end{tcolorbox}
\begin{tcolorbox}
\textsubscript{8} Рабы господствуют над нами, и некому избавить от руки их.
\end{tcolorbox}
\begin{tcolorbox}
\textsubscript{9} С опасностью жизни от меча, в пустыне достаем хлеб себе.
\end{tcolorbox}
\begin{tcolorbox}
\textsubscript{10} Кожа наша почернела, как печь, от жгучего голода.
\end{tcolorbox}
\begin{tcolorbox}
\textsubscript{11} Жен бесчестят на Сионе, девиц--в городах Иудейских.
\end{tcolorbox}
\begin{tcolorbox}
\textsubscript{12} Князья повешены руками их, лица старцев не уважены.
\end{tcolorbox}
\begin{tcolorbox}
\textsubscript{13} Юношей берут к жерновам, и отроки падают под ношами дров.
\end{tcolorbox}
\begin{tcolorbox}
\textsubscript{14} Старцы уже не сидят у ворот; юноши не поют.
\end{tcolorbox}
\begin{tcolorbox}
\textsubscript{15} Прекратилась радость сердца нашего; хороводы наши обратились в сетование.
\end{tcolorbox}
\begin{tcolorbox}
\textsubscript{16} Упал венец с головы нашей; горе нам, что мы согрешили!
\end{tcolorbox}
\begin{tcolorbox}
\textsubscript{17} От сего-то изнывает сердце наше; от сего померкли глаза наши.
\end{tcolorbox}
\begin{tcolorbox}
\textsubscript{18} Оттого, что опустела гора Сион, лисицы ходят по ней.
\end{tcolorbox}
\begin{tcolorbox}
\textsubscript{19} Ты, Господи, пребываешь во веки; престол Твой--в род и род.
\end{tcolorbox}
\begin{tcolorbox}
\textsubscript{20} Для чего совсем забываешь нас, оставляешь нас на долгое время?
\end{tcolorbox}
\begin{tcolorbox}
\textsubscript{21} Обрати нас к Тебе, Господи, и мы обратимся; обнови дни наши, как древле.
\end{tcolorbox}
\begin{tcolorbox}
\textsubscript{22} Неужели Ты совсем отверг нас, прогневался на нас безмерно?
\end{tcolorbox}
