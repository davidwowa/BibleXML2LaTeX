\section{BOOK 56}
\subsection{CHAPTER 1}
\begin{tcolorbox}
\textsubscript{1} Павло, в'язень Христа Ісуса, та брат Тимофій, улюбленому Филимонові й співробітникові нашому,
\end{tcolorbox}
\begin{tcolorbox}
\textsubscript{2} і сестрі любій Апфії, і співвойовникові нашому Архипові, і Церкві домашній твоїй:
\end{tcolorbox}
\begin{tcolorbox}
\textsubscript{3} благодать вам і мир від Бога Отця нашого й Господа Ісуса Христа!
\end{tcolorbox}
\begin{tcolorbox}
\textsubscript{4} Я завсіди дякую Богові моєму, коли тебе згадую в молитвах своїх.
\end{tcolorbox}
\begin{tcolorbox}
\textsubscript{5} Бо я чув про любов твою й віру, яку маєш до Господа Ісуса, і до всіх святих,
\end{tcolorbox}
\begin{tcolorbox}
\textsubscript{6} щоб спільність віри твоєї діяльна була в пізнанні всякого добра, що в нас для Христа.
\end{tcolorbox}
\begin{tcolorbox}
\textsubscript{7} Бо ми маємо радість велику й потіху в любові твоїй, серця бо святих заспокоїв ти, брате.
\end{tcolorbox}
\begin{tcolorbox}
\textsubscript{8} Через це, хоч я маю велику відвагу в Христі подавати накази тобі про потрібне,
\end{tcolorbox}
\begin{tcolorbox}
\textsubscript{9} але більше з любови благаю я, як Павло, старий, тепер же ще й в'язень Христа Ісуса.
\end{tcolorbox}
\begin{tcolorbox}
\textsubscript{10} Благаю тебе про сина свого, про Онисима, що його породив я в кайданах своїх.
\end{tcolorbox}
\begin{tcolorbox}
\textsubscript{11} Колись то для тебе він був непотрібний, тепер же для тебе й для мене він дуже потрібний.
\end{tcolorbox}
\begin{tcolorbox}
\textsubscript{12} Тобі я вертаю його, того, хто є неначе серце моє.
\end{tcolorbox}
\begin{tcolorbox}
\textsubscript{13} Я хотів був тримати його при собі, щоб він замість тебе мені послужив у кайданах за Євангелію,
\end{tcolorbox}
\begin{tcolorbox}
\textsubscript{14} та без волі твоєї нічого робити не хотів я, щоб твій добрий учинок не був ніби вимушений, але добровільний.
\end{tcolorbox}
\begin{tcolorbox}
\textsubscript{15} Бо може для того він був розлучився на час, щоб навіки прийняв ти його,
\end{tcolorbox}
\begin{tcolorbox}
\textsubscript{16} і вже не як раба, але вище від раба, як брата улюбленого, особливо для мене, а тим більше для тебе, і за тілом, і в Господі.
\end{tcolorbox}
\begin{tcolorbox}
\textsubscript{17} Отож, коли маєш за друга мене, то прийми його, як мене.
\end{tcolorbox}
\begin{tcolorbox}
\textsubscript{18} Коли ж він чим скривдив тебе або винен тобі, полічи це мені.
\end{tcolorbox}
\begin{tcolorbox}
\textsubscript{19} Я, Павло, написав це рукою своєю: Я віддам, щоб тобі не казати, що ти навіть самого себе мені винен.
\end{tcolorbox}
\begin{tcolorbox}
\textsubscript{20} Так, брате, нехай я одержу те, що від тебе прохаю в Господі. Заспокой моє серце в Христі!
\end{tcolorbox}
\begin{tcolorbox}
\textsubscript{21} Пересвідчений я про слухняність твою, і тобі написав оце, відаючи, що ти зробиш і більше, ніж я говорю.
\end{tcolorbox}
\begin{tcolorbox}
\textsubscript{22} А разом мені приготуй і помешкання, бо надіюся я, що за ваші молитви я буду дарований вам.
\end{tcolorbox}
\begin{tcolorbox}
\textsubscript{23} Вітає тебе Епафрас, мій співв'язень у Христі Ісусі,
\end{tcolorbox}
\begin{tcolorbox}
\textsubscript{24} Марко, Аристарх, Димас, Лука, мої співробітники.
\end{tcolorbox}
\begin{tcolorbox}
\textsubscript{25} Благодать Господа Ісуса Христа з вашим духом! Амінь.
\end{tcolorbox}
